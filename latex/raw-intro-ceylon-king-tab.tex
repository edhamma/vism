

\ifplastex
\begin{tabular}{l|l|l}
    Kings of Ceylon & Relevant event & Refs.\\
    Devānampiya Tissa:  BCE 307–267   & Arrival in Sri Lanka of the Arahant Mahinda bringing Pali Tipiṭaka with Commentaries; Commentaries translated into Sinhalese; Great Monastery founded.  & \emph{Mahāvaṃsa}, \textbf{\cite{Mhv} XIII}\\
    Duṭṭhagāmaṇi BCE 161–137 & Expulsion of invaders after 76 years of foreign occupation of capital; restoration of unity and independence.   & \textbf{\cite{Mhv} XXV–XXXII} \\
     & Many names of Great Monastery elders, noted in Commentaries for virtuous behaviour, traceable to this and following reign.  & Adikaram, \emph{Early History of Buddhism in Sri Lanka}, pp. 65–70 \\
    Vaṭṭagāmaṇi  BCE 104–88  & Reign interrupted after 5 months by rebellion of Brahman Tissa, famine, invasion, and king’s exile. & \textbf{\cite{Mhv} XXXIII.33f.} \\
     & Bhikkhus all disperse from Great Monastery to South SL and to India.   & \textbf{\cite{A-a} I 92}\\
     & Restoration of king after 14 years and return of bhikkhus. & \textbf{\cite{Mhv} XXXIII.78}\\
     & Foundation of Abhayagiri Monastery by king.  & \textbf{\cite{Mhv} XXXIII.81}  \\
     & Abhayagiri Monastery secedes from Great Monastery and becomes schismatic. & \textbf{\cite{Mhv} XXXIII.96}\\
     & Committal by Great Monastery of Pali Tipiṭaka to writing for first time (away from royal capital).  & \textbf{\cite{Mhv} XXXIII.100};  \textbf{\cite{Nikāya-s}} (translation) 10–11  \\
     & Abhayagiri Monastery adopts  “Dhammaruci Nikāya of Vajjiputtaka Sect” of India.  & \textbf{\cite{Nikāya-s} 11}\\
     & Meeting of Great Monastery bhikkhus  decides that care of texts and preaching  comes before practice of their contents.  & \textbf{\cite{A-a} I 92f}; \textbf{\cite{EHBC} 78}\\
     & Many Great Monastery elders’ names noted  in Commentaries for learning and contributions to decision of textual  problems, traceable to this reign. & \textbf{\cite{EHBC} 76} \\
    Kuṭakaṇṇa Tissa BCE 30–33 & Many elders as last stated traceable to this reign too. & \textbf{\cite{EHBC} 80} \\
     & Last Sri Lanka elders’ names in Vinaya Parivāra (p. 2) traceable to this reign; Parivāra can thus have been completed by Great Monastery any time later, before 5th cent   & \textbf{\cite{EHBC} 86}\\
    Bhātikābhaya BCE 20–CE 9 & Dispute between Great Monastery and Abhayagiri Monastery over Vinaya adjudged by Brahman Dīghakārāyana in favour of Great Monastery  & Vin-a 582; \textbf{\cite{EHBC} 99} \\
    Khanirājānu-Tissa 30–33 & 60 bhikkhus punished for treason. & \textbf{\cite{Mhv} XXXV.1}\\
    Vasabha  66–110   & Last reign to be mentioned in body of Commentaries.   & \textbf{\cite{EHBC} 3}, 86–7 \\
     & Sinhalese Commentaries can have been closed at any time after this reign & \textbf{\cite{EHBC} 3}, 86–7 \\
    Gajabāhu I  113–135   & Abhayagiri Monastery supported by king and enlarged.  & \textbf{\cite{Mhv} XXXV.119} \\
    6 kings  135–215   & Mentions of royal support for Great Monastery and Abhayagiri Monastery & \textbf{\cite{Mhv} XXXV.1}, 7, 24, 33, 65 \\
    Vohārika-Tissa 215 –237  & King supports both monasteries.  & \\
     & Abhayagiri Monastery has adopted Vetulya (Mahāyāna?) Piṭaka. & \textbf{\cite{Nikāya-s} 12}\\
     & King suppresses Vetulya doctrines. & \textbf{\cite{Mhv} XXXVI.41}\\
     & Vetulya books burnt and heretic bhikkhus disgraced   & \textbf{\cite{Nikāya-s} 12} \\
     & Corruption of bhikkhus by Vitaṇḍavadins (heretics or destructive critics).  & \emph{Dīpavaṃsa, }\textbf{\cite{Dīp}} XXII–XXIII\\
    Gothābhaya 254–267 & Great Monastery supported by king.  & \textbf{\cite{Mhv} XXXVI.102} \\
     & 60 bhikkhus in Abhayagiri Monastery banished by king for upholding Vetulya doctrines. & \textbf{\cite{Mhv} XXXVI.111}\\
     & Secession from Abhayagiri Monastery; new sect formed & \textbf{\cite{Nikāya-s} 13}\\
     & Indian bhikkhu Saṅghamitta supports Abhayagiri Monastery & \textbf{\cite{Mhv} XXXVI.112}\\
    Jeṭṭha-Tissa 267–277 & King favours Great Monastery; Saṅghamitta flees to India.  & \textbf{\cite{Mhv} XXXVI.123}  \\
    Mahāsena 277–304 & King protects Saṅghamitta, who returns Persecution of Great Monastery; its  bhikkhus driven from capital for 9 years.  & \textbf{\cite{Mhv} XXXVII.1–50}\\
     & Saṅghamitta assassinated. & \textbf{\cite{Mhv} XXXVII.27}\\
     &  Restoration of Great Monastery  & \textbf{\cite{EHBC} 92}\\
     & Vetulya books burnt again.  & \textbf{\cite{EHBC} 92}\\
     & Dispute over Great Monastery boundary; bhikkhus again absent from Great Monastery for 9 months. & \textbf{\cite{Mhv} XXXVII.32}\\
    Siri Meghavaṇṇa 304–332   & King favours Great Monastery & EHBC 92; \textbf{\cite{Mhv} XXXVII.51f}  \\
     & Sinhalese monastery established at Buddha Gayā in India & Malalasekera \textbf{\cite{PLC}, p.68}; Epigraphia Zeylanica iii, II\\
    Jeṭṭha-Tissa II 332–34 & \emph{Dīpavaṃsa} composed in this period. & Quoted in \textbf{\cite{Vin-a}}\\
    Buddhadāsa 341–70; Upatissa  370–412   & Also perhaps \emph{Mūlasikkhā} and \emph{Khuddasikkhā} (Vinaya summaries) and some of Buddhadatta Thera’s works. & \textbf{\cite{PLC}, p.77}\\
    Mahānāma 412–434 & Bhadantācariya Buddhaghosa arrives in Sri Lanka.   & \textbf{\cite{Mhv} XXXVII.215–46} \\
     & \emph{Samantapāsādikā} (Vinaya commentary) begun in 20th and finished in 21st year of this king’s reign. & \textbf{\cite{Vin-a}} Epilogue
\end{tabular}
\else
\begin{longtblr}[theme=vismLong]{colspec={X[2]|X[4]|X[3]},rowhead=1}
    Kings of Ceylon & Relevant event & Refs.\\
    Devānampiya Tissa:  BCE 307–267   & Arrival in Sri Lanka of the Arahant Mahinda bringing Pali Tipiṭaka with Commentaries; Commentaries translated into Sinhalese; Great Monastery founded.  & \emph{Mahāvaṃsa}, \textbf{\cite{Mhv} XIII}\\
    Duṭṭhagāmaṇi BCE 161–137 & Expulsion of invaders after 76 years of foreign occupation of capital; restoration of unity and independence.   & \textbf{\cite{Mhv} XXV–XXXII} \\
     & Many names of Great Monastery elders, noted in Commentaries for virtuous behaviour, traceable to this and following reign.  & Adikaram, \emph{Early History of Buddhism in Sri Lanka}, pp. 65–70 \\
    Vaṭṭagāmaṇi  BCE 104–88  & Reign interrupted after 5 months by rebellion of Brahman Tissa, famine, invasion, and king’s exile. & \textbf{\cite{Mhv} XXXIII.33f.} \\
     & Bhikkhus all disperse from Great Monastery to South SL and to India.   & \textbf{\cite{A-a} I 92}\\
     & Restoration of king after 14 years and return of bhikkhus. & \textbf{\cite{Mhv} XXXIII.78}\\
     & Foundation of Abhayagiri Monastery by king.  & \textbf{\cite{Mhv} XXXIII.81}  \\
     & Abhayagiri Monastery secedes from Great Monastery and becomes schismatic. & \textbf{\cite{Mhv} XXXIII.96}\\
     & Committal by Great Monastery of Pali Tipiṭaka to writing for first time (away from royal capital).  & \textbf{\cite{Mhv} XXXIII.100};  \textbf{\cite{Nikāya-s}} (translation) 10–11  \\
     & Abhayagiri Monastery adopts  “Dhammaruci Nikāya of Vajjiputtaka Sect” of India.  & \textbf{\cite{Nikāya-s} 11}\\
     & Meeting of Great Monastery bhikkhus  decides that care of texts and preaching  comes before practice of their contents.  & \textbf{\cite{A-a} I 92f}; \textbf{\cite{EHBC} 78}\\
     & Many Great Monastery elders’ names noted  in Commentaries for learning and contributions to decision of textual  problems, traceable to this reign. & \textbf{\cite{EHBC} 76} \\
    Kuṭakaṇṇa Tissa BCE 30–33 & Many elders as last stated traceable to this reign too. & \textbf{\cite{EHBC} 80} \\
     & Last Sri Lanka elders’ names in Vinaya Parivāra (p. 2) traceable to this reign; Parivāra can thus have been completed by Great Monastery any time later, before 5th cent   & \textbf{\cite{EHBC} 86}\\
    Bhātikābhaya BCE 20–CE 9 & Dispute between Great Monastery and Abhayagiri Monastery over Vinaya adjudged by Brahman Dīghakārāyana in favour of Great Monastery  & Vin-a 582; \textbf{\cite{EHBC} 99} \\
    Khanirājānu-Tissa 30–33 & 60 bhikkhus punished for treason. & \textbf{\cite{Mhv} XXXV.1}\\
    Vasabha  66–110   & Last reign to be mentioned in body of Commentaries.   & \textbf{\cite{EHBC} 3}, 86–7 \\
     & Sinhalese Commentaries can have been closed at any time after this reign & \textbf{\cite{EHBC} 3}, 86–7 \\
    Gajabāhu I  113–135   & Abhayagiri Monastery supported by king and enlarged.  & \textbf{\cite{Mhv} XXXV.119} \\
    6 kings  135–215   & Mentions of royal support for Great Monastery and Abhayagiri Monastery & \textbf{\cite{Mhv} XXXV.1}, 7, 24, 33, 65 \\
    Vohārika-Tissa 215 –237  & King supports both monasteries.  & \\
     & Abhayagiri Monastery has adopted Vetulya (Mahāyāna?) Piṭaka. & \textbf{\cite{Nikāya-s} 12}\\
     & King suppresses Vetulya doctrines. & \textbf{\cite{Mhv} XXXVI.41}\\
     & Vetulya books burnt and heretic bhikkhus disgraced   & \textbf{\cite{Nikāya-s} 12} \\
     & Corruption of bhikkhus by Vitaṇḍavadins (heretics or destructive critics).  & \emph{Dīpavaṃsa, }\textbf{\cite{Dīp}} XXII–XXIII\\
    Gothābhaya 254–267 & Great Monastery supported by king.  & \textbf{\cite{Mhv} XXXVI.102} \\
     & 60 bhikkhus in Abhayagiri Monastery banished by king for upholding Vetulya doctrines. & \textbf{\cite{Mhv} XXXVI.111}\\
     & Secession from Abhayagiri Monastery; new sect formed & \textbf{\cite{Nikāya-s} 13}\\
     & Indian bhikkhu Saṅghamitta supports Abhayagiri Monastery & \textbf{\cite{Mhv} XXXVI.112}\\
    Jeṭṭha-Tissa 267–277 & King favours Great Monastery; Saṅghamitta flees to India.  & \textbf{\cite{Mhv} XXXVI.123}  \\
    Mahāsena 277–304 & King protects Saṅghamitta, who returns Persecution of Great Monastery; its  bhikkhus driven from capital for 9 years.  & \textbf{\cite{Mhv} XXXVII.1–50}\\
     & Saṅghamitta assassinated. & \textbf{\cite{Mhv} XXXVII.27}\\
     &  Restoration of Great Monastery  & \textbf{\cite{EHBC} 92}\\
     & Vetulya books burnt again.  & \textbf{\cite{EHBC} 92}\\
     & Dispute over Great Monastery boundary; bhikkhus again absent from Great Monastery for 9 months. & \textbf{\cite{Mhv} XXXVII.32}\\
    Siri Meghavaṇṇa 304–332   & King favours Great Monastery & EHBC 92; \textbf{\cite{Mhv} XXXVII.51f}  \\
     & Sinhalese monastery established at Buddha Gayā in India & Malalasekera \textbf{\cite{PLC}, p.68}; Epigraphia Zeylanica iii, II\\
    Jeṭṭha-Tissa II 332–34 & \emph{Dīpavaṃsa} composed in this period. & Quoted in \textbf{\cite{Vin-a}}\\
    Buddhadāsa 341–70; Upatissa  370–412   & Also perhaps \emph{Mūlasikkhā} and \emph{Khuddasikkhā} (Vinaya summaries) and some of Buddhadatta Thera’s works. & \textbf{\cite{PLC}, p.77}\\
    Mahānāma 412–434 & Bhadantācariya Buddhaghosa arrives in Sri Lanka.   & \textbf{\cite{Mhv} XXXVII.215–46} \\
     & \emph{Samantapāsādikā} (Vinaya commentary) begun in 20th and finished in 21st year of this king’s reign. & \textbf{\cite{Vin-a}} Epilogue
\end{longtblr}
\fi
