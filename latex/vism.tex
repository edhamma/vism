\documentclass[a4paper]{book}
% for LaTex, disable plastex parts
\newif\ifplastex\plastexfalse
\ifplastex
	\let\RaggedRight\relax
	\def\marginnote#1{\textcolor{teal}{\footnotesize #1} }
	% \*matter are no-op in plastex, define what we need
	\def\frontmatter{\setcounter{part}{-1}\part{[Front]}} % Part 0 in roman will show as Part, giving Part [Front]
	\def\mainmatter{\setcounter{chapter}{0}}
	\def\backmatter{\setcounter{part}{-1}\part{[Back]}}
	\def\linkdest#1{\hypertarget{#1}{}}
	%\def\BPSed#1{\textbf{#1}}
\else
	\usepackage[paperwidth=18.2cm,paperheight=23.2cm,a4paper,margin=2.5cm]{geometry}
	% use the same-looking font like BPS2011
	% (they used URWPalladioPali, this is modernized, with unicode glyphs for all pali)
	\usepackage{fontspec}
	\setmainfont{Tex Gyre Pagella}
	% \RaggedRight used in two-column index
	\usepackage{ragged2e}
	% showing anchors and BPS pages in the margin
	\usepackage{marginnote}
	% no page numbers on empty pages
	\usepackage{emptypage}
	% make TOC with long (XVI.34) section titles nices
	\usepackage{tocloft}
		\setlength{\cftsubsecnumwidth}{4em}
		\setlength{\cftsecnumwidth}{4em}
		\setlength{\cftchapnumwidth}{4em}
	% less warnings
		\hbadness10000
		\vbadness10000
		\hfuzz=10cm
		\vfuzz=10cm
		\parskip=.4\baselineskip
	% https://tex.stackexchange.com/a/412381
	\makeatletter
	   \newcommand{\linkdest}[1]{\Hy@raisedlink{\hypertarget{#1}{}}}
	\makeatother
\fi

% number chapter with roman numbers
\renewcommand{\thechapter}{\Roman{chapter}}
% show parts as "Part II" in the TOC, but avoid "Part Part II" in the title itself
\renewcommand{\thepart}{Part \Roman{part}}
\renewcommand{\partname}{} 

\usepackage[bookmarksnumbered=true]{hyperref}
\usepackage{multicol}
\usepackage{xcolor}

\bibliographystyle{abstract}

\begin{document}
	\title{The Path of Purification \\ (\emph{Visuddhimagga})}
	\date{(decompiled preview full of errors)}
	\author{
		by \ifplastex\else\\\fi Bhadantácariya Buddhaghosa
		\ifplastex\\\else\and\fi
		Translated from the Pali by \ifplastex\else\\\fi Bhikkhu Ñáóamoli
	}
	\maketitle
	\frontmatter
		\tableofcontents
		

        \chapter[Prefaces]{Prefaces}
            \section{Copyright notice}
                \begin{verse}
                    © 1975, 1991, 2010 Buddhist Publication Society. All rights reserved.\\{}
                    First edition: 1956 by Mr. Ananda Semage, Colombo.\\{}
                    Second edition: 1964\\{}
                    Reprinted: 1979 by BPS\\{}
                    Third edition: 1991\\{}
                    Reprinted: 1999\\{}
                    Fourth edition: 2010
                \end{verse}

            \section{Buddhist Publication Society edition note}

                \textbf{The Buddhist Publication Society} is an approved charity dedicated to making known the Teaching of the Buddha, which has a vital message for people of all creeds.

                Founded in 1958, the BPS has published a wide variety of books and booklets covering a great range of topics. Its publications include accurate annotated translations of the Buddha’s discourses, standard reference works, as well as original  contemporary  expositions  of  Buddhist  thought  and  practice.  These works present Buddhism as it truly is — a dynamic force which has influenced receptive minds for the past 2500 years and is still as relevant today as it was when it first arose.
                \begin{verse}
                    Buddhist Publication Society\\{}
                    P.O. Box 61\\{}
                    54, Sangharaja Mawatha, Kandy, Sri Lanka\\{}
                    http://www.bps.lk
                \end{verse}


                \textbf{Printed Hardbound Copy:} A printed hardbound version of this book is available from the Buddhist Publication Society; see the “Translations from the Pali” page at the BPS online bookshop at http://www.bps.lk.

                \textbf{Terms of use:} Copyright revised, third, online edition. Copyright © 2011 Buddhist Publication Society.

                You may redistribute this PDF file in an unaltered form provided that: (1) you must only make such unaltered PDF copies available free of charge; (2) you clearly indicate that any passages of this work reproduced into other publications (printed as well as digital) are derived from this source document. Otherwise, all rights reserved.

                \textbf{Publisher’s Note:} The BPS thanks all those who assisted with making this book available in a digital as well as printed edition. This book is the result of the work of Mr John Bullitt of Access to Insight who initiated this digital edition, the several volunteers who helped him to convert the previous edition of this book to digital text, the other volunteers who helped the BPS with proofreading, Bhikkhu Nyanatusita, the BPS editor, who helped with and coordinated the proofreading, formatting, and typesetting, corrected the Pali, etc., and the work of the BPS typesetters Bhikkhu Sacramento Upatissa and Mr Nalin Ariyaratna who skilfully typeset the text.
            \section{Message from his Holiness the Dalai Lama }

                \marginnote{\textcolor{teal}{\footnotesize\{25|xxiii\}}}{}The history of the development of Buddhist literature seems to be marked by periods in which the received teachings and established scriptures are assimilated and consolidated and periods of mature creativity when the essence of that transmission is expressed afresh. Bhadantācariya Buddhaghosa’s \emph{Visuddhimagga} is a classic text of the latter type. It represents the epitome of Pali Buddhist literature, weaving together its many strands to create this wonderful meditation manual, which even today retains the clarity it revealed when it was written.

                There are occasions when people like to make much of the supposed differences in the various traditions of Buddhism that have evolved in different times and places. What I find especially encouraging about a book such as this is that it shows so clearly how much all schools of Buddhism have fundamentally in common. Within a structure based on the traditional three trainings of ethical discipline, concentration and wisdom are detailed instructions on how to take an ethical approach to life, how to meditate and calm the mind, and on the basis of those how to develop a correct understanding of reality. We find practical advice about creating an appropriate environment for meditation, the importance of developing love and compassion, and discussion of dependent origination that underlies the Buddhist view of reality. The very title of the work, the \emph{Path of Purification,} refers to the essential Buddhist understanding of the basic nature of the mind as clear and aware, unobstructed by disturbing emotions. This quality is possessed by all sentient beings which all may realize if we pursue such a path.

                Sometimes I am asked whether Buddhism is suitable for Westerners or not. I believe that the essence of all religions deals with basic human problems and Buddhism is no exception. As long as we continue to experience the basic human sufferings of birth, disease, old age, and death, there is no question of whether it is suitable or not as a remedy. Inner peace is the key. In that state of mind you can face difficulties with calm and reason. The teachings of love, kindness and tolerance, the conduct of non-violence, and especially the Buddhist theory that all things are relative can be a source of that inner peace.

                While the essence of Buddhism does not change, superficial cultural aspects will change. But how they will change in a particular place, we cannot say. This evolves over time. When Buddhism first came from India to countries like Sri Lanka or Tibet, it gradually evolved, and in time a unique tradition arose. This is also happening in the West, and gradually Buddhism may evolve with Western culture.

                Of course, what distinguishes the contemporary situation from past transmissions of Buddhism is that almost the entire array of traditions that evolved elsewhere is now accessible to anyone who is interested. And it is in such a context that I welcome this new edition of Bhikkhu Ñāṇamoli’s celebrated English translation of the \emph{Path of Purification}. I offer my prayers that readers, wherever they are, may find in it advice and inspiration to develop that inner peace that will contribute to creating a happier and more peaceful world.

                May 2000
            \section{Publisher’s Foreword to Third Edition }

                \marginnote{\textcolor{teal}{\footnotesize\{26|xxiv\}}}{}Bhikkhu Ñāṇamoli’s translation of the \emph{Visuddhimagga} not only makes available in fluent English this difficult and intricate classical work of Theravāda Buddhism, the high point of the commentarial era, but itself ranks as an outstanding cultural achievement perhaps unmatched by Pali Buddhist scholarship in the twentieth century. This achievement is even more remarkable in that the translator had completed the first draft within his first four years as a bhikkhu, which is also the amount of time he had been a student of Pali.

                The Buddhist Publication Society first issued this work beginning in 1975, with the kind consent of the original publisher, Mr. Ānanda Semage of Colombo. This was a reprint produced by photolithographic process from the 1964 edition. The 1979 reprint was also a photolithographic reprint, with some minor corrections..

                For this edition the text has been entirely recomposed, this time with the aid of the astonishing electronic typesetting equipment that has proliferated during the past few years. The text itself has not been altered except in a few places where the original translator had evidently made an oversight. However, numerous minor stylistic changes have been introduced, particularly in the lower casing of many technical terms that Ven. Ñāṇamoli had set in initial capitals and, occasionally, in the paragraphing.

                Buddhist Publication Society, 1991
            \section{Publisher’s Foreword to Fourth Edition }

                This fourth edition had to be retypeset again because the digital files of the previous edition, prepared “with the aid of the astonishing electronic typesetting equipment” (as mentioned in the Foreword to the Third Edition) were lost.

                Like in the previous edition, the text itself has not been altered except in a few places where Ven. Ñāṇamoli had evidently made an oversight. A few minor stylistic changes have been introduced again, such as the utilisation of the \emph{Critical Pali Dictionary }system of abbreviation instead of the PTS system

                The BPS would like to thank John Bullitt, Ester Barias-Wolf, Michael Zoll,

                Manfred Wierich and all others who helped with this project.

                Buddhist Publication Society, 2010
            \section{Translator’s Dedication}
                \begin{verse}
                    \emph{Ciraṃ tiṭṭhatu saddhammo}\\{}
                    \emph{sabbe sattā bhavantu sukhitattā}
                \end{verse}

                \begin{verse}
                    To  my  Upajjhāya,\\{}
                    the  late  venerable  Pälänē  Siri  Vajirañāṇa\\{}
                    Mahānāyakathera  of  Vajirārāma,\\{}
                    Colombo, Sri Lanka.
                \end{verse}

            \section{Translator’s Preface }

                \marginnote{\textcolor{teal}{\footnotesize\{27|xxv\}}}{}Originally I made this translation for my own instruction because the only published version was then no longer obtainable. So it was not done with any intention at all of publication; but rather it grew together out of notes made on some of the book’s passages. By the end of 1953 it had been completed, more or less, and put aside. Early in the following year a suggestion to publish it was put to me, and I eventually agreed, though not without a good deal of hesitation. Reasons for agreeing, however, seemed not entirely lacking. The only previous English version of this remarkable work had long been out of print. Justification too could in some degree be founded on the rather different angle from which this version is made.

                Over a year was then spent in typing out the manuscript during which time, and since, a good deal of revision has taken place, the intention of the revision being always to propitiate the demon of inaccuracy and at the same time to make the translation perspicuous and the translator inconspicuous. Had publication been delayed, it might well have been more polished. Nevertheless the work of polishing is probably endless. Somewhere a halt must be made.

                A guiding principle—the foremost, in fact—has throughout been avoidance of misrepresentation or distortion; for the ideal translation (which has yet to be made) should, like a looking glass, not discolour or blur or warp the original which it reflects. Literalness, however, on the one hand and considerations of clarity and style on the other make irreconcilable claims on a translator, who has to choose and to compromise. Vindication of his choice is sometimes difficult.

                I have dealt at the end of the Introduction with some particular problems. Not, however, with all of them or completely; for the space allotted to an introduction is limited.

                Much that is circumstantial has now changed since the Buddha discovered and made known his liberating doctrine 2,500 years ago, and likewise since this work was composed some nine centuries later. On the other hand, the Truth he discovered has remained untouched by all that circumstantial change. Old cosmologies give place to new; but the questions of consciousness, of pain and death, of responsibility for acts, and of what should be looked to in the scale it values as the highest of all, remain. Reasons for the perennial freshness of the Buddha’s teaching—of his handling of these questions—are several, but not least among them is its independence of any particular cosmology. Established as it is for its foundation on the self-evident insecurity of the human situation (the truth of suffering), the structure of the Four Noble Truths provides an unfailing standard of value, unique in its simplicity, its completeness and its ethical purity, by means of which any situation can be assessed and a profitable choice made.

                Now I should like to make acknowledgements, as follows, to all those without whose help this translation would never have been begun, persisted with or completed. \marginnote{\textcolor{teal}{\footnotesize\{28|xxvi\}}}{}To the venerable Ñāṇatiloka Mahāthera (from whom I first learned Pali) for his most kind consent to check the draft manuscript. However, although he had actually read through the first two chapters, a long spell of illness unfortunately prevented him from continuing with this himself.

                To the venerable Soma Thera for his unfailing assistance both in helping me to gain familiarity with the often difficult Pali idiom of the Commentaries and to get something of the feel—as it were, “from inside”—of Pali literature against its Indian background. Failing that, no translation would ever have been made: I cannot tell how far I have been able to express any of it in the rendering.

                To the venerable Nyanaponika Thera, German pupil of the venerable Ñāṇatiloka Mahāthera, for very kindly undertaking to check the whole manuscript in detail with the venerable Ñāṇatiloka Mahāthera’s German translation (I knowing no German).

                To all those with whom I have had discussions on the Dhamma, which have been many and have contributed to the clearing up of not a few unclear points.

                Lastly, and what is mentioned last bears its own special emphasis, it has been an act of singular merit on the part of Mr. A. Semage, of Colombo, to undertake to publish this translation.

                Island Hermitage Dodanduwa, Sri Lanka

                Ñāṇamoli Bhikkhu, Vesākhamāse, 2499: May, 1956
        \chapter[Bibliography]{Bibliography}\begin{vismHanging}
            \section{Printed Editions of the \emph{Visuddhimagga}}

                \emph{Sinhalese script:} Hewavitarne Bequest edition, Colombo.

                \emph{Burmese script:} Hanthawaddy Press edition, Rangoon, 1900.

                \emph{Siamese script:} Royal Siamese edition, Bangkok.

                \emph{Latin script:} Pali Text Society’s edition, London. Harvard University Press edition, Harvard Oriental Series, Vol. 41, Cambridge, Mass., 1950.
            \section{Translations of the \emph{Visuddhimagga}}

                \emph{English:}\emph{ The Path of Purity} by Pe Maung Tin, PTS, London. 3 vols., 1922–31.

                \emph{German:}\emph{ Visuddhimagga (der Weg zur Reinheit)} by Nyanatiloka, Verlag Christiani, Konstanz, 1952. Reprinted by Jhana-Verlag, Uttenbühl, 1997.

                \emph{Sinhala:}\emph{ Visuddhimārga-mahāsannē}, ed. Ratanapala Medhaṅkara et al, 2 vols., Kalutara, 1949. (Also called \emph{Parākramabāhu-sannaya}. A Pali-Sinhala paraphrase composed by King Paṇḍita Parākramabāhu II in the 13th cent. CE.) \emph{Visuddhimārgaya}, Sinhala translation by Paṇḍita Mātara Sri Dharmavaṃsa Sthavira, Mātara, 1953. Etc.

                \emph{French:}\emph{ Le Chemin de la pureté}, transl. by Christian Maës, Editions Fayard, Paris 2002.

                \emph{Italian:}\emph{ Visuddhimagga: Il sentiero della purificazione}, transl. of \emph{samādhi-bheda} by Antonella Serena Comba, Lulu.com, Raleigh, 2008.
            \section{Other works}

                \emph{Buddhaghosuppatti}, edited and translated into English, by J. Gray, Luzac and Co., London, 1892.

                \emph{Critical Pali Dictionary} (Pali-English), Vol. I (letter \emph{a}), Copenhagen, 1924–48.

                \emph{Cūḷavaṃsa or Minor Chronicle of Ceylon}\emph{ (or Mahāvaṃsa Part II)}, English translation by W. Geiger, PTS London.

                \emph{Dīpavaṃsa (Chronicle of Ceylon)}, English translation by H. Oldenberg, London, 1879.

                \emph{The Early History of Buddhism in Ceylon}, by E. W. Adikaram, Sri Lanka, 1946.

                \emph{Guide through Visuddhimagga}, U. Dhammaratana, Sarnath, 1964.

                \emph{History of Indian Literature}, by M. Winternitz, English translation by Mrs. S. Ketkar and Miss H. Kohn, Calcutta University, 1933.

                \emph{History of Pali Literature}, by B.C. Law, London, 1933 (2 Vols.).

                \emph{The Life and Work of Buddhaghosa}, by B.C. Law, Thacker, and Spink, Calcutta and Simla, 1923.

                \emph{Mahāvaṃsa or Great Chronicle of Ceylon}, English translation by W. Geiger, PTS, London.

                \emph{Pali-English Dictionary}, Pali Text Society, London.

                \emph{The Pali Literature of Ceylon}, by G.P. Malalasekera, Royal Asiatic Society, London, 1928. Reprinted by BPS, Kandy, 1994.

                \emph{Pali Literature and Language}, by W. Geiger, English translation by Batakrishna Ghosh, Calcutta University, 1943.

                \emph{Paramatthamañjūsā}, Ācariya Dhammapāla, commentary to the \emph{Visuddhimagga} (\emph{Visuddhimaggamahā-ṭīkā}). Vidyodaya ed. in Sinhalese script, Colombo (Chapters I to XVII only). P.C. Mundyne Pitaka Press ed. in Burmese script, Rangoon, 1909 (Chapters I to XI), 1910 (Chapters XII to XXIII). Siamese ed. in Siamese script, Bangkok. Latin script edition on \emph{Chaṭṭha Saṅgāyana} CDROM of Vipassana Research Institute, Igatpuri. No English translation.

                \emph{Theravada Buddhism in Burma}, by Niharranjan Ray, Calcutta University, 1946 (pp. 24 ff.).

                \emph{Vimuttimagga}, Chinese translation: \emph{Jiĕ-tu-dào-lùn} by Tipiṭaka Saṅghapāla of Funan (6th cent. CE). Taishō edition at T 32, no. 1648, p. 399c–461c (Nanjio no. 1293).

                \emph{The Path of Freedom} (\emph{Vimuttimagga}), privately circulated English translation from the Chinese by N.R.M. Ehara, V.E.P. Pulle and G.S. Prelis. Printed edition, Colombo 1961; reprinted by BPS, Kandy 1995. (Revised, BPS edition forthcoming in 2010.)

                \emph{Vimuttimagga and Visuddhimagga—Comparative Study}, by P.V. Bapat, Poona, 1937. (Reprinted by BPS, 2010)
            \section{List of abbreviations for texts used}

                \emph{All editions Pali Text Society unless otherwise stated.}
                \begin{thebibliography}{xxxxxxxxxxx}
                  \bibitem[A]{A}Aṅguttara Nikāya
                  \bibitem[A-a]{A-a}\emph{Aṅguttara Nikāya Aṭṭhakathā = Manorathapurāṇī}
                  \bibitem[Cp]{Cp}Cariyāpiṭaka
                  \bibitem[Cp-a]{Cp-a}\emph{Cariyāpiṭaka Aṭṭhakathā}
                  \bibitem[Dhp]{Dhp}Dhammapada
                  \bibitem[Dhp-a]{Dhp-a}\emph{Dhammapada Aṭṭhakathā}
                  \bibitem[Dhs]{Dhs}Dhammasaṅgaṇī
                  \bibitem[Dhs-a]{Dhs-a}\emph{Dhammasaṅgaṇi Aṭṭhakathā = Atthasālinī}
                  \bibitem[Dhs-ṭ]{Dhs-ṭ}\emph{Dhammasaṅgaṇī Ṭīkā = Mūla Ṭīkā II}
                  \bibitem[Dhātuk]{Dhātuk}Dhātukathā
                  \bibitem[D]{D}Dīgha Nikāya
                  \bibitem[D-a]{D-a}\emph{Dīgha Nikāya Aṭṭhakathā = Sumaṅgala-vilāsinī}
                  \bibitem[Dīp]{Dīp}\emph{Dīpavaṃsa}
                  \bibitem[It]{It}Itivuttaka
                  \bibitem[J-a]{J-a}\emph{Jātaka-aṭṭhakathā}
                  \bibitem[Kv]{Kv}Kathāvatthu
                  \bibitem[Mhv]{Mhv}\emph{Mahāvaṃsa}
                  \bibitem[M]{M}Majjhima Nikāya
                  \bibitem[M-a]{M-a}\emph{Majjhima Nikāya Aṭṭhakathā = Papañca-sūdanī}
                  \bibitem[Mil]{Mil}\emph{Milindapañhā}
                  \bibitem[Netti]{Netti}\emph{Nettipakaraṇa}
                  \bibitem[Nidd]{Nidd}Niddesa
                    \begin{enumerate}[I.,nosep]
                        \item Mahā Niddesa
                        \item Cūḷa Niddesa (Siamese ed.)
                    \end{enumerate}
                  \bibitem[Nikāya-s]{Nikāya-s}\emph{Nikāyasaṃgrahaya}
                  \bibitem[Paṭis]{Paṭis}Paṭisambhidāmagga
                  \bibitem[Paṭis-a]{Paṭis-a}\emph{Paṭisambhidāmagga Aṭṭhakathā = Saddhammappakāsinī} (Sinhalese Hewavitarne ed.)
                  \bibitem[Paṭṭh]{Paṭṭh}Paṭṭhāna
                    \begin{enumerate}[I.,nosep]
                        \item Paṭṭhāna, Tika Paṭṭhāna
                        \item Paṭṭhāna, Duka Paṭṭhāna (Se and Be.)
                    \end{enumerate}
                  \bibitem[Peṭ]{Peṭ}\emph{Peṭakopadesa}
                  \bibitem[Pv]{Pv}Petavatthu
                  \bibitem[S]{S}Saṃyutta Nikāya
                  \bibitem[S-a]{S-a}\emph{Saṃyutta Nikāya Aṭṭhakathā = Sāratthappakāsinī}
                  \bibitem[Sn]{Sn}Sutta-nipāta
                  \bibitem[Sn-a]{Sn-a}\emph{Sutta-nipāta Aṭṭhakathā = Paramatthajotikā}
                  \bibitem[Th]{Th}Thera-gāthā
                  \bibitem[Ud]{Ud}Udāna
                  \bibitem[Ud-a]{Ud-a}???
                  \bibitem[Vibh]{Vibh}Vibhaṅga
                  \bibitem[Vibh-a]{Vibh-a}\emph{Vibhaṅga Aṭṭhakathā = Sammohavinodanī}
                  \bibitem[Vibh-ṭ]{Vibh-ṭ}\emph{Vibhaṅga Ṭīkā = Mūla Ṭīkā II}
                  \bibitem[Vv]{Vv}Vimānavatthu
                  \bibitem[Vin]{Vin}Vinaya Piṭaka
                    \begin{enumerate}[I.,nosep]
                        \item Vinaya Piṭaka I: (3)—Mahāvagga
                        \item Vinaya Piṭaka II: (4)—Cūḷavagga
                        \item Vinaya Piṭaka III: (1)—Suttavibhaṅga 1
                        \item Vinaya Piṭaka IV: (2)—Suttavibhaṅga 2
                        \item Vinaya Piṭaka V: (5)—Parivāra
                    \end{enumerate}
                  \bibitem[Vin-a]{Vin-a}???
                  \bibitem[Vism]{Vism}\emph{Visuddhimagga}(PTS ed. [= Ee] and Harvard Oriental Series ed. [= Ae])
                  \bibitem[Vism-mhṭ]{Vism-mhṭ}Paramatthamañjūsā, Visuddhimagga Aṭṭhakathā = Mahā Ṭīkā (Chs. I to XVII Sinhalese Vidyodaya ed.; Chs. XVIII to XXIII Be ed.)
                \end{thebibliography}

                \subsection{Other abbreviations}
                    \begin{thebibliography}{xxxxxxxxxxx}
                      \bibitem[Ae]{Ae}American Edition (= Harvard Oriental Series)
                      \bibitem[Be]{Be}Burmese Edition
                      \bibitem[Ce]{Ce}Ceylonese Edition
                      \bibitem[CPD]{CPD}\emph{Critical Pali Dictionary}; Treckner
                      \bibitem[Ee]{Ee}European Edition (= PTS)
                      \bibitem[EHBC]{EHBC}\emph{The Early History of Buddhism in Ceylon}, E. W. Adikaram.
                      \bibitem[PED]{PED}\emph{Pali-English Dictionary}
                      \bibitem[PLC]{PLC}\emph{Pali Literature of Ceylon}, Malalasekera.
                      \bibitem[PTS]{PTS}Pali Text Society
                      \bibitem[Se]{Se}Siamese Edition
                    \end{thebibliography}


                Numbers in square brackets in the text thus [25] refer to the page numbers of the Pali Text Society's edition of the Pali.

                Paragraph numbers on the left correspond to the paragraph numbers of the Harvard edition of the Pali.

                Chapter and section headings and other numberings have been inserted for clarity.\end{vismHanging}
        \chapter[Introduction ]{Introduction }

            \marginnote{\textcolor{teal}{\footnotesize\{29|xxvii\}}}{}The \emph{Visuddhimagga—}here rendered \emph{Path of Purification—}is perhaps unique in the literature of the world. It systematically summarizes and interprets the teaching of the Buddha contained in the Pali \emph{Tipiṭaka, }which is now recognized in Europe as the oldest and most authentic record of the Buddha’s words. As the principal non-canonical authority of the \emph{Theravāda, }it forms the hub of a complete and coherent method of exegesis of the Tipiṭaka, using the “Abhidhamma method” as it is called. And it sets out detailed practical instructions for developing purification of mind.
            \section{Background and Main Facts}

                The works of Bhadantācariya Buddhaghosa fill more than thirty volumes in the Pali Text Society’s Latin-script edition; but what is known of the writer himself is meager enough for a page or two to contain the bare facts.

                Before dealing with those facts, however, and in order that they may appear oriented, it is worth while first to digress a little by noting how Pali literature falls naturally into three main historical periods. The early or classical period, which may be called the First Period, begins with the Tipiṭaka itself in the 6th century BCE and ends with the \emph{Milindapañhā }about five centuries later. These works, composed in India, were brought to Sri Lanka, where they were maintained in Pali but written about in Sinhalese. By the first century CE, Sanskrit (independently of the rise of Mahayana) or a vernacular had probably quite displaced Pali as the medium of study in all the Buddhist “schools” on the Indian mainland. Literary activity in Sri Lanka declined and, it seems, fell into virtual abeyance between CE 150 and 350, as will appear below. The first Pali renascence was under way in Sri Lanka and South India by about 400 and was made viable by Bhadantācariya Buddhaghosa. This can be called the Middle Period. Many of its principal figures were Indian. It developed in several centres in the South Indian mainland and spread to Burma, and it can be said to have lasted till about the 12th century. Meanwhile the renewed literary activity again declined in Sri Lanka till it was eclipsed by the disastrous invasion of Magha in the 11th\textbf{ }century. The second renascence, or the Third Period as it may be termed, begins in the\textbf{ }following century with Sri Lanka’s recovery, coinciding more or less with major political changes in Burma. In Sri Lanka it lasted for several centuries and in Burma for much longer, though India about that time or soon after lost all forms of Buddhism. But this period does not concern the present purpose and is only sketched in for the sake of perspective.

                The recorded facts relating from the standpoint of Sri Lanka to the rise of the Middle Period are very few, and it is worthwhile tabling them.\footnote{\vismAssertFootnoteCounter{1}Exact dates are not agreed. The \emph{Sri Lanka Chronicles} give the lengths of reigns of kings of Sri Lanka back to the time of the Buddha and also of kings of Magadha from Asoka back to the same time. Calculated backwards the list gives 543 BCE as the year of the Buddha’s parinibbāna (see list of kings in Codrington’s \emph{Short History of Ceylon}, Macmillan 1947, p. xvi.). For adjustments to this calculation that bring the date of the parinibbāna forward to 483 BCE (the date most generally accepted in Europe), see e.g. Geiger, \emph{Mahāvaṃsa} translation (introduction) \emph{Epigraphia Zeylanica }I, 156; E. J. Thomas, \emph{Life of the Buddha}, Kegan Paul, p. 26, n.1. It seems certain, however, that Mahānāma was reigning in the year 428 because of a letter sent by him to the Chinese court (Codrington p. 29; E.Z. III, 12). If the adjusted date is accepted then 60 extra years have somehow to be squeezed out without displacing Mahānāma’s reign. Here the older date has been used.} \marginnote{\textcolor{teal}{\footnotesize\{30|xxviii\}}}{}Why did Bhadantācariya Buddhaghosa come to Sri Lanka? And why did his work become famous beyond the island’s shores? The bare facts without some interpretation will hardly answer these questions. Certainly, any interpretation must be speculative; but if this is borne in mind, some attempt (without claim for originality) may perhaps be made on the following lines.

                

                    \ifplastex
                    \begin{tabular}{l|l|l}
                        Kings of Ceylon & Relevant event & Refs.\\
                        Devānampiya Tissa:  BCE 307–267 & Arrival in Sri Lanka of the Arahant Mahinda bringing Pali Tipiṭaka with Commentaries; Commentaries translated into Sinhalese; Great Monastery founded. & \emph{Mahāvaṃsa}, \textbf{\cite{Mhv}XIII}\\
                        Duṭṭhagāmaṇi BCE 161–137 & Expulsion of invaders after 76 years of foreign occupation of capital; restoration of unity and independence. & \textbf{\cite{Mhv}XXV–XXXII}\\
                        & Many names of Great Monastery elders, noted in Commentaries for virtuous behaviour, traceable to this and following reign. & Adikaram, \emph{Early History of Buddhism in Sri Lanka}, \textbf{\cite{EHBC}pp. 65–70}\\
                        Vaṭṭagāmaṇi  BCE 104–88 & Reign interrupted after 5 months by rebellion of Brahman Tissa, famine, invasion, and king’s exile. & \textbf{\cite{Mhv}XXXIII.33f.}\\
                        & Bhikkhus all disperse from Great Monastery to South SL and to India. & \textbf{\cite{A-a}I 92}\\
                        & Restoration of king after 14 years and return of bhikkhus. & \textbf{\cite{Mhv}XXXIII.78}\\
                        & Foundation of Abhayagiri Monastery by king. & \textbf{\cite{Mhv}XXXIII.81}\\
                        & Abhayagiri Monastery secedes from Great Monastery and becomes schismatic. & \textbf{\cite{Mhv}XXXIII.96}\\
                        & Committal by Great Monastery of Pali Tipiṭaka to writing for first time (away from royal capital). & \textbf{\cite{Mhv}XXXIII.100};  \textbf{\cite{Nikāya-s}(translation) 10–11}\\
                        & Abhayagiri Monastery adopts  “Dhammaruci Nikāya of Vajjiputtaka Sect” of India. & \textbf{\cite{Nikāya-s}11}\\
                        & Meeting of Great Monastery bhikkhus  decides that care of texts and preaching  comes before practice of their contents. & \textbf{\cite{A-a}I 92f}; \textbf{\cite{EHBC}78}\\
                        & Many Great Monastery elders’ names noted  in Commentaries for learning and contributions to decision of textual  problems, traceable to this reign. & \textbf{\cite{EHBC}76}\\
                        Kuṭakaṇṇa Tissa BCE 30–33 & Many elders as last stated traceable to this reign too. & \textbf{\cite{EHBC}80}\\
                        & Last Sri Lanka elders’ names in Vinaya Parivāra (p. 2) traceable to this reign; Parivāra can thus have been completed by Great Monastery any time later, before 5th century. & \textbf{\cite{EHBC}86}\\
                        Bhātikābhaya BCE 20–CE 9 & Dispute between Great Monastery and Abhayagiri Monastery over Vinaya adjudged by Brahman Dīghakārāyana in favour of Great Monastery. & \textbf{\cite{Vin-a}582}; \textbf{\cite{EHBC}99}\\
                        Khanirājānu-Tissa 30–33 & 60 bhikkhus punished for treason. & \textbf{\cite{Mhv}XXXV.1}\\
                        Vasabha 66–110 & Last reign to be mentioned in body of Commentaries. & \textbf{\cite{EHBC}3, 86–7}\\
                        & Sinhalese Commentaries can have been closed at any time after this reign. & \textbf{\cite{EHBC}3, 86–7}\\
                        Gajabāhu I  113–135 & Abhayagiri Monastery supported by king and enlarged. & \textbf{\cite{Mhv}XXXV.119}\\
                        6 kings  135–215 & Mentions of royal support for Great Monastery and Abhayagiri Monastery. & \textbf{\cite{Mhv}XXXV.1, 7, 24, 33, 65}\\
                        Vohārika-Tissa 215–237 & King supports both monasteries. &\\
                        & Abhayagiri Monastery has adopted Vetulya (Mahāyāna?) Piṭaka. & \textbf{\cite{Nikāya-s}12}\\
                        & King suppresses Vetulya doctrines. & \textbf{\cite{Mhv}XXXVI.41}\\
                        & Vetulya books burnt and heretic bhikkhus disgraced. & \textbf{\cite{Nikāya-s}12}\\
                        & Corruption of bhikkhus by Vitaṇḍavadins (heretics or destructive critics). & \emph{Dīpavaṃsa, }\textbf{\cite{Dīp}XXII–XXIII}\\
                        Gothābhaya 254–267 & Great Monastery supported by king. & \textbf{\cite{Mhv}XXXVI.102}\\
                        & 60 bhikkhus in Abhayagiri Monastery banished by king for upholding Vetulya doctrines. & \textbf{\cite{Mhv}XXXVI.111}\\
                        & Secession from Abhayagiri Monastery; new sect formed. & \textbf{\cite{Nikāya-s}13}\\
                        & Indian bhikkhu Saṅghamitta supports Abhayagiri Monastery. & \textbf{\cite{Mhv}XXXVI.112}\\
                        Jeṭṭha-Tissa 267–277 & King favours Great Monastery; Saṅghamitta flees to India. & \textbf{\cite{Mhv}XXXVI.123}\\
                        Mahāsena 277–304 & King protects Saṅghamitta, who returns Persecution of Great Monastery; its  bhikkhus driven from capital for 9 years. & \textbf{\cite{Mhv}XXXVII.1–50}\\
                        & Saṅghamitta assassinated. & \textbf{\cite{Mhv}XXXVII.27}\\
                        & Restoration of Great Monastery. & \textbf{\cite{EHBC}92}\\
                        & Vetulya books burnt again. & \textbf{\cite{EHBC}92}\\
                        & Dispute over Great Monastery boundary; bhikkhus again absent from Great Monastery for 9 months. & \textbf{\cite{Mhv}XXXVII.32}\\
                        Siri Meghavaṇṇa 304–332 & King favours Great Monastery. & \textbf{\cite{EHBC}92}; \textbf{\cite{Mhv}XXXVII.51f}\\
                        & Sinhalese monastery established at Buddha Gayā in India. & Malalasekera \textbf{\cite{PLC}, p.68}; Epigraphia Zeylanica iii, II\\
                        Jeṭṭha-Tissa II 332–34 & \emph{Dīpavaṃsa} composed in this period. & Quoted in \textbf{\cite{Vin-a}}\\
                        Buddhadāsa 341–70; Upatissa  370–412 & Also perhaps \emph{Mūlasikkhā} and \emph{Khuddasikkhā} (Vinaya summaries) and some of Buddhadatta Thera’s works. & \textbf{\cite{PLC}, p.77}\\
                        Mahānāma 412–434 & Bhadantācariya Buddhaghosa arrives in Sri Lanka. & \textbf{\cite{Mhv}XXXVII.215–46}\\
                        & \emph{Samantapāsādikā} (Vinaya commentary) begun in 20th and finished in 21st year of this king’s reign. & \textbf{\cite{Vin-a}Epilogue}
                    \end{tabular}
                    \else
                    \begin{longtblr}{colspec={X[2,l]|X[4,l]|X[3,l]}}
                        Kings of Ceylon & Relevant event & Refs.\\
                        Devānampiya Tissa:  BCE 307–267 & Arrival in Sri Lanka of the Arahant Mahinda bringing Pali Tipiṭaka with Commentaries; Commentaries translated into Sinhalese; Great Monastery founded. & \emph{Mahāvaṃsa}, \textbf{\cite{Mhv}XIII}\\
                        Duṭṭhagāmaṇi BCE 161–137 & Expulsion of invaders after 76 years of foreign occupation of capital; restoration of unity and independence. & \textbf{\cite{Mhv}XXV–XXXII}\\
                        & Many names of Great Monastery elders, noted in Commentaries for virtuous behaviour, traceable to this and following reign. & Adikaram, \emph{Early History of Buddhism in Sri Lanka}, \textbf{\cite{EHBC}pp. 65–70}\\
                        Vaṭṭagāmaṇi  BCE 104–88 & Reign interrupted after 5 months by rebellion of Brahman Tissa, famine, invasion, and king’s exile. & \textbf{\cite{Mhv}XXXIII.33f.}\\
                        & Bhikkhus all disperse from Great Monastery to South SL and to India. & \textbf{\cite{A-a}I 92}\\
                        & Restoration of king after 14 years and return of bhikkhus. & \textbf{\cite{Mhv}XXXIII.78}\\
                        & Foundation of Abhayagiri Monastery by king. & \textbf{\cite{Mhv}XXXIII.81}\\
                        & Abhayagiri Monastery secedes from Great Monastery and becomes schismatic. & \textbf{\cite{Mhv}XXXIII.96}\\
                        & Committal by Great Monastery of Pali Tipiṭaka to writing for first time (away from royal capital). & \textbf{\cite{Mhv}XXXIII.100};  \textbf{\cite{Nikāya-s}(translation) 10–11}\\
                        & Abhayagiri Monastery adopts  “Dhammaruci Nikāya of Vajjiputtaka Sect” of India. & \textbf{\cite{Nikāya-s}11}\\
                        & Meeting of Great Monastery bhikkhus  decides that care of texts and preaching  comes before practice of their contents. & \textbf{\cite{A-a}I 92f}; \textbf{\cite{EHBC}78}\\
                        & Many Great Monastery elders’ names noted  in Commentaries for learning and contributions to decision of textual  problems, traceable to this reign. & \textbf{\cite{EHBC}76}\\
                        Kuṭakaṇṇa Tissa BCE 30–33 & Many elders as last stated traceable to this reign too. & \textbf{\cite{EHBC}80}\\
                        & Last Sri Lanka elders’ names in Vinaya Parivāra (p. 2) traceable to this reign; Parivāra can thus have been completed by Great Monastery any time later, before 5th century. & \textbf{\cite{EHBC}86}\\
                        Bhātikābhaya BCE 20–CE 9 & Dispute between Great Monastery and Abhayagiri Monastery over Vinaya adjudged by Brahman Dīghakārāyana in favour of Great Monastery. & \textbf{\cite{Vin-a}582}; \textbf{\cite{EHBC}99}\\
                        Khanirājānu-Tissa 30–33 & 60 bhikkhus punished for treason. & \textbf{\cite{Mhv}XXXV.1}\\
                        Vasabha 66–110 & Last reign to be mentioned in body of Commentaries. & \textbf{\cite{EHBC}3, 86–7}\\
                        & Sinhalese Commentaries can have been closed at any time after this reign. & \textbf{\cite{EHBC}3, 86–7}\\
                        Gajabāhu I  113–135 & Abhayagiri Monastery supported by king and enlarged. & \textbf{\cite{Mhv}XXXV.119}\\
                        6 kings  135–215 & Mentions of royal support for Great Monastery and Abhayagiri Monastery. & \textbf{\cite{Mhv}XXXV.1, 7, 24, 33, 65}\\
                        Vohārika-Tissa 215–237 & King supports both monasteries. &\\
                        & Abhayagiri Monastery has adopted Vetulya (Mahāyāna?) Piṭaka. & \textbf{\cite{Nikāya-s}12}\\
                        & King suppresses Vetulya doctrines. & \textbf{\cite{Mhv}XXXVI.41}\\
                        & Vetulya books burnt and heretic bhikkhus disgraced. & \textbf{\cite{Nikāya-s}12}\\
                        & Corruption of bhikkhus by Vitaṇḍavadins (heretics or destructive critics). & \emph{Dīpavaṃsa, }\textbf{\cite{Dīp}XXII–XXIII}\\
                        Gothābhaya 254–267 & Great Monastery supported by king. & \textbf{\cite{Mhv}XXXVI.102}\\
                        & 60 bhikkhus in Abhayagiri Monastery banished by king for upholding Vetulya doctrines. & \textbf{\cite{Mhv}XXXVI.111}\\
                        & Secession from Abhayagiri Monastery; new sect formed. & \textbf{\cite{Nikāya-s}13}\\
                        & Indian bhikkhu Saṅghamitta supports Abhayagiri Monastery. & \textbf{\cite{Mhv}XXXVI.112}\\
                        Jeṭṭha-Tissa 267–277 & King favours Great Monastery; Saṅghamitta flees to India. & \textbf{\cite{Mhv}XXXVI.123}\\
                        Mahāsena 277–304 & King protects Saṅghamitta, who returns Persecution of Great Monastery; its  bhikkhus driven from capital for 9 years. & \textbf{\cite{Mhv}XXXVII.1–50}\\
                        & Saṅghamitta assassinated. & \textbf{\cite{Mhv}XXXVII.27}\\
                        & Restoration of Great Monastery. & \textbf{\cite{EHBC}92}\\
                        & Vetulya books burnt again. & \textbf{\cite{EHBC}92}\\
                        & Dispute over Great Monastery boundary; bhikkhus again absent from Great Monastery for 9 months. & \textbf{\cite{Mhv}XXXVII.32}\\
                        Siri Meghavaṇṇa 304–332 & King favours Great Monastery. & \textbf{\cite{EHBC}92}; \textbf{\cite{Mhv}XXXVII.51f}\\
                        & Sinhalese monastery established at Buddha Gayā in India. & Malalasekera \textbf{\cite{PLC}, p.68}; Epigraphia Zeylanica iii, II\\
                        Jeṭṭha-Tissa II 332–34 & \emph{Dīpavaṃsa} composed in this period. & Quoted in \textbf{\cite{Vin-a}}\\
                        Buddhadāsa 341–70; Upatissa  370–412 & Also perhaps \emph{Mūlasikkhā} and \emph{Khuddasikkhā} (Vinaya summaries) and some of Buddhadatta Thera’s works. & \textbf{\cite{PLC}, p.77}\\
                        Mahānāma 412–434 & Bhadantācariya Buddhaghosa arrives in Sri Lanka. & \textbf{\cite{Mhv}XXXVII.215–46}\\
                        & \emph{Samantapāsādikā} (Vinaya commentary) begun in 20th and finished in 21st year of this king’s reign. & \textbf{\cite{Vin-a}Epilogue}
                    \end{longtblr}\fi
                    \noindent
                    

                Up till the reign of King Vaṭṭagāmaṇi Abhaya in the first century BCE the Great Monastery, founded by Asoka’s son, the Arahant Mahinda, and hitherto without a rival for the royal favour, had preserved a reputation for the saintliness of its \marginnote{\textcolor{teal}{\footnotesize\{32|xxx\}}}{}bhikkhus. The violent upsets in his reign followed by his founding of the Abhayagiri Monastery, its secession and schism, changed the whole situation at home. Sensing insecurity, the Great Monastery took the precaution to commit the Tipiṭaka for the first time to writing, doing so in the provinces away from the king’s presence. Now by about the end of the first century BCE (dates are very vague), with Sanskrit Buddhist literature just launching out upon its long era of magnificence, Sanskrit was on its way to become a language of international culture. In Sri Lanka the Great Monastery, already committed by tradition to strict orthodoxy based on Pali, had been confirmed in that attitude by the schism of its rival, which now began publicly to study the new ideas from India. In the first century BCE probably the influx of Sanskrit thought was still quite small, so that the Great Monastery could well maintain its name in Anurādhapura as the principal centre of learning by developing its ancient Tipiṭaka commentaries in Sinhalese. This might account for the shift of emphasis from practice to scholarship in King Vaṭṭagāmani’s reign. Evidence shows great activity in this latter field throughout the first century BCE, and all this material was doubtless written down too.

                In the first century CE, Sanskrit Buddhism (“Hīnayāna,” and perhaps by then Mahāyāna) was growing rapidly and spreading abroad. The Abhayagiri Monastery would naturally have been busy studying and advocating some of these weighty \marginnote{\textcolor{teal}{\footnotesize\{33|xxxi\}}}{}developments while the Great Monastery had nothing new to offer: the rival was thus able, at some risk, to appear go-ahead and up-to-date while the old institution perhaps began to fall behind for want of new material, new inspiration and international connections, because its studies being restricted to the orthodox presentation in the Sinhalese language, it had already done what it could in developing Tipiṭaka learning (on the mainland Theravāda was doubtless deeper in the same predicament). Anyway we find that from the first century onwards its constructive scholarship dries up, and instead, with the reign of King Bhātika Abhaya (BCE 20–CE 9), public wrangles begin to break out between the two monasteries. This scene indeed drags on, gradually worsening through the next three centuries, almost bare as they are of illuminating information. King Vasabha’s reign (CE 66–110) seems to be the last mentioned in the Commentaries as we have them now, from which it may be assumed that soon afterwards they were closed (or no longer kept up), nothing further being added. Perhaps the Great Monastery, now living only on its past, was itself getting infected with heresies. But without speculating on the immediate reasons that induced it to let its chain of teachers lapse and to cease adding to its body of Sinhalese learning, it is enough to note that the situation went on deteriorating, further complicated by intrigues, till in Mahāsena’s reign (CE 277–304) things came to a head.

                With the persecution of the Great Monastery given royal assent and the expulsion of its bhikkhus from the capital, the Abhayagiri Monastery enjoyed nine years of triumph. But the ancient institution rallied its supporters in the southern provinces and the king repented. The bhikkhus returned and the king restored the buildings, which had been stripped to adorn the rival. Still, the Great Monastery must have foreseen, after this affair, that unless it could successfully compete with Sanskrit it had small hope of holding its position. With that the only course open was to launch a drive for the rehabilitation of Pali—a drive to bring the study of that language up to a standard fit to compete with the “modern” Sanskrit in the field of international Buddhist culture: by cultivating Pali at home and abroad it could assure its position at home. It was a revolutionary project, involving the displacement of Sinhalese by Pali as the language for the study and discussion of Buddhist teachings, and the founding of a school of Pali literary composition. Earlier it would doubtless have been impracticable; but the atmosphere had changed. Though various Sanskrit non-Mahayana sects are well known to have continued to flourish all over India, there is almost nothing to show the status of the Pali language there by now. Only the \emph{Mahāvaṃsa }[XXXVII.215f. quoted below] suggests that the Theravāda sect there had not only put aside but lost perhaps all of its old non-Piṭaka material dating from Asoka’s time.\footnote{\vismAssertFootnoteCounter{2}See also \emph{A Record of Buddhist Religion} by I-tsing, translation by J. Takakusu, Claren do Press, 1896, p. xxiii, where a geographical distribution of various schools gives Mūlasarvāstivāda mainly in the north and Ariyasthavira mainly in the south of India. I-tsing, who did not visit Sri Lanka, was in India at the end of the 7th cent.; but he does not mention whether the Ariyasthavira (Theravāda) Nikāya in India pursued its studies in the Pali of its Tipiṭaka or in Sanskrit or in a local vernacular.} One may guess that the pattern of things in Sri Lanka only echoed a process that had gone much further in India. But in the \marginnote{\textcolor{teal}{\footnotesize\{34|xxxii\}}}{}island of Sri Lanka the ancient body of learning, much of it pre-Asokan, had been kept lying by, as it were maturing in its two and a half centuries of neglect, and it had now acquired a new and great potential value due to the purity of its pedigree in contrast with the welter of new original thinking. Theravāda centres of learning on the mainland were also doubtless much interested and themselves anxious for help in a repristinization.\footnote{\vismAssertFootnoteCounter{3}In the epilogues and prologues of various works between the 5th and 12th centuries there is mention of e.g., Badaratittha (Vism-a prol.: near Chennai), Kañcipura (\textbf{\cite{A-a}} epil.: = Conjevaram near Chennai), and other places where different teachers accepting the Great Monastery tradition lived and worked. See also Malalasekera, \emph{Pali Literature of Ceylon,} p. 13; E.Z., IV, 69-71; Journal of Oriental Research, Madras, Vol. XIX, pp. 278f.}Without such cooperation there was little hope of success.

                It is not known what was the first original Pali composition in this period; but the \emph{Dīpavaṃsa }(dealing with historical evidence) belongs here (for it ends with Mahāsena’s reign and is quoted in the \emph{Samantapāsādikā), }and quite possibly the \emph{Vimuttimagga }(dealing with practice—see below) was another early attempt by the Great Monastery in this period (4th cent.) to reassert its supremacy through original Pali literary composition: there will have been others too.\footnote{\vismAssertFootnoteCounter{4}Possibly the Vinaya summaries, \emph{Mūlasikkhā} and \emph{Khuddasikkhā} (though Geiger places these much later), as well as some works of Buddhadatta Thera. It has not been satisfactorily explained why the \emph{Mahāvaṃsa,} composed in the late 4th or early 5th cent., ends abruptly in the middle of Chapter 37 with Mahāsena’s reign (the Chronicle being only resumed eight centuries later).}Of course, much of this is very conjectural. Still it is plain enough that by 400 CE a movement had begun, not confined to Sri Lanka, and that the time was ripe for the crucial work, for a Pali recension of the Sinhalese Commentaries with their unique tradition. Only the right personality, able to handle it competently, was yet lacking. That personality appeared in the first quarter of the fifth century.
            \section{The \emph{Visuddhimagga} and its Author}

                Sources of information about that person fall into three groups. There are firstly the scraps contained in the prologues and epilogues to the works ascribed to him. Then there is the account given in the second part of the Sri Lankan Chronicle, the \emph{Mahāvaṃsa }(or \emph{Cūḷavaṃsa }as the part of it is often called), written in about the 13th century, describing occurrences placed by it in the 5th century, and, lastly, the still later \emph{Buddhaghosuppatti }(15th cent.?) and other later works.

                It seems still uncertain how to evaluate the old Talaing records of Burma, which may not refer to the same person (see below). India herself tells us nothing at all.

                It seems worthwhile, therefore, to give a rendering here of the principal passage from the prologues and epilogues of the works ascribed to him by name; for they are few and short, and they have special authentic value as evidence. The \emph{Mahāvaṃsa }account will be reproduced in full, too, since it is held to have been composed from evidence and records before its author, and to have the ring of truth behind the legends it contains. But the later works (which European scholars hold to be legendary rather than historical in what they add to the accounts already mentioned) can only be dealt with very summarily here. \marginnote{\textcolor{teal}{\footnotesize\{35|xxxiii\}}}{}The books actually ascribed to Bhadantācariya Buddhaghosa have each a “postscript” identical in form with that at the end of \hyperlink{XXIII}{Chapter XXIII}{} of the present work, mentioning the title and author by name. This can be taken to have been appended, presumably contemporaneously, by the Great Monastery (the \emph{Mahāvaṃsa) }at Anurādhapura in Sri Lanka as their official seal of approval. Here is a list of the works (also listed in the modern \emph{Gandhavaṃsa }and \emph{Sāsanavaṃsa }with one or two discrepancies):\footnote{\vismAssertFootnoteCounter{5}The \emph{Gandhavaṃsa }also gives the Apadāna Commentary as by him.}

                \textbf{\emph{Commentaries to the Vinaya Piṭaka}}

                \ifplastex
                \begin{tabular}{ll}
                    Title & Commentary to\\
                    \emph{Samantapāsādikā} & Vinaya\\
                    \emph{Kaṅkhāvitaraṇī} & Pātimokkha
                \end{tabular}
                \else
                \begin{tblr}{colspec={Q[15em]Q[15em]}}
                    Title & Commentary to\\
                    \emph{Samantapāsādikā} & Vinaya\\
                    \emph{Kaṅkhāvitaraṇī} & Pātimokkha
                \end{tblr}\fi
                \noindent
                

                \textbf{\emph{Commentaries to the Sutta Piṭaka}}

                \ifplastex
                \begin{tabular}{ll}
                    Title & Commentary to\\
                    \emph{Sumaṅgalavilāsinī} & Dīgha Nikāya\\
                    \emph{Papañcasūdani} & Majjhima Nikāya\\
                    \emph{Sāratthappakāsinī} & Saṃyutta Nikāya\\
                    \emph{Manorathapurāṇī} & Aṅguttara Nikāya\\
                    \emph{Paramatthajotikā} & Khuddakapāṭha
                \end{tabular}
                \else
                \begin{tblr}{colspec={Q[15em]Q[15em]}}
                    Title & Commentary to\\
                    \emph{Sumaṅgalavilāsinī} & Dīgha Nikāya\\
                    \emph{Papañcasūdani} & Majjhima Nikāya\\
                    \emph{Sāratthappakāsinī} & Saṃyutta Nikāya\\
                    \emph{Manorathapurāṇī} & Aṅguttara Nikāya\\
                    \emph{Paramatthajotikā} & Khuddakapāṭha
                \end{tblr}\fi
                \noindent
                

                \textbf{\emph{Commentary to Suttanipāta}}

                \ifplastex
                \begin{tabular}{ll}
                    Title & Commentary to\\
                    \emph{Dhammapadaṭṭhakathā} & Dhammapada\\
                    \emph{Jātakaṭṭhakathā} & Jātaka
                \end{tabular}
                \else
                \begin{tblr}{colspec={Q[15em]Q[15em]}}
                    Title & Commentary to\\
                    \emph{Dhammapadaṭṭhakathā} & Dhammapada\\
                    \emph{Jātakaṭṭhakathā} & Jātaka
                \end{tblr}\fi
                \noindent
                

                \textbf{\emph{Commentaries to the Abhidhamma Piṭaka}}

                \ifplastex
                \begin{tabular}{ll}
                    Title & Commentary to\\
                    \emph{Atthasālinī} & Dhammasaṅgaṇī\\
                    \emph{Sammohavinodanī} & Vibhaṅga\\
                    \emph{Pañcappakaraṇaṭṭhakathā} & Remaining 5 books
                \end{tabular}
                \else
                \begin{tblr}{colspec={Q[15em]Q[15em]}}
                    Title & Commentary to\\
                    \emph{Atthasālinī} & Dhammasaṅgaṇī\\
                    \emph{Sammohavinodanī} & Vibhaṅga\\
                    \emph{Pañcappakaraṇaṭṭhakathā} & Remaining 5 books
                \end{tblr}\fi
                \noindent
                

                Beyond the bare hint that he came to Sri Lanka from India his actual works tell nothing about his origins or background. He mentions “The Elder Buddhamitta with whom I formerly lived at Mayūra suttapaṭṭana” (\textbf{\cite{M-a}} epil.),\footnote{\vismAssertFootnoteCounter{6}Other readings are: Mayūrarūpaṭṭana, Mayūradūtapaṭṭana. Identified with Mylapore near Chennai (J.O.R., Madras, Vol. XIX, p. 281).} and “The well known Elder Jotipāla, with whom I once lived at Kañcipura and elsewhere” (\textbf{\cite{A-a}} epil.).\footnote{\vismAssertFootnoteCounter{7}Identified with Conjevaram near Chennai: PLC, p. 113. Ācariya Ānanda, author of the sub-commentary to the Abhidhamma Pitaka (\emph{Mūla Ṭīkā}), also lived there, perhaps any time after the middle of the 5th century. The Elder Dhammapāla sometimes refers to the old Sinhalese commentaries as if they were still available to him.}Also the “postscript” attached to the \emph{Visuddhimagga }says, besides mentioning his name, that he “should be called ‘of Moraṇḍacetaka.’” \footnote{\vismAssertFootnoteCounter{8}Other readings are: Moraṇḍakheṭaka, Mudantakhedaka, Muraṇḍakheṭaka, etc.; not yet identified. Refers more probably to his birthplace than to his place of pabbajjā. See also J.O.R., Madras, Vol. XIX, p. 282, article “Buddhaghosa—His Place of Birth” by R. Subramaniam and S. P. Nainar, where a certain coincidence of names is mentioned that might suggest a possible identification of Moraṇḍakheṭaka (\emph{moraṇḍa} being Pali for ‘peacock egg’ and \emph{khedaka} Skr. for “village”—see Vism Ae ed., p. xv) with adjacent villages, 51 miles from Nāgārjunakoṇḍa and 58 miles from Amarāvatī, called Kotanemalipuri and Gundlapalli (\emph{nemali} and \emph{gundla} being Telegu respectively for “peacock” and “egg”). However, more specific information will be needed in support before it can be accepted as an indication that the \emph{Mahāvaṃsa} is wrong about his birthplace. More information about any connection between Sri Lanka and those great South Indian Buddhist centres is badly needed.} And that is all. \marginnote{\textcolor{teal}{\footnotesize\{36|xxxiv\}}}{}On coming to Sri Lanka, he went to Anurādhapura, the royal capital, and set himself to study. He seems to have lived and worked there during the whole of his stay in the island, though we do not know how long that stay lasted. To render his own words: “I learned three Sinhalese commentaries—the \emph{Mahā-aṭṭha-[kathā], Mahāpaccarī, Kuruṇḍī—}from the famed elder known by the name of Buddhamitta, who has expert knowledge of the Vinaya. Set in the grounds of the Mahā Meghavana Park [in Anurādhapura] there is the Great Monastery graced by the [sapling from the] Master’s Enlightenment Tree. A constant supporter of the Community, trusting with unwavering faith in the Three Jewels, belonging to an illustrious family and known by the name of Mahānigamasāmi (Lord of the Great City), had an excellent work-room built there on its southern side accessible to the ever virtuously conducted Community of Bhikkhus. The building was beautifully appointed, agreeably endowed with cool shade and had a lavish water supply. The Vinaya Commentary was begun by me for the sake of the Elder Buddhasiri of pure virtuous behaviour while I was living there in Mahānigamasāmi’s building, and it is now complete. It was begun by me in the twentieth year of the reign of peace of the King Sirinivāsa (Of Glorious Life), the renowned and glorious guardian who has kept the whole of Lanka’s island free from trouble. It was finished in one year without mishap in a world beset by mishaps, so may all beings attain…’’\emph{ (}\textbf{\cite{Vin-a}Epilogue}).

                Mostly it is assumed that he wrote and “published” his works one by one as authors do today. The assumption may not be correct. There is an unerring consistency throughout the system of explanation he adopts, and there are cross-references between works. This suggests that while the \emph{Visuddhimagga }itself may perhaps have been composed and produced first, the others as they exist now were more likely worked over contemporaneously and all more or less finished before any one of them was given out. They may well have been given out then following the order of the books in the Tipiṭaka which they explain. So in that way it may be taken that the Vinaya Commentary came next to the \emph{Visuddhimagga; }then the Commentaries on the four Nikāyas (Collections of Suttas), and after them the Abhidhamma Commentaries. Though it is not said that the Vinaya Commentary was given out first of these, still the prologue and epilogue contain the most information. The four Nikāya Commentaries all have the same basic prologue; but the Saṃyutta Nikāya Commentary inserts in its prologue a stanza referring the reader to “the two previous Collections” (i.e. the Dīgha and Majjhima Nikāyas) for explanations of the names of towns and for illustrative stories, while the Aṅguttara \marginnote{\textcolor{teal}{\footnotesize\{37|xxxv\}}}{}Nikāya Commentary replaces this stanza with another referring to “the Dīgha and Majjhima” by name for the same purpose. The point may seem laboured and even trivial, but it is not irrelevant; for if it is assumed that these works were written and “published” in some historical order of composition, one expects to find some corresponding development of thought and perhaps discovers what one’s assumption has projected upon them. The more likely assumption, based on consideration of the actual contents, is that their form and content was settled before any one of them was given out.

                Sometimes it is argued that the commentaries to the Dhammapada and the Jātaka may not be by the same author because the style is different. But that fact could be accounted for by the difference in the subject matter; for these two commentaries consist mainly of popular stories, which play only a very minor role in the other works. Besides, while this author is quite inexorably consistent throughout his works in his explanations of Dhamma, he by no means always maintains that consistency in different versions of the same story in, say, different Nikāya Commentaries (compare for instance, the version of the story of Elder Tissabhūti given in the commentary to \textbf{\cite{A}1:2.6}, with that at \textbf{\cite{M-a}I 66}; also the version of the story of the Elder Mahā Tissa in the \textbf{\cite{A-a}}, same ref., with that at \textbf{\cite{M-a}I 185}). Perhaps less need for strictness was felt with such story material. And there is also another possibility. It may not unreasonably be supposed that he did not work alone, without help, and that he had competent assistants. If so, he might well have delegated the drafting of the Khuddaka Nikāya commentaries—those of the Khuddakapāṭha and Suttanipāta, Dhammapada, and the Jātaka—or part of them, supervising and completing them himself, after which the official “postscript” was appended. This assumption seems not implausible and involves less difficulties than its alternatives.\footnote{\vismAssertFootnoteCounter{9}A definite statement that the \textbf{\cite{Dhp-a}} was written later by someone else can hardly avoid the inference that the “postscript” was a fraud, or at least misleading.}These secondary commentaries may well have been composed after the others.

                The full early history of the Pali Tipiṭaka and its commentaries in Sinhalese is given in the Sri Lanka Chronicle, the \emph{Dīpavaṃsa, }and \emph{Mahāvaṃsa, }and also in the introduction to the Vinaya Commentary. In the prologue to each of the four Nikāya Commentaries it is conveniently summarized by Bhadantācariya Buddhaghosa himself as follows: “[I shall now take] the commentary, whose object is to clarify the meaning of the subtle and most excellent Long Collection (Dīgha Nikāya) … set forth in detail by the Buddha and by his like [i.e. the Elder Sāriputta and other expounders of discourses in the Sutta Piṭaka]—the commentary that in the beginning was chanted [at the First Council] and later re-chanted [at the Second and Third], and was brought to the Sīhala Island (Sri Lanka) by the Arahant Mahinda the Great and rendered into the Sīhala tongue for the benefit of the islanders—and from that commentary I shall remove the Sīhala tongue, replacing it by the graceful language that conforms with Scripture and is purified and free from flaws. Not diverging from the standpoint of the elders residing in the Great Monastery [in Anurādhapura], who illumine the elders’ heritage and are all well \marginnote{\textcolor{teal}{\footnotesize\{38|xxxvi\}}}{}versed in exposition, and rejecting subject matter needlessly repeated, I shall make the meaning clear for the purpose of bringing contentment to good people and contributing to the long endurance of the Dhamma.”

                There are references in these works to “the Ancients” \emph{(porāṇā) }or “Former Teachers” \emph{(pubbācariyā) }as well as to a number of Sinhalese commentaries additional to the three referred to in the quotation given earlier. The fact is plain enough that a complete body of commentary had been built up during the nine centuries or so that separate Bhadantācariya Buddhaghosa from the Buddha. A good proportion of it dated no doubt from the actual time of the Buddha himself, and this core had been added to in India (probably in Pali), and later by learned elders in Sri Lanka (in Sinhalese) as references to their pronouncements show (e.g. \hyperlink{XII.105}{XII.105}{} and \hyperlink{XII.117}{117}{}).

                This body of material—one may guess that its volume was enormous—Bhadantācariya Buddhaghosa set himself to edit and render into Pali (the Tipiṭaka itself had been left in the original Pali). For this he had approval and express invitation (see, e.g., the epilogue to the present work, which the Elder Saṅghapāla invited him to compose). Modern critics have reproached him with lack of originality: but if we are to judge by his declared aims, originality, or to use his own phrase “advertising his own standpoint” (\hyperlink{XVII.25}{XVII.25}{}), seems likely to have been one of the things he would have wished to avoid. He says, for instance, “I shall expound the comforting \emph{Path of Purification, }pure in expositions, relying on the teaching of the dwellers in the Great Monastery” (\hyperlink{I.4}{I.4}{}; see also epilogue), and again “Now, as to the entire trustworthiness \emph{(samantapāsādikatta) }of this \emph{Samantapāsādika}: the wise see nothing untrustworthy here when they look—in the chain of teachers, in the citations of circumstance, instance and category [in each case], in the avoidance of others’ standpoints, in the purity of [our] own standpoint, in the correctness of details, in the word-meanings, in the order of construing the text, in the exposition of the training precepts, in the use of classification by the analytical method—which is why this detailed commentary on the Vinaya … is called \emph{Samantapāsādika }(\textbf{\cite{Vin-a}} epilogue). And then: “The commentary on the Pātimokkha, which I began at the request of the Elder Soṇa for the purpose of removing doubts in those uncertain of the Vinaya, and which covers the whole Sinhalese commentarial system based upon the arrangement adopted by the dwellers in the Great Monastery, is finished. The whole essence of the commentary and the entire meaning of the text has been extracted and there is no sentence here that might conflict with the text or with the commentaries of the dwellers in the Great Monastery or those of the Ancients” (Pātimokkha Commentary epilogue). Such examples could be multiplied (see especially also \hyperlink{XVII.25}{XVII.25}{}).

                There is only one instance in the \emph{Visuddhimagga }where he openly advances an opinion of his own, with the words “our preference here is this” (\hyperlink{XIII.123}{XIII.123}{}). He does so once in the Majjhima Nikāya Commentary, too, saying “the point is not dealt with by the Ancients, but this is my opinion” (\textbf{\cite{M-a}I 28}). The rarity of such instances and the caution expressed in them imply that he himself was disinclined to speculate and felt the need to point the fact out when he did. He actually says “one’s own opinion is the weakest authority of all and should only be accepted if it accords with the Suttas” (\textbf{\cite{D-a}567–568}). So it is likely that \marginnote{\textcolor{teal}{\footnotesize\{39|xxxvii\}}}{}he regarded what we should call original thinking as the province of the Buddha, and his own task as the fortification of that thought by coordinating the explanations of it. However, not every detail that he edited can claim direct support in the Suttas.

                The following considerations lend some support to the assumptions just made. It has been pointed out\footnote{\vismAssertFootnoteCounter{10}Adikaram, \emph{Early History of Buddhism in Ceylon}, pp. 3 and 86.}that in describing in the Vinaya Commentary how the tradition had been “maintained up to the present day by the chain of teachers and pupils” (\textbf{\cite{Vin-a}61–62}) the list of teachers’ names that follows contains names only traceable down to about the middle of the 2nd century CE, but not later. Again, there appear in his works numbers of illustrative stories, all of which are set either in India or Sri Lanka. However, no single one of them can be pointed to as contemporary. Stories about India in every case where a date can be assigned are not later than Asoka (3rd cent. BCE). Many stories about Sri Lanka cannot be dated, but of those that can none seems later than the 2nd century CE. This suggests that the material which he had before him to edit and translate had been already completed and fixed more than two centuries earlier in Sri Lanka, and that the words “present day” were not used by him to refer to his own time, but were already in the material he was coordinating. This final fixing, if it is a fact, might have been the aftermath of the decision taken in Sri Lanka in the first century BCE to commit the Pali Tipiṭaka to writing.

                Something now needs to be said about the relation of the \emph{Visuddhimagga }to the other books. This author’s work is characterized by relentless accuracy, consistency, and fluency of erudition, and much dominated by formalism. Not only is this formalism evident in the elaborate pattern of the \emph{Visuddhimagga }but also that work’s relationship to the others is governed by it. The \emph{Visuddhimagga }itself extracts from the Tipiṭaka all the central doctrines that pivot upon the Four Noble Truths, presenting them as a coherent systematic whole by way of quotation and explanation interspersed with treatises on subjects of more or less relative importance, all being welded into an intricate edifice. The work can thus stand alone. But the aim of the commentaries to the four main Nikāyas or Collections of Suttas is to explain the subject matter of individual discourses and, as well, certain topics and special doctrines not dealt with in the \emph{Visuddhimagga }(many passages commenting on identical material in the Suttas in different Nikāyas are reproduced \emph{verbatim }in each commentary, and elsewhere, e.g., MN 10, cf. DN 22, Satipaṭṭhāna Vibhaṅga, etc., etc., and respective commentaries). But these commentaries always refer the reader to the \emph{Visuddhimagga }for explanations of the central doctrines. And though the Vinaya and Abhidhamma (commentaries are less closely bound to the \emph{Visuddhimagga, }still they too either refer the reader to it or reproduce large blocks of it. The author himself says: “The treatises on virtue and on the ascetic’s rules, all the meditation subjects, the details of the attainments of the jhānas, together with the directions for each temperament, all the various kinds of direct-knowledge, the exposition of the definition of understanding, the aggregates, elements, bases, and faculties, the Four Noble Truths, the explanation \marginnote{\textcolor{teal}{\footnotesize\{40|xxxviii\}}}{}of the structure of conditions (dependent origination), and lastly the development of insight, by methods that are purified and sure and not divergent from Scripture—since these things have already been quite clearly stated in the \emph{Visuddhimagga }I shall no more dwell upon them here; for the \emph{Visuddhimagga }stands between and in the midst of all four Collections (Nikāyas) and will clarify the meaning of such things stated therein. It was made in that way: take it therefore along with this same commentary and know the meaning of the Long Collection (Dīgha Nikāya)” (prologue to the four Nikāyas).

                This is all that can, without unsafe inferences, be gleaned of Bhadantācariya Buddhaghosa himself from his own works (but see below). Now, there is the \emph{Mahāvaṃsa }account. The composition of the second part (often called \emph{Cūḷavaṃsa) }of that historical poem is attributed to an Elder Dhammakitti, who lived in or about the thirteenth century. Here is a translation of the relevant passage:

                “There was a Brahman student who was born near the site of the Enlightenment Tree. He was acquainted with the arts and accomplishments of the sciences and was qualified in the Vedas. He was well versed in what he knew and unhesitant over any phrase. Being interested in doctrines, he wandered over Jambudīpa (India) engaging in disputation.

                “He came to a certain monastery, and there in the night he recited Pātañjali’s system with each phrase complete and well rounded. The senior elder there, Revata by name, recognized, ‘This is a being of great understanding who ought to be tamed.’ He said, ‘Who is that braying the ass’s bray?’ The other asked, ‘What, then, do you know the meaning of the ass’s bray?’ The elder answered, ‘I know it,’ and he then not only expounded it himself, but explained each statement in the proper way and also pointed out contradictions. The other then urged him, ‘Now expound your own doctrine,’ and the elder repeated a text from the Abhidhamma, but the visitor could not solve its meaning. He asked, ‘Whose system is this?’ and the elder replied, ‘It is the Enlightened One’s system.’ ‘Give it to me,’ he said, but the elder answered, ‘You will have to take the going forth into homelessness.’ So he took the going forth, since he was interested in the system, and he learned the three Piṭakas, after which he believed, ‘This is the only way’ (\textbf{\cite{M}I 55}). Because his speech \emph{(ghosa) }was profound (voice was deep) like that of the Enlightened One \emph{(Buddha) }they called him Buddhaghosa, so that like the Enlightened One he might be voiced over the surface of the earth.

                “He prepared a treatise there called \emph{Ñāṇodaya, }and then the \emph{Atthasālinī, }a commentary on the Dhammasaṅgaṇī. Next he began work on a commentary to the \emph{Paritta}.\footnote{\vismAssertFootnoteCounter{11}\emph{Paritta} or “protection”: a name for certain suttas recited for that purpose. See \textbf{\cite{M-a}IV 114}.} When the Elder Revata saw that, he said, ‘Here only the text has been preserved. There is no commentary here, and likewise no Teachers’ Doctrine; for that has been allowed to go to pieces and is no longer known. However, a Sinhalese commentary still exists, which is pure. It was rendered into the Sinhalese tongue by the learned Mahinda with proper regard for the\marginnote{\textcolor{teal}{\footnotesize\{41|xxxix\}}}{} way of commenting that was handed down by the three Councils as taught by the Enlightened One and inculcated by Sāriputta and others. Go there, and after you have learnt it translate it into the language of the Magadhans. That will bring benefit to the whole world.’ As soon as this was said, he made up his mind to set out.

                \emph{“}He came from there to this island in the reign of this king (Mahānāma). He came to the (Great Monastery, the monastery of all true men. There he stayed in a large workroom, and he learnt the whole Sinhalese Commentary of the Elders’ Doctrine \emph{(theravāda) }under Saṅghapāla.\footnote{\vismAssertFootnoteCounter{12}See Vism epilogue.} He decided, ‘This alone is the intention of the Dhamma’s Lord.’ So he assembled the Community there and asked, ‘Give me all the books to make a commentary.’ Then in order to test him the Community gave him two stanzas, saying ‘Show your ability with these; when we have seen that you have it, we will give you all the books.’ On that text alone he summarized the three Piṭakas together with the Commentary as an epitome, which was named the \emph{Path of Purification (Visuddhimagga). }Then, in the precincts of the (sapling of the) Enlightenment Tree (in Anurādhapura), he assembled the Community expert in the Fully Enlightened One’s system, and he began to read it out. In order to demonstrate his skill to the multitude deities hid the book, and he was obliged to prepare it a second time, and again a third time. When the book was brought for the third time to be read out, the gods replaced the other two copies with it. Then the bhikkhus read out the three copies together, and it was found that there was no difference between the three in either the chapters or the meaning or the order of the material or the phrases and syllables of the Theravāda texts. With that the Community applauded in high delight and again and again it was said, ‘Surely this is (the Bodhisatta) Metteyya.’ “They gave him the books of the three Piṭakas together with the Commentary. Then, while staying undisturbed in the Library Monastery, he translated the Sinhalese Commentary into the Magadhan language, the root-speech of all, by which he brought benefit to beings of all tongues. The teachers of the Elders’ Tradition accepted it as equal in authority with the texts themselves. Then, when the tasks to be done were finished, he went back to Jambudīpa to pay homage to the Great Enlightenment Tree.

                \emph{“}And when Mahānāma had enjoyed twenty-two years’ reign upon earth and had performed a variety of meritorious works, he passed on according to his deeds”—(\textbf{\cite{Mhv}XXXVII.215–47}).

                King Mahānāma is identified with the “King Sirinivāsa” and the “King Sirikuḍḍa” mentioned respectively in the epilogues to the Vinaya and Dhammapada Commentaries. There is no trace, and no other mention anywhere, of the \emph{Ñāṇodaya. }The \emph{Atthasālinī }described as composed in India could not be the version extant today, which cites the Sri Lankan Commentaries and refers to the \emph{Visuddhimagga; }it will have been revised later.

                The prologues and epilogues of this author’s works are the only instances in which we can be sure that he is speaking of his own experience and not only simply editing; and while they point only to his residence in South India, they neither \marginnote{\textcolor{teal}{\footnotesize\{42|xl\}}}{}confute nor confirm the \emph{Mahāvaṃsa }statement that he was born in Magadha (see note 8). The Sri Lankan Chronicles survived the historical criticism to which they were subjected in the last hundred years. The independent evidence that could be brought to bear supported them, and Western scholars ended by pronouncing them reliable in essentials. The account just quoted is considered to be based on historical fact even if it contains legendary matter.

                It is not possible to make use of the body of Bhadantācariya Buddhaghosa’s works to test the \emph{Mahāvaṃsa’s} claim that he was a learned Brahman from central India, and so on. It has been shown already how the presumption is always, where the contrary is not explicitly stated, that he is editing and translating material placed before him rather than displaying his own private knowledge, experience and opinions. And so it would be a critical mistake to use any such passage in his work for assessing his personal traits; for in them it is, pretty certainly, not him we are dealing with at all but people who lived three or more centuries earlier. Those passages probably tell us merely that he was a scrupulously accurate and conscientious editor. His geographical descriptions are translations, not eyewitness accounts. Then such a sutta passage as that commented on in \hyperlink{I.86}{Chapter I, 86}{}–\hyperlink{I.97}{97}{} of the present work, which is a part of a sutta used by bhikkhus for daily reflection on the four requisites of the life of a bhikkhu, is certain to have been fully commented on from the earliest times, so that it would be just such a critical mistake to infer from this comment anything about his abilities as an original commentator, or anything else of a personal nature about him or his own past experience.\footnote{\vismAssertFootnoteCounter{13}For instance, Prof. Kosambi, in his preface to the \emph{Visuddhimagga}, Harvard ed., overlooks these considerations when he says: “More positive evidence (that he was not a North-Indian Brahman) is in the passage \emph{’Uṇhassa ti aggisantāpassa. Tassa vanadāhādisu sambhavo veditabbo’ }(\hyperlink{I.86}{I.86}{}). ’Heat: the heat of fire, such as occurs at the time of forest fires, etc.’” This is a comment upon protection against heat given by a \emph{cīvara}. His explanation is obviously ridiculous: “It is not known to Indian southerners that a bare skin is sure to be sunburnt in the northern summer” (p. xii). And Professor Kosambi has not only overlooked the fact that it is almost certainly translated material that he is criticizing as original composition, but he appears not to have even read the whole passage. The sutta sentence (\textbf{\cite{M}I 10}) commented on in the \emph{Visuddhimagga }(\hyperlink{I.86}{I.86}{}-\hyperlink{I.87}{87}{}) contains two words \emph{uṇha }and \emph{ātapa. }If, before condemning the explanation as “ridiculous,” he had read on, he would have found, a line or two below, the words \emph{Ātapo ti suriyātapo }(“‘Burning’ is burning of the sun”—\hyperlink{I.87}{I.87}{}).}And again, the controversial subject of the origin of the Brahman caste (see \textbf{\cite{M-a}II 418}) must have been fully explained from the Buddhist standpoint from the very start. If then that account disagrees with Brahmanical lore—and it would be odd, all things considered, if it did not—there is no justification for concluding on those grounds that the author of the \emph{Visuddhimagga} was not of Brahman origin and that the \emph{Mahāvaṃsa} is wrong. What does indeed seem improbable is that the authorities of the Great Monastery, resolutely committed to oppose unorthodoxy, would have given him a free hand to “correct” their traditions to accord with Brahmanical texts or with other alien sources, even if he had so wished. Again, the fact that there are allusions to extraneous, non-Buddhist literature (e.g. \hyperlink{VII.58}{VII.58}{}; \hyperlink{XVI.4n2}{XVI.4 n.2}{}; \hyperlink{XVI.85}{XVI.85}{}, etc.) hardly affects this issue because they too can have been already in the \marginnote{\textcolor{teal}{\footnotesize\{43|xli\}}}{}material he was editing or supplied to him by the elders with whom he was working. What might repay careful study are perhaps those things, such as certain Mahayana teachings and names, as well as much Brahmanical philosophy, which he ignores though he must have known about them. This ignoring cannot safely be ascribed to ignorance unless we are sure it was not dictated by policy; and we are not sure at all. His silences (in contrast to the author of the \emph{Paramatthamañjūsā}) are sometimes notable in this respect.

                The “popular novel” called \emph{Buddhaghosuppatti}, which was composed in Burma by an elder called Mahāmaṅgala, perhaps as early as the 15th century, is less dependable. But a survey without some account of it would be incomplete. So here is a \emph{précis}:

                Near the Bodhi Tree at Gayā there was a town called Ghosa. Its ruler had a Brahman chaplain called Kesi married to a wife called Kesinī. An elder bhikkhu, who was a friend of Kesi, used to wonder, when the Buddha’s teaching was recited in Sinhalese, and people did not therefore understand it, who would be able to translate it into Magadhan (Pāḷi). He saw that there was the son of a deity living in the Tāvatiṃsa heaven, whose name was Ghosa and who was capable of doing it. This deity was persuaded to be reborn in the human world as the son of the Brahman Kesi. He learnt the Vedas. One day he sat down in a place sacred to Vishnu and ate peas. Brahmans angrily rebuked him, but he uttered a stanza, “The pea itself is Vishnu; who is there called Vishnu? And how shall I know which is Vishnu?” and no one could answer him. Then one day while Kesi was instructing the town’s ruler in the Vedas a certain passage puzzled him, but Ghosa wrote down the explanations on a palm leaf, which was found later by his father—(Chapter I).

                Once when the elder bhikkhu was invited to Kesi’s house for a meal Ghosa’s mat was given to him to sit on. Ghosa was furious and abused the elder. Then he asked him if he knew the Vedas and any other system. The elder gave a recitation from the Vedas. Then Ghosa asked him for his own system, whereupon the elder expounded the first triad of the Abhidhamma schedule, on profitable, unprofitable, and indeterminate thought-arisings. Ghosa asked whose the system was. He was told that it was the Buddha’s and that it could only be learnt after becoming a bhikkhu. He accordingly went forth into homelessness as a bhikkhu, and in one month he learned the three Piṭakas. After receiving the full admission he acquired the four discriminations. The name given to him was Buddhaghosa—(Chapter II).

                One day the question arose in his mind: “Who has more understanding of the Buddha-word, I or my preceptor?” His preceptor, whose cankers were exhausted, read the thought in his mind and rebuked him, telling him to ask his forgiveness. The pupil was then very afraid, and after asking for forgiveness, he was told that in order to make amends he must go to Sri Lanka and translate the Buddha-word (\emph{sic}) from Sinhalese into Magadhan. He agreed, but asked that he might first be allowed to convert his father from the Brahman religion to the Buddha’s teaching. In order to achieve this he had a brick apartment fitted with locks and furnished with food and water. He set a contrivance so that when his father went inside he was trapped. He then preached to his father on the virtues of the Buddha, and on the pains of hell resulting from wrong belief. After three days his father was converted, and he took the Three Refuges. The son then opened the door and made \marginnote{\textcolor{teal}{\footnotesize\{44|xlii\}}}{}opened the door and made amends to his father with flowers and such things for the offence done to him. Kesi became a stream-enterer—(Chapter III).

                This done, he set sail in a ship for Sri Lanka. The Mahāthera Buddhadatta\footnote{\vismAssertFootnoteCounter{14}The allusion is to the author of various Pali works including the \emph{Abhidhammāvatāra; }see n. 4.} had set sail that day from Sri Lanka for India. The two ships met by the intervention of Sakka Ruler of Gods. When the two elders saw each other, the Elder Buddhaghosa told the other: “The Buddha’s Dispensation has been put into Sinhalese; I shall go and translate it and put it into Magadhan.” The other said, “I was sent to go and translate the Buddha-word and write it in Magadhan. I have only done the \emph{Jinālaṅkāra}, the \emph{Dantavaṃsa}, the \emph{Dhātuvaṃsa} and the \emph{Bodhivaṃsa}, not the commentaries and the sub-commentaries (\emph{ṭīkā}). If you, sir, are translating the Dispensation from Sinhalese into Magadhan, do the commentaries to the Three Piṭakas.” Then praising the Elder Buddhaghosa, he gave him the gall-nut, the iron stylus, and the stone given him by Sakka Ruler of Gods, adding, “If you have eye trouble or backache, rub the gall-nut on the stone and wet the place that hurts; then your ailment will vanish.” Then he recited a stanza from his \emph{Jinālaṅkāra. }The other said, “Venerable sir, your book is written in very ornate style. Future clansmen will not be able to follow its meaning. It is hard for simple people to understand it.”—“Friend Buddhaghosa, I went to Sri Lanka before you to work on the Blessed One’s Dispensation. But I have little time before me and shall not live long. So I cannot do it. Do it therefore yourself, and do it well.” Then the two ships separated. Soon after they had completed their voyages the Elder Buddhadatta died and was reborn in the Tusita heaven—(Chapter IV).

                The Elder Buddhaghosa stayed near the port of Dvijaṭhāna in Sri Lanka. While there he saw one woman water-carrier accidentally break another’s jar, which led to a violent quarrel between them with foul abuse. Knowing that he might be called as a witness, he wrote down what they said in a book. When the case came before the king, the elder was cited as a witness. He sent his notebook, which decided the case. The king then asked to see him—(Chapter V).

                After this the elder went to pay homage to the Saṅgharāja,\footnote{\vismAssertFootnoteCounter{15}\emph{Saṅgharāja }(“Ruler of the Community”—a title existing in Thailand today): possibly a mistake for Saṅghapāla here (see Vis. epil.).}the senior elder of Sri Lanka. One day while the senior elder was teaching bhikkhus he came upon a difficult point of Abhidhamma that he could not explain. The Elder Buddhaghosa knew its meaning and wrote it on a board after the senior elder had left. Next day it was discovered and then the senior elder suggested that he should teach the Order of Bhikkhus. The reply was: “I have come to translate the Buddha’s Dispensation into Magadhan.” The senior elder told him, “If so, then construe the Three Piṭakas upon the text beginning, ‘When a wise man, established well in virtue…’” He began the work that day, the stars being favourable, and wrote very quickly. When finished, he put it aside and went to sleep. Meanwhile Sakka, Ruler of Gods, abstracted the book. The elder awoke, and missing it, he wrote another copy very fast by lamplight then he put it aside and slept. Sakka abstracted that \marginnote{\textcolor{teal}{\footnotesize\{45|xliii\}}}{}too. The elder awoke, and not seeing his book, he wrote a third copy very fast by lamplight and wrapped it in his robe. Then he slept again. While he was asleep Sakka put the other two books beside him, and when he awoke he found all three copies. He took them to the senior elder and told him what had happened. When they were read over there was no difference even in a single letter. Thereupon the senior elder gave permission for the translating of the Buddha’s Dispensation. From then on the elder was known to the people of Sri Lanka by the name of Buddhaghosa—(Chapter VI).

                He was given apartments in the Brazen Palace, of whose seven floors he occupied the lowest. He observed the ascetic practices and was expert in all the scriptures. It was during his stay there that he translated the Buddha’s Dispensation. When on his alms round he saw fallen palm leaves he would pick them up; this was a duty undertaken by him. One day a man who had climbed a palm tree saw him. He left some palm leaves on the ground, watched him pick them up, and then followed him. Afterwards he brought him a gift of food. The elder concluded his writing of the Dispensation in three months. When the rainy season was over and he had completed the Pavāraṇā ceremony, he consigned the books to the senior elder, the Saṅgharāja. Then the Elder Buddhaghosa had the books written by Elder Mahinda piled up and burnt near the Great Shrine; the pile was as high as seven elephants. Now that this work was done, and wanting to see his parents, he took his leave before going back to India. Before he left, however, his knowledge of Sanskrit was queried by bhikkhus; but he silenced this by delivering a sermon in the language by the Great Shrine. Then he departed—(Chapter VII).

                On his return he went to his preceptor and cleared himself of his penance. His parents too forgave him his offences; and when they died they were reborn in the Tusita heaven. He himself, knowing that he would not live much longer, paid homage to his preceptor and went to the Great Enlightenment Tree. Foreseeing his approaching death, he considered thus: “There are three kinds of death: death as cutting off, momentary death, and conventional death. Death as cutting off belongs to those whose cankers are exhausted (and are Arahants). Momentary death is that of each consciousness of the cognitive series beginning with life-continuum consciousness, which arise each immediately on the cessation of the one preceding. Conventional death is that of all (so-called) living beings.\footnote{\vismAssertFootnoteCounter{16}A learned allusion to \hyperlink{VIII.1}{VIII.1}{}.}Mine will be conventional death.” After his death he was reborn in the Tusita heaven in a golden mansion seven leagues broad surrounded with divine nymphs. When the Bodhisatta Metteyya comes to this human world, he will be his disciple. After his cremation his relics were deposited near the Enlightenment Tree and shrines erected over them—(Chapter VIII).

                It has already been remarked that the general opinion of European scholars is that where this imaginative tale differs from, or adds to, the \emph{Mahāvaṃsa’s }account it is in legend rather than history.

                Finally there is the question of the Talaing Chronicles of Burma, which mention an elder named Buddhaghosa, of brahman stock, who went from Thatōn \marginnote{\textcolor{teal}{\footnotesize\{46|xliv\}}}{}(the ancient Buddhist stronghold in the Rāmaññadesa of Burma) to Sri Lanka (perhaps via India) to translate the Buddha-word into Talaing and bring it back. It is hard to evaluate this tradition on the evidence available; but according to the opinion of the more reliable Western scholars another elder of the same name is involved here.\footnote{\vismAssertFootnoteCounter{17}Hastings’ \emph{Encyclopaedia of Religion, }article “Buddhaghosa” by T. W. Rhys Davids. Note also that another elder of the same name invited the writing of the \emph{Sammohavinodanī. }The problem is discussed at some length by Prof. Niharranjan Ray, \emph{Theravada Buddhism in Burma, }pp. 24ff.}

                What can be said of the \emph{Visuddhimagga’s }author without venturing into unfounded speculation is now exhausted, at least in so far as the restricted scope of this introduction permits. The facts are tantalizingly few. Indeed this, like many scenes in Indian history, has something of the enigmatic transparencies and uncommunicative shadows of a moonlit landscape—at the same time inescapable and ungraspable.

                Some answer has, however, been furnished to the two questions: why did he come to Sri Lanka? And why did his work become famous beyond its shores? Trends such as have been outlined, working not quite parallel on the Theravāda of India and Sri Lanka, had evolved a situation favouring a rehabilitation of Pali, and consequently the question was already one of interest not only to Sri Lanka, where the old material was preserved. Again the author possessed outstandingly just those personal qualities most fitted to the need—accuracy, an indefatigable mental orderliness, and insight able to crystallize the vast, unwieldy, accumulated exegesis of the Tipiṭaka into a coherent workable whole with a dignified vigorous style, respect for authenticity and dislike of speculation, and (in the circumstances not at all paradoxically) preference for self-effacement. The impetus given by him to Pali scholarship left an indelible mark on the centuries that followed, enabling it to survive from then on the Sanskrit siege as well as the continuing schism and the political difficulties and disasters that harassed Sri Lanka before the “Second Renascence.” A long epoch of culture stems from him. His successors in the Great Monastery tradition continued to write in various centres in South India till the 12th century or so, while his own works spread to Burma and beyond. Today in Sri Lanka and South East Asia his authority is as weighty as it ever was and his name is venerated as before.
            \section{The \emph{Vimuttimagga}}

                Besides the books in Sinhala Bhadantācariya Buddhaghosa names as available to him (which have all disappeared) there was also a manual (existing now only in a Chinese translation of the 6th century CE), presumed to have been written in Pali. Bhadantācariya Buddhaghosa himself makes no mention of it; but his commentator, Bhadantācariya Dhammapāla (writing perhaps within two centuries of him), mentions it by name (see \hyperlink{III.n19}{Ch. III, n.19}{}). The \emph{Visuddhimagga }refutes a certain method of classifying temperaments as unsound. The Elder Dhammapāla ascribes the theory refuted to the \emph{Vimuttimagga. }The theory refuted is actually found in the Chinese version. Then other points rejected by the \emph{Visuddhimagga }are found in the \marginnote{\textcolor{teal}{\footnotesize\{47|xlv\}}}{}\emph{Vimuttimagga. }Some of these are attributed by the Elder Dhammapāla to the Abhayagiri Monastery. However, the \emph{Vimuttimagga }itself contains nothing at all of the Mahāyāna, its unorthodoxies being well within the “Hīnayāna” field.

                The book is much shorter than the \emph{Visuddhimagga. }Though set out in the same three general divisions of virtue, concentration, and understanding, it does not superimpose the pattern of the seven purifications. Proportionately much less space is devoted to understanding, and there are no stories. Though the appearance in both books of numbers of nearly identical passages suggests that they both drew a good deal from the same sources, the general style differs widely. The four measureless states and the four immaterial states are handled differently in the two books. Besides the “material octads,” “enneads” and “decads,” it mentions “endecads,” etc., too. Its description of the thirteen ascetic practices is quite different. Also Abhidhamma, which is the keystone of Bhadantācariya Buddhaghosa’s exegesis, is not used at all in the \emph{Vimuttimagga }(aggregates, truths, etc., do not \emph{in themselves }constitute Abhidhamma in the sense of that Piṭaka). There is for instance even in its description of the consciousness aggregate, no reference to the Dhammasaṅgaṇī’s classification of 89 types, and nothing from the Paṭṭhāna; and though the cognitive series is stated once in its full form (in Ch. 11) no use is made of it to explain conscious workings. This \emph{Vimuttimagga }is in fact a book of practical instructions, not of exegesis.

                Its authorship is ascribed to an Elder Upatissa. But the mere coincidence of names is insufficient to identify him with the Arahant Upatissa (prior to 3rd cent. CE) mentioned in the Vinaya Parivāra. A plausible theory puts its composition sometime before the \emph{Visuddhimagga, }possibly in India. That is quite compatible with its being a product of the Great Monastery before the \emph{Visuddhimagga }was written, though again evidence is needed to support the hypothesis. That it contains some minor points accepted by the Abhayagiri Monastery does not necessarily imply that it had any special connections with that centre. The source may have been common to both. The disputed points are not schismatical. Bhadantācariya Buddhaghosa himself never mentions it.
            \section{Trends in the Development of Theravāda doctrine}

                The doctrines (Dhamma) of the Theravāda Pali tradition can be conveniently traced in three main layers. (1) The first of these contains the main books of the Pali Sutta Piṭakas. (2) The second is the Abhidhamma Piṭaka, notably the closely related books, the Dhammasaṅgaṇī, Vibhaṅga, Paṭṭhāna. (3) The third is the system which the author of the \emph{Visuddhimagga }completed, or found completed, and which he set himself to edit and translate back into Pali (some further minor developments took place subsequently, particularly with the 12th century (?) \emph{Abhidhammatthasaṅgaha, }but they are outside the present scope). The point at issue here is not the much-debated historical question of how far the Abhidhamma books (leaving aside the Kathāvatthu) were contemporary with the Vinaya and Suttas, but rather what discernible direction they show in evolution of thought.

                (1) The Suttas being taken as the original exposition of the Buddha’s teaching, (2) the Abhidhamma Piṭaka itself appears as a highly technical and specialized systematization, or complementary set of modifications built \marginnote{\textcolor{teal}{\footnotesize\{48|xlvi\}}}{}upon that. Its immediate purpose is, one may say, to describe and pin-point mental constituents and characteristics and relate them to their material basis and to each other (with the secondary object, perhaps, of providing an efficient defence in disputes with heretics and exponents of outsiders’ doctrines). Its ultimate purpose is to furnish additional techniques for getting rid of unjustified assumptions that favour clinging and so obstruct the attainment of the extinction of clinging. Various instruments have been forged in it for sorting and re-sorting experience expressed as \emph{dhammas }(see \hyperlink{VII.n1}{Ch. VII, n.1}{}). These instruments are new to the Suttas, though partly traceable to them. The principal instruments peculiar to it are three: (a) the strict treatment of experience (or the knowable and knowledge, using the words in their widest possible sense) in terms of momentary cognizable states \emph{(dhamma) }and the definition of these states, which is done in the Dhammasaṅgaṇī and Vibhaṅga; (b) the creation of a ”schedule” \emph{(mātikā) }consisting of a set of triple \emph{(tika) }and double \emph{(duka) }classifications for sorting these states; and (c) the enumeration of twenty-four kinds of conditioning relations \emph{(paccaya), }which is done in the Paṭṭhāna. The states as defined are thus, as it were, momentary “stills”; the structure of relations combines the stills into continuities; the schedule classifications indicate the direction of the continuities.

                The three Abhidhamma books already mentioned are the essential basis for what later came to be called the “Abhidhamma method”: together they form an integral whole. The other four books, which may be said to support them in various technical fields, need not be discussed here. This, then, is a bare outline of what is in fact an enormous maze with many unexplored side-turnings.

                (3) The system found in the Commentaries has moved on (perhaps slightly diverged) from the strict Abhidhamma Piṭaka standpoint. The Suttas offered descriptions of discovery; the Abhidhamma map-making; but emphasis now is not on discovery, or even on mapping, so much as on consolidating, filling in and explaining. The material is worked over for consistency. Among the principal new developments here are these. The “cognitive series” (\emph{citta-vīthi}) in the occurrence of the conscious process is organized (see \hyperlink{IV.n13}{Ch. IV, n.13}{} and Table V) and completed, and its association with three different kinds of kamma is laid down. The term \emph{sabhāva} (“individual essence,” “own-being” or “it-ness,” see \hyperlink{VII.n68}{Ch. VII, n.68}{}) is introduced to explain the key word \emph{dhamma}, thereby submitting that term to ontological criticism, while the \emph{samaya} (“event” or “occasion”) of the Dhammasaṅgaṇī is now termed a \emph{khaṇa} (“moment”), thus shifting the weight and balance a little in the treatment of time. Then there is the specific ascription of the three “instants” (\emph{khaṇa}, too) of arising, presence and dissolution (\emph{uppāda-ṭṭhiti-bhaṅga}) to each “moment” (\emph{khaṇa}), one “material moment” being calculated to last as long as sixteen “mental moments” (\hyperlink{XX.24}{XX.24}{}; \textbf{\cite{Dhs-a}60}).\footnote{\vismAssertFootnoteCounter{18}The legitimateness of the mental moment of “presence” \emph{(ṭhiti) }as deducible from \textbf{\cite{A}I 152} is questioned by Ācariya Ānanda (\textbf{\cite{Vibh-ṭ}}), who wrote early in the Middle Period; he cites the Yamaka (refs.: II 13–14; and I 216-17) against it.}New to the Piṭakas are also the rather unwieldy enumeration of concepts \emph{(paññatti, }see \hyperlink{VIII.n11}{Ch. VIII, n.11}{}), and the \marginnote{\textcolor{teal}{\footnotesize\{49|xlvii\}}}{}handy defining-formula of word-meaning, characteristic, function, manifestation, and proximate cause (locus); also many minor instances such as the substitution of the specific “heart-basis” for the Paṭṭhāna’s “material basis of mind,” the conception of “material octads,” etc., the detailed descriptions of the thirty-two parts of the body instead of the bare enumeration of the names in the Suttas (thirty-one in the four Nikāyas and thirty-two in the Khuddakapāṭha and the Paṭisambhidāmagga), and many more. And the word \emph{paramattha }acquires a new and slightly altered currency. The question of how much this process of development owes to the post-Mauryan evolution of Sanskrit thought on the Indian mainland (either through assimilation or opposition) still remains to be explored, like so many others in this field. The object of this sketch is only to point to a few landmarks.
            \section{The \emph{Paramatthamañjūsā}}

                The notes to this translation contain many quotations from the commentary to the \emph{Visuddhimagga, }called the \emph{Paramatthamañjūsā }or \emph{Mahā-ṭīkā. }It is regarded as an authoritative work. The quotations are included both for the light they shed on difficult passages in the \emph{Visuddhimagga }and for the sake of rendering for the first time some of the essays interspersed in it. The prologue and epilogue give its author as an elder named Dhammapāla, who lived at Badaratittha (identified as near Chennai). This author, himself also an Indian, is usually held to have lived within two centuries or so of Bhadantācariya Buddhaghosa. There is nothing to say that he ever came to Sri Lanka.

                The\emph{ Visuddhimagga }quotes freely from the Paṭisambhidāmagga, the commentary to which was written by an elder named Mahānāma (date in the Middle Period and place of residence uncertain). Mostly but not quite always, the Elder Dhammapāla says the same thing, when commenting on these quoted passages, as the Elder Mahānāma but in more words.\footnote{\vismAssertFootnoteCounter{19}The Elder Dhammapāla, commenting on Vism \hyperlink{XXI.77}{XXI.77}{}, takes the reading \emph{phuṭṭhantaṃ sacchikato }and explains that (cf. \emph{Mūla Ṭīkā}, Pug-ṭ 32), but the Elder Mahānāma, commenting on the Paṭisambhidāmagga from which the passage is quoted, takes the reading \emph{phuṭṭhattā sacchikato }and comments differently (\textbf{\cite{Paṭis-a}396}, Hewavitarne ed.). Again, what is referred to as “said by some \emph{(keci)” }in the Elder Dhammapāla’s comment on the \emph{Visuddhimagga }(see Vism VIII, n.46) is put forward by the Elder Mahānāma with no such reservation (\textbf{\cite{Paṭis-a}351}). It is the usual standard of strict consistency that makes such very minor divergences noticeable. These two commentators, though, rarely reproduce each other \emph{verbatim. }Contrastingly, where the \emph{Paramatthamañjūsā }and the \emph{Mūlaṭīkā }similarly overlap, the sentences are mostly \emph{verbatim, }but the former, with extra material, looks like an expanded version of the latter, or the latter a cut version of the former.} He relies much on syllogisms and logical arguments. Also there are several discussions of some of the systems of the “Six Schools” of Brahmanical philosophy. There are no stories. This academic writer is difficult, formalistic, and often involved, very careful and accurate. Various other works are attributed to him.
            \section{Some Main Threads in the \emph{Visuddhimagga}}

                \marginnote{\textcolor{teal}{\footnotesize\{50|xlviii\}}}{}The \emph{Visuddhimagga }is probably best regarded as a detailed manual for meditation masters, and as a work of reference. As to its rather intricate construction, the List of Contents is given rather fully in order to serve as a guide to the often complicated form of the chapters and to the work as a whole. In addition, the following considerations may be noted.

                Chapters \hyperlink{I}{I}{} and \hyperlink{II}{II}{}, which deal with virtue as the practice of restraint, or withdrawal, need present no difficulties. It can be remarked here, though, that when the Buddhist ascetic goes into seclusion (restrains the sense doors), it would be incorrect to say of him that he “leaves the \emph{world”; }for where a man is, there is his world \emph{(loka),} as appears in the discourse quoted in \hyperlink{VII.36}{VII.36}{} (cf. also \textbf{\cite{S}IV 116} as well as many other suttas on the same subject). So when he retreats from the clamour of society to the woods and rocks, he takes his world with him, as though withdrawing to his laboratory, in order to better analyze it.

                Chapters \hyperlink{III}{III}{} to \hyperlink{XI}{XI}{} describe the process of concentration and give directions for attaining it by means of a choice of forty meditation subjects for developing concentration. The account of each single meditation subject as given here is incomplete unless taken in conjunction with the whole of \hyperlink{pIII}{Part III}{} (Understanding), which applies to all. Concentration is training in intensity and depth of focus and in single-mindedness. While Buddhism makes no exclusive claim to teach jhāna concentration \emph{(samatha = samādhi), }it does claim that the development of insight \emph{(vipassanā) }culminating in penetration of the Four Noble Truths is peculiar to it. The two have to be coupled together in order to attain the Truths\footnote{\vismAssertFootnoteCounter{20}See \textbf{\cite{A}II 56}; \textbf{\cite{Paṭis}II 92f.}}and the end of suffering. Insight is initially training to see experience as it occurs, without misperception, invalid assumptions, or wrong inferences.

                Chapters \hyperlink{XII}{XII}{} and \hyperlink{XIII}{XIII}{} describe the rewards of concentration fully developed without insight.

                Chapters \hyperlink{XIV}{XIV}{} to \hyperlink{XVII}{XVII}{} on understanding are entirely \emph{theoretical. }Experience in general is dissected, and the separated components are described and grouped in several alternative patterns in Chapters \hyperlink{XIV}{XIV}{} to \hyperlink{XVI.1}{XVI.1}{}–\hyperlink{XVI.12}{12}{}. The rest of \hyperlink{XVI}{Chapter XVI}{} expounds the Four Noble Truths, the centre of the Buddha’s teaching. After that, dependent origination, or the structure of conditionality, is dealt with in its aspect of \emph{arising, }or the process of being (\hyperlink{XVII}{Ch. XVII}{}; as \emph{cessation, }or Nibbāna, it is dealt with separately in Chapters \hyperlink{XVI}{XVI}{} and \hyperlink{XIX}{XIX}{}). The formula of dependent origination in its varying modes describes the working economics of the first two truths (suffering as outcome of craving, and craving itself—see also \hyperlink{XVII.n48}{Ch. XVII, n.48}{}). Without an understanding of conditionality the Buddha’s teaching cannot be grasped: “He who sees dependent origination sees the Dhamma” (\textbf{\cite{M}I 191}), though not all details in this work are always necessary. Since the detailed part of this chapter is very elaborate (\hyperlink{XVII.58}{§58}{}–\hyperlink{XVII.272}{272}{}), a first reading confined to \hyperlink{XVII.1}{§1}{}–\hyperlink{XVII.6}{6}{}, \hyperlink{XVII.20}{§20}{}–\hyperlink{XVII.57}{57}{}, and \hyperlink{XVII.273}{§273}{}–\hyperlink{XVII.314}{314}{}, might help to avoid losing the thread. These four chapters are “theoretical” because they contain in detailed form what needs to be learnt, if only in outline, as “book-learning” \marginnote{\textcolor{teal}{\footnotesize\{51|xlix\}}}{}\emph{(sotāvadhāna-ñāṇa). }They furnish techniques for describing the total experience and the experienceable rather as the branches of arithmetic and double-entry bookkeeping are to be learned as techniques for keeping accurate business accounts.

                Chapters \hyperlink{XVIII}{XVIII}{} to \hyperlink{XXI}{XXI}{}, on the contrary, are \emph{practical }and give instructions for applying the book-knowledge learnt from Chapters \hyperlink{XIV}{XIV}{} to \hyperlink{XVII}{XVII}{} by analyzing in its terms the meditator’s individual experience, dealing also with what may be expected to happen in the course of development. \hyperlink{XVIII}{Chapter XVIII}{} as “defining of mentality-materiality” (first application of Chapters \hyperlink{XIV}{XIV}{} to \hyperlink{XVI}{XVI}{}) and Chapter \hyperlink{XIX}{XIX}{} as “discerning conditions” (first application of Chapter XVII) are preparatory to insight proper, which begins in Chapter \hyperlink{XX}{XX}{} with contemplation of rise and fall. After this, progress continues through the “eight knowledges” with successive clarification—clarification of view of the object and consequent alterations of subjective attitude towards it—till a point, called “conformity knowledge,” is reached which, through one of the “three gateways to liberation,” heralds the attainment of the first supramundane path.

                In \hyperlink{XXII}{Chapter XXII}{}, the attainment of the four successive supramundane paths (or successive stages in realization) is described, with the first of which Nibbāna (extinction of the craving which originates suffering) is ‘seen’ for the first time, having till then been only intellectually conceived. At that moment suffering as a noble truth is fully understood, craving, its origin, is abandoned, suffering’s cessation is realized, and the way to its cessation is developed.\footnote{\vismAssertFootnoteCounter{21}In the present work the development of serenity (concentration) is carried to its limit before insight (understanding) is dealt with. This is for clarity. But in the commentary to the Satipaṭṭhāna Sutta (DN 22, MN 10) either the two are developed contemporaneously or insight is allowed to precede jhāna concentration. According to the Suttas, concentration of jhāna strength is necessary for the manifestation of the path (see e.g. \hyperlink{XIV.127}{XIV.127}{}; \hyperlink{XV.n7}{XV, n.7}{}; \textbf{\cite{D}II 313} = \textbf{\cite{M}III 252}; \textbf{\cite{A}II 156}, quoted at \textbf{\cite{Paṭis}II 92f.}).} The three remaining paths develop further and complete that vision.

                Finally, \hyperlink{XXIII}{Chapter XXIII}{}, as the counterpart of Chapters \hyperlink{XII}{XII}{} and \hyperlink{XIII}{XIII}{}, describes the benefits of understanding. The description of Nibbāna is given at \hyperlink{VIII.245}{Chapter VIII, §245ff.}{}, and a discussion of it at \hyperlink{XVI.66}{Chapter XVI, §66ff.}{}
            \section{Concerning the Translation}

                The pitfalls that await anyone translating from another European language into his own native English are familiar enough; there is no need for him to fall into them. But when he ventures upon rendering an Oriental language, he will often have to be his own guide.

                Naturally, a translator from Pali today owes a large debt to his predecessors and to the Pali Text Society’s publications, including in particular the Society’s invaluable \emph{Pali-English Dictionary}. A translator of the \emph{Visuddhimagga, }too, must make due acknowledgement of its pioneer translation\footnote{\vismAssertFootnoteCounter{22}Reprinted by the Pali Text Society as \emph{Path of Purity,} 1922–31.} U Pe Maung Tin. \marginnote{\textcolor{teal}{\footnotesize\{52|l\}}}{}The word \emph{pāḷi }is translatable by “text.” The \emph{pāḷi }language (the “text language,” which the commentators call Magadhan) holds a special position, with no European parallel, being reserved to one field, namely, the Buddha’s teaching. So there are no alien echoes. In the Suttas, the Sanskrit is silent, and it is heavily muted in the later literature. This fact, coupled with the richness and integrity of the subject itself, gives it a singular limpidness and depth in its early form, as in a string quartet or the clear ocean, which attains in the style of the Suttas to an exquisite and unrivalled beauty unreflectable by any rendering. Traces seem to linger even in the intricate formalism preferred by the commentators.

                This translation presents many formidable problems. Mainly either epistemological and psychological, or else linguistic, they relate either to what ideas and things are being discussed, or else to the manipulation of dictionary meanings of words used in discussion.

                The first is perhaps dominant. As mentioned earlier, the \emph{Visuddhimagga }can be properly studied only as part of the whole commentarial edifice, whose cornerstone it is. But while indexes of words and subjects to the PTS edition of the \emph{Visuddhimagga }exist, most of its author’s works have only indexes of Piṭaka words and names commented on but none for the mass of subject matter. So the student has to make his own. Of the commentaries too, only the \emph{Atthasālinī, }the Dhammapada Commentary, and the Jātaka Commentary have so far been translated (and the latter two are rather in a separate class). But that is a minor aspect.

                This book is largely technical and presents all the difficulties peculiar to technical translation: it deals, besides, with mental happenings. Now where many synonyms are used, as they often are in Pali, for public material objects—an elephant, say, or gold or the sun—the “material objects” should be pointable to, if there is doubt about what is referred to. Again even such generally recognized private experiences as those referred to by the words “consciousness” or “pain” seem too obvious to introspection for uncertainty to arise (communication to fail) if they are given variant symbols. Here the English translator can forsake the Pali allotment of synonyms and indulge a liking for “elegant variation,” if he has it, without fear of muddle. But mind is fluid, as it were, and materially negative, and its analysis needs a different and a strict treatment. In the Suttas, and still more in the Abhidhamma, charting by analysis and definition of pin-pointed mental states is carried far into unfamiliar waters. It was already recognized then that this is no more a solid landscape of “things” to be pointed to when variation has resulted in vagueness. As an instance of disregard of this fact: a greater scholar with impeccable historical and philological judgment (perhaps the most eminent of the English translators) has in a single work rendered the \emph{cattāro satipaṭṭhāna }(here represented by “four foundations of mindfulness”) by “four inceptions of deliberation,” “fourfold setting up of mindfulness,” “fourfold setting up of starting,” “four applications of mindfulness,” and other variants. The PED foreword observes: “No one needs now to use the one English word ‘desire’ as a translation of sixteen distinct Pali words, no one of which means precisely desire. Yet this was done in Vol. X of the\emph{ Sacred Books of the East} by Max Müller and Fausböll.” True; but need one go to the other extreme? How without looking up the Pali can one be sure if the same idea is \marginnote{\textcolor{teal}{\footnotesize\{53|li\}}}{}referred to by all these variants and not some other such as those referred to by \emph{cattāro iddhipādā (“}four roads to power” or “bases of success”), \emph{cattāro sammappadhānā }(“four right endeavours”), etc., or one of the many other “fours”? It is customary not to vary, say, the “call for the categorical imperative” in a new context by some such alternative as “uncompromising order” or “plain-speaking bidding” or “call for unconditional surrender,” which the dictionaries would justify, or “faith” which the exegetists might recommend; that is to say, if it is hoped to avoid confusion. The choosing of an adequate rendering is, however, a quite different problem.

                But there is something more to be considered before coming to that. So far only the difficulty of isolating, symbolizing, and describing individual mental states has been touched on. But here the whole mental structure with its temporal-dynamic process is dealt with too. Identified mental as well as material states (none of which can arise independently) must be recognizable with their associations when encountered in new circumstances: for here arises the central question of thought-association and its manipulation. That is tacitly recognized in the Pali. If disregarded in the English rendering the tenuous structure with its inferences and negations—the flexible pattern of thought-associations—can no longer be communicated or followed, because the pattern of speech no longer reflects it, and whatever may be communicated is only fragmentary and perhaps deceptive. Renderings of words have to be distinguished, too, from renderings of words used to explain those words. From this aspect the Oriental system of word-by-word translation, which transliterates the sound of the principal substantive and verb stems and attaches to them local inflections, has much to recommend it, though, of course, it is not readable as “literature.” One is handling instead of pictures of isolated ideas or even groups of ideas a whole coherent chart system. And besides, words, like maps and charts, are conventionally used to represent high dimensions.

                When already identified states or currents are encountered from new angles, the new situation can be verbalized in one of two ways at least: either by using in a new appropriate verbal setting the words already allotted to these states, or by describing the whole situation afresh in different terminology chosen \emph{ad hoc. }While the second may gain in individual brightness, connections with other allied references can hardly fail to be lost. Aerial photographs must be taken from consistent altitudes, if they are to be used for making maps. And words serve the double purpose of recording ideas already formed and of arousing new ones.

                Structural coherence between different parts in the Pali of the present work needs reflecting in the translation—especially in the last ten chapters—if the thread is not soon to be lost. In fact, in the Pali (just as much in the Tipiṭaka as in its Commentaries), when such subjects are being handled, one finds that a tacit rule, “One term and one flexible definition for one idea (or state or event or situation) referred to,” is adhered to pretty thoroughly. The reason has already been made clear. With no such rule, ideas are apt to disintegrate or coalesce or fictitiously multiply (and, of course, any serious attempt at indexing in English is stultified). \marginnote{\textcolor{teal}{\footnotesize\{54|lii\}}}{}One thing needs to be made clear, though; for there is confusion of thought on this whole subject (one so far only partly investigated).\footnote{\vismAssertFootnoteCounter{23}See Prof. I. A. Richards, \emph{Mencius on Mind,} Kegan Paul, 1932.}This “rule of parsimony in variants” has nothing to do with mechanical transliteration, which is a translator’s refuge when he is unsure of himself. The guiding rule, “One recognizable idea, one word, or phrase to symbolize it,” in no sense implies any such rule as, “One Pali word, one English word,” which is neither desirable nor practicable. Nor in translating need the rule apply beyond the scope reviewed.

                So much for the epistemological and psychological problems.

                The linguistic problem is scarcely less formidable though much better recognized. While English is extremely analytic, Pali (another Indo-European language) is one of the groups of tongues regarded as dominated by Sanskrit, strongly agglutinative, forming long compounds and heavily inflected. The vocabulary chosen occasioned much heart-searching but is still very imperfect. If a few of the words encountered seem a bit algebraical at first, contexts and definitions should make them clear. In the translation of an Oriental language, especially a classical one, the translator must recognize that such knowledge which the Oriental reader is taken for granted to possess is lacking in his European counterpart, who tends unawares to fill the gaps from his own foreign store: the result can be like taking two pictures on one film. Not only is the common background evoked by the words shadowy and patchy, but European thought and Indian thought tend to approach the problems of human existence from opposite directions. This affects word formations. And so double meanings (utraquisms, puns, and metaphors) and etymological links often follow quite different tracks, a fact which is particularly intrusive in describing mental events, where the terms employed are mainly “material” ones used metaphorically. Unwanted contexts constantly creep in and wanted ones stay out. Then there are no well-defined techniques for recognizing and handling idioms, literal rendering of which misleads (while, say, one may not wonder whether to render \emph{tour de force }by “enforced tour” or “tower of strength,” one cannot always be so confident in Pali).

                Then again in the \emph{Visuddhimagga }alone the actual words and word-meanings not in the PED come to more than two hundred and forty. The PED, as its preface states, is “essentially preliminary”; for when it was published many books had still not been collated; it leaves out many words even from the Sutta Piṭaka, and the Sub-commentaries are not touched by it. Also—and most important here—in the making of that dictionary the study of Pali literature had for the most part not been tackled much from, shall one say, the philosophical, or better, epistemological, angle,\footnote{\vismAssertFootnoteCounter{24}Exceptions are certain early works of Mrs. C.A.F. Rhys Davids. See also discussions in appendixes to the translations of the Kathāvatthu (\emph{Points of Controversy}, PTS) and the \emph{Abhidhammatthasaṅgaha} (\emph{Compendium of Philosophy}, PTS).}work and interest having been concentrated till then almost exclusively on history and philology. For instance, the epistemologically unimportant word \emph{vimāna }(divine mansion) is given more than twice the space allotted to the term \emph{paṭicca-samuppāda }(dependent origination), a difficult subject of central importance, the article on which is altogether inadequate and misleading (owing partly to misapplication of the “historical method”). Then \emph{gala }(throat) has been found more \marginnote{\textcolor{teal}{\footnotesize\{55|liii\}}}{}glossarialy interesting than \emph{paṭisandhi }(rebirth-linking), the original use of which word at \textbf{\cite{M}III 230} is ignored. Under \emph{nāma, }too, \emph{nāma-rūpa} is confused with \emph{nāma-kāya. }And so one might continue. By this, however, it is not intended at all to depreciate that great dictionary, but only to observe that in using it the Pali student has sometimes to be wary: if it is criticized in particular here (and it can well hold its own against criticism), tribute must also be paid to its own inestimable general value.
            \section{Concluding remarks}

                Current standard English has been aimed at and preference given always to simplicity. This has often necessitated cutting up long involved sentences, omitting connecting particles (such as \emph{pana, pan’ettha, yasmā }when followed by \emph{tasmā, hi, kho, }etc.), which serve simply as grammatical grease in long chains of subordinate periods. Conversely the author is sometimes extraordinarily elliptic (as in \hyperlink{XIV.46}{XIV.46}{} and \hyperlink{XVI.68}{XVI.68f.}{}), and then the device of square brackets has been used to add supplementary matter, without which the sentence would be too enigmatically shorthand. Such additions (kept to the minimum) are in almost every case taken from elsewhere in the work itself or from the \emph{Paramatthamañjūsā. }Round brackets have been reserved for references and for alternative renderings (as, e.g., in \hyperlink{I.140}{I.140}{}) where there is a sense too wide for any appropriate English word to straddle.

                A few words have been left untranslated (see individual notes). The choice is necessarily arbitrary. It includes \emph{kamma, dhamma} (sometimes), \emph{jhāna, Buddha }(sometimes), \emph{bhikkhu, Nibbāna, Pātimokkha, kasiṇa, Piṭaka}, and \emph{arahant}. There seemed no advantage and much disadvantage in using the Sanskrit forms, \emph{bhikṣu, dharma, dhyāna, arhat, }etc., as is sometimes done (even though ”karma” and “nirvana” are in the \emph{Concise Oxford Dictionary}), and no reason against absorbing the Pali words into English as they are by dropping the diacritical marks. Proper names appear in their Pali spelling without italics and with diacritical marks. Wherever Pali words or names appear, the stem form has been used (e.g. \emph{Buddha, kamma) }rather than the inflected nominative \emph{(Buddho, kammaṃ), }unless there were reasons against it.\footnote{\vismAssertFootnoteCounter{25}Pronounce letters as follows: \emph{a} as in countryman, \emph{ā} father, \emph{e} whey, \emph{i} chin, \emph{ī} machine, \emph{u} full, \emph{ū }rule; c church (always), \emph{g} give (always); \emph{h} always sounded separately, e.g. \emph{bh} in cab-horse, \emph{ch} in catch him (not kitchen), \emph{ph} in upholstery (not telephone), \emph{th} in hot-house (not pathos), etc.; \emph{j} joke; \emph{ṃ} and \emph{ṅ} as ng in singer, \emph{ñ} as \emph{ni} in onion; \emph{ḍ, ḷ, ṇ} and \emph{ṭ} are pronounced with tongue-tip on palate; \emph{d, t, n} and with tongue-tip on teeth; double consonants as in Italian, e.g. \emph{dd} as in mad dog (not madder), \emph{gg} as in big gun (not bigger); rest as in English.}

                Accepted renderings have not been departed from nor earlier translators gone against capriciously. It seemed advisable to treat certain emotionally charged words such as “real” (especially with a capital R) with caution. Certain other words have been avoided altogether. For example, \emph{vassa }(“rains”) signifies a three-month period of residence in one place during the rainy season, enjoined upon bhikkhus by the Buddha in order that they should not travel about trampling down crops and so \marginnote{\textcolor{teal}{\footnotesize\{56|liv\}}}{}annoy farmers. To translate it by “lent” as is sometimes done lets in a historical background and religious atmosphere of mourning and fasting quite alien to it (with no etymological support). “Metempsychosis” for \emph{paṭisandhi }is another notable instance.\footnote{\vismAssertFootnoteCounter{26}Of the principal English value words, “real,” “truth,” “beauty,” “good,” “absolute,” “being,” etc.: “real” has been used for \emph{tatha} (\hyperlink{XVI.24}{XVI.24}{}), “truth” allotted to \emph{sacca} (\hyperlink{XVI.25}{XVI.25}{}) and “beauty” to \emph{subha} (\hyperlink{IX.119}{IX.119}{}); “good” has been used sometimes for the prefix \emph{su-} and also for the adj. \emph{kalyāṇa }and the subst. \emph{attha.} “Absolute” has not been employed, though it might perhaps be used for the word \emph{advaya,} which qualifies the word \emph{kasiṇa }(“universality,” “totalization”) at \textbf{\cite{M}II 14}, and then: “One (man) perceives earth as a universality above, below, around, absolute, measureless” could be an alternative for the rendering given in \hyperlink{V.38}{V.38}{}. “Being” (as abstract subst.) has sometimes been used for \emph{bhava,} which is otherwise rendered by “becoming.”}

                The handling of three words, \emph{dhamma, citta, }and \emph{rūpa }(see Glossary and relevant notes) is admittedly something of a makeshift. The only English word that might with some agility be used consistently for \emph{dhamma }seems to be “idea”; but it has been crippled by philosophers and would perhaps mislead. \emph{Citta }might with advantage have been rendered throughout by “cognizance,” in order to preserve its independence, instead of rendering it sometimes by “mind” (shared with \emph{mano) }and sometimes by “consciousness” (shared with \emph{viññāṇa) }as has been done. But in many contexts all three Pali words are synonyms for the same general notion (see \hyperlink{XIV.82}{XIV.82}{}); and technically, the notion of “cognition,” referred to in its bare aspect by \emph{viññāṇa, }is also referred to along with its concomitant affective colouring, thought and memory, etc., by \emph{citta. }So the treatment accorded to \emph{citta }here finds support to that extent. Lastly “mentality-materiality” for \emph{nāma-rūpa} is inadequate and “name-and-form” in some ways preferable. “Name” (see \hyperlink{XVIII.n4}{Ch. XVIII, n.4}{}) still suggests \emph{nāma’s }function of “naming”; and “form” for the \emph{rūpa }of the \emph{rūpakkhandha }(“materiality aggregate”) can preserve the link with the \emph{rūpa }of the \emph{rūpāyatana, }(“visible-object base”) by rendering them respectively with “material form aggregate” and “visible form base”—a point not without philosophical importance. A compromise has been made at Chapter \hyperlink{X.13}{X.13}{}. “Materiality” or “matter” wherever used should not be taken as implying any hypostasis, any “permanent or semi-permanent substance behind appearances” (the objective counterpart of the subjective ego), which would find no support in the Pali.

                The editions of Sri Lanka, Burma and Thailand have been consulted as well as the two Latin-script editions; and Sinhalese translations, besides. The paragraph numbers of the Harvard University Press edition will be found at the start of paragraphs and the page numbers of the Pali Text Society’s edition in square brackets in the text (the latter, though sometimes appearing at the end of paragraphs, mark the beginnings of the PTS pages). Errors of readings and punctuation in the PTS edition not in the Harvard edition have not been referred to in the notes.

                For the quotations from the Tipiṭaka it was found impossible to make use of existing published translations because they lacked the kind of treatment sought. However, other translation work in hand served as the basis for all the Piṭaka quotations.

                Rhymes seemed unsuitable for the verses from the Tipiṭaka and the “Ancients”; but they have been resorted to for the summarizing verses belonging to the \emph{Visuddhimagga }itself. The English language is too weak in fixed stresses to lend \marginnote{\textcolor{teal}{\footnotesize\{57|lv\}}}{}itself to Pali rhythms, though one attempt to reproduce them was made in \hyperlink{IV}{Chapter IV}{}.

                Where a passage from a sutta is commented on, the order of the explanatory comments follows the Pali order of words in the original sentence, which is not always that of the translation of it.

                In Indian books the titles and subtitles are placed only at, the end of the subject matter. In the translations they have been inserted at the beginning, and some subtitles added for the sake of clarity. In this connection the title at the end of \hyperlink{XI}{Chapter XI}{}, “Description of Concentration” is a “heading” applying not only to that chapter but as far back as the beginning of \hyperlink{III}{Chapter III}{}. Similarly, the title at the end of \hyperlink{XIII}{Chapter XIII}{} refers back to the beginning of \hyperlink{XII}{Chapter XII}{}. The heading “Description of the Soil in which Understanding Grows” \emph{(paññā-bhūmi-niddesa) }refers back from the end of \hyperlink{XVII}{Chapter XVII}{} to the beginning of \hyperlink{XIV}{Chapter XIV}{}.

                The book abounds in “shorthand” allusions to the Piṭakas and to other parts of itself. They are often hard to recognize, and failure to do so results in a sentence with a half-meaning. It is hoped that most of them have been hunted down.

                Criticism has been strictly confined to the application of Pali Buddhist standards in an attempt to produce a balanced and uncoloured English counterpart of the original. The use of words has been stricter in the translation itself than the Introduction to it.

                The translator will, of course, have sometimes slipped or failed to follow his own rules; and there are many passages any rendering of which is bound to evoke query from some quarter where there is interest in the subject. As to the rules, however, and the vocabulary chosen, it has not been intended to lay down laws, and when the methods adopted are described above that is done simply to indicate the line taken: \emph{Janapada-niruttiṃ nābhiniveseyya, samaññaṃ nāti-dhāveyyā ti }(see \hyperlink{XVII.24}{XVII.24}{}).
    \mainmatter

    
    \part[Virtue (\emph{Sīla})]{Virtue (\emph{Sīla})\vismHypertarget{pI}}
        \label{pI}


        \chapter[Description of Virtue]{Description of Virtue\vismHypertarget{I}\newline{\textnormal{\emph{Sīla-niddesa}}}}
            \label{I}


            \marginnote{\textcolor{teal}{\footnotesize\{63|5\}}}{}\emph{Namo tassa bhagavato arahato sammāsambuddhassa}
            \section[\vismAlignedParas{§1–15}I. Introductory]{I. Introductory}

                \vismParagraph{I.1}{1}{}
                
                \begin{verse}
                    \textcolor{brown}{\textit{[1]}}“When a wise man, established well in virtue,\\{}
                    Develops consciousness and understanding,\\{}
                    Then as a bhikkhu ardent and sagacious\\{}
                    He succeeds in disentangling this tangle” (\textbf{\cite{S}I 13}).
                \end{verse}


                This was said. But why was it said? While the Blessed One was living at Sāvatthī, it seems, a certain deity came to him in the night, and in order to do away with his doubts, he asked this question:
                \begin{verse}
                    “The inner tangle and the outer tangle—\\{}
                    This generation is entangled in a tangle.\\{}
                    And so I ask of Gotama this question:\\{}
                    Who succeeds in disentangling this tangle?” (\textbf{\cite{S}I 13}).
                \end{verse}


                \vismParagraph{I.2}{2}{}
                Here is the meaning in brief. \emph{Tangle }is a term for the network of craving. For that is a tangle in the sense of lacing together, like the tangle called network of branches in bamboo thickets, etc., because it goes on arising again and again up and down\footnote{\vismAssertFootnoteCounter{1}\vismHypertarget{I.n1}{}“From a visible datum sometimes as far down as a mental datum, or vice versa, following the order of the six kinds of objects of consciousness as given in the teaching” (\textbf{\cite{Vism-mhṭ}5}, see \hyperlink{XV.32}{XV.32}{}).} among the objects [of consciousness] beginning with what is visible. But it is called \emph{the inner tangle and the outer tangle }because it arises [as craving] for one’s own requisites and another’s, for one’s own person and another’s, and for the internal and external bases [for consciousness]. Since it arises in this way, \emph{this generation is entangled in a tangle. }As the bamboos, etc., are entangled by the bamboo tangle, etc., so too this generation, in other words, this order of living beings, is all entangled by the tangle of craving—the meaning is that it is intertwined, interlaced by it. \textcolor{brown}{\textit{[2]}} And because it is entangled like this, \emph{so I ask of Gotama this question, }that is why I ask this. He addressed the Blessed One by his clan name as \emph{Gotama. Who }\marginnote{\textcolor{teal}{\footnotesize\{64|6\}}}{}\emph{succeeds in disentangling this tangle: }who may disentangle this tangle that keeps the three kinds of existence entangled in this way?—What he asks is, who is capable of disentangling it?

                \vismParagraph{I.3}{3}{}
                However, when questioned thus, the Blessed One, whose knowledge of all things is unimpeded, deity of deities, excelling Sakka (Ruler of Gods), excelling Brahmā, fearless in the possession of the four kinds of perfect confidence, wielder of the ten powers, all-seer with unobstructed knowledge, uttered this stanza in reply to explain the meaning:
                \begin{verse}
                    “When a wise man, established well in virtue,\\{}
                    Develops consciousness and understanding,\\{}
                    Then as a bhikkhu ardent and sagacious\\{}
                    He succeeds in disentangling this tangle.”
                \end{verse}


                \vismParagraph{I.4}{4}{}
                
                \begin{verse}
                    My task is now to set out the true sense,\\{}
                    Divided into virtue and the rest,\\{}
                    Of this same verse composed by the Great Sage.\\{}
                    There are here in the Victor’s Dispensation\\{}
                    Seekers gone forth from home to homelessness,\\{}
                    And who although desiring purity\\{}
                    Have no right knowledge of the sure straight way\\{}
                    Comprising virtue and the other two,\\{}
                    Right hard to find, that leads to purity—\\{}
                    Who, though they strive, here gain no purity.\\{}
                    To them I shall expound the comforting \emph{Path}\\{}
                    \emph{Of Purification}, pure in expositions,\\{}
                    Relying on the teaching of the dwellers\\{}
                    In the Great Monastery;\footnote{\vismAssertFootnoteCounter{2}\vismHypertarget{I.n2}{}The Great Monastery (Mahāvihāra) at Anurādhapura in Sri Lanka.} let all those\\{}
                    Good men who do desire purity\\{}
                    Listen intently to my exposition.
                \end{verse}


                \vismParagraph{I.5}{5}{}
                Herein, \emph{purification} should be understood as Nibbāna, which being devoid of all stains, is utterly pure. \emph{The path of purification} is the path to that purification; it is the means of approach that is called the path. The meaning is, I shall expound that path of purification.

                \vismParagraph{I.6}{6}{}
                In some instances this path of purification is taught by insight alone,\footnote{\vismAssertFootnoteCounter{3}\vismHypertarget{I.n3}{}“The words ‘insight alone’ are meant to exclude not virtue, etc., but serenity (i.e. \emph{jhāna}), which is the opposite number in the pair, serenity and insight. This is for emphasis. But the word ‘alone’ actually excludes only that concentration with distinction [of jhāna]; for concentration is classed as both access and absorption (see \hyperlink{IV.32}{IV.32}{}). Taking this stanza as the teaching for one whose vehicle is insight does not imply that there is no concentration; for no insight comes about without momentary concentration. And again, insight should be understood as the three contemplations of impermanence, pain, and not-self; not contemplation of impermanence alone” (\textbf{\cite{Vism-mhṭ}9–10}).} according as it is said:
                \begin{verse}
                    \marginnote{\textcolor{teal}{\footnotesize\{65|7\}}}{}“Formations are all impermanent:\\{}
                    When he sees thus with understanding\\{}
                    And turns away from what is ill,\\{}
                    That is the path to purity” (\textbf{\cite{Dhp}277}). \textcolor{brown}{\textit{[3]}}
                \end{verse}


                And in some instances by jhāna and understanding, according as it is said:
                \begin{verse}
                    “He is near unto Nibbāna\\{}
                    In whom are jhāna and understanding” (\textbf{\cite{Dhp}372}).
                \end{verse}


                And in some instances by deeds (kamma), etc., according as it is said:
                \begin{verse}
                    “By deeds, vision and righteousness,\\{}
                    By virtue, the sublimest life—\\{}
                    By these are mortals purified,\\{}
                    And not by lineage and wealth” (\textbf{\cite{M}III 262}).
                \end{verse}


                And in some instances by virtue, etc., according as it is said:
                \begin{verse}
                    “He who is possessed of constant virtue,\\{}
                    Who has understanding, and is concentrated,\\{}
                    Who is strenuous and diligent as well,\\{}
                    Will cross the flood so difficult to cross” (\textbf{\cite{S}I 53}).
                \end{verse}


                And in some instances by the foundations of mindfulness, etc., according as it is said: “Bhikkhus, this path is the only way for the purification of beings … for the realization of Nibbāna, that is to say, the four foundations of mindfulness” (\textbf{\cite{D}II 290}); and similarly in the case of the right efforts, and so on. But in the answer to this question it is taught by virtue and the other two.

                \vismParagraph{I.7}{7}{}
                Here is a brief commentary [on the stanza]. \emph{Established well in virtue}: standing on virtue. It is only one actually fulfilling virtue who is here said to “stand on virtue.” So the meaning here is this: being established well in virtue by fulfilling virtue. \emph{A man: }a living being. \emph{Wise: }possessing the kind of understanding that is born of kamma by means of a rebirth-linking with triple root-cause. \emph{Develops consciousness and understanding}: develops both concentration and insight. For it is concentration that is described here under the heading of “consciousness,” and insight under that of “understanding.”\footnote{\vismAssertFootnoteCounter{4}\vismHypertarget{I.n4}{}“‘Develops’ applies to both ‘consciousness’ and ‘understanding.’ But are they mundane or supramundane? They are supramundane, because the sublime goal is described; for one developing them is said to disentangle the tangle of craving by cutting it off at the path moment, and that is not mundane. But the mundane are included here too because they immediately precede, since supramundane (see \hyperlink{III.n5}{Ch. III n. 5}{}) concentration and insight are impossible without mundane concentration and insight to precede them; for without the access and absorption concentration in one whose vehicle is serenity, or without the momentary concentration in one whose vehicle is insight, and without the gateways to liberation (see \hyperlink{XXI.66}{XXI.66f.}{}), the supramundane can never in either case be reached” (\textbf{\cite{Vism-mhṭ}13}). “With triple root-cause” means with non-greed, none-hate, and non-delusion.} \emph{Ardent }(\emph{ātāpin}): possessing energy. For it is energy that is called “ardour” (\emph{ātāpa}) in the sense of burning up and consuming (\emph{ātāpana-paritāpana}) defilements. He has that, thus he is ardent. \emph{Sagacious: }it is \marginnote{\textcolor{teal}{\footnotesize\{66|8\}}}{}understanding that is called “sagacity”; possessing that, is the meaning. This word shows protective understanding. For understanding is mentioned three times in the reply to the question. Herein, the first is naïve understanding, the second is understanding consisting in insight, while the third is the protective understanding that guides all affairs. He sees fear (\emph{bhayaṃ ikkhati}) in the round of rebirths, thus he is a \emph{bhikkhu. He succeeds in disentangling this tangle}: \textcolor{brown}{\textit{[4]}} Just as a man standing on the ground and taking up a well-sharpened knife might disentangle a great tangle of bamboos, so too, he—this bhikkhu who possesses the six things, namely, this virtue, and this concentration described under the heading of consciousness, and this threefold understanding, and this ardour—standing on the ground of virtue and taking up with the hand of protective-understanding exerted by the power of energy the knife of insight-understanding well-sharpened on the stone of concentration, might disentangle, cut away and demolish all the tangle of craving that had overgrown his own life’s continuity. But it is at the moment of the path that he is said to be disentangling that tangle; at the moment of fruition he has disentangled the tangle and is worthy of the highest offerings in the world with its deities. That is why the Blessed One said:
                \begin{verse}
                    “When a wise man, established well in virtue,\\{}
                    Develops consciousness and understanding,\\{}
                    Then as a bhikkhu ardent and sagacious\\{}
                    He succeeds in disentangling this tangle.”
                \end{verse}


                \vismParagraph{I.8}{8}{}
                Herein there is nothing for him to do about the [naïve] understanding on account of which he is called \emph{wise}; for that has been established in him simply by the influence of previous kamma. But the words \emph{ardent }and\emph{ sagacious }mean that by persevering with energy of the kind here described and by acting in full awareness with understanding he should, having become well established in virtue, develop the serenity and insight that are described as \emph{concentration }and\emph{ understanding. }This is how the Blessed One shows the path of purification under the headings of virtue, concentration, and understanding there.

                \vismParagraph{I.9}{9}{}
                What has been shown so far is the three trainings, the dispensation that is good in three ways, the necessary condition for the threefold clear-vision, etc., the avoidance of the two extremes and the cultivation of the middle way, the means to surmounting the states of loss, etc., the abandoning of defilements in three aspects, prevention of transgression etc., purification from the three kinds of defilements, and the reason for the states of stream-entry and so on. How?

                \vismParagraph{I.10}{10}{}
                Here the training of higher virtue is shown by \emph{virtue; }the training of higher consciousness, by \emph{concentration; }and the training of higher understanding, by \emph{understanding.}

                \emph{The dispensation’s goodness in the beginning is shown by virtue. }Because of the passage, “And what is the beginning of profitable things? Virtue that is quite purified” (\textbf{\cite{S}V 143}), and because of the passage beginning, “The not doing of any evil” (\textbf{\cite{Dhp}183}), \emph{virtue }is the beginning of the dispensation. And that is good because it brings about the special qualities of non-remorse,\footnote{\vismAssertFootnoteCounter{5}\vismHypertarget{I.n5}{}One who is virtuous has nothing to be remorseful about.} and so on. Its goodness in the \marginnote{\textcolor{teal}{\footnotesize\{67|9\}}}{}middle is shown by \emph{concentration}. \textcolor{brown}{\textit{[5]}} Because of the passage beginning, “Entering upon the profitable” (\textbf{\cite{Dhp}183}), \emph{concentration }is the middle of the dispensation. And that is good because it brings about the special qualities of supernormal power, and so on. Its goodness in the end is shown by \emph{understanding. }Because of the passage, “The purifying of one’s own mind—this is the Buddhas’ dispensation” (\textbf{\cite{Dhp}183}), and because understanding is its culmination, \emph{understanding }is the end of the dispensation. And that is good because it brings about equipoise with respect to the desired and the undesired. For this is said:
                \begin{verse}
                    “Just as a solid massive rock\\{}
                    Remains unshaken by the wind,\\{}
                    So too, in face of blame and praise\\{}
                    The wise remain immovable” (\textbf{\cite{Dhp}81}).
                \end{verse}


                \vismParagraph{I.11}{11}{}
                Likewise the necessary condition for the triple clear-vision is shown by \emph{virtue. }For with the support of perfected virtue one arrives at the three kinds of clear-vision, but nothing besides that. The necessary condition for the six kinds of direct-knowledge is shown by \emph{concentration. }For with the support of perfected concentration one arrives at the six kinds of direct-knowledge, but nothing besides that. The necessary condition for the categories of discrimination is shown by \emph{understanding. }For with the support of perfected understanding one arrives at the four kinds of discrimination, but not for any other reason.\footnote{\vismAssertFootnoteCounter{6}\vismHypertarget{I.n6}{}The three kinds of clear-vision are: recollection of past lives, knowledge of the passing away and reappearance of beings (divine eye), and knowledge of destruction of cankers (\textbf{\cite{M}I 22–23}). The six kinds of direct-knowledge are: knowledge of supernormal power, the divine ear element, penetration of minds, recollection of past lives, knowledge of the passing away and reappearance of beings, and knowledge of destruction of cankers (\textbf{\cite{M}I 34–35}). The four discriminations are those of meaning, law, language, and intelligence (\textbf{\cite{A}II 160}).}

                And the avoidance of the extreme called devotion to indulgence of sense desires is shown by \emph{virtue. }The avoidance of the extreme called devotion to mortification of self is shown by \emph{concentration. }The cultivation of the middle way is shown by \emph{understanding.}

                \vismParagraph{I.12}{12}{}
                \emph{Likewise the means for surmounting the states of loss is shown by virtue}; the means for surmounting the element of sense desires, by \emph{concentration; }and the means for surmounting all becoming, by \emph{understanding.}

                \emph{And the abandoning of defilements by substitution of opposites is shown by virtue}; that by suppression is shown by \emph{concentration; }and that by cutting off is shown by \emph{understanding.}

                \vismParagraph{I.13}{13}{}
                \emph{Likewise prevention of defilements’ transgression is shown by virtue; }prevention of obsession (by defilement) is shown by \emph{concentration; }prevention of inherent tendencies is shown by \emph{understanding.}

                \emph{And purification from the defilement of misconduct is shown by virtue; }purification from the defilement of craving, by \emph{concentration}; and purification from the defilement of (false) views, by \emph{understanding.}

                \vismParagraph{I.14}{14}{}
                \marginnote{\textcolor{teal}{\footnotesize\{68|10\}}}{}\textcolor{brown}{\textit{[6]}} Likewise the reason for the states of stream-entry and once-return is shown by \emph{virtue; }that for the state of non-return, by \emph{concentration}; that for Arahantship by \emph{understanding. }For the stream-enterer is called “perfected in the kinds of virtue”; and likewise the once-returner. But the non-returner is called “perfected in con-centration.” And the Arahant is called “perfected in understanding” (see \textbf{\cite{A}I 233}).

                \vismParagraph{I.15}{15}{}
                So thus far these nine and other like triads of special qualities have been shown, that is, the three trainings, the dispensation that is good in three ways, the necessary condition for the threefold clear-vision, the avoidance of the two extremes and the cultivation of the middle way, the means for surmounting the states of loss, etc., the abandoning of defilements in three aspects, prevention of transgression, etc., purification from the three kinds of defilements, and the reason for the states of stream-entry, and so on.
            \section[\vismAlignedParas{§16–17}II. Virtue]{II. Virtue}

                \vismParagraph{I.16}{16}{}
                However, even when this path of purification is shown in this way under the headings of virtue, concentration and understanding, each comprising various special qualities, it is still only shown extremely briefly. And so since that is insufficient to help all, there is, in order to show it in detail, the following set of questions dealing in the first place with virtue:

                
                    \begin{enumerate}[(i),nosep]
                        \item What is virtue?
                        \item In what sense is it virtue?
                        \item What are its characteristic, function, manifestation, and proximate cause?
                        \item What are the benefits of virtue?
                        \item How many kinds of virtue are there?
                        \item What is the defiling of it?
                        \item What is the cleansing of it?
                    \end{enumerate}

                \vismParagraph{I.17}{17}{}
                Here are the answers:
            \section[\vismAlignedParas{§17–18}(i) What is virtue?]{(i) What is virtue?}

                It is the states beginning with volition present in one who abstains from killing living things, etc., or in one who fulfils the practice of the duties. For this is said in the Paṭisambhidā: “What is virtue? There is virtue as volition, virtue as consciousness-concomitant,\footnote{\vismAssertFootnoteCounter{7}\vismHypertarget{I.n7}{}“Consciousness-concomitants” (\emph{cetasikā}) is a collective term for feeling, perception, and formation, variously subdivided; in other words, aspects of mentality that arise together with consciousness.} virtue as restraint, \textcolor{brown}{\textit{[7]}} virtue as non-transgression” (\textbf{\cite{Paṭis}I 44}).

                Herein, \emph{virtue as volition }is the volition present in one who abstains from killing living things, etc., or in one who fulfils the practice of the duties. \emph{Virtue as consciousness-concomitant }is the abstinence in one who abstains from killing living things, and so on. Furthermore, \emph{virtue as volition }is the seven volitions [that accompany the first seven] of the [ten] courses of action (kamma) in one who abandons the killing of living things, and so on. \emph{Virtue as consciousness-concomitant }is the [three remaining] states consisting of non-covetousness, non-ill will, and right view, stated in the way beginning, “Abandoning covetousness, he dwells with a mind free from covetousness” (\textbf{\cite{D}I 71}).

                \vismParagraph{I.18}{18}{}
                \marginnote{\textcolor{teal}{\footnotesize\{69|11\}}}{}\emph{Virtue as restraint }should be understood here as restraint in five ways: restraint by the rules of the community (\emph{pātimokkha}), restraint by mindfulness, restraint by knowledge, restraint by patience, and restraint by energy. Herein, “restraint by the Pātimokkha” is this: “He is furnished, fully furnished, with this Pātimokkha restraint. (Vibh 246)” “Restraint by mindfulness” is this: “He guards the eye faculty, enters upon restraint of the eye faculty” (\textbf{\cite{D}I 70}). “Restraint by knowledge” is this:
                \begin{verse}
                    “The currents in the world that flow, Ajita,” said the Blessed One,\\{}
                    “Are stemmed by means of mindfulness;\\{}
                    Restraint of currents I proclaim,\\{}
                    By understanding they are dammed” (\textbf{\cite{Sn}1035});
                \end{verse}


                and use of requisites is here combined with this. But what is called “restraint by patience” is that given in the way beginning, “He is one who bears cold and heat” (\textbf{\cite{M}I 10}). And what is called “restraint by energy” is that given in the way beginning, “He does not endure a thought of sense desires when it arises” (\textbf{\cite{M}I 11}); purification of livelihood is here combined with this. So this fivefold restraint, and the abstinence, in clansmen who dread evil, from any chance of transgression met with, should all be understood to be “virtue as restraint.”

                \emph{Virtue as non-transgression }is the non-transgression, by body or speech, of precepts of virtue that have been undertaken.

                This, in the first place, is the answer to the question, “What is virtue?” \textcolor{brown}{\textit{[8]}} Now,
            \section[\vismAlignedParas{§19}(ii) In what sense is it virtue?]{(ii) In what sense is it virtue?}

                \vismParagraph{I.19}{19}{}
                as to the rest— (ii) \textsc{In what sense is it virtue?} It is virtue (\emph{sīla}) in the sense of composing (\emph{sīlana}).\footnote{\vismAssertFootnoteCounter{8}\vismHypertarget{I.n8}{}\emph{Sīlana} and \emph{upadhāraṇa} in this meaning (cf. \hyperlink{I.141}{Ch. I, §141}{} and \emph{sandhāraṇa}, \hyperlink{XIV.61}{XIV.61}{}) are not in PED.} What is this composing? It is either a coordinating (\emph{samādhāna}), meaning non-inconsistency of bodily action, etc., due to virtuousness; or it is an upholding (\emph{upadhāraṇa}),\vismAssertFootnoteCounter{8}\footnotemark[\value{footnote}] meaning a state of basis (\emph{ādhāra}) owing to its serving as foundation for profitable states. For those who understand etymology admit only these two meanings. Others, however, comment on the meaning here in the way beginning, “The meaning of virtue (\emph{sīla}) is the meaning of head (\emph{sira}), the meaning of virtue is the meaning of cool (\emph{sītala}).”
            \section[\vismAlignedParas{§20–22}(iii) What are its characteristic, etc.?]{(iii) What are its characteristic, etc.?}

                \vismParagraph{I.20}{20}{}
                (iii) Now, \textsc{what are its characteristic, function, manifestation, and proximate cause?} Here:
                \begin{verse}
                    The characteristic of it is composing\\{}
                    Even when analyzed in various ways,\\{}
                    As visibility is of visible data\\{}
                    Even when analyzed in various ways.
                \end{verse}


                Just as visibleness is the characteristic of the visible-data base even when analyzed into the various categories of blue, yellow, etc., because even when analyzed into these categories it does not exceed visible-ness, so also this same composing, described above as the coordinating of bodily action, etc., and as the foundation of \marginnote{\textcolor{teal}{\footnotesize\{70|12\}}}{}profitable states, is the characteristic of virtue even when analyzed into the various categories of volition, etc., because even when analyzed into these categories it does not exceed the state of coordination and foundation.

                \vismParagraph{I.21}{21}{}
                While such is its characteristic:
                \begin{verse}
                    Its function has a double sense:\\{}
                    \emph{Action} to stop misconduct, then\\{}
                    \emph{Achievement} as the quality\\{}
                    Of blamelessness in virtuous men.
                \end{verse}


                So what is called virtue should be understood to have the function (nature) of stopping misconduct as its function (nature) in the sense of action, and a blameless function (nature) as its function (nature) in the sense of achievement. For under [these headings of] characteristic, etc., it is action (\emph{kicca}) or it is achievement (\emph{sampatti}) that is called “function” (\emph{rasa—}nature).

                \vismParagraph{I.22}{22}{}
                
                \begin{verse}
                    Now, virtue, so say those who know,\\{}
                    Itself as purity will show;\\{}
                    And for its proximate cause they tell\\{}
                    The pair, conscience and shame, as well. \textcolor{brown}{\textit{[9]}}
                \end{verse}


                This virtue is manifested as the kinds of purity stated thus: “Bodily purity, verbal purity, mental purity” (\textbf{\cite{A}I 271}); it is manifested, comes to be apprehended, as a pure state. But conscience and shame are said by those who know to be its proximate cause; its near reason, is the meaning. For when conscience and shame are in existence, virtue arises and persists; and when they are not, it neither arises nor persists.

                This is how virtue’s characteristic, function, manifestation, and proximate cause, should be understood.
            \section[\vismAlignedParas{§23–24}(iv) What are the benefits of virtue?]{(iv) What are the benefits of virtue?}

                \vismParagraph{I.23}{23}{}
                (iv) \textsc{What are the benefits of virtue?} Its benefits are the acquisition of the several special qualities beginning with non-remorse. For this is said: “Ānanda, profitable habits (virtues) have non-remorse as their aim and non-remorse as their benefit” (\textbf{\cite{A}V 1}). Also it is said further: “Householder, there are these five benefits for the virtuous in the perfecting of virtue. What five? Here, householder, one who is virtuous, possessed of virtue, obtains a large fortune as a consequence of diligence; this is the first benefit for the virtuous in the perfecting of virtue. Again, of one who is virtuous, possessed of virtue, a fair name is spread abroad; this is the second benefit for the virtuous in the perfecting of virtue. Again, whenever one who is virtuous, possessed of virtue, enters an assembly, whether of khattiyas (warrior-nobles) or brahmans or householders or ascetics, he does so without fear or hesitation; this is the third benefit for the virtuous in the perfecting of virtue. Again, one who is virtuous, possessed of virtue, dies unconfused; this is the fourth benefit for the virtuous in the perfecting of virtue. Again, one who is virtuous, possessed of virtue, on the breakup of the body, after death, reappears in a happy destiny, in the heavenly world; this is the fifth benefit for the virtuous in the perfecting of virtue” (\textbf{\cite{D}II 86}). There are also the many benefits of virtue beginning with being dear and loved and ending with destruction of cankers described in the passage beginning, “If a bhikkhu should wish, ‘May I be dear to my fellows in the life of \marginnote{\textcolor{teal}{\footnotesize\{71|13\}}}{}purity and loved by them, held in respect and honoured by them,’ let him perfect the virtues” (\textbf{\cite{M}I 33}). This is how virtue has as its benefits the several special qualities beginning with non-remorse. \textcolor{brown}{\textit{[10]}}

                \vismParagraph{I.24}{24}{}
                Furthermore:
                \begin{verse}
                    Dare anyone a limit place\\{}
                    On benefits that virtue brings,\\{}
                    Without which virtue clansmen find\\{}
                    No footing in the dispensation?
                \end{verse}

                \begin{verse}
                    No Ganges, and no Yamunā\\{}
                    No Sarabhū, Sarassathī,\\{}
                    Or flowing Aciravatī,\\{}
                    Or noble River of Mahī,\\{}
                    Is able to wash out the stain\\{}
                    In things that breathe here in the world;\\{}
                    For only virtue’s water can\\{}
                    Wash out the stain in living things.
                \end{verse}

                \begin{verse}
                    No breezes that come bringing rain,\\{}
                    No balm of yellow sandalwood,\\{}
                    No necklaces beside, or gems\\{}
                    Or soft effulgence of moonbeams,\\{}
                    Can here avail to calm and soothe\\{}
                    Men’s fevers in this world; whereas\\{}
                    This noble, this supremely cool,\\{}
                    Well-guarded virtue quells the flame.
                \end{verse}

                \begin{verse}
                    Where is there to be found the scent\\{}
                    That can with virtue’s scent compare,\\{}
                    And that is borne against the wind\\{}
                    As easily as with it? Where\\{}
                    Can such another stair be found\\{}
                    That climbs, as virtue does, to heaven?\\{}
                    Or yet another door that gives\\{}
                    Onto the City of Nibbāna?
                \end{verse}

                \begin{verse}
                    Shine as they may, there are no kings\\{}
                    Adorned with jewellery and pearls\\{}
                    That shine as does a man restrained\\{}
                    Adorned with virtue’s ornament.\\{}
                    Virtue entirely does away\\{}
                    With dread of self-blame and the like;\\{}
                    Their virtue to the virtuous\\{}
                    Gives gladness always by its fame.
                \end{verse}

                \begin{verse}
                    From this brief sketch it may be known\\{}
                    How virtue brings reward, and how\\{}
                    This root of all good qualities\\{}
                    Robs of its power every fault.
                \end{verse}

            \section[\vismAlignedParas{§25–142}(v) How many kinds of virtue are there?]{(v) How many kinds of virtue are there?}

                \vismParagraph{I.25}{25}{}
                \textbf{[TODO: enumeration?]}\marginnote{\textcolor{teal}{\footnotesize\{72|14\}}}{}(v) Now, here is the answer to the question, \textsc{How many kinds of virtue are there?}

                1. Firstly all this virtue is of one kind by reason of its own characteristic of composing.

                2. It is of two kinds as keeping and avoiding.

                3. Likewise as that of good behaviour and that of the beginning of the life of purity,

                4. As abstinence and non-abstinence,

                5. As dependent and independent,

                6. As temporary and lifelong,

                7. As limited and unlimited,

                8. As mundane and supramundane. \textcolor{brown}{\textit{[11]}}

                9. It is of three kinds as inferior, medium, and superior.

                10. Likewise as giving precedence to self, giving precedence to the world, and giving precedence to the Dhamma,

                11. As adhered to, not adhered to, and tranquillized.

                12. As purified, unpurified, and dubious.

                13. As that of the trainer, that of the non-trainer, and that of the neither-trainer-nor-non-trainer.

                14. It is of four kinds as partaking of diminution, of stagnation, of distinction, of penetration.

                15. Likewise as that of bhikkhus, of bhikkhunīs, of the not-fully-admitted, of the laity,

                16. As natural, customary, necessary, due to previous causes,

                17. As virtue of Pātimokkha restraint, of restraint of sense faculties, of purification of livelihood, and that concerning requisites.

                18. It is of five kinds as virtue consisting in limited purification, etc.; for this is said in the Paṭisambhidā: “Five kinds of virtue: virtue consisting in limited purification, virtue consisting in unlimited purification, virtue consisting in fulfilled purification, virtue consisting in unadhered-to purification, virtue consisting in tranquillized purification” (\textbf{\cite{Paṭis}I 42}).

                19. Likewise as abandoning, refraining, volition, restraint, and non-transgression.
                \subsection[\vismAlignedParas{§26}1. Monad]{1. Monad}

                    \vismParagraph{I.26}{26}{}
                    \emph{1. }Herein, in the section dealing with that of one kind, the meaning should be understood as already stated.
                \subsection[\vismAlignedParas{§26–32}2.–8. Dyads]{2.–8. Dyads}

                    \emph{2. }In the section dealing with that of two kinds: fulfilling a training precept announced by the Blessed One thus: “This should be done” is \emph{keeping}; not doing what is prohibited by him thus: “This should not be done” is \emph{avoiding}. Herein, the word-meaning is this: they keep (\emph{caranti}) within that, they proceed as people who fulfil the virtues, thus it is keeping (\emph{cāritta}); they preserve, they protect, they avoid, thus it is \marginnote{\textcolor{teal}{\footnotesize\{73|15\}}}{}\emph{avoiding}. Herein, \emph{keeping} is accomplished by faith and energy; \emph{avoiding}, by faith and mindfulness. This is how it is of two kinds as keeping and avoiding.

                    \vismParagraph{I.27}{27}{}
                    \emph{3. }In the second dyad good behaviour is the best kind of behaviour. Good behaviour itself is \emph{that of good behaviour; }or what is announced for the sake of good behaviour is \emph{that of good behaviour. }This is a term for virtue other than that which has livelihood as eighth.\footnote{\vismAssertFootnoteCounter{9}\vismHypertarget{I.n9}{}The three kinds of profitable bodily kamma or action (not killing or stealing or indulging in sexual misconduct), the four kinds of profitable verbal kamma or action (refraining from lying, malicious speech, harsh speech, and gossip), and right livelihood as the eighth.} It is the initial stage of the life of purity consisting in the path, thus it is \emph{that of the beginning of the life of purity. }This is a term for the virtue that has livelihood as eighth. It is the initial stage of the path because it has actually to be purified in the prior stage too. Hence it is said: “But his bodily action, his verbal action, and his livelihood have already been purified earlier” (\textbf{\cite{M}III 289}). Or the training precepts called “lesser and minor” (\textbf{\cite{D}II 154}) \textcolor{brown}{\textit{[12]}} are \emph{that of good behaviour; }the rest are \emph{that of the beginning of the life of purity. }Or what is included in the Double Code (the bhikkhus’ and bhikkhunīs’ Pātimokkha) is \emph{that of the beginning of the life of purity; }and that included in the duties set out in the Khandhakas [of Vinaya] is \emph{that of good behaviour. }Through its perfection \emph{that of the beginning of the life of purity }comes to be perfected. Hence it is said also “that this bhikkhu shall fulfil the state consisting in the beginning of the life of purity without having fulfilled the state consisting in good behaviour—that is not possible” (\textbf{\cite{A}III 14–15}). So it is of two kinds as that of good behaviour and that of the beginning of the life of purity.

                    \vismParagraph{I.28}{28}{}
                    \emph{4. }In the third dyad virtue as \emph{abstinence }is simply abstention from killing living things, etc.; the other kinds consisting in volition, etc., are virtue as \emph{non-abstinence. }So it is of two kinds as abstinence and non-abstinence.

                    \vismParagraph{I.29}{29}{}
                    \emph{5. }In the fourth dyad there are two kinds of dependence: dependence through craving and dependence through [false] views. Herein, that produced by one who wishes for a fortunate kind of becoming thus, “Through this virtuous conduct [rite] I shall become a [great] deity or some [minor] deity” (\textbf{\cite{M}I 102}), is \emph{dependent }through craving. That produced through such [false] view about purification as “Purification is through virtuous conduct” (\textbf{\cite{Vibh}374}) is \emph{dependent }through [false] view. But the supramundane, and the mundane that is the prerequisite for the aforesaid supramundane, are \emph{independent. }So it is of two kinds as dependent and independent.

                    \vismParagraph{I.30}{30}{}
                    \emph{6. }In the fifth dyad \emph{temporary }virtue is that undertaken after deciding on a time limit. \emph{Lifelong }virtue is that practiced in the same way but undertaking it for as long as life lasts. So it is of two kinds as temporary and lifelong.

                    \vismParagraph{I.31}{31}{}
                    \emph{7. }In the sixth dyad the \emph{limited }is that seen to be limited by gain, fame, relatives, limbs, or life. The opposite is \emph{unlimited. }And this is said in the Paṭisambhidā: “What is the virtue that has a limit? There is virtue that has gain as its limit, there is virtue that has fame as its limit, there is virtue that has relatives as its limit, there is virtue that has limbs as its limit, there is virtue that has life as its limit. What is virtue that \marginnote{\textcolor{teal}{\footnotesize\{74|16\}}}{}has gain as its limit? Here someone with gain as cause, with gain as condition, with gain as reason, transgresses a training precept as undertaken: that virtue has gain as its limit” (\textbf{\cite{Paṭis}I 43}), \textcolor{brown}{\textit{[13]}} and the rest should be elaborated in the same way. Also in the answer dealing with the \emph{unlimited }it is said: “What is virtue that does not have gain as its limit? Here someone does not, with gain as cause, with gain as condition, with gain as reason, even arouse the thought of transgressing a training precept as undertaken, how then shall he actually transgress it? That virtue does not have gain as its limit” (\textbf{\cite{Paṭis}I 44}), and the rest should be elaborated in the same way. So it is of two kinds as limited and unlimited.

                    \vismParagraph{I.32}{32}{}
                    \emph{8. }In the seventh dyad all virtue subject to cankers is \emph{mundane; }that not subject to cankers is \emph{supramundane. }Herein, the \emph{mundane }brings about improvement in future becoming and is a prerequisite for the escape from becoming, according as it is said: “Discipline is for the purpose of restraint, restraint is for the purpose of non-remorse, non-remorse is for the purpose of gladdening, gladdening is for the purpose of happiness, happiness is for the purpose of tranquillity, tranquillity is for the purpose of bliss, bliss is for the purpose of concentration, concentration is for the purpose of correct knowledge and vision, correct knowledge and vision is for the purpose of dispassion, dispassion is for the purpose of fading away [of greed], fading away is for the purpose of deliverance, deliverance is for the purpose of knowledge and vision of deliverance, knowledge and vision of deliverance is for the purpose of complete extinction [of craving, etc.] through not clinging. Talk has that purpose, counsel has that purpose, support has that purpose, giving ear has that purpose, that is to say, the liberation of the mind through not clinging” (\textbf{\cite{Vin}V 164}). The \emph{supramundane }brings about the escape from becoming and is the plane of reviewing knowledge. So it is of two kinds as mundane and supramundane.
                \subsection[\vismAlignedParas{§33–38}9.–13. Triads]{9.–13. Triads}

                    \vismParagraph{I.33}{33}{}
                    \emph{9. }In the first of the triads the \emph{inferior }is produced by inferior zeal, [purity of] consciousness, energy, or inquiry; the \emph{medium }is produced by medium zeal, etc.; the \emph{superior}, by superior (zeal, and so on). That undertaken out of desire for fame is \emph{inferior; }that undertaken out of desire for the fruits of merit is \emph{medium}; that undertaken for the sake of the noble state thus, “This has to be done” is \emph{superior. }Or again, that defiled by self-praise and disparagement of others, etc., thus, “I am possessed of virtue, but these other bhikkhus are ill-conducted and evil-natured” (\textbf{\cite{M}I 193}), is \emph{inferior}; undefiled mundane virtue is \emph{medium}; supramundane is \emph{superior}. Or again, that motivated by craving, the purpose of which is to enjoy continued existence, is \emph{inferior}; that practiced for the purpose of one’s own deliverance is \emph{medium}; the virtue of the perfections practiced for the deliverance of all beings is \emph{superior. }So it is of three kinds as inferior, medium, and superior.

                    \vismParagraph{I.34}{34}{}
                    \emph{10. }In the second triad that practiced out of self-regard by one who regards self and desires to abandon what is unbecoming to self \textcolor{brown}{\textit{[14]}} is virtue \emph{giving precedence to self. }That practiced out of regard for the world and out of desire to ward off the censure of the world is virtue \emph{giving precedence to the world. }That practiced out of regard for the Dhamma and out of desire to honour the majesty of the Dhamma is virtue \emph{giving precedence to the Dhamma}. So it is of three kinds as giving precedence to self, and so on.

                    \vismParagraph{I.35}{35}{}
                    \emph{11. }In the third triad the virtue that in the dyads was called dependent (no. 5) is \emph{adhered-to }because it is adhered-to through craving and [false] view. That practiced \marginnote{\textcolor{teal}{\footnotesize\{75|17\}}}{}by the magnanimous ordinary man as the prerequisite of the path, and that associated with the path in trainers, are \emph{not-adhered-to. }That associated with trainers’ and non-trainers’ fruition is \emph{tranquillized. }So it is of three kinds as adhered-to, and so on.

                    \vismParagraph{I.36}{36}{}
                    \emph{12. }In the fourth triad that fulfilled by one who has committed no offence or has made amends after committing one is \emph{pure. }So long as he has not made amends after committing an offence it is \emph{impure}. Virtue in one who is dubious about whether a thing constitutes an offence or about what grade of offence has been committed or about whether he has committed an offence is \emph{dubious}. Herein, the meditator should purify impure virtue. If dubious, he should avoid cases about which he is doubtful and should get his doubts cleared up. In this way his mind will be kept at rest. So it is of three kinds as pure, and so on.

                    \vismParagraph{I.37}{37}{}
                    \emph{13. }In the fifth triad the virtue associated with the four paths and with the [first] three fruitions \emph{is that of the trainer. }That associated with the fruition of Arahantship \emph{is that of the non-trainer. }The remaining kinds are \emph{that of the neither-trainer-nor-non-trainer. }So it is of three kinds as that of the trainer, and so on.

                    \vismParagraph{I.38}{38}{}
                    But in the world the nature of such and such beings is called their “habit” (\emph{sīla}) of which they say: “This one is of happy habit (\emph{sukha-sīla}), this one is of unhappy habit, this one is of quarrelsome habit, this one is of dandified habit.” Because of that it is said in the Paṭisambhidā figuratively: “Three kinds of virtue (habit): profitable virtue, unprofitable virtue, indeterminate virtue” (\textbf{\cite{Paṭis}I 44}). So it is also called of three kinds as profitable, and so on. Of these, the unprofitable is not included here since it has nothing whatever to do with the headings beginning with the characteristic, which define virtue in the sense intended in this [chapter]. So the threefoldness should be understood only in the way already stated.
                \subsection[\vismAlignedParas{§39–130}14.–17. Tetrads]{14.–17. Tetrads}

                    \vismParagraph{I.39}{39}{}
                    \emph{14. }In the first of the tetrads:
                    \begin{verse}
                        The unvirtuous he cultivates,\\{}
                        He visits not the virtuous,\\{}
                        And in his ignorance he sees\\{}
                        No fault in a transgression here, \textcolor{brown}{\textit{[15]}}\\{}
                        With wrong thoughts often in his mind\\{}
                        His faculties he will not guard—\\{}
                        Virtue in such a constitution\\{}
                        Comes to \emph{partake of diminution.}
                    \end{verse}

                    \begin{verse}
                        But he whose mind is satisfied.\\{}
                        With virtue that has been achieved,\\{}
                        Who never thinks to stir himself\\{}
                        And take a meditation subject up,\\{}
                        Contented with mere virtuousness,\\{}
                        Nor striving for a higher state—\\{}
                        His virtue bears the appellation\\{}
                        Of that partaking of stagnation.
                    \end{verse}

                    \begin{verse}
                        But who, possessed of virtue, strives\\{}
                        With concentration for his aim—
                    \end{verse}

                    \begin{verse}
                        \marginnote{\textcolor{teal}{\footnotesize\{76|18\}}}{}That bhikkhu’s virtue in its function\\{}
                        Is called partaking of distinction.
                    \end{verse}

                    \begin{verse}
                        Who finds mere virtue not enough\\{}
                        But has dispassion for his goal—\\{}
                        His virtue through such aspiration\\{}
                        Comes to partake of penetration.
                    \end{verse}


                    So it is of four kinds as partaking of diminution, and so on.

                    \vismParagraph{I.40}{40}{}
                    \emph{15. }In the second tetrad there are training precepts announced for bhikkhus to keep irrespective of what is announced for bhikkhunīs. This is the virtue \emph{of bhikkhus}. There are training precepts announced for bhikkhunīs to keep irrespective of what is announced for bhikkhus. This is the virtue of \emph{bhikkhunīs. }The ten precepts of virtue for male and female novices are the virtue \emph{of the not fully admitted}. The five training precepts—ten when possible—as a permanent undertaking and eight as the factors of the Uposatha Day,\footnote{\vismAssertFootnoteCounter{10}\vismHypertarget{I.n10}{}\emph{Uposatha} (der. from \emph{upavasati}, to observe or to prepare) is the name for the day of “fasting” or “vigil” observed on the days of the new moon, waxing half moon, full moon, and waning half moon. On these days it is customary for laymen to undertake the Eight Precepts (\emph{sīla}) or Five Precepts. On the new-moon and full-moon days the Pātimokkha (see note 11) is recited by bhikkhus. The two quarter-moon days are called the “eighth of the half moon.” The Full-moon day is called the “fifteenth” (i.e. fifteen days from the new moon) and is the last day of the lunar month. That of the new moon is called the “fourteenth” when it is the second and fourth new moon of the four-month season (i.e. fourteen days from the full moon), the other two are called the “fifteenth.” This compensates for the irregularities of the lunar period.} for male and female lay followers are the virtue \emph{of the laity}. So it is of four kinds as the virtue of bhikkhus, and so on.

                    \vismParagraph{I.41}{41}{}
                    \emph{16. }In the third tetrad the non-transgression on the part of Uttarakuru human beings is \emph{natural virtue}. Each clan’s or locality’s or sect’s own rules of conduct are \emph{customary virtue}. The virtue of the Bodhisatta’s mother described thus: “It is the necessary rule, Ānanda, that when the Bodhisatta has descended into his mother’s womb, no thought of men that is connected with the cords of sense desire comes to her” (\textbf{\cite{D}II 13}), is \emph{necessary virtue}. But the virtue of such pure beings as Mahā Kassapa, etc., and of the Bodhisatta in his various births is virtue \emph{due to previous causes}. So it is of four kinds as natural virtue, and so on.
                    \subsubsection[\vismAlignedParas{§42–130}Virtue of the fourfold purification]{Virtue of the fourfold purification}

                        \vismParagraph{I.42}{42}{}
                        \emph{17. }In the fourth tetrad:

                        (a) The virtue described by the Blessed One thus: “Here a bhikkhu dwells restrained with the Pātimokkha restraint, possessed of the [proper] conduct and resort, and seeing fear in the slightest fault, he trains himself by undertaking the precepts of training, (\textbf{\cite{Vibh}244})” is \emph{virtue of Pātimokkha restraint}.

                        (b) That described thus: “On seeing a visible object with the eye, \textcolor{brown}{\textit{[16]}} he apprehends neither the signs nor the particulars through which, if he left the eye faculty unguarded, evil and unprofitable states of covetousness and grief might invade him; he enters upon the way of its restraint, he guards the eye faculty, undertakes the restraint of the eye faculty. On hearing a sound with the ear … On smelling an odour with the nose … On tasting a flavour with the tongue … On \marginnote{\textcolor{teal}{\footnotesize\{77|19\}}}{}touching a tangible object with the body … On cognizing a mental object with the mind, he apprehends neither the signs nor the particulars through which, if he left the mind faculty unguarded, evil and unprofitable states of covetousness and grief might invade him; he enters upon the way of its restraint, he guards the mind faculty, undertakes the restraint of the mind faculty (\textbf{\cite{M}I 180}), is \emph{virtue of restraint of the sense faculties}.

                        (c) Abstinence from such wrong livelihood as entails transgression of the six training precepts announced with respect to livelihood and entails the evil states beginning with “Scheming, talking, hinting, belittling, pursuing gain with gain” (\textbf{\cite{M}II 75}) is \emph{virtue of livelihood purification}.

                        (d) Use of the four requisites that is purified by the reflection stated in the way beginning, “Reflecting wisely, he uses the robe only for protection from cold” (\textbf{\cite{M}I 10}) is called \emph{virtue concerning requisites}.

                        \vismParagraph{I.43}{43}{}
                        Here is an explanatory exposition together with a word commentary starting from the beginning.
                        \par\noindent[\textsc{\textbf{(a) Virtue of Pātimokha restraint}}]

                            (a) \emph{Here}: in this dispensation. \emph{A bhikkhu}: a clansman who has gone forth out of faith and is so styled because he sees fear in the round of rebirths (\emph{saṃsāre bhayaṃ ikkhanatā}) or because he wears cloth garments that are torn and pieced together, and so on.

                            \emph{Restrained with the Pātimokkha restraint}: here “Pātimokkha” (Rule of the Community)\footnote{\vismAssertFootnoteCounter{11}\vismHypertarget{I.n11}{}The Suttavibhaṅga, the first book of the Vinaya Piṭaka, contains in its two parts the 227 rules for bhikkhus and the rules for bhikkhunīs, who have received the admission (\emph{upasampadā}), together with accounts of the incidents that led to the announcement of the rules, the modification of the rules and the explanations of them. The bare rules themselves form the Pātimokkha for bhikkhus and that for bhikkhunīs. They are also known as the “two codes” (\emph{dve mātikā}). The Pātimokkha is recited by bhikkhus on the Uposatha days of the full moon and new moon.} is the virtue of the training precepts; for it frees (\emph{mokkheti}) him who protects (\emph{pāti}) it, guards it, it sets him free (\emph{mocayati}) from the pains of the states of loss, etc., that is why it is called \emph{Pātimokkha}. “Restraint” is restraining; this is a term for bodily and verbal non-transgression. The Pātimokkha itself as restraint is “Pātimokkha restraint.” “Restrained with the Pātimokkha restraint” is restrained by means of the restraint consisting in that Pātimokkha; he has it, possesses it, is the meaning. \emph{Dwells}: bears himself in one of the postures. \textcolor{brown}{\textit{[17]}}

                            \vismParagraph{I.44}{44}{}
                            The meaning of\emph{ possessed of [the proper] conduct and resort}, etc., should be understood in the way in which it is given in the text. For this is said: “Possessed of [the proper] conduct and resort: there is [proper] conduct and improper conduct. Herein, what is improper conduct? Bodily transgression, verbal transgression, bodily and verbal transgression—this is called improper conduct. Also all unvirtuousness is improper conduct. Here someone makes a livelihood by gifts of bamboos, or by gifts of leaves, or by gifts of flowers, fruits, bathing powder, and tooth sticks, or by flattery, or by bean-soupery, or by fondling, or by going on errands on foot, or by one or other of the sorts of wrong livelihood condemned by the Buddhas—this is called improper conduct. Herein, what is [proper] conduct? Bodily \marginnote{\textcolor{teal}{\footnotesize\{78|20\}}}{}non-transgression, verbal non-transgression, bodily and verbal non-transgression—this is called [proper] conduct. Also all restraint through virtue is [proper] conduct. Here someone “does not make a livelihood by gifts of bamboos, or by gifts of leaves, or by gifts of flowers, fruits, bathing powder, and tooth sticks, or by flattery, or by bean-soupery, or by fondling, or by going on errands on foot, or by one or other of the sorts of wrong livelihood condemned by the Buddhas—this is called [proper] conduct.”

                            \vismParagraph{I.45}{45}{}
                            “[Proper] resort: there is [proper] resort and improper resort. Herein, what is improper resort? Here someone has prostitutes as resort, or he has widows, old maids, eunuchs, bhikkhunīs, or taverns as resort; or he dwells associated with kings, kings’ ministers, sectarians, sectarians’ disciples, in unbecoming association with laymen; or he cultivates, frequents, honours, such families as are faithless, untrusting, abusive and rude, who wish harm, wish ill, wish woe, wish no surcease of bondage, for bhikkhus and bhikkhunīs, for male and female devotees \textcolor{brown}{\textit{[18]}}—this is called improper resort. Herein, what is [proper] resort? Here someone does not have prostitutes as resort … or taverns as resort; he does not dwell associated with kings … sectarians’ disciples, in unbecoming association with laymen; he cultivates, frequents, honours, such families as are faithful and trusting, who are a solace, where the yellow cloth glows, where the breeze of sages blows, who wish good, wish well, wish joy, wish surcease of bondage, for bhikkhus and bhikkhunīs, for male and female devotees—this is called [proper] resort. Thus he is furnished with, fully furnished with, provided with, fully provided with, supplied with, possessed of, endowed with, this [proper] conduct and this [proper] resort. Hence it is said, ’Possessed of [the proper] conduct and resort’” (\textbf{\cite{Vibh}246–247}).

                            \vismParagraph{I.46}{46}{}
                            Furthermore, [proper] conduct and resort should also be understood here in the following way; for improper conduct is twofold as bodily and verbal. Herein, what is bodily improper conduct? “Here someone acts disrespectfully before the Community, and he stands jostling elder bhikkhus, sits jostling them, stands in front of them, sits in front of them, sits on a high seat, sits with his head covered, talks standing up, talks waving his arms … walks with sandals while elder bhikkhus walk without sandals, walks on a high walk while they walk on a low walk, walks on a walk while they walk on the ground … stands pushing elder bhikkhus, sits pushing them, prevents new bhikkhus from getting a seat … and in the bath house … without asking elder bhikkhus he puts wood on [the stove] … bolts the door … and at the bathing place he enters the water jostling elder bhikkhus, enters it in front of them, bathes jostling them, bathes in front of them, comes out jostling them, comes out in front of them … and entering inside a house he goes jostling elder bhikkhus, goes in front of them, pushing forward he goes in front of them … and where families have inner private screened rooms in which the women of the family … the girls of the family, sit, there he enters abruptly, and he strokes a child’s head” (\textbf{\cite{Nidd}I 228–229}). This is called bodily improper conduct.

                            \vismParagraph{I.47}{47}{}
                            Herein, what is verbal improper conduct? “Here someone acts disrespectfully before the Community. Without asking elder bhikkhus he talks on the Dhamma, answers questions, recites the Pātimokkha, talks standing up, \textcolor{brown}{\textit{[19]}} talks waving his arms … having entered inside a house, he speaks to a woman or a girl thus: ‘You, so-and-so\marginnote{\textcolor{teal}{\footnotesize\{79|21\}}}{} of such-and-such a clan, what is there? Is there rice gruel? Is there cooked rice? Is there any hard food to eat? What shall we drink? What hard food shall we eat? What soft food shall we eat? Or what will you give me?’—he chatters like this” (\textbf{\cite{Nidd}I 230}). This is called verbal improper conduct.

                            \vismParagraph{I.48}{48}{}
                            Proper conduct should be understood in the opposite sense to that. Furthermore, a bhikkhu is respectful, deferential, possessed of conscience and shame, wears his inner robe properly, wears his upper robe properly, his manner inspires confidence whether in moving forwards or backwards, looking ahead or aside, bending or stretching, his eyes are downcast, he has (a good) deportment, he guards the doors of his sense faculties, knows the right measure in eating, is devoted to wakefulness, possesses mindfulness and full awareness, wants little, is contented, is strenuous, is a careful observer of good behaviour, and treats the teachers with great respect. This is called (proper) conduct.

                            This firstly is how (proper) conduct should be understood.

                            \vismParagraph{I.49}{49}{}
                            (Proper) resort is of three kinds: (proper) resort as support, (proper) resort as guarding, and (proper) resort as anchoring. Herein, what is (proper) resort as support? A good friend who exhibits the instances of talk,\footnote{\vismAssertFootnoteCounter{12}\vismHypertarget{I.n12}{}The “ten instances of talk” (\emph{dasa kathāvatthūni}) refer to the kinds of talk given in the Suttas thus: “Such talk as is concerned with effacement, as favours the heart’s release, as leads to complete dispassion, fading, cessation, peace, direct knowledge, enlightenment, Nibbāna, that is to say: talk on wanting little, contentment, seclusion, aloofness from contact, strenuousness, virtue, concentration, understanding, deliverance, knowledge and vision of deliverance” (\textbf{\cite{M}I 145}; III 113).} in whose presence one hears what has not been heard, corrects what has been heard, gets rid of doubt, rectifies one’s view, and gains confidence; or by training under whom one grows in faith, virtue, learning, generosity and understanding—this is called (\emph{proper})\emph{ resort as support.}

                            \vismParagraph{I.50}{50}{}
                            What is (proper) resort as guarding? Here “A bhikkhu, having entered inside a house, having gone into a street, goes with downcast eyes, seeing the length of a plough yoke, restrained, not looking at an elephant, not looking at a horse, a carriage, a pedestrian, a woman, a man, not looking up, not looking down, not staring this way and that” (\textbf{\cite{Nidd}I 474}). This is called (proper) resort as guarding.

                            \vismParagraph{I.51}{51}{}
                            What is (proper) resort as anchoring? It is the four foundations of mindfulness on which the mind is anchored; for this is said by the Blessed One: “Bhikkhus, what is a bhikkhu’s resort, his own native place? It is these four foundations of mindfulness” (\textbf{\cite{S}V 148}). This is called (\emph{proper})\emph{ resort as anchoring}.

                            Being thus furnished with … endowed with, this (proper) conduct and this (proper) resort, he is also on that account called “one possessed of (proper) conduct and resort.” \textcolor{brown}{\textit{[20]}}

                            \vismParagraph{I.52}{52}{}
                            \emph{Seeing fear in the slightest fault }(\hyperlink{I.42}{§42}{}): one who has the habit (\emph{sīla}) of seeing fear in faults of the minutest measure, of such kinds as unintentional contravening of a minor training rule of the Pātimokkha, or the arising of unprofitable thoughts. \emph{He trains himself by undertaking }(\emph{samādāya})\emph{ the precepts of training: }whatever there is among the precepts of training to be trained in, in all that he trains by taking it up \marginnote{\textcolor{teal}{\footnotesize\{80|22\}}}{}rightly (\emph{sammā ādāya}). And here, as far as the words, “one restrained by the Pātimokkha restraint,” virtue of Pātimokkha restraint is shown by discourse in terms of persons.\footnote{\vismAssertFootnoteCounter{13}\vismHypertarget{I.n13}{}See \hyperlink{IV.n27}{Ch. IV n. 27}{}.} But all that beginning with the words, “possessed of [proper] conduct and resort” should be understood as said in order to show the way of practice that perfects that virtue in him who so practices it.
                        \par\noindent[\textsc{\textbf{(b) Virtue of restraint of the sense faculties}}]

                            \vismParagraph{I.53}{53}{}
                            (b) Now, as regards the virtue of restraint of faculties shown next to that in the way beginning, “on seeing a visible object with the eye,” herein \emph{he }is a bhikkhu established in the virtue of Pātimokkha restraint. \emph{On seeing a visible object with the eye}: on seeing a visible object with the eye-consciousness that is capable of seeing visible objects and has borrowed the name “eye” from its instrument. But the Ancients (\emph{porāṇā}) said: “The eye does not see a visible object because it has no mind. The mind does not see because it has no eyes. But when there is the impingement of door and object he sees by means of the consciousness that has eye-sensitivity as its physical basis. Now, (an idiom) such as this is called an ‘accessory locution’ (\emph{sasambhārakathā}), like ‘He shot him with his bow,’ and so on. So the meaning here is this: ‘On seeing a visible object with eye-consciousness.’”\footnote{\vismAssertFootnoteCounter{14}\vismHypertarget{I.n14}{}“‘\emph{On seeing a visible object with the eye}”: if the eye were to see the visible object, then (organs) belonging to other kinds of consciousness would see too; but that is not so. Why? Because the eye has no thought (\emph{acetanattā}). And then, were consciousness itself to see a visible object, it would see it even behind a wall because of being independent of sense resistance (\emph{appaṭighabhāvato}); but that is not so either because there is no seeing in all kinds of consciousness. And herein, it is consciousness dependent on the eye that sees, not just any kind. And that does not arise with respect to what is enclosed by walls, etc., where light is excluded. But where there is no exclusion of light, as in the case of a crystal or a mass of cloud, there it does arise even with respect to what is enclosed by them. So it is as a basis of consciousness that the eye sees.

                                    “‘\emph{When there is the impingement of door and object}’: what is intended is: when a visible datum as object has come into the eye’s focus. ‘\emph{One sees}’: one looks (\emph{oloketi}); for when the consciousness that has eye-sensitivity as its material support is disclosing (\emph{obhāsente}) by means of the special quality of its support a visible datum as object that is assisted by light (\emph{āloka}), then it is said that a person possessed of that sees the visible datum. And here the illuminating is the revealing of the visible datum according to its individual essence, in other words, the apprehending of it experientially (\emph{paccakkhato}).

                                    “Here it is the ‘\emph{sign of woman}’ because it is the cause of perceiving as ‘woman’ all such things as the shape that is grasped under the heading of the visible data (materiality) invariably found in a female continuity, the un-clear-cut-ness (\emph{avisadatā}) of the flesh of the breasts, the beardlessness of the face, the use of cloth to bind the hair, the un-clear-cut stance, walk, and so on. The ‘\emph{sign of man}’ is in the opposite sense.

                                    “‘\emph{The sign of beauty}’ here is the aspect of woman that is the cause for the arising of lust. By the word ‘etc.’ the sign of resentment (\emph{paṭigha}), etc., are included, which should be understood as the undesired aspect that is the cause for the arising of hate. And here admittedly only covetousness and grief are specified in the text but the sign of equanimity needs to be included too; since there is non-restraint in the delusion that arises due to overlooking, or since ‘forgetfulness of unknowing’ is said below (§57). And here the ‘sign of equanimity’ should be understood as an object that is the basis for the kind of equanimity associated with unknowing through overlooking it. So ‘\emph{the sign of beauty}, etc.’ given in brief thus is actually the cause of greed, hate, and delusion. “‘\emph{He stops at what is merely seen}’: according to the Sutta method, ‘The seen shall be merely seen’ (\textbf{\cite{Ud}8}). As soon as the colour basis has been apprehended by the consciousnesses of the cognitive series with eye-consciousness he stops; he does not fancy any aspect of beauty, etc., beyond that…. In one who fancies as beautiful, etc., the limbs of the opposite sex, defilements arisen with respect to them successively become particularized, which is why they are called ‘particulars.’ But these are simply modes of interpreting (\emph{sannivesākāra}) the kinds of materiality derived from the (four) primaries that are interpreted (\emph{sanniviṭṭha}) in such and such wise; for apart from that there is in the ultimate sense no such thing as a hand and so on” (\textbf{\cite{Vism-mhṭ}40–41}). See also \hyperlink{III.n31}{Ch. III, note 31}{}.}

                            \vismParagraph{I.54}{54}{}
                            \emph{Apprehends neither the signs}: he does not apprehend the sign of woman or man, or any sign that is a basis for defilement such as the sign of beauty, etc.; he stops at what is merely seen. \emph{Nor the particulars: }he does not apprehend any aspect classed as hand, foot, smile, laughter, talk, looking ahead, looking aside, etc., which has acquired the name “particular” (\emph{anubyañjana}) because of its particularizing (\emph{anu anu byañjanato}) defilements, because of its making them manifest themselves. \marginnote{\textcolor{teal}{\footnotesize\{81|23\}}}{}He only apprehends what is really there. Like the Elder Mahā Tissa who dwelt at Cetiyapabbata.

                            \vismParagraph{I.55}{55}{}
                            It seems that as the elder was on his way from Cetiyapabbata to Anurādhapura for alms, a certain daughterinlaw of a clan, who had quarrelled with her husband and had set out early from Anurādhapura all dressed up and tricked out like a celestial nymph to go to her relatives’ home, saw him on the road, and being low-minded, \textcolor{brown}{\textit{[21]}} she laughed a loud laugh. [Wondering] “What is that?” the elder looked up and finding in the bones of her teeth the perception of foulness (ugliness), he reached Arahantship.\footnote{\vismAssertFootnoteCounter{15}\vismHypertarget{I.n15}{}“As the elder was going along (occupied) only in keeping his meditation subject in mind, since noise is a thorn to those in the early stage, he looked up with the noise of the laughter, (wondering) ‘What is that?’ ‘Perception of foulness’ is perception of bones; for the elder was then making bones his meditation subject. The elder, it seems as soon as he saw her teeth-bones while she was laughing, got the counterpart sign with access jhāna because he had developed the preliminary-work well. While he stood there he reached the first jhāna. Then he made that the basis for insight, which he augmented until he attained the paths one after the other and reached destruction of cankers” (\textbf{\cite{Vism-mhṭ}41–42}).}Hence it was said:
                            \begin{verse}
                                “He saw the bones that were her teeth,\\{}
                                And kept in mind his first perception;\\{}
                                And standing on that very spot\\{}
                                The elder became an Arahant.”
                            \end{verse}


                            But her husband, who was going after her, saw the elder and asked, “Venerable sir, did you by any chance see a woman?” The elder told him:
                            \begin{verse}
                                “Whether it was a man or woman\\{}
                                That went by I noticed not,\\{}
                                \marginnote{\textcolor{teal}{\footnotesize\{82|24\}}}{}But only that on this high road\\{}
                                There goes a group of bones.”
                            \end{verse}


                            \vismParagraph{I.56}{56}{}
                            As to the words \emph{through which}, etc., the meaning is: by reason of which, because of which non-restraint of the eye faculty, \emph{if he}, if that person, \emph{left the eye faculty unguarded}, remained with the eye door unclosed by the door-panel of mindfulness, these \emph{states of covetousness}, etc., \emph{might invade}, might pursue, might threaten, \emph{him. He enters upon the way of its restraint}: he enters upon the way of closing that eye faculty by the door-panel of mindfulness. It is the same one of whom it is said \emph{he guards the eye faculty, undertakes the restraint of the eye faculty}.

                            \vismParagraph{I.57}{57}{}
                            Herein, there is neither restraint nor non-restraint in the actual eye faculty, since neither mindfulness nor forgetfulness arises in dependence on eye-sensitivity. On the contrary when a visible datum as object comes into the eye’s focus, then, after the life-continuum has arisen twice and ceased, the functional mind-element accomplishing the function of adverting arises and ceases. After that, eye-consciousness with the function of seeing; after that, resultant mind-element with the function of receiving; after that, resultant root-causeless mind-consciousness-element with the function of investigating; after that, functional root-causeless mind-consciousness-element accomplishing the function of determining arises and ceases. Next to that, impulsion impels.\footnote{\vismAssertFootnoteCounter{16}\vismHypertarget{I.n16}{}To expect to find in the \emph{Paramatthamañjūsā} an exposition of the “cognitive series” (\emph{citta-vīthi}), and some explanation of the individual members in addition to what is to be found in the \emph{Visuddhimagga} itself, is to be disappointed. There are only fragmentary treatments. All that is said here is this:

                                    “There is no unvirtuousness, in other words, bodily or verbal misconduct, in the five doors; consequently restraint of unvirtuousness happens through the mind door, and the remaining restraint happens through the six doors. For the arising of forgetfulness and the other three would be in the five doors since they are unprofitable states opposed to mindfulness, etc.; and there is no arising of unvirtuousness consisting in bodily and verbal transgression there because five-door impulsions do not give rise to intimation. And the five kinds of non-restraint beginning with unvirtuousness are stated here as the opposite of the five kinds of restraint beginning with restraint as virtue” (\textbf{\cite{Vism-mhṭ}42}). See also \hyperlink{IV.n13}{Ch. IV, note 13}{}.} Herein, there is neither restraint nor non-restraint on the occasion of the life-continuum, or on any of the occasions beginning with adverting. But there is non-restraint if unvirtuousness or forgetfulness or unknowing or impatience or idleness arises at the moment of impulsion. When this happens, it is called “non-restraint in the eye faculty.” \textcolor{brown}{\textit{[22]}}

                            \vismParagraph{I.58}{58}{}
                            Why is that? Because when this happens, the door is not guarded, nor are the life-continuum and the consciousnesses of the cognitive series. Like what? Just as, when a city’s four gates are not secured, although inside the city house doors, storehouses, rooms, etc., are secured, yet all property inside the city is unguarded and unprotected since robbers coming in by the city gates can do as they please, so too, when unvirtuousness, etc., arise in impulsion in which there is no restraint, then the door too is unguarded, and so also are the life-continuum and the consciousnesses of the cognitive series beginning with adverting. But when virtue, etc., has arisen in it, then the door too is guarded and so also are the life-continuum and the consciousnesses of \marginnote{\textcolor{teal}{\footnotesize\{83|25\}}}{}the cognitive series beginning with adverting. Like what? Just as, when the city gates are secured, although inside the city the houses, etc., are not secured, yet all property inside the city is well guarded, well protected, since when the city gates are shut there is no ingress for robbers, so too, when virtue, etc., have arisen in impulsion, the door too is guarded and so also are the life-continuum and the consciousnesses of the cognitive series beginning with adverting. Thus although it actually arises at the moment of impulsion, it is nevertheless called “restraint in the eye faculty.”

                            \vismParagraph{I.59}{59}{}
                            So also as regards the phrases \emph{on hearing a sound with the ear }and so on. So it is this virtue, which in brief has the characteristic of avoiding apprehension of signs entailing defilement with respect to visible objects, etc., that should be understood as \emph{virtue of restraint of faculties}.
                        \par\noindent[\textsc{\textbf{(c) Virtue of livelihood purification}}]

                            \vismParagraph{I.60}{60}{}
                            (c) Now, as regards the virtue of livelihood purification mentioned above next to the virtue of restraint of the faculties (\hyperlink{I.42}{§42}{}), the words \emph{of the six precepts announced on account of livelihood }mean, of the following six training precepts announced thus: “With livelihood as cause, with livelihood as reason, one of evil wishes, a prey to wishes, lays claim to a higher than human state that is non-existent, not a fact,” the contravention of which is defeat (expulsion from the Order); “with livelihood as cause, with livelihood as reason, he acts as go-between,” the contravention of which is an offence entailing a meeting of the Order; “with livelihood as cause, with livelihood as reason, he says, ‘A bhikkhu who lives in your monastery is an Arahant,’” the contravention of which is a serious offence in one who is aware of it; “with livelihood as cause, with livelihood as reason, a bhikkhu who is not sick eats superior food that he has ordered for his own use,” the contravention of which is an offence requiring expiation: “With livelihood as cause, with livelihood as reason, a bhikkhunī who is not sick eats superior food that she has ordered for her own use,” the contravention of which is an offence requiring confession; “with livelihood as cause, with livelihood as reason, one who is not sick eats curry or boiled rice \textcolor{brown}{\textit{[23]}} that he has ordered for his own use,” the contravention of which is an offence of wrongdoing (\textbf{\cite{Vin}V 146}). Of these six precepts.\footnote{\vismAssertFootnoteCounter{17}\vismHypertarget{I.n17}{}This apparently incomplete sentence is also in the Pāḷi text. It is not clear why. (BPS Ed.)}

                            \vismParagraph{I.61}{61}{}
                            As regards \emph{scheming}, etc. (\hyperlink{I.42}{§42}{}), this is the text: “Herein, what is scheming? It is the grimacing, grimacery, scheming, schemery, schemedness,\footnote{\vismAssertFootnoteCounter{18}\vismHypertarget{I.n18}{}The formula “\emph{kuhana kuhāyanā kuhitattaṃ},” i.e. verbal noun in two forms and abstract noun from pp., all from the same root, is common in Abhidhamma definitions. It is sometimes hard to produce a corresponding effect in English, yet to render such groups with words of different derivation obscures the meaning and confuses the effect.} by what is called rejection of requisites or by indirect talk, or it is the disposing, posing, composing, of the deportment on the part of one bent on gain, honour and renown, of one of evil wishes, a prey to wishes—this is called scheming.

                            \vismParagraph{I.62}{62}{}
                            ”Herein, what is talking? Talking at others, talking, talking round, talking up, continual talking up, persuading, continual persuading, suggesting, continual suggesting, ingratiating chatter, flattery, bean-soupery, fondling, on the part of one bent on gain, honour and renown, of one of evil wishes, a prey to wishes—this is called talking.

                            \vismParagraph{I.63}{63}{}
                            \marginnote{\textcolor{teal}{\footnotesize\{84|26\}}}{}”Herein, what is hinting? A sign to others, giving a sign, indication, giving indication, indirect talk, roundabout talk, on the part of one bent on gain, honour and renown, of one of evil wishes, a prey to wishes—this is called hinting.

                            \vismParagraph{I.64}{64}{}
                            ”Herein, what is belittling? Abusing of others, disparaging, reproaching, snubbing, continual snubbing, ridicule, continual ridicule, denigration, continual denigration, tale-bearing, backbiting, on the part of one bent on gain, honour and renown, of one of evil wishes, a prey to wishes—this is called belittling.

                            \vismParagraph{I.65}{65}{}
                            ”Herein, what is pursuing gain with gain? Seeking, seeking for, seeking out, going in search of, searching for, searching out material goods by means of material goods, such as carrying there goods that have been got from here, or carrying here goods that have been got from there, by one bent on gain, honour and renown, by one of evil wishes, a prey to wishes—this is called pursuing gain with gain.”\footnote{\vismAssertFootnoteCounter{19}\vismHypertarget{I.n19}{}The renderings “scheming” and so on in this context do not in all cases agree with PED. They have been chosen after careful consideration. The rendering “rejection of requisites” takes the preferable reading \emph{paṭisedhana} though the more common reading here is \emph{paṭisevana} (cultivation).} (\textbf{\cite{Vibh}352–353})

                            \vismParagraph{I.66}{66}{}
                            The meaning of this text should be understood as follows: Firstly, as regards description of \emph{scheming: on the part of one bent on gain, honour and renown }is on the part of one who is bent on gain, on honour, and on reputation; on the part of one who longs for them, is the meaning. \textcolor{brown}{\textit{[24]}} \emph{Of one of evil wishes}: of one who wants to show qualities that he has not got. \emph{A prey to wishes}:\footnote{\vismAssertFootnoteCounter{20}\vismHypertarget{I.n20}{}The Pali is: “\emph{Icchāpakatassā ti icchāya apakatassa; upaddutassā ti attho.” Icchāya apakatassa }simply resolves the compound \emph{icchāpakatassa} and is therefore untranslatable into English. Such resolutions are therefore sometimes omitted in this translation.} the meaning is, of one who is attacked by them. And after this the passage beginning \emph{or by what is called rejection of requisites }is given in order to show the three instances of scheming given in the Mahāniddesa as rejection of requisites, indirect talk, and that based on deportment.

                            \vismParagraph{I.67}{67}{}
                            Herein, [a bhikkhu] is invited to accept robes, etc., and, precisely because he wants them, he refuses them out of evil wishes. And then, since he knows that those householders believe in him implicitly when they think, “Oh, how few are our lord’s wishes! He will not accept a thing!” and they put fine robes, etc., before him by various means, he then accepts, making a show that he wants to be compassionate towards them—it is this hypocrisy of his, which becomes the cause of their subsequently bringing them even by cartloads, that should be understood as the instance of scheming called rejection of requisites.

                            \vismParagraph{I.68}{68}{}
                            For this is said in the Mahāniddesa: “What is the instance of scheming called rejection of requisites? Here householders invite bhikkhus [to accept] robes, alms food, resting place, and the requisite of medicine as cure for the sick. One who is of evil wishes, a prey to wishes, wanting robes … alms food … resting place … the requisite of medicine as cure for the sick, refuses robes … alms food … resting place … the requisite of medicine as cure for the sick, because he wants more. He says: ‘What has an ascetic to do with expensive robes? It is proper for an ascetic to gather rags from a charnel ground or from a rubbish heap or from a shop and make them into a patchwork cloak to wear. What has an ascetic to do with expensive \marginnote{\textcolor{teal}{\footnotesize\{85|27\}}}{}alms food? It is proper for an ascetic to get his living by the dropping of lumps [of food into his bowl] while he wanders for gleanings. What has an ascetic to do with an expensive resting place? It is proper for an ascetic to be a tree-root-dweller or an open-air-dweller. What has an ascetic to do with an expensive requisite of medicine as cure for the sick? It is proper for an ascetic to cure himself with putrid urine\footnote{\vismAssertFootnoteCounter{21}\vismHypertarget{I.n21}{}“‘\emph{Putrid urine}’ is the name for all kinds of cow’s urine whether old or not” (\textbf{\cite{Vism-mhṭ}45}). Fermented cow’s urine with gallnuts (myrobalan) is a common Indian medicine today.} and broken gallnuts.’ Accordingly he wears a coarse robe, eats coarse alms food, \textcolor{brown}{\textit{[25]}} uses a coarse resting place, uses a coarse requisite of medicine as cure for the sick. Then householders think, ‘This ascetic has few wishes, is content, is secluded, keeps aloof from company, is strenuous, is a preacher of asceticism,’ and they invite him more and more [to accept] robes, alms food, resting places, and the requisite of medicine as cure for the sick. He says: ‘With three things present a faithful clansman produces much merit: with faith present a faithful clansman produces much merit, with goods to be given present a faithful clansman produces much merit, with those worthy to receive present a faithful clansman produces much merit. You have faith; the goods to be given are here; and I am here to accept. If I do not accept, then you will be deprived of the merit. That is no good to me. Rather will I accept out of compassion for you.” Accordingly he accepts many robes, he accepts much alms food, he accepts many resting places, he accepts many requisites of medicine as cure for the sick. Such grimacing, grimacery, scheming, schemery, schemedness, is known as the instance of scheming called rejection of requisites’ (\textbf{\cite{Nidd}I 224–225}).

                            \vismParagraph{I.69}{69}{}
                            It is hypocrisy on the part of one of evil wishes, who gives it to be understood verbally in some way or other that he has attained a higher than human state, that should be understood as the instance of scheming called indirect talk, according as it is said: “What is the instance of scheming called indirect talk? Here someone of evil wishes, a prey to wishes, eager to be admired, [thinking] ‘Thus people will admire me’ speaks words about the noble state. He says, ‘He who wears such a robe is a very important ascetic.’ He says, ‘He who carries such a bowl, metal cup, water filler, water strainer, key, wears such a waist band, sandals, is a very important ascetic.’ He says, ‘He who has such a preceptor … teacher … who has the same preceptor, who has the same teacher, who has such a friend, associate, intimate, companion; he who lives in such a monastery, lean-to, mansion, villa,\footnote{\vismAssertFootnoteCounter{22}\vismHypertarget{I.n22}{}It is not always certain now what kind of buildings these names refer to.} cave, grotto, hut, pavilion, watch tower, hall, barn, meeting hall, \textcolor{brown}{\textit{[26]}} room, at such a tree root, is a very important ascetic.’ Or alternatively, all-gushing, all-grimacing, all-scheming, all-talkative, with an expression of admiration, he utters such deep, mysterious, cunning, obscure, supramundane talk suggestive of voidness as ‘This ascetic is an obtainer of peaceful abidings and attainments such as these.’ Such grimacing, grimacery, scheming, schemery, schemedness, is known as the instance of scheming called indirect talk” (\textbf{\cite{Nidd}I 226–227}).

                            \vismParagraph{I.70}{70}{}
                            It is hypocrisy on the part of one of evil wishes, which takes the form of deportment influenced by eagerness to be admired, that should be understood as the instance of \marginnote{\textcolor{teal}{\footnotesize\{86|28\}}}{}scheming dependent on deportment, according as it is said: “What is the instance of scheming called deportment? Here someone of evil wishes, a prey to wishes, eager to be admired, [thinking] ‘Thus people will admire me,’ composes his way of walking, composes his way of lying down; he walks studiedly, stands studiedly, sits studiedly, lies down studiedly; he walks as though concentrated, stands, sits, lies down as though concentrated; and he is one who meditates in public. Such disposing, posing, composing, of deportment, grimacing, grimacery, scheming, schemery, schemedness, is known as the instance of scheming called deportment” (\textbf{\cite{Nidd}I 225–226}).

                            \vismParagraph{I.71}{71}{}
                            Herein, the words \emph{by what is called rejection of requisites }(\hyperlink{I.61}{§61}{}) mean: by what is called thus “rejection of requisites”; or they mean: by means of the rejection of requisites that is so called. \emph{By indirect talk }means: by talking near to the subject. \emph{Of deportment }means: of the four modes of deportment (postures). \emph{Disposing }is initial posing, or careful posing. \emph{Posing }is the manner of posing. \emph{Composing }is prearranging; assuming a trust-inspiring attitude, is what is meant. \emph{Grimacing }is making grimaces by showing great intenseness; facial contraction is what is meant. One who has the habit of making grimaces is a grimacer. The grimacer’s state is \emph{grimacery. Scheming }is hypocrisy. The way (\emph{āyanā}) of a schemer (\emph{kuha}) is \emph{schemery }(\emph{kuhāyanā}). The state of what is schemed is \emph{schemedness}.

                            \vismParagraph{I.72}{72}{}
                            In the description of \emph{talking: talking at }is talking thus on seeing people coming to the monastery, “What have you come for, good people? What, to invite bhikkhus? If it is that, then go along and I shall come later with [my bowl],” etc.; or alternatively, \emph{talking at }is talking by advertising oneself thus, “I am Tissa, the king trusts me, such and such king’s ministers trust me.” \textcolor{brown}{\textit{[27]}} \emph{Talking }is the same kind of talking on being asked a question. \emph{Talking round }is roundly talking by one who is afraid of householders’ displeasure because he has given occasion for it. \emph{Talking up }is talking by extolling people thus, “He is a great land-owner, a great ship-owner, a great lord of giving.” \emph{Continual talking up }is talking by extolling [people] in all ways.

                            \vismParagraph{I.73}{73}{}
                            \emph{Persuading }is progressively involving\footnote{\vismAssertFootnoteCounter{23}\vismHypertarget{I.n23}{}\emph{Nahanā—}tying, from \emph{nayhati} (to tie). The noun in not in PED.} [people] thus, “Lay followers, formerly you used to give first-fruit alms at such a time; why do you not do so now?” until they say, “We shall give, venerable sir, we have had no opportunity,” etc.; entangling, is what is meant. Or alternatively, seeing someone with sugarcane in his hand, he asks, “Where are you coming from, lay follower?”—”From the sugarcane field, venerable sir”—”Is the sugarcane sweet there?”—”One can find out by eating, venerable sir”—”It is not allowed, lay follower, for bhikkhus to say ‘Give [me some] sugarcane.’” Such entangling talk from such an entangler is \emph{persuading}. Persuading again and again in all ways is \emph{continual persuading}.

                            \vismParagraph{I.74}{74}{}
                            \emph{Suggesting }is insinuating by specifying thus, “That family alone understands me; if there is anything to be given there, they give it to me only”; pointing to, is what is meant. And here the story of the oil-seller should be told.\footnote{\vismAssertFootnoteCounter{24}\vismHypertarget{I.n24}{}The story of the oil-seller is given in the \emph{Sammohavinodanī} (\textbf{\cite{Vibh-a}483}), which reproduces this part of Vism with some additions: “Two bhikkhus, it seems, went into a village and sat down in the sitting hall. Seeing a girl, they called her. Then one asked the other, ‘Whose girl is this, venerable sir?’—‘She is the daughter of our supporter the oil-seller, friend. When we go to her mother’s house and she gives us ghee, she gives it in the pot. And this girl too gives it in the pot as her mother does.’” Quoted at \textbf{\cite{Vism-mhṭ}46}.} Suggesting in all ways again and again is \emph{continual suggesting}.

                            \vismParagraph{I.75}{75}{}
                            \marginnote{\textcolor{teal}{\footnotesize\{87|29\}}}{}\emph{Ingratiating chatter }is endearing chatter repeated again and again without regard to whether it is in conformity with truth and Dhamma. \emph{Flattery }is speaking humbly, always maintaining an attitude of inferiority. \emph{Bean-soupery }is resemblance to bean soup; for just as when beans are being cooked only a few do not get cooked, the rest get cooked, so too the person in whose speech only a little is true, the rest being false, is called a “bean soup”; his state is \emph{bean-soupery}.

                            \vismParagraph{I.76}{76}{}
                            \emph{Fondling }is the state of the act of fondling. \textcolor{brown}{\textit{[28]}} For when a man fondles children on his lap or on his shoulder like a nurse—he nurses, is the meaning—that fondler’s act is the act of fondling. The state of the act of fondling is \emph{fondling}.

                            \vismParagraph{I.77}{77}{}
                            In the description of \emph{hinting }(\emph{nemittikatā}): a sign (\emph{nimitta}) is any bodily or verbal act that gets others to give requisites. \emph{Giving a sign }is making a sign such as “What have you got to eat?”, etc., on seeing [people] going along with food. \emph{Indication }is talk that alludes to requisites. \emph{Giving indication}: on seeing cowboys, he asks, “Are these milk cows’ calves or buttermilk cows’ calves?” and when it is said, “They are milk cows’ calves, venerable sir,” [he remarks] “They are not milk cows’ calves. If they were milk cows’ calves the bhikkhus would be getting milk,” etc.; and his getting it to the knowledge of the boys’ parents in this way, and so making them give milk, is \emph{giving indication}.

                            \vismParagraph{I.78}{78}{}
                            \emph{Indirect talk }is talk that keeps near [to the subject]. And here there should be told the story of the bhikkhu supported by a family. A bhikkhu, it seems, who was supported by a family went into the house wanting to eat and sat down. The mistress of the house was unwilling to give. On seeing him she said, “There is no rice,” and she went to a neighbour’s house as though to get rice. The bhikkhu went into the storeroom. Looking round, he saw sugarcane in the corner behind the door, sugar in a bowl, a string of salt fish in a basket, rice in a jar, and ghee in a pot. He came out and sat down. When the housewife came back, she said, “I did not get any rice.” The bhikkhu said, “Lay follower, I saw a sign just now that alms will not be easy to get today.”—“What, venerable sir?”—”I saw a snake that was like sugarcane put in the corner behind the door; looking for something to hit it with, I saw a stone like a lump of sugar in a bowl. When the snake had been hit with the clod, it spread out a hood like a string of salt fish in a basket, and its teeth as it tried to bite the clod were like rice grains in a jar. Then the saliva mixed with poison that came out to its mouth in its fury was like ghee put in a pot.” She thought, “There is no hoodwinking the shaveling,” so she gave him the sugarcane \textcolor{brown}{\textit{[29]}} and she cooked the rice and gave it all to him with the ghee, the sugar and the fish.

                            \vismParagraph{I.79}{79}{}
                            Such talk that keeps near [to the subject] should be understood as indirect talk. \emph{Roundabout talk }is talking round and round [the subject] as much as is allowed.

                            \vismParagraph{I.80}{80}{}
                            In the description of \emph{belittling: abusing }is abusing by means of the ten instances of abuse.\footnote{\vismAssertFootnoteCounter{25}\vismHypertarget{I.n25}{}The “ten instances of abuse” (\emph{akkosa-vatthu}) are given in the \emph{Sammohavinodanī} (\textbf{\cite{Vibh-a}340}) as: “You are a thief, you are a fool, you are an idiot, you are a camel (\emph{oṭṭha}), you are an ox, you are a donkey, you belong to the states of loss, you belong to hell, you are a beast, there is not even a happy or an unhappy destiny to be expected for you” (see also \textbf{\cite{Sn-a}364}).} \emph{Disparaging }is contemptuous talk. \emph{Reproaching }is enumeration of faults such as “He is faithless, he is an unbeliever.” \emph{Snubbing }is taking up verbally thus, \marginnote{\textcolor{teal}{\footnotesize\{88|30\}}}{}“Don’t say that here.” Snubbing in all ways, giving grounds and reasons, is \emph{continual snubbing}. Or alternatively, when someone does not give, taking him up thus, “Oh, the prince of givers!” is \emph{snubbing; }and the thorough snubbing thus, “A mighty prince of givers!” is \emph{continual snubbing. Ridicule }is making fun of someone thus, “What sort of a life has this man who eats up his seed [grain]?” \emph{Continual ridicule }is making fun of him more thoroughly thus, “What, you say this man is not a giver who always gives the words ‘There is nothing’ to everyone?”

                            \vismParagraph{I.81}{81}{}
                            \emph{Denigration}\footnote{\vismAssertFootnoteCounter{26}\vismHypertarget{I.n26}{}The following words of this paragraph are not in PED: \emph{Pāpanā} (denigration), \emph{pāpanaṃ} (nt. denigrating), \emph{nippeseti} (scrapes off—from \emph{piṃsati}? cf. \emph{nippesikatā—}“belittling” §§42, 64), \emph{nippuñchati} (wipes off—only \emph{puñchati} in PED), \emph{pesikā} (scraper—not in this sense in PED: from same root as \emph{nippeseti}), \emph{nippiṃsitvā} (grinding, pounding), \emph{abbhaṅga} (unguent = \emph{abbhañjana}, \textbf{\cite{Vism-mhṭ}47}).}\emph{ is denigrating someone by saying that he is not a giver, or by censuring him. All-round denigration is continual denigration. Tale-bearing }is bearing tales from house to house, from village to village, from district to district, [thinking] “So they will give to me out of fear of my bearing tales.” \emph{Backbiting }is speaking censoriously behind another’s back after speaking kindly to his face; for this is like biting the flesh of another’s back, when he is not looking, on the part of one who is unable to look him in the face; therefore it is called \emph{backbiting}. This is called \emph{belittling }(\emph{nippesikatā}) because it scrapes off (\emph{nippeseti}), wipes off, the virtuous qualities of others as a bamboo scraper (\emph{veḷupesikā}) does unguent, or because it is a pursuit of gain by grinding (\emph{nippiṃsitvā}) and pulverizing others’ virtuous qualities, like the pursuit of perfume by grinding perfumed substances; that is why it is called \emph{belittling}.

                            \vismParagraph{I.82}{82}{}
                            In the description of \emph{pursuing gain with gain: pursuing }is hunting after. \emph{Got from here }is got from this house. \emph{There }is into that house. \emph{Seeking }is wanting. \emph{Seeking for }is hunting after. \emph{Seeking out }is hunting after again and again. \textcolor{brown}{\textit{[30]}} The story of the bhikkhu who went round giving away the alms he had got at first to children of families here and there and in the end got milk and gruel should be told here. \emph{Searching}, etc., are synonyms for “seeking,” etc., and so the construction here should be understood thus: \emph{going in search of }is seeking; \emph{searching for }is seeking for; \emph{searching out }is seeking out.

                            This is the meaning of \emph{scheming, }and so on.

                            \vismParagraph{I.83}{83}{}
                            Now, [as regards the words] \emph{The evil states beginning with }(\hyperlink{I.42}{§42}{}): here the words \emph{beginning with }should be understood to include the many evil states given in the Brahmajāla Sutta in the way beginning, “Or just as some worthy ascetics, while eating the food given by the faithful, make a living by wrong livelihood, by such low arts as these, that is to say, by palmistry, by fortune-telling, by divining omens, by interpreting dreams, marks on the body, holes gnawed by mice; by fire sacrifice, by spoon oblation …” (\textbf{\cite{D}I 9}).

                            \vismParagraph{I.84}{84}{}
                            \marginnote{\textcolor{teal}{\footnotesize\{89|31\}}}{}So this wrong livelihood entails the transgression of these six training precepts announced on account of livelihood, and it entails the evil states beginning with “Scheming, talking, hinting, belittling, pursuing gain with gain.” And so it is the abstinence from all sorts of wrong livelihood that is \emph{virtue of livelihood purification}, the word-meaning of which is this: on account of it they live, thus it is livelihood. What is that? It is the effort consisting in the search for requisites. “Purification” is purifiedness. “Livelihood purification” is purification of livelihood.
                        \par\noindent[\textsc{\textbf{(d) Virtue concerning requisites}}]

                            \vismParagraph{I.85}{85}{}
                            (d) As regards the next kind called \emph{virtue concerning requisites}, [here is the text: “Reflecting wisely, he uses the robe only for protection from cold, for protection from heat, for protection from contact with gadflies, flies, wind, burning and creeping things, and only for the purpose of concealing the private parts. Reflecting wisely, he uses alms food neither for amusement nor for intoxication nor for smartening nor for embellishment, but only for the endurance and continuance of this body, for the ending of discomfort, and for assisting the life of purity: ‘Thus I shall put a stop to old feelings and shall not arouse new feelings, and I shall be healthy and blameless and live in comfort.’ Reflecting wisely, he uses the resting place only for the purpose of protection from cold, for protection from heat, for protection from contact with gadflies, flies, wind, burning and creeping things, and only for the purpose of warding off the perils of climate and enjoying retreat. Reflecting wisely, he uses the requisite of medicine as cure for the sick only for protection from arisen hurtful feelings and for complete immunity from affliction” (\textbf{\cite{M}I 10}). Herein, \emph{reflecting wisely} is reflecting as the means and as the way;\footnote{\vismAssertFootnoteCounter{27}\vismHypertarget{I.n27}{}For attention (\emph{manasi-kāra}) as the means (\emph{upāya}) and the way (\emph{patha}) see \textbf{\cite{M-a}I 64}.} by knowing, by reviewing, is the meaning. And here it is the reviewing stated in the way beginning, “For protection from cold” that should be understood as “reflecting wisely.”

                            \vismParagraph{I.86}{86}{}
                            Herein, \emph{the robe }is any one of those beginning with the inner cloth. \emph{He uses}: he employs; dresses in [as inner cloth], or puts on [as upper garment]. \emph{Only }\textcolor{brown}{\textit{[31]}} is a phrase signifying invariability in the definition of a limit\footnote{\vismAssertFootnoteCounter{28}\vismHypertarget{I.n28}{}\emph{Avadhi—}“limit” = \emph{odhi}: this form is not in PED (see \textbf{\cite{M-a}II 292}).} of a purpose; the purpose in the meditator’s making use of the robes is that much only, namely, protection from cold, etc., not more than that. \emph{From cold}: from any kind of cold arisen either through disturbance of elements internally or through change in temperature externally. \emph{For protection}: for the purpose of warding off; for the purpose of eliminating it so that it may not arouse affliction in the body. For when the body is afflicted by cold, the distracted mind cannot be wisely exerted. That is why the Blessed One permitted the robe to be used for protection from cold. So in each instance, except that \emph{from heat }means from the heat of fire, the origin of which should be understood as forest fires, and so on.

                            \vismParagraph{I.87}{87}{}
                            \emph{From contact with gadflies and flies, wind and burning and creeping things: }here \emph{gadflies }are flies that bite; they are also called “blind flies.” \emph{Flies }are just flies. \emph{Wind }is distinguished as that with dust and that without dust. \emph{Burning }is burning of the sun. \emph{Creeping things }are any long creatures such as snakes and so on that move by crawling. Contact with them is of two kinds: contact by being bitten and contact \marginnote{\textcolor{teal}{\footnotesize\{90|32\}}}{}by being touched. And that does not worry him who sits with a robe on. So he uses it for the purpose of protection from such things.

                            \vismParagraph{I.88}{88}{}
                            \emph{Only}: the word is repeated in order to define a subdivision of the invariable purpose; for the concealment of the private parts is an invariable purpose; the others are purposes periodically. Herein, \emph{private parts }are any parts of the pudendum. For when a member is disclosed, conscience (\emph{hiri}) is disturbed (\emph{kuppati}), offended. It is called “private parts” (\emph{hirikopīna}) because of the disturbance of conscience (\emph{hiri-kopana}). \emph{For the purpose of concealing the private parts: }for the purpose of the concealment of those private parts. [As well as the reading “\emph{hiriko-pīna-paṭicchādanatthaṃ}] there is a reading “\emph{hirikopīnaṃ paṭicchādanatthaṃ}.”

                            \vismParagraph{I.89}{89}{}
                            \emph{Alms food }is any sort of food. For any sort of nutriment is called “alms food” (\emph{piṇḍapāta—}lit. “lump-dropping”) because of its having been dropped (\emph{patitattā}) into a bhikkhu’s bowl during his alms round (\emph{piṇḍolya}). Or alms food (\emph{piṇḍapāta}) is the dropping (\emph{pāta}) of the lumps (\emph{piṇḍa}); it is the concurrence (\emph{sannipāta}), the collection, of alms (\emph{bhikkhā}) obtained here and there, is what is meant.

                            \emph{Neither for amusement}: neither for the purpose of amusement, as with village boys, etc.; for the sake of sport, is what is meant. \emph{Nor for intoxication}: not for the purpose of intoxication, as with boxers, etc.; for the sake of intoxication with strength and for the sake of intoxication with manhood, is what is meant. \textcolor{brown}{\textit{[32]}} \emph{Nor for smartening}: not for the purpose of smartening, as with royal concubines, courtesans, etc.; for the sake of plumpness in all the limbs, is what is meant. \emph{Nor for embellishment}: not for the purpose of embellishment, as with actors, dancers, etc.; for the sake of a clear skin and complexion, is what is meant.

                            \vismParagraph{I.90}{90}{}
                            And here the clause \emph{neither for amusement }is stated for the purpose of abandoning support for delusion; \emph{nor for intoxication }is said for the purpose of abandoning support for hate; \emph{nor for smartening nor for embellishment }is said for the purpose of abandoning support for greed. And \emph{neither for amusement nor for intoxication }is said for the purpose of preventing the arising of fetters for oneself. \emph{Nor for smartening nor for embellishment }is said for the purpose of preventing the arising of fetters for another. And the abandoning of both unwise practice and devotion to indulgence of sense pleasures should be understood as stated by these four. \emph{Only }has the meaning already stated.

                            \vismParagraph{I.91}{91}{}
                            \emph{Of this body}: of this material body consisting of the four great primaries. \emph{For the endurance}: for the purpose of continued endurance. \emph{And continuance}: for the purpose of not interrupting [life’s continued] occurrence, or for the purpose of endurance for a long time. He makes use of the alms food for the purpose of the endurance, for the purpose of the continuance, of the body, as the owner of an old house uses props for his house, and as a carter uses axle grease, not for the purpose of amusement, intoxication, smartening, and embellishment. Furthermore, endurance is a term for the life faculty. So what has been said as far as the words \emph{for the endurance and continuance of this body }can be understood to mean: for the purpose of maintaining the occurrence of the life faculty in this body.

                            \vismParagraph{I.92}{92}{}
                            \emph{For the ending of discomfort}: hunger is called “discomfort” in the sense of afflicting. He makes use of alms food for the purpose of ending that, like anointing \marginnote{\textcolor{teal}{\footnotesize\{91|33\}}}{}a wound, like counteracting heat with cold, and so on. \emph{For assisting the life of purity}: for the purpose of assisting the life of purity consisting in the whole dispensation and the life of purity consisting in the path. For while this [bhikkhu] is engaged in crossing the desert of existence by means of devotion to the three trainings depending on bodily strength whose necessary condition is the use of alms food, he makes use of it to assist the life of purity just as those seeking to cross the desert used their child’s flesh,\footnote{\vismAssertFootnoteCounter{29}\vismHypertarget{I.n29}{}“\emph{Child’s flesh}” (\emph{putta-maṃsa}) is an allusion to the story (\textbf{\cite{S}II 98}) of the couple who set out to cross a desert with an insufficient food supply but got to the other side by eating the flesh of their child who died on the way. The derivation given in PED, “A metaphor probably distorted from \emph{pūtamaṃsa},” has no justification. The reference to rafts might be to \textbf{\cite{D}II 89}.} just as those seeking to cross a river use a raft, and just as those seeking to cross the ocean use a ship.

                            \vismParagraph{I.93}{93}{}
                            \emph{Thus I shall put a stop to old feelings and shall not arouse new feelings}: \textcolor{brown}{\textit{[33]}} thus as a sick man uses medicine, he uses [alms food, thinking]: “By use of this alms food I shall put a stop to the old feeling of hunger, and I shall not arouse a new feeling by immoderate eating, like one of the [proverbial] brahmans, that is, one who eats till he has to be helped up by hand, or till his clothes will not meet, or till he rolls there [on the ground], or till crows can peck from his mouth, or until he vomits what he has eaten. Or alternatively, there is that which is called ‘old feelings’ because, being conditioned by former kamma, it arises now in dependence on unsuitable immoderate eating—I shall put a stop to that old feeling, forestalling its condition by suitable moderate eating. And there is that which is called ‘new feeling’ because it will arise in the future in dependence on the accumulation of kamma consisting in making improper use [of the requisite of alms food] now—I shall also not arouse that new feeling, avoiding by means of proper use the production of its root.” This is how the meaning should be understood here. What has been shown so far can be understood to include proper use [of requisites], abandoning of devotion to self-mortification, and not giving up lawful bliss (pleasure).

                            \vismParagraph{I.94}{94}{}
                            \emph{And I shall be healthy}: “In this body, which exists in dependence on requisites, I shall, by moderate eating, have health called ‘long endurance’ since there will be no danger of severing the life faculty or interrupting the [continuity of the] postures.” [Reflecting] in this way, he makes use [of the alms food] as a sufferer from a chronic disease does of his medicine. \emph{And blameless and live in comfort }(lit. “and have blamelessness and a comfortable abiding”): he makes use of them thinking: “I shall have blamelessness by avoiding improper search, acceptance and eating, and I shall have a comfortable abiding by moderate eating.” Or he does so thinking: “I shall have blamelessness due to absence of such faults as boredom, sloth, sleepiness, blame by the wise, etc., that have unseemly immoderate eating as their condition; and I shall have a comfortable abiding by producing bodily strength that has seemly moderate eating as its condition.” Or he does so thinking: “I shall have blamelessness by abandoning the pleasure of lying down, lolling and torpor, through refraining from eating as much as possible to stuff the belly; and I shall have a comfortable abiding by controlling the four postures through eating four or five mouthfuls less than the maximum.” For this is said:
                            \begin{verse}
                                \marginnote{\textcolor{teal}{\footnotesize\{92|34\}}}{}With four or five lumps still to eat\\{}
                                Let him then end by drinking water;\\{}
                                For energetic bhikkhus’ needs\\{}
                                This should suffice to live in comfort (\textbf{\cite{Th}983}). \textcolor{brown}{\textit{[34]}}
                            \end{verse}


                            Now, what has been shown at this point can be understood as discernment of purpose and practice of the middle way.

                            \vismParagraph{I.95}{95}{}
                            \emph{Resting place} (\emph{senāsana}): this is the bed (\emph{sena}) and seat (\emph{āsana}). For wherever one sleeps (\emph{seti}), whether in a monastery or in a lean-to, etc., that is the bed (\emph{sena}); wherever one seats oneself (\emph{āsati}), sits \emph{(nisīdati}), that is the seat (\emph{āsana}). Both together are called “resting-place” (or “abode”—\emph{senāsana}).

                            \emph{For the purpose of warding off the perils of climate and enjoying retreat}: the climate itself in the sense of imperilling (\emph{parisahana}) is “perils of climate” (\emph{utu-parissaya}). Unsuitable climatic conditions that cause mental distraction due to bodily affliction can be warded off by making use of the resting place; it is for the purpose of warding off these and for the purpose of the pleasure of solitude, is what is meant. Of course, the warding off of the perils of climate is stated by [the phrase] “protection from cold,” etc., too; but, just as in the case of making use of the robes the concealment of the private parts is stated as an invariable purpose while the others are periodical [purposes], so here also this [last] should be understood as mentioned with reference to the invariable warding off of the perils of climate. Or alternatively, this “climate” of the kind stated is just climate; but “perils” are of two kinds: evident perils and concealed perils (see \textbf{\cite{Nidd}I 12}). Herein, evident perils are lions, tigers, etc., while concealed perils are greed, hate, and so on. When a bhikkhu knows and reflects thus in making use of the kind of resting place where these [perils] do not, owing to unguarded doors and sight of unsuitable visible objects, etc., cause affliction, he can be understood as one who “reflecting wisely makes use of the resting place for the purpose of warding off the perils of climate.”

                            \vismParagraph{I.96}{96}{}
                            \emph{The requisite of medicine as cure for the sick}: here “cure” (\emph{paccaya = }going against) is in the sense of going against (\emph{pati-ayana}) illness; in the sense of countering, is the meaning. This is a term for any suitable remedy. It is the medical man’s work (\emph{bhisakkassa kammaṃ}) because it is permitted by him, thus it is medicine (\emph{bhesajja}). Or the cure for the sick itself as medicine is “medicine as cure for the sick.” Any work of a medical man such as oil, honey, ghee, etc., that is suitable for one who is sick, is what is meant. A “requisite” (\emph{parikkhāra}), however, in such passages as “It is well supplied with the requisites of a city” (\textbf{\cite{A}IV 106}) is equipment; in such passages as “The chariot has the requisite of virtue, the axle of jhāna, the wheel of energy” (\textbf{\cite{S}V 6}) \textcolor{brown}{\textit{[35]}} it is an ornament; in such passages as “The requisites for the life of one who has gone into homelessness that should be available” (\textbf{\cite{M}I 104}), it is an accessory. But here both equipment and accessory are applicable. For that medicine as a cure for the sick is equipment for maintaining life because it protects by preventing the arising of affliction destructive to life; and it is an accessory too because it is an instrument for prolonging life. That is why it is called “requisite.” So it is medicine as cure for the sick and that is a requisite, thus it is a “requisite of medicine as cure for the sick.” [He makes use of] that requisite of medicine as cure \marginnote{\textcolor{teal}{\footnotesize\{93|35\}}}{}for the sick; any requisite for life consisting of oil, honey, molasses, ghee, etc., that is allowed by a medical man as suitable for the sick, is what is meant.

                            \vismParagraph{I.97}{97}{}
                            \emph{From arisen}: from born, become, produced. \emph{Hurtful: }here “hurt (affliction)” is a disturbance of elements, and it is the leprosy, tumours, boils, etc., originated by that disturbance. \emph{Hurtful }(\emph{veyyābādhika}) because arisen in the form of hurt (\emph{byābādha}). \emph{Feelings}: painful feelings, feelings resulting from unprofitable kamma—from those hurtful feelings. \emph{For complete immunity from affliction}: for complete freedom from pain; so that all that is painful is abandoned, is the meaning.

                            This is how this \emph{virtue concerning requisites }should be understood. In brief its characteristic is the use of requisites after wise reflection. The word-meaning here is this: because breathing things go (\emph{ayanti}), move, proceed, using [what they use] in dependence on these robes, etc., these robes, etc., are therefore called requisites (\emph{paccaya = }ger. of \emph{paṭi + ayati}); “concerning requisites” is concerning those requisites.
                        \par\noindent[\textsc{\textbf{(a) Pātimokha restraint by means of faith}}]

                            \vismParagraph{I.98}{98}{}
                            (a) So, in this fourfold virtue, \emph{Pātimokkha restraint }has to be undertaken by means of faith. For that is accomplished by faith, since the announcing of training precepts is outside the disciples’ province; and the evidence here is the refusal of the request to [allow disciples to] announce training precepts (see \textbf{\cite{Vin}III 9–10}). Having therefore undertaken through faith the training precepts without exception as announced, one should completely perfect them without regard for life. For this is said: \textcolor{brown}{\textit{[36]}}
                            \begin{verse}
                                “As a hen guards her eggs,\\{}
                                Or as a yak her tail,\\{}
                                Or like a darling child,\\{}
                                Or like an only eye—\\{}
                                So you who are engaged\\{}
                                Your virtue to protect,\\{}
                                Be prudent at all times\\{}
                                And ever scrupulous.” (\emph{Source untraced})
                            \end{verse}


                            Also it is said further: “So too, sire, when a training precept for disciples is announced by me, my disciples do not transgress it even for the sake of life” (\textbf{\cite{A}IV 201}).

                            \vismParagraph{I.99}{99}{}
                            And the story of the elders bound by robbers in the forest should be understood in this sense.

                            It seems that robbers in the Mahāvaṭṭanī Forest bound an elder with black creepers and made him lie down. While he lay there for seven days he augmented his insight, and after reaching the fruition of non-return, he died there and was reborn in the Brahmā-world. Also they bound another elder in Tambapaṇṇi Island (Sri Lanka) with string creepers and made him lie down. When a forest fire came and the creepers were not cut, he established insight and attained Nibbāna simultaneously with his death. When the Elder Abhaya, a preacher of the Dīgha Nikāya, passed by with five hundred bhikkhus, he saw [what had happened] and he had the elder’s body cremated and a shrine built. Therefore let other clansmen also:
                            \begin{verse}
                                Maintain the rules of conduct pure,\\{}
                                Renouncing life if there be need,\\{}
                                \marginnote{\textcolor{teal}{\footnotesize\{94|36\}}}{}Rather than break virtue’s restraint\\{}
                                By the World’s Saviour decreed.
                            \end{verse}

                        \par\noindent[\textsc{\textbf{(b) Restraint of the sense faculties by means of mindfulness}}]

                            \vismParagraph{I.100}{100}{}
                            (b) And as Pātimokkha restraint is undertaken out of faith, so \emph{restraint of the sense faculties }should be undertaken with \emph{mindfulness}. For that is accomplished by mindfulness, because when the sense faculties’ functions are founded on mindfulness, there is no liability to invasion by covetousness and the rest. So, recollecting the Fire Discourse, which begins thus, “Better, bhikkhus, the extirpation of the eye faculty by a red-hot burning blazing glowing iron spike than the apprehension of signs in the particulars of visible objects cognizable by the eye” (\textbf{\cite{S}IV 168}), this [restraint] should be properly undertaken by preventing with unremitting mindfulness any apprehension, in the objective fields consisting of visible data, etc., of any signs, etc., likely to encourage covetousness, etc., to invade consciousness occurring in connection with the eye door, and so on.

                            \vismParagraph{I.101}{101}{}
                            \textcolor{brown}{\textit{[37]}} When not undertaken thus, virtue of Pātimokkha restraint is unenduring: it does not last, like a crop not fenced in with branches. And it is raided by the robber defilements as a village with open gates is by thieves. And lust leaks into his mind as rain does into a badly-roofed house. For this is said:
                            \begin{verse}
                                “Among the visible objects, sounds, and smells,\\{}
                                And tastes, and tangibles, guard the faculties;\\{}
                                For when these doors are open and unguarded,\\{}
                                Then thieves will come and raid as ’twere a village (?).
                            \end{verse}

                            \begin{verse}
                                And just as with an ill-roofed house\\{}
                                The rain comes leaking in, so too\\{}
                                Will lust come leaking in for sure\\{}
                                Upon an undeveloped mind” (\textbf{\cite{Dhp}13}).
                            \end{verse}


                            \vismParagraph{I.102}{102}{}
                            When it is undertaken thus, virtue of Pātimokkha restraint is enduring: it lasts, like a crop well fenced in with branches. And it is not raided by the robber defilements, as a village with well-guarded gates is not by thieves. And lust does not leak into his mind, as rain does not into a well-roofed house. For this is said:
                            \begin{verse}
                                “Among the visible objects, sounds and smells,\\{}
                                And tastes and tangibles, guard the faculties;\\{}
                                For when these doors are closed and truly guarded,\\{}
                                Thieves will not come and raid as ’twere a village (?).
                            \end{verse}

                            \begin{verse}
                                “And just as with a well-roofed house\\{}
                                No rain comes leaking in, so too\\{}
                                No lust comes leaking in for sure\\{}
                                Upon a well-developed mind” (\textbf{\cite{Dhp}14}).
                            \end{verse}


                            \vismParagraph{I.103}{103}{}
                            This, however, is the teaching at its very highest.

                            This mind is called “quickly transformed” (\textbf{\cite{A}I 10}), so restraint of the faculties should be undertaken by removing arisen lust with the contemplation of foulness, as was done by the Elder Vaṅgīsa soon after he had gone forth. \textcolor{brown}{\textit{[38]}}

                            As the elder was wandering for alms, it seems, soon after going forth, lust arose in him on seeing a woman. Thereupon he said to the venerable Ānanda:
                            \begin{verse}
                                \marginnote{\textcolor{teal}{\footnotesize\{95|37\}}}{}“I am afire with sensual lust.\\{}
                                And burning flames consume my mind;\\{}
                                In pity tell me, Gotama,\\{}
                                How to extinguish it for good” (\textbf{\cite{S}I 188}).
                            \end{verse}


                            The elder said:
                            \begin{verse}
                                “You do perceive mistakenly,\\{}
                                That burning flames consume your mind.\\{}
                                Look for no sign of beauty there,\\{}
                                For that it is which leads to lust.\\{}
                                See foulness there and keep your mind\\{}
                                Harmoniously concentrated;\\{}
                                Formations see as alien,\\{}
                                As ill, not self, so this great lust\\{}
                                May be extinguished, and no more\\{}
                                Take fire thus ever and again” (\textbf{\cite{S}I 188}).
                            \end{verse}


                            The elder expelled his lust and then went on with his alms round.

                            \vismParagraph{I.104}{104}{}
                            Moreover, a bhikkhu who is fulfilling restraint of the faculties should be like the Elder Cittagutta resident in the Great Cave at Kuraṇḍaka, and like the Elder Mahā Mitta resident at the Great Monastery of Coraka.

                            \vismParagraph{I.105}{105}{}
                            In the Great Cave of Kuraṇḍaka, it seems, there was a lovely painting of the Renunciation of the Seven Buddhas. A number of bhikkhus wandering about among the dwellings saw the painting and said, “What a lovely painting, venerable sir!” The elder said: “For more than sixty years, friends, I have lived in the cave, and I did not know whether there was any painting there or not. Now, today, I know it through those who have eyes.” The elder, it seems, though he had lived there for so long, had never raised his eyes and looked up at the cave. And at the door of his cave there was a great ironwood tree. And the elder had never looked up at that either. He knew it was in flower when he saw its petals on the ground each year.

                            \vismParagraph{I.106}{106}{}
                            The king heard of the elder’s great virtues, and he sent for him three times, desiring to pay homage to him. When the elder did not go, he had the breasts of all the women with infants in the town bound and sealed off, [saying] “As long as the elder does not come let the children go without milk,” \textcolor{brown}{\textit{[39]}} Out of compassion for the children the elder went to Mahāgāma. When the king heard [that he had come, he said] “Go and bring the elder in. I shall take the precepts.” Having had him brought up into the inner palace, he paid homage to him and provided him with a meal. Then, saying, “Today, venerable sir, there is no opportunity. I shall take the precepts tomorrow,” he took the elder’s bowl. After following him for a little, he paid homage with the queen and turned back. As seven days went by thus, whether it was the king who paid homage or whether it was the queen, the elder said, “May the king be happy.”

                            \vismParagraph{I.107}{107}{}
                            Bhikkhus asked: “Why is it, venerable sir, that whether it is the king who pays the homage or the queen you say ‘May the king be happy’?” The elder replied: “Friends, I do not notice whether it is the king or the queen.” At the end of seven days [when it was found that] the elder was not happy living there, he was dismissed \marginnote{\textcolor{teal}{\footnotesize\{96|38\}}}{}by the king. He went back to the Great Cave at Kuraṇḍaka. When it was night he went out onto his walk. A deity who dwelt in the ironwood tree stood by with a torch of sticks. Then his meditation subject became quite clear and plain. The elder, [thinking] “How clear my meditation subject is today!” was glad, and immediately after the middle watch he reached Arahantship, making the whole rock resound.\footnote{\vismAssertFootnoteCounter{30}\vismHypertarget{I.n30}{}“‘\emph{Making the whole rock resound}’: ‘making the whole rock reverberate as one doing so by means of an earth tremor. But some say that is was owing to the cheering of the deities who lived there’” (\textbf{\cite{Vism-mhṭ}58}).}

                            \vismParagraph{I.108}{108}{}
                            So when another clansman seeks his own good:
                            \begin{verse}
                                Let him not be hungry-eyed,\\{}
                                Like a monkey in the groves,\\{}
                                Like a wild deer in the woods,\\{}
                                Like a nervous little child.\\{}
                                Let him go with eyes downcast\\{}
                                Seeing a plough yoke’s length before,\\{}
                                That he fall not in the power\\{}
                                Of the forest-monkey mind.
                            \end{verse}


                            \vismParagraph{I.109}{109}{}
                            The Elder Mahā Mitta’s mother was sick with a poisoned tumour. She told her daughter, who as a bhikkhunī had also gone forth, “Lady, go to your brother. Tell him my trouble and bring back some medicine.” She went and told him. The elder said: “I do not know how to gather root medicines and such things and concoct a medicine from them. But rather I will tell you a medicine: since I went forth I have not broken [my virtue of restraint of] the sense faculties by looking at the bodily form of the opposite sex with a lustful mind. By this \textcolor{brown}{\textit{[40]}} declaration of truth may my mother get well. Go and tell the lay devotee and rub her body.” She went and told her what had happened and then did as she had been instructed. At that very moment the lay devotee’s tumour vanished, shrinking away like a lump of froth. She got up and uttered a cry of joy: “If the Fully Enlightened One were still alive, why should he not stroke with his netadorned hand the head of a bhikkhu like my son?” So:

                            \vismParagraph{I.110}{110}{}
                            
                            \begin{verse}
                                Let another noble clansman\\{}
                                Gone forth in the Dispensation\\{}
                                Keep, as did the Elder Mitta,\\{}
                                Perfect faculty restraint.
                            \end{verse}

                        \par\noindent[\textsc{\textbf{(c) Livelihood purification by means of energy}}]

                            \vismParagraph{I.111}{111}{}
                            (c) As restraint of the faculties is to be undertaken by means of mindfulness, so \emph{livelihood purification }is to be undertaken by means of \emph{energy}. For that is accomplished by energy, because the abandoning of wrong livelihood is effected in one who has rightly applied energy. Abandoning, therefore, unbefitting wrong search, this should be undertaken with energy by means of the right kind of search consisting in going on alms round, etc., avoiding what is of impure origin as though it were a poisonous snake, and using only requisites of pure origin.

                            \vismParagraph{I.112}{112}{}
                            Herein, for one who has not taken up the ascetic practices, any requisites obtained from the Community, from a group of bhikkhus, or from laymen who have confidence in his special qualities of teaching the Dhamma, etc., are called “of pure origin.” But \marginnote{\textcolor{teal}{\footnotesize\{97|39\}}}{}those obtained on alms round, etc., are of extremely pure origin. For one who has taken up the ascetic practices, those obtained on alms round, etc., and—as long as this is in accordance with the rules of the ascetic practices—from people who have confidence in his special qualities of asceticism, are called “of pure origin.” And if he has got putrid urine with mixed gall nuts and “four-sweets”\footnote{\vismAssertFootnoteCounter{31}\vismHypertarget{I.n31}{}“Four-sweets”—\emph{catumadhura}: a medicinal sweet made of four ingredients: honey, palm-sugar, ghee and sesame oil.} for the purpose of curing a certain affliction, and he eats only the broken gall nuts, thinking, “Other companions in the life of purity will eat the ‘four-sweets’,” his undertaking of the ascetic practices is befitting, for he is then called a bhikkhu who is supreme in the Noble Ones’ heritages (\textbf{\cite{A}II 28}).

                            \vismParagraph{I.113}{113}{}
                            As to the robe and the other requisites, no hint, indication, roundabout talk, or intimation about robes and alms food is allowable for a bhikkhu who is purifying his livelihood. But a hint, indication, or roundabout talk about a resting place is allowable for one who has not taken up the ascetic practices. \textcolor{brown}{\textit{[41]}}

                            \vismParagraph{I.114}{114}{}
                            Herein, a “hint” is when one who is getting the preparing of the ground, etc., done for the purpose of [making] a resting place is asked, “What is being done, venerable sir? Who is having it done?” and he replies, “No one”; or any other such giving of hints. An “indication” is saying, “Lay follower, where do you live?”—”In a mansion, venerable sir”—”But, lay follower, a mansion is not allowed for bhikkhus.” Or any other such giving of indication. “Roundabout talk” is saying, “The resting place for the Community of Bhikkhus is crowded”; or any other such oblique talk.

                            \vismParagraph{I.115}{115}{}
                            All, however, is allowed in the case of medicine. But when the disease is cured, is it or is it not allowed to use the medicine obtained in this way? Herein, the Vinaya specialists say that the opening has been given by the Blessed One, therefore it is allowable. But the Suttanta specialists say that though there is no offence, nevertheless the livelihood is sullied, therefore it is not allowable.

                            \vismParagraph{I.116}{116}{}
                            But one who does not use hints, indications, roundabout talk, or intimation, though these are permitted by the Blessed One, and who depends only on the special qualities of fewness of wishes, etc., and makes use only of requisites obtained otherwise than by indication, etc., even when he thus risks his life, is called supreme in living in effacement, like the venerable Sāriputta.

                            \vismParagraph{I.117}{117}{}
                            It seems that the venerable one was cultivating seclusion at one time, living in a certain forest with the Elder Mahā Moggallāna. One day an affliction of colic arose in him, causing him great pain. In the evening the Elder Mahā Moggallāna went to attend upon him. Seeing him lying down, he asked what the reason was. And then he asked, “What used to make you better formerly, friend?” The elder said, “When I was a layman, friend, my mother used to mix ghee, honey, sugar and so on, and give me rice gruel with pure milk. That used to make me better.” Then the other said, “So be it, friend. If either you or I have merit, perhaps tomorrow we shall get some.”

                            \vismParagraph{I.118}{118}{}
                            Now, a deity who dwelt in a tree at the end of the walk overheard their conversation. [Thinking] “I will find rice gruel for the lord tomorrow,” he went \marginnote{\textcolor{teal}{\footnotesize\{98|40\}}}{}meanwhile to the family who was supporting the elder \textcolor{brown}{\textit{[42]}} and entered into the body of the eldest son, causing him discomfort. Then he told the assembled relatives the price of the cure: “If you prepare rice gruel of such a kind tomorrow for the elder, I will set this one free.” They said: “Even without being told by you we regularly supply the elder’s needs,” and on the following day they prepared rice gruel of the kind needed.

                            \vismParagraph{I.119}{119}{}
                            The Elder Mahā Moggallāna came in the morning and said, “Stay here, friend, till I come back from the alms round.” Then he went into the village. Those people met him. They took his bowl, filled it with the stipulated kind of rice gruel, and gave it back to him. The elder made as though to go, but they said, “Eat, venerable sir, we shall give you more.” When the elder had eaten, they gave him another bowlful. The elder left. Bringing the alms food to the venerable Sāriputta, he said, “Here, friend Sāriputta, eat.” When the elder saw it, he thought, “The gruel is very nice. How was it got?” and seeing how it had been obtained, he said, “Friend, the alms food cannot be used.”

                            \vismParagraph{I.120}{120}{}
                            Instead of thinking, “He does not eat alms food brought by the likes of me,” the other at once took the bowl by the rim and turned it over on one side. As the rice gruel fell on the ground the elder’s affliction vanished. From then on it did not appear again during forty-five years.

                            \vismParagraph{I.121}{121}{}
                            Then he said to the venerable Mahā Moggallāna, “Friend, even if one’s bowels come out and trail on the ground, it is not fitting to eat gruel got by verbal intimation,” and he uttered this exclamation:
                            \begin{verse}
                                My livelihood might well be blamed\\{}
                                If I were to consent to eat\\{}
                                The honey and the gruel obtained\\{}
                                By influence of verbal hints.
                            \end{verse}

                            \begin{verse}
                                And even if my bowels obtrude\\{}
                                And trail outside, and even though\\{}
                                My life is to be jeopardized,\\{}
                                I will not blot my livelihood (\textbf{\cite{Mil}370}).
                            \end{verse}

                            \begin{verse}
                                For I will satisfy my heart\\{}
                                By shunning all wrong kinds of search;\\{}
                                And never will I undertake\\{}
                                The search the Buddhas have condemned. \textcolor{brown}{\textit{[43]}}
                            \end{verse}


                            \vismParagraph{I.122}{122}{}
                            And here too should be told the story of the Elder Mahā Tissa the Mango-eater who lived at Cīragumba\footnote{\vismAssertFootnoteCounter{32}\vismHypertarget{I.n32}{}“The Elder Mahā Tissa, it seems, was going on a journey during a famine, and being tired in body and weak through lack of food and travel weariness, he lay down at the root of a mango tree covered with fruit. There were many fallen mangoes here and there” (\textbf{\cite{Vism-mhṭ}60}). “Through ownerless mangoes were lying fallen on the ground near him, he would not eat them in the absence of someone to accept them from” (Vism-mhṭ 65). “Then a lay devotee, who was older than he, went to the elder, and learning of his exhaustion, gave him mango juice to drink. Then he mounted him on his back and took him to his home. Meanwhile the elder admonished himself as follows: ‘Nor your mother nor your father,’ etc. (see §133). And beginning the comprehension [of formations], and augmenting insight, he realized Arahantship after the other paths in due succession while he was still mounted on his back” (\textbf{\cite{Vism-mhṭ}60}).} (see \hyperlink{I.132}{§132}{} below). So in all respects:
                            \begin{verse}
                                \marginnote{\textcolor{teal}{\footnotesize\{99|41\}}}{}A man who has gone forth in faith\\{}
                                Should purify his livelihood\\{}
                                And, seeing clearly, give no thought\\{}
                                To any search that is not good.
                            \end{verse}

                        \par\noindent[\textsc{\textbf{(d) Virtue dependent on requisites by means of understanding}}]

                            \vismParagraph{I.123}{123}{}
                            (d) And as livelihood purification is to be undertaken by means of energy, so \emph{virtue dependent on requisites }is to be undertaken by means of \emph{understanding}. For that is accomplished by understanding, because one who possesses understanding is able to see the advantages and the dangers in requisites. So one should abandon greed for requisites and undertake that virtue by using requisites obtained lawfully and properly, after reviewing them with understanding in the way aforesaid.

                            \vismParagraph{I.124}{124}{}
                            Herein, reviewing is of two kinds: at the time of receiving requisites and at the time of using them. For use (\emph{paribhoga}) is blameless in one who at the time of receiving robes, etc., reviews them either as [mere] elements or as repulsive,\footnote{\vismAssertFootnoteCounter{33}\vismHypertarget{I.n33}{}“‘\emph{As elements}’ in this way: ‘This robe, etc., consists merely of [the four] elements and occurs when its conditions are present; and the person who uses it [likewise].’ ‘\emph{As repulsive}’ in this way: Firstly perception of repulsiveness in nutriment in the case of alms food; then as bringing repulsiveness to mind thus: ‘But all these robes, etc., which are not in themselves disgusting, become utterly disgusting on reaching this filthy body’” (\textbf{\cite{Vism-mhṭ}61}).} and puts them aside for later use, and in one who reviews them thus at the time of using them.

                            \vismParagraph{I.125}{125}{}
                            Here is an explanation to settle the matter. There are four kinds of use: use as theft,\footnote{\vismAssertFootnoteCounter{34}\vismHypertarget{I.n34}{}“‘\emph{Use as theft}’: use by one who is unworthy. And the requisites are allowed by the Blessed One to one in his own dispensation who is virtuous, not unvirtuous; and the generosity of the givers is towards one who is virtuous, not towards one who is not, since they expect great fruit from their actions” (\textbf{\cite{Vism-mhṭ}61}; cf. MN 142 and commentary).} use as a debt?, use as an inheritance, use as a master. Herein, use by one who is unvirtuous and makes use [of requisites], even sitting in the midst of the Community, is called “use as theft.” Use without reviewing by one who is virtuous is “use as a debt”; therefore the robe should be reviewed every time it is used, and the alms food lump by lump. One who cannot do this [should review it] before the meal, after the meal, in the first watch, in the middle watch, and in the last watch. If dawn breaks on him without his having reviewed it, he finds himself in the position of one who has used it as a debt. Also the resting place should be reviewed each time it is used. Recourse to mindfulness both in the accepting and the use of medicine is proper; but while this is so, though there is an offence for one who uses it without mindfulness after mindful acceptance, there is no offence for one who is mindful in using after accepting without mindfulness.

                            \vismParagraph{I.126}{126}{}
                            Purification is of four kinds: purification by the Teaching, purification by restraint, purification by search, and purification by reviewing. Herein, \emph{virtue of }\marginnote{\textcolor{teal}{\footnotesize\{100|42\}}}{}\emph{the Pātimokkha restraint }is called “purification by the Teaching”; \textcolor{brown}{\textit{[44]}} for that is so called because it purifies by means of teaching. \emph{Virtue of restraint of faculties }is called “purification by restraint”; for that is so called because it purifies by means of the restraint in the mental resolution, “I shall not do so again.” \emph{Virtue of livelihood purification }is called “purification by search”; for that is so called because search is purified in one who abandons wrong search and gets requisites lawfully and properly. \emph{Virtue dependent on requisites }is called “purification by reviewing”; for that is so called because it purifies by the reviewing of the kind already described. Hence it was said above (\hyperlink{I.125}{§125}{}): “There is no offence for one who is mindful in using after accepting without mindfulness.”

                            \vismParagraph{I.127}{127}{}
                            Use of the requisites by the seven kinds of trainers is called “use as an inheritance”; for they are the Buddha’s sons, therefore they make use of the requisites as the heirs of requisites belonging to their father. But how then, is it the Blessed One’s requisites or the laity’s requisites that are used? Although given by the laity, they actually belong to the Blessed One, because it is by the Blessed One that they are permitted. That is why it should be understood that the Blessed One’s requisites are used. The confirmation here is in the Dhammadāyāda Sutta (MN 3).

                            Use by those whose cankers are destroyed is called “use as a master”; for they make use of them as masters because they have escaped the slavery of craving.

                            \vismParagraph{I.128}{128}{}
                            As regards these kinds of use, use as a master and use as an inheritance are allowable for all. Use as a debt is not allowable, to say nothing of use as theft. But this use of what is reviewed by one who is virtuous is use freed from debt because it is the opposite of use as a debt or is included in use as an inheritance too. For one possessed of virtue is called a trainer too because of possessing this training.

                            \vismParagraph{I.129}{129}{}
                            As regards these three kinds of use, since use as a master is best, when a bhikkhu undertakes \emph{virtue dependent on requisites}, he should aspire to that and use them after reviewing them in the way described. And this is said: \textcolor{brown}{\textit{[45]}}
                            \begin{verse}
                                “The truly wise disciple\\{}
                                Who listens to the Dhamma\\{}
                                As taught by the Sublime One\\{}
                                Makes use, after reviewing,\\{}
                                Of alms food, and of dwelling,\\{}
                                And of a resting place,\\{}
                                And also of the water\\{}
                                For washing dirt from robes” (\textbf{\cite{Sn}391}).
                            \end{verse}

                            \begin{verse}
                                “So like a drop of water\\{}
                                Lying on leaves of lotus,\\{}
                                A bhikkhu is unsullied\\{}
                                By any of these matters,\\{}
                                By alms food, [and by dwelling,]\\{}
                                And by a resting place,\\{}
                                And also by the water\\{}
                                For washing dirt from robes” (\textbf{\cite{Sn}392}).
                            \end{verse}

                            \begin{verse}
                                \marginnote{\textcolor{teal}{\footnotesize\{101|43\}}}{}“Since aid it is and timely\\{}
                                Procured from another\\{}
                                The right amount he reckons,\\{}
                                Mindful without remitting\\{}
                                In chewing and in eating,\\{}
                                In tasting food besides:\\{}
                                He treats it as an ointment\\{}
                                Applied upon a wound.” (Source untraced)
                            \end{verse}

                            \begin{verse}
                                “So like the child’s flesh in the desert\\{}
                                Like the greasing for the axle,\\{}
                                He should eat without delusion\\{}
                                Nutriment to keep alive.” (Source untraced)
                            \end{verse}


                            \vismParagraph{I.130}{130}{}
                            And in connection with the fulfilling of this virtue dependent on requisites there should be told the story of the novice Saṅgharakkhita the Nephew. For he made use of requisites after reviewing, according as it is said:
                            \begin{verse}
                                “Seeing me eat a dish of rice\\{}
                                Quite cold, my preceptor observed:\\{}
                                ‘Novice, if you are not restrained,\\{}
                                Be careful not to burn your tongue.’\\{}
                                On hearing my Preceptor’s words,\\{}
                                I then and there felt urged to act\\{}
                                And, sitting in a single session,\\{}
                                I reached the goal of Arahantship.\\{}
                                Since I am now waxed full in thought\\{}
                                Like the full moon of the fifteenth (\textbf{\cite{M}III 277}),\\{}
                                And all my cankers are destroyed,\\{}
                                There is no more becoming now.” \textcolor{brown}{\textit{[46]}}
                            \end{verse}

                            \begin{verse}
                                And so should any other man\\{}
                                Aspiring to end suffering\\{}
                                Make use of all the requisites\\{}
                                Wisely after reviewing them.
                            \end{verse}


                            So virtue is of four kinds as “virtue of Pātimokkha restraint,” and so on.
                \subsection[\vismAlignedParas{§131–142}18.–19. Pentads]{18.–19. Pentads}

                    \vismParagraph{I.131}{131}{}
                    \emph{18.} In the first pentad in the fivefold section the meaning should be understood in accordance with the virtue of those not fully admitted to the Order, and so on. For this is said in the Paṭisambhidā: “(a) What is virtue consisting in limited purification? That of the training precepts for those not fully admitted to the Order: such is virtue consisting in limited purification. (b) What is virtue consisting in unlimited purification? That of the training precepts for those fully admitted to the Order: such is virtue consisting in unlimited purification. (c) What is virtue consisting in fulfilled purification? That of magnanimous ordinary men devoted to profitable things, who are perfecting [the course] that ends in trainership, regardless of the physical body and life, having given up [attachment to] life: such is virtue of fulfilled purification, (d) What is virtue consisting in purification not adhered to? That of the seven kinds of trainer: such is virtue consisting in purification not adhered to. (e) What is virtue consisting in tranquillized purification? That of the Perfect One’s \marginnote{\textcolor{teal}{\footnotesize\{102|44\}}}{}disciples with cankers destroyed, of the Paccekabuddhas, of the Perfect Ones, accomplished and fully enlightened: such is virtue consisting in tranquillized purification” (\textbf{\cite{Paṭis}I 42–43}).

                    \vismParagraph{I.132}{132}{}
                    (a) Herein, the virtue of those not fully admitted to the Order should be understood as \emph{virtue consisting in limited purification}, because it is limited by the number [of training precepts, that is, five or eight or ten].

                    (b) That of those fully admitted to the Order is [describable] thus:
                    \begin{verse}
                        Nine thousand millions, and a hundred\\{}
                        And eighty millions then as well,\\{}
                        And fifty plus a hundred thousand,\\{}
                        And thirty-six again to swell.
                    \end{verse}

                    \begin{verse}
                        The total restraint disciplines:\\{}
                        These rules the Enlightened One explains\\{}
                        Told under heads for filling out,\\{}
                        Which the Discipline restraint contains.\footnote{\vismAssertFootnoteCounter{35}\vismHypertarget{I.n35}{}The figures depend on whether \emph{koṭi} is taken as 1,000,000 or 100,000 or 10,000.}
                    \end{verse}


                    So although limited in number, \textcolor{brown}{\textit{[47]}} it should yet be understood as \emph{virtue consisting in unlimited purification}, since it is undertaken without reserve and has no obvious limit such as gain, fame, relatives, limbs or life. Like the virtue of the Elder Mahā Tissa the Mango-eater who lived at Cīragumba (see \hyperlink{I.122}{§122}{} above).

                    \vismParagraph{I.133}{133}{}
                    For that venerable one never abandoned the following good man’s recollection:
                    \begin{verse}
                        “Wealth for a sound limb’s sake should be renounced,\\{}
                        And one who guards his life gives up his limbs;\\{}
                        And wealth and limbs and life, each one of these,\\{}
                        A man gives up who practices the Dhamma.”
                    \end{verse}


                    And he never transgressed a training precept even when his life was in the balance, and in this way he reached Arahantship with that same virtue of unlimited purification as his support while he was being carried on a lay devotee’s back. According to as it is said:
                    \begin{verse}
                        “Nor your mother nor your father\\{}
                        Nor your relatives and kin\\{}
                        Have done as much as this for you\\{}
                        Because you are possessed of virtue.”\\{}
                        So, stirred with urgency, and wisely\\{}
                        Comprehending\footnote{\vismAssertFootnoteCounter{36}\vismHypertarget{I.n36}{}“Comprehending” (\emph{sammasana}) is a technical term that will become clear in \hyperlink{XX}{Chapter XX}{}. In short, it is inference that generalizes the “three characteristics” from one’s own directly-known experience to all possible formed experience at all times (see \textbf{\cite{S}II 107}). Commenting on “\emph{He comprehended that same illness}” (§138), Vism-mhṭ says: “He exercised insight by discerning the feeling in the illness under the heading of the feeling [aggregate] and the remaining material dhammas as materiality” (\textbf{\cite{Vism-mhṭ}65}).} with insight,\\{}
                        \marginnote{\textcolor{teal}{\footnotesize\{103|45\}}}{}While carried on his helper’s back\\{}
                        He reached the goal of Arahantship.
                    \end{verse}


                    \vismParagraph{I.134}{134}{}
                    (c) The magnanimous ordinary man’s virtue, which from the time of admission to the Order is devoid even of the stain of a [wrong] thought because of its extreme purity, like a gem of purest water, like well-refined gold, becomes the proximate cause for Arahantship itself, which is why it is called \emph{consisting of fulfilled purification}; like that of the lders Saṅgharakkhita the Great and Saṅgharakkhita the Nephew.

                    \vismParagraph{I.135}{135}{}
                    The Elder Saṅgharakkhita the Great (\emph{Mahā Saṅgharakkhita}), aged over sixty, was lying, it seems, on his deathbed. The Order of Bhikkhus questioned him about attainment of the supramundane state. The elder said: “I have no supramundane state.” Then the young bhikkhu who was attending on him said: “Venerable sir, people have come as much as twelve leagues, thinking that you have reached Nibbāna. It will be a disappointment for many if you die as an ordinary man.”—“Friend, thinking to see the Blessed One Metteyya, I did not try for insight. \textcolor{brown}{\textit{[48]}} So help me to sit up and give me the chance.” He helped the elder to sit up and went out. As he went out the elder reached Arahantship and he gave a sign by snapping his fingers. The Order assembled and said to him: “Venerable sir, you have done a difficult thing in achieving the supramundane state in the hour of death.”—“That was not difficult, friends. But rather I will tell you what is difficult. Friends, I see no action done [by me] without mindfulness and unknowingly since the time I went forth.” His nephew also reached Arahantship in the same way at the age of fifty years.

                    \vismParagraph{I.136}{136}{}
                    
                    \begin{verse}
                        “Now, if a man has little learning\\{}
                        And he is careless of his virtue,\\{}
                        They censure him on both accounts\\{}
                        For lack of virtue and of learning.
                    \end{verse}

                    \begin{verse}
                        “But if he is of little learning\\{}
                        Yet he is careful of his virtue,\\{}
                        They praise him for his virtue, so\\{}
                        It is as though he too had learning.
                    \end{verse}

                    \begin{verse}
                        “And if he is of ample learning\\{}
                        Yet he is careless of his virtue,\\{}
                        They blame him for his virtue, so\\{}
                        It is as though he had no learning.
                    \end{verse}

                    \begin{verse}
                        “But if he is of ample learning\\{}
                        And he is careful of his virtue,\\{}
                        They give him praise on both accounts\\{}
                        For virtue and as well for learning.
                    \end{verse}

                    \begin{verse}
                        “The Buddha’s pupil of much learning\\{}
                        Who keeps the Law with understanding—\\{}
                        A jewel of Jambu River gold\footnote{\vismAssertFootnoteCounter{37}\vismHypertarget{I.n37}{}A story of the Jambu River and its gold is given at \textbf{\cite{M-a}IV 147}.}\\{}
                        Who is here fit to censure him?
                    \end{verse}

                    \begin{verse}
                        \marginnote{\textcolor{teal}{\footnotesize\{104|46\}}}{}Deities praise him [constantly],\\{}
                        By Brahmā also is he praised (\textbf{\cite{A}II 7}).
                    \end{verse}


                    \vismParagraph{I.137}{137}{}
                    (d) What should be understood as \emph{virtue consisting in purification not adhered to }is trainers’ virtue, because it is not adhered to by [false] view, and ordinary men’s virtue when not adhered to by greed. Like the virtue of the Elder Tissa the Landowner’s Son (\emph{Kuṭumbiyaputta-Tissa-thera}). Wanting to become established in Arahantship in dependence on such virtue, this venerable one told his enemies:
                    \begin{verse}
                        I broke the bones of both my legs\\{}
                        To give the pledge you asked from me.\\{}
                        I am revolted and ashamed\\{}
                        At death accompanied by greed. \textcolor{brown}{\textit{[49]}}
                    \end{verse}

                    \begin{verse}
                        “And after I had thought on this,\\{}
                        And wisely then applied insight,\\{}
                        When the sun rose and shone on me,\\{}
                        I had become an Arahant” (\textbf{\cite{M-a}I 233}).
                    \end{verse}


                    \vismParagraph{I.138}{138}{}
                    Also there was a certain senior elder who was very ill and unable to eat with his own hand. He was writhing smeared with his own urine and excrement. Seeing him, a certain young bhikkhu said, “Oh, what a painful process life is!” The senior elder told him: “If I were to die now, friend, I should obtain the bliss of heaven; I have no doubt of that. But the bliss obtained by breaking this virtue would be like the lay state obtained by disavowing the training,” and he added: “I shall die together with my virtue.” As he lay there, he comprehended that same illness [with insight], and he reached Arahantship. Having done so, he pronounced these verses to the Order of Bhikkhus:
                    \begin{verse}
                        “I am victim of a sickening disease\\{}
                        That racks me with its burden of cruel pain;\\{}
                        As flowers in the dust burnt by the sun,\\{}
                        So this my corpse will soon have withered up.
                    \end{verse}

                    \begin{verse}
                        “Unbeautiful called beautiful,\\{}
                        Unclean while reckoned as if clean,\\{}
                        Though full of ordure seeming fair\\{}
                        To him that cannot see it clear.
                    \end{verse}

                    \begin{verse}
                        “So out upon this ailing rotting body,\\{}
                        Fetid and filthy, punished with affliction,\\{}
                        Doting on which this silly generation\\{}
                        Has lost the way to be reborn in heaven!” (\textbf{\cite{J-a}II 437})
                    \end{verse}


                    \vismParagraph{I.139}{139}{}
                    (e) It is the virtue of the Arahants, etc., that should be understood \emph{as tranquillized purification}, because of tranquillization of all disturbance and because of purifiedness.

                    So it is of five kinds as “consisting in limited purification,” and so on.

                    \vismParagraph{I.140}{140}{}
                    \emph{19. }In the second pentad the meaning should be understood as the abandoning, etc., of killing living things, etc.; for this is said in the Paṭisambhidā: “Five kinds of virtue: (1) In the case of killing living things, (a) abandoning is \marginnote{\textcolor{teal}{\footnotesize\{105|47\}}}{}virtue, (b) abstention is virtue, (c) volition is virtue, (d) restraint is virtue, (e) non-transgression is virtue. (2) In the case of taking what is not given … (3) In the case of sexual misconduct … (4) In the case of false speech … (5) In the case of malicious speech … (6) In the case of harsh speech … (7) In the case of gossip … \textcolor{brown}{\textit{[50]}} (8) In the case of covetousness … (9) In the case of ill will … (10) In the case of wrong view …

                    (11) “Through renunciation in the case of lust, (a) abandoning is virtue … (12) Through non-ill-will in the case of ill-will … (13) Through perception of light in the case of stiffness-and-torpor … (14) Through non-distraction … agitation … (15) Through definition of states (\emph{dhamma}) … uncertainty … (16) Through knowledge … ignorance … (17) Through gladdening in the case of boredom …

                    (18) “Through the first jhāna in the case of the hindrances, (a) abandoning is virtue … (19) Through the second jhāna … applied and sustained thought … (20) Through the third jhāna … happiness … (21) Through the fourth jhāna in the case of pleasure and pain, (a) abandoning is virtue … (22) Through the attainment of the base consisting of boundless space in the case of perceptions of matter, perceptions of resistance, and perceptions of variety, (a) abandoning is virtue … (23) Through the attainment of the base consisting of boundless consciousness in the case of the perception of the base consisting of boundless space … (24) Through the attainment of the base consisting of nothingness in the case of the perception of the base consisting of boundless consciousness … (25) Through the attainment of the base consisting of neither perception nor non-perception in the case of the perception of the base consisting of nothingness …

                    (26) “Through the contemplation of impermanence in the case of the perception of permanence, (a) abandoning is virtue … (27) Through the contemplation of pain in the case of the perception of pleasure … (28) Through the contemplation of not-self in the case of the perception of self … (29) Through the contemplation of dispassion in the case of the perception of delighting … (30) Through the contemplation of fading away in the case of greed … (31) Through the contemplation of cessation in the case of originating … (32) Through the contemplation of relinquishment in the case of grasping …

                    (33) “Through the contemplation of destruction in the case of the perception of compactness, (a) abandoning is virtue … (34) Through the contemplation of fall [of formations] in the case of accumulating [kamma] … (35) Through the contemplation of change in the case of the perception of lastingness … (36) Through the contemplation of the signless in the case of a sign … (37) Through the contemplation of the desireless in the case of desire … (38) Through the contemplation of voidness in the case of misinterpreting (insistence) … (39) Through insight into states that is higher understanding in the case of misinterpreting (insistence) due to grasping … (40) Through correct knowledge and vision in the case of misinterpreting (insistence) due to confusion … (41) Through the contemplation of danger in the case of misinterpreting (insistence) due to reliance [on formations] … (42) Through reflection in the case of non-reflection … (43) Through the contemplation of turning away in the case of misinterpreting (insistence) due to bondage … \marginnote{\textcolor{teal}{\footnotesize\{106|48\}}}{}(44) “Through the path of stream-entry in the case of defilements coefficient with [false] view, (a) abandoning is virtue … (45) Through the path of once-return in the case of gross defilements … (46) Through the path of non-return in the case of residual defilements … (47) Through the path of Arahantship in the case of all defilements, (a) abandoning is virtue, (b) abstention is virtue, (c) volition is virtue, (d) restraint is virtue, (e) non-transgression is virtue.

                    “Such virtues lead to non-remorse in the mind, to gladdening, to happiness, to tranquillity, to joy, to repetition, to development, to cultivation, to embellishment, to the requisite [for concentration], to the equipment [of concentration], to fulfilment, to complete dispassion, to fading away, to cessation, to peace, to direct-knowledge, to enlightenment, to Nibbāna.”\footnote{\vismAssertFootnoteCounter{38}\vismHypertarget{I.n38}{}This list describes, in terms of abandoning, etc., the stages in the normal progress from ignorance to Arahantship, and it falls into the following groups: I. \emph{Virtue}: the abandoning of the ten unprofitable courses of action (1–10). II. \emph{Concentration}: A. abandoning the seven hindrances to concentration by means of their opposites (11–17); B. The eight attainments of concentration, and what is abandoned by each (18–25). III. \emph{Understanding}: A. Insight: the eighteen principal insights beginning with the seven contemplations (26–43). B. Paths: The four paths and what is abandoned by each (44–47).} (\textbf{\cite{Paṭis}I 46–47})

                    \vismParagraph{I.141}{141}{}
                    And here there is no state called abandoning other than the mere non-arising of the killing of living things, etc., as stated. But the abandoning of a given [unprofitable state] upholds \textcolor{brown}{\textit{[51]}} a given profitable state in the sense of providing a foundation for it, and concentrates it by preventing wavering, so it is called “virtue” (\emph{sīla}) in the sense of composing (\emph{sīlana}), reckoned as upholding and concentrating as stated earlier (\hyperlink{I.19}{§19}{}).

                    The other four things mentioned refer to the presence\footnote{\vismAssertFootnoteCounter{39}\vismHypertarget{I.n39}{}\emph{Sabbhāva—}“presence” ( = \emph{sat + bhāva}): not in PED. Not to be confused with \emph{sabhāva—}“individual essence” ( = \emph{sa} (Skr. \emph{sva}) + \emph{bhāva}, or \emph{saha + bhāva}).} of occurrence of will as abstention from such and such, as restraint of such and such, as the volition associated with both of these, and as non-transgression in one who does not transgress such and such. But their meaning of virtue has been explained already.

                    So it is of five kinds as “virtue consisting in abandoning” and so on.

                    \vismParagraph{I.142}{142}{}
                    At this point the answers to the questions, “What is virtue? In what sense is it virtue? What are its characteristic, function, manifestation, and proximate cause? What are the benefits of virtue? How many kinds of virtue are there?” are complete.
            \section[\vismAlignedParas{§143–161}(vi), (vii) What are the defiling and the cleansing of it?]{(vi), (vii) What are the defiling and the cleansing of it?}

                \vismParagraph{I.143}{143}{}
                However, it was also asked (vi) \textsc{What is the defiling of it?} and (vii) \textsc{What is the cleansing of it?}

                We answer that virtue’s tornness, etc., is its defiling, and that its untornness, etc., is its cleansing. Now, that tornness, etc., are comprised under the breach that has gain, fame, etc., as its cause, and under the seven bonds of sexuality. When a man has broken the training course at the beginning or at the end in any instance of the seven classes of offences,\footnote{\vismAssertFootnoteCounter{40}\vismHypertarget{I.n40}{}The seven consisting of \emph{pārājikā, saṅghādisesā, pācittiyā, pāṭidesanīyā, dukkaṭā, thullaccayā, dubbhāsitā} (mentioned at \textbf{\cite{M-a}II 33}).} his virtue is called torn, like a cloth that is cut at the edge. But when he has broken it in the middle, it is called rent, like a cloth that \marginnote{\textcolor{teal}{\footnotesize\{107|49\}}}{}is rent in the middle. When he has broken it twice or thrice in succession, it is called blotched, like a cow whose body is some such colour as black or red with a discrepant colour appearing on the back or the belly. When he has broken it [all over] at intervals, it is called mottled, like a cow speckled [all over] with discrepant-coloured spots at intervals. This in the first place, is how there comes to be tornness with the breach that has gain, etc., as its cause.

                \vismParagraph{I.144}{144}{}
                And likewise with the seven bonds of sexuality; for this is said by the Blessed One: “Here, brahman, some ascetic or brahman claims to lead the life of purity rightly; for he does not \textcolor{brown}{\textit{[52]}} enter into actual sexual intercourse with women. Yet he agrees to massage, manipulation, bathing and rubbing down by women. He enjoys it, desires it and takes satisfaction in it. This is what is torn, rent, blotched and mottled in one who leads the life of purity. This man is said to lead a life of purity that is unclean. As one who is bound by the bond of sexuality, he will not be released from birth, ageing and death … he will not be released from suffering, I say.

                \vismParagraph{I.145}{145}{}
                “Furthermore, brahman, … while he does not agree to [these things], yet he jokes, plays and amuses himself with women …

                \vismParagraph{I.146}{146}{}
                “Furthermore, brahman, … while he does not agree to [these things], yet he gazes and stares at women eye to eye …

                \vismParagraph{I.147}{147}{}
                “Furthermore, brahman, … while he does not agree to [these things], yet he listens to the sound of women through a wall or through a fence as they laugh or talk or sing or weep …

                \vismParagraph{I.148}{148}{}
                “Furthermore, brahman, … while he does not agree to [these things], yet he recalls laughs and talks and games that he formerly had with women …

                \vismParagraph{I.149}{149}{}
                “Furthermore, brahman, … while he does not agree to [these things], \textcolor{brown}{\textit{[53]}} yet he sees a householder or a householder’s son possessed of, endowed with, and indulging in, the five cords of sense desire …

                \vismParagraph{I.150}{150}{}
                “Furthermore, brahman, while he does not agree to [these things], yet he leads the life of purity aspiring to some order of deities, [thinking] ‘Through this rite (virtue) or this ritual (vow) or this asceticism I shall become a [great] deity or some [lesser] deity.’ He enjoys it, desires it, and takes satisfaction in it. This, brahman, is what is torn, rent, blotched and mottled in one who leads the life of purity. This man … will not be released from suffering, I say” (\textbf{\cite{A}IV 54–56}).

                This is how tornness, etc., should be understood as included under the breach that has gain, etc., as its cause and under the seven bonds of sexuality.

                \vismParagraph{I.151}{151}{}
                Untornness, however, is accomplished by the complete non-breaking of the training precepts, by making amends for those broken for which amends should be made, by the absence of the seven bonds of sexuality, and, as well, by the non-arising of such evil things as anger, enmity, contempt, domineering, envy, avarice, deceit, fraud, obduracy, presumption, pride (conceit), haughtiness, conceit (vanity), and negligence (MN 7), and by the arising of such qualities as fewness of wishes, contentment, and effacement (MN 24).

                \vismParagraph{I.152}{152}{}
                \marginnote{\textcolor{teal}{\footnotesize\{108|50\}}}{}Virtues not broken for the purpose of gain, etc., and rectified by making amends after being broken by the faults of negligence, etc., and not damaged by the bonds of sexuality and by such evil things as anger and enmity, are called entirely untorn, unrent, unblotched, and unmottled. And those same virtues are \emph{liberating }since they bring about the state of a freeman, and praised by the wise since it is by the wise that they are praised, and \emph{unadhered-to }since they are not adhered to by means of craving and views, and \emph{conducive to concentration }since they conduce to access concentration or to absorption concentration. That is why their untornness, etc., should be understood as “cleansing” (see also \hyperlink{VII.101}{VII.101f.}{}).

                \vismParagraph{I.153}{153}{}
                This cleansing comes about in two ways: through seeing the danger of failure in virtue, and through seeing the benefit of perfected virtue. \textcolor{brown}{\textit{[54]}} Herein, the danger of failure in virtue can be seen in accordance with such suttas as that beginning, “Bhikkhus, there are these five dangers for the unvirtuous in the failure of virtue” (\textbf{\cite{A}III 252}).

                \vismParagraph{I.154}{154}{}
                Furthermore, on account of his unvirtuousness an unvirtuous person is displeasing to deities and human beings, is uninstructable by his fellows in the life of purity, suffers when unvirtuousness is censured, and is remorseful when the virtuous are praised. Owing to that unvirtuousness he is as ugly as hemp cloth. Contact with him is painful because those who fall in with his views are brought to long-lasting suffering in the states of loss. He is worthless because he causes no great fruit [to accrue] to those who give him gifts. He is as hard to purify as a cesspit many years old. He is like a log from a pyre (see It 99); for he is outside both [recluseship and the lay state]. Though claiming the bhikkhu state he is no bhikkhu, so he is like a donkey following a herd of cattle. He is always nervous, like a man who is everyone’s enemy. He is as unfit to live with as a dead carcase. Though he may have the qualities of learning, etc., he is as unfit for the homage of his fellows in the life of purity as a charnel-ground fire is for that of brahmans. He is as incapable of reaching the distinction of attainment as a blind man is of seeing a visible object. He is as careless of the Good Law as a guttersnipe is of a kingdom. Though he fancies he is happy, yet he suffers because he reaps suffering as told in the Discourse on the Mass of Fire (\textbf{\cite{A}IV 128–134}).

                \vismParagraph{I.155}{155}{}
                Now, the Blessed One has shown that when the unvirtuous have their minds captured by pleasure and satisfaction in the indulgence of the five cords of sense desires, in [receiving] salutation, in being honoured, etc., the result of that kamma, directly visible in all ways, is very violent pain, with that [kamma] as its condition, capable of producing a gush of hot blood by causing agony of heart with the mere recollection of it. Here is the text:

                “Bhikkhus, do you see that great mass of fire burning, blazing and glowing?—Yes, venerable sir.—What do you think, bhikkhus, which is better, that one [gone forth] should sit down or lie down embracing that mass of fire burning, blazing and glowing, or that he should sit down or lie down embracing a warrior-noble maiden or a brahman maiden or a maiden of householder family, with soft, delicate hands and feet?—It would be better, venerable sir, that he should sit down or lie down embracing a warrior-noble maiden … \textcolor{brown}{\textit{[55]}} It would be painful, venerable sir, \marginnote{\textcolor{teal}{\footnotesize\{109|51\}}}{}if he sat down or lay down embracing that great mass of fire burning, blazing and glowing.

                \vismParagraph{I.156}{156}{}
                “I say to you, bhikkhus, I declare to you, bhikkhus, that it would be better for one [gone forth] who is unvirtuous, who is evil-natured, of unclean and suspect habits, secretive of his acts, who is not an ascetic and claims to be one, who does not lead the life of purity and claims to do so, who is rotten within, lecherous, and full of corruption, to sit down or lie down embracing that great mass of fire burning, blazing and glowing. Why is that? By his doing so, bhikkhus, he might come to death or deadly suffering, yet he would not on that account, on the breakup of the body, after death, reappear in states of loss, in an unhappy destiny, in perdition, in hell. But if one who is unvirtuous, evil-natured … and full of corruption, should sit down or lie down embracing a warrior-noble maiden … that would be long for his harm and suffering: on the break-up of the body, after death, he would reappear in states of loss, in an unhappy destiny, in perdition, in hell” (\textbf{\cite{A}IV 128–129}).

                \vismParagraph{I.157}{157}{}
                Having thus shown by means of the analogy of the mass of fire the suffering that is bound up with women and has as its condition the indulgence of the five cords of sense desires [by the unvirtuous], to the same intent he showed, by the following similes of the horse-hair rope, the sharp spear, the iron sheet, the iron ball, the iron bed, the iron chair, and the iron cauldron, the pain that has as its condition [acceptance of] homage and reverential salutation, and the use of robes, alms food, bed and chair, and dwelling [by unvirtuous bhikkhus]:

                “What do you think, bhikkhus, which is better, that one should have a strong horse-hair rope twisted round both legs by a strong man and tightened so that it cut through the outer skin, and having cut through the outer skin it cut through the inner skin, and having cut through the inner skin it cut through the flesh, and having cut through the flesh it cut through the sinews, and having cut through the sinews it cut through the bones, and having cut through the bones it remained crushing the bone marrow—or that he should consent to the homage of great warrior-nobles, great brahmans, great householders?” (\textbf{\cite{A}IV 129}). \textcolor{brown}{\textit{[56]}}

                And: “What do you think, bhikkhus, which is better, that one should have a strong man wound one’s breast with a sharp spear tempered in oil—or that he should consent to the reverential salutation of great warrior-nobles, great brahmans, great householders?” (\textbf{\cite{A}IV 130}).

                And: “What do you think, bhikkhus, which is better, that one’s body should be wrapped by a strong man in a red-hot iron sheet burning, blazing and glowing—or that he should use robes given out of faith by great warrior-nobles, great brahmans, great householders?” (\textbf{\cite{A}IV 130–131}).

                And: “What do you think, bhikkhus, which is better, that one’s mouth should be prised open by a strong man with red-hot iron tongs burning, blazing and glowing, and that into his mouth should be put a red-hot iron ball burning, blazing and glowing, which burns his lips and burns his mouth and tongue and throat and belly and passes out below carrying with it his bowels and entrails—or that he should use alms food given out of faith by great warrior-nobles …?” (\textbf{\cite{A}IV 131–132}). \marginnote{\textcolor{teal}{\footnotesize\{110|52\}}}{}And: “What do you think, bhikkhus, which is better, that one should have a strong man seize him by the head or seize him by the shoulders and seat him or lay him on a red-hot iron bed or iron chair, burning, blazing and glowing—or that he should use a bed or chair given out of faith by great warrior-nobles … ?” (\textbf{\cite{A}IV 132–133}).

                And: “What do you think, bhikkhus, which is better, that one should have a strong man take him feet up and head down and plunge him into a red-hot metal cauldron burning, blazing and glowing, to be boiled there in a swirl of froth, and as he boils in the swirl of froth to be swept now up, now down, and now across—or that he should use a dwelling given out of faith by great warrior-nobles … ?” (\textbf{\cite{A}IV 133–134}).

                \vismParagraph{I.158}{158}{}
                
                \begin{verse}
                    What pleasure has a man of broken virtue\\{}
                    Forsaking not sense pleasures, which bear fruit\\{}
                    Of pain more violent even than the pain\\{}
                    In the embracing of a mass of fire?
                \end{verse}

                \begin{verse}
                    What pleasure has he in accepting homage\\{}
                    Who, having failed in virtue, must partake\\{}
                    Of pain that will excel in agony\\{}
                    The crushing of his legs with horse-hair ropes? \textcolor{brown}{\textit{[57]}}
                \end{verse}

                \begin{verse}
                    What pleasure has a man devoid of virtue\\{}
                    Accepting salutations of the faithful,\\{}
                    Which is the cause of pain acuter still\\{}
                    Than pain produced by stabbing with a spear?
                \end{verse}

                \begin{verse}
                    What is the pleasure in the use of garments\\{}
                    For one without restraint, whereby in hell\\{}
                    He will for long be forced to undergo\\{}
                    The contact of the blazing iron sheet?
                \end{verse}

                \begin{verse}
                    Although to him his alms food may seem tasty,\\{}
                    Who has no virtue, it is direst poison,\\{}
                    Because of which he surely will be made\\{}
                    For long to swallow burning iron balls.
                \end{verse}

                \begin{verse}
                    And when the virtueless make use of couches\\{}
                    And chairs, though reckoned pleasing, it is pain\\{}
                    Because they will be tortured long indeed\\{}
                    On red-hot blazing iron beds and chairs.
                \end{verse}

                \begin{verse}
                    Then what delight is there for one unvirtuous\\{}
                    Inhabiting a dwelling given in faith,\\{}
                    Since for that reason he will have to dwell\\{}
                    Shut up inside a blazing iron pan?
                \end{verse}

                \begin{verse}
                    The Teacher of the world, in him condemning,\\{}
                    Described him in these terms: “Of suspect habits,\\{}
                    Full of corruption, lecherous as well,\\{}
                    By nature evil, rotten too within.”
                \end{verse}

                \begin{verse}
                    So out upon the life of him abiding\\{}
                    Without restraint, of him that wears the guise\\{}
                    \marginnote{\textcolor{teal}{\footnotesize\{111|53\}}}{}Of the ascetic that he will not be,\\{}
                    And damages and undermines himself!
                \end{verse}

                \begin{verse}
                    What is the life he leads, since any person,\\{}
                    No matter who, with virtue to his credit\\{}
                    Avoids it here, as those that would look well\\{}
                    Keep far away from dung or from a corpse?
                \end{verse}

                \begin{verse}
                    He is not free from any sort of terror,\\{}
                    Though free enough from pleasure of attainment;\\{}
                    While heaven’s door is bolted fast against him,\\{}
                    He is well set upon the road to hell.
                \end{verse}

                \begin{verse}
                    Who else if not one destitute of virtue\\{}
                    More fit to be the object of compassion?\\{}
                    Many indeed and grave are the defects\\{}
                    That brand a man neglectful of his virtue.
                \end{verse}


                Seeing danger in the failure of virtue should be understood as reviewing in such ways as these. And seeing benefits in perfected vir-tue should be understood in the opposite sense.

                \vismParagraph{I.159}{159}{}
                Furthermore: \textcolor{brown}{\textit{[58]}}
                \begin{verse}
                    His virtue is immaculate,\\{}
                    His wearing of the bowl and robes\\{}
                    Gives pleasure and inspires trust,\\{}
                    His going forth will bear its fruit.
                \end{verse}

                \begin{verse}
                    A bhikkhu in his virtue pure\\{}
                    Has never fear that self-reproach\\{}
                    Will enter in his heart: indeed\\{}
                    There is no darkness in the sun.
                \end{verse}

                \begin{verse}
                    A bhikkhu in his virtue bright\\{}
                    Shines forth in the Ascetics’ Wood\footnote{\vismAssertFootnoteCounter{41}\vismHypertarget{I.n41}{}An allusion to the Gosiṅga Suttas (MN 31, 32).}\\{}
                    As by the brightness of his beams\\{}
                    The moon lights up the firmament.
                \end{verse}

                \begin{verse}
                    Now, if the bodily perfume\\{}
                    Of virtuous bhikkhus can succeed\\{}
                    In pleasing even deities,\\{}
                    What of the perfume of his virtue?
                \end{verse}

                \begin{verse}
                    It is more perfect far than all\\{}
                    The other perfumes in the world,\\{}
                    Because the perfume virtue gives\\{}
                    Is borne unchecked in all directions.
                \end{verse}

                \begin{verse}
                    The deeds done for a virtuous man,\\{}
                    Though they be few, will bear much fruit,\\{}
                    \marginnote{\textcolor{teal}{\footnotesize\{112|54\}}}{}And so the virtuous man becomes\\{}
                    A vessel of honour and renown.
                \end{verse}

                \begin{verse}
                    There are no cankers here and now\\{}
                    To plague the virtuous man at all;\\{}
                    The virtuous man digs out the root\\{}
                    Of suffering in lives to come.
                \end{verse}

                \begin{verse}
                    Perfection among human kind\\{}
                    And even among deities.\\{}
                    If wished for, is not hard to gain\\{}
                    For him whose virtue is perfected;
                \end{verse}

                \begin{verse}
                    But once his virtue is perfected,\\{}
                    His mind then seeks no other kind\\{}
                    han the perfection of Nibbāna,\\{}
                    The state where utter peace prevails.
                \end{verse}

                \begin{verse}
                    Such is the blessed fruit of virtue,\\{}
                    Showing full many a varied form,\\{}
                    So let a wise man know it well\\{}
                    This root of all perfection’s branches.
                \end{verse}


                \vismParagraph{I.160}{160}{}
                The mind of one who understands thus, shudders at failure in virtue and reaches out towards the perfecting of virtue. So virtue should be cleansed with all care, seeing this danger of failure in virtue and this benefit of the perfection of virtue in the way stated.

                \vismParagraph{I.161}{161}{}
                And at this point in the \emph{Path of Purification}, which is shown under the headings of virtue, concentration and understanding by the stanza, “When a wise man, established well in virtue” (\hyperlink{I.1}{§1}{}), virtue, firstly, has been fully illustrated.

                The first chapter called “The Description of Virtue” in the \emph{Path of Purification }composed for the purpose of gladdening good people.
        \chapter[The Ascetic Practices]{The Ascetic Practices\vismHypertarget{II}\newline{\textnormal{\emph{Dhutaṅga-niddesa}}}}
            \label{II}


            \vismParagraph{II.1}{1}{}
            \marginnote{\textcolor{teal}{\footnotesize\{113|55\}}}{}\textcolor{brown}{\textit{[59]}} Now, while a meditator is engaged in the pursuit of virtue, he should set about undertaking the ascetic practices in order to perfect those special qualities of fewness of wishes, contentment, etc., by which the virtue of the kind already described, is cleansed. For when his virtue is thus washed clean of stains by the waters of such special qualities as fewness of wishes, contentment, effacement, seclusion, dispersal, energy, and modest needs, it will become quite purified; and his vows will succeed as well. And–so, when his whole behaviour has been purified by the special quality of blameless virtue and vows and he has become established in the [first] three of the ancient Noble Ones’ heritages, he may become worthy to attain to the fourth called “delight in development” (\textbf{\cite{A}II 27}). We shall therefore begin the explanation of the ascetic practices.
            \section[\vismAlignedParas{§2–3}The 13 kinds of Ascetic Practices]{The 13 kinds of Ascetic Practices}

                \vismParagraph{II.2}{2}{}
                Thirteen kinds of ascetic practices have been allowed by the Blessed One to clansmen who have given up the things of the flesh and, regardless of body and life, are desirous of undertaking a practice in conformity [with their aim]. They are:

                
                    \begin{enumerate}[i.,nosep]
                        \item the refuse-rag-wearer’s practice,
                        \item the triple-robe-wearer’s practice,
                        \item the alms-food-eater’s practice,
                        \item the house-to-house-seeker’s practice,
                        \item the one-sessioner’s practice,
                        \item the bowl-food-eater’s practice,
                        \item the later-food-refuser’s practice,
                        \item the forest-dweller’s practice,
                        \item the tree-root-dweller’s practice,
                        \item the open-air-dweller’s practice,
                        \item the charnel-ground-dweller’s practice,
                        \item the any-bed-user’s practice,
                        \item the sitter’s practice.
                    \end{enumerate}

                \vismParagraph{II.3}{3}{}
                Herein:
                \begin{verse}
                    (1) As to meaning, (2) characteristic, et cetera,\\{}
                    (3) The undertaking and directions,\\{}
                    \marginnote{\textcolor{teal}{\footnotesize\{114|56\}}}{}And then the grade, and breach as well,\\{}
                    And benefits of each besides,\\{}
                    (4) As to the profitable triad,\\{}
                    (5) “Ascetic” and so on distinguished,\\{}
                    (6) And as to groups, and also (7) singly,\\{}
                    The exposition should be known.\textcolor{brown}{\textit{[60]}}
                \end{verse}

            \section[\vismAlignedParas{§4–11}Meaning]{Meaning}

                \vismParagraph{II.4}{4}{}
                1. Herein, \emph{as to meaning}, in the first place.

                i. It is “refuse” (\emph{paṃsukūla}) since, owing to its being found on refuse in any such place as a street, a charnel ground, or a midden, it belongs, as it were, to the refuse in the sense of being dumped in anyone of these places. Or alternatively: like refuse it gets to a vile state (\emph{PAṂSU viya KUcchitabhāvaṃ ULAti}), thus it is “refuse” (\emph{paṃsukūla}); it goes to a vile state, is what is meant. The wearing of a refuse-[rag], which has acquired its derivative name\footnote{\vismAssertFootnoteCounter{1}\vismHypertarget{II.n1}{}\emph{Nibbacana—”}derivative name (or verbal derivative)”; gram. term not in PED; \textbf{\cite{M-a}I 61},105; Vism \hyperlink{XVI.16}{XVI.16}{}.} in this way, is “refuse-[rag-wearing]” (\emph{paṃsukūla}). That is his habit, thus he is a “refuse-[rag-wear-]er” (\emph{paṃsukūlika}). The practice (\emph{aṅga}) of the refuse-[rag-wear-]er is the “refuse-[rag-wear-]er’s practice” (\emph{paṃsukūlikaṅga}). It is the action that is called the “practice.” Therefore it should be understood as a term for that by undertaking which one becomes a refuse-[rag-wear-]er.

                ii. In the same way, he has the habit of [wearing] the triple robe (\emph{ti-cīvara})—in other words, the cloak of patches, the upper garment, and the inner clothing—thus he is a “triple-robe-[wear-]er” (\emph{tecīvarika}). His practice is called the “triple-robe-wearer’s practice.”

                \vismParagraph{II.5}{5}{}
                iii. The dropping (\emph{pāta}) of the lumps (\emph{piṇḍa}) of material sustenance (\emph{āmisa}) called alms (\emph{bhikkhā}) is “alms food” (\emph{piṇḍapāta}); the falling (\emph{nipatana}) into the bowl of lumps (\emph{piṇḍa}) given by others, is what is meant. He gleans that alms food (that falling of lumps), he seeks it by approaching such and such a family, thus he is called an “alms-food [eat-]er” (\emph{piṇḍapātika}). Or his vow is to gather (\emph{patituṃ})\footnote{\vismAssertFootnoteCounter{2}\vismHypertarget{II.n2}{}\emph{Patati—”}to gather (or to wander)”: not in PED.} the lump (\emph{piṇḍa}), thus he is a “lump-gatherer” (\emph{piṇḍapātin}). To “gather” is to wander for. A “lump-gatherer” (\emph{piṇḍapātin}) is the same as an “alms-food-eater” (\emph{piṇḍapātika}). The practice of the alms-food-eater is the “alms-food-eater’s practice.”

                \vismParagraph{II.6}{6}{}
                iv. It is a hiatus (\emph{avakhaṇḍana}) that is called a “gap” (\emph{dāna}).\footnote{\vismAssertFootnoteCounter{3}\vismHypertarget{II.n3}{}\emph{Avakhaṇḍana—}”hiatus” and \emph{dāna—”}gap”: not in PED.} It is removed (\emph{apeta}) from a gap, thus it is called “gapless” (\emph{apadāna}); the meaning is, it is without hiatus. It is together with (\emph{saha}) what is gapless (\emph{apadāna}), thus it is “with the gapless” (\emph{sapadāna}); devoid of hiatus—from house to house—is what is meant. His habit is to wander on what-is-with-the-gapless, thus he is a “gapless wanderer” (\emph{sapadāna-cārin}). A gapless wanderer is the same as a “house-to-house-seeker” (\emph{sapadāna-cārika}). His practice is the “house-to-house-seeker’s practice.”

                \vismParagraph{II.7}{7}{}
                v. Eating in one session is “one-session.” He has that habit, thus he is a “one-sessioner.” His practice is the “one-sessioner’s practice.” \marginnote{\textcolor{teal}{\footnotesize\{115|57\}}}{}vi. Alms (\emph{piṇḍa}) in one bowl (\emph{patta}) only because of refusing a second vessel, is “bowl-alms” (\emph{patta-piṇḍa}). Now, making “bowl alms” (\emph{patta-piṇḍa}) the name for the taking of alms food in the bowl: bowl-alms-food is his habit, thus he is a “bowl-food-eater” (\emph{pattapiṇḍika}). His practice is the “bowl-food-eater’s practice.”

                \vismParagraph{II.8}{8}{}
                vii. “No” (\emph{khalu}) is a particle in the sense of refusing. \textcolor{brown}{\textit{[61]}} Food (\emph{bhatta}) obtained later by one who has shown that he is satisfied is called “later-food” (\emph{pacchā-bhatta}). The eating of that later food is “later-food-eating.” Making “later-food” (\emph{pacchā-bhatta}) the name for that later-food-eating: later-food is his habit, thus he is a “later-food-[eat-]er” (\emph{pacchābhattika}). Not a later-food-eater is a “no-later-food-[eat-]er” (\emph{khalu-pacchābhattika}), [that is, a “later-food-refuser”]. This is the name for one who as an undertaking refuses extra food. But it is said in the commentary\footnote{\vismAssertFootnoteCounter{4}\vismHypertarget{II.n4}{}Such references to “the Commentary” are to the old Sinhalese commentary, no longer extant, from which Bhadantācariya Buddhaghosa drew his material.} “Khalu is a certain kind of bird. When it has taken a fruit into its beak and that drops, it does not eat any more. This [bhikkhu] is like that.” Thus he is “a later-food-refuser” (\emph{khalu-pacchā-bhattika}). His practice is the “later-food-refuser’s practice.”

                \vismParagraph{II.9}{9}{}
                viii. His habit is dwelling in the forest, thus he is a “forest-dweller.” His practice is the “forest-dweller’s practice.”

                ix. Dwelling at the root of a tree is “tree-root-dwelling.” He has that habit, thus he is a “tree-root-dweller.” The practice of the tree-root-dweller is the “tree-root-dweller’s practice.” x., xi. Likewise with the open-air-dweller and the charnel-ground-dweller.

                \vismParagraph{II.10}{10}{}
                xii. Only what has been distributed (\emph{yad eva santhata}) is “as distributed” (\emph{yathāsanthata}). This is a term for the resting place first allotted thus “This one falls to you.” He has the habit of dwelling in that as distributed, thus he is an “as-distributed-user” (\emph{yathāsanthatika}), [that is, an “any-bed-user”]. His practice is the “any-bed-user’s practice.”

                xiii. He has the habit of keeping to the sitting [posture when resting], refusing to lie down, thus he is a “sitter.” His practice is the “sitter’s practice.”

                \vismParagraph{II.11}{11}{}
                All these, however, are the practices (\emph{aṅga}) of a bhikkhu who is ascetic (\emph{dhuta}) because he has shaken off (\emph{dhuta}) defilement by undertaking one or other of them. Or the knowledge that has got the name “ascetic” (\emph{dhuta}) because it shakes off (\emph{dhunana}) defilement is a practice (\emph{aṅga}) belonging to these, thus they are “ascetic practices” (\emph{dhutaṅga}). Or alternatively, they are ascetic (\emph{dhuta}) because they shake off (\emph{niddhunana}) opposition, and they are practices (\emph{aṅga}) because they are a way (\emph{paṭipatti}).

                This, firstly, is how the exposition should be known here as to meaning.
            \section[\vismAlignedParas{§12–13}Characteristic]{Characteristic}

                \vismParagraph{II.12}{12}{}
                \emph{2. }All of them have as their characteristic the volition of undertaking. For this is said [in the commentary]: “He who does the undertaking is a person. That whereby he does the undertaking is states of consciousness and consciousness-concomitants. The volition of the act of undertaking is the ascetic practice. What it rejects is the instance.” All have the function of eliminating cupidity, and they \marginnote{\textcolor{teal}{\footnotesize\{116|58\}}}{}manifest themselves with the production of non-cupidity. For their proximate cause they have the noble states consisting of fewness of wishes, and so on. \textcolor{brown}{\textit{[62]}} This is how the exposition should be known \emph{as to characteristic, etc}., here.

                \vismParagraph{II.13}{13}{}
                \emph{3. }As regards the five beginning with \emph{the undertaking }and\emph{ directions}: during the Blessed One’s lifetime all ascetic practices should be undertaken in the Blessed One’s presence. After his attainment of Nibbāna this should be done in the presence of a principal disciple. When he is not available it should be done in the presence of one whose cankers are destroyed, of a non-returner, of a once-returner, of a stream-enterer, of one who knows the three Piṭakas, of one who knows two of the Piṭakas, of one who knows one of the Piṭakas, of one who knows one Collection,\footnote{\vismAssertFootnoteCounter{5}\vismHypertarget{II.n5}{}“\emph{‘Ekasaṅgītika’:} one who knows one of the five collections \emph{(nikāya) }beginning with the Collection of Long Discourses (Dīgha Nikāya). (\textbf{\cite{Vism-mhṭ}76})”} of a teacher of the Commentaries. When he is not available it should be done in the presence of an observer of an ascetic practice. When he is not available, then after one has swept out the shrine terrace they can be undertaken seated in a reverential posture as though pronouncing them in the Fully Enlightened One’s presence. Also it is permitted to undertake them by oneself.

                And here should be told the story of the senior of the two brothers who were elders at Cetiyapabbata and their fewness of wishes with respect to the ascetic practices\footnote{\vismAssertFootnoteCounter{6}\vismHypertarget{II.n6}{}“That elder, it seems, was a sitter, but no one knew it. Then one night the other saw him by the light of a flash of lightning sitting up on his bed. He asked, ‘Are you a sitter, venerable sir?’ Out of fewness of wishes that his ascetic practice should get known, the elder lay down. Afterwards he undertook the practice anew. So the story has come down. (\textbf{\cite{Vism-mhṭ}77})”} (\textbf{\cite{M-a}II 140}).

                This, firstly, is what applies to all [the practices].
            \section[\vismAlignedParas{§14–77}Undertaking, directions, etc.]{Undertaking, directions, etc.}

                \vismParagraph{II.14}{14}{}
                Now, we shall proceed to comment on the undertaking, directions, grade, breach and benefits, of each one [separately].
                \subsection[\vismAlignedParas{§14–22}Refuse-rag-wearer]{Refuse-rag-wearer}

                    i. First, the \emph{refuse-rag-wearer’s practice} is undertaken with one of these two statements: “I refuse robes given by householders” or “I undertake the refuse-rag-wearer’s practice.” This, firstly, is the \emph{undertaking}.

                    \vismParagraph{II.15}{15}{}
                    One who has done this should get a robe of one of the following kinds: one from a charnel ground, one from a shop, a cloth from a street, a cloth from a midden, one from a childbed, an ablution cloth, a cloth from a washing place, one worn going to and returning from [the charnel ground], one scorched by fire, one gnawed by cattle, one gnawed by ants, one gnawed by rats, one cut at the end, one cut at the edge, one carried as a flag, a robe from a shrine, an ascetic’s robe, one from a consecration, one produced by supernormal power, one from a highway, one borne by the wind, one presented by deities, one from the sea. Taking one of these robe cloths, he should tear off and throw away the weak parts, and then wash the sound parts and make up a robe. He can use it after getting rid of his old robe given by householders.

                    \vismParagraph{II.16}{16}{}
                    Herein, “\emph{one from a charnel ground}” is one dropped on a charnel ground.

                    \marginnote{\textcolor{teal}{\footnotesize\{117|59\}}}{}“\emph{One from a shop}” is one dropped at the door of a shop.

                    “\emph{A cloth from a street}” is a cloth thrown into a street from inside a window by those who seek merit.

                    “\emph{A cloth from a midden}” \textcolor{brown}{\textit{[63]}} is a cloth thrown onto a place for rubbish.

                    “\emph{One from a childbed}” is a cloth thrown away after wiping up the stains of childbirth with it. The mother of Tissa the Minister, it seems, had the stains of childbirth wiped up with a cloth worth a hundred [pieces], and thinking, “The refuse-rag wearers will take it,” she had it thrown onto the Tālaveli Road.\footnote{\vismAssertFootnoteCounter{7}\vismHypertarget{II.n7}{}“The name of a street in Mahāgāma (S.E. Sri Lanka). Also in Anurādhapura, they say” (\textbf{\cite{Vism-mhṭ}77}).} Bhikkhus took it for the purpose of mending worn places.

                    \vismParagraph{II.17}{17}{}
                    “\emph{An ablution cloth}” is one that people who are made by devil doctors to bathe themselves, including their heads, are accustomed to throw away as a “cloth of ill luck.”

                    “\emph{A cloth from washing place}” is rags thrown away at a washing place where bathing is done.

                    “\emph{One worn going to and coming from}” is one that people throw away after they have gone to a charnel ground and returned and bathed.

                    “\emph{One scorched by fire}” is one partly scorched by fire; for people throw that away.

                    “\emph{One gnawed by cattle,}” etc., are obvious; for people throw away such as these too.

                    “\emph{One carried as a flag}”: Those who board a ship do so after hoisting a flag. It is allowable to take this when they have gone out of sight. Also it is allowable, when the two armies have gone away, to take a flag that has been hoisted on a battlefield.

                    \vismParagraph{II.18}{18}{}
                    “\emph{A robe from a shrine}” is an offering made by draping a termite-mound [in cloth].

                    “\emph{An ascetic’s robe}” is one belonging to a bhikkhu.

                    “\emph{One from a consecration}” is one thrown away at the king’s consecration place. “\emph{One produced by supernormal power}” is a “come-bhikkhu” robe.\footnote{\vismAssertFootnoteCounter{8}\vismHypertarget{II.n8}{}On certain occasions, when the going forth was given by the Buddha with only the words, \emph{“Ehi bhikkhu }(Come, bhikkhu),” owing to the disciple’s past merit robes appeared miraculously upon him (see e.g. \textbf{\cite{Vin}} Mahāvagga, Kh. 1).} “\emph{One from a highway}” is one dropped in the middle of a road. But one dropped by the owner’s negligence should be taken only after waiting a while.

                    “\emph{One borne by the wind}” is one that falls a long way off, having been carried by the wind. It is allowable to take it if the owners are not in sight.

                    “\emph{One presented by deities}” is one given by deities like that given to the Elder Anuruddha (\textbf{\cite{Dhp-a}II 173–174}).

                    “\emph{One from the sea}” is one washed up on dry land by the sea waves.

                    \vismParagraph{II.19}{19}{}
                    One given thus “We give it to the Order” or got by those who go out for alms-cloth is not a refuse-rag. And in the case of one presented by a bhikkhu, one given \marginnote{\textcolor{teal}{\footnotesize\{118|60\}}}{}after it has been got [at a presentation of robes by householders] at the end of the Rains, or a “resting-place robe” [that is, one automatically supplied by a householder to the occupant of a certain resting place] is not a refuse-rag. It is a refuse-rag only when given after not having been so obtained. And herein, that placed by the donors at a bhikkhu’s feet but given by that bhikkhu to the refuse-rag wearer by placing it in his hand is called pure in one way. That given to a bhikkhu by placing it in his hand but placed by him at the [refuse-rag wearer’s] feet is also pure in one way. That which is both placed at a bhikkhu’s feet and then given by him in the same way is pure in both ways. \textcolor{brown}{\textit{[64]}} One obtained by being placed in the hand and [given by being] placed in the hand too is not a strict man’s robe. So a refuse-rag wearer should use the robe after getting to know about the kinds of refuse-rags. These are the \emph{directions} for it in this instance.

                    \vismParagraph{II.20}{20}{}
                    The \emph{grades} are these. There are three kinds of refuse-rag wearers: the strict, the medium, and the mild. Herein, one who takes it only from a charnel ground is strict. One who takes one left [by someone, thinking] “One gone forth will take it” is medium. One who takes one given by being placed at his feet [by a bhikkhu] is mild.

                    The moment anyone of these of his own choice or inclination agrees to [accept] a robe given by a householder, his ascetic practice is broken. This is the \emph{breach} in this instance.

                    \vismParagraph{II.21}{21}{}
                    The \emph{benefits} are these. He actually practices in conformity with the dependence, because of the words “The going forth by depending on the refuse-rag robe” (\textbf{\cite{Vin}I 58}, 96); he is established in the first of the Noble Ones’ heritages (\textbf{\cite{A}II 27}); there is no suffering due to protecting; he exists independent of others; there is no fear of robbers; there is no craving connected with use [of robes]; it is a requisite suitable for an ascetic; it is a requisite recommended by the Blessed One thus “valueless, easy to get, and blameless” (\textbf{\cite{A}II 26}); it inspires confidence; it produces the fruits of fewness of wishes, etc.; the right way is cultivated; a good example is set\footnote{\vismAssertFootnoteCounter{9}\vismHypertarget{II.n9}{}\emph{Apādana—}”institution (or production),” not in PED.} to later generations.

                    \vismParagraph{II.22}{22}{}
                    While striving for Death’s army’s rout
                    \begin{verse}
                        The ascetic clad in rag-robe clout\\{}
                        Got from a rubbish heap, shines bright\\{}
                        As mail-clad warrior in the fight.
                    \end{verse}

                    \begin{verse}
                        This robe the world’s great teacher wore,\\{}
                        Leaving rare Kāsi cloth and more;\\{}
                        Of rags from off a rubbish heap\\{}
                        Who would not have a robe to keep?
                    \end{verse}

                    \begin{verse}
                        Minding the words he did profess\\{}
                        When he went into homelessness,\\{}
                        Let him to wear such rags delight\\{}
                        As one in seemly garb bedight.
                    \end{verse}


                    This, firstly, is the commentary on the undertaking, directions, grades, breach, and benefits, in the case of the refuse-rag-wearer’s practice.
                \subsection[\vismAlignedParas{§23–26}Triple-robe-wearer]{Triple-robe-wearer}

                    \vismParagraph{II.23}{23}{}
                    \marginnote{\textcolor{teal}{\footnotesize\{119|61\}}}{}ii. Next there is the \emph{triple-robe-wearer’s practice}. This is undertaken with one of the following statements: “I refuse a fourth robe” or “I undertake the triple-robe-wearer’s practice.” \textcolor{brown}{\textit{[65]}}

                    When a triple-robe wearer has got cloth for a robe, he can put it by for as long as, owing to ill-health, he is unable to make it up, or for as long as he does not find a helper, or lacks a needle, etc., and there is no fault in his putting it by. But it is not allowed to put it by once it has been dyed. That is called cheating the ascetic practice. These are the \emph{directions} for it.

                    \vismParagraph{II.24}{24}{}
                    This too has three \emph{grades}. Herein, one who is strict should, at the time of dyeing, first dye either the inner cloth or the upper garment, and having dyed it, he should wear that round the waist and dye the other. Then he can put that on over the shoulder and dye the cloak of patches. But he is not allowed to wear the cloak of patches round the waist. This is the duty when in an abode inside a village. But it is allowable for him in the forest to wash and dye two together. However, he should sit in a place near [to the robes] so that, if he sees anyone, he can pull a yellow cloth over himself. But for the medium one there is a yellow cloth in the dyeing room for use while dyeing, and it is allowable for him to wear that [as an inner cloth] or to put it on [as an upper garment] in order to do the work of dyeing. For the mild one it is allowable to wear, or put on, the robes of bhikkhus who are in communion (i.e. not suspended, etc.) in order to do the work of dyeing. A bedspread that remains where it is\footnote{\vismAssertFootnoteCounter{10}\vismHypertarget{II.n10}{}\emph{Tatraṭṭhaka-paccattharaṇa—}”a bedspread that remains there”; “A name for what has been determined upon as a bedspread in one’s own resting place or in someone else’s. They say accordingly (it is said in a commentary) that there is no breach of the ascetic practice even when these two, that is, the bedspread and the undyed cloth, are kept as extra robes” (\textbf{\cite{Vism-mhṭ}78–79}). For \emph{tatraṭṭhaka }(fixture) see also §61.} is also allowable for him, but he must not take it about him. And it is allowed for him to use from time to time the robes of bhikkhus who are in communion. It is allowed to one who wears the triple robe as an ascetic practice to have a yellow shoulder-cloth too as a fourth; but it must be only a span wide and three hands long.

                    The moment anyone of these three agrees to [accept] a fourth robe, his ascetic practice is broken. This is the \emph{breach} in this instance.

                    \vismParagraph{II.25}{25}{}
                    The \emph{benefits} are these. The bhikkhu who is a triple-robe wearer is content with the robe as a protection for the body. Hence he goes taking it with him as a bird does its wings (\textbf{\cite{M}I 180}); and such special qualities as having few undertakings, avoidance of storage of cloth, a frugal existence, the abandoning of greed for many robes, living in effacement by observing moderation even in what is permitted, production of the fruits of fewness of wishes, etc., are perfected. \textcolor{brown}{\textit{[66]}}

                    \vismParagraph{II.26}{26}{}
                    No risk of hoarding haunts the man of wit
                    \begin{verse}
                        Who wants no extra cloth for requisite;\\{}
                        Using the triple robe where’er he goes\\{}
                        The pleasant relish of content he knows.
                    \end{verse}

                    \begin{verse}
                        \marginnote{\textcolor{teal}{\footnotesize\{120|62\}}}{}So, would the adept wander undeterred\\{}
                        With naught else but his robes, as flies the bird\\{}
                        With its own wings, then let him too rejoice\\{}
                        That frugalness in garments be his choice.
                    \end{verse}


                    This is the commentary on the undertaking, directions, grades, breach, and benefits, in the case of the triple-robe-wearer’s practice.
                \subsection[\vismAlignedParas{§27–30}Alms-food-eater]{Alms-food-eater}

                    \vismParagraph{II.27}{27}{}
                    iii. The \emph{alms-food-eater’s practice} is \emph{undertaken} with one of the following statements: “I refuse a supplementary [food] supply” or “I undertake the alms-food-eater’s practice.”

                    Now, this alms-food eater should not accept the following fourteen kinds of meal: a meal offered to the Order, a meal offered to specified bhikkhus, an invitation, a meal given by a ticket, one each half-moon day, one each Uposatha day, one each first of the half-moon, a meal given for visitors, a meal for travellers, a meal for the sick, a meal for sick-nurses, a meal supplied to a [particular] residence, a meal given in a principal house,\footnote{\vismAssertFootnoteCounter{11}\vismHypertarget{II.n11}{}“A meal to be given by setting it out in a principal house only.” (\textbf{\cite{Vism-mhṭ}79}) This meaning of \emph{dhura-bhatta }not in PED.} a meal given in turn.

                    If, instead of saying “Take a meal given to the Order”, [meals] are given saying “The Order is taking alms in our house; you may take alms too”, it is allowable to consent. Tickets from the Order that are not for actual food,\footnote{\vismAssertFootnoteCounter{12}\vismHypertarget{II.n12}{}“Tickets that are not for actual food, but deal with medicine, etc.” (\textbf{\cite{Vism-mhṭ}79}) \emph{Paṭikkamana—”refectory” }(28) = \emph{bojun hal }(eating hall) in Sinhalese translation.} and also a meal cooked in a monastery, are allowable as well.

                    These are the \emph{directions} for it.

                    \vismParagraph{II.28}{28}{}
                    This too has three \emph{grades}. Herein, one who is strict takes alms brought both from before and from behind, and he gives the bowl to those who take it while he stands outside a door. He also takes alms brought to the refectory and given there. But he does not take alms by sitting [and waiting for it to be brought later] that day. The medium one takes it as well by sitting [and waiting for it to be brought later] that day; but he does not consent to [its being brought] the next day. The mild one consents to alms [being brought] on the next day and on the day after. Both these last miss the joy of an independent life. There is, perhaps, a preaching on the Noble Ones’ heritages (\textbf{\cite{A}II 28}) in some village. The strict one says to the others “Let us go, friends, and listen to the Dhamma.” One of them says, “I have been made to sit [and wait] by a man, venerable sir,” and the other, “I have consented to [receive] alms tomorrow, venerable sir.” So they are both losers. The other wanders for alms in the morning and then he goes and savours the taste of the Dhamma. \textcolor{brown}{\textit{[67]}}

                    The moment anyone of these three agrees to the extra gain consisting of a meal given to the Order, etc., his ascetic practice is broken. This is the \emph{breach} in this instance.

                    \vismParagraph{II.29}{29}{}
                    The \emph{benefits} are these. He actually practices in conformity with the dependence because of the words “The going forth by depending on the eating of lumps of \marginnote{\textcolor{teal}{\footnotesize\{121|63\}}}{}alms food” (\textbf{\cite{Vin}II 58}, 96); he is established in the second of the Noble Ones’ heritages; his existence is independent of others; it is a requisite recommended by the Blessed One thus “Valueless, easy to get, blameless” (\textbf{\cite{A}II 26}); idleness is eliminated; livelihood is purified; the practice of the minor training rule [of the Pātimokkha] is fulfilled; he is not maintained by another; he helps others; pride is abandoned; craving for tastes is checked; the training precepts about eating as a group, substituting one meal [invitation for another] (see Vinaya, Pācittiya 33 and Comy.), and good behaviour, are not contravened; his life conforms to [the principles of] fewness of wishes; he cultivates the right way; he has compassion for later generations.

                    \vismParagraph{II.30}{30}{}
                    
                    \begin{verse}
                        The monk content with alms for food\\{}
                        Has independent livelihood,\\{}
                        And greed in him no footing finds;\\{}
                        He is as free as the four winds.\\{}
                        He never need be indolent,\\{}
                        His livelihood is innocent,\\{}
                        So let a wise man not disdain\\{}
                        Alms-gathering for his domain.
                    \end{verse}


                    Since it is said:
                    \begin{verse}
                        “If a bhikkhu can support himself on alms\\{}
                        And live without another’s maintenance,\\{}
                        And pay no heed as well to gain and fame,\\{}
                        The very gods indeed might envy him” (\textbf{\cite{Ud}31}).
                    \end{verse}


                    This is the commentary on the undertaking, directions, grades, breach and benefits, in the case of the alms-food-eater’s practice.
                \subsection[\vismAlignedParas{§31–34}House-to-house seeker]{House-to-house seeker}

                    \vismParagraph{II.31}{31}{}
                    iv. The \emph{house-to-house seeker’s practice} is \emph{undertaken} with one of the following statements “I refuse a greedy alms round” or “I undertake the house-to-house seeker’s practice.”

                    Now, the house-to-house seeker should stop at the village gate and make sure that there is no danger. If there is danger in any street or village, it is allowable to leave it out and wander for alms elsewhere. When there is a house door or a street or a village where he [regularly] gets nothing at all, he can go [past it] not counting it as a village. But wherever he gets anything at all it is not allowed [subsequently] to go [past] there and leave it out. This bhikkhu should enter the village early so that he will be able to leave out any inconvenient place and go elsewhere. \textcolor{brown}{\textit{[68]}} But if people who are giving a gift [of a meal] in a monastery or who are coming along the road take his bowl and give alms food, it is allowable. And as this [bhikkhu] is going along the road, he should, when it is the time, wander for alms in any village he comes to and not pass it by. If he gets nothing there or only a little, he should wander for alms in the next village in order. These are the \emph{directions} for it.

                    \vismParagraph{II.32}{32}{}
                    This too has three \emph{grades}. Herein, one who is strict does not take alms brought from before or brought from behind or brought to the refectory and given there. He hands over his bowl at a door, however; for in this ascetic practice there is none equal to the Elder Mahā Kassapa, yet an instance in which even he handed over his \marginnote{\textcolor{teal}{\footnotesize\{122|64\}}}{}bowl is mentioned (see \textbf{\cite{Ud}29}). The medium one takes what is brought from before and from behind and what is brought to the refectory, and he hands over his bowl at a door. But he does not sit waiting for alms. Thus he conforms to the rule of the strict alms-food eater. The mild one sits waiting [for alms to be brought] that day.

                    The ascetic practice of these three is broken as soon as the greedy alms round starts [by going only to the houses where good alms food is given]. This is the \emph{breach} in this instance.

                    \vismParagraph{II.33}{33}{}
                    The \emph{benefits} are these. He is always a stranger among families and is like the moon (\textbf{\cite{S}II 197}); he abandons avarice about families; he is compassionate impartially; he avoids the dangers in being supported by a family; he does not delight in invitations; he does not hope for [meals] to be brought; his life conforms to [the principles of] fewness of wishes, and so on.

                    \vismParagraph{II.34}{34}{}
                    
                    \begin{verse}
                        The monk who at each house his begging plies\\{}
                        Is moonlike, ever new to families,\\{}
                        Nor does he grudge to help all equally,\\{}
                        Free from the risks of house-dependency.\\{}
                        Who would the self-indulgent round forsake\\{}
                        And roam the world at will, the while to make\\{}
                        His downcast eyes range a yoke-length before,\\{}
                        Then let him wisely seek from door to door.
                    \end{verse}


                    This is the commentary on the undertaking, directions, grades, breach, and benefits, in the case of the house-to-house-seeker’s practice. \textcolor{brown}{\textit{[69]}}
                \subsection[\vismAlignedParas{§35–38}One-sessioner]{One-sessioner}

                    \vismParagraph{II.35}{35}{}
                    v. The \emph{one-sessioner’s practice} is \emph{undertaken} with one of the following statements: “I refuse eating in several sessions” or “I undertake the one-sessioner’s practice.”

                    When the one-sessioner sits down in the sitting hall, instead of sitting on an elder’s seat, he should notice which seat is likely to fall to him and sit down on that. If his teacher or preceptor arrives while the meal is still unfinished, it is allowable for him to get up and do the duties. But the Elder Tipiṭaka Cūla-Abhaya said: “He should either keep his seat [and finish his meal] or [if he gets up he should leave the rest of] his meal [in order not to break the ascetic practice]. And this is one whose meal is still unfinished; therefore let him do the duties, but in that case let him not eat the [rest of the] meal.” These are the \emph{directions}.

                    \vismParagraph{II.36}{36}{}
                    This too has three \emph{grades}. Herein, one who is strict may not take anything more than the food that he has laid his hand on whether it is little or much. And if people bring him ghee, etc., thinking “The elder has eaten nothing,” while these are allowable for the purpose of medicine, they are not so for the purpose of food. The medium one may take more as long as the meal in the bowl is not exhausted; for he is called “one who stops when the food is finished.” The mild one may eat as long as he does not get up from his seat. He is either “one who stops with the water” because he eats until he takes [water for] washing the bowl, or “one who stops with the session” because he eats until he gets up.

                    The ascetic practice of these three is broken at the moment when food has been eaten at more than one session. This is the breach in this instance.

                    \vismParagraph{II.37}{37}{}
                    \marginnote{\textcolor{teal}{\footnotesize\{123|65\}}}{}The \emph{benefits} are these. He has little affliction and little sickness; he has lightness, strength, and a happy life; there is no contravening [rules] about food that is not what is left over from a meal; craving for tastes is eliminated; his life conforms to the [principles of] fewness of wishes, and so on.

                    \vismParagraph{II.38}{38}{}
                    No illness due to eating shall he feel
                    \begin{verse}
                        Who gladly in one session takes his meal;\\{}
                        No longing to indulge his sense of taste\\{}
                        Tempts him to leave his work to go to waste.\\{}
                        His own true happiness a monk may find\\{}
                        In eating in one session, pure in mind.\\{}
                        Purity and effacement wait on this;\\{}
                        For it gives reason to abide in bliss.
                    \end{verse}


                    This is the commentary on the undertaking, directions, grades, breach, and benefits, in the case of the one-sessioner’s practice. \textcolor{brown}{\textit{[70]}}

                    
                \subsection[\vismAlignedParas{§39–42}Bowl-food-eater]{Bowl-food-eater}

                    \vismParagraph{II.39}{39}{}
                    vi. The \emph{bowl-food-eater’s practice} is \emph{undertaken} with one of the following statements: “I refuse a second vessel” or “I undertake the bowl-food-eater’s practice.”

                    When at the time of drinking rice gruel, the bowl-food eater gets curry that is put in a dish; he can first either eat the curry or drink the rice gruel. If he puts it in the rice gruel, the rice gruel becomes repulsive when a curry made with cured fish, etc., is put into it. So it is allowable [to do this] only in order to use it without making it repulsive. Consequently this is said with reference to such curry as that. But what is unrepulsive, such as honey, sugar,\footnote{\vismAssertFootnoteCounter{13}\vismHypertarget{II.n13}{}\emph{Sakkarā—}”sugar”: spelt \emph{sakkharā} in PED.} etc., should be put into it. And in taking it he should take the right amount. It is allowable to take green vegetables with the hand and eat them. But unless he does that they should be put into the bowl. Because a second vessel has been refused it is not allowable [to use] anything else, not even the leaf of a tree. These are its \emph{directions}.

                    \vismParagraph{II.40}{40}{}
                    This too has three \emph{grades}. Herein, for one who is strict, except at the time of eating sugarcane, it is not allowed [while eating] to throw rubbish away, and it is not allowed while eating to break up rice-lumps, fish, meat and cakes. [The rubbish should be thrown away and the rice-lumps, etc., broken up before starting to eat.] The medium one is allowed to break them up with one hand while eating; and he is called a “hand ascetic.” The mild one is called a “bowl ascetic”; anything that can be put into his bowl he is allowed, while eating, to break up, [that is, rice lumps, etc.,] with his hand or [such things as palm sugar, ginger, etc.,] with his teeth.

                    The moment anyone of these three agrees to a second vessel his ascetic practice is broken. This is the \emph{breach} in this instance.

                    \vismParagraph{II.41}{41}{}
                    The \emph{benefits} are these. Craving for variety of tastes is eliminated; excessiveness of wishes is abandoned; he sees the purpose and the [right] amount in nutriment; he is not bothered with carrying saucers, etc., about; his life conforms to [the principles of] fewness of wishes and so on.

                    \vismParagraph{II.42}{42}{}
                    He baffles doubts that might arise With extra dishes; downcast eyes
                    \begin{verse}
                        \marginnote{\textcolor{teal}{\footnotesize\{124|66\}}}{}The true devotedness imply\footnote{\vismAssertFootnoteCounter{14}\vismHypertarget{II.n14}{}\emph{Subbata—}”truly devoted”: fm. \emph{su + vata }(having good vows). See also §59.}\\{}
                        Of one uprooting gluttony.\\{}
                        Wearing content as if ‘twere part\\{}
                        Of his own nature, glad at heart;\\{}
                        None but a bowl-food eater may\\{}
                        Consume his food in such a way.
                    \end{verse}


                    This is the commentary on the undertaking, directions, grades, breach, and benefits, in the case of the bowl-food-eater’s practice. \textcolor{brown}{\textit{[71]}}

                    
                \subsection[\vismAlignedParas{§43–46}Late-food-refuser]{Late-food-refuser}

                    \vismParagraph{II.43}{43}{}
                    vii. The \emph{later-food-refuser’s practice} is \emph{undertaken} with one of the following statements: “I refuse additional food” or “I undertake the later-food-refuser’s practice.”

                    Now, when that later-food refuser has shown that he is satisfied, he should not again have the food made allowable [by having it put into his hands according to the rule for bhikkhus] and eat it. These are the \emph{directions} for it.

                    \vismParagraph{II.44}{44}{}
                    This too has three \emph{grades}. Herein, there is no showing that he has had enough with respect to the first lump, but there is when he refuses more while that is being swallowed. So when one who is strict has thus shown that he has had enough [with respect to the second lump], he does not eat the second lump after swallowing the first. The medium one eats also that food with respect to which he has shown that he has had enough. But the mild one goes on eating until he gets up from his seat.

                    The moment any one of these three has eaten what has been made allowable [again] after he has shown that he has had enough, his ascetic practice is broken. This is the \emph{breach} in this instance.

                    \vismParagraph{II.45}{45}{}
                    The \emph{benefits} are these. One is far from committing an offence concerned with extra food; there is no overloading of the stomach; there is no keeping food back; there is no renewed search [for food]; he lives in conformity with [the principles of] fewness of wishes, and so on.

                    \vismParagraph{II.46}{46}{}
                    When a wise man refuses later food
                    \begin{verse}
                        He needs no extra search in weary mood,\\{}
                        Nor stores up food till later in the day,\\{}
                        Nor overloads his stomach in this way.\\{}
                        So, would the adept from such faults abstain,\\{}
                        Let him assume this practice for his gain,\\{}
                        Praised by the Blessed One, which will augment\\{}
                        The special qualities such as content.
                    \end{verse}


                    This is the commentary on the undertaking, directions, grades, breach, and benefits, in the case of the later-food-refuser’s practice.
                \subsection[\vismAlignedParas{§47–55}Forest-dweller]{Forest-dweller}

                    \vismParagraph{II.47}{47}{}
                    viii. The \emph{forest-dweller’s practice} is \emph{undertaken} with one of the following statements: “I refuse an abode in a village” or “I undertake the forest-dweller’s practice.”

                    \vismParagraph{II.48}{48}{}
                    \marginnote{\textcolor{teal}{\footnotesize\{125|67\}}}{}Now, that forest dweller must leave an abode in a village in order to meet the dawn in the forest. Herein, a village abode is the village itself with its precincts. A “village” may consist of one cottage or several cottages, it may be enclosed by a wall or not, have human inhabitants or not, and it can also be a caravan that is inhabited for more than four months. \textcolor{brown}{\textit{[72]}} The “village precincts” cover the range of a stone thrown by a man of medium stature standing between the gate-posts of a walled village, if there are two gate-posts, as at Anurādhapura (cf. \textbf{\cite{Vin}III 46}). The Vinaya experts say that this [stone’s throw] is characterized as up to the place where a thrown stone falls, as, for instance, when young men exercise their arms and throw stones in order to show off their strength. But the Suttanta experts say that it is up to where one thrown to scare crows normally falls. In the case of an unwalled village, the house precinct is where the water falls when a woman standing in the door of the outermost house of all throws water from a basin. Within a stone’s throw of the kind already described from that point is the village. Within a second stone’s throw is the village precinct.

                    \vismParagraph{II.49}{49}{}
                    “Forest,” according to the Vinaya method firstly, is described thus: “Except the village and its precincts, all is forest” (\textbf{\cite{Vin}III 46}). According to the Abhidhamma method it is described thus: “Having gone out beyond the boundary post, all that is forest” (Vibh 251; \textbf{\cite{Paṭis}I 176}). But according to the Suttanta method its characteristic is this: “A forest abode is five hundred bow-lengths distant” (\textbf{\cite{Vin}IV 183}). That should be defined by measuring it with a strung instructor’s bow from the gate-post of a walled village, or from the range of the first stone’s throw from an unwalled one, up to the monastery wall.

                    \vismParagraph{II.50}{50}{}
                    But if the monastery is not walled, it is said in the Vinaya commentaries, it should be measured by making the first dwelling of all the limit, or else the refectory or regular meeting place or Bodhi Tree or shrine, even if that is far from a dwelling [belonging to the monastery]. But in the Majjhima commentary it is said that, omitting the precincts of the monastery and the village, the distance to be measured is that between where the two stones fall. This is the measure here.

                    \vismParagraph{II.51}{51}{}
                    Even if the village is close by and the sounds of men are audible to people in the monastery, still if it is not possible to go straight to it because of rocks, rivers, etc., in between, the five hundred bow-lengths can be reckoned by that road even if one has to go by boat. But anyone who blocks the path to the village here and there for the purpose of [lengthening it so as to be able to say that he is] taking up the practice is cheating the ascetic practice.

                    \vismParagraph{II.52}{52}{}
                    If a forest-dwelling bhikkhu’s preceptor or teacher is ill and does not get what he needs in the forest, \textcolor{brown}{\textit{[73]}} he should take him to a village abode and attend him there. But he should leave in time to meet the dawn in a place proper for the practice. If the affliction increases towards the time of dawn, he must attend him and not bother about the purity of his ascetic practice. These are the \emph{directions}.

                    \vismParagraph{II.53}{53}{}
                    This too has three \emph{grades}. Herein, one who is strict must always meet the dawn in the forest. The medium one is allowed to live in a village for the four months of the Rains. And the mild one, for the winter months too. \marginnote{\textcolor{teal}{\footnotesize\{126|68\}}}{}If in the period defined any one of these three goes from the forest and hears the Dhamma in a village abode, his ascetic practice is not broken if he meets the dawn there, nor is it broken if he meets it as he is on his way back after hearing [the Dhamma]. But if, when the preacher has got up, he thinks “We shall go after lying down awhile” and he meets the dawn while asleep or if of his own choice he meets the dawn while in a village abode, then his ascetic practice is broken. This is the \emph{breach} in this instance.

                    \vismParagraph{II.54}{54}{}
                    The \emph{benefits} are these. A forest-dwelling bhikkhu who has given attention to the perception of forest (see MN 121) can obtain hitherto unobtained concentration, or preserve that already obtained. And the Master is pleased with him, according as it is said: “So, Nāgita, I am pleased with that bhikkhu’s dwelling in the forest” (\textbf{\cite{A}III 343}). And when he lives in a remote abode his mind is not distracted by unsuitable visible objects, and so on. He is free from anxiety; he abandons attachment to life; he enjoys the taste of the bliss of seclusion, and the state of the refuse-rag wearer, etc., becomes him.

                    \vismParagraph{II.55}{55}{}
                    He lives secluded and apart,
                    \begin{verse}
                        Remote abodes delight his heart;\\{}
                        The Saviour of the world, besides,\\{}
                        He gladdens that in groves abides.
                    \end{verse}

                    \begin{verse}
                        The hermit that in woods can dwell\\{}
                        Alone, may gain the bliss as well\\{}
                        Whose savour is beyond the price\\{}
                        Of royal bliss in paradise.
                    \end{verse}

                    \begin{verse}
                        Wearing the robe of rags he may\\{}
                        Go forth into the forest fray;\\{}
                        Such is his mail, for weapons too\\{}
                        The other practices will do.
                    \end{verse}

                    \begin{verse}
                        One so equipped can be assured\\{}
                        Of routing Māra and his horde.\\{}
                        So let the forest glades delight\\{}
                        A wise man for his dwelling’s site.
                    \end{verse}


                    This is the commentary on the undertaking, directions, grades, breach, and benefits, in the case of the forest-dweller’s practice. \textcolor{brown}{\textit{[74]}}

                    
                \subsection[\vismAlignedParas{§56–59}Tree-root-dweller]{Tree-root-dweller}

                    \vismParagraph{II.56}{56}{}
                    ix. The \emph{tree-root-dweller’s practice} is \emph{undertaken} with one of the following statements: “I refuse a roof” or “I undertake the tree-root-dweller’s practice.”

                    The tree-root dweller should avoid such trees as a tree near a frontier, a shrine tree, a gum tree, a fruit tree, a bats’ tree, a hollow tree, or a tree standing in the middle of a monastery. He can choose a tree standing on the outskirts of a monastery. These are the \emph{directions}.

                    \vismParagraph{II.57}{57}{}
                    This has three \emph{grades} too. Herein, one who is strict is not allowed to have a tree that he has chosen tidied up. He can move the fallen leaves with his foot while dwelling there. The medium one is allowed to get it tidied up by those who happen to come along. The mild one can take up residence there after summoning \marginnote{\textcolor{teal}{\footnotesize\{127|69\}}}{}monastery attendants and novices and getting them to clear it up, level it, strew sand and make a fence round with a gate fixed in it. On a special day, a tree-root dweller should sit in some concealed place elsewhere rather than there.

                    The moment any one of these three makes his abode under a roof, his ascetic practice is broken. The reciters of the Aṅguttara say that it is broken as soon as he knowingly meets the dawn under a roof. This is the breach in this instance.

                    \vismParagraph{II.58}{58}{}
                    The \emph{benefits} are these. He practices in conformity with the dependence, because of the words “The going forth by depending on the root of a tree as an abode” (\textbf{\cite{Vin}I 58}, 96); it is a requisite recommended by the Blessed One thus “Valueless, easy to get, and blameless” (\textbf{\cite{A}II 26}); perception of impermanence is aroused through seeing the continual alteration of young leaves; avarice about abodes and love of [building] work are absent; he dwells in the company of deities; he lives in conformity with [the principles of] fewness of wishes, and so on.

                    \vismParagraph{II.59}{59}{}
                    
                    \begin{verse}
                        The Blessed One praised roots of trees\\{}
                        As one of the dependencies (\textbf{\cite{Vin}I 58});\\{}
                        Can he that loves secludedness\\{}
                        Find such another dwelling place?
                    \end{verse}

                    \begin{verse}
                        Secluded at the roots of trees\\{}
                        And guarded well by deities\\{}
                        He lives in true devotedness\\{}
                        Nor covets any dwelling place. \textcolor{brown}{\textit{[75]}}
                    \end{verse}

                    \begin{verse}
                        And when the tender leaves are seen\\{}
                        Bright red at first, then turning green,\\{}
                        And then to yellow as they fall,\\{}
                        He sheds belief once and for all
                    \end{verse}

                    \begin{verse}
                        In permanence. Tree roots have been\\{}
                        Bequeathed by him; secluded scene\\{}
                        No wise man will disdain at all\\{}
                        For contemplating [rise and fall].
                    \end{verse}


                    This is the commentary on the undertaking, directions, grades, breach, and benefits, in the case of the tree-root-dweller’s practice.
                \subsection[\vismAlignedParas{§60–63}Open-air-dweller]{Open-air-dweller}

                    \vismParagraph{II.60}{60}{}
                    x. The \emph{open-air-dweller’s practice} is \emph{undertaken} with one of the following statements: “I refuse a roof and a tree root” or “I undertake the open-air-dweller’s practice.”

                    An open-air dweller is allowed to enter the Uposatha-house for the purpose of hearing the Dhamma or for the purpose of the Uposatha. If it rains while he is inside, he can go out when the rain is over instead of going out while it is still raining. He is allowed to enter the eating hall or the fire room in order to do the duties, or to go under a roof in order to ask elder bhikkhus in the eating hall about a meal, or when teaching and taking lessons, or to take beds, chairs, etc., inside that have been wrongly left outside. If he is going along a road with a requisite belonging to a senior and it rains, he is allowed to go into a wayside rest house. If he has nothing with him, he is not allowed to hurry in order to get to a rest house; \marginnote{\textcolor{teal}{\footnotesize\{128|70\}}}{}but he can go at his normal pace and enter it and stay there as long as it rains. These are the \emph{directions }for it. And the same rule applies to the tree-root dweller too.

                    \vismParagraph{II.61}{61}{}
                    This has three \emph{grades} too. Herein, one who is strict is not allowed to live near a tree or a rock or a house. He should make a robe-tent right out in the open and live in that. The medium one is allowed to live near a tree or a rock or a house so long as he is not covered by them. The mild one is allowed these: a [rock] overhang without a drip-ledge cut in it,\footnote{\vismAssertFootnoteCounter{15}\vismHypertarget{II.n15}{}Reading \emph{acchinna-mariyādaṃ }with \textbf{\cite{Vism-mhṭ}}, which says: \emph{“‘Without a drip-ledge cut (acchinna-mariyādaṃ)’ }means without a drip-ledge \emph{(mariyāda) }made above, which might come under the heading of a drip-ledge \emph{(mariyāda-saṅkhepena) }made to prevent rain water from coming in. But if the rain water comes under the overhang \emph{(pabbhāra) }and is allowed to go in under it, then this comes under the heading of the open air \emph{(abbhokāsika-saṅkhepa)” }(\textbf{\cite{Vism-mhṭ}84}). This seems to refer to the widespread habit in ancient Sri Lanka of cutting a drip-ledge on overhanging rocks used for bhikkhus’ dwellings so that the rain that falls on top of the rock drips down in front of the space under the overhang instead of trickling down under the rock and wetting the back and floor. \emph{Pabbhāra }in this context is “over hang” rather than “slope.”} a hut of branches, cloth stiffened with paste, and a tent treated as a fixture, that has been left by field watchers, and so on.

                    The moment any one of these three goes under a roof or to a tree root to dwell there, \textcolor{brown}{\textit{[76]}} his ascetic practice is broken. The reciters of the Aṅguttara say that it is broken as soon as he knowingly meets the dawn there. This is the breach in this case.

                    \vismParagraph{II.62}{62}{}
                    The benefits are these: the impediment of dwellings is severed; stiffness and torpor are expelled; his conduct deserves the praise “Like deer the bhikkhus live unattached and homeless” (\textbf{\cite{S}I 199}); he is detached; he is [free to go in] any direction; he lives in conformity with [the principles of] fewness of wishes, and so on.

                    \vismParagraph{II.63}{63}{}
                    
                    \begin{verse}
                        The open air provides a life\\{}
                        That aids the homeless bhikkhu’s strife,\\{}
                        Easy to get, and leaves his mind\\{}
                        Alert as a deer, so he shall find
                    \end{verse}

                    \begin{verse}
                        Stiffness and torpor brought to halt.\\{}
                        Under the star-bejewelled vault\\{}
                        The moon and sun furnish his light,\\{}
                        And concentration his delight.\\{}
                        The joy seclusion’s savour gives\\{}
                        He shall discover soon who lives\\{}
                        In open air; and that is why\\{}
                        The wise prefer the open sky.
                    \end{verse}


                    This is the commentary on the undertaking, directions, grades, breach, and benefits, in the case of the open-air-dweller’s practice.
                \subsection[\vismAlignedParas{§64–68}Charnel-groud-dweller]{Charnel-groud-dweller}

                    \vismParagraph{II.64}{64}{}
                    xi. The \emph{charnel-ground-dweller’s practice} is \emph{undertaken} with one of the following statements: “I refuse what is not a charnel ground” or “I undertake the charnel-ground-dweller’s practice.” \marginnote{\textcolor{teal}{\footnotesize\{129|71\}}}{}Now, the charnel-ground dweller should not live in some place just because the people who built the village have called it “the charnel ground” for it is not a charnel ground unless a dead body has been burnt on it. But as soon as one has been burnt on it, it becomes a charnel ground. And even if it has been neglected for a dozen years, it is so still.

                    \vismParagraph{II.65}{65}{}
                    One who dwells there should not be the sort of person who gets walks, pavilions, etc., built, has beds and chairs set out and drinking and washing water kept ready, and preaches Dhamma; for this ascetic practice is a momentous thing. Whoever goes to live there should be diligent. And he should first inform the senior elder of the Order or the king’s local representative in order to prevent trouble. When he walks up and down, he should do so looking at the pyre with half an eye. \textcolor{brown}{\textit{[77]}} On his way to the charnel ground he should avoid the main roads and take a by-path. He should define all the objects [there] while it is day, so that they will not assume frightening shapes for him at night. Even if non-human beings wander about screeching, he must not hit them with anything. It is not allowed to miss going to the charnel ground even for a single day. The reciters of the Aṅguttara say that after spending the middle watch in the charnel ground he is allowed to leave in the last watch. He should not take such foods as sesame flour, pease pudding, fish, meat, milk, oil, sugar, etc., which are liked by non-human beings. He should not enter the homes of families.\footnote{\vismAssertFootnoteCounter{16}\vismHypertarget{II.n16}{}“He should not go into families’ houses because he smells of the dead and is followed by \emph{pisāca }goblins” (\textbf{\cite{Vism-mhṭ}84}).} These are the \emph{directions} for it.

                    \vismParagraph{II.66}{66}{}
                    This has three \emph{grades} too. Herein, one who is strict should live where there are always burnings and corpses and mourning. The medium one is allowed to live where there is one of these three. The mild one is allowed to live in a place that possesses the bare characteristics of a charnel ground already stated.

                    When any one of these three makes his abode in some place not a charnel ground, his ascetic practice is broken. It is on the day on which he does not go to the charnel ground, the Aṅguttara reciters say. This is the \emph{breach} in this case.

                    \vismParagraph{II.67}{67}{}
                    The \emph{benefits} are these. He acquires mindfulness of death; he lives diligently; the sign of foulness is available (see \hyperlink{VI}{Ch. VI}{}); greed for sense desires is removed; he constantly sees the body’s true nature; he has a great sense of urgency; he abandons vanity of health, etc.; he vanquishes fear and dread (MN 4); non-human beings respect and honour him; he lives in conformity with [the principles of] fewness of wishes, and so on.

                    \vismParagraph{II.68}{68}{}
                    
                    \begin{verse}
                        Even in sleep the dweller in a charnel ground shows naught\\{}
                        Of negligence, for death is ever present to his thought;\\{}
                        He may be sure there is no lust after sense pleasure preys\\{}
                        Upon his mind, with many corpses present to his gaze.
                    \end{verse}

                    \begin{verse}
                        Rightly he strives because he gains a sense of urgency,\\{}
                        While in his search for final peace he curbs all vanity.\\{}
                        Let him that feels a leaning to Nibbāna in his heart\\{}
                        Embrace this practice for it has rare virtues to impart.
                    \end{verse}


                    \marginnote{\textcolor{teal}{\footnotesize\{130|72\}}}{}This is the commentary on the undertaking, directions, grades, breach, and benefits, in the case of the charnel-ground dweller’s practice. \textcolor{brown}{\textit{[78]}}
                \subsection[\vismAlignedParas{§69–72}Any-bed-user]{Any-bed-user}

                    \vismParagraph{II.69}{69}{}
                    xii. The \emph{any-bed-user’s practice} is \emph{undertaken} with one of the following statements: “I refuse greed for resting places” or “I undertake the any-bed-user’s practice.”

                    The any-bed user should be content with whatever resting place he gets thus: “This falls to your lot.” He must not make anyone else shift [from his bed]. These are the \emph{directions}.

                    \vismParagraph{II.70}{70}{}
                    This has three \emph{grades} too. Herein, one who is strict is not allowed to ask about the resting place that has fallen to his lot: “Is it far?” or “Is it too near?” or “Is it infested by non-human beings, snakes, and so on?” or “Is it hot?” or “Is it cold?”. The medium one is allowed to ask, but not to go and inspect it. The mild one is allowed to inspect it and, if he does not like it, to choose another.

                    As soon as greed for resting places arises in any one of these three, his ascetic practice is broken. This is the \emph{breach} in this instance.

                    

                    \vismParagraph{II.71}{71}{}
                    The \emph{benefits} are these. The advice “He should be content with what he gets” (\textbf{\cite{J-a}I 476}; \textbf{\cite{Vin}IV 259}) is carried out; he regards the welfare of his fellows in the life of purity; he gives up caring about inferiority and superiority; approval and disapproval are abandoned; the door is closed against excessive wishes; he lives in conformity with [the principles] of fewness of wishes, and so on.

                    \vismParagraph{II.72}{72}{}
                    One vowed to any bed will be
                    \begin{verse}
                        Content with what he gets, and he\\{}
                        Can sleep in bliss without dismay\\{}
                        On nothing but a spread of hay.
                    \end{verse}

                    \begin{verse}
                        He is not eager for the best,\\{}
                        No lowly couch does he detest,\\{}
                        He aids his young companions too\\{}
                        That to the monk’s good life are new.
                    \end{verse}

                    \begin{verse}
                        So for a wise man to delight\\{}
                        In any kind of bed is right;\\{}
                        A Noble One this custom loves\\{}
                        As one the sages’ Lord approves.
                    \end{verse}


                    This is the commentary on the undertaking, directions, grades, breach, and benefits, in the case of the any-bed-user’s practice.
                \subsection[\vismAlignedParas{§73–77}Sitter]{Sitter}

                    \vismParagraph{II.73}{73}{}
                    xiii. The \emph{sitter’s practice} is \emph{undertaken} with one of the following statements: “I refuse lying down” or “I undertake the sitter’s practice.”

                    The sitter can get up in any one of three watches of the night and walk up and down: for lying down is the only posture not allowed. These are the \emph{directions}. \textcolor{brown}{\textit{[79]}}

                    \vismParagraph{II.74}{74}{}
                    This has three \emph{grades} too. Herein, one who is strict is not allowed a back-rest or cloth band or binding-strap [to prevent falling while asleep].\footnote{\vismAssertFootnoteCounter{17}\vismHypertarget{II.n17}{}\emph{Āyogapatta—}”a binding-strap”: this is probably the meaning. But cf. \textbf{\cite{Vin}II 135} and \textbf{\cite{Vin-a}891}.} The medium one \marginnote{\textcolor{teal}{\footnotesize\{131|73\}}}{}is allowed any one of these three. The mild one is allowed a back-rest, a cloth band, a binding-strap, a cushion, a “five-limb” and a “seven-limb.” A “five-limb” is [a chair] made with [four legs and] a support for the back. A “seven-limb” is one made with [four legs,] a support for the back and an [arm] support on each side. They made that, it seems, for the Elder Pīṭhābhaya (Abhaya of the Chair). The elder became a non-returner, and then attained Nibbāna.

                    As soon as any one of these three lies down, his ascetic practice is broken. This is the \emph{breach} in this instance.

                    

                    \vismParagraph{II.75}{75}{}
                    The \emph{benefits} are these. The mental shackle described thus, “He dwells indulging in the pleasure of lying prone, the pleasure of lolling, the pleasure of torpor” (\textbf{\cite{M}I 102}), is severed; his state is suitable for devotion to any meditation subject; his deportment inspires confidence; his state favours the application of energy; he develops the right practice.

                    \vismParagraph{II.76}{76}{}
                    
                    \begin{verse}
                        The adept that can place crosswise\\{}
                        His feet to rest upon his thighs\\{}
                        And sit with back erect shall make\\{}
                        Foul Māra’s evil heart to quake.\\{}
                        No more in supine joys to plump\\{}
                        And wallow in lethargic dump;\\{}
                        Who sits for rest and finds it good\\{}
                        Shines forth in the Ascetics’ Wood.
                    \end{verse}

                    \begin{verse}
                        The happiness and bliss it brings\\{}
                        Has naught to do with worldly things;\\{}
                        So must the sitter’s vow befit\\{}
                        The manners of a man of wit.
                    \end{verse}


                    This is the commentary on the undertaking, directions, grades, breach, and benefits, in the case of the sitter’s practice.

                    \vismParagraph{II.77}{77}{}
                    Now, there is the commentary according to the stanza:
                    \begin{verse}
                        (4) As to the profitable triad,\\{}
                        (5) “Ascetic” and so on distinguished,\\{}
                        (6) As to groups, and also (7) singly,\\{}
                        The exposition should be known (see \hyperlink{II.3}{§3}{}).
                    \end{verse}

            \section[\vismAlignedParas{§78–79}Profitable triad]{Profitable triad}

                \vismParagraph{II.78}{78}{}
                \emph{4. }Herein, as to the \emph{profitable triad}: (\textbf{\cite{Dhs}, p. 1}) all the ascetic practices, that is to say, those of trainers, ordinary men, and men whose cankers have been destroyed, may be either profitable or [in the Arahant’s case] indeterminate. \textcolor{brown}{\textit{[80]}} No ascetic practice is unprofitable.

                \vismParagraph{II.79}{79}{}
                But if someone should say: There is also an unprofitable ascetic practice because of the words “One of evil wishes, a prey to wishes, becomes a forest dweller” (\textbf{\cite{A}III 219}), etc., he should be told: We have not said that he does not live in the forest with unprofitable consciousness. Whoever has his dwelling in the forest is a forest dweller; and he may be one of evil wishes or of few wishes. But, as it was said above (\hyperlink{II.11}{§11}{}), they “are the practices (\emph{aṅga}) of a bhikkhu who is ascetic (\emph{dhuta}) because he has shaken off (\emph{dhuta}) defilement by undertaking one or other of them. Or the \marginnote{\textcolor{teal}{\footnotesize\{132|74\}}}{}knowledge that has got the name “ascetic” (\emph{dhuta}) because it shakes off (\emph{dhunana}) defilement is a practice (\emph{aṅga}) belonging to these, thus they are “ascetic practices” (\emph{dhutaṅga}). Or alternatively, they are ascetic (\emph{dhuta}) because they shake off (\emph{niddhunana}) opposition, and they are practices (\emph{aṅga}) because they are a way (\emph{paṭipatti}).” Now, no one called “ascetic” on account of what is unprofitable could have these as his practices; nor does what is unprofitable shake off anything so that those things to which it belonged as a practice could be called “ascetic practices.” And what is unprofitable does not both shake off cupidity for robes, etc., and become the practice of the way. Consequently it was rightly said that no ascetic practice is unprofitable. And those who hold that an ascetic practice is outside the profitable triad\footnote{\vismAssertFootnoteCounter{18}\vismHypertarget{II.n18}{}For the triads of the \emph{Abhidhamma Mātikā} (Abhidhamma Schedule) see \hyperlink{XIII.n20}{Ch. XIII, n.20}{}. “‘\emph{Those who hold}’: a reference to the inhabitants of the Abhayagiri Monastery at Anurādhapura. For they say that ascetic practice is a concept consisting in a name \emph{(nāma-paññatti). }That being so, they could have no meaning of shaking off defilements, or possibility of being undertaken, because in the ultimate sense they would be non-existent [concepts having no existence]” (\textbf{\cite{Vism-mhṭ}87}). Cf. \hyperlink{IV.29}{IV.29}{}.} have no ascetic practice as regards meaning. Owing to the shaking off of what is non-existent could it be called an ascetic practice? Also there are the words “Proceeded to undertake the ascetic qualities” (\textbf{\cite{Vin}III 15}), and it follows\footnote{\vismAssertFootnoteCounter{19}\vismHypertarget{II.n19}{}\emph{Āpajjati }(and its noun \emph{āpatti) }is the normal word used for undesirable consequences that follow on some unsound logical proposition. See \hyperlink{XVI.68}{XVI.68f.}{} This meaning is not in PED.} that those words are contradicted. So that should not be accepted.

                This, in the first place, is the commentary on the profitable triad.
            \section[\vismAlignedParas{§80–86}Ascetic and so on distinguished]{Ascetic and so on distinguished}

                \vismParagraph{II.80}{80}{}
                \emph{5. As to “ascetic and so on distinguished},” the following things should be understood, that is to say, ascetic, a preacher of asceticism, ascetic states, ascetic practices, and for whom the cultivation of ascetic practices is suitable.

                \vismParagraph{II.81}{81}{}
                Herein, \emph{ascetic} means either a person whose defilements are shaken off, or a state that entails shaking off defilements.

                A \emph{preacher of asceticism}: one is ascetic but not a preacher of asceticism, another is not ascetic but a preacher of asceticism, another is neither ascetic nor a preacher of asceticism, and another is both ascetic and a preacher of asceticism.

                \vismParagraph{II.82}{82}{}
                Herein, one who has shaken off his defilements with an ascetic practice but does not advise and instruct another in an ascetic practice, like the Elder Bakkula, is “ascetic but not a preacher of asceticism,” according as it is said: “Now, the venerable Bakkula was ascetic but not a preacher of asceticism.”

                One who \textcolor{brown}{\textit{[81]}} has not shaken off his own defilements but only advises and instructs another in an ascetic practice, like the Elder Upananda, is “not ascetic but a preacher of asceticism,” according as it is said: “Now, the venerable Upananda son of the Sakyans was not ascetic but a preacher of asceticism.”

                One who has failed in both, like Lāḷudāyin, is “neither ascetic nor a preacher of asceticism,” according as it is said: “Now, the venerable Lāḷudāyin was neither ascetic nor a preacher of asceticism.” \marginnote{\textcolor{teal}{\footnotesize\{133|75\}}}{}One who has succeeded in both, like the General of the Dhamma, is “both ascetic and a preacher of asceticism,” according as it is said: “Now, the venerable Sāriputta was ascetic and a preacher of asceticism.”

                \vismParagraph{II.83}{83}{}
                \emph{Ascetic states}: the five states that go with the volition of an ascetic practice, that is to say, fewness of wishes, contentment, effacement, seclusion, and that specific quality\footnote{\vismAssertFootnoteCounter{20}\vismHypertarget{II.n20}{}\emph{Idamatthitā—}”that specific quality”: “Owing to these profitable states it exists, (thus it is ‘specific by those’; \emph{imehi kusaladhammehi atthi = idam-atthi). }The knowledge by means of which one who has gone forth should be established in the refuse-rag-wearer’s practice, etc., and by means of which, on being so instructed one undertakes and persists in the ascetic qualities—that knowledge is \emph{idamatthitā” }(\textbf{\cite{Vism-mhṭ}88}).} are called “ascetic states’ because of the words “Depending on fewness of wishes” (\textbf{\cite{A}III 219}), and so on.

                \vismParagraph{II.84}{84}{}
                Herein, \emph{fewness of wishes} and \emph{contentment} are non-greed. \emph{Effacement} and \emph{seclusion} belong to the two states, non-greed and non-delusion. \emph{That specific quality }is knowledge. Herein, by means of non-greed a man shakes off greed for things that are forbidden. By means of non-delusion he shakes off the delusion that hides the dangers in those same things. And by means of non-greed he shakes off indulgence in pleasure due to sense desires that occurs under the heading of using what is allowed. And by means of non-delusion he shakes off indulgence in self-mortification that occurs under the heading of excessive effacement in the ascetic practices. That is why these states should be understood as “ascetic states.”

                \vismParagraph{II.85}{85}{}
                \emph{Ascetic practices}: these should be understood as the thirteen, that is to say, the refuse-rag-wearer’s practice … the sitter’s practice, which have already been described as to meaning and as to characteristic, and so forth.

                \vismParagraph{II.86}{86}{}
                \emph{For whom the cultivation of ascetic practices is suitable}: [they are suitable] for one of greedy temperament and for one of deluded temperament. Why? Because the cultivation of ascetic practices is both a difficult progress\footnote{\vismAssertFootnoteCounter{21}\vismHypertarget{II.n21}{}See \hyperlink{XXI.117}{XXI.117}{}.} and an abiding in effacement; and greed subsides with the difficult progress, while delusion is got rid of in those diligent by effacement. Or the cultivation of the forest-dweller’s practice and the tree-root-dweller’s practice here are suitable for one of hating temperament; for hate too subsides in one who dwells there without coming into conflict.

                This is the commentary “as to ‘ascetic’ and so on distinguished.” \textcolor{brown}{\textit{[82]}}
            \section[\vismAlignedParas{§87–89}Groups]{Groups}

                \vismParagraph{II.87}{87}{}
                \emph{6. and 7. As to groups and also singly. }Now, \emph{6. as to groups}: these ascetic practices are in fact only eight, that is to say, three principal and five individual practices. Herein, the three, namely, the house-to-house-seeker’s practice, the one-sessioner’s practice, and the open-air-dweller’s practice, are principal practices. For one who keeps the house-to-house-seeker’s practice will keep the alms-food-eater’s practice; and the bowl-food-eater’s practice and the later-food-refuser’s practice will be well kept by one who keeps the one-sessioner’s practice. And what need has one who keeps the open-air-dweller’s practice to keep the tree-root-dweller’s practice or the any-bed-user’s practice? So there are these three principal practices that, \marginnote{\textcolor{teal}{\footnotesize\{134|76\}}}{}together with the five individual practices, that is to say, the forest-dweller’s practice, the refuse-rag-wearer’s practice, the triple-robe-wearer’s practice, the sitter’s practice, and the charnel-ground-dweller’s practice, come to eight only.

                \vismParagraph{II.88}{88}{}
                Again they come to four, that is to say, two connected with robes, five connected with alms food, five connected with the resting place, and one connected with energy. Herein, it is the sitter’s practice that is connected with energy; the rest are obvious.

                Again they all amount to two only, since twelve are dependent on requisites and one on energy. Also they are two according to what is and what is not to be cultivated. For when one cultivating an ascetic practice finds that his meditation subject improves, he should cultivate it; but when he is cultivating one and finds that his meditation subject deteriorates, he should not cultivate it. But when he finds that, whether he cultivates one or not, his meditation subject only improves and does not deteriorate, he should cultivate them out of compassion for later generations. And when he finds that, whether he cultivates them or not, his meditation subject does not improve, he should still cultivate them for the sake of acquiring the habit for the future. So they are of two kinds as what is and what is not to be cultivated.

                \vismParagraph{II.89}{89}{}
                And all are of one kind as volition. For there is only one ascetic practice, namely, that consisting in the volition of undertaking. Also it is said in the Commentary: “It is the volition that is the ascetic practice, they say.”
            \section[\vismAlignedParas{§90–93}Singly]{Singly}

                \vismParagraph{II.90}{90}{}
                \emph{7. Singly}: with thirteen for bhikkhus, eight for bhikkhunīs, twelve for novices, seven for female probationers and female novices, and two for male and female lay followers, there are thus forty-two.

                \vismParagraph{II.91}{91}{}
                If there is a charnel ground in the open that complies with the forest-dweller’s practice, one bhikkhu is able to put all the ascetic practices into effect simultaneously.

                But the two, namely, the forest-dweller’s practice and the later-food-refuser’s practice, are forbidden to bhikkhunīs by training precept. \textcolor{brown}{\textit{[83]}} And it is hard for them to observe the three, namely, the open-air-dweller’s practice, the tree-root-dweller’s practice, and the charnel-ground-dweller’s practice, because a bhikkhunī is not allowed to live without a companion, and it is hard to find a female companion with like desire for such a place, and even if available, she would not escape having to live in company. This being so, the purpose of cultivating the ascetic practice would scarcely be served. It is because they are reduced by five owing to this inability to make use of certain of them that they are to be understood as eight only for bhikkhunīs.

                \vismParagraph{II.92}{92}{}
                Except for the triple-robe-wearer’s practice all the other twelve as stated should be understood to be for novices, and all the other seven for female probationers and female novices.

                The two, namely, the one-sessioner’s practice and the bowl-food-eater’s practice, are proper for male and female lay followers to employ. In this way there are two ascetic practices.

                This is the commentary “as to groups and also singly.”

                \vismParagraph{II.93}{93}{}
                \marginnote{\textcolor{teal}{\footnotesize\{135|77\}}}{}And this is the end of the treatise on the ascetic practices to be undertaken for the purpose of perfecting those special qualities of fewness of wishes, contentment, etc., by means of which there comes about the cleansing of virtue as described in the \emph{Path of Purification}, which is shown under the three headings of virtue, concentration, and understanding, contained in the stanza, “When a wise man, established well in virtue” (\hyperlink{I.1}{I.1}{}).

                The second chapter called “The Description of the Ascetic Practices” in the \emph{Path of Purification} composed for the purpose of gladdening good people.
    
    \part[Concentration (\emph{Samādhi})]{Concentration (\emph{Samādhi})\vismHypertarget{pII}}
        \label{pII}


        \chapter[Taking a Meditation Subject]{Taking a Meditation Subject\vismHypertarget{III}\newline{\textnormal{\emph{Kammaṭṭhāna-gahaṇa-niddesa}}}}
            \label{III}


            \vismParagraph{III.1}{1}{}
            \marginnote{\textcolor{teal}{\footnotesize\{139|81\}}}{}\textcolor{brown}{\textit{[84]}} Now, concentration is described under the heading of “consciousness” in the phrase “develops consciousness and understanding” (\hyperlink{I.1}{I.1}{}). It should be developed by one who has taken his stand on virtue that has been purified by means of the special qualities of fewness of wishes, etc., and perfected by observance of the ascetic practices. But that concentration has been shown only very briefly and so it is not even easy to understand, much less to develop. There is therefore the following set of questions, the purpose of which is to show the method of its development in detail:

            (i) What is concentration? (ii) In what sense is it concentration? (iii) What are its characteristic, function, manifestation, and proximate cause?

            (iv) How many kinds of concentration are there? (v) What is its defilement? (vi) What is its cleansing? (vii) How should it be developed? (viii) What are the benefits of the development of concentration?\footnote{\vismAssertFootnoteCounter{1}\vismHypertarget{III.n1}{}The answer to question (vii) stretches from \hyperlink{III.27}{III.27}{} to \hyperlink{XI.119}{XI.119}{}. That to question (viii) from XI. 120 up to the end of \hyperlink{XIII}{Ch. XIII}{}.}

            \vismParagraph{III.2}{2}{}
            Here are the answers:
            \section[\vismAlignedParas{§2}(i) What is concentration?]{(i) What is concentration?}

                Concentration is of many sorts and has various aspects. An answer that attempted to cover it all would accomplish neither its intention nor its purpose and would, besides, lead to distraction; so we shall confine ourselves to the kind intended here, calling concentration profitable unification of mind.\footnote{\vismAssertFootnoteCounter{2}\vismHypertarget{III.n2}{}\emph{“Cittass’ ekaggatā” }is rendered here as “unification of mind” in the sense of agreement or harmony (cf. \emph{samagga) }of consciousness and its concomitants in focusing on a single object (see \textbf{\cite{A}I 70}). It is sometimes rendered “one-pointedness” in that sense, or in the sense of the focusing of a searchlight. It may be concluded that this term is simply a synonym for \emph{samādhi }and nothing more, firstly from its use in the suttas, and secondly from the fact that it is given no separate definition in the description of the formations aggregate in \hyperlink{XIV}{Ch. XIV}{}. Cf. gloss at \textbf{\cite{M-a}I 124}.}
            \section[\vismAlignedParas{§3}(ii) In what sense is it concentration?]{(ii) In what sense is it concentration?}

                \vismParagraph{III.3}{3}{}
                \marginnote{\textcolor{teal}{\footnotesize\{140|82\}}}{}(ii) \textsc{In what sense is it concentration?} It is concentration (\emph{samādhi}) in the sense of concentrating (\emph{samādhāna}). What is this concentrating? It is the centring (\emph{ādhāna}) of consciousness and consciousness-concomitants evenly (\emph{samaṃ}) and rightly (\emph{sammā}) on a single object; placing, is what is meant. \textcolor{brown}{\textit{[85]}} So it is the state in virtue of which consciousness and its concomitants remain evenly and rightly on a single object, undistracted and unscattered, that should be understood as concentrating.
            \section[\vismAlignedParas{§4}(iii) What are its characteristic, etc.?]{(iii) What are its characteristic, etc.?}

                \vismParagraph{III.4}{4}{}
                (iii) \textsc{What are its characteristic, function, manifestation, and proximate cause? }Concentration has non-distraction as its characteristic.\footnote{\vismAssertFootnoteCounter{3}\vismHypertarget{III.n3}{}“The characteristic of non-distraction is the individual essence peculiar to concentration. Hence no analysis of it is possible, which is why he said: \emph{‘It is of one kind with the characteristic of non-distraction’” }(\textbf{\cite{Vism-mhṭ}91}).} Its function is to eliminate distraction. It is manifested as non-wavering. Because of the words, “Being blissful, his mind becomes concentrated” (\textbf{\cite{D}I 73}), its proximate cause is bliss.
            \section[\vismAlignedParas{§5–25}(iv) How many kinds of concentration are there?]{(iv) How many kinds of concentration are there?}

                \vismParagraph{III.5}{5}{}
                (iv) \textsc{How many kinds of concentration are there?}

                (1) First of all it is of one kind with the characteristic of non-distraction. (2) Then it is of two kinds as access and absorption;\footnote{\vismAssertFootnoteCounter{4}\vismHypertarget{III.n4}{}“Applied thought that occurs as though absorbing \emph{(appento) }associated states in the object is absorption \emph{(appanā). }Accordingly it is described as ‘absorption, absorbing \emph{(appanā vyappanā)’ }(\textbf{\cite{M}III 73}). Now since that is the most important, the usage of the Commentaries is to call all exalted and unsurpassed jhāna states ‘absorption’ [as well as the applied thought itself], and likewise to apply the term of common usage ‘access’ to the limited [i.e. sense-sphere] jhāna that heralds the arising of the former, just as the term ‘village access,’ etc. is applied to the neighbourhood of a village” (\textbf{\cite{Vism-mhṭ}91}).} (3) likewise as mundane and supramundane,\footnote{\vismAssertFootnoteCounter{5}\vismHypertarget{III.n5}{}“The round \emph{(vaṭṭa, }see \hyperlink{XVII.298}{XVII.298}{}) [including fine-material and immaterial heavens] is called the world \emph{(loka) }because of its crumbling \emph{(lujjana) }and disintegrating \emph{(palujjana). ‘Mundane’ (lokiya) }means connected with the world because of being included in it or found there. \emph{‘Supramundane’ (lokuttara) }means beyond the world, excepted from it, because of not being included in it [through being associated with Nibbāna]” (Vism-mhṭ 91). See also “nine supramundane states. (\hyperlink{VII.68}{VII.68}{}, \hyperlink{VII.74}{74f.}{})”} (4) as with happiness and without happiness, and (5) as accompanied by bliss and accompanied by equanimity.\footnote{\vismAssertFootnoteCounter{6}\vismHypertarget{III.n6}{}In loose usage \emph{pīti} (happiness) and \emph{sukha }(pleasure or bliss) are almost synonyms. They become differentiated in the jhāna formulas (see \hyperlink{IV.100}{IV.100}{}), and then technically \emph{pīti}, as the active thrill of rapture, is classed under the formations aggregate and \emph{sukha }under the feeling aggregate. The valuable word “happiness” was chosen for \emph{pīti }rather than the possible alternatives of “joy” (needed for \emph{somanassa), }“interest” (which is too flat), “rapture” (which is overcharged), or “zest.” For \emph{sukha, }while “pleasure” seemed to fit admirably where ordinary pleasant feeling is intended, another, less crass, word seemed necessary for the refined pleasant feeling of jhāna and the “bliss” of Nibbāna (which is not feeling aggregate—see \textbf{\cite{M}I 400}). “Ease” is sometimes used.

                        “Neither-painful-nor-pleasant feeling is intended here by ‘equanimity’ \emph{(upekkhā, }lit, onlooking); for it ‘looks on’ \emph{(upekkhati) }at the occurrence of [bodily] pleasure and pain by maintaining the neutral (central) mode” (\textbf{\cite{Vism-mhṭ}92}).} It is of three kinds (6) as inferior, medium and superior; likewise (7) as with applied thought and sustained thought, etc., (8) as accompanied by happiness, etc., and (9) as limited, exalted, and measureless. It is of four kinds (10) as of difficult progress and sluggish \marginnote{\textcolor{teal}{\footnotesize\{141|83\}}}{}direct-knowledge, etc.; likewise (11) as limited with limited object, etc., (12) according to the factors of the four jhānas, (13) as partaking of diminution, etc., (14) as of the sense sphere, etc., and (15) as predominance, and so on. (16) It is of five kinds according to the factors of the five jhānas reckoned by the fivefold method.

                \vismParagraph{III.6}{6}{}
                \emph{1. }Herein, the section dealing with that of one kind is evident in meaning.

                \emph{2. }In the section dealing with that of two kinds, \emph{access }concentration is the unification of mind obtained by the following, that is to say, the six recollections, mindfulness of death, the recollection of peace, the perception of repulsiveness in nutriment, and the defining of the four elements, and it is the unification that precedes absorption concentration. \emph{Absorption }concentration is the unification that follows immediately upon the preliminary-work (\hyperlink{IV.74}{IV.74}{}) because of the words, “The first-jhāna preliminary-work is a condition, as proximity condition, for the first jhāna” (\textbf{\cite{Paṭṭh}II 350} (Se). So it is of two kinds as access and absorption.

                \vismParagraph{III.7}{7}{}
                \emph{3. }In the second dyad \emph{mundane }concentration is profitable unification of mind in the three planes. \emph{Supramundane }concentration is the unification associated with the noble paths. So it is of two kinds as mundane and supramundane.

                \vismParagraph{III.8}{8}{}
                \emph{4. }In the third dyad concentration \emph{with happiness }is the unification of mind in two jhānas in the fourfold reckoning and in three jhānas in the fivefold reckoning. \textcolor{brown}{\textit{[86]}} Concentration \emph{without happiness }is the unification in the remaining two jhānas. But access concentration may be with happiness or without happiness. So it is of two kinds as with happiness and without happiness.

                \vismParagraph{III.9}{9}{}
                \emph{5. }In the fourth dyad concentration \emph{accompanied by bliss }is the unification in three jhānas in the fourfold and four in the fivefold reckoning. That \emph{accompanied by equanimity }is that in the remaining jhāna. Access concentration may be accompanied by bliss or accompanied by equanimity. So it is of two kinds as accompanied by bliss and accompanied by equanimity.

                \vismParagraph{III.10}{10}{}
                \emph{6. }In the first of the triads what has only just been acquired is \emph{inferior}. What is not very well developed is \emph{medium}. What is well developed and has reached mastery is \emph{superior}. So it is of three kinds as inferior, medium, and superior.

                \vismParagraph{III.11}{11}{}
                \emph{7. }In the second triad that \emph{with applied thought and sustained thought }is the concentration of the first jhāna together with access concentration. That \emph{without applied thought, with sustained thought only}, is the concentration of the second jhāna in the fivefold reckoning. For when a man sees danger only in applied thought and not in sustained thought, he aspires only to abandon applied thought when he passes beyond the first jhāna, and so he obtains concentration without applied thought and with sustained thought only. This is said with reference to him. Concentration\emph{ without applied thought and sustained thought }is the unification in the three jhānas beginning with the second in the fourfold reckoning and with the third in the fivefold reckoning (see \textbf{\cite{D}III 219}). So it is of three kinds as with applied thought and sustained thought, and so on.

                \vismParagraph{III.12}{12}{}
                \emph{8. }In the third triad concentration \emph{accompanied by happiness }is the unification in the two first jhānas in the fourfold reckoning and in the three first jhānas in the fivefold reckoning. Concentration \emph{accompanied by bliss }is the unification in those same jhānas and in the third and the fourth respectively in the two reckonings. \marginnote{\textcolor{teal}{\footnotesize\{142|84\}}}{}That \emph{accompanied by equanimity }is that in the remaining jhāna. Access concentration may be accompanied by bliss and happiness or accompanied by equanimity. So it is of three kinds as accompanied by happiness, and so on.

                \vismParagraph{III.13}{13}{}
                \emph{9. }In the fourth triad \emph{limited }concentration is unification on the plane of access. \emph{Exalted }concentration is unification in profitable [consciousness, etc.,] of the fine-material sphere and immaterial sphere. \emph{Measureless }concentration is unification associated with the noble paths. So it is of three kinds as limited, exalted, and measureless.

                \vismParagraph{III.14}{14}{}
                \emph{10. }In the first of the tetrads there is concentration of \emph{difficult progress and sluggish direct-knowledge}. There is that of difficult progress and swift direct-knowledge. There is that of easy progress and sluggish direct-knowledge. And there is that of easy progress and swift direct-knowledge.

                \vismParagraph{III.15}{15}{}
                Herein, the development of concentration that occurs from the time of the first conscious reaction up to the arising of the access of a given jhāna is called \emph{progress}. And the understanding that occurs from the time of access until absorption is called \emph{direct-knowledge}. That progress is difficult for some, being troublesome owing to the tenacious resistance of the inimical states beginning with the hindrances. The meaning is that it is cultivated without ease. \textcolor{brown}{\textit{[87]}} It is easy for others because of the absence of those difficulties. Also the direct-knowledge is sluggish in some and occurs slowly, not quickly. In others it is swift and occurs rapidly, not slowly.

                \vismParagraph{III.16}{16}{}
                Herein, we shall comment below upon the suitable and unsuitable (\hyperlink{IV.35}{IV.35f.}{}), the preparatory tasks consisting in the severing of impediments (\hyperlink{IV.20}{IV.20}{}), etc., and skill in absorption (\hyperlink{IV.42}{IV.42}{}). When a man cultivates what is unsuitable, his progress is difficult and his direct-knowledge sluggish. When he cultivates what is suitable, his progress is easy and his direct-knowledge swift. But if he cultivates the unsuitable in the earlier stage and the suitable in the later stage, or if he cultivates the suitable in the earlier stage and the unsuitable in the later stage, then it should be understood as mixed in his case. Likewise if he devotes himself to development without carrying out the preparatory tasks of severing impediments, etc., his progress is difficult. It is easy in the opposite case. And if he is not accomplished in skill in absorption, his direct-knowledge is sluggish. It is swift if he is so accomplished.

                \vismParagraph{III.17}{17}{}
                Besides, they should be understood as classed according to craving and ignorance, and according to whether one has had practice in serenity and insight.\footnote{\vismAssertFootnoteCounter{7}\vismHypertarget{III.n7}{}\emph{Samatha—}”serenity” is a synonym for absorption concentration, and “insight” \emph{(vipassanā) }a synonym for understanding. \emph{Samatha }is sometimes rendered by “tranquillity” (reserved here for \emph{passaddhi) }or “calm” or “quiet.”} For if a man is overwhelmed by craving, his progress is difficult. If not, it is easy. And if he is overwhelmed by ignorance, his direct-knowledge is sluggish. If not, it is swift. And if he has had no practice in serenity, his progress is difficult. If he has, it is easy. And if he has had no practice in insight, his direct-knowledge is sluggish. If he has, it is swift.

                \vismParagraph{III.18}{18}{}
                Also they should be understood as classed according to defilements and faculties. For if a man’s defilements are sharp and his faculties dull, then his progress \marginnote{\textcolor{teal}{\footnotesize\{143|85\}}}{}is difficult and his direct-knowledge sluggish; but if his faculties are keen, his direct-knowledge is swift. And if his defilements are blunt and his faculties dull, then his progress is easy and his direct-knowledge sluggish; but if his faculties are keen, his direct-knowledge is swift.

                \vismParagraph{III.19}{19}{}
                So as regards this progress and this direct-knowledge, when a person reaches concentration with difficult progress and sluggish direct-knowledge, his concentration is called \emph{concentration of difficult progress and sluggish direct-knowledge}; similarly in the cases of the remaining three.

                So it is of four kinds as of difficult progress and sluggish direct-knowledge, and so on.

                \vismParagraph{III.20}{20}{}
                \emph{11. }In the second tetrad there is limited concentration with a limited object, there is limited concentration with a measureless object, there is measureless concentration with a limited object, and there is measureless concentration with a measureless object. Herein, concentration that is unfamiliar and incapable of being a condition for a higher jhāna \textcolor{brown}{\textit{[88]}} is \emph{limited}. When it occurs with an unextended object (\hyperlink{IV.126}{IV.126}{}), it is \emph{with a limited object}. When it is familiar, well developed, and capable of being a condition for a higher jhāna, it is \emph{measureless}. And when it occurs with an extended object, it is \emph{with a measureless object}. The mixed method can be understood as the mixture of the characteristics already stated. So it is of four kinds as limited with limited object, and so on.

                \vismParagraph{III.21}{21}{}
                \emph{12. }In the third tetrad the first jhāna has five factors, that is to say, applied thought, sustained thought, happiness, bliss, and concentration, following suppression of the hindrances. The second has the three factors remaining after the elimination of applied and sustained thought. The third has two factors with the fading away of happiness. The fourth, where bliss is abandoned, has two factors with concentration and the equanimous feeling that accompanies it. Thus there are four kinds of concentration according to the factors of these four jhānas. So it is of four kinds according to the factors of the four jhānas.

                \vismParagraph{III.22}{22}{}
                \emph{13. }In the fourth tetrad there is concentration partaking of diminution, there is concentration partaking of stagnation, there is concentration partaking of distinction, and there is concentration partaking of penetration. Herein, it should be understood that the state of \emph{partaking of diminution }is accessibility to opposition, the state of \emph{partaking of stagnation }(\emph{ṭhiti}) is stationariness (\emph{saṇṭhāna}) of the mindfulness that is in conformity with that [concentration], the state of \emph{partaking of distinction }is the attaining of higher distinction, and the state of \emph{partaking of penetration }is accessibility to perception and attention accompanied by dispassion, according as it is said: “When a man has attained the first jhāna and he is accessible to perception and attention accompanied by sense desire, then his understanding partakes of diminution. When his mindfulness that is in conformity with that stagnates, then his understanding partakes of stagnation. When he is accessible to perception and attention unaccompanied by applied thought, then his understanding partakes of distinction. When he is accessible to perception and attention accompanied by dispassion and directed to fading away, then his understanding partakes of penetration” (\textbf{\cite{Vibh}330}). The kinds of concentration \marginnote{\textcolor{teal}{\footnotesize\{144|86\}}}{}associated with that [fourfold] understanding are also four in number. So it is of four kinds as partaking of diminution, and so on.

                \vismParagraph{III.23}{23}{}
                \emph{14. }In the fifth tetrad there are the following four kinds of concentration, that is to say, sense-sphere concentration, fine-material-sphere concentration, immaterial-sphere concentration, and unincluded [that is, path] concentration. Herein, \emph{sense-sphere concentration }is all kinds of access unification. Likewise the other three are respectively profitable unification of mind associated with fine-material, [immaterial, and path, jhāna]. So it is of four kinds as of the sense-sphere, and so on.

                \vismParagraph{III.24}{24}{}
                \emph{15. }In the sixth tetrad: “If a bhikkhu obtains concentration, obtains unification of mind, by making zeal (desire) predominant, \textcolor{brown}{\textit{[89]}} this is called concentration due to zeal. If … by making energy predominant … If … by making [natural purity of] consciousness predominant… If … by making inquiry predominant, this is called concentration due to inquiry” (\textbf{\cite{Vibh}216–219}). So it is of four kinds as predominance.

                \vismParagraph{III.25}{25}{}
                \emph{16. }In the pentad there are five jhānas by dividing in two what is called the second jhāna in the fourfold reckoning (see \hyperlink{III.21}{§21}{}), taking the second jhāna to be due to the surmounting of only applied thought and the third jhāna to be due to the surmounting of both applied and sustained thought. There are five kinds of concentration according to the factors of these five jhānas. So its fivefoldness should be understood according to the five sets of jhāna factors.
            \section[\vismAlignedParas{§26}(v), (vi) What are the defiling and the cleansing of it?]{(v), (vi) What are the defiling and the cleansing of it?}

                \vismParagraph{III.26}{26}{}
                (v) \emph{What is its defilement?} (vi) \emph{What is its cleansing?}\textbf{ }Here the answer is given in the Vibhaṅga: “Defilement is the state partaking of diminution, cleansing is the state partaking of distinction” (\textbf{\cite{Vibh}343}). Herein, the state partaking of diminution should be understood in this way: “When a man has attained the first jhāna and he is accessible to perception and attention accompanied by sense desire, then his understanding partakes of diminution” (\textbf{\cite{Vibh}330}). And the state partaking of distinction should be understood in this way: “When he is accessible to perception and attention unaccompanied by applied thought, then his understanding partakes of distinction” (\textbf{\cite{Vibh}330}).
            \section[\vismAlignedParas{§27}(vii) How is it developed?]{(vii) How is it developed?}

                \vismParagraph{III.27}{27}{}
                (vii) \emph{How should it be developed?}
            \section[\vismAlignedParas{§27–28}A. Development in Brief]{A. Development in Brief}

                The method of developing the kind of concentration associated with the noble paths mentioned (\hyperlink{III.7}{§7}{}) under that “of two kinds as mundane and supramundane,” etc., is included in the method of developing understanding; (\hyperlink{XXII}{Ch. XXII}{}) for in developing [path] understanding that is developed too. So we shall say nothing separately [here] about how that is to be developed.

                \vismParagraph{III.28}{28}{}
                But mundane concentration should be developed by one who has taken his stand on virtue that is quite purified in the way already stated. He should sever any of the ten impediments that he may have. He should then approach the good friend, the giver of a meditation subject, and he should apprehend from among the forty meditation subjects one that suits his own temperament. After that he should avoid a monastery unfavourable to the development of concentration and \marginnote{\textcolor{teal}{\footnotesize\{145|87\}}}{}go to live in one that is favourable. Then he should sever the lesser impediments and not overlook any of the directions for development. This is in brief.
            \section[\vismAlignedParas{§29–133}B. Development in Detail]{B. Development in Detail}

                \vismParagraph{III.29}{29}{}
                The detail is this:
                \subsection[\vismAlignedParas{§29–56}The ten impediments]{The ten impediments}

                    Firstly it was said above, \emph{he should sever any of the ten impediments that he may have}. \textcolor{brown}{\textit{[90]}} Now, the “ten impediments” are:
                    \begin{verse}
                        A dwelling, family, and gain,\\{}
                        A class, and building too as fifth,\\{}
                        And travel, kin, affliction, books,\\{}
                        And supernormal powers: ten.
                    \end{verse}


                    Herein, the dwelling itself is the “impediment due to the dwelling.” So too with the family and so on.

                    \vismParagraph{III.30}{30}{}
                    \emph{1. }Herein, a single inner room or a single hut or a whole monastery for the Community is called a \emph{dwelling}. This is not an impediment for everyone. It is an impediment only for anyone whose mind is exercised about the building, etc., that goes on there, or who has many belongings stored there, or whose mind is caught up by some business connected with it. For any other it is not an impediment.

                    \vismParagraph{III.31}{31}{}
                    Here is a relevant story. Two clansmen left Anurādhapura, it seems, and eventually went forth at the Thūpārāma.\footnote{\vismAssertFootnoteCounter{8}\vismHypertarget{III.n8}{}One of the principal monasteries in Anurādhapura.} One of them made himself familiar with the Two Codes,\footnote{\vismAssertFootnoteCounter{9}\vismHypertarget{III.n9}{}\emph{Dve mātikā—}the “two codes”: see \hyperlink{I.n11}{Ch. I, n. 11}{}. But \textbf{\cite{Vism-mhṭ}} says here: \emph{“‘Observers of the codes’ }are observers of the codes (summaries) of the Dhamma and Vinaya” (\textbf{\cite{Vism-mhṭ}117}).} and when he had acquired five years’ seniority, he took part in the Pavāraṇā\footnote{\vismAssertFootnoteCounter{10}\vismHypertarget{III.n10}{}\emph{Pavāraṇa: }ceremony held at the end of the rains, during three months of which season bhikkhus have to undertake to live in one place in order to avoid travel while crops are growing. It consists in a meeting of the bhikkhus who have spent the rains together, at which each member present invites \emph{(pavāreti) }the Community to point out his faults (breaches of Vinaya rules) committed during the preceding three months (\textbf{\cite{Vin}I 155}).} and then left for the place called Pācīnakhaṇḍarājī.\footnote{\vismAssertFootnoteCounter{11}\vismHypertarget{III.n11}{}“\emph{Pācinakhaṇḍarājā ti puratthimadisāya pabbatakhaṇḍānaṃ antare vanarājīṭṭhānaṃ}” (\textbf{\cite{Vism-mhṭ}97}).} The other stayed on where he was. Now, when the one who had gone to Pācīnakhaṇḍarājī had lived there a long time and had become an elder,\footnote{\vismAssertFootnoteCounter{12}\vismHypertarget{III.n12}{}For the first five years after the admission \emph{(upasampadā) }a bhikkhu is called a “new \emph{(nava) }bhikkhu”; from five to ten years he is called a “middle \emph{(majjhima) }bhikkhu”; with ten or more years’ seniority he is called an “elder \emph{(thera) }bhikkhu.”} he thought, “This place is good for retreat; suppose I told my friend about it?” So he set out, and in due course he entered the Thūpārāma. As he entered, the elder of the same seniority saw him, went to meet him, took his bowl and robe and did the duties.

                    \vismParagraph{III.32}{32}{}
                    \marginnote{\textcolor{teal}{\footnotesize\{146|88\}}}{}The visiting elder went into his lodging. He thought, “Now my friend will be sending me ghee or molasses or a drink; for he has lived long in this city.” He got nothing that night, and in the morning he thought, “Now he will be sending me rice gruel and solid food sent by his supporters.” When he saw none, he thought, “There is no one to bring it. No doubt they will give it when we go into the town.” Early in the morning they went into the town together. When they had wandered through one street and had got only a ladleful of gruel, they sat down in a sitting hall to drink it.\footnote{\vismAssertFootnoteCounter{13}\vismHypertarget{III.n13}{}The last sentence here might refer to a free mass distribution of gruel \emph{(yāgu), }which appears to have been more or less constantly maintained at Anurādhapura.}

                    \vismParagraph{III.33}{33}{}
                    Then the visitor thought, “Perhaps there is no individual giving of gruel. But as soon as it is the time for the meal people will give special food.” But when it was time for the meal, they ate what they had got by wandering for alms. Then the visitor said, “Venerable sir, how is this? Do you live in this way all the time?”—“Yes, friend.”—”Venerable sir, Pācīnakhaṇḍarājī is comfortable; let us go there.” Now, as the elder came out from the city \textcolor{brown}{\textit{[91]}} by the southern gate he took the Kumbhakāragāma road [which leads to Pācīnakhaṇḍarājī]. The visitor asked, “But, venerable sir, why do you take this road?”—”Did you not recommend Pācīnakhaṇḍarājī, friend?”—”But how is this, venerable sir, have you no extra belongings in the place you have lived in for so long?”—”That is so, friend. The bed and chair belong to the Community, and they are put away [as usual]. There is nothing else.”—”But, venerable sir, I have left my staff and my oil tube and my sandal bag there.”—”Have you already collected so much, friend, living there for just one day?”—“Yes, venerable sir.”

                    \vismParagraph{III.34}{34}{}
                    He was glad in his heart, and he paid homage to the elder: “For those like you, venerable sir, everywhere is a forest dwelling. The Thūpārāma is a place where the relics of four Buddhas are deposited; there is suitable hearing of the Dhamma in the Brazen Palace; there is the Great Shrine to be seen; and one can visit elders. It is like the time of the Buddha. It is here that you should live.” On the following day he took his bowl and [outer] robe and went away by himself. It is no impediment for one like that.

                    \vismParagraph{III.35}{35}{}
                    \emph{2. Family }means a family consisting of relatives or of supporters. For even a family consisting of supporters is an impediment for someone who lives in close association with it in the way beginning, “He is pleased when they are pleased” (\textbf{\cite{S}III 11}), and who does not even go to a neighbouring monastery to hear the Dhamma without members of the family.

                    \vismParagraph{III.36}{36}{}
                    But even mother and father are not an impediment for another, as in the case of the young bhikkhu, the nephew of the elder who lived at the Koraṇḍaka Monastery. He went to Rohaṇa for instruction, it seems. The elder’s sister, who was a lay devotee, was always asking the elder how her son was getting on. One day the elder set out for Rohaṇa to fetch him back.

                    \vismParagraph{III.37}{37}{}
                    The young bhikkhu too thought, “I have lived here for a long time. Now I might go and visit my preceptor and find out how the lay devotee is,” and he left \marginnote{\textcolor{teal}{\footnotesize\{147|89\}}}{}Rohaṇa. The two met on the banks of the [Mahaveli] River. He did the duties to the elder at the foot of a tree. When asked, “Where are you going?” he told him his purpose. The elder said: “You have done well. The lay devotee is always asking after you. That was why I came. You may go, but I shall stay here for the Rains,” and he dismissed him. \textcolor{brown}{\textit{[92]}} He arrived at the monastery on the actual day for taking up residence for the Rains. The lodging allotted to him happened to be the one for which his father had undertaken responsibility.

                    \vismParagraph{III.38}{38}{}
                    His father came on the following day and asked, “To whom was our lodging allotted, venerable sirs?” When he heard that it had fallen to a young visitor, he went to him. After paying homage to him, he said, “Venerable sir, there is an obligation for him who has taken up residence for the Rains in our lodging.”—”What is it, lay follower?”—”It is to take alms food only in our house for the three months, and to let us know the time of departure after the Pavāraṇā ceremony.” He consented in silence. The lay devotee went home and told his wife. “There is a visiting lord who has taken up residence for the Rains in our lodging. He must be carefully looked after,” and she agreed. She prepared good food of various kinds for him.\footnote{\vismAssertFootnoteCounter{14}\vismHypertarget{III.n14}{}It is usual to render the set phrase \emph{paṇītaṃ khādanīyaṃ bhojanīyaṃ }by some such phrase as “sumptuous food both hard and soft,” which is literal but unfamiliar-sounding.} Though the youth went to his relatives’ home at the time of the meal, no one recognized him.

                    \vismParagraph{III.39}{39}{}
                    When he had eaten alms food there during the three months and had completed the residence for the Rains, he announced his departure. Then his relatives said, “Let it be tomorrow, venerable sir,” and on the following day, when they had fed him in their house and filled his oil tube and given him a lump of sugar and a nine-cubit length of cloth, they said, “Now you are leaving, venerable sir.” He gave his blessing and set out for Rohaṇa.

                    \vismParagraph{III.40}{40}{}
                    His preceptor had completed the Pavāraṇā ceremony and was on his way back. They met at the same place as before. He did the duties to the elder at the foot of a tree. The elder asked him, “How was it, my dear, did you see the good woman lay devotee?” He replied, “Yes, venerable sir,” and he told him all that had happened. He then anointed the elder’s feet with the oil, made him a drink with the sugar, and presented him with the length of cloth. He then, after paying homage to the elder, told him, “Venerable sir, only Rohaṇa suits me,” and he departed. The elder too arrived back at his monastery, and next day he went into the village of Koraṇḍaka.

                    \vismParagraph{III.41}{41}{}
                    The lay devotee, his sister, had always kept looking down the road, thinking, “My brother is now coming with my son.” When she saw him coming alone, she thought, “My son must be dead; that is why the elder is coming alone,” and she fell at the elder’s feet, lamenting and weeping. Suspecting that it must have been out of fewness of wishes that the youth had gone away without announcing himself, \textcolor{brown}{\textit{[93]}} the elder comforted her and told her all that had happened, and he took the length of cloth out of his bag and showed it to her.

                    \vismParagraph{III.42}{42}{}
                    \marginnote{\textcolor{teal}{\footnotesize\{148|90\}}}{}She was appeased. She prostrated herself in the direction taken by her son, and she said: “Surely the Blessed One taught the way of the Rathavinīta, the way of the Nālaka, the way of the Tuvaṭaka, and the way of the great Noble Ones’ heritages\footnote{\vismAssertFootnoteCounter{15}\vismHypertarget{III.n15}{}“The way of the Rathavinīta \emph{(Rathavinīta-paṭipadā)”: }this is a reference to certain suttas that were adopted by bhikkhus as a “way” \emph{(paṭipadā) }or guide to practice. The suttas mentioned here are Rathavinīta (\textbf{\cite{M}I 145}), Nālaka (\textbf{\cite{Sn}, p. 131}), Tuvaṭaka (\textbf{\cite{Sn}179}), Noble One’s Heritages (\emph{ariyavaṃsa—}\textbf{\cite{A}II 27}). Others are mentioned at \textbf{\cite{M-a}I 92}; III 6; \textbf{\cite{S-a}III 291}. The Ariyavaṃsa Sutta itself has a long commentary on practice, and it is mentioned in the Commentaries as a popular subject for preaching (see e.g. commentary to AN III 42).} showing contentment with the four requisites and delight in development, making a bhikkhu such as my son a body-witness. So, although for three months he ate in the house of the mother who bore him, yet he never said ‘I am your son, you are my mother!’ Oh, admirable man!” Even mother and father are no impediment for one such as him, so how much less any other family that supports him.

                    \vismParagraph{III.43}{43}{}
                    \emph{3. Gain }is the four requisites. How are they an impediment? Wherever a meritorious bhikkhu goes, people give him a large supply of requisites. With giving blessings to them and teaching them the Dhamma he gets no chance to do the ascetic’s duties. From sunrise till the first watch of the night he never breaks his association with people. Again, even at dawn, alms-food eaters fond of opulence come and say, “Venerable sir, such and such a man lay follower, woman lay follower, friend, friend’s daughter, wants to see you,” and being ready to go, he replies, “Take the bowl and robe, friend.” So he is always on the alert. Thus these requisites are an impediment for him. He should leave his group and wander by himself where he is not known. This is the way his impediment is severed.

                    \vismParagraph{III.44}{44}{}
                    \emph{4. Class }is a class (group) of students of suttas or students of Abhidhamma. If with the group’s instruction and questioning he gets no opportunity for the ascetic’s duties, then that group is an impediment for him. He should sever that impediment in this way: if those bhikkhus have already acquired the main part and little still remains, he should finish that off and then go to the forest. If they have only acquired little and much still remains, \textcolor{brown}{\textit{[94]}} he should, without travelling more than a league, approach another instructor of a class within the radius of a league and say, “Help those venerable ones with instruction, etc.” If he does not find anyone in this way, he should take leave of the class, saying. “I have a task to see to, friends; go where it suits you,” and he should do his own work.

                    \vismParagraph{III.45}{45}{}
                    \emph{5. Building} (\emph{kamma}) is new building work (\emph{nava-kamma}). Since one engaged in this must know about what [material] has and has not been got by carpenters, etc., and must see about what has and has not been done, it is always an impediment. It should be severed in this way. If little remains it should be completed. If much remains, it should be handed over to the Community or to bhikkhus who are entrusted with the Community’s affairs, if it is a new building for the Community; or if it is for himself, it should be handed over to those whom he entrusts with his own affairs, but if these are not available, he should relinquish it to the Community and depart.

                    \vismParagraph{III.46}{46}{}
                    \marginnote{\textcolor{teal}{\footnotesize\{149|91\}}}{}\emph{6. Travel }is going on a journey. If someone is expected to give the going forth somewhere else, or if some requisite is obtainable there and he cannot rest content without getting it [that will be an impediment; for] even if he goes into the forest to do the ascetic’s duties, he will find it hard to get rid of thoughts about the journey. So one in this position should apply himself to the ascetic’s duties after he has done the journey and transacted the business.

                    \vismParagraph{III.47}{47}{}
                    \emph{7. Kin }in the case of the monastery means teacher, preceptor, co-resident, pupil, those with the same preceptor as oneself, and those with the same teacher as oneself; and in the case of the house it means mother, father, brother, and so on. When they are sick they are an impediment for him. Therefore that impediment should be severed by curing them with nursing.

                    \vismParagraph{III.48}{48}{}
                    Herein, when the preceptor is sick he must be cared for as long as life lasts if the sickness does not soon depart. Likewise the teacher at the going forth, the teacher at the admission, the co-resident, the pupils to whom one has given the admission and the going forth, and those who have the same preceptor. But the teacher from whom one takes the dependence, the teacher who gives one instruction, the pupil to whom one has given the dependence, the pupil to whom one is giving instruction, and those who have that same teacher as oneself, should be looked after as long as the dependence or the instruction has not been terminated. If one is able to do so, one should look after them even beyond that [period].

                    \vismParagraph{III.49}{49}{}
                    Mother and father should be treated like the preceptor; if they live within the kingdom and look to their son for help, it should be given. \textcolor{brown}{\textit{[95]}} Also if they have no medicine, he should give them his own. If he has none, he should go in search of it as alms and give that. But in the case of brothers or sisters, one should only give them what is theirs. If they have none, then one should give one’s own temporarily and later get it back, but one should not complain if one does not get it back. It is not allowed either to make medicine for or to give it to a sister’s husband who is not related by blood; but one can give it to one’s sister saying, “Give it to your husband.” The same applies to one’s brother’s wife. But it is allowed to make it for their children since they are blood relatives.

                    \vismParagraph{III.50}{50}{}
                    \emph{8. Affliction }is any kind of illness. It is an impediment when it is actually afflicting; therefore it should be severed by treatment with medicine. But if it is not cured after taking medicine for a few days, then the ascetic’s duties should be done after apostrophizing one’s person in this way: “I am not your slave, or your hireling. I have come to suffering through maintaining you through the beginningless round of rebirths.”

                    \vismParagraph{III.51}{51}{}
                    \emph{9. Books }means responsibility for the scriptures. That is an impediment only for one who is constantly busy with recitations, etc., but not for others. Here are relevant stories. The Elder Revata, it seems, the Majjhima reciter, went to the Elder Revata, the dweller in Malaya (the Hill Country), and asked him for a meditation subject. The elder asked him, “How are you in the scriptures, friend?”—”I am studying the Majjhima [Nikāya], venerable sir.”—”The Majjhima is a hard responsibility, friend. When a man is still learning the First Fifty by heart, he is faced with the Middle Fifty; and when he is still learning that by heart, he is faced \marginnote{\textcolor{teal}{\footnotesize\{150|92\}}}{}with the Last Fifty. How can you take up a meditation subject?”—”Venerable sir, when I have taken a meditation subject from you, I shall not look at the scriptures again.” He took the meditation subject, and doing no recitation for nineteen years, he reached Arahantship in the twentieth year. He told bhikkhus who came for recitation: “I have not looked at the scriptures for twenty years, friends, \textcolor{brown}{\textit{[96]}} yet I am familiar with them. You may begin.” And from beginning to end he had no hesitation even over a single syllable.

                    \vismParagraph{III.52}{52}{}
                    The Elder Mahā-Nāga, too, who lived at Karuliyagiri (Karaliyagiri) put aside the scriptures for eighteen years, and then he recited the Dhātukathā to the bhikkhus. When they checked this with the town-dwelling elders [of Anurādhapura], not a single question was found out of its order.

                    \vismParagraph{III.53}{53}{}
                    In the Great Monastery too the Elder Tipiṭaka-Cūḷa-Abhaya had the golden drum struck, saying: “I shall expound the three Piṭakas in the circle of [experts in] the Five Collections of discourses,” and this was before he had learnt the commentaries. The Community of Bhikkhus said, “‘Which teachers’ teaching is it? Unless you give only the teaching of our own teachers we shall not let you speak.” Also his preceptor asked him when he went to wait on him, “Did you have the drum beaten, friend?”—”Yes, venerable sir.”—”For what reason?”—”I shall expound the scriptures, venerable sir.”—”Friend Abhaya, how do the teachers explain this passage?”—”They explain it in this way, venerable sir.” The elder dissented, saying “Hum.” Again three times, each time in a different way, he said, “They explain it in this way, venerable sir.” The elder always dissented, saying, “Hum.” Then he said, “Friend, your first explanation was the way of the teachers. But it is because you have not actually learnt it from the teachers’ lips that you are unable to maintain that the teachers say such and such. Go and learn it from our own teachers.”—”Where shall I go, venerable sir?”—”There is an elder named Mahā Dhammarakkhita living in the Tulādhārapabbata Monastery in the Rohaṇa country beyond the [Mahaveli] River. He knows all the scriptures. Go to him.” Saying, “Good, venerable sir,” he paid homage to the elder. He went with five hundred bhikkhus to the Elder Mahā-Dhammarakkhita, and when he had paid homage to him, he sat down. The elder asked, “Why have you come?”—”To hear the Dhamma, venerable sir.”—”Friend Abhaya, they ask me about the Dīgha and the Majjhima from time to time, but I have not looked at the others for thirty years. Still you may repeat them in my presence by night, and I shall explain them to you by day.” He said, “Good, venerable sir,” and he acted accordingly.

                    \vismParagraph{III.54}{54}{}
                    The inhabitants of the village had a large pavilion built at the door of his dwelling, and they came daily to hear the Dhamma. Explaining by day what had been repeated by night, \textcolor{brown}{\textit{[97]}} the Elder [Dhammarakkhita] eventually completed the instruction. Then he sat down on a mat on the ground before the Elder Abhaya and said, “Friend, explain a meditation subject to me.”—”What are you saying, venerable sir, have I not heard it all from you? What can I explain to you that you do not already know?” The senior elder said, “This path is different for one who has actually travelled by.”

                    \vismParagraph{III.55}{55}{}
                    The Elder Abhaya was then, it seems, a stream-enterer. When the Elder Abhaya had given his teacher a meditation subject, he returned to Anurādhapura. Later, \marginnote{\textcolor{teal}{\footnotesize\{151|93\}}}{}while he was expounding the Dhamma in the Brazen Palace, he heard that the elder had attained Nibbāna. On hearing this, he said, “Bring me [my] robe, friends.” Then he put on the robe and said, “The Arahant path befits our teacher, friends. Our teacher was a true thoroughbred. He sat down on a mat before his own Dhamma pupil and said, ‘Explain a meditation subject to me.’ The Arahant path befits our teacher, friends.”

                    For such as these, books are no impediment.

                    \vismParagraph{III.56}{56}{}
                    \emph{10. Supernormal powers }are the supernormal powers of the ordinary man. They are hard to maintain, like a prone infant or like young corn, and the slightest thing breaks them. But they are an impediment for insight, not for concentration, since they are obtainable through concentration. So the supernormal powers are an impediment that should be severed by one who seeks insight; the others are impediments to be severed by one who seeks concentration.

                    This, in the first place, is the detailed explanation of the impediments.
                \subsection[\vismAlignedParas{§57–60}Meditation subjects etc.]{Meditation subjects etc.}

                    \vismParagraph{III.57}{57}{}
                    \emph{Approach the good friend, the giver of a meditation subject }(\hyperlink{III.28}{§28}{}): meditation subjects are of two kinds, that is, generally useful meditation subjects and special meditation subjects. Herein, loving-kindness towards the Community of Bhikkhus, etc., and also mindfulness of death are what are called generally useful meditation subjects. Some say perception of foulness, too.

                    \vismParagraph{III.58}{58}{}
                    When a bhikkhu takes up a meditation subject, he should first develop loving-kindness towards the Community of Bhikkhus within the boundary,\footnote{\vismAssertFootnoteCounter{16}\vismHypertarget{III.n16}{}\emph{Sīmā—}”boundary”: loosely used in this sense, it corresponds vaguely to what is meant by “parish.” In the strict sense it is the actual area (usually a “chapter house”) agreed according to the rules laid down in the Vinaya and marked by boundary stones, within which the Community \emph{(saṅgha) }carries out its formal acts.} limiting it at first [to “all bhikkhus in this monastery”], in this way: “May they be happy and free from affliction.” Then he should develop it towards all deities within the boundary. Then towards all the principal people in the village that is his alms resort; then to [all human beings there and to] all living beings dependent on the human beings. With loving-kindness towards the Community of Bhikkhus he produces kindliness in his co-residents; then they are easy for him to live with. With loving-kindness towards the deities within the boundary he is protected by kindly deities with lawful protection. \textcolor{brown}{\textit{[98]}} With loving-kindness towards the principal people in the village that is his alms resort his requisites are protected by well-disposed principal people with lawful protection. With loving-kindness to all human beings there he goes about without incurring their dislike since they trust him. With loving-kindness to all living beings he can wander unhindered everywhere.

                    With mindfulness of death, thinking, “I have got to die,” he gives up improper search (see \textbf{\cite{S}II 194}; \textbf{\cite{M-a}I 115}), and with a growing sense of urgency he comes to live without attachment. When his mind is familiar with the perception of foulness, then even divine objects do not tempt his mind to greed.

                    \vismParagraph{III.59}{59}{}
                    \marginnote{\textcolor{teal}{\footnotesize\{152|94\}}}{}So these are called “generally useful” and they are “called meditation subjects” since they are needed\footnote{\vismAssertFootnoteCounter{17}\vismHypertarget{III.n17}{}\emph{Atthayitabba—”}needed”: not in PED, not in CPD.} generally and desirable owing to their great helpfulness and since they are subjects for the meditation work intended.

                    \vismParagraph{III.60}{60}{}
                    What is called a “special meditation subject” is that one from among the forty meditation subjects that is suitable to a man’s own temperament. It is “special” (\emph{pārihāriya}) because he must carry it (\emph{pariharitabbattā}) constantly about with him, and because it is the proximate cause for each higher stage of development.

                    So it is the one who gives this twofold meditation subject that is called the \emph{giver of a meditation subject}.
                \subsection[\vismAlignedParas{§61–73}The good friend]{The good friend}

                    \vismParagraph{III.61}{61}{}
                    \emph{The good friend }is one who possesses such special qualities as these:

                    He is revered and dearly loved, And one who speaks and suffers speech; The speech he utters is profound, He does not urge without a reason (\textbf{\cite{A}IV 32}) and so on. He is wholly solicitous of welfare and partial to progress.

                    \vismParagraph{III.62}{62}{}
                    Because of the words beginning, “Ānanda, it is owing to my being a good friend to them that living beings subject to birth are freed from birth” (\textbf{\cite{S}I 88}), it is only the Fully Enlightened One who possesses all the aspects of the good friend. Since that is so, while he is available only a meditation subject taken in the Blessed One’s presence is well taken.

                    But after his final attainment of Nibbāna, it is proper to take it from anyone of the eighty great disciples still living. When they are no more available, one who wants to take a particular meditation subject should take it from someone with cankers destroyed, who has, by means of that particular meditation subject, produced the fourfold and fivefold jhāna, and has reached the destruction of cankers by augmenting insight that had that jhāna as its proximate cause.

                    \vismParagraph{III.63}{63}{}
                    But how then, does someone with cankers destroyed declare himself thus: “I am one whose cankers are destroyed?” Why not? He declares himself when he knows that his instructions will be carried out. Did not the Elder Assagutta \textcolor{brown}{\textit{[99]}} spread out his leather mat in the air and sitting cross-legged on it explain a meditation subject to a bhikkhu who was starting his meditation subject, because he knew that that bhikkhu was one who would carry out his instructions for the meditation subject?

                    \vismParagraph{III.64}{64}{}
                    So if someone with cankers destroyed is available, that is good. If not, then one should take it from a non-returner, a once-returner, a stream-enterer, an ordinary man who has obtained jhāna, one who knows three Piṭakas, one who knows two Piṭakas, one who knows one Piṭaka, in descending order [according as available]. If not even one who knows one Piṭaka is available, then it should be taken from one who is familiar with one Collection together with its commentary and one who is himself conscientious. For a teacher such as this, who knows the texts, guards the heritage, and protects the tradition, will follow the teachers’ opinion rather than his own. Hence the Ancient Elders said three times, “One who is conscientious will guard it.”

                    \vismParagraph{III.65}{65}{}
                    \marginnote{\textcolor{teal}{\footnotesize\{153|95\}}}{}Now, those beginning with one whose cankers are destroyed, mentioned above, will describe only the path they have themselves reached. But with a learned man, his instructions and his answers to questions are purified by his having approached such and such teachers, and so he will explain a meditation subject showing a broad track, like a big elephant going through a stretch of jungle, and he will select suttas and reasons from here and there, adding [explanations of] what is suitable and unsuitable. So a meditation subject should be taken by approaching the good friend such as this, the giver of a meditation subject, and by doing all the duties to him.

                    \vismParagraph{III.66}{66}{}
                    If he is available in the same monastery, it is good. If not, one should go to where he lives.

                    When [a bhikkhu] goes to him, he should not do so with feet washed and anointed, wearing sandals, with an umbrella, surrounded by pupils, and bringing oil tube, honey, molasses, etc.; he should do so fulfilling the duties of a bhikkhu setting out on a journey, carrying his bowl and robes himself, doing all the duties in each monastery on the way, with few belongings, and living in the greatest effacement. When entering that monastery, he should do so [expecting nothing, and even provided] with a tooth-stick that he has had made allowable on the way [according to the rules]. And he should not enter some other room, thinking, “I shall go to the teacher after resting awhile and after washing and anointing my feet, and so on.”

                    \vismParagraph{III.67}{67}{}
                    Why? If there are bhikkhus there who are hostile to the teacher, they might ask him the reason for his coming and speak dispraise of the teacher, saying, “You are done for if you go to him”; \textcolor{brown}{\textit{[100]}} they might make him regret his coming and turn him back. So he should ask for the teacher’s dwelling and go straight there.

                    \vismParagraph{III.68}{68}{}
                    If the teacher is junior, he should not consent to the teacher’s receiving his bowl and robe, and so on. If the teacher is senior, then he should go and pay homage to him and remain standing. When told, “Put down the bowl and robe, friend,” he may put them down. When told, “Have some water to drink,” he can drink if he wants to. When told, “You may wash your feet,” he should not do so at once, for if the water has been brought by the teacher himself, it would be improper. But when told “Wash, friend, it was not brought by me, it was brought by others,” then he can wash his feet, sitting in a screened place out of sight of the teacher, or in the open to one side of the dwelling.

                    \vismParagraph{III.69}{69}{}
                    If the teacher brings an oil tube, he should get up and take it carefully with both hands. If he did not take it, it might make the teacher wonder, “Does this bhikkhu resent sharing so soon?” but having taken it, he should not anoint his feet at once. For if it were oil for anointing the teacher’s limbs, it would not be proper. So he should first anoint his head, then his shoulders, etc.; but when told, “This is meant for all the limbs, friend, anoint your feet,” he should put a little on his head and then anoint his feet. Then he should give it back, saying when the teacher takes it, “May I return this oil tube, venerable sir?”

                    \vismParagraph{III.70}{70}{}
                    He should not say, “Explain a meditation subject to me, venerable sir” on the very day he arrives. But starting from the next day, he can, if the teacher has a \marginnote{\textcolor{teal}{\footnotesize\{154|96\}}}{}habitual attendant, ask his permission to do the duties. If he does not allow it when asked, they can be done when the opportunity offers. When he does them, three tooth-sticks should be brought, a small, a medium and a big one, and two kinds of mouth-washing water and bathing water, that is, hot and cold, should be set out. Whichever of these the teacher uses for three days should then be brought regularly. If the teacher uses either kind indiscriminately, he can bring whatever is available.

                    \vismParagraph{III.71}{71}{}
                    Why so many words? All should be done as prescribed by the Blessed One in the Khandhakas as the right duties in the passage beginning: “Bhikkhus, a pupil should perform the duties to the teacher \textcolor{brown}{\textit{[101]}} rightly. Herein, this is the right performance of duties. He should rise early; removing his sandals and arranging his robe on one shoulder, he should give the tooth-sticks and the mouth-washing water, and he should prepare the seat. If there is rice gruel, he should wash the dish and bring the rice gruel” (\textbf{\cite{Vin}I 61}).

                    \vismParagraph{III.72}{72}{}
                    To please the teacher by perfection in the duties he should pay homage in the evening, and he should leave when dismissed with the words, “You may go.” When the teacher asks him, “Why have you come?” he can explain the reason for his coming. If he does not ask but agrees to the duties being done, then after ten days or a fortnight have gone by he should make an opportunity by staying back one day at the time of his dismissal, and announcing the reason for his coming; or he should go at an unaccustomed time, and when asked, “What have you come for?” he can announce it.

                    \vismParagraph{III.73}{73}{}
                    If the teacher says, “Come in the morning,” he should do so. But if his stomach burns with a bile affliction at that hour, or if his food does not get digested owing to sluggish digestive heat, or if some other ailment afflicts him, he should let it be known, and proposing a time that suits himself, he should come at that time. For if a meditation subject is expounded at an inconvenient time, one cannot give attention.

                    This is the detailed explanation of the words “approach the good friend, the giver of a meditation subject.”

                    
                \subsection[\vismAlignedParas{§74–102}Temperaments]{Temperaments}

                    \vismParagraph{III.74}{74}{}
                    Now, as to the words, \emph{one that suits his temperament }(\hyperlink{III.28}{§28}{}): there are six kinds of temperament, that is, greedy temperament, hating temperament, deluded temperament, faithful temperament, intelligent temperament, and speculative temperament. Some would have fourteen, taking these six single ones together with the four made up of the three double combinations and one triple combination with the greed triad and likewise with the faith triad. But if this classification is admitted, there are many more kinds of temperament possible by combining greed, etc., with faith, etc.; therefore the kinds of temperament should be understood briefly as only six. As to meaning the temperaments are one, that is to say, personal nature, idiosyncrasy. According to \textcolor{brown}{\textit{[102]}} these there are only six types of persons, that is, one of greedy temperament, one of hating temperament, one of deluded temperament, one of faithful temperament, one of intelligent temperament, and one of speculative temperament.

                    \vismParagraph{III.75}{75}{}
                    Herein, one of faithful temperament is parallel to one of greedy temperament because faith is strong when profitable [kamma] occurs in one of greedy \marginnote{\textcolor{teal}{\footnotesize\{155|97\}}}{}temperament, owing to its special qualities being near to those of greed. For, in an unprofitable way, greed is affectionate and not over-austere, and so, in a profitable way, is faith. Greed seeks out sense desires as object, while faith seeks out the special qualities of virtue and so on. And greed does not give up what is harmful, while faith does not give up what is beneficial.

                    \vismParagraph{III.76}{76}{}
                    One of intelligent temperament is parallel to one of hating temperament because understanding is strong when profitable [kamma] occurs in one of hating temperament, owing to its special qualities being near to those of hate. For, in an unprofitable way, hate is disaffected and does not hold to its object, and so, in a profitable way, is understanding. Hate seeks out only unreal faults, while understanding seeks out only real faults. And hate occurs in the mode of condemning living beings, while understanding occurs in the mode of condemning formations.

                    \vismParagraph{III.77}{77}{}
                    One of speculative temperament is parallel to one of deluded temperament because obstructive applied thoughts arise often in one of deluded temperament who is striving to arouse unarisen profitable states, owing to their special qualities being near to those of delusion. For just as delusion is restless owing to perplexity, so are applied thoughts that are due to thinking over various aspects. And just as delusion vacillates owing to superficiality, so do applied thoughts that are due to facile conjecturing.

                    \vismParagraph{III.78}{78}{}
                    Others say that there are three more kinds of temperament with craving, pride, and views. Herein craving is simply greed; and pride\footnote{\vismAssertFootnoteCounter{18}\vismHypertarget{III.n18}{}\emph{Māna, }usually rendered by “pride,” is rendered here both by “pride” and “conceit.” Etymologically it is derived perhaps from \emph{māneti }(to honour) or \emph{mināti }(to measure). In sense, however, it tends to become associated with \emph{maññati, }to conceive (false notions, see \textbf{\cite{M}I 1}), to imagine, to think (as e.g. at \textbf{\cite{Nidd}I 80}, Vibh 390 and comy.). As one of the “defilements” (see \textbf{\cite{M}I 36}) it is probably best rendered by “pride.” In the expression \emph{asmi-māna }(often rendered by “the pride that says ‘I am’”) it more nearly approaches \emph{maññanā }(false imagining, misconception, see \textbf{\cite{M}III 246}) and is better rendered by the “conceit ‘I am,’” since the word “conceit” straddles both the meanings of “pride” (i.e. haughtiness) and “conception.”} is associated with that, so neither of them exceeds greed. And since views have their source in delusion, the temperament of views falls within the deluded temperament.

                    \vismParagraph{III.79}{79}{}
                    What is the source of these temperaments? And how is it to be known that such a person is of greedy temperament, that such a person is of one of those beginning with hating temperament? What suits one of what kind of temperament?
                    \subsubsection[\vismAlignedParas{§80–86}Source of temperaments]{Source of temperaments}

                        \vismParagraph{III.80}{80}{}
                        Herein, as some say,\footnote{\vismAssertFootnoteCounter{19}\vismHypertarget{III.n19}{}“\emph{‘Some’ }is said with reference to the Elder Upatissa. For it is put in this way by him in the \emph{Vimuttimagga}. The word \emph{‘apparently’ }indicates dissent from what follows” (\textbf{\cite{Vism-mhṭ}103}). A similar passage to that referred to appears in Ch. 6 (Taisho ed. p. 410a) of the Chinese version of the \emph{Vimuttimagga}, the only one extant.} the first three kinds of temperament to begin with have their source in previous habit; and they have their source in elements and humours. Apparently one of greedy temperament has formerly had plenty of desirable tasks and gratifying work to do, or has reappeared here after dying in a heaven. And one \marginnote{\textcolor{teal}{\footnotesize\{156|98\}}}{}of hating temperament has formerly had plenty of stabbing and torturing and brutal work to do or has reappeared here after dying in one of the hells or the nāga (serpent) existences. And one \textcolor{brown}{\textit{[103]}} of deluded temperament has formerly drunk a lot of intoxicants and neglected learning and questioning, or has reappeared here after dying in the animal existence. It is in this way that they have their source in previous habit, they say.

                        \vismParagraph{III.81}{81}{}
                        Then a person is of deluded temperament because two elements are prominent, that is to say, the earth element and the water element. He is of hating temperament because the other two elements are prominent. But he is of greedy temperament because all four are equal. And as regards the humours, one of greedy temperament has phlegm in excess and one of deluded temperament has wind in excess. Or one of deluded temperament has phlegm in excess and one of greedy temperament has wind in excess. So they have their source in the elements and the humours, they say.

                        \vismParagraph{III.82}{82}{}
                        [Now, it can rightly be objected that] not all of those who have had plenty of desirable tasks and gratifying work to do, and who have reappeared here after dying in a heaven, are of greedy temperament, or the others respectively of hating and deluded temperament; and there is no such law of prominence of elements (see \hyperlink{XIV.43}{XIV.43f.}{}) as that asserted; and only the pair, greed and delusion, are given in the law of humours, and even that subsequently contradicts itself; and no source for even one among those beginning with one of faithful temperament is given. Consequently this definition is indecisive.

                        \vismParagraph{III.83}{83}{}
                        The following is the exposition according to the opinion of the teachers of the commentaries; or this is said in the “explanation of prominence”: “The fact that these beings have prominence of greed, prominence of hate, prominence of delusion, is governed by previous root-cause.

                        “For when in one man, at the moment of his accumulating [rebirth-producing] kamma, greed is strong and non-greed is weak, non-hate and non-delusion are strong and hate and delusion are weak, then his weak non-greed is unable to prevail over his greed, but his non-hate and non-delusion being strong are able to prevail over his hate and delusion. That is why, on being reborn through rebirth-linking given by that kamma, he has greed, is good-natured and unangry, and possesses understanding with knowledge like a lightning flash.

                        \vismParagraph{III.84}{84}{}
                        “When, at the moment of another’s accumulating kamma, greed and hate are strong and non-greed and non-hate weak, and non-delusion is strong and delusion weak, then in the way already stated he has both greed and hate but possesses understanding with knowledge like a lightning flash, like the Elder Datta-Abhaya.

                        “When, at the moment of his accumulating kamma, greed, non-hate and delusion are strong and the others are weak, then in the way already stated he both has greed and is dull but is good-tempered\footnote{\vismAssertFootnoteCounter{20}\vismHypertarget{III.n20}{}\emph{Sīlaka—}”good-tempered”—\emph{sukhasīla }(good-natured—see §83), which = \emph{sakhila }(kindly—\textbf{\cite{Vism-mhṭ}104}). Not in PED.} and unangry, like the Elder Bahula. \marginnote{\textcolor{teal}{\footnotesize\{157|99\}}}{}“Likewise when, at the moment of his accumulating kamma, the three, namely, greed, hate and delusion are strong and non-greed, etc., are weak, then in the way already stated he has both greed and hate and is deluded. \textcolor{brown}{\textit{[104]}}

                        \vismParagraph{III.85}{85}{}
                        “When, at the moment of his accumulating kamma, non-greed, hate and delusion are strong and the others are weak, then in the way already stated he has little defilement and is unshakable even on seeing a heavenly object, but he has hate and is slow in understanding.

                        “When, at the moment of his accumulating kamma, non-greed, non-hate and non-delusion are strong and the rest weak, then in the way already stated he has no greed and no hate, and is good-tempered but slow in understanding.

                        “Likewise when, at the moment of his accumulating kamma, non-greed, hate and non-delusion are strong and the rest weak, then in the way already stated he both has no greed and possesses understanding but has hate and is irascible.

                        “Likewise when, at the moment of his accumulating kamma, the three, that is, non-hate, non-greed, and non-delusion, are strong and greed, etc., are weak, then in the way already stated he has no greed and no hate and possesses understanding, like the Elder Mahā-Saṅgharakkhita.”

                        \vismParagraph{III.86}{86}{}
                        One who, as it is said here, “has greed” is one of greedy temperament; one who “has hate” and one who “is dull” are respectively of hating temperament and deluded temperament. One who “possesses understanding” is one of intelligent temperament. One who “has no greed” and one who “has no hate” are of faithful temperament because they are naturally trustful. Or just as one who is reborn through kamma accompanied by non-delusion is of intelligent temperament, so one who is reborn through kamma accompanied by strong faith is of faithful temperament, one who is reborn through kamma accompanied by thoughts of sense desire is of speculative temperament, and one who is reborn through kamma accompanied by mixed greed, etc., is of mixed temperament. So it is the kamma productive of rebirth-linking and accompanied by someone among the things beginning with greed that should be understood as the source of the temperaments.
                    \subsubsection[\vismAlignedParas{§87–96}Recognition of temperament]{Recognition of temperament}

                        \vismParagraph{III.87}{87}{}
                        But it was asked, \emph{and how is it to be known that “This person is of greedy temperament?” }(\hyperlink{III.79}{§79}{}), and so on. This is explained as follows:
                        \begin{verse}
                            By the posture, by the action,\\{}
                            By eating, seeing, and so on,\\{}
                            By the kind of states occurring,                                
May temperament be recognized.
                        \end{verse}


                        \vismParagraph{III.88}{88}{}
                        Herein, \emph{by the posture}: when one of greedy temperament is walking in his usual manner, he walks carefully, puts his foot down slowly, puts it down evenly, lifts it up evenly, and his step is springy.\footnote{\vismAssertFootnoteCounter{21}\vismHypertarget{III.n21}{}\emph{Ukkuṭika—”}springy” is glossed here by \emph{asamphuṭṭhamajjhaṃ }(“not touching in the middle”—\textbf{\cite{Vism-mhṭ}106}). This meaning is not in PED.}

                        One of hating temperament walks as though he were digging with the points of his feet, puts his foot down quickly, lifts it up quickly, and his step is dragged along.

                        \marginnote{\textcolor{teal}{\footnotesize\{158|100\}}}{}One of deluded temperament walks with a perplexed gait, puts his foot down hesitantly, lifts it up hesitantly, \textcolor{brown}{\textit{[105]}} and his step is pressed down suddenly.

                        And this is said in the account of the origin of the Māgandiya Sutta:
                        \begin{verse}
                            The step of one of greedy nature will be springy;\\{}
                            The step of one of hating nature, dragged along;\\{}
                            Deluded, he will suddenly press down his step;\\{}
                            And one without defilement has a step like this.\footnote{\vismAssertFootnoteCounter{22}\vismHypertarget{III.n22}{}See \textbf{\cite{Sn-a}544}, \textbf{\cite{A-a}} 436.}
                        \end{verse}


                        \vismParagraph{III.89}{89}{}
                        The stance of one of greedy temperament is confident and graceful. That of one of hating temperament is rigid. That of one of deluded temperament is muddled, likewise in sitting. And one of greedy temperament spreads his bed unhurriedly, lies down slowly, composing his limbs, and he sleeps in a confident manner. When woken, instead of getting up quickly, he gives his answer slowly as though doubtful. One of hating temperament spreads his bed hastily anyhow; with his body flung down he sleeps with a scowl. When woken, he gets up quickly and answers as though annoyed. One of deluded temperament spreads his bed all awry and sleeps mostly face downwards with his body sprawling. When woken, he gets up slowly, saying, “Hum.”

                        \vismParagraph{III.90}{90}{}
                        Since those of faithful temperament, etc., are parallel to those of greedy temperament, etc., their postures are therefore like those described above.

                        This firstly is how the temperaments may be recognized by the posture.

                        \vismParagraph{III.91}{91}{}
                        \emph{By the action}: also in the acts of sweeping, etc., one of greedy temperament grasps the broom well, and he sweeps cleanly and evenly without hurrying or scattering the sand, as if he were strewing \emph{sinduvāra }flowers. One of hating temperament grasps the broom tightly, and he sweeps uncleanly and unevenly with a harsh noise, hurriedly throwing up the sand on each side. One of deluded temperament grasps the broom loosely, and he sweeps neither cleanly nor evenly, mixing the sand up and turning it over.

                        \vismParagraph{III.92}{92}{}
                        As with sweeping, so too with any action such as washing and dyeing robes, and so on. One of greedy temperament acts skilfully, gently, evenly and carefully. One of hating temperament acts tensely, stiffly and unevenly. One of deluded temperament acts unskilfully as if muddled, unevenly and indecisively. \textcolor{brown}{\textit{[106]}}

                        Also one of greedy temperament wears his robe neither too tightly nor too loosely, confidently and level all round. One of hating temperament wears it too tight and not level all round. One of deluded temperament wears it loosely and in a muddled way.

                        Those of faithful temperament, etc., should be understood in the same way as those just described, since they are parallel.

                        This is how the temperaments may be recognized by the actions.

                        \vismParagraph{III.93}{93}{}
                        \emph{By eating}: One of greedy temperament likes eating rich sweet food. When eating, he makes a round lump not too big and eats unhurriedly, savouring the various tastes. He enjoys getting something good. One of hating temperament likes eating rough sour food. When eating he makes a lump that fills his mouth, and he \marginnote{\textcolor{teal}{\footnotesize\{159|101\}}}{}eats hurriedly without savouring the taste. He is aggrieved when he gets something not good. One of deluded temperament has no settled choice. When eating, he makes a small un-rounded lump, and as he eats he drops bits into his dish, smearing his face, with his mind astray, thinking of this and that.

                        Also those of faithful temperament, etc., should be understood in the same way as those just described since they are parallel.

                        This is how the temperament may be recognized by eating.

                        \vismParagraph{III.94}{94}{}
                        And \emph{by seeing and so on}: when one of greedy temperament sees even a slightly pleasing visible object, he looks long as if surprised, he seizes on trivial virtues, discounts genuine faults, and when departing, he does so with regret as if unwilling to leave. When one of hating temperament sees even a slightly unpleasing visible object, he avoids looking long as if he were tired, he picks out trivial faults, discounts genuine virtues, and when departing, he does so without regret as if anxious to leave. When one of deluded temperament sees any sort of visible object, he copies what others do: if he hears others criticizing, he criticizes; if he hears others praising, he praises; but actually he feels equanimity in himself—the equanimity of unknowing. So too with sounds, and so on.

                        And those of faithful temperament, etc., should be understood in the same way as those just described since they are parallel.

                        This is how the temperaments may be recognized by seeing and so on.

                        \vismParagraph{III.95}{95}{}
                        \emph{By the kind of states occurring}: in one of greedy temperament there is frequent occurrence of such states as deceit, fraud, pride, evilness of wishes, greatness of wishes, discontent, foppery and personal vanity.\footnote{\vismAssertFootnoteCounter{23}\vismHypertarget{III.n23}{}\emph{Siṅga—}”foppery” is not in PED in this sense. See \textbf{\cite{Vibh}351} and commentary.

                                \emph{Cāpalya (cāpalla)—”}personal vanity”: noun from adj. \emph{capala. }The word \emph{“capala” }comes in an often-repeated passage: \emph{“saṭhā māyāvino keṭubhino uddhatā unnalā capalā mukharā …” }(\textbf{\cite{M}I 32}); cf. \textbf{\cite{S}I 203}; \textbf{\cite{A}III 199}, etc.) and also \textbf{\cite{M}I 470} \emph{“uddhato hoti capalo,” }with two lines lower \emph{“uddhaccaṃ cāpalyaṃ.” Cāpalya }also occurs at Vibh 351 (and \textbf{\cite{M}II 167}). At M-a I 152 (commenting on \textbf{\cite{M}I 32}) we find: \emph{capalā ti pattacīvaramaṇḍanādinā cāpallena yuttā }(“interested in personal vanity consisting in adorning bowl and robe and so on”), and at \textbf{\cite{M-a}III 185} (commenting on \textbf{\cite{M}I 470}): \emph{Uddhato hoti capalo ti uddhaccapakatiko c’eva hoti cīvaramaṇḍanā pattamaṇḍanā senāsanamaṇḍanā imassa vā pūtikāyassa kelāyanamaṇḍanā ti evaṃ vuttena taruṇadārakacāpallena samannāgato }(“‘he is distracted—or puffed up—and personally vain’: he is possessed of the callow youth’s personal vanity described as adorning the robe, adorning the bowl, adorning the lodging, or prizing and adorning this filthy body”). This meaning is confirmed in the commentary to \textbf{\cite{Vibh}251}. PED does not give this meaning at all but only “fickle,” which is unsupported by the commentary. CPD \emph{(acapala) }also does not give this meaning.

                                As to the other things listed here in the \emph{Visuddhimagga} text, most will be found at M\textbf{\cite{M}I 36}. For “holding on tenaciously,” etc., see \textbf{\cite{M}I 43}.} \textcolor{brown}{\textit{[107]}} In one of hating temperament there is frequent occurrence of such states as anger, enmity, disparaging, domineering, envy and avarice. In one of deluded temperament there is frequent occurrence of such states as stiffness, torpor, agitation, worry, uncertainty, and holding on tenaciously with refusal to relinquish. \marginnote{\textcolor{teal}{\footnotesize\{160|102\}}}{}In one of faithful temperament there is frequent occurrence of such states as free generosity, desire to see Noble Ones, desire to hear the Good Dhamma, great gladness, ingenuousness, honesty, and trust in things that inspire trust. In one of intelligent temperament there is frequent occurrence of such states as readiness to be spoken to, possession of good friends, knowledge of the right amount in eating, mindfulness and full awareness, devotion to wakefulness, a sense of urgency about things that should inspire a sense of urgency, and wisely directed endeavour. In one of speculative temperament there is frequent occurrence of such states as talkativeness, sociability, boredom with devotion to the profitable, failure to finish undertakings, smoking by night and flaming by day (see \textbf{\cite{M}I 144}—that is to say, hatching plans at night and putting them into effect by day), and mental running hither and thither (see \textbf{\cite{Ud}37}).

                        This is how the temperaments may be recognized by the kind of states occurring.

                        \vismParagraph{III.96}{96}{}
                        However, these directions for recognizing the temperaments have not been handed down in their entirety in either the texts or the commentaries; they are only expressed according to the opinion of the teachers and cannot therefore be treated as authentic. For even those of hating temperament can exhibit postures, etc., ascribed to the greedy temperament when they try diligently. And postures, etc., never arise with distinct characteristics in a person of mixed temperament. Only such directions for recognizing temperament as are given in the commentaries should be treated as authentic; for this is said: “A teacher who has acquired penetration of minds will know the temperament and will explain a meditation subject accordingly; one who has not should question the pupil.” So it is by penetration of minds or by questioning the person, that it can be known whether he is one of greedy temperament or one of those beginning with hating temperament.
                    \subsubsection[\vismAlignedParas{§97–102}What suits one of what kind of temperament?]{What suits one of what kind of temperament?}

                        \vismParagraph{III.97}{97}{}
                        \emph{What suits one of what kind of temperament}? (\hyperlink{III.79}{§79}{}). A suitable lodging for one of greedy temperament has an unwashed sill and stands level with the ground, and it can be either an overhanging [rock with an] unprepared [drip-ledge] (see \hyperlink{II.n15}{Ch. II, note 15}{}), a grass hut, or a leaf house, etc. It ought to be spattered with dirt, full of bats,\footnote{\vismAssertFootnoteCounter{24}\vismHypertarget{III.n24}{}\emph{Jatukā—}”a bat”: not in PED. Also at \hyperlink{XI}{Ch. XI}{}. §7.} dilapidated, too high or too low, in bleak surroundings, threatened [by lions, tigers, etc.,] with a muddy, uneven path, \textcolor{brown}{\textit{[108]}} where even the bed and chair are full of bugs. And it should be ugly and unsightly, exciting loathing as soon as looked at. Suitable inner and outer garments are those that have torn-off edges with threads hanging down all round like a “net cake,”\footnote{\vismAssertFootnoteCounter{25}\vismHypertarget{III.n25}{}\emph{Jalapūvasadisa—}”like a net cake”: “A cake made like a net” (\textbf{\cite{Vism-mhṭ}108}); possibly what is now known in Sri Lanka as a “string hopper,” or something like it.} harsh to the touch like hemp, soiled, heavy and hard to wear. And the right kind of bowl for him is an ugly clay bowl disfigured by stoppings and joints, or a heavy and misshapen iron bowl as unappetizing as a skull. The right kind of road for him on which to wander for alms is disagreeable, with no village near, and uneven. The right kind of village for him in which to wander for alms is where people wander about as if oblivious of him, where, as he is about to leave without getting alms even from a single family, people call him into the sitting hall, saying, “Come, venerable sir,” and give him \marginnote{\textcolor{teal}{\footnotesize\{161|103\}}}{}gruel and rice, but do so as casually as if they were putting a cow in a pen. Suitable people to serve him are slaves or workmen who are unsightly, ill-favoured, with dirty clothes, ill-smelling and disgusting, who serve him his gruel and rice as if they were throwing it rudely at him. The right kind of gruel and rice and hard food is poor, unsightly, made up of millet, \emph{kudusaka}, broken rice, etc., stale buttermilk, sour gruel, curry of old vegetables, or anything at all that is merely for filling the stomach. The right kind of posture for him is either standing or walking. The object of his contemplation should be any of the colour kasiṇas, beginning with the blue, whose colour is not pure. This is what suits one of greedy temperament.

                        \vismParagraph{III.98}{98}{}
                        A suitable resting place for one of hating temperament is not too high or too low, provided with shade and water, with well-proportioned walls, posts and steps, with well-prepared frieze work and lattice work, brightened with various kinds of painting, with an even, smooth, soft floor, adorned with festoons of flowers and a canopy of many-coloured cloth like a Brahmā-god’s divine palace, with bed and chair covered with well-spread clean pretty covers, smelling sweetly of flowers, and perfumes and scents set about for homely comfort, which makes one happy and glad at the mere sight of it.

                        \vismParagraph{III.99}{99}{}
                        The right kind of road to his lodging is free from any sort of danger, traverses clean, even ground, and has been properly prepared. \textcolor{brown}{\textit{[109]}} And here it is best that the lodging’s furnishings are not too many in order to avoid hiding-places for insects, bugs, snakes and rats: even a single bed and chair only. The right kind of inner and outer garments for him are of any superior stuff such as China cloth, Somāra cloth, silk, fine cotton, fine linen, of either single or double thickness, quite light, and well dyed, quite pure in colour to befit an ascetic. The right kind of bowl is made of iron, as well shaped as a water bubble, as polished as a gem, spotless, and of quite pure colour to befit an ascetic. The right kind of road on which to wander for alms is free from dangers, level, agreeable, with the village neither too far nor too near. The right kind of village in which to wander for alms is where people, thinking, “Now our lord is coming,” prepare a seat in a sprinkled, swept place, and going out to meet him, take his bowl, lead him to the house, seat him on a prepared seat and serve him carefully with their own hands.

                        \vismParagraph{III.100}{100}{}
                        Suitable people to serve him are handsome, pleasing, well bathed, well anointed, scented\footnote{\vismAssertFootnoteCounter{26}\vismHypertarget{III.n26}{}\emph{Surabhi—}”scented, perfume”: not in PED; also at \hyperlink{VI.90}{VI.90}{}; \hyperlink{X.60}{X.60}{} and \textbf{\cite{Vism-mhṭ}445}.} with the perfume of incense and the smell of flowers, adorned with apparel made of variously-dyed clean pretty cloth, who do their work carefully. The right kind of gruel, rice, and hard food has colour, smell and taste, possesses nutritive essence, and is inviting, superior in every way, and enough for his wants. The right kind of posture for him is lying down or sitting. The object of his contemplation should be anyone of the colour kasiṇas, beginning with the blue, whose colour is quite pure. This is what suits one of hating temperament.

                        \vismParagraph{III.101}{101}{}
                        The right lodging for one of deluded temperament has a view and is not shut in, where the four quarters are visible to him as he sits there. As to the postures, walking is right. The right kind of object for his contemplation is not small, that is to say, the size of a winnowing basket or the size of a saucer; for his mind becomes more confused \marginnote{\textcolor{teal}{\footnotesize\{162|104\}}}{}in a confined space; so the right kind is an amply large kasiṇa. The rest is as stated for one of hating temperament. This is what suits one of deluded temperament.

                        \vismParagraph{III.102}{102}{}
                        For one of faithful temperament all the directions given for one of hating temperament are suitable. As to the object of his contemplation, one of the recollections is right as well.

                        For one of intelligent temperament there is nothing unsuitable as far as concerns the lodging and so on.

                        For one of speculative temperament an open lodging with a view, \textcolor{brown}{\textit{[110]}} where gardens, groves and ponds, pleasant prospects, panoramas of villages, towns and countryside, and the blue gleam of mountains, are visible to him as he sits there, is not right; for that is a condition for the running hither and thither of applied thought. So he should live in a lodging such as a deep cavern screened by woods like the Overhanging Rock of the Elephant’s Belly (\emph{Hatthikucchipabbhāra}), or Mahinda’s Cave. Also an ample-sized object of contemplation is not suitable for him; for one like that is a condition for the running hither and thither of applied thought. A small one is right. The rest is as stated for one of greedy temperament. This is what suits one of speculative temperament.

                        These are the details, with definition of the kind, source, recognition, and what is suitable, as regards the various temperaments handed down here with the words “that suits his own temperament” (\hyperlink{III.60}{§60}{}).
                \subsection[\vismAlignedParas{§103–122}Definition of meditation subjects]{Definition of meditation subjects}

                    \vismParagraph{III.103}{103}{}
                    However, the meditation subject that is suitable to the temperament has not been cleared up in all its aspects yet. This will become clear automatically when those in the following list are treated in detail.

                    Now, it was said above, “and he should apprehend from among the forty meditation subjects one that suits his own temperament” (\hyperlink{III.60}{§60}{}). Here the exposition of the meditation subject should be first understood in these ten ways: (1) as to enumeration, (2) as to which bring only access and which absorption, (3) at to the kinds of jhāna, (4) as to surmounting, (5) as to extension and non-extension, (6) as to object, (7) as to plane, (8) as to apprehending, (9) as to condition, (10) as to suitability to temperament.

                    \vismParagraph{III.104}{104}{}
                    \emph{1. }Herein, as to \emph{enumeration}: it was said above, “from among the forty meditation subjects” (\hyperlink{III.28}{§28}{}). Herein, the forty meditation subjects are these:
                    \begin{verse}
                        ten kasiṇas (totalities),\\{}
                        ten kinds of foulness,\\{}
                        ten recollections,\\{}
                        four divine abidings,\\{}
                        four immaterial states,\\{}
                        one perception,\\{}
                        one defining.
                    \end{verse}


                    \vismParagraph{III.105}{105}{}
                    Herein, the ten kasiṇas are these: earth kasiṇa, water kasiṇa, fire kasiṇa, air kasiṇa, blue kasiṇa, yellow kasiṇa, red kasiṇa, white kasiṇa, light kasiṇa, and limited-space kasiṇa.\footnote{\vismAssertFootnoteCounter{27}\vismHypertarget{III.n27}{}“\emph{‘Kasiṇa’ }is in the sense of entirety \emph{(sakalaṭṭhena)” }(\textbf{\cite{M-a}III 260}). See \hyperlink{IV.119}{IV.119}{}.} \marginnote{\textcolor{teal}{\footnotesize\{163|105\}}}{}The ten kinds of foulness are these: the bloated, the livid, the festering, the cut-up, the gnawed, the scattered, the hacked and scattered, the bleeding, the worm-infested, and a skeleton.\footnote{\vismAssertFootnoteCounter{28}\vismHypertarget{III.n28}{}Here ten kinds of foulness are given. But in the Suttas only either five or six of this set appear to be mentioned, that is, “Perception of a skeleton, perception of the worm-infested, perception of the livid, perception of the cut-up, perception of the bloated. (see \textbf{\cite{A}I 42} and \textbf{\cite{S}V 131}; \textbf{\cite{A}II 17} adds “perception of the festering”)” No details are given. All ten appear at Dhs 263–64 and \textbf{\cite{Paṭis}I 49}. It will be noted that no order of progress of decay in the kinds of corpse appears here; also the instructions in \hyperlink{VI}{Ch. VI}{} are for contemplating actual corpses in these states. The primary purpose here is to cultivate “repulsiveness.”

                            Another set of nine progressive stages in the decay of a corpse, mostly different from these, is given at \textbf{\cite{M}I 58}, 89, etc., beginning with a corpse one day old and ending with bones turned to dust. From the words “suppose a bhikkhu saw a corpse thrown on a charnel ground … he compares this same body of his with it thus, ‘This body too is of like nature, awaits a like fate, is not exempt from that’”(\textbf{\cite{M}I 58}), it can be assumed that these nine, which are given in progressive order of decay in order to demonstrate the body’s impermanence, are not necessarily intended as contemplations of actual corpses so much as mental images to be created, the primary purpose being to cultivate impermanence. This may be why these nine are not used here (see \hyperlink{VIII.43}{VIII.43}{}).

                            The word \emph{asubha }(foul, foulness) is used both of the contemplations of corpses as here and of the contemplation of the parts of the body (\textbf{\cite{A}V 109}).}

                    The ten kinds of recollection are these: recollection of the Buddha (the Enlightened One), recollection of the Dhamma (the Law), recollection of the Sangha (the Community), recollection of virtue, recollection of generosity, recollection of deities, recollection (or mindfulness) of death, mindfulness occupied with the body, mindfulness of breathing, and recollection of peace. \textcolor{brown}{\textit{[111]}}

                    The four divine abidings are these: loving-kindness, compassion, gladness, and equanimity.

                    The four immaterial states are these: the base consisting of boundless space, the base consisting of boundless consciousness, the base consisting of nothingness, and the base consisting of neither perception nor non-perception.

                    The one perception is the perception of repulsiveness in nutriment.

                    The one defining is the defining of the four elements.

                    This is how the exposition should be understood “as to enumeration.”

                    \vismParagraph{III.106}{106}{}
                    \emph{2 As to which bring access only and which absorption}: the eight recollections—excepting mindfulness occupied with the body and mindfulness of breathing—the perception of repulsiveness in nutriment, and the defining of the four elements, are ten meditation subjects that bring access only. The others bring absorption. This is “as to which bring access only and which absorption.”

                    \vismParagraph{III.107}{107}{}
                    \emph{3. As to the kind of jhāna}: among those that bring absorption, the ten kasiṇas together with mindfulness of breathing bring all four jhānas. The ten kinds of foulness together with mindfulness occupied with the body bring the first jhāna. \marginnote{\textcolor{teal}{\footnotesize\{164|106\}}}{}The first three divine abidings bring three jhānas. The fourth divine abiding and the four immaterial states bring the fourth jhāna. This is “as to the kind of jhāna.”

                    \vismParagraph{III.108}{108}{}
                    \emph{4. As to surmounting}: there are two kinds of surmounting, that is to say, surmounting of factors and surmounting of object. Herein, there is surmounting of factors in the case of all meditation subjects that bring three and four jhānas because the second jhāna, etc., have to be reached in those same objects by surmounting the jhāna factors of applied thought and sustained thought, and so on. Likewise in the case of the fourth divine abiding; for that has to be reached by surmounting joy in the same object as that of loving-kindness, and so on. But in the case of the four immaterial states there is surmounting of the object; for the base consisting of boundless space has to be reached by surmounting one or other of the first nine kasiṇas, and the base consisting of boundless consciousness, etc., have respectively to be reached by surmounting space, and so on. With the rest there is no surmounting. This is “as to surmounting.”

                    \vismParagraph{III.109}{109}{}
                    \emph{5. As to extension and non-extension}: only the ten kasiṇas among these forty meditation subjects need be extended. For it is within just so much space as one is intent upon with the kasiṇa that one can hear sounds with the divine ear element, see visible objects with the divine eye, and know the minds of other beings with the mind.

                    \vismParagraph{III.110}{110}{}
                    Mindfulness occupied with the body and the ten kinds of foulness need not be extended. Why? Because they have a definite location and because there is no benefit in it. The definiteness of their location will become clear in explaining the method of development (\hyperlink{VIII.83}{VIII.83}{}–\hyperlink{VIII.138}{138}{} and \hyperlink{VI.40}{VI.40}{}, \hyperlink{VI.41}{41}{}, \hyperlink{VI.79}{79}{}). If the latter are extended, it is only a quantity of corpses that is extended \textcolor{brown}{\textit{[112]}} and there is no benefit. And this is said in answer to the question of Sopāka: “Perception of visible forms is quite clear, Blessed One, perception of bones is not clear” (Source untraced\footnote{\vismAssertFootnoteCounter{29}\vismHypertarget{III.n29}{}Also quoted in \textbf{\cite{A-a}V 79} on AN 11:9. Cf. \textbf{\cite{Sn}1119}. A similar quotation with Sopāka is found in \textbf{\cite{Vism-mhṭ}334–335}, see note 1 to \hyperlink{XI.2}{XI.2}{}.}); for here the perception of visible forms is called “quite clear” in the sense of extension of the sign, while the perception of bones is called “not quite clear” in the sense of its non-extension.

                    \vismParagraph{III.111}{111}{}
                    But the words “I was intent upon this whole earth with the perception of a skeleton” (\textbf{\cite{Th}18}) are said of the manner of appearance to one who has acquired that perception. For just as in [the Emperor] Dhammāsoka’s time the \emph{Karavīka }bird uttered a sweet song when it saw its own reflection in the looking glass walls all round and perceived \emph{Karavīkas} in every direction,\footnote{\vismAssertFootnoteCounter{30}\vismHypertarget{III.n30}{}The full story, which occurs at \textbf{\cite{M-a}III 382–383} and elsewhere, is this: “It seems that when the Karavīka bird has pecked a sweet-flavoured mango wth its beak and savoured the dripping juice, and flapping its wings, begins to sing, then quadrupeds caper as if mad. Quadrupeds grazing in their pastures drop the grass in their mouths and listen to the sound. Beasts of prey hunting small animals pause with one foot raised. Hunted animals lose their fear of death and halt in their tracks. Birds flying in the air stay with wings outstretched. Fishes in the water keep still, not moving their fins. All listen to the sound, so beautiful is the Karavīka’s song. Dhammāsoka’s queen Asandhamittā asked the Community: ‘Venerable sirs, is there anything that sounds like the Buddha?’—‘The Karavīka birds does.’—‘Where are those birds, venerable sirs?’—‘In the Himalaya.’

                            She told the king: ‘Sire, I wish to hear a Karavīka bird.’ The king dispatched a gold cage with the order, ‘Let a Karavīka bird come and sit in this cage.’ The cage travelled and halted in front of a Karavīka. Thinking, ‘The cage has come at the king’s command; it is impossible not to go,’ the bird got in. The cage returned and stopped before the king. They could not get the Karavīka to utter a sound. When the king asked, ‘When do they utter a sound?’ they replied, ‘On seeing their kin.’ Then the king had it surrounded with looking-glasses. Seeing its own reflection and imagining that its relatives had come, it flapped its wings and cried out with an exquisite voice as if sounding a crystal trumpet. All the people in the city rushed about as if mad. Asandhamittā thought: ‘If the sound of this creature is so fine, what indeed can the sound of the Blessed One have been like since he had reached the glory of omniscient knowledge?’ and arousing a happiness that she never again relinquished, she became established in the fruition of stream-entry.”} so the Elder [Siṅgāla Pitar] \marginnote{\textcolor{teal}{\footnotesize\{165|107\}}}{}thought, when he saw the sign appearing in all directions through his acquisition of the perception of a skeleton, that the whole earth was covered with bones.

                    \vismParagraph{III.112}{112}{}
                    If that is so, then is what is called “the measurelessness of the object of jhāna produced on foulness”\footnote{\vismAssertFootnoteCounter{31}\vismHypertarget{III.n31}{}See \textbf{\cite{Dhs}55}; but it comes under the “… \emph{pe }…,” which must be filled in from pp. 37–38, §182 and §184.} contradicted? It is not contradicted. For one man apprehends the sign in a large bloated corpse or skeleton, another in a small one. In this way the jhāna of the one has a limited object and of the other a measureless object. Or alternatively, “With a measureless object” (\textbf{\cite{Dhs}182–84} in elision) is said of it referring to one who extends it, seeing no disadvantage in doing so. But it need not be extended because no benefit results.

                    \vismParagraph{III.113}{113}{}
                    The rest need not be extended likewise. Why? When a man extends the sign of in-breaths and out-breaths, only a quantity of wind is extended, and it has a definite location, [the nose-tip]. So it need not be extended because of the disadvantage and because of the definiteness of the location. And the divine abidings have living beings as their object. When a man extends the sign of these, only the quantity of living beings would be extended, and there is no purpose in that. So that also need not be extended.

                    \vismParagraph{III.114}{114}{}
                    When it is said, “Intent upon one quarter with his heart endued with loving-kindness” (\textbf{\cite{D}I 250}), etc., that is said for the sake of comprehensive inclusion. For it is when a man develops it progressively by including living beings in one direction by one house, by two houses, etc., that he is said to be “intent upon one direction,” \textcolor{brown}{\textit{[113]}} not when he extends the sign. And there is no counterpart sign here that he might extend. Also the state of having a limited or measureless object can be understood here according to the way of inclusion, too.

                    \vismParagraph{III.115}{115}{}
                    As regards the immaterial states as object, space need not be extended since it is the mere removal of the kasiṇa [materiality]; for that should be brought to mind only as the disappearance of the kasiṇa [materiality]; if he extends it, nothing further happens. And consciousness need not be extended since it is a state consisting in an individual essence, and it is not possible to extend a state consisting in an individual essence. The disappearance of consciousness need not be extended since it is mere non-existence of consciousness. And the base consisting of neither \marginnote{\textcolor{teal}{\footnotesize\{166|108\}}}{}perception nor non-perception as object need not be extended since it too is a state consisting in an individual essence.\footnote{\vismAssertFootnoteCounter{32}\vismHypertarget{III.n32}{}“It is because only an abstract \emph{(parikappaja) }object can be extended, not any other kind, that he said, \emph{‘it is not possible to extend a state consisting in an individual essence’” }(\textbf{\cite{Vism-mhṭ}110}).}

                    \vismParagraph{III.116}{116}{}
                    The rest need not be extended because they have no sign. For it is the counterpart sign\footnote{\vismAssertFootnoteCounter{33}\vismHypertarget{III.n33}{}The word \emph{“nimitta” }in its technical sense is consistently rendered here by the word “sign,” which corresponds very nearly if not exactly to most uses of it. It is sometimes rendered by “mark” (which over-emphasizes the concrete), and by “image” (which is not always intended). The three kinds, that is, the preliminary-work sign, learning sign and counterpart sign, do not appear in the Piṭakas. There the use rather suggests association of ideas as, for example, at \textbf{\cite{M}I 180}, \textbf{\cite{M}I 119}, \textbf{\cite{A}I 4}, etc., than the more definitely visualized “image” in some instances of the “counterpart sign” described in the following chapters.} that would be extendable, and the object of the recollection of the Buddha, etc., is not a counterpart sign. Consequently there is no need for extension there.

                    This is “as to extension and non-extension.”

                    \vismParagraph{III.117}{117}{}
                    \emph{6. As to object}: of these forty meditation subjects, twenty-two have counterpart signs as object, that is to say, the ten kasiṇas, the ten kinds of foulness, mindfulness of breathing, and mindfulness occupied with the body; the rest do not have counterpart signs as object. Then twelve have states consisting in individual essences as object, that is to say, eight of the ten recollections—except mindfulness of breathing and mindfulness occupied with the body—the perception of repulsiveness in nutriment, the defining of the elements, the base consisting of boundless consciousness, and the base consisting of neither perception nor non-perception; and twenty-two have [counterpart] signs as object, that is to say, the ten kasiṇas, the ten kinds of foulness, mindfulness of breathing, and mindfulness occupied with the body; while the remaining six have “not-so-classifiable”\footnote{\vismAssertFootnoteCounter{34}\vismHypertarget{III.n34}{}\emph{Na-vattabba—}”not so-classifiable” is an Abhidhamma shorthand term for something that, when considered under one of the triads or dyads of the \emph{Abhidhamma Mātikā} (\textbf{\cite{Dhs}1f.}), cannot be placed under any one of the three, or two, headings.} objects. Then eight have mobile objects in the early stage though the counterpart is stationary, that is to say, the festering, the bleeding, the worm-infested, mindfulness of breathing, the water kasiṇa, the fire kasiṇa, the air kasiṇa, and in the case of the light kasiṇa the object consisting of a circle of sunlight, etc.; the rest have immobile objects.\footnote{\vismAssertFootnoteCounter{35}\vismHypertarget{III.n35}{}“‘\emph{The festering’ }is a mobile object because of the oozing of the pus, \emph{‘the bleeding’ }because of the trickling of the blood, \emph{‘the worm-infested’ }because of the wriggling of the worms. The mobile aspect of the sunshine coming in through a window opening is evident, which explains why an object consisting of a circle of sunlight is called mobile” (\textbf{\cite{Vism-mhṭ}110}).} This is “as to object.”

                    \vismParagraph{III.118}{118}{}
                    \emph{7. As to plane}: here the twelve, namely, the ten kinds of foulness, mindfulness occupied with the body, and perception of repulsiveness in nutriment, do not occur among deities. These twelve and mindfulness of breathing do not occur in the \marginnote{\textcolor{teal}{\footnotesize\{167|109\}}}{}Brahmā-world. But none except the four immaterial states occur in the immaterial becoming. All occur among human beings. This is “as to plane.” \textcolor{brown}{\textit{[114]}}

                    \vismParagraph{III.119}{119}{}
                    \emph{8. As to apprehending}: here the exposition should be understood according to the seen, the touched and the heard. Herein, these nineteen, that is to say, nine kasiṇas omitting the air kasiṇa and the ten kinds of foulness, must be apprehended by the seen. The meaning is that in the early stage their sign must be apprehended by constantly looking with the eye. In the case of mindfulness occupied with the body the five parts ending with skin must be apprehended by the seen and the rest by the heard, so its object must be apprehended by the seen and the heard. Mindfulness of breathing must be apprehended by the touched; the air kasiṇa by the seen and the touched; the remaining eighteen by the heard. The divine abiding of equanimity and the four immaterial states are not apprehendable by a beginner; but the remaining thirty-five are. This is “as to apprehending.”

                    \vismParagraph{III.120}{120}{}
                    \emph{9. As to condition}: of these meditation subjects nine kasiṇas omitting the space kasiṇa are conditions for the immaterial states. The ten kasiṇas are conditions for the kinds of direct-knowledge. Three divine abidings are conditions for the fourth divine abiding. Each lower immaterial state is a condition for each higher one. The base consisting of neither perception nor non-perception is a condition for the attainment of cessation. All are conditions for living in bliss, for insight, and for the fortunate kinds of becoming. This is “as to condition.”

                    \vismParagraph{III.121}{121}{}
                    \emph{10. As to suitability to temperament}: here the exposition should be understood according to what is suitable to the temperaments. That is to say: first, the ten kinds of foulness and mindfulness occupied with the body are eleven meditation subjects suitable for one of greedy temperament. The four divine abidings and four colour kasiṇas are eight suitable for one of hating temperament. Mindfulness of breathing is the one [recollection as a] meditation subject suitable for one of deluded temperament and for one of speculative temperament. The first six recollections are suitable for one of faithful temperament. Mindfulness of death, the recollection of peace, the defining of the four elements, and the perception of repulsiveness in nutriment, are four suitable for one of intelligent temperament. The remaining kasiṇas and the immaterial states are suitable for all kinds of temperament. And anyone of the kasiṇas should be limited for one of speculative temperament and measureless for one of deluded temperament. This is how the exposition should be understood here “as to suitability to temperament.”

                    \vismParagraph{III.122}{122}{}
                    All this has been stated in the form of direct opposition and complete suitability. But there is actually no profitable development that does not suppress greed, etc., and help faith, and so on. And this is said in the Meghiya Sutta: “[One] should, in addition,\footnote{\vismAssertFootnoteCounter{36}\vismHypertarget{III.n36}{}“In addition to the five things” (not quoted) dealt with earlier in the sutta, namely, perfection of virtue, good friendship, hearing suitable things, energy, and understanding.} develop these four things: foulness should be developed for the purpose of abandoning greed (lust). Loving-kindness should be developed for \marginnote{\textcolor{teal}{\footnotesize\{168|110\}}}{}the purpose of abandoning ill will. \textcolor{brown}{\textit{[115]}} Mindfulness of breathing should be developed for the purpose of cutting off applied thought. Perception of impermanence should be cultivated for the purpose of eliminating the conceit, ‘I am’” (\textbf{\cite{A}IV 358}). Also in the Rāhula Sutta, in the passage beginning, “Develop loving-kindness, Rāhula” (\textbf{\cite{M}I 424}), seven meditation subjects are given for a single temperament. So instead of insisting on the mere letter, the intention should be sought in each instance.

                    This is the explanatory exposition of the meditation subject referred to by the words \emph{he should apprehend…one} [meditation subject] (\hyperlink{III.28}{§28}{}).

                    
                \subsection[\vismAlignedParas{§123–129}Self-dedication]{Self-dedication}

                    \vismParagraph{III.123}{123}{}
                    Now the words \emph{and he should apprehend }are illustrated as follows. After approaching the good friend of the kind described in the explanation of the words \emph{then approach the good friend, the giver of a meditation subject }(\hyperlink{III.28}{§28}{} and \hyperlink{III.57}{§57}{}–\hyperlink{III.73}{73}{}), the meditator should dedicate himself to the Blessed One, the Enlightened One, or to a teacher, and he should ask for the meditation subject with a sincere inclination [of the heart] and sincere resolution.

                    \vismParagraph{III.124}{124}{}
                    Herein, he should dedicate himself to the Blessed One, the Enlightened One, in this way: “Blessed One, I relinquish this my person to you.” For without having thus dedicated himself, when living in a remote abode he might be unable to stand fast if a frightening object made its appearance, and he might return to a village abode, become associated with laymen, take up improper search and come to ruin. But when he has dedicated himself in this way no fear arises in him if a frightening object makes its appearance; in fact only joy arises in him as he reflects: “Have you not wisely already dedicated yourself to the Enlightened One?”

                    \vismParagraph{III.125}{125}{}
                    Suppose a man had a fine piece of Kāsi cloth. He would feel grief if it were eaten by rats or moths; but if he gave it to a bhikkhu needing robes, he would feel only joy if he saw the bhikkhu tearing it up [to make his patched cloak]. And so it is with this.

                    \vismParagraph{III.126}{126}{}
                    When he dedicates himself to a teacher, he should say: “I relinquish this my person to you, venerable sir.” For one who has not dedicated his person thus becomes unresponsive to correction, hard to speak to, and unamenable to advice, or he goes where he likes without asking the teacher. Consequently the teacher does not help him with either material things or the Dhamma, and he does not train him in the cryptic books.\footnote{\vismAssertFootnoteCounter{37}\vismHypertarget{III.n37}{}“‘\emph{Cryptic books’: }the meditation-subject books dealing with the truths, the dependent origination, etc., which are profound and associated with voidness” (Vism-mhṭ 111). Cf. \textbf{\cite{M-a}II 264}, \textbf{\cite{A-a}} commentary to AN 4:180.} Failing to get these two kinds of help, \textcolor{brown}{\textit{[116]}} he finds no footing in the Dispensation, and he soon comes down to misconducting himself or to the lay state. But if he has dedicated his person, he is not unresponsive to correction, does not go about as he likes, is easy to speak to, and lives only in dependence on the teacher. He gets the twofold help from the teacher and attains growth, increase, and fulfilment in the Dispensation. Like the Elder Cūḷa-Piṇḍapātika-Tissa’s pupils.

                    \vismParagraph{III.127}{127}{}
                    Three bhikkhus came to the elder, it seems. One of them said, “Venerable sir, I am ready to fall from a cliff the height of one hundred men, if it is said to be to your advantage.” The second said, “Venerable sir, I am ready to grind away this body from the heels up without remainder on a flat stone, if it is said to be to your advantage.” The third said, “Venerable sir, I am ready to die by stopping breathing, \marginnote{\textcolor{teal}{\footnotesize\{169|111\}}}{}if it is said to be to your advantage.” Observing, “These bhikkhus are certainly capable of progress,” the elder expounded a meditation subject to them. Following his advice, the three attained Arahantship.

                    This is the benefit in self-dedication. Hence it was said above “dedicating himself to the Blessed One, the Enlightened One, or to a teacher.”

                    

                    \vismParagraph{III.128}{128}{}
                    \emph{With a sincere inclination [of the heart] and sincere resolution }(\hyperlink{III.123}{§123}{}): the meditator’s inclination should be sincere in the six modes beginning with non-greed. For it is one of such sincere inclination who arrives at one of the three kinds of enlightenment, according as it is said: “Six kinds of inclination lead to the maturing of the enlightenment of the Bodhisattas. With the inclination to non-greed, Bodhisattas see the fault in greed. With the inclination to non-hate, Bodhisattas see the fault in hate. With the inclination to non-delusion, Bodhisattas see the fault in delusion. With the inclination to renunciation, Bodhisattas see the fault in house life. With the inclination to seclusion, Bodhisattas see the fault in society. With the inclination to relinquishment, Bodhisattas see the fault in all kinds of becoming and destiny (\emph{Source untraced}.)” For stream-enterers, once-returners, non-returners, those with cankers destroyed (i.e. Arahants), Paccekabuddhas, and Fully Enlightened Ones, whether past, future or present, all arrive at the distinction peculiar to each by means of these same six modes. That is why he should have sincerity of inclination in these six modes.

                    \vismParagraph{III.129}{129}{}
                    He should be whole-heartedly resolved on that. The meaning is \textcolor{brown}{\textit{[117]}} that he should be resolved upon concentration, respect concentration, incline to concentration, be resolved upon Nibbāna, respect Nibbāna, incline to Nibbāna.
                \subsection[\vismAlignedParas{§130–133}Ways of expounding]{Ways of expounding}

                    \vismParagraph{III.130}{130}{}
                    When, with sincerity of inclination and whole-hearted resolution in this way, he asks for a meditation subject, then a teacher who has acquired the penetration of minds can know his temperament by surveying his mental conduct; and a teacher who has not can know it by putting such questions to him as: “What is your temperament?” or “What states are usually present in you?” or “What do you like bringing to mind?” or “What meditation subject does your mind favour?” When he knows, he can expound a meditation subject suitable to that temperament. And in doing so, he can expound it in three ways: it can be expounded to one who has already learnt the meditation subject by having him recite it at one or two sessions; it can be expounded to one who lives in the same place each time he comes; and to one who wants to learn it and then go elsewhere it can be expounded in such a manner that it is neither too brief nor too long.

                    \vismParagraph{III.131}{131}{}
                    Herein, when first he is explaining the earth kasiṇa, there are nine aspects that he should explain. They are the four faults of the kasiṇa, the making of a kasiṇa, the method of development for one who has made it, the two kinds of sign, the two kinds of concentration, the seven kinds of suitable and unsuitable, the ten kinds of skill in absorption, evenness of energy, and the directions for absorption.

                    In the case of the other meditation subjects, each should be expounded in the way appropriate to it. All this will be made clear in the directions for development. But when the meditation subject is being expounded in this way, the meditator must apprehend the sign as he listens.

                    \vismParagraph{III.132}{132}{}
                    \marginnote{\textcolor{teal}{\footnotesize\{170|112\}}}{}\emph{Apprehend the sign} means that he must connect each aspect thus: “This is the preceding clause, this is the subsequent clause, this is its meaning, this is its intention, this is the simile.” When he listens attentively, apprehending the sign in this way, his meditation subject is well apprehended. Then, and because of that, he successfully attains distinction, but not otherwise. This clarifies the meaning of the words “and he must apprehend.”

                    \vismParagraph{III.133}{133}{}
                    At this point the clauses approach the good friend, the giver of a meditation subject, and he should apprehend from among the forty meditation subjects one that suits his own temperament (\hyperlink{III.28}{§28}{}) have been expounded in detail in all their aspects.

                    The third chapter called “The Description of Taking a Meditation Subject” in the Treatise on the Development of Concentration in the \emph{Path of Purification} composed for the purpose of gladdening good people.
        \chapter[The Earth Kasiṇa]{The Earth Kasiṇa\vismHypertarget{IV}\newline{\textnormal{\emph{Pathavī-kasiṇa-niddesa}}}}
            \label{IV}


            \vismParagraph{IV.1}{1}{}
            \marginnote{\textcolor{teal}{\footnotesize\{171|113\}}}{}\textcolor{brown}{\textit{[118]}} Now, it was said earlier: After that he should avoid a monastery unfavourable to the development of concentration and go to live in one that is favourable (\hyperlink{III.28}{III.28}{}). In the first place one who finds it convenient to live with the teacher in the same monastery can live there while he is making certain of the meditation subject. If it is inconvenient there, he can live in another monastery—a suitable one—a quarter or a half or even a whole league distant. In that case, when he finds he is in doubt about, or has forgotten, some passage in the meditation subject, then he should do the duties in the monastery in good time and set out afterwards, going for alms on the way and arriving at the teacher’s dwelling place after his meal. He should make certain about the meditation subject that day in the teacher’s presence. Next day, after paying homage to the teacher, he should go for alms on his way back and so he can return to his own dwelling place without fatigue. But one who finds no convenient place within even a league should clarify all difficulties about the meditation subject and make quite sure it has been properly attended to. Then he can even go far away and, avoiding a monastery unfavourable to development of concentration, live in one that is favourable.
            \section[\vismAlignedParas{§2–18}The eighteen faults of a monastery]{The eighteen faults of a monastery}

                \vismParagraph{IV.2}{2}{}
                Herein, one that is unfavourable has anyone of eighteen faults. These are: (1) largeness, (2) newness, (3) dilapidatedness, (4) a nearby road, (5) a pond, (6) [edible] leaves, (7) flowers, (8) fruits, (9) famousness, (10) a nearby city, (11) nearby timber trees, (12) nearby arable fields, (13) presence of incompatible persons, (14) a nearby port of entry, (15) nearness to the border countries, (16) nearness to the frontier of a kingdom, (17) unsuitability, (18) lack of good friends. \textcolor{brown}{\textit{[119]}} One with any of these faults is not favourable. He should not live there. Why?

                \vismParagraph{IV.3}{3}{}
                \emph{1. }Firstly, people with varying aims collect in a \emph{large monastery}. They conflict with each other and so neglect the duties. The Enlightenment-tree terrace, etc., remain unswept, the water for drinking and washing is not set out. So if he thinks, “I shall go to the alms-resort village for alms” and takes his bowl and robe and sets out, perhaps he sees that the duties have not been done or that a drinking-water pot is empty, and so the duty has to be done by him unexpectedly. Drinking water must be maintained. By not doing it he would commit a \marginnote{\textcolor{teal}{\footnotesize\{172|114\}}}{}wrongdoing in the breach of a duty. But if he does it, he loses time. He arrives too late at the village and gets nothing because the alms giving is finished. Also, when he goes into retreat, he is distracted by the loud noises of novices and young bhikkhus, and by acts of the Community [being carried out]. However, he can live in a large monastery where all the duties are done and where there are none of the other disturbances.

                \vismParagraph{IV.4}{4}{}
                \emph{2.} In \emph{a new monastery }there is much new building activity. People criticize someone who takes no part in it. But he can live in such a monastery where the bhikkhus say, “Let the venerable one do the ascetic’s duties as much as he likes. We shall see to the building work.”

                \vismParagraph{IV.5}{5}{}
                \emph{3.} In \emph{a dilapidated monastery} there is much that needs repair. People criticize someone who does not see about the repairing of at least his own lodging. When he sees to the repairs, his meditation subject suffers.

                \vismParagraph{IV.6}{6}{}
                \emph{4.} In a monastery with a \emph{nearby road}, by a main street, visitors keep arriving night and day. He has to give up his own lodging to those who come late, and he has to go and live at the root of a tree or on top of a rock. And next day it is the same. So there is no opportunity [to practice] his meditation subject. But he can live in one where there is no such disturbance by visitors.

                \vismParagraph{IV.7}{7}{}
                \emph{5.} A \emph{pond }is a rock pool. Numbers of people come there for drinking water. Pupils of city-dwelling elders supported by the royal family come to do dyeing work. When they ask for vessels, wood, tubs, etc., \textcolor{brown}{\textit{[120]}} they must be shown where these things are. So he is kept all the time on the alert.

                \vismParagraph{IV.8}{8}{}
                \emph{6.} If he goes with his meditation subject to sit by day where there are many sorts of edible \emph{leaves}, then women vegetable-gatherers, singing as they pick leaves nearby, endanger his meditation subject by disturbing it with sounds of the opposite sex.

                \emph{7.} And where there are many sorts of \emph{flowering }shrubs in bloom there is the same danger too.

                \vismParagraph{IV.9}{9}{}
                \emph{8.} Where there are many sorts of \emph{fruits }such as mangoes, rose-apples and jak-fruits, people who want fruits come and ask for them, and they get angry if he does not give them any, or they take them by force. When walking in the monastery in the evening he sees them and asks, “Why do you do so, lay followers?” they abuse him as they please and even try to evict him.

                \vismParagraph{IV.10}{10}{}
                \emph{9.} When he lives in a monastery that is \emph{famous }and renowned in the world, like Dakkhiṇagiri\footnote{\vismAssertFootnoteCounter{1}\vismHypertarget{IV.n1}{}“They say it is the Dakkhiṇagiri in the Magadha country” (\textbf{\cite{Vism-mhṭ}116}). There is mention of a Dakkhiṇagiri-vihāra at \textbf{\cite{M-a}II 293} and elsewhere.} Hatthikucchi, Cetiyagiri or Cittalapabbata, there are always people coming who want to pay homage to him, supposing that he is an Arahant, which inconveniences him. But if it suits him, he can live there at night and go elsewhere by day.

                \vismParagraph{IV.11}{11}{}
                \emph{10.} In one with a \emph{nearby city }objects of the opposite sex come into focus. Women-pot carriers go by bumping into him with their jars and giving no room \marginnote{\textcolor{teal}{\footnotesize\{173|115\}}}{}to pass. Also important people spread out carpets in the middle of the monastery and sit down.

                \vismParagraph{IV.12}{12}{}
                \emph{11.} One with \emph{nearby timber trees} where there are timber trees and osiers useful for making framework is inconvenient because of the wood-gatherers there, like the gatherers of branches and fruits already mentioned. If there are trees in a monastery, people come and cut them down to build houses with. When he has come out of his meditation room in the evening and is walking up and down in the monastery, if he sees them and asks, “Why do you do so, lay followers?” they abuse him as they please and even try to evict him.

                \vismParagraph{IV.13}{13}{}
                \emph{12.} People make use of one with \emph{nearby arable fields}, quite surrounded by fields. They make a threshing floor in the middle of the monastery itself. They thresh corn there, dry it in the forecourts,\footnote{\vismAssertFootnoteCounter{2}\vismHypertarget{IV.n2}{}Read \emph{pamukhesu sosayanti. Pamukha} not thus in PED.} and cause great inconvenience. And where there is extensive property belonging to the Community, the monastery attendants impound cattle belonging to families and deny the water supply [to their crops]. \textcolor{brown}{\textit{[121]}} Then people bring an ear of paddy and show it to the Community saying “Look at your monastery attendants’ work.” For one reason or another he has to go to the portals of the king or the king’s ministers. This [matter of property belonging to the Community] is included by [a monastery that is] near arable fields.

                \vismParagraph{IV.14}{14}{}
                \emph{13. Presence of incompatible persons}: where there are bhikkhus living who are incompatible and mutually hostile, when they clash and it is protested, “Venerable sirs, do not do so,” they exclaim, “We no longer count now that this refuse-rag wearer has come.” 15.

                \vismParagraph{IV.15}{15}{}
                \emph{14.} One with a nearby water \emph{port of entry} or land port of entry\footnote{\vismAssertFootnoteCounter{3}\vismHypertarget{IV.n3}{}“A \emph{‘water port of entry’ }is a port of entry on the sea or on an estuary. A \emph{‘land port of entry’ }is one on the edge of a forest and acts as the gateway on the road of approach to great cities” (\textbf{\cite{Vism-mhṭ}116}).} is made inconvenient by people constantly arriving respectively by ship or by caravan and crowding round, asking for space or for drinking water or salt.

                \vismParagraph{IV.16}{16}{}
                \emph{15.} In the case of one \emph{near the border countries}, people have no trust in the Buddha, etc., there.

                \emph{16.} In one \emph{near the frontier of a kingdom} there is fear of kings. For perhaps one king attacks that place, thinking, “It does not submit to my rule,” and the other does likewise, thinking, “It does not submit to my rule.” A bhikkhu lives there when it is conquered by one king and when it is conquered by the other. Then they suspect him of spying, and they bring about his undoing.

                \vismParagraph{IV.17}{17}{}
                \emph{17. Unsuitability} is that due to the risk of encountering visible data, etc., of the opposite sex as objects or to haunting by non-human beings. Here is a story. An elder lived in a forest, it seems. Then an ogress stood in the door of his leaf hut and sang. The elder came out and stood in the door. She went to the end of the walk and sang. The elder went to the end of the walk. She stood in a chasm a hundred fathoms deep and sang. The elder recoiled. Then she suddenly \marginnote{\textcolor{teal}{\footnotesize\{174|116\}}}{}grabbed him saying, “Venerable sir, it is not just one or two of the likes of you I have eaten.”

                \vismParagraph{IV.18}{18}{}
                \emph{18. Lack of good friends}: where it is not possible to find a good friend as a teacher or the equivalent of a teacher or a preceptor or the equivalent of a preceptor, the lack of good friends there is a serious fault.

                One that has any of those eighteen faults should be understood as unfavourable. And this is said in the commentaries:
                \begin{verse}
                    A large abode, a new abode,\\{}
                    One tumbling down, one near a road,\\{}
                    One with a pond, or leaves, or flowers,\\{}
                    Or fruits, or one that people seek; \textcolor{brown}{\textit{[122]}}
                \end{verse}

                \begin{verse}
                    In cities, among timber, fields,\\{}
                    Where people quarrel, in a port,\\{}
                    In border lands, on frontiers,\\{}
                    Unsuitableness, and no good friend—
                \end{verse}

                \begin{verse}
                    These are the eighteen instances\\{}
                    A wise man needs to recognize\\{}
                    And give them full as wide a berth\\{}
                    As any footpad-hunted road.
                \end{verse}

            \section[\vismAlignedParas{§19}The five factors of the resting place]{The five factors of the resting place}

                \vismParagraph{IV.19}{19}{}
                One that has the five factors beginning with “not too far from and not too near to” the alms resort is called favourable. For this is said by the Blessed One: “And how has a lodging five factors, bhikkhus? Here, bhikkhus, (1) a lodging is not too far, not too near, and has a path for going and coming. (2) It is little frequented by day with little sound and few voices by night. (3) There is little contact with gadflies, flies, wind, burning [sun] and creeping things. (4) One who lives in that lodging easily obtains robes, alms food, lodging, and the requisite of medicine as cure for the sick. (5) In that lodging there are elder bhikkhus living who are learned, versed in the scriptures, observers of the Dhamma, observers of the Vinaya, observers of the Codes, and when from time to time one asks them questions, ‘How is this, venerable sir? What is the meaning of this?’ then those venerable ones reveal the unrevealed, explain the unexplained, and remove doubt about the many things that raise doubts. This, bhikkhus, is how a lodging has five factors”(\textbf{\cite{A}V 15}).

                These are the details for the clause, “After that he should avoid a monastery unfavourable to the development of concentration and go to live in one that is favourable” (\hyperlink{III.28}{III.28}{}).
            \section[\vismAlignedParas{§20}The lesser impediments]{The lesser impediments}

                \vismParagraph{IV.20}{20}{}
                \emph{Then he should sever the lesser impediments} (\hyperlink{III.28}{III.28}{}): one living in such a favourable monastery should sever any minor impediments that he may still have, that is to say, long head hair, nails, and body hair should be cut, mending and patching of old robes should be done, or those that are soiled should be \marginnote{\textcolor{teal}{\footnotesize\{175|117\}}}{}dyed. If there is a stain on the bowl, the bowl should be baked. The bed, chair, etc., should be cleaned up. These are the details for the clause, “Then he should sever the lesser impediments.”
            \section[\vismAlignedParas{§21}Detailed instructions for development]{Detailed instructions for development}

                \vismParagraph{IV.21}{21}{}
                Now, with the clause, \emph{And not overlook any of the directions for development }(\hyperlink{III.28}{III.28}{}), the time has come for the detailed exposition of all meditation subjects, starting with the earth kasiṇa.
            \section[\vismAlignedParas{§22–31}The earth kasiṇa]{The earth kasiṇa}

                \vismParagraph{IV.22}{22}{}
                \textcolor{brown}{\textit{[123]}} When a bhikkhu has thus severed the lesser impediments, then, on his return from his alms round after his meal and after he has got rid of drowsiness due to the meal, he should sit down comfortably in a secluded place and apprehend the sign in earth that is either made up or not made up. For this is said:\footnote{\vismAssertFootnoteCounter{4}\vismHypertarget{IV.n4}{}“Said in the Old Commentary. \emph{‘One who is learning the earth kasiṇa’: }one who is apprehending, grasping, an earth kasiṇa as a ‘learning sign’. The meaning is, one who is producing an earth kasiṇa that has become the sign of learning; and here ‘arousing’ should be regarded as the establishing of the sign in that way. \emph{‘In earth’: }in an earth disk of the kind about to be described. \emph{‘Apprehends the sign’: }he apprehends in that, with knowledge connected with meditative development, the sign of earth of the kind about to be described, as one does with the eye the sign of the face in a looking-glass. \emph{‘Made up’: }prepared in the manner about to be described. \emph{‘Not made up’: }in a disk of earth consisting of an ordinary threshing-floor disk, and so on. \emph{‘Bounded’: }only in one that has bounds. As regard the words \emph{‘the size of a bushel’, }etc., it would be desirable that a bushel and a saucer were of equal size, but some say that \emph{‘the size of a saucer’ }is a span and four fingers, and the \emph{‘the size of a bushel’ }is larger than that. \emph{‘He sees to it that that sign is well apprehended’: }that meditator makes that disk of earth a well-apprehended sign. When, after apprehending the sign in it by opening the eyes, and looking and then closing them again, it appears to him as he adverts to it just as it did at the moment of looking with open eyes, then he has made it well apprehended. Having thoroughly established his mindfulness there, observing it again and again with his mind not straying outside, he sees that it is \emph{‘well attended to’. }When it is well attended to thus by adverting and attending again and again by producing much repetition and development instigated by that, he sees that it is \emph{‘well defined’. ‘To that object’: }to that object called earth kasiṇa, which has appeared rightly owing to its having been well apprehended. \emph{‘He anchors his mind’: }by bringing his own mind to access jhāna he anchors it, keeps it from other objects” (\textbf{\cite{Vism-mhṭ}119}).} “One who is learning the earth kasiṇa apprehends the sign in earth that is either made up or not made up; that is bounded, not unbounded; limited, not unlimited; with a periphery, not without a periphery; circumscribed, not uncircumscribed; either the size of a bushel (\emph{suppa}) or the size of a saucer (\emph{sarāva}). He sees to it that that sign is well apprehended, well attended to, well defined. Having done that, and seeing its advantages and perceiving it as a treasure, building up respect for it, making it dear to him, he anchors his mind to that object, thinking, ‘Surely in this way I shall be freed from aging and death.’ Secluded from sense desires … he enters upon and dwells in the first jhāna …”

                \vismParagraph{IV.23}{23}{}
                \marginnote{\textcolor{teal}{\footnotesize\{176|118\}}}{}Herein, when in a previous becoming a man has gone forth into homelessness in the Dispensation or [outside it] with the rishis’ going forth and has already produced the jhāna tetrad or pentad on the earth kasiṇa, and so has such merit and the support [of past practice of jhāna] as well, then the sign arises in him on earth that is not made up, that is to say, on a ploughed area or on a threshing floor, as in the Elder Mallaka’s case.

                It seems that while that venerable one was looking at a ploughed area the sign arose in him the size of that area. He extended it and attained the jhāna pentad. Then by establishing insight with the jhāna as the basis for it, he reached Arahantship.
                \subsection[\vismAlignedParas{§24–26}Making an earth kasiṇa]{Making an earth kasiṇa}

                    \vismParagraph{IV.24}{24}{}
                    But when a man has had no such previous practice, he should make a kasiṇa, guarding against the four faults of a kasiṇa and not overlooking any of the directions for the meditation subject learnt from the teacher. Now, the four faults of the earth kasiṇa are due to the intrusion of blue, yellow, red or white. So instead of using clay of such colours, he should make the kasiṇa of clay like that in the stream of the Gangā,\footnote{\vismAssertFootnoteCounter{5}\vismHypertarget{IV.n5}{}\emph{“Gaṅgā }(= ‘river’) is the name for the Ganges in India and for the Mahavaeligaṅgā, Sri Lanka’s principal river. However, in the Island of Sri Lanka there is a river, it seems, called the Rāvanagaṅgā. The clay in the places where the banks are cut away by its stream is the colour of dawn” (\textbf{\cite{Vism-mhṭ}119}).} which is the colour of the dawn. \textcolor{brown}{\textit{[124]}} And he should make it not in the middle of the monastery in a place where novices, etc., are about but on the confines of the monastery in a screened place, either under an overhanging rock or in a leaf hut. He can make it either portable or as a fixture.

                    \vismParagraph{IV.25}{25}{}
                    Of these, a portable one should be made by tying rags of leather or matting onto four sticks and smearing thereon a disk of the size already mentioned, using clay picked clean of grass, roots, gravel, and sand, and well kneaded. At the time of the preliminary work it should be laid on the ground and looked at.

                    A fixture should be made by knocking stakes into the ground in the form of a lotus calyx, lacing them over with creepers. If the clay is insufficient, then other clay should be put underneath and a disk a span and four fingers across made on top of that with the quite pure dawn-coloured clay. For it was with reference only to measurement that it was said above \emph{either the size of a bushel or the size of a saucer} (\hyperlink{IV.22}{§22}{}). But \emph{that is bounded, not unbounded }was said to show its delimitedness.

                    \vismParagraph{IV.26}{26}{}
                    So, having thus made it delimited and of the size prescribed, he should scrape it down with a stone trowel—a wooden trowel turns it a bad colour, so that should not be employed—and make it as even as the surface of a drum. Then he should sweep the place out and have a bath. On his return he should seat himself on a well-covered chair with legs a span and four fingers high, prepared in a place that is two and a half cubits [that is, two and a half times elbow to finger-tip] from the kasiṇa disk. For the kasiṇa does not appear plainly to him if he sits further off than that; and if he sits nearer than that, faults in the \marginnote{\textcolor{teal}{\footnotesize\{177|119\}}}{}kasiṇa appear. If he sits higher up, he has to look at it with his neck bent; and if he sits lower down, his knees ache.
                \subsection[\vismAlignedParas{§27–30}Starting contemplation]{Starting contemplation}

                    \vismParagraph{IV.27}{27}{}
                    So, after seating himself in the way stated, he should review the dangers in sense desires in the way beginning, “Sense desires give little enjoyment” (\textbf{\cite{M}I 91}) and arouse longing for the escape from sense desires, for the renunciation that is the means to the surmounting of all suffering. He should next arouse joy of happiness by recollecting the special qualities of the Buddha, the Dhamma, and the Sangha; then awe by thinking, “Now, this is the way of renunciation entered upon by all Buddhas, Paccekabuddhas and noble disciples”; and then eagerness by thinking, “In this way I shall surely come to know the taste of the bliss of seclusion.” \textcolor{brown}{\textit{[125]}} After that he should open his eyes moderately, apprehend the sign, and so proceed to develop it.\footnote{\vismAssertFootnoteCounter{6}\vismHypertarget{IV.n6}{}\emph{“‘Apprehend the sign’: }apprehend with the mind the sign apprehended by the eye in the earth kasiṇa. \emph{‘And develop it’: }the apprehending of the sign as it occurs should be continued intensively and constantly practiced” (\textbf{\cite{Vism-mhṭ}120}).}

                    \vismParagraph{IV.28}{28}{}
                    If he opens his eyes too wide, they get fatigued and the disk becomes too obvious, which prevents the sign becoming apparent to him. If he opens them too little, the disk is not obvious enough, and his mind becomes drowsy, which also prevents the sign becoming apparent to him. So he should develop it by apprehending the sign (\emph{nimitta}), keeping his eyes open moderately, as if he were seeing the reflection of his face (\emph{mukha-nimitta}) on the surface of a looking-glass.\footnote{\vismAssertFootnoteCounter{7}\vismHypertarget{IV.n7}{}“Just as one who sees his reflection \emph{(mukha-nimitta—}lit. “face-sign”) on the surface of a looking-glass does not open his eyes too widely or too little (in order to get the effect), nor does he review the colour of the looking-glass or give attention to its characteristic, but rather looks with moderately opened eyes and sees only the sign of his face, so too this meditator looks with moderately opened eyes at the earth kasiṇa and is occupied only with the sign” (\textbf{\cite{Vism-mhṭ}121}).}

                    \vismParagraph{IV.29}{29}{}
                    The colour should not be reviewed. The characteristic should not be given attention.\footnote{\vismAssertFootnoteCounter{8}\vismHypertarget{IV.n8}{}“The dawn colour that is there in the kasiṇa should not be thought about, though it cannot be denied that it is apprehended by eye-consciousness. That is why, instead of saying here, ‘should not be looked at,’ he says that it should not be apprehended by reviewing. Also the earth element’s characteristic of hardness, which is there, should not be given attention because the apprehension has to be done through the channel of seeing. And after saying, ‘while not ignoring the colour’ he said, ‘relegating the colour to the position of a property of the physical support,’ showing that here the concern is not with the colour, which is the channel, but rather that this colour should be treated as an accessory of the physical support; the meaning is that the kasiṇa (disk) should be given attention with awareness of both the accompanying earth-aspect and its ancillary colour-aspect, but taking the earth-aspect with its ancillary concomitant colour as both supported equally by that physical support [the disk]. ‘On the concept as the mental datum since that is what is outstanding’: the term of ordinary usage ‘earth’ (\emph{pathavī}) as applied to earth with its accessories, since the prominence of its individual effect is due to outstandingness of the earth element: ‘setting the mind’ on that mental datum consisting of a [name-] concept (\emph{paññatti-dhamma}), the kasiṇa should be given attention as ‘earth, earth.’—If the mind is to be set on a mere concept by means of a term of common usage, ought earth to be given attention by means of different names?—It can be. What is wrong? It is to show that that is done he said, ‘\emph{Mahī, medinī},’ and so on” (\textbf{\cite{Vism-mhṭ}122}).} But rather, while not ignoring the colour, attention should be given \marginnote{\textcolor{teal}{\footnotesize\{178|120\}}}{}by setting the mind on the [name] concept as the most outstanding mental datum, relegating the colour to the position of a property of its physical support. That [conceptual state] can be called by anyone he likes among the names for earth (\emph{pathavī}) such as “earth” (\emph{pathavī}), “the Great One” (\emph{mahī}), “the Friendly One” (\emph{medinī}), “ground” (\emph{bhūmi}), “the Provider of Wealth” (\emph{vasudhā}), “the Bearer of Wealth” (\emph{vasudharā}), etc., whichever suits his manner of perception. Still “earth” is also a name that is obvious, so it can be developed with the obvious one by saying “earth, earth.” It should be adverted to now with eyes open, now with eyes shut. And he should go on developing it in this way a hundred times, a thousand times, and even more than that, until the learning sign arises.

                    \vismParagraph{IV.30}{30}{}
                    When, while he is developing it in this way, it comes into focus\footnote{\vismAssertFootnoteCounter{9}\vismHypertarget{IV.n9}{}“‘\emph{Comes into focus’: }becomes the resort of mind-door impulsion” (\textbf{\cite{Vism-mhṭ}122}).} as he adverts with his eyes shut exactly as it does with his eyes open, then the learning sign is said to have been produced. After its production he should no longer sit in that place;\footnote{\vismAssertFootnoteCounter{10}\vismHypertarget{IV.n10}{}“Why should he not? If, after the learning sign was produced, he went on developing it by looking at the disk of the earth, there would be no arising of the counterpart sign” (\textbf{\cite{Vism-mhṭ}122}).}\textbf{ }he should return to his own quarters and go on developing it sitting there. But in order to avoid the delay of foot washing, a pair of single-soled sandals and a walking stick are desirable. Then if the new concentration vanishes through some unsuitable encounter, he can put his sandals on, take his walking stick, and go back to the place to re-apprehend the sign there. When he returns he should seat himself comfortably and develop it by reiterated reaction to it and by striking at it with thought and applied thought.
                \subsection[\vismAlignedParas{§31}The counterpart sign]{The counterpart sign}

                    \vismParagraph{IV.31}{31}{}
                    As he does so, the hindrances eventually become suppressed, the defilements subside, the mind becomes concentrated with access concentration, and the counterpart sign arises.

                    The difference between the earlier learning sign and the counterpart sign is this. In the learning sign any fault in the kasiṇa is apparent. But the counterpart sign \textcolor{brown}{\textit{[126]}} appears as if breaking out from the learning sign, and a hundred times, a thousand times more purified, like a looking-glass disk drawn from its case, like a mother-of-pearl dish well washed, like the moon’s disk coming out from behind a cloud, like cranes against a thunder cloud. But it has neither colour nor shape; for if it had, it would be cognizable by the eye, gross, susceptible of comprehension [by insight—(see \hyperlink{XX.2}{XX.2f.}{})] and stamped with the three characteristics.\footnote{\vismAssertFootnoteCounter{11}\vismHypertarget{IV.n11}{}“Stamped with the three characteristics of the formed beginning with rise (see \textbf{\cite{A}I 152}), or marked with the three characteristics beginning with impermanence” (Vism-mhṭ 122).}\textbf{ }But it is not like that. For it is born only of perception in one who has obtained concentration, being a mere mode of appearance.\footnote{\vismAssertFootnoteCounter{12}\vismHypertarget{IV.n12}{}“If ‘it is not like that’—is not possessed of colour, etc.—then how is it the object of jhāna? It is in order to answer that question that the sentence beginning, ‘For it is …’ is given. ‘Born of the perception’: produced by the perception during development, simply born from the perception during development. Since there is no arising from anywhere of what has no individual essence, he therefore said, ‘Being the mere mode of appearance’” (\textbf{\cite{Vism-mhṭ}122}). See \hyperlink{VIII.n11}{Ch. VIII, n. 11}{}.} But as \marginnote{\textcolor{teal}{\footnotesize\{179|121\}}}{}soon as it arises the hindrances are quite suppressed, the defilements subside, and the mind becomes concentrated in access concentration.
            \section[\vismAlignedParas{§32–33}The two kinds of concentration]{The two kinds of concentration}

                \vismParagraph{IV.32}{32}{}
                Now, concentration is of two kinds, that is to say, access concentration and absorption concentration: the mind becomes concentrated in two ways, that is, on the plane of access and on the plane of obtainment. Herein, the mind becomes concentrated on the plane of access by the abandonment of the hindrances, and on the plane of obtainment by the manifestation of the jhāna factors.

                \vismParagraph{IV.33}{33}{}
                The difference between the two kinds of concentration is this. The factors are not strong in access. It is because they are not strong that when access has arisen, the mind now makes the sign its object and now re-enters the life-continuum,\footnote{\vismAssertFootnoteCounter{13}\vismHypertarget{IV.n13}{}\emph{Bhavaṅga }(life-continuum, lit. “constituent of becoming”) and \emph{javana }(impulsion) are first mentioned in this work at \hyperlink{I.57}{I.57}{} (see n. 16); this is the second mention. The “cognitive series” \emph{(citta-vīthi) }so extensively used here is unknown as such in the Piṭakas. Perhaps the seed from which it sprang may exist in, say, such passages as: “Due to eye and to visible data eye-consciousness arises. The coincidence of the three is contact. With contact as condition there is feeling. What he feels he perceives. What he perceives he thinks about4. What he thinks about he diversifies [by means of craving, pride and false view] … Due to mind and to mental data …” (\textbf{\cite{M}I 111}). And: “Is the eye permanent or impermanent … Are visible objects permanent or impermanent? … Is the mind permanent or impermanent? Are mental data … Is mind-consciousness … Is mind-contact … Is any feeling, any perception, any formation, any consciousness, that arises with mind-contact as condition permanent or impermanent?” (\textbf{\cite{M}III 279}). And: “These five faculties [of eye, etc.] each with its separate objective field and no one of them experiencing as its objective field the province of any other, have mind as their refuge, and mind experiences their provinces as its objective field” (\textbf{\cite{M}I 295}). This treatment of consciousness implies, as it were, more than even a “double thickness” of consciousness. An already-formed nucleus of the cognitive series, based on such Sutta Piṭakas material, appears in the Abhidhamma Piṭakas. The following two quotations show how the commentary (bracketed italics) expands the Abhidhamma Piṭakas treatment.

                        (i) “Herein, what is eye-consciousness element? Due to eye and to visible data (\emph{as support condition, and to functional mind element} (= \emph{5-door adverting}), \emph{as disappearance condition, and to the remaining three immaterial aggregates as conascence condition}) there arises consciousness … which is eye-consciousness element. [Similarly with the other four sense elements.] Herein, what is mind element? Eye-consciousness having arisen and ceased, next to that there arises consciousness … which is appropriate (profitable or unprofitable) mind element (in the mode of receiving). [Similarly with the other four sense elements.] Or else it is the first reaction to any mental datum (to be taken as functional mind element in the mode of mind-door adverting). Herein, what is mind-consciousness element? Eye-consciousness having arisen and ceased, next to that there arises mind element. (Resultant) mind element having arisen and ceased, also (next to that there arises resultant mind-consciousness element in the mode of investigating; and that having arisen and ceased, next to that there arises functional mind-consciousness element in the mode of determining; and that having arisen and ceased) next to that there arises consciousness … which is appropriate mind-consciousness element (in the mode of impulsion). [Similarly with the other four sense elements.] Due to (life-continuum) mind and to mental data there arises consciousness … which is appropriate (impulsion) mind-consciousness element (following on the above-mentioned mind-door adverting)” (\textbf{\cite{Vibh}87–90} and \textbf{\cite{Vibh-a}81f.}).

                        (ii) “Eye-consciousness and its associated states are a condition, as proximity condition, for \emph{(resultant) }mind element and for its associated states. Mind element and its associated states are a condition, as proximity condition, for \emph{(root-causeless resultant) }mind-consciousness element \emph{(in the mode of investigating) }and for its associated states. \emph{(Next to that, the mind-consciousness elements severally in the modes of determining, impulsion, registration, and life-continuum should be mentioned, though they are not, since the teaching is abbreviated.) }[Similarly for the other four senses and mind-consciousness element]. Preceding profitable \emph{(impulsion) }states are a condition, as proximity condition, for subsequent indeterminate \emph{(registration, life-continuum) }states [etc.]” (\textbf{\cite{Paṭṭh}} II, and Comy., 33–34).

                        The form that the two kinds (5-door and mind-door) of the cognitive series take is shown in Table V. The following are some Piṭakas references for the individual modes: \emph{bhavaṅga }(life-continuum): \textbf{\cite{Paṭṭh}I 159}, 160, 169, 324; \emph{āvajjana }(adverting) \textbf{\cite{Paṭṭh}I 159}, 160, 169, 324; \emph{sampaṭicchana }(receiving), \emph{santīraṇa }(investigating), \emph{voṭṭhapana }(determining), and \emph{tadārammaṇa }(registration) appear only in the Commentaries. \emph{Javana }(impulsion): \textbf{\cite{Paṭis}II 73}, 76. The following references may also be noted here: \emph{anuloma }(conformity), \textbf{\cite{Paṭṭh}I 325}. \emph{Cuti-citta }(death consciousness), \textbf{\cite{Paṭṭh}I 324}. \emph{Paṭisandhi }(rebirth-linking), Vism-mhṭ 1, 320, etc.; \textbf{\cite{Paṭis}II 72}, etc.} just as when a young child is lifted up and stood on its feet, it \marginnote{\textcolor{teal}{\footnotesize\{180|122\}}}{}repeatedly falls down on the ground. But the factors are strong in absorption. It is because they are strong that when absorption concentration has arisen, the mind, having once interrupted the flow of the life-continuum, carries on with a stream of profitable impulsion for a whole night and for a whole day, just as a healthy man, after rising from his seat, could stand for a whole day.
            \section[\vismAlignedParas{§34–41}Guarding the sign]{Guarding the sign}

                \vismParagraph{IV.34}{34}{}
                The arousing of the counterpart sign, which arises together with access concentration, is very difficult. Therefore if he is able to arrive at absorption in that same session by extending the sign, it is good. If not, then he must guard the sign diligently as if it were the foetus of a Wheel-turning Monarch (World-ruler).
                \begin{verse}
                    \marginnote{\textcolor{teal}{\footnotesize\{181|123\}}}{}So guard the sign, nor count the cost,\\{}
                    And what is gained will not be lost;\\{}
                    Who fails to have this guard maintained\\{}
                    Will lose each time what he has gained. \textcolor{brown}{\textit{[127]}}
                \end{verse}


                \vismParagraph{IV.35}{35}{}
                Herein, the way of guarding it is this:
                \begin{verse}
                    (1) Abode, (2) resort, (3) and speech, (4) and person,\\{}
                    (5) The food, (6) the climate, (7) and the posture—\\{}
                    Eschew these seven different kinds
                \end{verse}

                \begin{verse}
                    Whenever found unsuitable.\\{}
                    But cultivate the suitable;\\{}
                    For one perchance so doing finds\\{}
                    He need not wait too long until\\{}
                    Absorption shall his wish fulfil.
                \end{verse}


                \vismParagraph{IV.36}{36}{}
                \emph{1. }Herein, an \emph{abode }is unsuitable if, while he lives in it, the unarisen sign does not arise in him or is lost when it arises, and where unestablished mindfulness fails to become established and the unconcentrated mind fails to become concentrated. That is suitable in which the sign arises and becomes confirmed, in which mindfulness becomes established and the mind becomes concentrated, as in the Elder Padhāniya-Tissa, resident at Nāgapabbata. So if a monastery has many abodes he can try them one by one, living in each for three days, and stay on where his mind becomes unified. For it was due to suitability of abode that five hundred bhikkhus reached Arahantship while still dwelling in the Lesser Nāga Cave (\emph{Cūḷa-nāga-leṇa}) in Tambapaṇṇi Island (Sri Lanka) after apprehending their meditation subject there. There is no counting the stream-enterers who have reached Arahantship there after reaching the noble plane elsewhere; so too in the monastery of Cittalapabbata, and others.

                \vismParagraph{IV.37}{37}{}
                \emph{2.} An alms-\emph{resort} village lying to the north or south of the lodging, not too far, within one \emph{kosa }and a half, and where alms food is easily obtained, is suitable. The opposite kind is unsuitable.\footnote{\vismAssertFootnoteCounter{14}\vismHypertarget{IV.n14}{}North or south to avoid facing the rising sun in coming or going. \emph{Kosa }is not in PED; “one and a half \emph{kosa} = 3,000 bows” (\textbf{\cite{Vism-mhṭ}123}).}

                \vismParagraph{IV.38}{38}{}
                \emph{3. Speech}: that included in the thirty-two kinds of aimless talk is unsuitable; for it leads to the disappearance of the sign. But talk based on the ten examples of talk is suitable, though even that should be discussed with moderation.\footnote{\vismAssertFootnoteCounter{15}\vismHypertarget{IV.n15}{}Twenty-six kinds of “aimless” (lit. “animal”) talk are given in the Suttas (e.g. \textbf{\cite{M}II 1}; III 113), which the commentary increases to thirty-two (\textbf{\cite{M-a}III 233}). The ten instances of talk are those given in the Suttas (e.g. \textbf{\cite{M}I 145}; III 113). See \hyperlink{I.n12}{Ch. I, n.12}{}.}

                \vismParagraph{IV.39}{39}{}
                \emph{4. Person}: one not given to aimless talk, who has the special qualities of virtue, etc., by acquaintanceship with whom the unconcentrated mind becomes concentrated, or the concentrated mind becomes more so, is suitable. One who is much concerned with his body,\footnote{\vismAssertFootnoteCounter{16}\vismHypertarget{IV.n16}{}“One who is occupied with exercising and caring for the body” (\textbf{\cite{Vism-mhṭ}124}).} who is addicted to aimless talk, is unsuitable; for he only creates disturbances, like muddy water added to clear water. And it \marginnote{\textcolor{teal}{\footnotesize\{182|124\}}}{}was owing to one such as this that the attainments of the young bhikkhu who lived at Koṭapabbata vanished, not to mention the sign. \textcolor{brown}{\textit{[128]}}

                \vismParagraph{IV.40}{40}{}
                \emph{5. Food}: Sweet food suits one, sour food another.

                \emph{6. Climate}: a cool climate suits one, a warm one another. So when he finds that by using certain food or by living in a certain climate he is comfortable, or his unconcentrated mind becomes concentrated, or his concentrated mind becomes more so, then that food or that climate is suitable. Any other food or climate is unsuitable.

                \vismParagraph{IV.41}{41}{}
                \emph{7. Postures}: walking suits one; standing or sitting or lying down suits another. So he should try them, like the abode, for three days each, and that posture is suitable in which his unconcentrated mind becomes concentrated or his concentrated mind becomes more so. Any other should be understood as unsuitable.

                So he should avoid the seven unsuitable kinds and cultivate the suitable. For when he practices in this way, assiduously cultivating the sign, then, “he need not wait too long until absorption shall his wish fulfil.”
            \section[\vismAlignedParas{§42–65}The ten kinds of skill in absorption]{The ten kinds of skill in absorption}

                \vismParagraph{IV.42}{42}{}
                However, if this does not happen while he is practicing in this way, then he should have recourse to the ten kinds of skill in absorption. Here is the method. Skill in absorption needs [to be dealt with in] ten aspects: (1) making the basis clean, (2) maintaining balanced faculties, (3) skill in the sign, (4) he exerts the mind on an occasion when it should be exerted, (5) he restrains the mind on an occasion when it should be restrained, (6) he encourages the mind on an occasion when it should be encouraged, (7) he looks on at the mind with equanimity when it should be looked on at with equanimity, (8) avoidance of unconcentrated persons, (9) cultivation of concentrated persons, (10) resoluteness upon that (concentration).

                \vismParagraph{IV.43}{43}{}
                \emph{1. }Herein, \emph{making the basis clean} is cleansing the internal and the external basis. For when his head hair, nails and body hair are long, or when the body is soaked with sweat, then the internal basis is unclean and unpurified. But when an old dirty smelly robe is worn or when the lodging is dirty, then the external basis is unclean and unpurified. \textcolor{brown}{\textit{[129]}} When the internal and external bases are unclean, then the knowledge in the consciousness and consciousness-concomitants that arise is unpurified, like the light of a lamp’s flame that arises with an unpurified lamp-bowl, wick and oil as its support; formations do not become evident to one who tries to comprehend them with unpurified knowledge, and when he devotes himself to his meditation subject, it does not come to growth, increase and fulfilment.

                \vismParagraph{IV.44}{44}{}
                But when the internal and external bases are clean, then the knowledge in the consciousness and consciousness-concomitants that arise is clean and purified, like the light of a lamp’s flame that arises with a purified lamp bowl, wick and oil as its support; formations become evident to one who tries to comprehend them with purified knowledge, and as he devotes himself to his meditation subject, it comes to growth, increase and fulfilment.

                \vismParagraph{IV.45}{45}{}
                \marginnote{\textcolor{teal}{\footnotesize\{183|125\}}}{}\emph{2. Maintaining balanced faculties} is equalizing the [five] faculties of faith and the rest. For if his faith faculty is strong and the others weak, then the energy faculty cannot perform its function of exerting, the mindfulness faculty its function of establishing, the concentration faculty its function of not distracting, and the understanding faculty its function of seeing. So in that case the faith faculty should be modified either by reviewing the individual essences of the states [concerned, that is, the objects of attention] or by not giving [them] attention in the way in which the faith faculty became too strong. And this is illustrated by the story of the Elder Vakkali (\textbf{\cite{S}III 119}).

                \vismParagraph{IV.46}{46}{}
                Then if the energy faculty is too strong, the faith faculty cannot perform its function of resolving, nor can the rest of the faculties perform their several functions. So in that case the energy faculty should be modified by developing tranquillity, and so on. And this should be illustrated by the story of the Elder Soṇa (\textbf{\cite{Vin}I 179–185}; \textbf{\cite{A}III 374–376}). So too with the rest; for it should be understood that when anyone of them is too strong the others cannot perform their several functions.

                \vismParagraph{IV.47}{47}{}
                However, what is particularly recommended is balancing faith with understanding, and concentration with energy. For one strong in faith and weak in understanding has confidence uncritically and groundlessly. One strong in understanding and weak in faith errs on the side of cunning and is as hard to cure as one sick of a disease caused by medicine. With the balancing of the two a man has confidence only when there are grounds for it.

                Then idleness overpowers one strong in concentration and weak in energy, since concentration favours idleness. \textcolor{brown}{\textit{[130]}} Agitation overpowers one strong in energy and weak in concentration, since energy favours agitation. But concentration coupled with energy cannot lapse into idleness, and energy coupled with concentration cannot lapse into agitation. So these two should be balanced; for absorption comes with the balancing of the two.

                \vismParagraph{IV.48}{48}{}
                Again, [concentration and faith should be balanced]. One working on concentration needs strong faith, since it is with such faith and confidence that he reaches absorption. Then there is [balancing of] concentration and understanding. One working on concentration needs strong unification, since that is how he reaches absorption; and one working on insight needs strong understanding, since that is how he reaches penetration of characteristics; but with the balancing of the two he reaches absorption as well.

                \vismParagraph{IV.49}{49}{}
                Strong mindfulness, however, is needed in all instances; for mindfulness protects the mind from lapsing into agitation through faith, energy and understanding, which favour agitation, and from lapsing into idleness through concentration, which favours idleness. So it is as desirable in all instances as a seasoning of salt in all sauces, as a prime minister in all the king’s business. Hence it is said [in the commentaries (D-a 788, \textbf{\cite{M-a}I 292}, etc)]: “And mindfulness has been called universal by the Blessed One. For what reason? Because the mind has mindfulness as its refuge, and mindfulness is manifested as protection, and there is no exertion and restraint of the mind without mindfulness.”

                \vismParagraph{IV.50}{50}{}
                \marginnote{\textcolor{teal}{\footnotesize\{184|126\}}}{}\emph{3. Skill in the sign} is skill in producing the as yet unproduced sign of unification of mind through the earth kasiṇa, etc.; and it is skill in developing [the sign] when produced, and skill in protecting [the sign] when obtained by development. The last is what is intended here.

                \vismParagraph{IV.51}{51}{}
                \emph{4.} How does he \emph{exert the mind on an occasion when it should be exerted}? When his mind is slack with over-laxness of energy, etc., then, instead of developing the three enlightenment factors beginning with tranquillity, he should develop those beginning with investigation-of-states. For this is said by the Blessed One: “Bhikkhus, suppose a man wanted to make a small fire burn up, and he put wet grass on it, put wet cow-dung on it, put wet sticks on it, sprinkled it with water, and scattered dust on it, would that man be able to make the small fire burn up?” \textcolor{brown}{\textit{[131]}}—“No, venerable sir.”—“So too, bhikkhus, when the mind is slack, that is not the time to develop the tranquillity enlightenment factor, the concentration enlightenment factor or the equanimity enlightenment factor. Why is that? Because a slack mind cannot well be roused by those states. When the mind is slack, that is the time to develop the investigation-of-states enlightenment factor, the energy enlightenment factor and the happiness enlightenment factor. Why is that? Because a slack mind can well be roused by those states.

                “Bhikkhus, suppose a man wanted to make a small fire burn up, and he put dry grass on it, put dry cow-dung on it, put dry sticks on it, blew on it with his mouth, and did not scatter dust on it, would that man be able to make that small fire burn up?”—“Yes, venerable sir” (\textbf{\cite{S}V 112}).

                \vismParagraph{IV.52}{52}{}
                And here the development of the investigation-of-states enlightenment factor, etc., should be understood as the nutriment for each one respectively, for this is said: “Bhikkhus, there are profitable and unprofitable states, reprehensible and blameless states, inferior and superior states, dark and bright states the counterpart of each other. Wise attention much practiced therein is the nutriment for the arising of the unarisen investigation-of-states enlightenment factor, or leads to the growth, fulfilment, development and perfection of the arisen investigation-of-states enlightenment factor.” Likewise: “Bhikkhus there is the element of initiative, the element of launching, and the element of persistence. Wise attention much practiced therein is the nutriment for the arising of the unarisen energy enlightenment factor, or leads to the growth, fulfilment, development and perfection of the arisen energy enlightenment factors.” Likewise: “Bhikkhus, there are states productive of the happiness enlightenment factor. Wise attention much practiced therein is the nutriment for the arising of the unarisen happiness enlightenment factor, or leads to the growth, fulfilment, development and perfection of the arisen happiness enlightenment factor” (\textbf{\cite{S}V 104}). \textcolor{brown}{\textit{[132]}}

                \vismParagraph{IV.53}{53}{}
                Herein, \emph{wise attention given to the profitable}, etc., is attention occurring in penetration of individual essences and of [the three] general characteristics. \emph{Wise attention given to the element of initiative}, etc., is attention occurring in the arousing of the element of initiative, and so on. Herein, initial energy is called the \emph{element of initiative}. The \emph{element of launching }is stronger than that because it launches out from idleness. \emph{The element of persistence }is still stronger than that \marginnote{\textcolor{teal}{\footnotesize\{185|127\}}}{}because it goes on persisting in successive later stages. \emph{States productive of the happiness enlightenment factor} is a name for happiness itself; and attention that arouses that is \emph{wise attention}.

                \vismParagraph{IV.54}{54}{}
                There are, besides, seven things that lead to the arising of the investigation-of-states enlightenment factor: (i) asking questions, (ii) making the basis clean, (iii) balancing the faculties, (iv) avoidance of persons without understanding, (v) cultivation of persons with understanding, (vi) reviewing the field for the exercise of profound knowledge, (vii) resoluteness upon that [investigation of states].

                \vismParagraph{IV.55}{55}{}
                Eleven things lead to the arising of the energy enlightenment factor: (i) reviewing the fearfulness of the states of loss such as the hell realms, etc., (ii) seeing benefit in obtaining the mundane and supramundane distinctions dependent on energy, (iii) reviewing the course of the journey [to be travelled] thus: “The path taken by the Buddhas, Paccekabuddhas, and the great disciples has to be taken by me, and it cannot be taken by an idler,” (iv) being a credit to the alms food by producing great fruit for the givers, (v) reviewing the greatness of the Master thus: “My Master praises the energetic, and this unsurpassable Dispensation that is so helpful to us is honoured in the practice, not otherwise,” (vi) reviewing the greatness of the heritage thus: “It is the great heritage called the Good Dhamma that is to be acquired by me, and it cannot be acquired by an idler,” (vii) removing stiffness and torpor by attention to perception of light, change of postures, frequenting the open air, etc., (viii) avoidance of idle persons, (ix) cultivation of energetic persons, (x) reviewing the right endeavours, (xi) resoluteness upon that [energy].

                \vismParagraph{IV.56}{56}{}
                Eleven things lead to the arising of the happiness enlightenment factor: the recollections (i) of the Buddha, (ii) of the Dhamma, (iii) of the Sangha, (iv) of virtue, (v) of generosity, and (vi) of deities, (vii) the recollection of peace, \textcolor{brown}{\textit{[133]}} (viii) avoidance of rough persons, (ix) cultivation of refined persons, (x) reviewing encouraging discourses, (xi) resoluteness upon that [happiness].

                So by arousing these things in these ways he develops the investigation-of-states enlightenment factor, and the others. This is how he exerts the mind on an occasion when it should be exerted.

                \vismParagraph{IV.57}{57}{}
                \emph{5.} How does he \emph{restrain the mind on an occasion when it should be restrained}? When his mind is agitated through over-energeticness, etc., then, instead of developing the three enlightenment factors beginning with investigation-of-states, he should develop those beginning with tranquillity; for this is said by the Blessed One: “Bhikkhus, suppose a man wanted to extinguish a great mass of fire, and he put dry grass on it … and did not scatter dust on it, would that man be able to extinguish that great mass of fire?”—“No, venerable sir.”—“So too, bhikkhus, when the mind is agitated, that is not the time to develop the investigation-of-states enlightenment factor, the energy enlightenment factor or the happiness enlightenment factor. Why is that? Because an agitated mind cannot well be quieted by those states. When the mind is agitated, that is the time to develop the tranquillity enlightenment factor, the concentration enlightenment \marginnote{\textcolor{teal}{\footnotesize\{186|128\}}}{}factor and the equanimity enlightenment factor. Why is that? Because an agitated mind can well be quieted by those states.”

                “Bhikkhus, suppose a man wanted to extinguish a great mass of fire, and he put wet grass on it … and scattered dust on it, would that man be able to extinguish that great mass of fire?”—“Yes, venerable sir” (\textbf{\cite{S}V 114}).

                \vismParagraph{IV.58}{58}{}
                And here the development of the tranquillity enlightenment factor, etc., should be understood as the nutriment for each one respectively, for this is said: “Bhikkhus, there is bodily tranquillity and mental tranquillity. \textcolor{brown}{\textit{[134]}} Wise attention much practiced therein is the nutriment for the arising of the unarisen tranquillity enlightenment factor, or leads to the growth, fulfilment, development and perfection of the arisen tranquillity enlightenment factor.” Likewise: “Bhikkhus, there is the sign of serenity, the sign of non-diversion. Wise attention, much practiced, therein is the nutriment for the arising of the unarisen concentration enlightenment factor, or it leads to the growth, fulfilment, development and perfection of the arisen concentration enlightenment factor.” Likewise: “Bhikkhus, there are states productive of the equanimity enlightenment factor. Wise attention, much practiced, therein is the nutriment for the arising of the unarisen equanimity enlightenment factor, or it leads to the growth, fulfilment, development and perfection of the arisen equanimity enlightenment factor” (\textbf{\cite{S}V 104}).

                \vismParagraph{IV.59}{59}{}
                Herein \emph{wise attention} given to the three instances is attention occurring in arousing tranquillity, etc., by observing the way in which they arose in him earlier. The \emph{sign of serenity }is a term for serenity itself, and \emph{non-diversion }is a term for that too in the sense of non-distraction.

                \vismParagraph{IV.60}{60}{}
                There are, besides, seven things that lead to the arising of the tranquillity enlightenment factor: (i) using superior food, (ii) living in a good climate, (iii) maintaining a pleasant posture, (iv) keeping to the middle, (v) avoidance of violent persons, (vi) cultivation of persons tranquil in body, (vii) resoluteness upon that [tranquillity].

                \vismParagraph{IV.61}{61}{}
                Eleven things lead to the arising of the concentration enlightenment factor: (i) making the basis clean, (ii) skill in the sign, (iii) balancing the faculties, (iv) restraining the mind on occasion, (v) exerting the mind on occasion, (vi) encouraging the listless mind by means of faith and a sense of urgency, (vii) looking on with equanimity at what is occurring rightly, (viii) avoidance of unconcentrated persons, (ix) cultivation of concentrated persons, (x) reviewing of the jhānas and liberations, (xi) resoluteness upon that [concentration].

                \vismParagraph{IV.62}{62}{}
                Five things lead to the arising of the equanimity enlightenment factor: (i) maintenance of neutrality towards living beings; (ii) maintenance of neutrality towards formations (inanimate things); (iii) avoidance of persons who show favouritism towards beings and formations; (iv) cultivation of persons who maintain neutrality towards beings and formations; (v) resoluteness upon that [equanimity]. \textcolor{brown}{\textit{[135]}}

                So by arousing these things in these ways he develops the tranquillity enlightenment factor, as well as the others. This is how he restrains the mind on an occasion when it should be restrained.

                \vismParagraph{IV.63}{63}{}
                \marginnote{\textcolor{teal}{\footnotesize\{187|129\}}}{}\emph{6.} How does he \emph{encourage the mind on an occasion when it should be encouraged}? When his mind is listless owing to sluggishness in the exercise of understanding or to failure to attain the bliss of peace, then he should stimulate it by reviewing the eight grounds for a sense of urgency. These are the four, namely, birth, aging, sickness, and death, with the suffering of the states of loss as the fifth, and also the suffering in the past rooted in the round [of rebirths], the suffering in the future rooted in the round [of rebirths], and the suffering in the present rooted in the search for nutriment. And he creates confidence by recollecting the special qualities of the Buddha, the Dhamma, and the Sangha. This is how he encourages the mind on an occasion when it should be encouraged.

                \vismParagraph{IV.64}{64}{}
                \emph{7.} How does he \emph{look on at the mind with equanimity on an occasion when it should be looked on at with equanimity}? When he is practicing in this way and his mind follows the road of serenity, occurs evenly on the object, and is unidle, unagitated and not listless, then he is not interested to exert or restrain or encourage it; he is like a charioteer when the horses are progressing evenly. This is how he looks on at the mind with equanimity on an occasion when it should be looked on at with equanimity.

                \vismParagraph{IV.65}{65}{}
                \emph{8. Avoidance of unconcentrated persons} is keeping far away from persons who have never trodden the way of renunciation, who are busy with many affairs, and whose hearts are distracted.

                \emph{9. Cultivation of concentrated persons} is approaching periodically persons who have trodden the way of renunciation and obtained concentration.

                \emph{10. Resoluteness upon that} is the state of being resolute upon concentration; the meaning is, giving concentration importance, tending, leaning and inclining to concentration.

                This is how the tenfold skill in concentration should be undertaken.
            \section[\vismAlignedParas{§66–73}Balancing the effort]{Balancing the effort}

                \vismParagraph{IV.66}{66}{}
                
                \begin{verse}
                    Any man who acquires this sign,\\{}
                    This tenfold skill will need to heed\\{}
                    In order for absorption to gain\\{}
                    Thus achieving his bolder goal.\\{}
                    But if in spite of his efforts\\{}
                    No result comes that might requite\\{}
                    His work, still a wise wight persists,\\{}
                    Never this task relinquishing, \textcolor{brown}{\textit{[136]}}\\{}
                    Since a tiro, if he gives up,\\{}
                    Thinking not to continue in\\{}
                    The task, never gains distinction\\{}
                    Here no matter how small at all.\\{}
                    A man wise in temperament\footnote{\vismAssertFootnoteCounter{17}\vismHypertarget{IV.n17}{}\emph{Buddha—}“possessed of wit”: not in PED; see \textbf{\cite{M-a}I 39}.}\\{}
                    Notices how his mind inclines:\\{}
                    Energy and serenity\\{}
                    Always he couples each to each.
                \end{verse}

                \begin{verse}
                    \marginnote{\textcolor{teal}{\footnotesize\{188|130\}}}{}Now, his mind, seeing that it holds back,\\{}
                    He prods, now the restraining rein\\{}
                    Tightening, seeing it pull too hard;\\{}
                    Guiding with even pace the race.\\{}
                    Well-controlled bees get the pollen;\\{}
                    Well-balanced efforts meet to treat\\{}
                    Leaves, thread, and ships, and oil-tubes too,\\{}
                    Gain thus, not otherwise, the prize.\\{}
                    Let him set aside this lax\\{}
                    Also this agitated state,\\{}
                    Steering here his mind at the sign\\{}
                    As the bee and the rest suggest.
                \end{verse}

                \subsection[\vismAlignedParas{§67–73}The five similes]{The five similes}

                    \vismParagraph{IV.67}{67}{}
                    Here is the explanation of the meaning.

                    When a too clever bee learns that a flower on a tree is blooming, it sets out hurriedly, overshoots the mark, turns back, and arrives when the pollen is finished; and another, not clever enough bee, who sets out with too slow a speed, arrives when the pollen is finished too; but a clever bee sets out with balanced speed, arrives with ease at the cluster of flowers, takes as much pollen as it pleases and enjoys the honey-dew.

                    \vismParagraph{IV.68}{68}{}
                    Again, when a surgeon’s pupils are being trained in the use of the scalpel on a lotus leaf in a dish of water, one who is too clever applies the scalpel hurriedly and either cuts the lotus leaf in two or pushes it under the water, and another who is not clever enough does not even dare to touch it with the scalpel for fear of cutting it in two or pushing it under; but one who is clever shows the scalpel stroke on it by means of a balanced effort, and being good at his craft he is rewarded on such occasions.

                    \vismParagraph{IV.69}{69}{}
                    Again when the king announces, “Anyone who can draw out a spider’s thread four fathoms long shall receive four thousand,” one man who is too clever breaks the spider’s thread here and there by pulling it hurriedly, and another who is not clever enough does not dare to touch it with his hand for fear of breaking it, but a clever man pulls it out starting from the end with a balanced effort, winds it on a stick, and so wins the prize.

                    \vismParagraph{IV.70}{70}{}
                    Again, a too clever \textcolor{brown}{\textit{[137]}} skipper hoists full sails in a high wind and sends his ship adrift, and another, not clever enough skipper, lowers his sails in a light wind and remains where he is, but a clever skipper hoists full sails in a light wind, takes in half his sails in a high wind, and so arrives safely at his desired destination.

                    \vismParagraph{IV.71}{71}{}
                    Again, when a teacher says, “Anyone who fills the oil-tube without spilling any oil will win a prize,” one who is too clever fills it hurriedly out of greed for the prize, and he spills the oil, and another who is not clever enough does not dare to pour the oil at all for fear of spilling it, but one who is clever fills it with a balanced effort and wins the prize.

                    \vismParagraph{IV.72}{72}{}
                    Just as in these five similes, so too when the sign arises, one bhikkhu forces his energy, thinking “I shall soon reach absorption.” Then his mind lapses into \marginnote{\textcolor{teal}{\footnotesize\{189|131\}}}{}agitation because of his mind’s over-exerted energy and he is prevented from reaching absorption. Another who sees the defect in over-exertion slacks off his energy, thinking, “What is absorption to me now?” Then his mind lapses into idleness because of his mind’s too lax energy and he too is prevented from reaching absorption. Yet another who frees his mind from idleness even when it is only slightly idle and from agitation when only slightly agitated, confronting the sign with balanced effort, reaches absorption. One should be like the last-named.

                    \vismParagraph{IV.73}{73}{}
                    It was with reference to this meaning that it was said above:
                    \begin{verse}
                        “Well-controlled bees get the pollen;\\{}
                        Well-balanced efforts meet to treat\\{}
                        Leaves, thread, and ships, and oil-tubes too,\\{}
                        Gain thus, not otherwise, the prize.\\{}
                        Let him set aside then this lax\\{}
                        Also this agitated state,\\{}
                        Steering here his mind at the sign\\{}
                        As the bee and the rest suggest”.
                    \end{verse}

            \section[\vismAlignedParas{§74–78}Absorption in the cognitive series]{Absorption in the cognitive series}

                \vismParagraph{IV.74}{74}{}
                So, while he is guiding his mind in this way, confronting the sign, [then knowing]: “Now absorption will succeed,” there arises in him mind-door adverting with that same earth kasiṇa as its object, interrupting the [occurrence of consciousness as] life-continuum, and evoked by the constant repeating of “earth, earth.” After that, either four or five impulsions impel on that same object, the last one of which is an impulsion of the fine-material sphere. The rest are of the sense sphere, but they have stronger applied thought, sustained thought, happiness, bliss, and unification of mind than the normal ones. They are called “preliminary work” [consciousnesses] because they are the preliminary work for absorption; \textcolor{brown}{\textit{[138]}} and they are also called “access” [consciousnesses] because of their nearness to absorption because they happen in its neighbourhood, just as the words “village access” and “city access” are used for a place near to a village, etc.; and they are also called “conformity” [consciousnesses] because they conform to those that precede the “preliminary work” [consciousnesses] and to the absorption that follows. And the last of these is also called “change-of-lineage” because it transcends the limited [sense-sphere] lineage and brings into being the exalted [fine-material-sphere] lineage.\footnote{\vismAssertFootnoteCounter{18}\vismHypertarget{IV.n18}{}“It guards the line (\emph{gaṃ tāyati}), thus it is lineage (\emph{gotta}). When it occurs limitedly, it guards the naming (\emph{abhidhāna}) and the recognition (\emph{buddhi}) of the naming as restricted to a definite scope (\emph{ekaṃsa-visayatā}). For just as recognition does not take place without a meaning (\emph{attha}) for its objective support (\emph{ārammaṇa}), so naming (\emph{abhidhāna}) does not take place without what is named (\emph{abhidheyya}). So it (the \emph{gotta}) is said to protect and keep these. But the limited should be regarded as the materiality peculiar to sense-sphere states, which are the resort of craving for sense desires, and destitute of the exalted (fine-material and immaterial) or the unsurpassed (supramundane). The exalted lineage is explainable in the same way” (\textbf{\cite{Vism-mhṭ}134}).}

                \vismParagraph{IV.75}{75}{}
                \marginnote{\textcolor{teal}{\footnotesize\{190|132\}}}{}But omitting repetitions,\footnote{\vismAssertFootnoteCounter{19}\vismHypertarget{IV.n19}{}See \hyperlink{XVII.189}{XVII.189}{} and note.} then either the first is the “preliminary work,” the second “access,” the third “conformity,” and the fourth, “change-of-lineage,” or else the first is “access,” the second “conformity,” and the third “change-of-lineage.” Then either the fourth [in the latter case] or the fifth [in the former case] is the absorption consciousness. For it is only either the fourth or the fifth that fixes in absorption. And that is according as there is swift or sluggish direct-knowledge. (cf. \hyperlink{XXI.117}{XXI.117}{}) Beyond that, impulsion lapses and the life-continuum\footnote{\vismAssertFootnoteCounter{20}\vismHypertarget{IV.n20}{}“The intention is that it is as if the sixth and seventh impulsions had lapsed since impulsion beyond the fifth is exhausted. The elder’s opinion was that just as the first impulsion, which lacks the quality of repetition, does not arouse change-of-lineage because of its weakness, while the second or the third, which have the quality of repetition, can do so because they are strong on that account, so too the sixth and seventh fix in absorption owing to their strength due to their quality of repetition. But it is unsupported by a sutta or by any teacher’s statement in conformity with a sutta. And the text quoted is not a reason because strength due to the quality of repetition is not a principle without exceptions \emph{(anekantikattā); }for the first volition, which is not a repetition, has result experienceable here and now, while the second to the sixth, which are repetitions, have result experienceable in future becomings” (\textbf{\cite{Vism-mhṭ}135}).} takes over.

                \vismParagraph{IV.76}{76}{}
                But the Abhidhamma scholar, the Elder Godatta, quoted this text: “Preceding profitable states are a condition, as repetition condition, for succeeding profitable states” (\textbf{\cite{Paṭṭh}I 5}). Adding, “It is owing to the repetition condition that each succeeding state is strong, so there is absorption also in the sixth and seventh.”

                \vismParagraph{IV.77}{77}{}
                That is rejected by the commentaries with the remark that it is merely that elder’s opinion, adding that, “It is only either in the fourth or the fifth\footnote{\vismAssertFootnoteCounter{21}\vismHypertarget{IV.n21}{}“‘\emph{Either in the fourth or the fifth,’ }etc., is said for the purpose of concluding [the discussion] with a paragraph showing the correctness of the meaning already stated.—Herein, if the sixth and seventh impulsions are said to have lapsed because impulsion is exhausted, how does seventh-impulsion volition come to have result experienceable in the next rebirth and to be of immediate effect on rebirth?—This is not owing to strength got through a repetition condition.—What then?—It is owing to the difference in the function’s position \emph{(kiriyāvatthā). }For the function [of impulsion] has three positions, that is, initial, medial and final. Herein, experienceability of result in the next rebirth and immediateness of effect on rebirth are due to the last volition’s final position, not to its strength … So the fact that the sixth and seventh lapse because impulsion is used up cannot be objected to” (\textbf{\cite{Vism-mhṭ}135}). See Table V.} that there is absorption. Beyond that, impulsion lapses. It is said to do so because of nearness of the life-continuum.” And that has been stated in this way after consideration, so it cannot be rejected. For just as a man who is running towards a precipice and wants to stop cannot do so when he has his foot on the edge but falls over it, so there can be no fixing in absorption in the sixth or the seventh because of the nearness to the life-continuum. That is why it should be understood that there is absorption only in the fourth or the fifth.

                \vismParagraph{IV.78}{78}{}
                \marginnote{\textcolor{teal}{\footnotesize\{191|133\}}}{}But that absorption is only of a single conscious moment. For there are seven instances in which the normal extent\footnote{\vismAssertFootnoteCounter{22}\vismHypertarget{IV.n22}{}“‘\emph{The normal extent does not apply’ }here \emph{‘in the seven instances’ }because of the immeasurability of the conscious moment in some, and the extreme brevity of the moment in others; for \emph{‘extent’ }is inapplicable here in the sense of complete cognitive series, which is why ‘in fruition next to the path,’ etc., is said” (\textbf{\cite{Vism-mhṭ}} 136).} [of the cognitive series] does not apply. They are in the cases of the first absorption, the mundane kinds of direct-knowledge, the four paths, fruition next after the path, life-continuum jhāna in the fine-material and immaterial kinds of becoming, the base consisting of neither perception nor non-perception as condition for cessation [of perception and feeling], and the fruition attainment in one emerging from cessation. Here the fruition next after the path does not exceed three [consciousnesses in number]; \textcolor{brown}{\textit{[139]}} the [consciousnesses] of the base consisting of neither perception nor non-perception as condition for cessation do not exceed two [in number]; there is no measure of the [number of consciousnesses in the] life-continuum in the fine-material and immaterial [kinds of becoming]. In the remaining instances [the number of consciousnesses is] one only. So absorption is of a single consciousness moment. After that, it lapses into the life-continuum. Then the life-continuum is interrupted by adverting for the purpose of reviewing the jhāna, next to which comes the reviewing of the jhāna.
            \section[\vismAlignedParas{§79–126}The first jhāna]{The first jhāna}

                \vismParagraph{IV.79}{79}{}
                At this point, “Quite secluded from sense desires, secluded from unprofitable things he enters upon and dwells in the first jhāna, which is accompanied by applied and sustained thought with happiness and bliss born of seclusion” (\textbf{\cite{Vibh}245}), and so he has attained the first jhāna, which abandons five factors, possesses five factors, is good in three ways, possesses ten characteristics, and is of the earth kasiṇa.

                \vismParagraph{IV.80}{80}{}
                Herein, \emph{quite secluded from sense desires} means having secluded himself from, having become without, having gone away from, sense desires. Now, this word \emph{quite }(\emph{eva}) should be understood to have the meaning of absoluteness. Precisely because it has the meaning of absoluteness it shows how, on the actual occasion of entering upon and dwelling in the first jhāna, sense desires as well as being non-existent then are the first jhāna’s contrary opposite, and it also shows that the arrival takes place only (\emph{eva}) through the letting go of sense desires. How?

                \vismParagraph{IV.81}{81}{}
                When absoluteness is introduced thus, “quite secluded from sense desires,” what is expressed is this: sense desires are certainly incompatible with this jhāna; when they exist, it does not occur, just as when there is darkness, there is no lamplight; and it is only by letting go of them that it is reached, just as the further bank is reached only by letting go of the near bank. That is why absoluteness is introduced.

                \vismParagraph{IV.82}{82}{}
                Here it might be asked: But why is this [word “quite”] mentioned only in the first phrase and not in the second? How is this, might he enter upon and \marginnote{\textcolor{teal}{\footnotesize\{192|134\}}}{}dwell in the first jhāna even when not secluded from unprofitable things?—It should not be regarded in that way. It is mentioned in the first phrase as the escape from them; for this jhāna is the escape from sense desires since it surmounts the sense-desire element and since it is incompatible with greed for sense desires, according as it is said: “The escape from sense desires is this, that is to say, renunciation” (\textbf{\cite{D}III 275}). But in the second phrase \textcolor{brown}{\textit{[140]}} the word \emph{eva }should be adduced and taken as said, as in the passage, “Bhikkhus, only (\emph{eva}) here is there an ascetic, here a second ascetic” (\textbf{\cite{M}I 63}). For it is impossible to enter upon and dwell in jhāna unsecluded also from unprofitable things, in other words, the hindrances other than that [sense desire]. So this word must be read in both phrases thus: “Quite secluded from sense desires, quite secluded from unprofitable things.” And although the word “secluded” as a general term includes all kinds of seclusion, that is to say, seclusion by substitution of opposites, etc., and bodily seclusion, etc.,\footnote{\vismAssertFootnoteCounter{23}\vismHypertarget{IV.n23}{}The five (see e.g. \textbf{\cite{Paṭis}II 220}; \textbf{\cite{M-a}I 85}) are suppression (by concentration), substitution of opposites (by insight), cutting off (by the path), tranquillization (by fruition), and escape (as Nibbāna); cf. five kinds of deliverance (e.g. \textbf{\cite{M-a}IV 168}). The \emph{three }(see e.g. \textbf{\cite{Nidd}I 26}; \textbf{\cite{M-a}II 143}) are bodily seclusion (retreat), mental seclusion (jhāna), and seclusion from the substance or circumstances of becoming (Nibbāna).} still only the three, namely, bodily seclusion, mental seclusion, and seclusion by suppression (suspension) should be regarded here.

                \vismParagraph{IV.83}{83}{}
                But this term “sense desires” should be regarded as including all kinds, that is to say, sense desires as object as given in the Niddesa in the passage beginning, “What are sense desires as object? They are agreeable visible objects …” (\textbf{\cite{Nidd}I 1}), and the sense desires as defilement given there too and in the Vibhaṅga thus: “Zeal as sense desire (\emph{kāma}), greed as sense desire, zeal and greed as sense desire, thinking as sense desire, greed as sense desire, thinking and greed as sense desire”\footnote{\vismAssertFootnoteCounter{24}\vismHypertarget{IV.n24}{}Here \emph{saṅkappa }(“thinking”) has the meaning of “hankering.” \emph{Chanda, kāma }and \emph{rāga }and their combinations need sorting out. \emph{Chanda }(zeal, desire) is much used, neutral in colour, good or bad according to context and glossed by “desire to act”; technically also one of the four roads to power and four predominances. \emph{Kāma }(sense desire, sensuality) loosely represents enjoyment of the five sense pleasures (e.g. sense-desire sphere). More narrowly it refers to sexual enjoyment (third of the Five Precepts). Distinguished as subjective desire (defilement) and objective things that arouse it (\textbf{\cite{Nidd}I 1}; cf. \hyperlink{XIV.n36}{Ch. XIV, n.36}{}). The figure “five cords of sense desire” signifies simply these desires with the five sense objects that attract them. \emph{Rāga }(greed) is the general term for desire in its bad sense and identical with \emph{lobha, }which latter, however, appears technically as one of the three root-causes of unprofitable action. \emph{Rāga }is renderable also by “lust” in its general sense. \emph{Kāmacchanda }(lust): a technical term for the first of the five hindrances. \emph{Chanda-rāga }(zeal and greed) and \emph{kāma-rāga }(greed for sense desires) have no technical use.} (\textbf{\cite{Nidd}I 2}; \textbf{\cite{Vibh}256}). That being so, the words “quite secluded from sense desires” properly mean “quite secluded from sense desires as object,” and express bodily seclusion, while the words “secluded from unprofitable things” properly mean “secluded from sense desires as defilement or from all unprofitable things,” and express mental seclusion. And in this case giving up of pleasure in sense desires is indicated by the first since it only expresses seclusion from sense desires as object, while acquisition of pleasure \marginnote{\textcolor{teal}{\footnotesize\{193|135\}}}{}in renunciation is indicated by the second since it expresses seclusion from sense desire as defilement.

                \vismParagraph{IV.84}{84}{}
                And with sense desires as object and sense desires as defilement expressed in this way, it should also be recognized that the abandoning of the objective basis for defilement is indicated by the first of these two phrases and the abandoning of the [subjective] defilement by the second; also that the giving up of the cause of cupidity is indicated by the first and [the giving up of the cause] of stupidity by the second; also that the purification of one’s occupation is indicated by the first and the educating of one’s inclination by the second.

                This, firstly, is the method here when the words \emph{from sense desires }are treated as referring to sense desires as object.

                \vismParagraph{IV.85}{85}{}
                But if they are treated as referring to sense desires as defilement, then it is simply just zeal for sense desires (\emph{kāmacchanda}) in the various forms of zeal (\emph{chanda}), greed (\emph{rāga}), etc., that is intended as “sense desires” (\emph{kāma}) (\hyperlink{IV.83}{§83}{}, 2nd quotation). \textcolor{brown}{\textit{[141]}} And although that [lust] is also included by [the word] “unprofitable,” it is nevertheless stated separately in the Vibhaṅga in the way beginning, “Herein, what are sense desires? Zeal as sense desire …” (\textbf{\cite{Vibh}256}) because of its incompatibility with jhāna. Or, alternatively, it is mentioned in the first phrase because it is sense desire as defilement and in the second phrase because it is included in the “unprofitable.” And because this [lust] has various forms, therefore “from sense desires” is said instead of “from sense desire.”

                \vismParagraph{IV.86}{86}{}
                And although there may be unprofitableness in other states as well, nevertheless only the hindrances are mentioned subsequently in the Vibhaṅga thus, “Herein, what states are unprofitable? Lust …” (\textbf{\cite{Vibh}256}), etc., in order to show their opposition to, and incompatibility with, the jhāna factors. For the hindrances are the contrary opposites of the jhāna factors: what is meant is that the jhāna factors are incompatible with them, eliminate them, abolish them. And it is said accordingly in the \emph{Peṭaka} (\emph{Peṭakopadesa}): “Concentration is incompatible with lust, happiness with ill will, applied thought with stiffness and torpor, bliss with agitation and worry, and sustained thought with uncertainty” (not in \emph{Peṭakopadesa}).

                \vismParagraph{IV.87}{87}{}
                So in this case it should be understood that seclusion by suppression (suspension) of lust is indicated by the phrase quite secluded from sense desires, and seclusion by suppression (suspension) of [all] five hindrances by the phrase secluded from unprofitable things. But omitting repetitions, that of lust is indicated by the first and that of the remaining hindrances by the second. Similarly with the three unprofitable roots, that of greed, which has the five cords of sense desire (\textbf{\cite{M}I 85}) as its province, is indicated by the first, and that of hate and delusion, which have as their respective provinces the various grounds for annoyance (\textbf{\cite{A}IV 408}; V 150), etc., by the second. Or with the states consisting of the floods, etc., that of the flood of sense desires, of the bond of sense desires, of the canker of sense desires, of sense-desire clinging, of the bodily tie of \marginnote{\textcolor{teal}{\footnotesize\{194|136\}}}{}covetousness, and of the fetter of greed for sense desires, is indicated by the first, and that of the remaining floods, bonds, cankers, clingings, ties, and fetters, is indicated by the second. Again, that of craving and of what is associated with craving is indicated by the first, and that of ignorance and of what is associated with ignorance is indicated by the second. Furthermore, that of the eight thought-arisings associated with greed (\hyperlink{XIV.90}{XIV.90}{}) is indicated by the first, and that of the remaining kinds of unprofitable thought-arisings is indicated by the second.

                This, in the first place, is the explanation of the meaning of the words “quite secluded from sense desires, secluded from unprofitable things.”

                

                \vismParagraph{IV.88}{88}{}
                So far the factors abandoned by the jhāna have been shown. And now, in order to show the factors associated with it, \emph{which is accompanied by applied and sustained thought} is said. \textcolor{brown}{\textit{[142]}} Herein, applied thinking (\emph{vitakkana}) is \emph{applied thought }(\emph{vitakka}); hitting upon, is what is meant.\footnote{\vismAssertFootnoteCounter{25}\vismHypertarget{IV.n25}{}\emph{Ūhana—}“hitting upon”: possibly connected with \emph{ūhanati }(to disturb—see \textbf{\cite{M}I 243}; II 193). Obviously connected here with the meaning of \emph{āhananapariyāhanana }(“striking and threshing”) in the next line. For the similes that follow here, see \textbf{\cite{Peṭ}142}.} It has the characteristic of directing the mind on to an object (mounting the mind on its object). Its function is to strike at and thresh—for the meditator is said, in virtue of it, to have the object struck at by applied thought, threshed by applied thought. It is manifested as the leading of the mind onto an object. Sustained thinking (\emph{vicaraṇa}) is \emph{sustained thought }(\emph{vicāra}); continued sustainment (\emph{anusañcaraṇa}), is what is meant. It has the characteristic of continued pressure on (occupation with) the object. Its function is to keep conascent [mental] states [occupied] with that. It is manifested as keeping consciousness anchored [on that object].

                \vismParagraph{IV.89}{89}{}
                And, though sometimes not separate, \emph{applied thought }is the first impact of the mind in the sense that it is both gross and inceptive, like the striking of a bell. \emph{Sustained thought }is the act of keeping the mind anchored, in the sense that it is subtle with the individual essence of continued pressure, like the ringing of the bell. \emph{Applied thought }intervenes, being the interference of consciousness at the time of first arousing [thought], like a bird’s spreading out its wings when about to soar into the air, and like a bee’s diving towards a lotus when it is minded to follow up the scent of it. The behaviour of \emph{sustained thought }is quiet, being the near non-interference of consciousness, like the bird’s planing with outspread wings after soaring into the air, and like the bee’s buzzing above the lotus after it has dived towards it.

                \vismParagraph{IV.90}{90}{}
                In the commentary to the Book of Twos\footnote{\vismAssertFootnoteCounter{26}\vismHypertarget{IV.n26}{}Of the Aṅguttara Nikāya? [The original could not be traced anywhere in the Tipiṭaka, Aṭṭhakathā, and other texts contained in the digitalised Chaṭṭha Saṅgāyana edition of the Vipassana Research Institute. \textbf{\cite{Dhs-a}114} quotes the same passage, but gives the source as \emph{aṭṭhakathāyaṃ}, “in the commentary.” BPS ed.]} this is said: “Applied thought occurs as a state of directing the mind onto an object, like the movement of a large bird taking off into the air by engaging the air with both wings and forcing them downwards. For it causes absorption by being unified. Sustained thought occurs with the individual essence of continued pressure, like the bird’s movement when it is using (activating) its wings for the purpose of keeping \marginnote{\textcolor{teal}{\footnotesize\{195|137\}}}{}hold on the air. For it keeps pressing the object\footnote{\vismAssertFootnoteCounter{27}\vismHypertarget{IV.n27}{}These two sentences, \emph{“So hi ekaggo hutvā appeti” }and \emph{“So hi ārammaṇaṃ anumajjati,” }are not in Be and Ae.}”. That fits in with the latter’s occurrence as anchoring. This difference of theirs becomes evident in the first and second jhānas [in the fivefold reckoning].

                \vismParagraph{IV.91}{91}{}
                Furthermore, \emph{applied thought }is like the hand that grips firmly and \emph{sustained thought }is like the hand that rubs, when one grips a tarnished metal dish firmly with one hand and rubs it with powder and oil and a woollen pad with the other hand. Likewise, when a potter has spun his wheel with a stroke on the stick and is making a dish \textcolor{brown}{\textit{[143]}}, his supporting hand is like \emph{applied thought }and his hand that moves back and forth is like \emph{sustained thought}. Likewise, when one is drawing a circle, the pin that stays fixed down in the centre is like \emph{applied thought}, which directs onto the object, and the pin that revolves round it is like \emph{sustained thought}, which continuously presses.

                \vismParagraph{IV.92}{92}{}
                So this jhāna occurs together with this applied thought and this sustained thought and it is called, “accompanied by applied and sustained thought” as a tree is called “accompanied by flowers and fruits.” But in the Vibhaṅga the teaching is given in terms of a person\footnote{\vismAssertFootnoteCounter{28}\vismHypertarget{IV.n28}{}\emph{Puggalādhiṭṭhāna—}“in terms of a person”; a technical commentarial term for one of the ways of presenting a subject. They are \emph{dhammā-desanā }(discourse about principles), and \emph{puggala-desanā }(discourse about persons), both of which may be treated either as \emph{dhammādhiṭṭhāna }(in terms of principles) or \emph{puggalādhiṭṭhāna }(in terms of persons). See \textbf{\cite{M-a}I 24}.} in the way beginning, “He is possessed, fully possessed, of this applied thought and this sustained thought” (\textbf{\cite{Vibh}257}). The meaning should be regarded in the same way there too.

                \vismParagraph{IV.93}{93}{}
                \emph{Born of seclusion}: here secludedness (\emph{vivitti}) is seclusion (\emph{viveka}); the meaning is, disappearance of hindrances. Or alternatively, it is secluded (\emph{vivitta}), thus it is seclusion; the meaning is, the collection of states associated with the jhāna is secluded from hindrances. “Born of seclusion” is born of or in that kind of seclusion.

                \vismParagraph{IV.94}{94}{}
                \emph{Happiness and bliss}: it refreshes (\emph{pīnayati}), thus it is happiness (\emph{pīti}). It has the characteristic of endearing (\emph{sampiyāyanā}). Its function is to refresh the body and the mind; or its function is to pervade (thrill with rapture). It is manifested as elation. But it is of five kinds as minor happiness, momentary happiness, showering happiness, uplifting happiness, and pervading (rapturous) happiness.

                Herein, \emph{minor happiness }is only able to raise the hairs on the body. \emph{Momentary happiness }is like flashes of lightning at different moments. \emph{Showering happiness }breaks over the body again and again like waves on the sea shore.

                \vismParagraph{IV.95}{95}{}
                \emph{Uplifting happiness }can be powerful enough to levitate the body and make it spring up into the air. For this was what happened to the Elder Mahā-Tissa, resident at Puṇṇavallika. He went to the shrine terrace on the evening of the full-moon day. Seeing the moonlight, he faced in the direction of the Great Shrine [at Anurādhapura], thinking, “At this very hour the four \marginnote{\textcolor{teal}{\footnotesize\{196|138\}}}{}assemblies\footnote{\vismAssertFootnoteCounter{29}\vismHypertarget{IV.n29}{}The four assemblies (\emph{parisā}) are the bhikkhus, bhikkhunīs, laymen followers and laywomen followers.} are worshipping at the Great Shrine!” By means of objects formerly seen [there] he aroused uplifting happiness with the Enlightened One as object, and he rose into the air like a painted ball bounced off a plastered floor and alighted on the terrace of the Great Shrine.

                \vismParagraph{IV.96}{96}{}
                And this was what happened to the daughter of a clan in the village of Vattakālaka near the Girikaṇḍaka Monastery when she sprang up into the air owing to strong uplifting happiness with the Enlightened One as object. As her parents were about to go to the monastery in the evening, it seems, in order to hear the Dhamma \textcolor{brown}{\textit{[144]}}, they told her: “My dear, you are expecting a child; you cannot go out at an unsuitable time. We shall hear the Dhamma and gain merit for you.” So they went out. And though she wanted to go too, she could not well object to what they said. She stepped out of the house onto a balcony and stood looking at the Ākāsacetiya Shrine at Girikaṇḍaka lit by the moon. She saw the offering of lamps at the shrine, and the four communities as they circumambulated it to the right after making their offerings of flowers and perfumes; and she heard the sound of the massed recital by the Community of Bhikkhus. Then she thought: “How lucky they are to be able to go to the monastery and wander round such a shrine terrace and listen to such sweet preaching of Dhamma!” Seeing the shrine as a mound of pearls and arousing uplifting happiness, she sprang up into the air, and before her parents arrived she came down from the air into the shrine terrace, where she paid homage and stood listening to the Dhamma.

                \vismParagraph{IV.97}{97}{}
                When her parents arrived, they asked her, “What road did you come by?” She said, “I came through the air, not by the road,” and when they told her, “My dear, those whose cankers are destroyed come through the air. But how did you come?” she replied: “As I was standing looking at the shrine in the moonlight a strong sense of happiness arose in me with the Enlightened One as its object. Then I knew no more whether I was standing or sitting, but only that I was springing up into the air with the sign that I had grasped, and I came to rest on this shrine terrace.”

                So uplifting happiness can be powerful enough to levitate the body, make it spring up into the air.

                \vismParagraph{IV.98}{98}{}
                But when \emph{pervading} (\emph{rapturous}) \emph{happiness} arises, the whole body is completely pervaded, like a filled bladder, like a rock cavern invaded by a huge inundation.

                \vismParagraph{IV.99}{99}{}
                Now, this fivefold happiness, when conceived and matured, perfects the twofold tranquillity, that is, bodily and mental tranquillity. When tranquillity is conceived and matured, it perfects the twofold bliss, that is, bodily and mental bliss. When bliss is conceived and matured, it perfects the threefold concentration, that is, momentary concentration, access concentration, and absorption concentration.

                Of these, what is intended in this context by happiness is pervading happiness, which is the root of absorption and comes by growth into association with absorption. \textcolor{brown}{\textit{[145]}}

                \vismParagraph{IV.100}{100}{}
                \marginnote{\textcolor{teal}{\footnotesize\{197|139\}}}{}But as to the other word: pleasing (\emph{sukhana}) is bliss (\emph{sukha}). Or alternatively: it thoroughly (\emph{SUṭṭhu}) devours (\emph{KHĀdati}), consumes (\emph{KHAṇati}),\footnote{\vismAssertFootnoteCounter{30}\vismHypertarget{IV.n30}{}For this word play see also \hyperlink{XVII.48}{XVII.48}{}. \emph{Khaṇati }is only given in normal meaning of “to dig” in PED. There seems to be some confusion of meaning with \emph{khayati }(to destroy) here, perhaps suggested by \emph{khādati }(to eat). This suggests a rendering here and in \hyperlink{XVII}{Ch. XVII}{} of “to consume” which makes sense. Glossed by \emph{avadāriyati, }to break or dig: not in PED. See CPD \emph{“avadārana.”}} bodily and mental affliction, thus it is bliss (\emph{sukha}). It has gratifying as its characteristic. Its function is to intensify associated states. It is manifested as aid.

                And wherever the two are associated, happiness is the contentedness at getting a desirable object, and bliss is the actual experiencing of it when got. Where there is happiness there is bliss (pleasure); but where there is bliss there is not necessarily happiness. Happiness is included in the formations aggregate; bliss is included in the feeling aggregate. If a man, exhausted\footnote{\vismAssertFootnoteCounter{31}\vismHypertarget{IV.n31}{}\emph{Kantāra-khinna—}“exhausted in a desert”; \emph{khinna }is not in PED.} in a desert, saw or heard about a pond on the edge of a wood, he would have happiness; if he went into the wood’s shade and used the water, he would have bliss. And it should be understood that this is said because they are obvious on such occasions.

                \vismParagraph{IV.101}{101}{}
                Accordingly, (a) this happiness and this bliss are of this jhāna, or in this jhāna; so in this way this jhāna is qualified by the words \emph{with happiness and bliss }[and also \emph{born of seclusion}]. Or alternatively: (b) the words \emph{happiness and bliss }(\emph{pītisukhaṃ}) can be taken as “the happiness and the bliss” independently, like “the Dhamma and the Discipline” (\emph{dhammavinaya}), and so then it can be taken as seclusion-born happiness-and-bliss of this jhāna, or in this jhāna; so in this way it is the happiness and bliss [rather than the jhāna] that are born of seclusion. For just as the words “born of seclusion” can [as at (a)] be taken as qualifying the word “jhāna,” so too they can be taken here [as at (b)] as qualifying the expression “happiness and bliss,” and then that [total expression] is predicated of this [jhāna]. So it is also correct to call “happiness-and-bliss born-of-seclusion” a single expression. In the Vibhaṅga it is stated in the way beginning, “This bliss accompanied by this happiness” (\textbf{\cite{Vibh}257}). The meaning should be regarded in the same way there too.

                \vismParagraph{IV.102}{102}{}
                \emph{First jhāna}: this will be explained below (\hyperlink{IV.119}{§119}{}).

                \emph{Enters upon }(\emph{upasampajja}): arrives at; reaches, is what is meant; or else, taking it as “makes enter” (\emph{upasampādayitvā}), then producing, is what is meant. In the Vibhaṅga this is said: “‘Enters upon’: the gaining, the regaining, the reaching, the arrival at, the touching, the realizing of, the entering upon (\emph{upasampadā}, the first jhāna” (\textbf{\cite{Vibh}257}), the meaning of which should be regarded in the same way.

                \vismParagraph{IV.103}{103}{}
                \emph{And dwells in }(\emph{viharati}): by becoming possessed of jhāna of the kind described above through dwelling in a posture favourable to that [jhāna], he produces a posture, a procedure, a keeping, an enduring, a lasting, a behaviour, a dwelling, of the person. For this is said in the Vibhaṅga: “‘Dwells in’: poses, \marginnote{\textcolor{teal}{\footnotesize\{198|140\}}}{}proceeds, keeps, endures, lasts, behaves, dwells; \textcolor{brown}{\textit{[146]}} hence ‘dwells’ is said” (\textbf{\cite{Vibh}252}).

                \vismParagraph{IV.104}{104}{}
                Now, it was also said above \emph{which abandons five factors, possesses five factors }(\hyperlink{IV.79}{§79}{}; cf. \textbf{\cite{M}I 294}). Herein, the abandoning of the five factors should be understood as the abandoning of these five hindrances, namely, lust, ill will, stiffness and torpor, agitation and worry, and uncertainty; for no jhāna arises until these have been abandoned, and so they are called the factors of abandoning. For although other unprofitable things too are abandoned at the moment of jhāna, still only these are specifically obstructive to jhāna.

                \vismParagraph{IV.105}{105}{}
                The mind affected through lust by greed for varied objective fields does not become concentrated on an object consisting in unity, or being overwhelmed by lust, it does not enter on the way to abandoning the sense-desire element. When pestered by ill will towards an object, it does not occur uninterruptedly. When overcome by stiffness and torpor, it is unwieldy. When seized by agitation and worry, it is unquiet and buzzes about. When stricken by uncertainty, it fails to mount the way to accomplish the attainment of jhāna. So it is these only that are called factors of abandoning because they are specifically obstructive to jhāna.

                \vismParagraph{IV.106}{106}{}
                But applied thought directs the mind onto the object; sustained thought keeps it anchored there. Happiness produced by the success of the effort refreshes the mind whose effort has succeeded through not being distracted by those hindrances; and bliss intensifies it for the same reason. Then unification aided by this directing onto, this anchoring, this refreshing and this intensifying, evenly and rightly centres (\hyperlink{III.3}{III.3}{}) the mind with its remaining associated states on the object consisting in unity. Consequently, possession of five factors should be understood as the arising of these five, namely, applied thought, sustained thought, happiness, bliss and unification of mind.

                \vismParagraph{IV.107}{107}{}
                For it is when these are arisen that jhāna is said to be arisen, which is why they are called the five factors of possession. Therefore it should not be assumed that the jhāna is something other which possesses them. But just as “The army with the four factors” (\textbf{\cite{Vin}IV 104}) and “Music with the five factors” (\textbf{\cite{M-a}II 300}) and “The path with the eight factors (eightfold path)” are stated simply in terms of their factors, so this too \textcolor{brown}{\textit{[147]}} should be understood as stated simply in terms of its factors, when it is said to “have five factors” or “possess five factors.”

                \vismParagraph{IV.108}{108}{}
                And while these five factors are present also at the moment of access and are stronger in access than in normal consciousness, they are still stronger here than in access and acquire the characteristic of the fine-material sphere. For applied thought arises here directing the mind on to the object in an extremely lucid manner, and sustained thought does so pressing the object very hard, and the happiness and bliss pervade the entire body. Hence it is said: “And there is nothing of his whole body not permeated by the happiness and bliss born of seclusion” (\textbf{\cite{D}I 73}). And unification too arises in the complete contact with the object that the surface of a box’s lid has with the surface of its base. This is how they differ from the others.

                \vismParagraph{IV.109}{109}{}
                \marginnote{\textcolor{teal}{\footnotesize\{199|141\}}}{}Although unification of mind is not actually listed among these factors in the [summary] version [beginning] “which is accompanied by applied and sustained thought” (\textbf{\cite{Vibh}245}), nevertheless it is mentioned [later] in the Vibhaṅga as follows: “‘Jhāna’: it is applied thought, sustained thought, happiness, bliss, unification”(\textbf{\cite{Vibh}257}), and so it is a factor too; for the intention with which the Blessed One gave the summary is the same as that with which he gave the exposition that follows it.

                \vismParagraph{IV.110}{110}{}
                \emph{Is good in three ways, possesses ten characteristics }(\hyperlink{IV.79}{§79}{}): the goodness in three ways is in the beginning, middle, and end. The possession of the ten characteristics should be understood as the characteristics of the beginning, middle, and end, too. Here is the text:

                \vismParagraph{IV.111}{111}{}
                “Of the first jhāna, purification of the way is the beginning, intensification of equanimity is the middle, and satisfaction is the end.

                “‘Of the first jhāna, purification of the way is the beginning’: how many characteristics has the beginning? The beginning has three characteristics: the mind is purified of obstructions to that [jhāna]; because it is purified the mind makes way for the central [state of equilibrium, which is the] sign of serenity; because it has made way the mind enters into that state. And it is since the mind becomes purified of obstructions and, through being purified, makes way for the central [state of equilibrium, which is the] sign of serenity and, having made way, enters into that state, that the purification of the way is the beginning of the first jhāna. These are the three characteristics of the beginning. Hence it is said: ‘The first jhāna is good in the beginning which possesses three characteristics.’ \textcolor{brown}{\textit{[148]}}

                \vismParagraph{IV.112}{112}{}
                “‘Of the first jhāna intensification of equanimity is the middle’: how many characteristics has the middle? The middle has three characteristics. He [now] looks on with equanimity at the mind that is purified; he looks on with equanimity at it as having made way for serenity; he looks on with equanimity at the appearance of unity.\footnote{\vismAssertFootnoteCounter{32}\vismHypertarget{IV.n32}{}Four unities \emph{(ekatta) }are given in the preceding paragraph of the same Paṭisambhidā ref.: “The unity consisting in the appearance of relinquishment in the act of giving, which is found in those resolved upon generosity (giving up); the unity consisting in the appearance of the sign of serenity, which is found in those who devote themselves to the higher consciousness; the unity consisting in the appearance of the characteristic of fall, which is found in those with insight; the unity consisting in the appearance of cessation, which is found in noble persons” (\textbf{\cite{Paṭis}I 167}). The second is meant here.} And in that he [now] looks on with equanimity at the mind that is purified and looks on with equanimity at it as having made way for serenity and looks on with equanimity at the appearance of unity, that intensification of equanimity is the middle of the first jhāna. These are the three characteristics of the middle. Hence it is said: ‘The first jhāna is good in the middle which possesses three characteristics.’

                \vismParagraph{IV.113}{113}{}
                “‘Of the first jhāna satisfaction is the end’: how many characteristics has the end? The end has four characteristics. The satisfaction in the sense that there was non-excess of any of the states arisen therein, and the satisfaction in the sense that the faculties had a single function, and the satisfaction in the sense \marginnote{\textcolor{teal}{\footnotesize\{200|142\}}}{}that the appropriate energy was effective, and the satisfaction in the sense of repetition, are the satisfaction in the end of the first jhāna. These are the four characteristics of the end. Hence it is said: ‘The first jhāna is good in the end which possesses four characteristics’” (\textbf{\cite{Paṭis}I 167–168}).

                \vismParagraph{IV.114}{114}{}
                Herein, \emph{purification of the way }is access together with its concomitants. \emph{Intensification of equanimity }is absorption. \emph{Satisfaction }is reviewing. So some comment.\footnote{\vismAssertFootnoteCounter{33}\vismHypertarget{IV.n33}{}“The inmates of the Abhayagiri Monastery in Anurādhapura” (\textbf{\cite{Vism-mhṭ}144}).} But it is said in the text, “The mind arrived at unity enters into purification of the way, is intensified in equanimity, and is satisfied by knowledge” (\textbf{\cite{Paṭis}I 167}), and therefore it is from the standpoint within actual absorption that \emph{purification of the way }firstly should be understood as the approach, with \emph{intensification of equanimity }as the function of equanimity consisting in specific neutrality, and \emph{satisfaction} as the manifestation of clarifying knowledge’s function in accomplishing non-excess of states. How?

                \vismParagraph{IV.115}{115}{}
                Firstly, in a cycle [of consciousness] in which absorption arises the mind becomes purified from the group of defilements called hindrances that are an obstruction to jhāna. Being devoid of obstruction because it has been purified, it makes way for the central [state of equilibrium, which is the] sign of serenity. Now, it is the absorption concentration itself occurring evenly that is called \emph{the sign of serenity}. But the consciousness immediately before that \textcolor{brown}{\textit{[149]}} reaches that state by way of change in a single continuity (cf. \hyperlink{XXII.1}{XXII.1}{}–\hyperlink{XXII.6}{6}{}), and so it is said that it \emph{makes way for the central [state of equilibrium, which is the] sign of serenity}. And it is said that it \emph{enters into that state} by approaching it through having made way for it. That is why in the first place \emph{purification of the way}, while referring to aspects existing in the preceding consciousness, should nevertheless be understood as the approach at the moment of the first jhāna’s actual arising.

                \vismParagraph{IV.116}{116}{}
                Secondly, when he has more interest in purifying, since there is no need to re-purify what has already been purified thus, it is said that \emph{he looks on with equanimity at the mind that is purified}. And when he has no more interest in concentrating again what has already made way for serenity by arriving at the state of serenity, it is said that \emph{he looks on with equanimity at it as having made way for serenity}. And when he has no more interest in again causing appearance of unity in what has already appeared as unity through abandonment of its association with defilement in making way for serenity, it is said that \emph{he looks on with equanimity at the appearance of unity}. That is why \emph{intensification of equanimity }should be understood as the function of equanimity that consists in specific neutrality.

                \vismParagraph{IV.117}{117}{}
                And lastly, when equanimity was thus intensified, the states called concentration and understanding produced there, occurred coupled together without either one exceeding the other. And also the [five] faculties beginning with faith occurred with the single function (taste) of deliverance owing to deliverance from the various defilements. And also the energy appropriate to that, which was favourable to their state of non-excess and single function, was \marginnote{\textcolor{teal}{\footnotesize\{201|143\}}}{}effective. And also its repetition occurs at that moment.\footnote{\vismAssertFootnoteCounter{34}\vismHypertarget{IV.n34}{}“‘\emph{Its}’: of that jhāna consciousness. ‘\emph{At that moment’: }at the moment of dissolution; for when the moment of arising is past, repetition occurs starting with the moment of presence” (\textbf{\cite{Vism-mhṭ}145}). A curious argument; see §182.} Now, all these [four] aspects are only produced because it is after seeing with knowledge the various dangers in defilement and advantages in cleansing that satisfiedness, purifiedness and clarifiedness ensue accordingly. That is the reason why it was said that \emph{satisfaction }should be understood as the manifestation of clarifying knowledge’s function in accomplishing non-excess, etc., of states (\hyperlink{IV.114}{§114}{}).

                \vismParagraph{IV.118}{118}{}
                Herein, satisfaction as a function of knowledge is called “the end” since the knowledge is evident as due to onlooking equanimity, according as it is said: “He looks on with complete equanimity at the mind thus exerted; then the understanding faculty is outstanding as understanding due to equanimity. Owing to equanimity the mind is liberated from the many sorts of defilements; then the understanding faculty is outstanding as understanding due to liberation. Because of being liberated these states come to have a single function; then [the understanding faculty is outstanding as understanding due to] development in the sense of the single function”\footnote{\vismAssertFootnoteCounter{35}\vismHypertarget{IV.n35}{}The quotation is incomplete and the end should read, “… \emph{ekarasaṭṭhena bhāvanāvasena paññāvasena paññindriyaṃ adhimattaṃ hoti}.”} (\textbf{\cite{Paṭis}II 25}).

                \vismParagraph{IV.119}{119}{}
                Now, as to the words \emph{and so he has attained the first jhāna … of the earth kasiṇa }(\hyperlink{IV.79}{§79}{}): Here it is \emph{first }because it starts a numerical series; \textcolor{brown}{\textit{[150]}} also it is first because it arises first. It is called \emph{jhāna }because of lighting (\emph{upanijjhāna}) the object and because of burning up (\emph{jhāpana}) opposition (\textbf{\cite{Paṭis}I 49}). The disk of earth is called \emph{earth kasiṇa }(\emph{paṭhavīkasiṇa—}lit. “earth universal”) in the sense of entirety,\footnote{\vismAssertFootnoteCounter{36}\vismHypertarget{IV.n36}{}“In the sense of the jhāna’s entire object. It is not made its partial object” (\textbf{\cite{Vism-mhṭ}147}).} and the sign acquired with that as its support and also the jhāna acquired in the earth-kasiṇa sign are so called too. So that jhāna should be understood as \emph{of the earth kasiṇa }in this sense, with reference to which it was said above “and so he has attained to the first jhāna … of the earth kasiṇa.”

                \vismParagraph{IV.120}{120}{}
                When it has been attained in this way, the mode of its attainment must be discerned by the meditator as if he were a hair-splitter or a cook. For when a very skilful archer, who is working to split a hair, actually splits the hair on one occasion, he discerns the modes of the position of his feet, the bow, the bowstring, and the arrow thus: “I split the hair as I stood thus, with the bow thus, the bowstring thus, the arrow thus.” From then on he recaptures those same modes and repeats the splitting of the hair without fail. So too the meditator must discern such modes as that of suitable food, etc., thus: “I attained this after eating this food, attending on such a person, in such a lodging, in this posture at this time.” In this way, when that [absorption] is lost, he will be able to recapture those modes and renew the absorption, or while familiarizing himself with it he will be able to repeat that absorption again and again.

                \vismParagraph{IV.121}{121}{}
                And just as when a skilled cook is serving his employer, he notices whatever he chooses to eat and from then on brings only that sort and so obtains \marginnote{\textcolor{teal}{\footnotesize\{202|144\}}}{}a reward, so too this meditator discerns such modes as that of the food, etc., at the time of the attaining, and he recaptures them and re-obtains absorption each time it is lost. So he must discern the modes as a hair-splitter or a cook does.

                \vismParagraph{IV.122}{122}{}
                And this has been said by the Blessed One: “Bhikkhus, suppose a wise, clever, skilful cook set various kinds of sauces before a king or a king’s minister, such as sour, bitter, sharp, \textcolor{brown}{\textit{[151]}} sweet, peppery and unpeppery, salty and unsalty sauces; then the wise, clever, skilful cook learned his master’s sign thus ‘today this sauce pleased my master’ or ‘he held out his hand for this one’ or ‘he took a lot of this one’ or ‘he praised this one’ or ‘today the sour kind pleased my master’ or ‘he held out his hand for the sour kind’ or ‘he took a lot of the sour kind’ or ‘he praised the sour kind’ … or ‘he praised the unsalty kind’; then the wise, clever, skilful cook is rewarded with clothing and wages and presents. Why is that? Because that wise, clever, skilful cook learned his master’s sign in this way. So too, bhikkhus, here a wise, clever, skilful bhikkhu dwells contemplating the body as a body … He dwells contemplating feelings as feelings … consciousness as consciousness … mental objects as mental objects, ardent, fully aware and mindful, having put away covetousness and grief for the world. As he dwells contemplating mental objects as mental objects, his mind becomes concentrated, his defilements are abandoned. He learns the sign of that. Then that wise, clever, skilful bhikkhu is rewarded with a happy abiding here and now, he is rewarded with mindfulness and full awareness. Why is that? Because that wise, clever, skilful bhikkhu learned his consciousness’s sign” (\textbf{\cite{S}V 151–152}).

                \vismParagraph{IV.123}{123}{}
                And when he recaptures those modes by apprehending the sign, he just succeeds in reaching absorption, but not in making it last. It lasts when it is absolutely purified from states that obstruct concentration.

                \vismParagraph{IV.124}{124}{}
                When a bhikkhu enters upon a jhāna without [first] completely suppressing lust by reviewing the dangers in sense desires, etc., and without [first] completely tranquillizing bodily irritability\footnote{\vismAssertFootnoteCounter{37}\vismHypertarget{IV.n37}{}\emph{Kāya-duṭṭhulla—“}bodily irritability”: explained here as “bodily disturbance \emph{(daratha), }excitement of the body \emph{(kāya-sāraddhatā)” }by\textbf{\cite{Vism-mhṭ}(p. 148)}; here it represents the hindrance of ill will; cf. \textbf{\cite{M}III 151}, 159, where commented on as \emph{kāyālasiya—}“bodily inertia” (\textbf{\cite{M-a}IV 202}, 208). PED, only gives meaning of “wicked, lewd” for \emph{duṭṭhulla}, for which meaning see e.g. \textbf{\cite{A}I 88}, \textbf{\cite{Vin-a}528}; cf. \hyperlink{IX.69}{IX.69}{}.} by tranquillizing the body, and without [first] completely removing stiffness and torpor by bringing to mind the elements of initiative, etc., (\hyperlink{IV.55}{§55}{}), and without [first] completely abolishing agitation and worry by bringing to mind the sign of serenity, etc., \textcolor{brown}{\textit{[152]}} and without [first] completely purifying his mind of other states that obstruct concentration, then that bhikkhu soon comes out of that jhāna again, like a bee that has gone into an unpurified hive, like a king who has gone into an unclean park.

                \vismParagraph{IV.125}{125}{}
                But when he enters upon a jhāna after [first] completely purifying his mind of states that obstruct concentration, then he remains in the attainment even for a whole day, like a bee that has gone into a completely purified hive, like a king who has gone into a perfectly clean park. Hence the Ancients said:
                \begin{verse}
                    \marginnote{\textcolor{teal}{\footnotesize\{203|145\}}}{}“So let him dispel any sensual lust, and resentment,\\{}
                    Agitation as well, and then torpor, and doubt as the fifth;\\{}
                    There let him find joy with a heart that is glad in seclusion,\\{}
                    Like a king in a garden where all and each corner is clean.”
                \end{verse}


                \vismParagraph{IV.126}{126}{}
                So if he wants to remain long in the jhāna, he must enter upon it after [first] purifying his mind from obstructive states.
            \section[\vismAlignedParas{§126–130}Extension of the sign]{Extension of the sign}

                In order to perfect the development of consciousness he should besides extend the counterpart sign according as acquired. Now, there are two planes for extension, namely, access and absorption; for it is possible to extend it on reaching access and on reaching absorption. But the extending should be done consistently in one [or the other], which is why it was said “he should besides extend the counterpart sign according as acquired.”

                \vismParagraph{IV.127}{127}{}
                The way to extend it is this. The meditator should not extend the sign as a clay bowl or a cake or boiled rice or a creeper or a piece of cloth is extended. He should first delimit with his mind successive sizes for the sign, according as acquired, that is to say, one finger, two fingers, three fingers, four fingers, and then extend it by the amount delimited, just as a ploughman delimits with the plough the area to be ploughed and then ploughs within the area delimited, or just as bhikkhus fixing a boundary first observe the marks and then fix it. He should not, in fact, extend it without having delimited [the amount it is to be extended by]. After that has been done, he can further extend it, doing so by delimiting successive boundaries of, say, a span, a \emph{ratana }(=2 spans), the veranda, the surrounding space,\footnote{\vismAssertFootnoteCounter{38}\vismHypertarget{IV.n38}{}For \emph{pamukha—}“veranda” see n. 2 above. \emph{Pariveṇa—}“surrounding space”: this meaning, not given in PED, is brought out clearly in \hyperlink{XI.7}{XI.7}{}.} the monastery, and the boundaries of the village, the town, the district, the kingdom and the ocean, \textcolor{brown}{\textit{[153]}} making the extreme limit the world-sphere or even beyond.

                \vismParagraph{IV.128}{128}{}
                Just as young swans first starting to use their wings soar a little distance at a time, and by gradually increasing it eventually reach the presence of the moon and sun, so too when a bhikkhu extends the sign by successive delimitations in the way described, he can extend it up to the limit of the world-sphere or even beyond.

                \vismParagraph{IV.129}{129}{}
                Then that sign [appears] to him like an ox hide stretched out with a hundred pegs\footnote{\vismAssertFootnoteCounter{39}\vismHypertarget{IV.n39}{}\emph{Samabbhāhata—}“stretch flat”: not in this sense in PED. This word replaces the word \emph{suvihata} used at \textbf{\cite{M}III 105} where this clause is borrowed from. At \hyperlink{XI.92}{XI.92}{}, the same word (apparently in another sense) is glossed by \emph{pellana} = “pushing” (not in PED) at Vism-mhṭ 362. \textbf{\cite{M-a}IV 153} glosses \emph{suvihata} with “\emph{pasāretvā suṭṭhu vihata}” which suggests “stretched” rather than “beaten”; \emph{harati} rather than \emph{hanati}.} over the earth’s ridges and hollows, river ravines, tracts of scrub and thorns, and rocky inequalities (see \textbf{\cite{M}III 105}) in any area to which it has been extended. \marginnote{\textcolor{teal}{\footnotesize\{204|146\}}}{}When a beginner has reached the first jhāna in this sign, he should enter upon it often without reviewing it much. For the first jhāna factors occur crudely and weakly in one who reviews it much. Then because of that they do not become conditions for higher endeavour. While he is endeavouring for the unfamiliar [higher jhāna] he falls away from the first jhāna and fails to reach the second.

                \vismParagraph{IV.130}{130}{}
                Hence the Blessed One said: “Bhikkhus, suppose there were a foolish stupid mountain cow, with no knowledge of fields and no skill in walking on craggy mountains, who thought: ‘What if I walked in a direction I never walked in before, ate grass I never ate before, drank water I never drank before?’ and without placing her forefoot properly she lifted up her hind foot; then she would not walk in the direction she never walked in before or eat the grass she never ate before or drink the water she never drank before, and also she would not get back safely to the place where she had thought, ‘What if I walked in a direction I never walked in before … drank water I never drank before?’ Why is that? Because that mountain cow was foolish and stupid with no knowledge of fields and no skill in walking on craggy mountains. So too, bhikkhus, here is a certain foolish stupid bhikkhu with no knowledge of fields and no skill, quite secluded from sense desires, secluded from unprofitable things, in entering upon and dwelling in the first jhāna, which is accompanied by applied thought and sustained thought with happiness and bliss born of seclusion; he does not repeat, develop or cultivate that sign or properly establish it. He thinks: ‘What if with the subsiding of applied and sustained thought I entered upon and dwelt in the second jhāna, which is … with happiness and bliss born of concentration?’ \textcolor{brown}{\textit{[154]}} He is unable with the subsiding of applied and sustained thought to enter upon and dwell in the second jhāna, which is … with happiness and bliss born of concentration. Then he thinks: ‘What if, quite secluded from sense desires, secluded from unprofitable things, I entered upon and dwelt in the first jhāna, which is … with happiness and bliss born of seclusion?’ He is unable, quite secluded from sense desires, secluded from unprofitable things, to enter upon and dwell in the first jhāna which is … with happiness and bliss born of seclusion. This bhikkhu is called one who has slipped between the two, who has fallen between the two, just like the foolish stupid mountain cow with no knowledge of fields and no skill in walking on craggy mountains …” (\textbf{\cite{A}IV 418–419}).
            \section[\vismAlignedParas{§131–138}Mastery in five ways]{Mastery in five ways}

                \vismParagraph{IV.131}{131}{}
                Therefore he should acquire mastery in the five ways first of all with respect to the first jhāna. Herein, these are the five kinds of mastery: mastery in adverting, mastery in attaining, mastery in resolving (steadying the duration), mastery in emerging, and mastery in reviewing. “He adverts to the first jhāna where, when, and for as long as, he wishes; he has no difficulty in adverting; thus it is mastery in adverting. He attains the first jhāna where … he has no difficulty in attaining; thus it is mastery in attaining” (\textbf{\cite{Paṭis}I 100}), and all the rest should be quoted in detail (\hyperlink{XXIII.27}{XXIII.27}{}).

                \vismParagraph{IV.132}{132}{}
                The explanation of the meaning here is this. When he emerges from the first jhāna and first of all adverts to the applied thought, then, next to the \marginnote{\textcolor{teal}{\footnotesize\{205|147\}}}{}adverting that arose interrupting the life-continuum, either four or five impulsions impel with that applied thought as their object. Then there are two life-continuum [consciousnesses]. Then there is adverting with the sustained thought as its object and followed by impulsions in the way just stated. When he is able to prolong his conscious process uninterruptedly in this way with the five jhāna factors, then his mastery of adverting is successful. But this mastery is found at its acme of perfection in the Blessed One’s Twin Marvel (\textbf{\cite{Paṭis}I 125}), or for others on the aforesaid occasions. There is no quicker mastery in adverting than that.

                \vismParagraph{IV.133}{133}{}
                The venerable Mahā-Moggallāna’s ability to enter upon jhāna quickly, as in the taming of the royal nāga-serpent Nandopananda (\hyperlink{XII.106}{XII.106f.}{}), is called mastery in attaining.

                \vismParagraph{IV.134}{134}{}
                Ability to remain in jhāna for a moment consisting in exactly a finger-snap or exactly ten finger-snaps is called mastery in resolving (steadying the duration).

                Ability to emerge quickly in the same way is called mastery in emerging.

                \vismParagraph{IV.135}{135}{}
                The story of the Elder Buddharakkhita may be told in order to illustrate both these last. \textcolor{brown}{\textit{[155]}} Eight years after his admission to the Community that elder was sitting in the midst of thirty thousand bhikkhus possessed of supernormal powers who had gathered to attend upon the sickness of the Elder Mahā-Rohanagutta at Therambatthala. He saw a royal supaṇṇa (bird) swooping down from the sky intending to seize an attendant royal nāga-serpent as he was getting rice-gruel accepted for the elder. The Elder Buddharakkhita created a rock meanwhile, and seizing the royal nāga by the arm, he pushed him inside it. The royal supaṇṇa gave the rock a blow and made off. The senior elder remarked: “Friends, if Rakkhita had not been there, we should all have been put to shame.”\footnote{\vismAssertFootnoteCounter{40}\vismHypertarget{IV.n40}{}What the story is trying to illustrate is the rapidity with which the elder entered the jhāna, controlled its duration, and emerged, which is the necessary preliminary to the working of a marvel (the creation of a rock in this case; \hyperlink{XII.57}{XII.57}{}). The last remark seems to indicate that all the others would have been too slow (see \textbf{\cite{Vism-mhṭ}150}).}

                \vismParagraph{IV.136}{136}{}
                Mastery in reviewing is described in the same way as mastery in adverting; for the reviewing impulsions are in fact those next to the adverting mentioned there (\hyperlink{IV.132}{§132}{}).

                \vismParagraph{IV.137}{137}{}
                When he has once acquired mastery in these five ways, then on emerging from the now familiar first jhāna he can regard the flaws in it in this way: “This attainment is threatened by the nearness of the hindrances, and its factors are weakened by the grossness of the applied and sustained thought.” He can bring the second jhāna to mind as quieter and so end his attachment to the first jhāna and set about doing what is needed for attaining the second.

                \vismParagraph{IV.138}{138}{}
                When he has emerged from the first jhāna, applied and sustained thought appear gross to him as he reviews the jhāna factors with mindfulness and full awareness, while happiness and bliss and unification of mind appear peaceful. Then, as he brings that same sign to mind as “earth, earth” again and again \marginnote{\textcolor{teal}{\footnotesize\{206|148\}}}{}with the purpose of abandoning the gross factors and obtaining the peaceful factors, [knowing] “now the second jhāna will arise,” there arises in him mind-door adverting with that same earth kasiṇa as its object, interrupting the life-continuum. After that, either four or five impulsions impel on that same object, the last one of which is an impulsion of the fine-material sphere belonging to the second jhāna. The rest are of the sense sphere of the kinds already stated (\hyperlink{IV.74}{§74}{}).
            \section[\vismAlignedParas{§139–152}The second jhāna]{The second jhāna}

                \vismParagraph{IV.139}{139}{}
                And at this point, “With the stilling of applied and sustained thought he enters upon and dwells in the second jhāna, which has internal confidence and singleness of mind without applied thought, without sustained thought, with happiness and bliss born of concentration” (\textbf{\cite{Vibh}245}), and so he has attained the second jhāna, which abandons two factors, possesses three factors, is good in three ways, possesses ten characteristics and is of the earth kasiṇa. \textcolor{brown}{\textit{[156]}}

                \vismParagraph{IV.140}{140}{}
                Herein, \emph{with the stilling of applied and sustained thought}: with the stilling, with the surmounting, of these two, namely, applied thought and sustained thought; with their non-manifestation at the moment of the second jhāna, is what is meant. Herein, although none of the states belonging to the first jhāna exist in the second jhāna—for the contact, etc. (see \textbf{\cite{M}III 25}), in the first jhāna are one and here they are another—it should be understood all the same that the phrase “with the stilling of applied and sustained thought” is expressed in this way in order to indicate that the attaining of the other jhānas, beginning with that of the second from the first, is effected by the surmounting of the gross factor in each case.

                \vismParagraph{IV.141}{141}{}
                \emph{Internal}: here one’s own internal\footnote{\vismAssertFootnoteCounter{41}\vismHypertarget{IV.n41}{}See \hyperlink{XIV.192}{XIV.192}{} and note.} is intended; but that much is actually stated in the Vibhaṅga too with the words “internally in oneself” (\textbf{\cite{Vibh}258}). And since one’s own internal is intended, the meaning here is this: born in oneself, generated in one’s own continuity.

                \vismParagraph{IV.142}{142}{}
                \emph{Confidence}: it is faith that is called confidence. The jhāna “has confidence” because it is associated with confidence as a cloth “has blue colour” because it is associated with blue colour. Or alternatively, that jhāna is stated to “have confidence” because it makes the mind confident with the confidence possessed by it and by stilling the disturbance created by applied and sustained thought. And with this conception of the meaning the word construction must be taken as “confidence of mind.” But with the first-mentioned conception of the meaning the words “of mind” must be construed with “singleness\footnote{\vismAssertFootnoteCounter{42}\vismHypertarget{IV.n42}{}In the Pali, \emph{sampasādanaṃ cetaso ekodibhāvaṃ: cetaso} (“of mind”) comes between \emph{sampasādanaṃ }(“confidence”) and \emph{ekodibhāvaṃ }(“singleness”) and so can be construed with either.}”.

                \vismParagraph{IV.143}{143}{}
                Here is the construction of the meaning in that case. Unique (\emph{eka}) it comes up (\emph{udeti}), thus it is single (\emph{ekodi}); the meaning is, it comes up as the superlative, the best, because it is not overtopped by applied and sustained thought, for the best is called “unique” in the world. Or it is permissible to say that when deprived \marginnote{\textcolor{teal}{\footnotesize\{207|149\}}}{}of applied and sustained thought it is unique, without companion. Or alternatively: it evokes (\emph{udāyati}) associated states, thus it is an evoker (\emph{udi}); the meaning is, it arouses. And that is unique (\emph{eka}) in the sense of best, and it is an evoker (\emph{udi}), thus it is a unique evoker (\emph{ekodi }= single). This is a term for concentration. Then, since the second jhāna gives existingness to (\emph{bhāveti}), augments, this single [thing], it “gives singleness” (\emph{ekodibhāva}). But as this single [thing] is a mind’s, not a being’s or a soul’s, so singleness of mind is said.

                \vismParagraph{IV.144}{144}{}
                It might be asked: But does not this faith exist in the first jhāna too, and also this concentration with the name of the “single [thing]?” Then why is only this second jhāna said to have confidence and singleness of mind?—It may be replied as follows: It is because that first jhāna \textcolor{brown}{\textit{[157]}} is not fully confident owing to the disturbance created by applied and sustained thought, like water ruffled by ripples and wavelets. That is why, although faith does exist in it, it is not called “confidence.” And there too concentration is not fully evident because of the lack of full confidence. That is why it is not called “singleness” there. But in this second jhāna faith is strong, having got a footing in the absence of the impediments of applied and sustained thought; and concentration is also evident through having strong faith as its companion. That may be understood as the reason why only this jhāna is described in this way.

                \vismParagraph{IV.145}{145}{}
                But that much is actually stated in the Vibhaṅga too with the words: “‘Confidence’ is faith, having faith, trust, full confidence. ‘Singleness of mind’ is steadiness of consciousness … right concentration” (\textbf{\cite{Vibh}258}). And this commentary on the meaning should not be so understood as to conflict with the meaning stated in that way, but on the contrary so as to agree and concur with it.

                \vismParagraph{IV.146}{146}{}
                \emph{Without applied thought, without sustained thought}: since it has been abandoned by development, there is no applied thought in this, or of this, [jhāna], thus it is without applied thought. The same explanation applies to sustained thought. Also it is said in the Vibhaṅga: “So this applied thought and this sustained thought are quieted, quietened, stilled, set at rest, set quite at rest, done away with, quite done away with,\footnote{\vismAssertFootnoteCounter{43}\vismHypertarget{IV.n43}{}\emph{Appita—}“done away with”: \emph{Appitā ti vināsaṃ gamitā} (“\emph{Appita}” means “made to go to annihilation”) (\textbf{\cite{Vism-mhṭ}153}). This meaning, though not in PED, is given in CPD.} dried up, quite dried up, made an end of; hence it is said: without applied thought, without sustained thought” (\textbf{\cite{Vibh}258}).

                Here it may be asked: Has not this meaning already been established by the words “with the stilling of applied and sustained thought?” So why is it said again “without applied thought, without sustained thoughts?”—It may be replied: Yes, that meaning has already been established. But this does not indicate that meaning. Did we not say earlier: “The phrase ‘with the stilling of applied and sustained thought’ is expressed in this way in order to indicate that the act of attaining the other jhānas, beginning with that of the second from the first, is effected by the surmounting of the gross factor in each case?” (\hyperlink{IV.140}{§140}{}).

                \vismParagraph{IV.147}{147}{}
                Besides, this confidence comes about with the act of stilling, not the darkness of defilement, but the applied and sustained thought. And the \marginnote{\textcolor{teal}{\footnotesize\{208|150\}}}{}singleness comes about, not as in access jhāna with the abandoning of the hindrances, nor as in the first jhāna with the manifestation of the factors, but with the act of stilling the applied and sustained thought. So that [first] clause indicates the cause of the confidence and singleness. In the same way this jhāna is without applied thought and without sustained thought, not as in the third and fourth jhānas or as in eye-consciousness, etc., with just absence, but with the actual act of stilling the applied and sustained thought. So that [first clause] also indicates the cause of the state without applied and sustained thought; it does not indicate the bare absence of applied and sustained thought. \textcolor{brown}{\textit{[158]}} The bare absence of applied and sustained thought is indicated by this [second] clause, namely, “without applied thought, without sustained thought.” Consequently it needs to be stated notwithstanding that the first has already been stated.

                \vismParagraph{IV.148}{148}{}
                \emph{Born of concentration}: born of the first-jhāna concentration, or born of associated concentration, is the meaning. Herein, although the first was born of associated concentration too, still it is only this concentration that is quite worthy to be called “concentration” because of its complete confidence and extreme immobility due to absence of disturbance by applied and sustained thought. So only this [jhāna] is called “born of concentration,” and that is in order to recommend it.

                \emph{With happiness and bliss} is as already explained. \emph{Second}: second in numerical series. Also second because entered upon second.

                \vismParagraph{IV.149}{149}{}
                Then it was also said above \emph{which abandons two factors, possesses three factors }(\hyperlink{IV.139}{§139}{}). Herein, the abandoning of two factors should be understood as the abandoning of applied thought and sustained thought. But while the hindrances are abandoned at the moment of the access of the first jhāna, in the case of this jhāna the applied thought and sustained thought are not abandoned at the moment of its access. It is only at the moment of actual absorption that the jhāna arises without them. Hence they are called its factors of abandoning.

                \vismParagraph{IV.150}{150}{}
                Its possession of three factors should be understood as the arising of the three, that is, happiness, bliss, and unification of mind. So when it is said in the Vibhaṅga, “‘Jhāna’: confidence, happiness, bliss, unification of mind” (\textbf{\cite{Vibh}258}), this is said figuratively in order to show that jhāna with its equipment. But, excepting the confidence, this jhāna has literally three factors \emph{qua }factors that have attained to the characteristic of lighting (see \hyperlink{IV.119}{§119}{}), according as it is said: “What is jhāna of three factors on that occasion? It is happiness, bliss, unification of mind” (\textbf{\cite{Vibh}263}).

                The rest is as in the case of the first jhāna.

                \vismParagraph{IV.151}{151}{}
                Once this has been obtained in this way, and he has mastery in the five ways already described, then on emerging from the now familiar second jhāna he can regard the flaws in it thus: “This attainment is threatened by the nearness of applied and sustained thought; ‘Whatever there is in it of happiness, of mental excitement, proclaims its grossness’ (\textbf{\cite{D}I 37}), and its factors are weakened by the grossness of the happiness so expressed.” He can bring the third jhāna to mind \marginnote{\textcolor{teal}{\footnotesize\{209|151\}}}{}as quieter and so end his attachment to the second jhāna and set about doing what is needed for attaining the third.

                \vismParagraph{IV.152}{152}{}
                When he has emerged from the second jhāna \textcolor{brown}{\textit{[159]}} happiness appears gross to him as he reviews the jhāna factors with mindfulness and full awareness, while bliss and unification appear peaceful. Then as he brings that same sign to mind as “earth, earth” again and again with the purpose of abandoning the gross factor and obtaining the peaceful factors, [knowing] “now the third jhāna will arise,” there arises in him mind-door adverting with that same earth kasiṇa as its object, interrupting the life-continuum. After that, either four or five impulsions impel on that same object, the last one of which is an impulsion of the fine-material sphere belonging to the third jhāna. The rest are of the kinds already stated (\hyperlink{IV.74}{§74}{}).
            \section[\vismAlignedParas{§153–182}The third jhāna]{The third jhāna}

                \vismParagraph{IV.153}{153}{}
                And at this point, “With the fading away of happiness as well he dwells in equanimity, and mindful and fully aware, he feels bliss with his body; he enters upon and dwells in the third jhāna, on account of which the Noble Ones announce: ‘He dwells in bliss who has equanimity and is mindful’ (\textbf{\cite{Vibh}245}), and so he has attained the third jhāna, which abandons one factor, possesses two factors, is good in three ways, possesses ten characteristics, and is of the earth kasiṇa.

                \vismParagraph{IV.154}{154}{}
                Herein, \emph{with the fading away of happiness as well }(\emph{pītiyā ca virāgā}): fading away is distaste for, or surmounting of, happiness of the kind already described. But the words “as well” (\emph{ca}) between the two [words \emph{pītiyā }and \emph{virāgā}] have the meaning of a conjunction;\footnote{\vismAssertFootnoteCounter{44}\vismHypertarget{IV.n44}{}\emph{Sampiṇḍana—}“conjunction”: gram. term for the word \emph{ca }(and). This meaning not given in PED. Cf. \textbf{\cite{M-a}I 40}.} they conjoin [to them] either the word “stilling” or the expression “the stilling of applied and sustained thought” [in the description of the second jhāna]. Herein, when taken as conjoining “stilling” the construction to be understood is “with the fading away and, what is more, with the stilling, of happiness.” With this construction “fading away” has the meaning of distaste; so the meaning can be regarded as “with distaste for, and with the stilling of, happiness.” But when taken as conjoining the words “stilling of applied and sustained thought,” then the construction to be understood is “with the fading of happiness and, further, with the stilling of applied and sustained thought.” With this construction “fading away” has the meaning of surmounting; so this meaning can be regarded as “with the surmounting of happiness and with the stilling of applied and sustained thought.”

                \vismParagraph{IV.155}{155}{}
                Of course, applied and sustained thought have already been stilled in the second jhāna, too. However, this is said in order to show the path to this third jhāna and in order to recommend it. For when “with the stilling of applied and sustained thought” is said, it is declared that the path to this jhāna is necessarily by the stilling of applied and sustained thought. And just as, although mistaken view of individuality, etc., are not abandoned in the attaining of the third noble path [but in the first], yet when it is recommended by describing their \marginnote{\textcolor{teal}{\footnotesize\{210|152\}}}{}abandonment thus, “With the abandoning of the five lower fetters” (\textbf{\cite{A}I 232}), \textcolor{brown}{\textit{[160]}} then it awakens eagerness in those trying to attain that third noble path—so too, when the stilling of applied and sustained thought is mentioned, though they are not actually stilled here [but in the second], this is a recommendation. Hence the meaning expressed is this: “With the surmounting of happiness and with the stilling of applied and sustained thought.”

                \vismParagraph{IV.156}{156}{}
                \emph{He dwells in equanimity}: it watches [things] as they arise (\emph{UPApattito IKKHATI}), thus it is equanimity (\emph{upekkhā—}or onlooking); it sees fairly, sees without partiality (\emph{a-pakkha-patita}), is the meaning. A possessor of the third jhāna is said to “dwell in equanimity” since he possesses equanimity that is clear, abundant and sound.

                Equanimity is of ten kinds; six-factored equanimity, equanimity as a divine abiding, equanimity as an enlightenment factor, equanimity of energy, equanimity about formations, equanimity as a feeling, equanimity about insight, equanimity as specific neutrality, equanimity of jhāna and equanimity of purification.

                \vismParagraph{IV.157}{157}{}
                Herein, \emph{six factored equanimity }is a name for the equanimity in one whose cankers are destroyed. It is the mode of non-abandonment of the natural state of purity when desirable or undesirable objects of the six kinds come into focus in the six doors described thus: “Here a bhikkhu whose cankers are destroyed is neither glad nor sad on seeing a visible object with the eye: he dwells in equanimity, mindful and fully aware” (\textbf{\cite{A}III 279}).

                \vismParagraph{IV.158}{158}{}
                \emph{Equanimity as a divine abiding }is a name for equanimity consisting in the mode of neutrality towards beings described thus: “He dwells intent upon one quarter with his heart endued with equanimity” (\textbf{\cite{D}I 251}).

                \vismParagraph{IV.159}{159}{}
                \emph{Equanimity as an enlightenment factor }is a name for equanimity consisting in the mode of neutrality in conascent states described thus: “He develops the equanimity enlightenment factor depending on relinquishment” (\textbf{\cite{M}I 11}).

                \vismParagraph{IV.160}{160}{}
                \emph{Equanimity of energy }is a name for the equanimity otherwise known as neither over-strenuous nor over-lax energy described thus: “From time to time he brings to mind the sign of equanimity” (\textbf{\cite{A}I 257}).

                \vismParagraph{IV.161}{161}{}
                \emph{Equanimity about formations }is a name for equanimity consisting in neutrality about apprehending reflexion and composure regarding the hindrances, etc., described thus: “How many kinds of equanimity about formations arise through concentration? How many kinds of equanimity about formations arise through insight? Eight kinds of equanimity about formations arise through concentration. Ten kinds of equanimity about formations arise through insight”\footnote{\vismAssertFootnoteCounter{45}\vismHypertarget{IV.n45}{}The “eight kinds” are those connected with the eight jhānas, the “ten kinds” those connected with the four paths, the four fruitions, the void liberation, and the signless liberation.} (\textbf{\cite{Paṭis}I 64}). \textcolor{brown}{\textit{[161]}}

                \vismParagraph{IV.162}{162}{}
                \marginnote{\textcolor{teal}{\footnotesize\{211|153\}}}{}\emph{Equanimity as a feeling }is a name for the equanimity known as neither-pain-nor-pleasure described thus: “On the occasion on which a sense-sphere profitable consciousness has arisen accompanied by equanimity” (\textbf{\cite{Dhs}§156}).

                \vismParagraph{IV.163}{163}{}
                \emph{Equanimity about insight }is a name for equanimity consisting in neutrality about investigation described thus: “What exists, what has become, that he abandons, and he obtains equanimity” (\textbf{\cite{M}II 264–265}, \textbf{\cite{A}IV 70}f).

                \vismParagraph{IV.164}{164}{}
                \emph{Equanimity as specific neutrality} is a name for equanimity consisting in the equal efficiency of conascent states; it is contained among the “or-whatever states” beginning with zeal (\hyperlink{XIV.133}{XIV.133}{}; \textbf{\cite{Dhs-a}132}).

                \vismParagraph{IV.165}{165}{}
                \emph{Equanimity of jhāna} is a name for equanimity producing impartiality towards even the highest bliss described thus: “He dwells in equanimity” (\textbf{\cite{Vibh}245}).

                \vismParagraph{IV.166}{166}{}
                \emph{Purifying equanimity }is a name for equanimity purified of all opposition, and so consisting in uninterestedness in stilling opposition described thus: “The fourth jhāna, which … has mindfulness purified by equanimity” (\textbf{\cite{Vibh}245}).

                \vismParagraph{IV.167}{167}{}
                Herein, six-factored equanimity, equanimity as a divine abiding, equanimity as an enlightenment factor, equanimity as specific neutrality, equanimity of jhāna and purifying equanimity are one in meaning, that is, equanimity as specific neutrality. Their difference, however, is one of position,\footnote{\vismAssertFootnoteCounter{46}\vismHypertarget{IV.n46}{}\emph{Avatthā—“}position, occasion.” Not in PED; see CPD.} like the difference in a single being as a boy, a youth, an adult, a general, a king, and so on. Therefore of these it should be understood that equanimity as an enlightenment factor, etc., are not found where there is six-factored equanimity; or that six-factored equanimity, etc., are not found where there is equanimity as an enlightenment factor.

                And just as these have one meaning, so also equanimity about formations and equanimity about insight have one meaning too; for they are simply understanding classed in these two ways according to function.

                \vismParagraph{IV.168}{168}{}
                Just as, when a man has seen a snake go into his house in the evening and has hunted for it with a forked stick, and then when he has seen it lying in the grain store and has looked to discover whether it is actually a snake or not, and then by seeing three marks\footnote{\vismAssertFootnoteCounter{47}\vismHypertarget{IV.n47}{}\emph{Sovatthika-ttaya—}”three marks;” cf. \hyperlink{XXI.49}{XXI.49}{}.} has no more doubt, and so there is neutrality in him about further investigating whether or not it is a snake, \textcolor{brown}{\textit{[162]}} so too, when a man has begun insight, and he sees with insight knowledge the three characteristics, then there is neutrality in him about further investigating the impermanence, etc., of formations, and that neutrality is called \emph{equanimity about insight}.

                \vismParagraph{IV.169}{169}{}
                But just as, when the man has caught hold of the snake securely with the forked stick and thinks, “How shall I get rid of the snake without hurting it or getting bitten by it?” then as he is seeking only the way to get rid of it, there is neutrality in him about the catching hold of it, so too, when a man, through seeking the three characteristics, sees the three kinds of becoming as if burning, \marginnote{\textcolor{teal}{\footnotesize\{212|154\}}}{}then there is neutrality in him about catching hold of formations, and that neutrality is called \emph{equanimity about formations}.

                \vismParagraph{IV.170}{170}{}
                So when equanimity about insight is established, equanimity about formations is established too. But it is divided into two in this way according to function, in other words, according to neutrality about investigating and about catching hold.

                \emph{Equanimity of energy }and \emph{equanimity as feeling }are different both from each other and from the rest.

                \vismParagraph{IV.171}{171}{}
                So, of these kinds of equanimity, it is equanimity of jhāna that is intended here. That has the characteristic of neutrality. Its function is to be unconcerned. It is manifested as uninterestedness. Its proximate cause is the fading away of happiness.

                Here it may be said: Is this not simply equanimity as specific neutrality in the meaning? And that exists in the first and second jhānas as well; so this clause, “He dwells in equanimity,” ought to be stated of those also. Why is it not?—[It may be replied:] Because its function is unevident there since it is overshadowed by applied thought and the rest. But it appears here with a quite evident function, with head erect, as it were, because it is not overshadowed by applied thought and sustained thought and happiness. That is why it is stated here.

                The commentary on the meaning of the clause “He dwells in equanimity” is thus completed in all its aspects.

                \vismParagraph{IV.172}{172}{}
                Now, as to \emph{mindful and fully aware}: here, he remembers (\emph{sarati}), thus he is mindful (\emph{sata}). He has full awareness (\emph{sampajānāti}), thus he is fully aware (\emph{sampajāna}). This is mindfulness and full awareness stated as personal attributes. Herein, mindfulness has the characteristic of remembering. Its function is not to forget. It is manifested as guarding. Full awareness has the characteristic of non-confusion. Its function is to investigate (judge). It is manifested as scrutiny.

                \vismParagraph{IV.173}{173}{}
                Herein, although this mindfulness and this full awareness exist in the earlier jhānas as well—for one who is forgetful and not fully aware does not attain even access, let alone absorption—yet, because of the [comparative] grossness of those jhānas, the mind’s going is easy [there], like that of a man on [level] ground, and so the functions of mindfulness and full awareness are not evident in them. \textcolor{brown}{\textit{[163]}} But it is only stated here because the subtlety of this jhāna, which is due to the abandoning of the gross factors, requires that the mind’s going always includes the functions of mindfulness and full awareness, like that of a man on a razor’s edge.

                \vismParagraph{IV.174}{174}{}
                What is more, just as a calf that follows a cow returns to the cow when taken away from her if not prevented, so too, when this third jhāna is led away from happiness, it would return to happiness if not prevented by mindfulness and full awareness, and would rejoin happiness. And besides, beings are greedy for bliss, and this kind of bliss is exceedingly sweet since there is none greater. But here there is non-greed for the bliss owing to the influence of the mindfulness and full awareness, not for any other reason. And so it should also be understood that it is stated only here in order to emphasize this meaning too.

                \vismParagraph{IV.175}{175}{}
                \marginnote{\textcolor{teal}{\footnotesize\{213|155\}}}{}Now, as to the clause \emph{he feels bliss with his body}: here, although in one actually possessed of the third jhāna there is no concern about feeling bliss, nevertheless he would feel the bliss associated with his mental body, and after emerging from the jhāna he would also feel bliss since his material body would have been affected by the exceedingly superior matter originated by that bliss associated with the mental body.\footnote{\vismAssertFootnoteCounter{48}\vismHypertarget{IV.n48}{}For consciousness-originated materiality see \hyperlink{XX.30}{XX.30}{} ff.} It is in order to point to this meaning that the words “he feels bliss with his body” are said.

                \vismParagraph{IV.176}{176}{}
                Now, as to the clause, that … on account of which the Noble Ones announce: He dwells in bliss who has equanimity and is mindful: here it is the jhāna, on account of which as cause, on account of which as reason, the Noble Ones, that is to say, the Enlightened Ones, etc., “announce, teach, declare, establish, reveal, expound, explain, clarify” (\textbf{\cite{Vibh}259}) that person who possesses the third jhāna—they praise, is what is intended. Why? Because “he dwells in bliss who has equanimity and is mindful. He enters upon and dwells in that third jhāna” (\emph{taṃ … tatiyaṃ jhānaṃ upasampajja viharati}) is how the construction should be understood here. But why do they praise him thus? Because he is worthy of praise.

                \vismParagraph{IV.177}{177}{}
                For this man is worthy of praise since he has equanimity towards the third jhāna though it possesses exceedingly sweet bliss and has reached the perfection of bliss, and he is not drawn towards it by a liking for the bliss, and he is mindful with the mindfulness established in order to prevent the arising of happiness, and he feels with his mental body the undefiled bliss beloved of Noble Ones, cultivated by Noble Ones. Because he is worthy of praise in this way, it should be understood, Noble Ones praise him with the words, “He dwells in bliss who has equanimity and is mindful,” thus declaring the special qualities that are worthy of praise.

                \textcolor{brown}{\textit{[164]}} \emph{Third}: it is the third in the numerical series; and it is third because it is entered upon third.

                \vismParagraph{IV.178}{178}{}
                Then it was said, \emph{which abandons one factor, possesses two factors }(\hyperlink{IV.153}{§153}{}): here the abandoning of the one factor should be understood as the abandoning of happiness. But that is abandoned only at the moment of absorption, as applied thought and sustained thought are at that of the second jhāna; hence it is called its factor of abandoning.

                \vismParagraph{IV.179}{179}{}
                The possession of the two factors should be understood as the arising of the two, namely, bliss and unification. So when it is said in the Vibhaṅga, “‘Jhāna’: equanimity, mindfulness, full awareness, bliss, unification of mind” (\textbf{\cite{Vibh}260}), this is said figuratively in order to show that jhāna with its equipment. But, excepting the equanimity and mindfulness and full awareness, this jhāna has literally only two factors \emph{qua }factors that have attained to the characteristic of lighting (see \hyperlink{IV.119}{§119}{}), according as it is said, “What is the jhāna of two factors on that occasion? It is bliss and unification of mind” (\textbf{\cite{Vibh}264}).

                The rest is as in the case of the first jhāna.

                \vismParagraph{IV.180}{180}{}
                \marginnote{\textcolor{teal}{\footnotesize\{214|156\}}}{}Once this has been obtained in this way, and once he has mastery in the five ways already described, then on emerging from the now familiar third jhāna, he can regard the flaws in it thus: “This attainment is threatened by the nearness of happiness; ‘Whatever there is in it of mental concern about bliss proclaims its grossness’ (\textbf{\cite{D}I 37}; see \hyperlink{IX.n20}{Ch. IX, n. 20}{}), and its factors are weakened by the grossness of the bliss so expressed.” He can bring the fourth jhāna to mind as quieter and so end his attachment to the third jhāna and set about doing what is needed for attaining the fourth.

                \vismParagraph{IV.181}{181}{}
                When he has emerged from the third jhāna, the bliss, in other words, the mental joy, appears gross to him as he reviews the jhāna factors with mindfulness and full awareness, while the equanimity as feeling and the unification of mind appear peaceful. Then, as he brings that same sign to mind as “earth, earth” again and again with the purpose of abandoning the gross factor and obtaining the peaceful factors, [knowing] “now the fourth jhāna will arise,” there arises in him mind-door adverting with that same earth kasiṇa as its object, interrupting the life-continuum. After that either four or five impulsions impel on that same object, \textcolor{brown}{\textit{[165]}} the last one of which is an impulsion of the fine-material sphere belonging to the fourth jhāna. The rest are of the kinds already stated (\hyperlink{IV.74}{§74}{}).

                \vismParagraph{IV.182}{182}{}
                But there is this difference: blissful (pleasant) feeling is not a condition, as repetition condition, for neither-painful-nor-pleasant feeling, and [the preliminary work] must be aroused in the case of the fourth jhāna with neither-painful-nor-pleasant feeling; consequently these [consciousnesses of the preliminary work] are associated with neither-painful-nor-pleasant feeling, and here happiness vanishes simply owing to their association with equanimity.
            \section[\vismAlignedParas{§183–197}The fourth jhāna]{The fourth jhāna}

                \vismParagraph{IV.183}{183}{}
                And at this point, “With the abandoning of pleasure and pain and with the previous disappearance of joy and grief he enters upon and dwells in the fourth jhāna, which has neither-pain-nor-pleasure and has purity of mindfulness due to equanimity” (\textbf{\cite{Vibh}245}), and so he has attained the fourth jhāna, which abandons one factor, possesses two factors, is good in three ways, possesses ten characteristics, and is of the earth kasiṇa.

                \vismParagraph{IV.184}{184}{}
                Herein, \emph{with the abandoning of pleasure and pain}: with the abandoning of bodily pleasure and bodily pain. \emph{With the previous}: which took place before, not in the moment of the fourth jhāna. \emph{Disappearance of joy and grief}: with the previous disappearance of the two, that is, mental bliss (pleasure) and mental pain; with the abandoning, is what is meant.

                \vismParagraph{IV.185}{185}{}
                But when does the abandoning of these take place? At the moment of access of the four jhānas. For [mental] joy is only abandoned at the moment of the fourth-jhāna access, while [bodily] pain, [mental] grief, and [bodily] bliss (pleasure) are abandoned respectively at the moments of access of the first, second, and third jhānas. So although the order in which they are abandoned is not actually mentioned, nevertheless the abandoning of the pleasure, pain, joy, and grief, is stated here according to the order in which the faculties are summarized in the Indriya Vibhaṅga (\textbf{\cite{Vibh}122}).

                \vismParagraph{IV.186}{186}{}
                \marginnote{\textcolor{teal}{\footnotesize\{215|157\}}}{}But if these are only abandoned at the moments of access of the several jhānas, why is their cessation said to. take place in the jhāna itself in the following passage: “And where does the arisen pain faculty cease without remainder? Here, bhikkhus, quite secluded from sense desires, secluded from unprofitable things, a bhikkhu enters upon and dwells in the first jhāna, which is … born of seclusion. It is here that the arisen pain faculty ceases without remainder … Where does the arisen grief faculty [cease without remainder? … in the second jhāna] … Where does the arisen pleasure faculty [cease without remainder? … in the third jhāna] … Where does the arisen joy faculty cease without remainder? \textcolor{brown}{\textit{[166]}} Here, bhikkhus, with the abandoning of pleasure and pain [and with the previous disappearance of joy and grief] a bhikkhu enters upon and dwells in the fourth jhāna, which … has mindfulness purified by equanimity. It is here that the arisen joy faculty ceases without remainder” (\textbf{\cite{S}V 213–215}).

                It is said in that way there referring to reinforced cessation. For in the first jhāna, etc., it is their reinforced cessation, not just their cessation, that takes place. At the moment of access it is just their cessation, not their reinforced cessation, that takes place.

                \vismParagraph{IV.187}{187}{}
                For accordingly, during the first jhāna access, which has multiple adverting, there could be rearising of the [bodily] pain faculty\footnote{\vismAssertFootnoteCounter{49}\vismHypertarget{IV.n49}{}“They say that with the words, ‘There could be the arising of the pain faculty,’ it is shown that since grief arises even in obtainers of jhāna, it is demonstrated thereby that hate can exist without being a hindrance just as greed can; for grief does not arise without hate. Nor, they say, is there any conflict with the Paṭṭhāna text to be fancied here, since what is shown there is only grief that occurs making lost jhāna its object because the grief that occurs making its object a jhāna that has not been lost is not relevant there. And they say that it cannot be maintained that grief does not arise at all in those who have obtained jhāna since it did arise in Asita who had the eight attainments (Sn 691), and he was not one who had lost jhāna. So they say. That is wrong because there is no hate without the nature of a hindrance. If there were, it would arise in fine-material and immaterial beings, and it does not. Accordingly when in such passages as, ‘In the immaterial state, due to the hindrance of lust there is the hindrance of stiffness and torpor … the hindrance of agitation, the hindrance of ignorance’ (\textbf{\cite{Paṭṭh}II 291}), ill will and worry are not mentioned as hindrances, that does not imply that they are not hindrances even by supposing that it was because lust, etc., were not actually hindrances and were called hindrances there figuratively because of resemblance to hindrances. And it is no reason to argue, ‘it is because it arose in Asita,’ since there is falling away from jhāna with the arising of grief. The way to regard that is that when the jhāna is lost for some trivial reason such men reinstate it without difficulty” (\textbf{\cite{Vism-mhṭ}158–159}).} due to contact with gadflies, flies, etc. or the discomfort of an uneven seat, though that pain faculty had already ceased, but not so during absorption. Or else, though it has ceased during access, it has not absolutely ceased there since it is not quite beaten out by opposition. But during absorption the whole body is showered with bliss owing to pervasion by happiness. And the pain faculty has absolutely ceased in one whose body is showered with bliss, since it is beaten out then by opposition.

                \vismParagraph{IV.188}{188}{}
                \marginnote{\textcolor{teal}{\footnotesize\{216|158\}}}{}And during the second-jhāna access too, which has multiple advertings, there could be rearising of the [mental] grief faculty, although it had already ceased there, because it arises when there is bodily weariness and mental vexation, which have applied thought and sustained thought as their condition, but it does not arise when applied and sustained thought are absent. When it arises, it does so in the presence of applied and sustained thought, and they are not abandoned in the second-jhāna access; but this is not so in the second jhāna itself because its conditions are abandoned there.

                \vismParagraph{IV.189}{189}{}
                Likewise in the third-jhāna access there could be rearising of the abandoned [bodily] pleasure faculty in one whose body was pervaded by the superior materiality originated by the [consciousness associated with the] happiness. But not so in the third jhāna itself. For in the third jhāna the happiness that is a condition for the [bodily] bliss (pleasure) has ceased entirely. Likewise in the fourth-jhāna access there could be re-arising of the abandoned [mental] joy faculty because of its nearness and because it has not been properly surmounted owing to the absence of equanimity brought to absorption strength. But not so in the fourth jhāna itself. And that is why in each case (\hyperlink{IV.186}{§186}{}) the words “without remainder” are included thus: “It is here that the arisen pain faculty ceases without remainder.”

                \vismParagraph{IV.190}{190}{}
                Here it may be asked: Then if these kinds of feeling are abandoned in the access in this way, why are they brought in here? It is done so that they can be readily grasped. For the neither-painful-nor-pleasant feeling described here by the words “which has neither-pain-nor-pleasure” is subtle, hard to recognize and not readily grasped. So just as, when a cattle-herd\footnote{\vismAssertFootnoteCounter{50}\vismHypertarget{IV.n50}{}\emph{Gopa—}“cowherd (or guardian)”: not in PED.} wants to catch a refractory ox that cannot be caught at all by approaching it, he collects all the cattle into one pen \textcolor{brown}{\textit{[167]}} and lets them out one by one, and then [he says] “That is it; catch it,” and so it gets caught as well, so too the Blessed One has collected all these [five kinds of feeling] together so that they can be grasped readily; for when they are shown collected together in this way, then what is not [bodily] pleasure (bliss) or [bodily] pain or [mental] joy or [mental] grief can still be grasped in this way: “This is neither-painful-nor-pleasant feeling.”

                \vismParagraph{IV.191}{191}{}
                Besides, this may be understood as said in order to show the condition for the neither-painful-nor-pleasant mind-deliverance. For the abandoning of [bodily] pain, etc., are conditions for that, according as it is said: “There are four conditions, friend, for the attainment of the neither-painful-nor-pleasant mind-deliverance. Here, friend, with the abandoning of pleasure and pain and with the previous disappearance of joy and grief a bhikkhu enters upon and dwells in the fourth jhāna … equanimity. These are the four conditions for the attainment of the neither-painful-nor-pleasant mind-deliverance” (\textbf{\cite{M}I 296}).

                \vismParagraph{IV.192}{192}{}
                Or alternatively, just as, although mistaken view of individuality, etc., have already been abandoned in the earlier paths, they are nevertheless mentioned as abandoned in the description of the third path for the purpose of recommending it (cf. \hyperlink{IV.155}{§155}{}), so too these kinds of feeling can be understood \marginnote{\textcolor{teal}{\footnotesize\{217|159\}}}{}as mentioned here for the purpose of recommending this jhāna. Or alternatively, they can be understood as mentioned for the purpose of showing that greed and hate are very far away owing to the removal of their conditions; for of these, pleasure (bliss) is a condition for joy, and joy for greed; pain is a condition for grief and grief for hate. So with the removal of pleasure (bliss), etc., greed and hate are very far away since they are removed along with their conditions.

                \vismParagraph{IV.193}{193}{}
                \emph{Which has neither-pain-nor-pleasure}: no pain owing to absence of pain; no pleasure owing to absence of pleasure (bliss). By this he indicates the third kind of feeling that is in opposition both to pain and to pleasure, not the mere absence of pain and pleasure. This third kind of feeling named neither-pain-nor-pleasure is also called “equanimity.” It has the characteristic of experiencing what is contrary to both the desirable and the undesirable. Its function is neutral. Its manifestation is unevident. Its proximate cause should be understood as the cessation of pleasure (bliss).

                \vismParagraph{IV.194}{194}{}
                \emph{And has purity of mindfulness due to equanimity}: has purity of mindfulness brought about by equanimity. For the mindfulness in this jhāna is quite purified, and its purification is effected by equanimity, not by anything else. That is why it is said to have purity of mindfulness due to equanimity. Also it is said in the Vibhaṅga: “This mindfulness is cleared, purified, clarified, by equanimity; hence it is said to have purity of mindfulness due to equanimity” (\textbf{\cite{Vibh}261}). \textcolor{brown}{\textit{[168]}} And the equanimity due to which there comes to be this purity of mindfulness should be understood as specific neutrality in meaning. And not only mindfulness is purified by it here, but also all associated states. However, the teaching is given under the heading of mindfulness.

                \vismParagraph{IV.195}{195}{}
                Herein, this equanimity exists in the three lower jhānas too; but just as, although a crescent moon exists by day but is not purified or clear since it is outshone by the sun’s radiance in the daytime or since it is deprived of the night, which is its ally owing to gentleness and owing to helpfulness to it, so too, this crescent moon of equanimity consisting in specific neutrality exists in the first jhāna, etc., but it is not purified since it is outshone by the glare of the opposing states consisting in applied thought, etc., and since it is deprived of the night of equanimity-as-feeling for its ally; and because it is not purified, the conascent mindfulness and other states are not purified either, like the unpurified crescent moon’s radiance by day. That is why no one among these [first three jhānas] is said to have purity of mindfulness due to equanimity. But here this crescent moon consisting in specific neutrality is utterly pure because it is not outshone by the glare of the opposing states consisting in applied thought, etc., and because it has the night of equanimity-as-feeling for its ally. And since it is purified, the conascent mindfulness and other states are purified and clear also, like the purified crescent moon’s radiance. That, it should be understood, is why only this jhāna is said to have purity of mindfulness due to equanimity.

                \vismParagraph{IV.196}{196}{}
                \emph{Fourth}: it is fourth in numerical series; and it is fourth because it is entered upon fourth.

                \vismParagraph{IV.197}{197}{}
                \marginnote{\textcolor{teal}{\footnotesize\{218|160\}}}{}Then it was said, \emph{which abandons one factor, possesses two factors }(\hyperlink{IV.183}{§183}{}); here the abandoning of the one factor should be understood as the abandoning of joy. But that joy is actually abandoned in the first impulsions of the same cognitive series (cf. \hyperlink{IV.185}{§185}{}). Hence it is called its factor of abandoning.

                The possession of the two factors should be understood as the arising of the two, namely, equanimity as feeling and unification of mind.

                The rest is as stated in the case of the first jhāna.

                This, in the first place, is according to the fourfold reckoning of jhāna.
            \section[\vismAlignedParas{§198–202}The fivefold reckoning of jhāna]{The fivefold reckoning of jhāna}

                \vismParagraph{IV.198}{198}{}
                When, however, he is developing fivefold jhāna, then, on emerging from the now familiar first jhāna, he can regard the flaws in it in this way: “This attainment is threatened by the nearness of the hindrances, and its factors are weakened by the grossness of applied thought.” \textcolor{brown}{\textit{[169]}} He can bring the second jhāna to mind as quieter and so end his attachment to the first jhāna and set about doing what is needed for attaining the second.

                \vismParagraph{IV.199}{199}{}
                Now, he emerges from the first jhāna mindfully and fully aware; and only applied thought appears gross to him as he reviews the jhāna factors, while the sustained thought, etc., appear peaceful. Then, as he brings that same sign to mind as “earth, earth” again and again with the purpose of abandoning the gross factor and obtaining the peaceful factors, the second jhāna arises in him in the way already described.

                Its factor of abandoning is applied thought only. The four beginning with sustained thought are the factors that it possesses. The rest is as already stated.

                \vismParagraph{IV.200}{200}{}
                When this has been obtained in this way, and once he has mastery in the five ways already described, then on emerging from the now familiar second jhāna he can regard the flaws in it in this way: “This attainment is threatened by the nearness of applied thought, and its factors are weakened by the grossness of sustained thought.” He can bring the third jhāna to mind as quieter and so end his attachment to the second jhāna and set about doing what is needed for attaining the third.

                \vismParagraph{IV.201}{201}{}
                Now, he emerges from the second jhāna mindfully and fully aware; only sustained thought appears gross to him as he reviews the jhāna factors, while happiness, etc., appear peaceful. Then, as he brings that same sign to mind as “earth, earth” again and again with the purpose of abandoning the gross factor and obtaining the peaceful factors, the third jhāna arises in him in the way already described.

                Its factor of abandoning is sustained thought only. The three beginning with happiness, as in the second jhāna in the fourfold reckoning, are the factors that it possesses. The rest is as already stated.

                \vismParagraph{IV.202}{202}{}
                So that which is the second in the fourfold reckoning becomes the second and third in the fivefold reckoning by being divided into two. And those which \marginnote{\textcolor{teal}{\footnotesize\{219|161\}}}{}are the third and fourth in the former reckoning become the fourth and fifth in this reckoning. The first remains the first in each case.

                The fourth chapter called “The Description of the Earth Kasiṇa” in the Treatise on the Development of Concentration in the \emph{Path of Purification} composed for the purpose of gladdening good people.
        \chapter[The Remaining Kasiṇas]{The Remaining Kasiṇas\vismHypertarget{V}\newline{\textnormal{\emph{Sesa-kasiṇa-niddesa}}}}
            \label{V}

            \section[\vismAlignedParas{§1–4}The Water Kasiṇa]{The Water Kasiṇa}

                \vismParagraph{V.1}{1}{}
                \marginnote{\textcolor{teal}{\footnotesize\{220|162\}}}{}\textcolor{brown}{\textit{[170]}} Now, the water kasiṇa comes next after the earth kasiṇa (\hyperlink{III.105}{III.105}{}). Here is the detailed explanation.

                One who wants to develop the water kasiṇa should, as in the case of the earth kasiṇa, seat himself comfortably and apprehend the sign in water that “is either made up or not made up,” etc.; and so all the rest should be repeated in detail (\hyperlink{IV.22}{IV.22}{}). And as in this case, so with all those that follow [in this chapter]. We shall in fact not repeat even this much and shall only point out what is different.

                \vismParagraph{V.2}{2}{}
                Here too, when someone has had practice in previous [lives], the sign arises for him in water that is not made up, such as a pool, a lake, a lagoon, or the ocean as in the case of the Elder Cūḷa-Sīva. The venerable one, it seems, thought to abandon gain and honour and live a secluded life. He boarded a ship at Mahātittha (Mannar) and sailed to Jambudīpa (India). As he gazed at the ocean meanwhile, the kasiṇa sign, the counterpart of that ocean, arose in him.

                \vismParagraph{V.3}{3}{}
                Someone with no such previous practice should guard against the four faults of a kasiṇa (\hyperlink{IV.24}{IV.24}{}) and not apprehend the water as one of the colours, blue, yellow, red or white. He should fill a bowl or a four-footed water pot\footnote{\vismAssertFootnoteCounter{1}\vismHypertarget{V.n1}{}\emph{Kuṇḍika—}“a four-footed water pot”: not in PED.} to the brim with water uncontaminated by soil, taken in the open through a clean cloth [strainer], or with any other clear unturbid water. He should put it in a screened place on the outskirts of the monastery as already described and seat himself comfortably. He should neither review its colour nor bring its characteristic to mind. Apprehending the colour as belonging to its physical support, he should set his mind on the [name] concept as the most outstanding mental datum, and using any among the [various] names for water (\emph{āpo}) such as “rain” (\emph{ambu}), “liquid” (\emph{udaka}), “dew” (\emph{vāri}), “fluid” (\emph{salila}),\footnote{\vismAssertFootnoteCounter{2}\vismHypertarget{V.n2}{}English cannot really furnish five words for water.} he should develop [the kasiṇa] by using [preferably] the obvious “water, water.”

                \vismParagraph{V.4}{4}{}
                As he develops it in this way, the two signs eventually arise in him in the way already described. Here, however, the learning sign has the appearance of moving. \textcolor{brown}{\textit{[171]}} If the water has bubbles of froth mixed with it, the learning sign has the \marginnote{\textcolor{teal}{\footnotesize\{221|163\}}}{}same appearance, and it is evident as a fault in the kasiṇa. But the counterpart sign appears inactive, like a crystal fan set in space, like the disk of a looking-glass made of crystal. With the appearance of that sign he reaches access jhāna and the jhāna tetrad and pentad in the way already described.
            \section[\vismAlignedParas{§5–8}The Fire Kasiṇa]{The Fire Kasiṇa}

                \vismParagraph{V.5}{5}{}
                Anyone who wants to develop the fire kasiṇa should apprehend the sign in fire. Herein, when someone with merit, having had previous practice, is apprehending the sign, it arises in him in any sort of fire, not made up, as he looks at the fiery combustion in a lamp’s flame or in a furnace or in a place for baking bowls or in a forest conflagration, as in the Elder Cittagutta’s case. The sign arose in that elder as he was looking at a lamp’s flame while he was in the Uposatha house on the day of preaching the Dhamma.

                \vismParagraph{V.6}{6}{}
                Anyone else should make one up. Here are the directions for making it. He should split up some damp heartwood, dry it, and break it up into short lengths. He should go to a suitable tree root or to a shed and there make a pile in the way done for baking bowls, and have it lit. He should make a hole a span and four fingers wide in a rush mat or a piece of leather or a cloth, and after hanging it in front of the fire, he should sit down in the way already described. Instead of giving attention to the grass and sticks below or the smoke above, he should apprehend the sign in the dense combustion in the middle.

                \vismParagraph{V.7}{7}{}
                He should not review the colour as blue or yellow, etc., or give attention to its characteristic as heat, etc., but taking the colour as belonging to its physical support, and setting his mind on the [name] concept as the most outstanding mental datum, and using any among the names for fire (\emph{tejo}) such as “the Bright One” (\emph{pāvaka}), “the Leaver of the Black Trail” (\emph{kaṇhavattani}), “the Knower of Creatures” (\emph{jātaveda}), “the Altar of Sacrifice” (\emph{hutāsana}), etc., he should develop [the kasiṇa] by using [preferably] the obvious “fire, fire.”

                \vismParagraph{V.8}{8}{}
                As he develops it in this way the two signs eventually arise in him as already described. Herein, the learning sign appears like [the fire to keep] sinking down as the flame keeps detaching itself. \textcolor{brown}{\textit{[172]}} But when someone apprehends it in a kasiṇa that is not made up, any fault in the kasiṇa is evident [in the learning sign], and any firebrand, or pile of embers or ashes, or smoke appears in it. The counterpart sign appears motionless like a piece of red cloth set in space, like a gold fan, like a gold column. With its appearance he reaches access jhāna and the jhāna tetrad and pentad in the way already described.
            \section[\vismAlignedParas{§9–11}The Air Kasiṇa]{The Air Kasiṇa}

                \vismParagraph{V.9}{9}{}
                Anyone who wants to develop the air kasiṇa should apprehend the sign in air. And that is done either by sight or by touch. For this is said in the Commentaries: “One who is learning the air kasiṇa apprehends the sign in air. He notices the tops of [growing] sugarcane moving to and fro; or he notices the tops of bamboos, or the tops of trees, or the ends of the hair, moving to and fro; or he notices the touch of it on the body.”

                \vismParagraph{V.10}{10}{}
                \marginnote{\textcolor{teal}{\footnotesize\{222|164\}}}{}So when he sees sugarcanes with dense foliage standing with tops level or bamboos or trees, or else hair four fingers long on a man’s head, being struck by the wind, he should establish mindfulness in this way: “This wind is striking on this place.” Or he can establish mindfulness where the wind strikes a part of his body after entering by a window opening or by a crack in a wall, and using any among the names for wind (\emph{vāta}) beginning with “wind” (\emph{vāta}), “breeze” (\emph{māluta}), “blowing” (\emph{anila}), he should develop [the kasiṇa] by using [preferably] the obvious “air, air.”

                \vismParagraph{V.11}{11}{}
                Here the learning sign appears to move like the swirl of hot [steam] on rice gruel just withdrawn from an oven. The counterpart sign is quiet and motionless. The rest should be understood in the way already described.
            \section[\vismAlignedParas{§12–14}The Blue Kasiṇa]{The Blue Kasiṇa}

                \vismParagraph{V.12}{12}{}
                Next it is said [in the Commentaries]: “One who is learning the blue kasiṇa apprehends the sign in blue, whether in a flower or in a cloth or in a colour element.”\footnote{\vismAssertFootnoteCounter{3}\vismHypertarget{V.n3}{}\emph{Vaṇṇa-dhātu—}“colour element” should perhaps have been rendered simply by “paint.” The one Pali word “\emph{nīla}” has to serve for the English blue, green, and sometimes black.} Firstly, when someone has merit, having had previous practice, the sign arises in him when he sees a bush with blue flowers, or such flowers spread out on a place of offering, or any blue cloth or gem.

                \vismParagraph{V.13}{13}{}
                \textcolor{brown}{\textit{[173]}} But anyone else should take flowers such as blue lotuses, \emph{girikaṇṇikā }(morning glory) flowers, etc., and spread them out to fill a tray or a flat basket completely so that no stamen or stalk shows or with only their petals. Or he can fill it with blue cloth bunched up together; or he can fasten the cloth over the rim of the tray or basket like the covering of a drum. Or he can make a kasiṇa disk, either portable as described under the earth kasiṇa or on a wall, with one of the colour elements such as bronze-green, leaf-green, \emph{añjana}-ointment black, surrounding it with a different colour. After that, he should bring it to mind as “blue, blue” in the way already described under the earth kasiṇa.

                \vismParagraph{V.14}{14}{}
                And here too any fault in the kasiṇa is evident in the learning sign; the stamens and stalks and the gaps between the petals, etc., are apparent. The counterpart sign appears like a crystal fan in space, free from the kasiṇa disk. The rest should be understood as already described.
            \section[\vismAlignedParas{§15–16}The Yellow Kasiṇa]{The Yellow Kasiṇa}

                \vismParagraph{V.15}{15}{}
                Likewise with the yellow kasiṇa; for this is said: “One who is learning the yellow kasiṇa apprehends the sign in yellow, either in a flower or in a cloth or in a colour element.” Therefore here too, when someone has merit, having had previous practice, the sign arises in him when he sees a flowering bush or flowers spread out, or yellow cloth or colour element, as in the case of the Elder Cittagutta. That venerable one, it seems, saw an offering being made on the flower altar, with \emph{pattaṅga} flowers\footnote{\vismAssertFootnoteCounter{4}\vismHypertarget{V.n4}{}\emph{Pattaṅga}: not in PED. \emph{Āsana—}“altar”: not in this sense in PED.} at Cittalapabbata, and as soon as he saw it the sign arose in him the size of the flower altar.

                \vismParagraph{V.16}{16}{}
                \marginnote{\textcolor{teal}{\footnotesize\{223|165\}}}{}Anyone else should make a kasiṇa, in the way described for the blue kasiṇa, with \emph{kaṇikāra} flowers, etc., or with yellow cloth or with a colour element. He should bring it to mind as “yellow, yellow.” The rest is as before.
            \section[\vismAlignedParas{§17–18}The Red Kasiṇa]{The Red Kasiṇa}

                \vismParagraph{V.17}{17}{}
                Likewise with the red kasiṇa; for this is said: “One who is learning the red kasiṇa apprehends the sign in red, \textcolor{brown}{\textit{[174]}} either in a flower or in a cloth or in a colour element.” Therefore here too, when someone has merit, having had previous practice, the sign arises in him when he sees a \emph{bandhujīvaka} (hibiscus) bush, etc., in flower, or such flowers spread out, or a red cloth or gem or colour element.

                \vismParagraph{V.18}{18}{}
                But anyone else should make a kasiṇa, in the way already described for the blue kasiṇa, with \emph{jayasumana} flowers or \emph{bandhujīvaka} or red \emph{koraṇḍaka} flowers, etc., or with red cloth or with a colour element. He should bring it to mind as “red, red.” The rest is as before.
            \section[\vismAlignedParas{§19–20}The White Kasiṇa]{The White Kasiṇa}

                \vismParagraph{V.19}{19}{}
                Of the white kasiṇa it is said: “One who is learning the white kasiṇa apprehends the sign in white, either in a flower or in a cloth or in a colour element.” So firstly, when someone has merit, having had previous practice, the sign arises in him when he sees a flowering bush of such a kind or \emph{vassikasumana }(jasmine) flowers, etc., spread out, or a heap of white lotuses or lilies, white cloth or colour element; and it also arises in a tin disk, a silver disk, and the moon’s disk.

                \vismParagraph{V.20}{20}{}
                Anyone else should make a kasiṇa, in the way already described for the blue kasiṇa, with the white flowers already mentioned, or with cloth or colour element. He should bring it to mind as “white, white.” The rest is as before.
            \section[\vismAlignedParas{§21–23}The Light Kasiṇa]{The Light Kasiṇa}

                \vismParagraph{V.21}{21}{}
                Of the light kasiṇa it is said: “One who is learning the light kasiṇa apprehends the sign in light in a hole in a wall, or in a keyhole, or in a window opening.” So firstly, when someone has merit, having had previous practice, the sign arises in him when he sees the circle thrown on a wall or a floor by sunlight or moonlight entering through a hole in a wall, etc., or when he sees a circle thrown on the ground by sunlight or moonlight coming through a gap in the branches of a dense-leaved tree or through a gap in a hut made of closely packed branches.

                \vismParagraph{V.22}{22}{}
                Anyone else should use that same kind of circle of luminosity just described, developing it as “luminosity, luminosity” or “light, light.” If he cannot do so, he can light a lamp inside a pot, close the pot’s mouth, make a hole in it and place it with the hole facing a wall. The lamplight coming out of the hole throws a circle on the wall. He should develop that \textcolor{brown}{\textit{[175]}} as “light, light.” This lasts longer than the other kinds.

                \vismParagraph{V.23}{23}{}
                Here the learning sign is like the circle thrown on the wall or the ground. The counterpart sign is like a compact bright cluster of lights. The rest is as before.
            \section[\vismAlignedParas{§24–26}The Limited-Space Kasiṇa]{The Limited-Space Kasiṇa}

                \vismParagraph{V.24}{24}{}
                \marginnote{\textcolor{teal}{\footnotesize\{224|166\}}}{}Of the limited-space kasiṇa it is said: “One who is learning the space kasiṇa apprehends the sign in a hole in a wall, or in a keyhole, or in a window opening.” So firstly, when someone has merit, having had previous practice, the sign arises in him when he sees any [such gap as a] hole in a wall.

                \vismParagraph{V.25}{25}{}
                Anyone else should make a hole a span and four fingers broad in a well-thatched hut, or in a piece of leather, or in a rush mat, and so on. He should develop one of these, or a hole such as a hole in a wall, as “space, space.”

                \vismParagraph{V.26}{26}{}
                Here the learning sign resembles the hole together with the wall, etc., that surrounds it. Attempts to extend it fail. The counterpart sign appears only as a circle of space. Attempts to extend it succeed. The rest should be understood as described under the earth kasiṇa.\footnote{\vismAssertFootnoteCounter{5}\vismHypertarget{V.n5}{}In the Suttas the first eight kasiṇas are the same as those given here, and they are the only ones mentioned in the Dhammasaṅgaṇī (§160–203) and Paṭisambhidā (\textbf{\cite{Paṭis}I 6}). The Suttas give space and consciousness as ninth and tenth respectively (\textbf{\cite{M}II 14–15}; \textbf{\cite{D}III 268}; \textbf{\cite{Netti}89}, etc.). But these last two appear to coincide with the first two immaterial states, that is, boundless space and boundless consciousness. The light kasiṇa given here as ninth does not appear in the Suttas. It is perhaps a development from the “perception of light” (\emph{āloka-saññā}) (\textbf{\cite{A}II 45}). The limited-space kasiṇa given here as tenth has perhaps been made “limited’ in order to differentiate it from the first immaterial state. The commentary on the consciousness kasiṇa (\textbf{\cite{M-a}III 261}) says nothing on this aspect. As to space, \textbf{\cite{Vism-mhṭ}(p. 373)} says: “The attainment of the immaterial states is not produced by means of the space kasiṇa, and with the words ‘ending with the white kasiṇa’ (\hyperlink{XXI.2}{XXI.2}{}) the light kasiṇa is included in the white kasiṇa.” For description of space (\emph{ākāsa}) see \textbf{\cite{Dhs-a}325}, \textbf{\cite{Netti}29}. Also \textbf{\cite{Vism-mhṭ}(p. 393)} defines space thus: “Wherever there is no obstruction, that is called space.” Again the \emph{Majjhima Nikāya Ṭīkā} (commenting on MN 106) remarks: “[Sense desires] are not called empty (\emph{ritta}) in the sense that space, which is entirely devoid of individual essence, is called empty.”}
            \section[\vismAlignedParas{§27–42}General]{General}

                \vismParagraph{V.27}{27}{}
                
                \begin{verse}
                    He with Ten Powers, who all things did see,\\{}
                    Tells ten kasiṇas, each of which can be\\{}
                    The cause of fourfold and of fivefold jhāna,\\{}
                    The fine-material sphere’s own master key.\\{}
                    Now, knowing their descriptions and the way\\{}
                    To tackle each and how they are developed,\\{}
                    There are some further points that will repay\\{}
                    Study, each with its special part to play.
                \end{verse}


                \vismParagraph{V.28}{28}{}
                Of these, the earth kasiṇa is the basis for such powers as the state described as “Having been one, he becomes many” (\textbf{\cite{D}I 78}), etc., and stepping or standing or sitting on space or on water by creating earth, and the acquisition of the bases of mastery (\textbf{\cite{M}II 13}) by the limited and measureless method.

                \vismParagraph{V.29}{29}{}
                The water kasiṇa is the basis for such powers as diving in and out of the earth (\textbf{\cite{D}I 78}), causing rain, storms, creating rivers and seas, making the earth and rocks and palaces quake (\textbf{\cite{M}I 253}).

                \vismParagraph{V.30}{30}{}
                \marginnote{\textcolor{teal}{\footnotesize\{225|167\}}}{}The fire kasiṇa is the basis for such powers as smoking, flaming, causing showers of sparks, countering fire with fire, ability to burn only what one wants to burn (\textbf{\cite{S}IV 290}), \textcolor{brown}{\textit{[176]}} causing light for the purpose of seeing visible objects with the divine eye, burning up the body by means of the fire element at the time of attaining Nibbāna (\textbf{\cite{M-a}IV 196}).

                \vismParagraph{V.31}{31}{}
                The air kasiṇa is the basis for such powers as going with the speed of the wind, causing wind storms.

                \vismParagraph{V.32}{32}{}
                The blue kasiṇa is the basis for such powers as creating black forms, causing darkness, acquisition of the bases of mastery by the method of fairness and ugliness, and attainment of the liberation by the beautiful (see \textbf{\cite{M}II 12})

                \vismParagraph{V.33}{33}{}
                The yellow kasiṇa is the basis for such powers as creating yellow forms, resolving that something shall be gold (\textbf{\cite{S}I 116}), acquisition of the bases of mastery in the way stated, and attainment of the liberation by the beautiful.

                \vismParagraph{V.34}{34}{}
                The red kasiṇa is the basis for such powers as creating red forms, acquisition of the bases of mastery in the way stated, and attainment of the liberation by the beautiful.

                \vismParagraph{V.35}{35}{}
                The white kasiṇa is the basis for such powers as creating white forms, banishing stiffness and torpor, dispelling darkness, causing light for the purpose of seeing visible objects with the divine eye.

                \vismParagraph{V.36}{36}{}
                The light kasiṇa is the basis for such powers as creating luminous forms, banishing stiffness and torpor, dispelling darkness, causing light for the purpose of seeing visible objects with the divine eye.

                \vismParagraph{V.37}{37}{}
                The space kasiṇa is the basis for such powers as revealing the hidden, maintaining postures inside the earth and rocks by creating space inside them, travelling unobstructed through walls, and so on.

                \vismParagraph{V.38}{38}{}
                The classification “above, below, around, exclusive, measureless” applies to all kasiṇas; for this is said: “He perceives the earth kasiṇa above, below, around, exclusive, measureless” (\textbf{\cite{M}II 14}), and so on.

                \vismParagraph{V.39}{39}{}
                Herein, \emph{above} is upwards towards the sky’s level. \emph{Below} is downwards towards the earth’s level. \emph{Around} is marked off all around like the perimeter of a field. For one extends a kasiṇa upwards only, another downwards, another all round; or for some reason another projects it thus as one who wants to see visible objects with the divine eye projects light.\textbf{ }\textcolor{brown}{\textit{[177]}} Hence “above, below, around” is said. The word \emph{exclusive}, however, shows that anyone such state has nothing to do with any other. Just as there is water and nothing else in all directions for one who is actually in water, so too, the earth kasiṇa is the earth kasiṇa only; it has nothing in common with any other kasiṇa. Similarly in each instance. \emph{Measureless} means measureless intentness. He is intent upon the entirety with his mind, taking no measurements in this way: “This is its beginning, this is its middle.”

                \vismParagraph{V.40}{40}{}
                No kasiṇa can be developed by any living being described as follows: “Beings hindered by kamma, by defilement or by kamma-result, who lack faith, zeal and understanding, will be incapable of entering into the certainty of rightness in profitable states” (\textbf{\cite{Vibh}341}).

                \vismParagraph{V.41}{41}{}
                \marginnote{\textcolor{teal}{\footnotesize\{226|168\}}}{}Herein, the words \emph{hindered by kamma} refer to those who possess bad kamma entailing immediate effect [on rebirth].\footnote{\vismAssertFootnoteCounter{6}\vismHypertarget{V.n6}{}The five kinds of bad kamma with immediate effect on rebirth are, in that order of priority: matricide, parricide, arahanticide, intentional shedding of a Buddha’s blood, and causing a schism in the Community, all of which cause rebirth in hell and remaining there for the remainder of the aeon (\emph{kappa}), whatever other kinds of kamma may have been performed (\textbf{\cite{M-a}IV 109f.}).} \emph{By defilement}: who have fixed wrong view\footnote{\vismAssertFootnoteCounter{7}\vismHypertarget{V.n7}{}The no-cause view, moral-inefficacy-of-action view, the nihilistic view that there is no such thing as giving, and so on (see DN 2).} or are hermaphrodites or eunuchs. \emph{By kamma-result}: who have had a rebirth-linking with no [profitable] root-cause or with only two [profitable] root-causes. \emph{Lack faith}: are destitute of faith in the Buddha, Dhamma and Sangha. \emph{Zeal}: are destitute of zeal for the unopposed way. \emph{Understanding}: are destitute of mundane and supramundane right view. \emph{Will be incapable of entering into the certainty of rightness in profitable states} means that they are incapable of entering into the noble path called “certainty” and “rightness in profitable states.”

                \vismParagraph{V.42}{42}{}
                And this does not apply only to kasiṇas; for none of them will succeed in developing any meditation subject at all. So the task of devotion to a meditation subject must be undertaken by a clansman who has no hindrance by kamma-result, who shuns hindrance by kamma and by defilement, and who fosters faith, zeal and understanding by listening to the Dhamma, frequenting good men, and so on.

                The fifth chapter called “The Description of the Remaining Kasiṇas” in the Treatise on the Development of Concentration in the \emph{Path of Purification} composed for the purpose of gladdening good people.
        \chapter[Foulness as a Meditation Subject]{Foulness as a Meditation Subject\vismHypertarget{VI}\newline{\textnormal{\emph{Asubha-kammaṭṭhāna-niddesa}}}}
            \label{VI}

            \section[\vismAlignedParas{§1–11}General Definitions]{General Definitions}

                \vismParagraph{VI.1}{1}{}
                \marginnote{\textcolor{teal}{\footnotesize\{227|169\}}}{}\textcolor{brown}{\textit{[178]}} Now, ten kinds of foulness, [as corpses] without consciousness, were listed next after the kasiṇas thus: the bloated, the livid, the festering, the cut up, the gnawed, the scattered, the hacked and scattered, the bleeding, the worm infested, a skeleton (\hyperlink{III.105}{III.105}{}).

                \emph{The bloated}: it is bloated (\emph{uddhumāta}) because bloated by gradual dilation and swelling after (\emph{uddhaṃ}) the close of life, as a bellows is with wind. What is bloated (\emph{uddhumāta}) is the same as “the bloated” (\emph{uddhumātaka}). Or alternatively, what is bloated (\emph{uddhumāta}) is vile (\emph{kucchita}) because of repulsiveness, thus it is “the bloated” (\emph{uddhumātaka}). This is a term for a corpse in that particular state.

                \vismParagraph{VI.2}{2}{}
                \emph{The livid}: what has patchy discolouration is called livid (\emph{vinīla}). What is livid is the same as “the livid” (\emph{vinīlaka}). Or alternatively, what is livid (\emph{vinīla}) is vile (\emph{kucchita}) because of repulsiveness, thus it is “the livid” (\emph{vinīlaka}).\footnote{\vismAssertFootnoteCounter{1}\vismHypertarget{VI.n1}{}It is not possible to render such associative and alliterative derivations of meaning into English. They have nothing to do with the historical development of words, and their purpose is purely mnemonic.} This is a term for a corpse that is reddish-coloured in places where flesh is prominent, whitish-coloured in places where pus has collected, but mostly blue-black (\emph{nīla}), as if draped with blue-black cloth in the blue-black places.

                \vismParagraph{VI.3}{3}{}
                \emph{The festering}: what is trickling with pus in broken places is festering (\emph{vipubba}). What is festering is the same as “the festering” (\emph{vipubbaka}). Or alternatively, what is festering (\emph{vipubba}) is vile (\emph{kucchita}) because of repulsiveness, thus it is “the festering” (\emph{vipubbaka}). This is a term for a corpse in that particular state.

                \vismParagraph{VI.4}{4}{}
                \emph{The cut up}: what has been opened up\footnote{\vismAssertFootnoteCounter{2}\vismHypertarget{VI.n2}{}\emph{Apavārita—}“opened up”: not in PED.} by cutting it in two is called cut up (\emph{vicchidda}). What is cut up is the same as “the cut up” (\emph{vicchiddaka}). Or alternatively, what is cut up (\emph{vicchidda}) is vile (\emph{kucchita}) because of repulsiveness, thus it is “the cut up” (\emph{vicchiddaka}). This is a term for a corpse cut in the middle. \textcolor{brown}{\textit{[179]}}

                \vismParagraph{VI.5}{5}{}
                \emph{The gnawed}: what has been chewed here and there in various ways by dogs, jackals, etc., is what is gnawed (\emph{vikkhāyita}). What is gnawed is the same as “the gnawed” (\emph{vikkhāyitaka}). Or alternatively, what is gnawed (\emph{vikkhāyita}) is vile \marginnote{\textcolor{teal}{\footnotesize\{228|170\}}}{}(\emph{kucchita}) because of repulsiveness, thus it is “the gnawed” (\emph{vikkhāyitaka}). This is a term for a corpse in that particular state.

                \vismParagraph{VI.6}{6}{}
                \emph{The scattered}: what is strewed about (\emph{vividhaṃ khittaṃ}) is scattered (\emph{vikkhittaṃ}). What is scattered is the same as “the scattered” (\emph{vikkhittaka}). Or alternatively, what is scattered (\emph{vikkhitta}) is vile (\emph{kucchita}) because of repulsiveness, thus it is “the scattered” (\emph{vikkhittaka}). This is a term for a corpse that is strewed here and there in this way: “Here a hand, there a foot, there the head” (cf. \textbf{\cite{M}I 58}).

                \vismParagraph{VI.7}{7}{}
                \emph{The hacked and scattered}: it is hacked, and it is scattered in the way just described, thus it is “hacked and scattered” (\emph{hata-vikkhittaka}). This is a term for a corpse scattered in the way just described after it has been hacked with a knife in a crow’s-foot pattern on every limb.

                \vismParagraph{VI.8}{8}{}
                \emph{The bleeding}: it sprinkles (\emph{kirati}), scatters, blood (\emph{lohita}), and it trickles here and there, thus it is “the bleeding” (\emph{lohitaka}). This is a term for a corpse smeared with trickling blood.

                \vismParagraph{VI.9}{9}{}
                \emph{The worm-infested}: it is maggots that are called worms (\emph{puḷuva}); it sprinkles worms (\emph{puḷuve kirati}), thus it is worm-infested (\emph{puḷuvaka}). This is a term for a corpse full of maggots.

                \vismParagraph{VI.10}{10}{}
                \emph{A skeleton}: bone (\emph{aṭṭhi}) is the same as skeleton (\emph{aṭṭhika}). Or alternatively, bone (\emph{aṭṭhi}) is vile (\emph{kucchita}) because of repulsiveness, thus it is a skeleton (\emph{aṭṭhika}). This is a term both for a single bone and for a framework of bones.

                \vismParagraph{VI.11}{11}{}
                These names are also used both for the signs that arise with the bloated, etc., as their support, and for the jhānas obtained in the signs.
            \section[\vismAlignedParas{§12–69}The Bloated]{The Bloated}

                \vismParagraph{VI.12}{12}{}
                Herein, when a meditator wants to develop the jhāna called “of the bloated” by arousing the sign of the bloated on a bloated body, he should in the way already described approach a teacher of the kind mentioned under the earth kasiṇa and learn the meditation subject from him. In explaining the meditation subject to him, the teacher should explain it all, that is, the directions for going with the aim of acquiring the sign of foulness, the characterizing of the surrounding signs, the eleven ways of apprehending the sign, the reviewing of the path gone by and come by, concluding with the directions for absorption. And when the meditator has learnt it all well, he should go to an abode of the kind already described and live there while seeking the sign of the bloated.

                \vismParagraph{VI.13}{13}{}
                Meanwhile, when he hears people saying that at some village gate or on some road or at some forest’s edge or at the base of some rock or at the root of some tree \textcolor{brown}{\textit{[180]}} or on some charnel ground a bloated corpse is lying, he should not go there at once, like one who plunges into a river where there is no ford.

                \vismParagraph{VI.14}{14}{}
                Why not? Because this foulness is beset by wild beasts and non-human beings, and he might risk his life there. Or perhaps the way to it goes by a village gate or a bathing place or an irrigated field, and there a visible object of the opposite sex might come into focus. Or perhaps the body is of the opposite sex; for a female body is unsuitable for a man, and a male body for a woman. If only recently dead, it may even look beautiful; hence there might be danger to the life \marginnote{\textcolor{teal}{\footnotesize\{229|171\}}}{}of purity. But if he judges himself thus, “This is not difficult for one like me,” then he can go there.

                \vismParagraph{VI.15}{15}{}
                And when he goes, he should do so only after he has spoken to the senior elder of the Community or to some well-known bhikkhu.

                \vismParagraph{VI.16}{16}{}
                Why? Because if all his limbs are seized with shuddering at the charnel ground, or if his gorge rises when he is confronted with disagreeable objects such as the visible forms and sounds of non-human beings, lions, tigers, etc., or something else afflicts him, then he whom he told will have his bowl and robe well looked after in the monastery, or he will care for him by sending young bhikkhus or novices to him.

                \vismParagraph{VI.17}{17}{}
                Besides, robbers may meet there thinking a charnel ground a safe place for them whether or not they have done anything wrong. And when men chase them, they drop their goods near the bhikkhu and run away. Perhaps the men seize the bhikkhu, saying “We have found the thief with the goods,” and bully him. Then he whom he told will explain to the men “Do not bully him; he went to do this special work after telling me,” and he will rescue him. This is the advantage of going only after informing someone.

                \vismParagraph{VI.18}{18}{}
                Therefore he should inform a bhikkhu of the kind described and then set out eager to see the sign, and as happy and joyful as a warrior-noble (\emph{khattiya}) on his way to the scene of anointing, as one going to offer libations at the hall of sacrifice, or as a pauper on his way to unearth a hidden treasure. And he should go there in the way advised by the Commentaries.

                \vismParagraph{VI.19}{19}{}
                For this is said: “One who is learning the bloated sign of foulness goes alone with no companion, with unremitting mindfulness established, with his sense faculties turned inwards, with his mind not turned outwards, reviewing the path gone by and come by. In the place where the bloated sign of foulness \textcolor{brown}{\textit{[181]}} has been left he notes any stone or termite-mound or tree or bush or creeper there each with its particular sign and in relation to the object. When he has done this, he characterizes the bloated sign of foulness by the fact of its having attained that particular individual essence. (see \hyperlink{VI.84}{§84}{}) Then he sees that the sign is properly apprehended, that it is properly remembered, that it is properly defined, by its colour, by its mark, by its shape, by its direction, by its location, by its delimitation, by its joints, by its openings, by its concavities, by its convexities, and all round.

                \vismParagraph{VI.20}{20}{}
                “When he has properly apprehended the sign, properly remembered it, properly defined it, he goes alone with no companion, with unremitting mindfulness established, with his sense faculties turned inwards, with his mind not turned outwards, reviewing the path gone by and come by. When he walks, he resolves that his walk is oriented towards it; when he sits, he prepares a seat that is oriented towards it.

                \vismParagraph{VI.21}{21}{}
                “What is the purpose, what is the advantage of characterizing the surrounding signs? Characterizing the surrounding signs has non-delusion for its purpose, it has non-delusion for its advantage. What is the purpose, what is the advantage of apprehending the sign in the [other] eleven ways? \marginnote{\textcolor{teal}{\footnotesize\{230|172\}}}{}Apprehending the sign in the [other] eleven ways has anchoring [the mind] for its purpose, it has anchoring [the mind] for its advantage. What is the purpose, what is the advantage of reviewing the path gone by and come by? Reviewing the path gone by and come by has keeping [the mind] on the track for its purpose, it has keeping [the mind] on the track for its advantage.

                \vismParagraph{VI.22}{22}{}
                “When he has established reverence for it by seeing its advantages and by perceiving it as a treasure and so come to love it, he anchors his mind upon that object: ‘Surely in this way I shall be liberated from ageing and death.’ Quite secluded from sense desires, secluded from unprofitable things he enters upon and dwells in the first jhāna … [seclusion]. He has arrived at the first jhāna of the fine-material sphere. His is a heavenly abiding and an instance of the meritorious action consisting in [meditative] development.” (\emph{Source untraced.})

                \vismParagraph{VI.23}{23}{}
                So if he goes to the charnel ground to test his control of mind, let him do so after striking the gong or summoning a chapter. If he goes there mainly for [developing that] meditation subject, let him go alone with no companion, without renouncing his basic meditation subject and keeping it always in mind, taking a walking stick or a staff to keep off attacks by dogs, etc., \textcolor{brown}{\textit{[182]}} ensuring unremitting mindfulness by establishing it well, with his mind not turned outwards because he has ensured that his faculties, of which his mind is the sixth, are turned inwards.

                \vismParagraph{VI.24}{24}{}
                As he goes out of the monastery he should note the gate: “I have gone out in such a direction by such a gate.” After that he should define the path by which he goes: “This path goes in an easterly direction … westerly … northerly … southerly direction” or “It goes in an intermediate direction”; and “In this place it goes to the left, in this place to the right”; and “In this place there is a stone, in this a termite-mound, in this a tree, in this a bush, in this a creeper.” He should go to the place where the sign is, defining in this way the path by which he goes.

                \vismParagraph{VI.25}{25}{}
                And he should not approach it upwind; for if he did so and the smell of corpses assailed his nose, his brain\footnote{\vismAssertFootnoteCounter{3}\vismHypertarget{VI.n3}{}This does not imply what we, now, might suppose. See the description of “brain” in \hyperlink{VIII.126}{VIII.126}{} and especially \hyperlink{VIII.136}{VIII.136}{}. What is meant is perhaps that he might get a cold or catarrh.} might get upset, or he might throw up his food, or he might repent his coming, thinking “What a place of corpses I have come to!” So instead of approaching it upwind, he should go downwind. If he cannot go by a downwind path—if there is a mountain or a ravine or a rock or a fence or a patch of thorns or water or a bog in the way—then he should go stopping his nose with the corner of his robe. These are the duties in going.

                \vismParagraph{VI.26}{26}{}
                When he has gone there in this way, he should not at once look at the sign of foulness; he should make sure of the direction. For perhaps if he stands in a certain direction, the object does not appear clearly to him and his mind is not wieldy. So rather than there he should stand where the object appears clearly and his mind is wieldy. And he should avoid standing to leeward or to windward of it. For if he stands to leeward he is bothered by the corpse smell and his mind strays; and if he stands to windward and non-human beings are dwelling there, \marginnote{\textcolor{teal}{\footnotesize\{231|173\}}}{}they may get annoyed and do him a mischief. So he should move round a little and not stand too much to windward. \textcolor{brown}{\textit{[183]}}

                \vismParagraph{VI.27}{27}{}
                Then he should stand not too far off or too near, or too much towards the feet or the head. For if he stands too far off, the object is not clear to him, and if he stands too near, he may get frightened. If he stands too much towards the feet or the head, not all the foulness becomes manifest to him equally. So he should stand not too far off or too near, opposite the middle of the body, in a place convenient for him to look at it.

                \vismParagraph{VI.28}{28}{}
                Then he should characterize the surrounding signs in the way stated thus: “In the place where the bloated sign of foulness has been left he notes any stone … or creeper there with its sign” (\hyperlink{VI.19}{§19}{}).

                \vismParagraph{VI.29}{29}{}
                These are the directions for characterizing them. If there is a rock in the eye’s focus near the sign, he should define it in this way: “This rock is high or low, small or large, brown or black or white, long or round,” after which he should observe [the relative positions] thus: “In this place, this is a rock, this is the sign of foulness; this is the sign of foulness, this is a rock.”

                \vismParagraph{VI.30}{30}{}
                If there is a termite-mound, he should define it in this way: “This is high or low, small or large, brown or black or white, long or round,” after which he should observe [the relative positions] thus: “In this place, this is a termite-mound, this is the sign of foulness.”

                \vismParagraph{VI.31}{31}{}
                If there is a tree, he should define it in this way: “This is a pipal fig tree or a banyan fig tree or a \emph{kacchaka }fig tree or a \emph{kapittha }fig tree; it is tall or short, small or large, black or white,” after which he should observe [the relative positions] thus: “In this place, this is a tree, this is the sign of foulness.”

                \vismParagraph{VI.32}{32}{}
                If there is a bush, he should define it in this way: “This is a \emph{sindi }bush or a \emph{karamanda }bush or a \emph{kaṇavīra }bush or a \emph{koraṇḍaka }bush; it is tall or short, small or large,” after which he should observe [the relative positions] thus: “In this place, this is a bush, this is the sign of foulness.”

                \vismParagraph{VI.33}{33}{}
                If there is a creeper, he should define it in this way: “This is a pumpkin creeper or a gourd creeper or a brown creeper or a black creeper or a stinking creeper,” after which he should observe [the relative positions] thus: “In this place, this is a creeper, this is the sign of foulness; this is the sign of foulness, this is a creeper.”

                \vismParagraph{VI.34}{34}{}
                Also \emph{with its particular sign and in relation to the object }was said (\hyperlink{VI.19}{§19}{}); but that is included by what has just been said; for he “characterizes it with its particular sign” when he defines it again and again, and he “characterizes it in relation to the object” when he defines it by combining it each time in pairs thus: “This is a rock, this is the sign of foulness; this is the sign of foulness, this is a rock.”

                \vismParagraph{VI.35}{35}{}
                Having done this, again he should bring to mind the fact that it has an individual essence, its own state of being bloated, which is not common to anything else, since it was said that he defines\footnote{\vismAssertFootnoteCounter{4}\vismHypertarget{VI.n4}{}Reference back to §19 requires \emph{sabhāvato upalakkhati }rather than\emph{ sabhāvato vavaṭṭhāpeti}, but so the readings have it.} it \emph{by the fact of its having attained }\marginnote{\textcolor{teal}{\footnotesize\{232|174\}}}{}\emph{that particular individual essence}. The meaning is that it should be defined according to individual essence, according to its own nature, as “the inflated,\footnote{\vismAssertFootnoteCounter{5}\vismHypertarget{VI.n5}{}\emph{Vaṇita—}“inflated”: glossed by \textbf{\cite{Vism-mhṭ}} with \emph{sūna} (swollen). Not in PED in this sense.} the bloated.”

                Having defined it in this way, he should apprehend the sign in the following six ways, that is to say, (1) by its colour, (2) by its mark, (3) by its shape, \textcolor{brown}{\textit{[184]}} (4) by its direction, (5) by its location, (6) by its delimitation. How?

                \vismParagraph{VI.36}{36}{}
                (1) The meditator should define it \emph{by its colour} thus: “This is the body of one who is black or white or yellow-skinned.”

                \vismParagraph{VI.37}{37}{}
                (2) Instead of defining it by the female mark or the male mark, he should define it \emph{by its mark }thus: “This is the body of one who was in the first phase of life, in the middle phase, in the last phase.”

                \vismParagraph{VI.38}{38}{}
                (3) \emph{By its shape}: he should define it only by the shape of the bloated thus: “This is the shape of its head, this is the shape of its neck, this is the shape of its hand, this is the shape of its chest, this is the shape of its belly, this is the shape of its navel, this is the shape of its hips, this is the shape of its thigh, this is the shape of its calf, this is the shape of its foot.”

                \vismParagraph{VI.39}{39}{}
                (4) He should define it \emph{by its direction }thus: “There are two directions in this body, that is, down from the navel as the lower direction, and up from it as the upper direction.” Or alternatively, he can define it thus: “I am standing in this direction; the sign of foulness is in that direction.”

                \vismParagraph{VI.40}{40}{}
                (5) He should define it \emph{by its location }thus: “The hand is in this location, the foot in this, the head in this, the middle of the body in this.” Or alternatively, he can define it thus: “I am in this location; the sign of foulness is in that.”

                \vismParagraph{VI.41}{41}{}
                (6) He should define it \emph{by its delimitation }thus: “This body is delimited below by the soles of the feet, above by the tips of the hair, all round by the skin; the space so delimited is filled up with thirty-two pieces of corpse.” Or alternatively, he can define it thus: “This is the delimitation of its hand, this is the delimitation of its foot, this is the delimitation of its head, this is the delimitation of the middle part of its body.” Or alternatively, he can delimit as much of it as he has apprehended thus: “Just this much of the bloated is like this.”

                \vismParagraph{VI.42}{42}{}
                However, a female body is not appropriate for a man or a male one for a woman; for the object, [namely, the repulsive aspect], does not make its appearance in a body of the opposite sex, which merely becomes a condition for the wrong kind of excitement.\footnote{\vismAssertFootnoteCounter{6}\vismHypertarget{VI.n6}{}\emph{Vipphandana—}“wrong kind of excitement”: \textbf{\cite{Vism-mhṭ}} says here “\emph{Kilesa-paripphandanass’ eva nimittaṃ hotī ti attho} (the meaning is, it becomes the sign for interference by (activity of) defilement” (\textbf{\cite{Vism-mhṭ}170}). \emph{Phandati} and \emph{vipphandati} are both given only such meanings as “to throb, stir, twitch” and \emph{paripphandati} is not in PED. For the sense of wrong (\emph{vi}-) excitement (\emph{phandana}) cf. \hyperlink{IV.89}{IV.89}{} and \hyperlink{XIV.132}{XIV.132}{} and note. There seems to be an association of meaning between \emph{vipphāra, vyāpāra, vipphandana, īhaka}, and \emph{paripphandana} (perhaps also \emph{ābhoga}) in the general senses of interestedness, activity, concern, interference, intervention, etc.} To quote the Majjhima Commentary: “Even \marginnote{\textcolor{teal}{\footnotesize\{233|175\}}}{}when decaying,\footnote{\vismAssertFootnoteCounter{7}\vismHypertarget{VI.n7}{}The Harvard text has \emph{ugghāṭita}, but \textbf{\cite{Vism-mhṭ}(p. 170)} reads “\emph{ugghāṇitā }(not in PED)\emph{ pī-tī uddhumātakabhāvappattā pi sabbaso kuthita-sarīrā-pī-ti attho}.”} a woman invades a man’s mind and stays there.” That is why the sign should be apprehended in the six ways only in a body of the same sex.

                \vismParagraph{VI.43}{43}{}
                But when a clansman has cultivated the meditation subject under former Enlightened Ones, kept the ascetic practices, threshed out the great primary elements, discerned formations, defined mentality-materiality, eliminated the perception of a being, done the ascetic’s \textcolor{brown}{\textit{[185]}} duties, lived the moral life, and developed the development, when he contains the seed [of turning away from formations], and has mature knowledge and little defilement, then the counterpart sign appears to him in the place while he keeps looking. If it does not appear in that way, then it appears to him as he is apprehending the sign in the six ways.

                \vismParagraph{VI.44}{44}{}
                But if it does not appear to him even then, he should apprehend the sign again in five more ways: (7) by its joints, (8) by its openings, (9) by its concavities, (10) by its convexities, and (11) all round.

                \vismParagraph{VI.45}{45}{}
                Herein, (7) \emph{by its joints} is [properly] by its hundred and eighty joints. But how can he define the hundred and eighty joints in the bloated? Consequently he can define it by its fourteen major joints thus: Three joints in the right arm, three in the left arm, three in the right leg, three in the left leg, one neck joint, one waist joint.

                \vismParagraph{VI.46}{46}{}
                (8) \emph{By its openings}: an “opening” is the hollow between the arm [and the side], the hollow between the legs, the hollow of the stomach, the hollow of the ear. He should define it by its openings in this way. Or alternatively, the opened or closed state of the eyes and the opened or closed state of the mouth can be defined.

                \vismParagraph{VI.47}{47}{}
                (9) \emph{By its concavities}: he should define any concave place on the body such as the eye sockets or the inside of the mouth or the base of the neck. Or he can define it thus: “I am standing in a concave place, the body is in a convex place.”

                \vismParagraph{VI.48}{48}{}
                (10) \emph{By its convexities}: he should define any raised place on the body such as the knee or the chest or the forehead. Or he can define it thus: “I am standing in a convex place, the body is in a concave place.”

                \vismParagraph{VI.49}{49}{}
                (11) \emph{All round}: the whole body should be defined all round. After working over the whole body with knowledge, he should establish his mind thus, “The bloated, the bloated,” upon any part that appears clearly to him. If it has not appeared even yet, and if there is special intensity of the bloatedness in the belly,\footnote{\vismAssertFootnoteCounter{8}\vismHypertarget{VI.n8}{}“\emph{Udara-pariyosānaṃ uparisarīram}” (\textbf{\cite{Vism-mhṭ}172}). \emph{Pariyosāna} here means “intensity” though normally it means “end”; but see PED \emph{pariyosita}.} he should establish his mind thus, “The bloated, the bloated,” on that.

                \vismParagraph{VI.50}{50}{}
                Now, as to the words, \emph{he sees that the sign is properly apprehended}, etc., the explanation is this. The meditator should apprehend the sign thoroughly in that body in the way of apprehending the sign already described. He should \marginnote{\textcolor{teal}{\footnotesize\{234|176\}}}{}advert to it with well-established mindfulness. He should see that it is properly remembered, properly defined, by doing that again and again. Standing in a place not too far from and not too near to the body, he should open his eyes, look and apprehend the sign. \textcolor{brown}{\textit{[186]}} He should open his eyes and look a hundred times, a thousand times, [thinking], “Repulsiveness of the bloated, repulsiveness of the bloated,” and he should close his eyes and advert to it.

                \vismParagraph{VI.51}{51}{}
                As he does so again and again, the learning sign becomes properly apprehended by him. When is it properly apprehended? When it comes into focus alike whether he opens his eyes and looks or closes his eyes and adverts, then it is called properly apprehended.

                \vismParagraph{VI.52}{52}{}
                When he has thus properly apprehended the sign, properly remembered it, and properly defined it, then if he is unable to conclude his development on the spot, he can go to his own lodging, alone, in the same way as described of his coming, with no companion, keeping that same meditation subject in mind, with mindfulness well established, and with his mind not turned outwards owing to his faculties being turned inwards.

                \vismParagraph{VI.53}{53}{}
                As he leaves the charnel ground he should define the path he comes back by thus: “The path by which I have left goes in an easterly direction, westerly … northerly … southerly direction,” or “It goes in an intermediate direction”; or “In this place it goes to the left, in this place to the right”; and “In this place there is a stone, in this a termite-mound, in this a tree, in this a bush, in this a creeper.”

                \vismParagraph{VI.54}{54}{}
                When he has defined the path he has come back by and when, once back, he is walking up and down, he should see that his walk is oriented towards it too; the meaning is that he should walk up and down on a piece of ground that faces in the direction of the sign of foulness. And when he sits, he should prepare a seat oriented towards it too.

                \vismParagraph{VI.55}{55}{}
                But if there is a bog or a ravine or a tree or a fence or a swamp in that direction, if he cannot walk up and down on a piece of ground facing in that direction, if he cannot prepare his seat thus because there is no room for it, then he can both walk up and down and sit in a place where there is room, even though it does not face that way; but he should turn his mind in that direction.

                \vismParagraph{VI.56}{56}{}
                Now, as to the questions beginning with \emph{what is the purpose … characterizing the surrounding signs}? The intention of the answer that begins with the words, \emph{has non-delusion for its purpose}, is this: If someone goes at the wrong time to the place where the sign of the bloated is, and opens his eyes for the purpose of apprehending the sign by characterizing the surrounding signs, then as soon as he looks the dead body appears \textcolor{brown}{\textit{[187]}} as if it were standing up and threatening\footnote{\vismAssertFootnoteCounter{9}\vismHypertarget{VI.n9}{}There is no sense of \emph{ajjhottharati} given in PED that fits here. Cf. \hyperlink{I.56}{I.56}{}.} and pursuing him, and when he sees the hideous and fearful object, his mind reels, he is like one demented, gripped by panic, fear and terror, and his hair stands on end. For among the thirty-eight meditation subjects expounded in the texts no object is so frightening as this one. There are some who lose jhāna in this meditation subject. Why? Because it is so frightening.

                \vismParagraph{VI.57}{57}{}
                \marginnote{\textcolor{teal}{\footnotesize\{235|177\}}}{}So the meditator must stand firm. Establishing his mindfulness well, he should remove his fears in this way: “No dead body gets up and pursues one. If that stone or that creeper close to it were to come, the body might come too; but since that stone or that creeper does not come, the body will not come either. Its appearance to you in this way is born: of your perception, created by your perception. Today your meditation subject has appeared to you. Do not be afraid, bhikkhu.” He should laugh it off and direct his mind to the sign. In that way he will arrive at distinction. The words “Characterizing the surrounding signs has non-delusion for its purpose” are said on this account.

                \vismParagraph{VI.58}{58}{}
                To succeed in apprehending the sign in the eleven ways is to anchor the meditation subject. For the opening of his eyes and looking conditions the arising of the learning sign; and as he exercises his mind on that the counterpart sign arises; and as he exercises his mind on that he reaches absorption. When he is sure of absorption, he works up insight and realizes Arahantship. Hence it was said: \emph{apprehending the sign in the [other] eleven ways has anchoring [the mind] for its purpose}.

                \vismParagraph{VI.59}{59}{}
                \emph{The reviewing of the path gone by and come by has keeping [the mind] on the track for its purpose}: the meaning is that the reviewing of the path gone by and of the path come back by mentioned is for the purpose of keeping properly to the track of the meditation subject.

                \vismParagraph{VI.60}{60}{}
                For if this bhikkhu is going along with his meditation subject and people on the way ask him about the day, “What is today, venerable sir?” or they ask him some question [about Dhamma], or they welcome him, he ought not to go on in silence, thinking “I have a meditation subject.” The day must be told, the question must be answered, even by saying “I do not know” if he does not know, a legitimate welcome must be responded to. \textcolor{brown}{\textit{[188]}} As he does so, the newly acquired sign vanishes. But even if it does vanish, he should still tell the day when asked; if he does not know the answer to the question, he should still say “I do not know,” and if he does know it, he should explain it surely;\footnote{\vismAssertFootnoteCounter{10}\vismHypertarget{VI.n10}{}Reading \emph{ekaṃsena} (surely) with Harvard text rather than \emph{ekadesena} (partly).} and he must respond to a welcome. Also reception of visitors must be attended to on seeing a visiting bhikkhu, and all the remaining duties in the Khandhakas must be carried out too, that is, the duties of the shrine terrace, the duties of the Bodhi-tree terrace, the duties of the Uposatha house, the duties of the refectory and the bath house, and those to the teacher, the preceptor, visitors, departing bhikkhus, and the rest.

                \vismParagraph{VI.61}{61}{}
                And the newly acquired sign vanishes while he is carrying out these too. When he wants to go again, thinking “I shall go and take up the sign,” he finds he cannot go to the charnel ground because it has been invaded by non-human beings or by wild beasts, or the sign has disappeared. For a bloated corpse only lasts one or two days and then turns into a livid corpse. Of all the meditation subjects there is none so hard to come by as this.

                \vismParagraph{VI.62}{62}{}
                So when the sign has vanished in this way, the bhikkhu should sit down in his night quarters or in his day quarters and first of all review the path gone by and come by up to the place where he is actually sitting cross-legged, doing it in \marginnote{\textcolor{teal}{\footnotesize\{236|178\}}}{}this way: “I went out of the monastery by this gate, I took a path leading in such and such a direction, I turned left at such and such a place, I turned right at such and such a place, in one part of it there was a stone, in another a termite-mound or a tree or a bush or a creeper; having gone by that path, I saw the foulness in such and such a place, I stood there facing in such and such a direction and observed such and such surrounding signs, I apprehended the sign of foulness in this way; I left the charnel ground in such and such a direction, I came back by such and such a path doing this and this, and I am now sitting here.”

                \vismParagraph{VI.63}{63}{}
                As he reviews it in this way, the sign becomes evident and appears as if placed in front of him; the meditation subject rides in its track as it did before. Hence it was said:\emph{ the reviewing of the path gone by and come by has keeping [the mind] on the track for its purpose}.

                \vismParagraph{VI.64}{64}{}
                Now, as to the words, when he has established reverence for it by seeing its advantages and by perceiving it as a treasure and so come to love it, he anchors the mind on that object: here, having gained jhāna by exercising his mind on the repulsiveness in the bloated, he should increase insight with the jhāna as its proximate cause, and then he should see the advantages in this way: \textcolor{brown}{\textit{[189]}} “Surely in this way I shall be liberated from ageing and death.”

                \vismParagraph{VI.65}{65}{}
                Just as a pauper who acquired a treasure of gems would guard and love it with great affection, feeling reverence for it as one who appreciates the value of it, “I have got what is hard indeed to get!” so too [this bhikkhu] should guard the sign, loving it and feeling reverence for it as one who appreciates the value of it, “I have got this meditation subject, which is indeed as hard to get as a very valuable treasure is for a pauper to get. For one whose meditation subject is the four elements discerns the four primary elements in himself, one whose meditation subject is breathing discerns the wind in his own nostrils, and one whose meditation subject is a kasiṇa makes a kasiṇa and develops it at his ease, so these other meditation subjects are easily got. But this one lasts only one, or two days, after which it turns into a livid corpse. There is none harder to get than this one.” In his night quarters and in his day quarters he should keep his mind anchored there thus, “Repulsiveness of the bloated, repulsiveness of the bloated.” And he should advert to the sign, bring it to mind and strike at it with thought and applied thought over and over again.

                \vismParagraph{VI.66}{66}{}
                As he does so, the counterpart sign arises. Here is the difference between the two signs. The learning sign appears as a hideous, dreadful and frightening sight; but the counterpart sign appears like a man with big limbs lying down after eating his fill.

                \vismParagraph{VI.67}{67}{}
                Simultaneously with his acquiring the counterpart sign, his lust is abandoned by suppression owing to his giving no attention externally to sense desires [as object]. And owing to his abandoning of approval, ill will is abandoned too, as pus is with the abandoning of blood. Likewise stiffness and torpor are abandoned through exertion of energy, agitation and worry are abandoned through devotion to peaceful things that cause no remorse; and uncertainty about the Master who teaches the way, about the way, and about the fruit of the way, is abandoned through the actual experience of the distinction attained. So \marginnote{\textcolor{teal}{\footnotesize\{237|179\}}}{}the five hindrances are abandoned. And there are present applied thought with the characteristic of directing the mind on to that same sign, and sustained thought accomplishing the function of pressing on the sign, and happiness due to the acquisition of distinction, and tranquillity due to the production of tranquillity in one whose mind is happy, and bliss with that tranquillity as its sign, \textcolor{brown}{\textit{[190]}} and unification that has bliss as its sign due to the production of concentration in one whose mind is blissful. So the jhāna factors become manifest.

                \vismParagraph{VI.68}{68}{}
                Thus access, which is the obverse of the first jhāna, is produced in him too at that same moment. All after that up to absorption in the first jhāna and mastery in it should be understood as described under the earth kasiṇa.

                \vismParagraph{VI.69}{69}{}
                As regards the livid and the rest: the characterizing already described, starting with the going in the way beginning “One who is learning the bloated sign of foulness goes alone with no companion, with unremitting mindfulness established” (\hyperlink{VI.19}{§19}{}), should all be understood with its exposition and intention, substituting for the word “bloated” the appropriate word in each case thus: “One who is learning the livid sign of foulness …”, “One who is learning the festering sign of foulness …” But the differences are as follows.
            \section[\vismAlignedParas{§70}The Livid]{The Livid}

                \vismParagraph{VI.70}{70}{}
                \emph{The livid }should be brought to mind as “Repulsiveness of the livid, repulsiveness of the livid.” Here the learning sign appears blotchy-coloured; but the counterpart sign’s appearance has the colour which is most prevalent.
            \section[\vismAlignedParas{§71}The Festering]{The Festering}

                \vismParagraph{VI.71}{71}{}
                \emph{The festering }should be brought to mind as “Repulsiveness of the festering, repulsiveness of the festering.” Here the learning sign appears as though trickling; but the counterpart sign appears motionless and quiet.
            \section[\vismAlignedParas{§72}The Cut Up]{The Cut Up}

                \vismParagraph{VI.72}{72}{}
                \emph{The cut up }is found on a battlefield or in a robbers’ forest or on a charnel ground where kings have robbers cut up or in the jungle in a place where men are torn up by lions and tigers. So, if when he goes there, it comes into focus at one adverting although lying in different places, that is good. If not, then he should not touch it with his own hand; for by doing so he would become familiar with it.\footnote{\vismAssertFootnoteCounter{11}\vismHypertarget{VI.n11}{}“He would come to handle it without disgust as a corpse-burner would” (\textbf{\cite{Vism-mhṭ}176}).} He should get a monastery attendant or one studying to become an ascetic or someone else to put it together in one place. If he cannot find anyone to do it, he should put it together with a walking stick or a staff in such a way that there is only a finger’s breadth separating [the parts]. Having put it together thus, he should bring it to mind as “Repulsiveness of the cut up, repulsiveness of the cut up.” Herein, the learning sign appears as though cut in the middle; but the counterpart sign appears whole. \textcolor{brown}{\textit{[191]}}
            \section[\vismAlignedParas{§73}The Gnawed]{The Gnawed}

                \vismParagraph{VI.73}{73}{}
                \marginnote{\textcolor{teal}{\footnotesize\{238|180\}}}{}\emph{The gnawed }should be brought to mind as “Repulsiveness of the gnawed, repulsiveness of the gnawed.” Here the learning sign appears as though gnawed here and there; but the counterpart sign appears whole.
            \section[\vismAlignedParas{§74}The Scattered]{The Scattered}

                \vismParagraph{VI.74}{74}{}
                After getting \emph{the scattered }put together or putting it together in the way described under the cut up so that there is only a finger’s breadth, separating [the pieces], it should be brought to mind as “Repulsiveness of the scattered, repulsiveness of the scattered.” Here the learning sign appears with the gaps evident; but the counterpart sign appears whole.
            \section[\vismAlignedParas{§75}The Hacked and Scattered]{The Hacked and Scattered}

                \vismParagraph{VI.75}{75}{}
                \emph{The hacked and scattered }is found in the same places as those described under the cut up. Therefore, after going there and getting it put together or putting it together in the way described under the cut up so that there is only a finger’s breadth separating [the pieces], it should be brought to mind as “Repulsiveness of the hacked and scattered, repulsiveness of the hacked and scattered.” Here, when the learning sign becomes evident, it does so with the fissures of the wounds; but the counterpart sign appears whole.
            \section[\vismAlignedParas{§76}The Bleeding]{The Bleeding}

                \vismParagraph{VI.76}{76}{}
                \emph{The bleeding }is found at the time when [blood] is trickling from the openings of wounds received on battlefields, etc., or from the openings of burst boils and abscesses when the hands and feet have been cut off. So on seeing that, it should be brought to mind as “Repulsiveness of the bleeding, repulsiveness of the bleeding.” Here the learning sign appears to have the aspect of moving like a red banner struck by wind; but the counterpart sign appears quiet.
            \section[\vismAlignedParas{§77}The Worm-Infested]{The Worm-Infested}

                \vismParagraph{VI.77}{77}{}
                There is a \emph{worm-infested }corpse when at the end of two or three days a mass of maggots oozes out from the corpse’s nine orifices, and the mass lies there like a heap of paddy or boiled rice as big as the body, whether the body is that of a dog, a jackal, a human being,\footnote{\vismAssertFootnoteCounter{12}\vismHypertarget{VI.n12}{}Reading \emph{manussa} with Sinhalese ed.} an ox, a buffalo, an elephant, a horse, a python, or what you will. It can be brought to mind with respect to anyone of these as “Repulsiveness of the worm-infested, repulsiveness of the worm-infested.” For the sign arose for the Elder Cūḷa-Piṇḍapātika-Tissa in the corpse of an elephant’s carcass in the Kāḷadīghavāpi reservoir. Here the learning sign appears as though moving; but the counterpart sign appears quiet, like a ball of boiled rice.
            \section[\vismAlignedParas{§78–82}A Skeleton]{A Skeleton}

                \vismParagraph{VI.78}{78}{}
                \emph{A skeleton }is described in various aspects in the way beginning “As though he were looking at a corpse thrown onto a charnel ground, a skeleton with flesh \marginnote{\textcolor{teal}{\footnotesize\{239|181\}}}{}and blood, held together by sinews” (\textbf{\cite{D}II 296}). \textcolor{brown}{\textit{[192]}} So he should go in the way already described to where it has been put, and noticing any stones, etc., with their surrounding signs and in relation, to the object, he should characterize it \emph{by the fact of its having attained that particular individual essence} thus, “This is a skeleton,” and he should apprehend the sign in the eleven ways by colour and the rest. But if he looks at it, [apprehending it only] by its colour as white, it does not appear to him [with its individual essence as repulsive], but only as a variant of the white kasiṇa. Consequently he should only look at it as ‘a skeleton’ in the repulsive aspect.

                \vismParagraph{VI.79}{79}{}
                “Mark” is a term for the hand, etc., here, so he should define it \emph{by its mark }according to hand, foot, head, chest, arm, waist, thigh, and shin. He should define it \emph{by its shape}, however, according as it is long, short, square, round, small or large. \emph{By its direction and by its location} are as already described (\hyperlink{VI.39}{§39}{}–\hyperlink{VI.40}{40}{}). Having defined it \emph{by its delimitation} according to the periphery of each bone, he should reach absorption by apprehending whichever appears most evident to him. But it can also be defined \emph{by its concavities} and \emph{by its convexities} according to the concave and convex places in each bone. And it can also be defined by position thus: “I am standing in a concave place, the skeleton is in a convex place; or I am standing in a convex place, the skeleton is in a concave place.” It should be defined \emph{by its joints} according as any two bones are joined together. It should be defined \emph{by its openings} according to the gaps separating the bones. It should be defined \emph{all round }by directing knowledge to it comprehensively thus: “In this place there is this skeleton.” If the sign does not arise even in this way, then the mind should be established on the frontal bone. And in this case, just as in the case of those that precede it beginning with the worm-infested, the apprehending of the sign should be observed in this elevenfold manner as appropriate.

                \vismParagraph{VI.80}{80}{}
                This meditation subject is successful with a whole skeleton frame and even with a single bone as well. So having learnt the sign in anyone of these in the eleven ways, he should bring it to mind as “Repulsiveness of a skeleton, repulsiveness of a skeleton.” Here the learning sign and the counterpart sign are alike, so it is said. That is correct for a single bone. But when the learning sign becomes manifest in a skeleton frame, what is correct [to say] is that there are gaps in the learning sign while the counterpart sign appears whole. \textcolor{brown}{\textit{[193]}} And the learning sign even in a single bone should be dreadful and terrifying but the counterpart sign produces happiness and joy because it brings access.

                \vismParagraph{VI.81}{81}{}
                What is said in the Commentaries in this context allows that deduction. For there, after saying this, “There is no counterpart sign in the four divine abidings and in the ten kinds of foulness; for in the case of the divine abidings the sign is the breaking down of boundaries itself, and in the case of the ten kinds of foulness the sign comes into being as soon as the repulsiveness is seen, without any thinking about it,” it is again said, immediately next: “Here the sign is twofold: the learning sign and the counterpart sign. The learning sign appears hideous, dreadful and terrifying,” and so on. So what we said was well considered. And it is only this that is correct here. Besides, the appearance of a \marginnote{\textcolor{teal}{\footnotesize\{240|182\}}}{}woman’s whole body as a collection of bones to the Elder Mahā-Tissa through his merely looking at her teeth demonstrates this here (see \hyperlink{I.55}{I.55}{}).

                \vismParagraph{VI.82}{82}{}
                
                \begin{verse}
                    The Divine Ruler with ten hundred eyes\\{}
                    Did him with the Ten Powers eulogize,\\{}
                    Who, fair in fame, made known as cause of jhāna\\{}
                    This foulness of ten species in such wise.\\{}
                    Now, knowing their description and the way\\{}
                    To tackle each and how they are developed,\\{}
                    There are some further points that will repay\\{}
                    Study, each with its special part to play.
                \end{verse}

            \section[\vismAlignedParas{§83–94}General]{General}

                \vismParagraph{VI.83}{83}{}
                One who has reached jhāna in anyone of these goes free from cupidity; he resembles [an Arahant] without greed because his greed has been well suppressed. At the same time, however, this classification of foulness should be understood as stated in accordance with the particular individual essences successively reached by the [dead] body and also in accordance with the particular subdivisions of the greedy temperament.

                \vismParagraph{VI.84}{84}{}
                When a corpse has entered upon the repulsive state, it may have reached the individual essence of the bloated or anyone of the individual essences beginning with that of the livid. So the sign should be apprehended as “Repulsiveness of the bloated,” “Repulsiveness of the livid,” according to whichever he has been able to find. This, it should be understood, is how the classification of foulness comes to be tenfold with the body’s arrival at each particular individual essence.

                \vismParagraph{VI.85}{85}{}
                And individually \emph{the bloated }suits one who is greedy about shape since it makes evident the disfigurement of the body’s shape. \emph{The livid }suits one who is greedy about the body’s colour since it makes evident the disfigurement of the skin’s colour. \emph{The festering }\textcolor{brown}{\textit{[194]}} suits one who is greedy about the smell of the body aroused by scents, perfumes, etc., since it makes evident the evil smells connected with this sore, the body. \emph{The cut up} suits one who is greedy about compactness in the body since it makes evident the hollowness inside it. \emph{The gnawed }suits one who is greedy about accumulation of flesh in such parts of the body as the breasts since it makes it evident how a fine accumulation of flesh comes to nothing. \emph{The scattered }suits one who is greedy about the grace of the limbs since it makes it evident how limbs can be scattered. \emph{The hacked }and scattered suits one who is greedy about a fine body as a whole since it makes evident the disintegration and alteration of the body as a whole. \emph{The bleeding }suits one who is greedy about elegance produced by ornaments since it makes evident its repulsiveness when smeared with blood. \emph{The worm-infested }suits one who is greedy about ownership of the body since it makes it evident how the body is shared with many families of worms. \emph{A skeleton }suits one who is greedy about fine teeth since it makes evident the repulsiveness of the bones in the body. This, it should be understood, is how the classification of foulness comes to be tenfold according to the subdivisions of the greedy temperament.

                \vismParagraph{VI.86}{86}{}
                \marginnote{\textcolor{teal}{\footnotesize\{241|183\}}}{}But as regards the tenfold foulness, just as it is only by virtue of its rudder that a boat keeps steady in a river with turbulent\footnote{\vismAssertFootnoteCounter{13}\vismHypertarget{VI.n13}{}\emph{Aparisaṇṭhita—}“turbulent.” \emph{Parisaṇṭhāti} (to quiet) is not in PED. \emph{Aparisaṇṭhita} is not in CPD.} waters and a rapid current, and it cannot be steadied without a rudder, so too [here], owing to the weak hold on the object, consciousness when unified only keeps steady by virtue of applied thought, and it cannot be steadied without applied thought, which is why there is only the first jhāna here, not the second and the rest.

                \vismParagraph{VI.87}{87}{}
                And repulsive as this object is, still it arouses joy and happiness in him by his seeing its advantages thus, “\emph{Surely in this way I shall be liberated from ageing and death},” and by his abandoning the hindrances’ oppression; just as a garbage heap does in a flower-scavenger by his seeing the advantages thus, “Now I shall get a high wage,” and as the workings of purges and emetics do in a man suffering the pains of sickness.

                \vismParagraph{VI.88}{88}{}
                This foulness, while of ten kinds, has only one characteristic. For though it is of ten kinds, nevertheless its characteristic is only its impure, stinking, disgusting and repulsive state (essence). And foulness appears with this characteristic not only in a dead body but also in a living one, as it did to the Elder Mahā-Tissa who lived at Cetiyapabbata (\hyperlink{I.55}{I.55}{}), and to the novice attendant on the Elder Saṅgharakkhita while he was watching the king riding an elephant. For a living body is just as foul as a dead one, \textcolor{brown}{\textit{[195]}} only the characteristic of foulness is not evident in a living body, being hidden by adventitious embellishments.

                \vismParagraph{VI.89}{89}{}
                This is the body’s nature: it is a collection of over three hundred bones, jointed by one hundred and eighty joints, bound together by nine hundred sinews, plastered over with nine hundred pieces of flesh, enveloped in the moist inner skin, enclosed in the outer cuticle, with orifices here and there, constantly dribbling and trickling like a grease pot, inhabited by a community of worms, the home of disease, the basis of painful states, perpetually oozing from the nine orifices like a chronic open carbuncle, from both of whose eyes eye-filth trickles, from whose ears comes ear-filth, from whose nostrils snot, from whose mouth food and bile and phlegm and blood, from whose lower outlets excrement and urine, and from whose ninety-nine thousand pores the broth of stale sweat seeps, with bluebottles and their like buzzing round it, which when untended with tooth sticks and mouth-washing and head-anointing and bathing and underclothing and dressing would, judged by the universal repulsiveness of the body, make even a king, if he wandered from village to village with his hair in its natural wild disorder, no different from a flower-scavenger or an outcaste or what you will. So there is no distinction between a king’s body and an outcaste’s in so far as its impure stinking nauseating repulsiveness is concerned.

                \vismParagraph{VI.90}{90}{}
                But by rubbing out the stains on its teeth with tooth sticks and mouth-washing and all that, by concealing its private parts under several cloths, by daubing it with various scents and salves, by pranking it with nosegays and such things, it is worked up into a state that permits of its being taken as “I” and \marginnote{\textcolor{teal}{\footnotesize\{242|184\}}}{}“mine.” So men delight in women and women in men without perceiving the true nature of its characteristic foulness, now masked by this adventitious adornment. But in the ultimate sense there is no place here even the size of an atom fit to lust after.

                \vismParagraph{VI.91}{91}{}
                And then, when any such bits of it as head hairs, body hairs, nails, teeth, spittle, snot, excrement or urine have dropped off the body, beings will not touch them; they are ashamed, humiliated and disgusted. But as long as anyone of these things remains in it, though it is just as repulsive, they take it as agreeable, desirable, permanent, \textcolor{brown}{\textit{[196]}} pleasant, self, because they are wrapped in the murk of ignorance and dyed with affection and greed for self. Taking it as they do, they resemble the old jackal who saw a flower not yet fallen from a \emph{kiṃsuka }tree in a forest and yearned after it, thinking, “This is a piece of meat, it is a piece of meat.”

                \vismParagraph{VI.92}{92}{}
                
                \begin{verse}
                    There was a jackal chanced to see\\{}
                    A flowering \emph{kiṃsuka }in a wood;\\{}
                    In haste he went to where it stood:\\{}
                    “I have found a meat-bearing tree!”
                \end{verse}

                \begin{verse}
                    He chewed the blooms that fell, but could,\\{}
                    Of course, find nothing fit to eat;\\{}
                    He took it thus: “Unlike the meat\\{}
                    There on the tree, this is no good.”
                \end{verse}

                \begin{verse}
                    A wise man will not think to treat\\{}
                    As foul only the part that fell,\\{}
                    But treats as foul the part as well\\{}
                    That in the body has its seat.
                \end{verse}

                \begin{verse}
                    Fools cannot in their folly tell;\\{}
                    They take the body to be fair,\\{}
                    And soon get caught in Evil’s snare\\{}
                    Nor can escape its painful spell.
                \end{verse}

                \begin{verse}
                    But since the wise have thus laid bare\\{}
                    This filthy body’s nature, so,\\{}
                    Be it alive or dead, they know\\{}
                    There is no beauty lurking there.
                \end{verse}


                \vismParagraph{VI.93}{93}{}
                For this is said:
                \begin{verse}
                    “This filthy body stinks outright\\{}
                    Like ordure, like a privy’s site;\\{}
                    This body men that have insight\\{}
                    Condemn, as object of a fool’s delight.
                \end{verse}

                \begin{verse}
                    “A tumour where nine holes abide\\{}
                    Wrapped in a coat of clammy hide\\{}
                    And trickling filth on every side,\\{}
                    Polluting the air with stenches far and wide.
                \end{verse}

                \begin{verse}
                    \marginnote{\textcolor{teal}{\footnotesize\{243|185\}}}{}“If it perchance should come about\\{}
                    That what is inside it came out,\\{}
                    Surely a man would need a knout\\{}
                    With which to put the crows and dogs to rout.”
                \end{verse}


                \vismParagraph{VI.94}{94}{}
                So a capable bhikkhu should apprehend the sign wherever the aspect of foulness is manifest, whether in a living body or in a dead one, and he should make the meditation subject reach absorption.

                The sixth chapter called “The Description of Foulness as a Meditation Subject” in the Treatise on the Development of Concentration in the \emph{Path of Purification }composed for the purpose of gladdening good people.
        \chapter[Six Recollections]{Six Recollections\vismHypertarget{VII}\newline{\textnormal{\emph{Cha-anussati-niddesa}}}}
            \label{VII}


            \vismParagraph{VII.1}{1}{}
            \marginnote{\textcolor{teal}{\footnotesize\{244|186\}}}{}\textcolor{brown}{\textit{[197]}} Now, ten recollections were listed next after the ten kinds of foulness (\hyperlink{III.105}{III.105}{}). As to these:

            Mindfulness (\emph{sati}) itself is recollection (\emph{anussati}) because it arises again and again; or alternatively, the mindfulness (\emph{sati}) that is proper (\emph{anurūpa}) for a clansman gone forth out of faith, since it occurs only in those instances where it should occur, is “recollection” (\emph{anussati}).

            The recollection arisen inspired by the Enlightened One is the \emph{recollection of the Buddha}. This is a term for mindfulness with the Enlightened One’s special qualities as its object.

            The recollection arisen inspired by the Law is the \emph{recollection of the Dhamma}.\footnote{\vismAssertFootnoteCounter{1}\vismHypertarget{VII.n1}{}The word \emph{dhamma—}perhaps the most important and frequently used of Pali words—has no single equivalent in English because no English word has both a generalization so wide and loose as the word \emph{dhamma} in its widest sense (which includes “everything” that can be known or thought of in any way) and at the same time an ability to be, as it were, focused in a set of well-defined specific uses. Roughly \emph{dhamma} = what-can-be-remembered or what-can-be-borne-in-mind (\emph{dhāretabba}) as \emph{kamma} = what-can-be-done (\emph{kātabba}). The following two principal (and overlapping) senses are involved here: (i) the Law as taught, and (ii) objects of consciousness. (i) In the first case the word has either been left untranslated as “Dhamma” or “dhamma” or it has been tendered as “Law” or “law.” This ranges from the loose sense of the “Good Law,” “cosmic law,” and “teaching” to such specific technical senses as the “discrimination of law,” “causality,” “being subject to or having the nature of.” (ii) In the second case the word in its looser sense of “something known or thought of” has either been left untranslated as “dhamma” or rendered by “state” (more rarely by “thing” or “phenomenon”), while in its technical sense as one of the twelve bases or eighteen elements “mental object” and “mental datum” have been used. The sometimes indiscriminate use of “dhamma,” “state” and “law” in both the looser senses is deliberate. The English words have been reserved as far as possible for rendering \emph{dhamma} (except that “state” has sometimes been used to render \emph{bhāva}, etc., in the sense of “-ness”). Other subsidiary meanings of a non-technical nature have occasionally been otherwise rendered according to context.

                    In order to avoid muddle it is necessary to distinguish renderings of the word \emph{dhamma} and renderings of the words used to define it. The word itself is a gerundive of the verb \emph{dharati }(caus. \emph{dhāreti—}“to bear”) and so is the literal equivalent of “[quality] that is to be borne.” But since the grammatical meanings of the two words \emph{dharati} (“to bear”) and \emph{dahati }(“to put or sort out,” whence \emph{dhātu—}“element”) sometimes coalesce, it often comes very close to \emph{dhātu} (but see VIII n. 68 and \hyperlink{XI.104}{XI.104}{}). If it is asked, what bears the qualities to be borne? A correct answer here would probably be that it is the event (\emph{samaya}), as stated in the Dhammasaṅgaṇī (§1, etc.), in which the various dhammas listed there arise and are present, variously related to each other. The word \emph{dhammin} (thing qualified or “bearer of what is to be borne”) is a late introduction as a logical term (perhaps first used in Pali by \textbf{\cite{Vism-mhṭ}}, see p. 534).

                    As to the definitions of the word, there are several. At D-a I 99 four meanings are given: moral (meritorious) special quality (\emph{guṇa}), preaching of the Law (\emph{desanā}), scripture (\emph{pariyatti}), and “no-living-being-ness” (\emph{nissattatā}). Four meanings are also given at \textbf{\cite{Dhs-a}38}: scripture (\emph{pariyatti}), cause (of effect) as law (\emph{hetu}), moral (meritorious) special quality (\emph{guṇa}), and “no-living-being-ness and soullessness” (\emph{nissatta-nijjīvatā}). A wider definition is given at \textbf{\cite{M-a}I 17}, where the following meanings are distinguished: scriptural mastery, (\emph{pariyatti—}\textbf{\cite{A}III 86}) truth, (\emph{sacca—}\textbf{\cite{Vin}I 12}) concentration, (\emph{samādhi—}\textbf{\cite{D}II 54}) understanding, (\emph{paññā—}\textbf{\cite{J-a}I 280}) nature, (\emph{pakati—}\textbf{\cite{M}I 162}) individual essence, (\emph{sabhāva—}\textbf{\cite{Dhs}1}) voidness, (\emph{suññatā—}\textbf{\cite{Dhs}25}) merit, (\emph{puñña—}\textbf{\cite{S}I 82}) offence, (\emph{āpatti—}\textbf{\cite{Vin}III 187}) what is knowable, (\emph{ñeyya—}\textbf{\cite{Paṭis}II 194}) “\emph{and so on}” (see also VIII n. 68).} This is a term for mindfulness with the special qualities of the Law’s being well proclaimed, etc., as its object. \marginnote{\textcolor{teal}{\footnotesize\{245|187\}}}{}The recollection arisen inspired by the Community is the \emph{recollection of the Saṅgha}. This is a term for mindfulness with the Community’s special qualities of being entered on the good way, etc., as its object.

            The recollection arisen inspired by virtue is the \emph{recollection of virtue}. This is a term for mindfulness with the special qualities of virtue’s untornness, etc., as its object.

            The recollection arisen inspired by generosity is the \emph{recollection of generosity}. This is a term for mindfulness with generosity’s special qualities of free generosity, etc., as its object.

            The recollection arisen inspired by deities is the \emph{recollection of deities}. This is a term for mindfulness with the special qualities of one’s own faith, etc., as its object with deities standing as witnesses.

            The recollection arisen inspired by death is the \emph{recollection of death}. This is a term for mindfulness with the termination of the life faculty as its object.

            [\emph{Mindfulness occupied with the body (kāya-gatā sati—}lit. “body-gone mindfulness”):] it is gone (\emph{gata}) to the material body (\emph{kāya}) that is analyzed into head hairs, etc., or it is gone into the body, thus it is “body-gone” (\emph{kāya-gatā}). It is body-gone (\emph{kāya-gatā}) and it is mindfulness (\emph{sati}), thus it is “body-gone-mindfulness” (\emph{kāyagatasati—}single compound); but instead of shortening [the vowel] thus in the usual way, “body-gone mindfulness” (\emph{kāyagatā sati—}compound adj. + noun) is said. This is a term for mindfulness that has as its object the sign of the bodily parts consisting of head hairs and the rest.

            The mindfulness arisen inspired by breathing (\emph{ānāpāna}) is \emph{mindfulness of breathing}. This is a term for mindfulness that has as its object the sign of in-breaths and out-breaths.

            \marginnote{\textcolor{teal}{\footnotesize\{246|188\}}}{}The recollection arisen inspired by peace is the \emph{recollection of peace}. This is a term that has as its object the stilling of all suffering.
            \section[\vismAlignedParas{§2–67}(1) Recollection of the Enlightened One]{(1) Recollection of the Enlightened One}

                \vismParagraph{VII.2}{2}{}
                \textcolor{brown}{\textit{[198]}} Now, a meditator with absolute confidence\footnote{\vismAssertFootnoteCounter{2}\vismHypertarget{VII.n2}{}“‘\emph{Absolute confidence’} is the confidence afforded by the noble path. Development of the recollection comes to success in him who has that, not in any other” (Vism-mhṭ 181). “Absolute confidence” is a constituent of the first three “factors of stream-entry” (see \textbf{\cite{S}V 196}).} who wants to develop firstly the recollection of the Enlightened One among these ten should go into solitary retreat in a favourable abode and recollect the special qualities of the Enlightened One, the Blessed One, as follows:

                That Blessed One is such since he is accomplished, fully enlightened, endowed with [clear] vision and [virtuous] conduct, sublime, the knower of worlds, the incomparable leader of men to be tamed, the teacher of gods and men, enlightened and blessed (\textbf{\cite{M}I 37}; \textbf{\cite{A}III 285}).

                \vismParagraph{VII.3}{3}{}
                Here is the way he recollects: “That Blessed One is such since he is accomplished, he is such since he is fully enlightened, … he is such since he is blessed”—he is so for these several reasons, is what is meant.
                \subsection[\vismAlignedParas{§4–7}Accomplished]{Accomplished}

                    \vismParagraph{VII.4}{4}{}
                    Herein, what he recollects firstly is that the Blessed One is \emph{accomplished }(\emph{arahanta}) for the following reasons: (i) because of remoteness (\emph{āraka}), and (ii) because of his enemies (\emph{ari}) and (iii) the spokes (\emph{ara}) having been destroyed (\emph{hata}), and (iv) because of his worthiness (\emph{araha}) of requisites, etc., and (v) because of absence of secret (\emph{rahābhāva}) evil-doing.\footnote{\vismAssertFootnoteCounter{3}\vismHypertarget{VII.n3}{}Cf. derivation of the word \emph{ariya} (“noble”) at \textbf{\cite{M-a}I 21}.}

                    \vismParagraph{VII.5}{5}{}
                    (i) He stands utterly remote and far away from all defilements because he has expunged all trace of defilement by means of the path—because of such remoteness (\emph{āraka}) he is accomplished (\emph{arahanta}).
                    \begin{verse}
                        A man remote (\emph{āraka}) indeed we call\\{}
                        From something he has not at all;\\{}
                        The Saviour too that has no stain\\{}
                        May well the name “accomplished” (\emph{arahanta}) gain.
                    \end{verse}


                    \vismParagraph{VII.6}{6}{}
                    (ii) And these enemies (\emph{ari}), these defilements, are destroyed (\emph{hata}) by the path—because the enemies are thus destroyed he is accomplished (\emph{arahanta}) also.
                    \begin{verse}
                        The enemies (\emph{ari}) that were deployed,\\{}
                        Greed and the rest, have been destroyed (\emph{hata})\\{}
                        By his, the Helper’s, wisdom’s sword,\\{}
                        So he is “accomplished” (\emph{arahanta}), all accord.
                    \end{verse}


                    \vismParagraph{VII.7}{7}{}
                    (iii) Now, this wheel of the round of rebirths with its hub made of ignorance and of craving for becoming, with its spokes consisting of formations of merit and the rest, with its rim of ageing and death, which is joined to the chariot of \marginnote{\textcolor{teal}{\footnotesize\{247|189\}}}{}the triple becoming by piercing it with the axle made of the origins of cankers (see \textbf{\cite{M}I 55}), has been revolving throughout time that has no beginning. All of this wheel’s spokes (\emph{ara}) were destroyed (\emph{hata}) by him at the Place of Enlightenment, as he stood firm with the feet of energy on the ground of virtue, wielding with the hand of faith the axe of knowledge that destroys kamma—because the spokes are thus destroyed he is \emph{accomplished} (\emph{arahanta}) also.
                \subsection[\vismAlignedParas{§8–25}Dependent origination]{Dependent origination}

                    \vismParagraph{VII.8}{8}{}
                    Or alternatively, it is the beginningless round of rebirths that is called the “wheel of the round of rebirths.” Ignorance is its hub because it is its root. Ageing-and-death is its rim because it terminates it. The remaining ten states [of the dependent origination] are its spokes because ignorance is their root and ageing-and-death their termination.

                    \vismParagraph{VII.9}{9}{}
                    Herein, ignorance is unknowing about suffering and the rest. And ignorance in sensual becoming \textcolor{brown}{\textit{[199]}} is a condition for formations in sensual becoming. Ignorance in fine-material becoming is a condition for formations in fine-material becoming. Ignorance in immaterial becoming is a condition for formations in immaterial becoming.

                    \vismParagraph{VII.10}{10}{}
                    Formations in sensual becoming are a condition for rebirth-linking consciousness in sensual becoming. And similarly with the rest.

                    \vismParagraph{VII.11}{11}{}
                    Rebirth-linking consciousness in sensual becoming is a condition for mentality-materiality in sensual becoming. Similarly in fine-material becoming. In immaterial becoming it is a condition for mentality only.

                    \vismParagraph{VII.12}{12}{}
                    Mentality-materiality in sensual becoming is a condition for the sixfold base in sensual becoming. Mentality-materiality in fine-material becoming is a condition for three bases in fine-material becoming. Mentality in immaterial becoming is a condition for one base in immaterial becoming.

                    \vismParagraph{VII.13}{13}{}
                    The sixfold base in sensual becoming is a condition for six kinds of contact in sensual becoming. Three bases in fine-material becoming are conditions for three kinds of contact in fine-material becoming. The mind base alone in immaterial becoming is a condition for one kind of contact in immaterial becoming.

                    \vismParagraph{VII.14}{14}{}
                    The six kinds of contact in sensual becoming are conditions for six kinds of feeling in sensual becoming. Three kinds of contact in fine-material becoming are conditions for three kinds of feeling there too. One kind of contact in immaterial becoming is a condition for one kind of feeling there too.

                    \vismParagraph{VII.15}{15}{}
                    The six kinds of feeling in sensual becoming are conditions for the six groups of craving in sensual becoming. Three in the fine-material becoming are for three there too. One kind of feeling in the immaterial becoming is a condition for one group of craving in the immaterial becoming. The craving in the several kinds of becoming is a condition for the clinging there.

                    \vismParagraph{VII.16}{16}{}
                    Clinging, etc., are the respective conditions for becoming and the rest. In what way? Here someone thinks, “I shall enjoy sense desires,” and with sense-desire clinging as condition he misconducts himself in body, speech, and mind. Owing to the fulfilment of his misconduct he reappears in a state of loss (deprivation). The kamma that is the cause of his reappearance there is kamma-process\marginnote{\textcolor{teal}{\footnotesize\{248|190\}}}{} becoming, the aggregates generated by the kamma are rebirth-process becoming, the generating of the aggregates is birth, their maturing is ageing, their dissolution is death.

                    \vismParagraph{VII.17}{17}{}
                    Another thinks, “I shall enjoy the delights of heaven,” and in the parallel manner he conducts himself well. Owing to the fulfilment of his good conduct he reappears in a [sensual-sphere] heaven. The kamma that is the cause of his reappearance there is kamma-process becoming, and the rest as before.

                    \vismParagraph{VII.18}{18}{}
                    Another thinks, “I shall enjoy the delights of the Brahmā-world,” and with sense-desire clinging as condition he develops loving-kindness, compassion, gladness, and equanimity.\footnote{\vismAssertFootnoteCounter{4}\vismHypertarget{VII.n4}{}“Because of the words, ‘Also all dhammas of the three planes are sense desires (\emph{kāma}) in the sense of being desirable (\emph{kamanīya}) (Cf. \textbf{\cite{Nidd}I 1}: \emph{sabbepi kāmāvacarā dhammā, sabbepi rūpāvacarā dhammā, sabbepi arūpāvacarā dhammā … kāmanīyaṭṭhena … kāmā}), greed for becoming is sense-desire clinging’ (\textbf{\cite{Vism-mhṭ}184}). See \hyperlink{XII.72}{XII.72}{}. For the “way to the Brahmā-world” see \textbf{\cite{M}II 194–196}; 207f.} \textcolor{brown}{\textit{[200]}} Owing to the fulfilment of the meditative development he is reborn in the Brahmā-world. The kamma that is the cause of his rebirth there is kamma-process becoming, and the rest is as before.

                    \vismParagraph{VII.19}{19}{}
                    Yet another thinks, “I shall enjoy the delights of immaterial becoming,” and with the same condition he develops the attainments beginning with the base consisting of boundless space. Owing to the fulfilment of the development he is reborn in one of these states. The kamma that is the cause of his rebirth there is kamma-process becoming, the aggregates generated by the kamma are rebirth-process becoming, the generating of the aggregates is birth, their maturing is ageing, their dissolution is death (see \textbf{\cite{M}II 263}). The remaining kinds of clinging are construable in the same way.

                    \vismParagraph{VII.20}{20}{}
                    So, “Understanding of discernment of conditions thus, ‘Ignorance is a cause, formations are causally arisen, and both these states are causally arisen,’ is knowledge of the causal relationship of states. Understanding of discernment of conditions thus, ‘In the past and in the future ignorance is a cause, formations are causally arisen, and both these states are causally arisen,’ is knowledge of the causal relationship of states” (\textbf{\cite{Paṭis}I 50}), and all the clauses should be given in detail in this way.

                    \vismParagraph{VII.21}{21}{}
                    Herein, ignorance and formations are one summarization; consciousness, mentality-materiality, the sixfold base, contact, and feeling are another; craving, clinging, and becoming are another; and birth and ageing-and-death are another. Here the first summarization is past, the two middle ones are present, and birth and ageing-and-death are future. When ignorance and formations are mentioned, thentates, became dispassionate towards them, when his greed faded away, when he was liberated, then he destroyed, quite destroyed, abolished, the spokes of this wheel of the round of rebirths of the kind just described.

                    \vismParagraph{VII.22}{22}{}
                    Now, the Blessed One knew, saw, understood, and penetrated in all aspects this dependent origination with its four summarizations, its three times, its twenty aspects, and its three links. “Knowledge is in the sense of that being known,\footnote{\vismAssertFootnoteCounter{5}\vismHypertarget{VII.n5}{}Reading “\emph{taṃ ñātaṭṭthena ñāṇaṃ”} with \textbf{\cite{Vism-mhṭ}}.} and understanding is in the sense of the act of understanding that. \marginnote{\textcolor{teal}{\footnotesize\{249|191\}}}{}Hence it was said: ‘Understanding of discernment of conditions is knowledge of the causal relationship of states’” (\textbf{\cite{Paṭis}I 52}). Thus when the Blessed One, by correctly knowing these states with knowledge of relations of states, became dispassionate towards them, when his greed faded away, when he was liberated, then he destroyed, quite destroyed, abolished, the spokes of this wheel of the round of rebirths of the kind just described.

                    Because the spokes are thus destroyed he is accomplished (\emph{arahanta}) also.
                    \begin{verse}
                        \textcolor{brown}{\textit{[201]}}The spokes (\emph{ara}) of rebirth’s wheel have been\\{}
                        Destroyed (\emph{hata}) with wisdom’s weapon keen\\{}
                        By him, the Helper of the World,\\{}
                        And so “accomplished” (\emph{arahanta}) he is called.
                    \end{verse}


                    \vismParagraph{VII.23}{23}{}
                    (iv) And he is worthy (\emph{arahati}) of the requisites of robes, etc., and of the distinction of being accorded homage because it is he who is most worthy of offerings. For when a Perfect One has arisen, important deities and human beings pay homage to none else; for Brahmā Sahampati paid homage to the Perfect One with a jewelled garland as big as Sineru, and other deities did so according to their means, as well as human beings as King Bimbisāra [of Magadha] and the king of Kosala. And after the Blessed One had finally attained Nibbāna, King Asoka renounced wealth to the amount of ninety-six million for his sake and founded eight-four thousand monasteries throughout all Jambudīpa (India). And so, with all these, what need to speak of others? Because of worthiness of requisites he is accomplished (\emph{arahanta}) also.
                    \begin{verse}
                        So he is worthy, the Helper of the World,\\{}
                        Of homage paid with requisites; the word\\{}
                        “Accomplished” (\emph{arahanta}) has this meaning in the world:\\{}
                        Hence the Victor is worthy of that word.
                    \end{verse}


                    \vismParagraph{VII.24}{24}{}
                    (v) And he does not act like those fools in the world who vaunt their cleverness and yet do evil, but in secret for fear of getting a bad name. Because of absence of secret (\emph{rahābhāva}) evil-doing he is accomplished (\emph{arahanta}) also.
                    \begin{verse}
                        No secret evil deed may claim\\{}
                        An author so august; the name\\{}
                        “Accomplished” (\emph{arahanta}) is his deservedly\\{}
                        By absence of such secrecy (\emph{rahābhāva}).
                    \end{verse}


                    \vismParagraph{VII.25}{25}{}
                    So in all ways:
                    \begin{verse}
                        The Sage of remoteness unalloyed,\\{}
                        Vanquished defiling foes deployed,\\{}
                        The spokes of rebirth’s wheel destroyed,\\{}
                        Worthy of requisites employed,\\{}
                        Secret evil he does avoid:\\{}
                        For these five reasons he may claim\\{}
                        This word “accomplished” for his name.
                    \end{verse}

                \subsection[\vismAlignedParas{§26–29}Fully Enlightened]{Fully Enlightened}

                    \vismParagraph{VII.26}{26}{}
                    \marginnote{\textcolor{teal}{\footnotesize\{250|192\}}}{}He is fully enlightened (\emph{sammāsambuddha}) because he has discovered (\emph{buddha}) all things rightly (\emph{sammā}) and by himself (\emph{sāmaṃ}).

                    In fact, all things were discovered by him rightly by himself in that he discovered, of the things to be directly known, that they must be directly known (that is, learning about the four truths), of the things to be fully understood that they must be fully understood (that is, penetration of suffering), of the things to be abandoned that they must be abandoned (that is, penetration of the origin of suffering), of the things to be realized that they must be realized (that is, penetration of the cessation of suffering), and of the things to be developed that they must be developed (that is, penetration of the path). Hence it is said:
                    \begin{verse}
                        What must be directly known is directly known,\\{}
                        What has to be developed has been developed,\\{}
                        What has to be abandoned has been abandoned;\\{}
                        And that, brahman, is why I am enlightened (\textbf{\cite{Sn}558}).
                    \end{verse}


                    \vismParagraph{VII.27}{27}{}
                    \textcolor{brown}{\textit{[202]}} Besides, he has discovered all things rightly by himself step by step thus: The eye is the truth of suffering; the prior craving that originates it by being its root-cause is the truth of origin; the non-occurrence of both is the truth of cessation; the way that is the act of understanding cessation is the truth of the path. And so too in the case of the ear, the nose, the tongue, the body, and the mind.

                    \vismParagraph{VII.28}{28}{}
                    And the following things should be construed in the same way:

                    the six bases beginning with visible objects;

                    the six groups of consciousness beginning with eye-consciousness;

                    the six kinds of contact beginning with eye-contact;

                    the six kinds of feeling beginning with the eye-contact-born;

                    the six kinds of perception beginning with perception of visible objects;

                    the six kinds of volition beginning with volition about visible objects;

                    the six groups of craving beginning with craving for visible objects;

                    the six kinds of applied thought beginning with applied thought about visible objects;

                    the six kinds of sustained thought beginning with sustained thought about visible objects;

                    the five aggregates beginning with the aggregate of matter;

                    the ten kasiṇas;

                    the ten recollections;

                    the ten perceptions beginning with perception of the bloated;

                    the thirty-two aspects [of the body] beginning with head hairs;

                    the twelve bases;

                    the eighteen elements; the nine kinds of becoming beginning with sensual becoming;\footnote{\vismAssertFootnoteCounter{6}\vismHypertarget{VII.n6}{}See \hyperlink{XVII.253}{XVII.253f.}{} The word \emph{bhava} is rendered here both by “existence” and by “becoming.” The former, while less awkward to the ear, is inaccurate if it is allowed a flavour of staticness. “Becoming” will be more frequently used as this work proceeds. Loosely the two senses tend to merge. But technically, “existence” should perhaps be used only for \emph{atthitā}, which signifies the momentary existence of a \emph{dhamma} “possessed of the three instants of arising, presence, and dissolution.” “Becoming” then signifies the continuous flow or flux of such triple-instant moments; and it occurs in three main modes: sensual, fine-material, and immaterial. For remarks on the words “being” and “essence” see VIII n. 68.}

                    \marginnote{\textcolor{teal}{\footnotesize\{251|193\}}}{}the four jhānas beginning with the first;

                    the four measureless states beginning with the development of loving-kindness;

                    the four immaterial attainments;

                    the factors of the dependent origination in reverse order beginning with ageing-and-death and in forward order beginning with ignorance (cf. \hyperlink{XX.9}{XX.9}{}).

                    \vismParagraph{VII.29}{29}{}
                    Herein, this is the construction of a single clause [of the dependent origination]: Ageing-and-death is the truth of suffering, birth is the truth of origin, the escape from both is the truth of cessation, the way that is the act of understanding cessation is the truth of the path.

                    In this way he has discovered, progressively discovered, completely discovered, all states rightly and by himself step by step. Hence it was said above: “He is fully enlightened because he has discovered all things rightly and by himself” (\hyperlink{VII.26}{§26}{}).\footnote{\vismAssertFootnoteCounter{7}\vismHypertarget{VII.n7}{}“Is not unobstructed knowledge (\emph{anāvaraṇa-ñāṇa}) different from omniscient knowledge (\emph{sabbaññuta-ñāṇa})? Otherwise the words “Six kinds of knowledge unshared [by disciples]” (\textbf{\cite{Paṭis}I 3}) would be contradicted? [Note: The six kinds are: knowledge of what faculties prevail in beings, knowledge of the inclinations and tendencies of beings, knowledge of the Twin Marvel, knowledge of the attainment of the great compassion, omniscient knowledge, and unobstructed knowledge (see \textbf{\cite{Paṭis}I 133})].—There is no contradiction, because two ways in which a single kind of knowledge’s objective field occurs are described for the purpose of showing by means of this difference how it is not shared by others.

                            It is only one kind of knowledge; but it is called omniscient knowledge because its objective field consists of formed, unformed, and conventional (\emph{sammuti}) [i.e. conceptual] dhammas without remainder, and it is called unobstructed knowledge because of its unrestricted access to the objective field, because of absence of obstruction. And it is said accordingly in the Paṭisambhidā: “It knows all the formed and the unformed without remainder, thus it is omniscient knowledge. It has no obstruction therein, thus it is unobstructed knowledge” (\textbf{\cite{Paṭis}I 131}), and so on. So they are not different kinds of knowledge. And there must be no reservation, otherwise it would follow that omniscient and unobstructed knowledge had obstructions and did not make all dhammas its object. There is not in fact a minimal obstruction to the

                            Blessed One’s knowledge: and if his unobstructed knowledge did not have all dhammas as its object, there would be presence of obstruction where it did not occur, and so it would not be unobstructed.

                            “Or alternatively, even if we suppose that they are different, still it is omniscient knowledge itself that is intended as ‘unhindered’ since it is that which occurs unhindered universally. And it is by his attainment of that that the Blessed One is known as Omniscient, All-seer, Fully Enlightened, not because of awareness (\emph{avabodha}) of every dhamma at once, simultaneously (see \textbf{\cite{M}II 127}). And it is said accordingly in the Paṭisambhidā: ‘This is a name derived from the final liberation of the Enlightened Ones, the Blessed Ones, together with the acquisition of omniscient knowledge at the root of the Enlightenment Tree; this name “Buddha” is a designation based on realization’ (\textbf{\cite{Paṭis}I 174}). For the ability in the Blessed One’s continuity to penetrate all dhammas without exception was due to his having completely attained to knowledge capable of becoming aware of all dhammas.

                            “Here it may be asked: But how then? When this knowledge occurs, does it do so with respect to every field simultaneously, or successively? For firstly, if it occurs simultaneously with respect to every objective field, then with the simultaneous appearance of formed dhammas classed as past, future and present, internal and external, etc., and of unformed and conventional (conceptual) dhammas, there would be no awareness of contrast (\emph{paṭibhāga}), as happens in one who looks at a painted canvas from a distance. That being so, it follows that all dhammas become the objective field of the Blessed One’s knowledge in an undifferentiated form (\emph{anirūpita-rūpana}), as they do through the aspect of not-self to those who are exercising insight thus ’All dhammas are not-self’ (Dhp 279; Th 678; \textbf{\cite{M}I 230}; II 64; \textbf{\cite{S}III 132}; \textbf{\cite{A}I 286}; IV 14; \textbf{\cite{Paṭis}II 48}, 62; \textbf{\cite{Vin}I 86}. Cf. also \textbf{\cite{A}III 444}; IV 88, 338; \textbf{\cite{Sn}1076}). And those do not escape this difficulty who say that the Enlightened One’s knowledge occurs with the characteristic of presence of all knowable dhammas as its objective field, devoid of discriminative thinking (\emph{vikappa-rahita}), and universal in time (\emph{sabba-kāla}) and that is why they are called ’All-seeing’ and why it is said, ’The Nāga is concentrated walking and he is concentrated standing’ (?).

                            They do not escape the difficulty since the Blessed One’s knowledge would then have only a partial objective field, because, by having the characteristic of presence as its object, past, future and conventional dhammas, which lack that characteristic, would be absent. So it is wrong to say that it occurs simultaneously with respect to every objective field. Then secondly, if we say that it occurs successively with respect to every objective field, that is wrong too. For when the knowable, classed in the many different ways according to birth, place, individual essence, etc., and direction, place, time, etc., is apprehended successively, then penetration without remainder is not effected since the knowable is infinite. And those are wrong too who say that the Blessed One is All-seeing owing to his doing his defining by taking one part of the knowable as that actually experienced (\emph{paccakkha}) and deciding that the rest is the same because of the unequivocalness of its meaning, and that such knowledge is not inferential (\emph{anumānika}) since it is free from doubt, because it is what is doubtfully discovered that is meant by inferential knowledge in the world. And they are wrong because there is no such defining by taking one part of the knowable as that actually experienced and deciding that the rest is the same because of the unequivocalness of its meaning, without making all of it actually experienced. For then that ‘rest’ is not actually experienced; and if it were actually experienced, it would no longer be ‘the rest.’

                            “All that is no argument.—Why not?—Because this is not a field for ratiocination; for the Blessed One has said this: ‘The objective field of Enlightened Ones is unthinkable, it cannot be thought out; anyone who tried to think it out would reap madness and frustration’ (\textbf{\cite{A}II 80}). The agreed explanation here is this: Whatever the Blessed One wants to know—either entirely or partially—there his knowledge occurs as actual experience because it does so without hindrance. And it has constant concentration because of the absence of distraction. And it cannot occur in association with wishing of a kind that is due to absence from the objective field of something that he wants to know. There can be no exception to this because of the words, ‘All dhammas are available to the adverting of the Enlightened One, the Blessed One, are available at his wish, are available to his attention, are available to his thought’ (\textbf{\cite{Paṭis}II 195}). And the Blessed One’s knowledge that has the past and future as its objective field is entirely actual experience since it is devoid of assumption based on inference, tradition or conjecture.

                            “And yet, even in that case, suppose he wanted to know the whole in its entirety, then would his knowledge not occur without differentiation in the whole objective field simultaneously? And so there would still be no getting out of that difficulty? “That is not so, because of its purifiedness. Because the Enlightened One’s objective field is purified and it is unthinkable. Otherwise there would be no unthinkableness in the knowledge of the Enlightened One, the Blessed One, if it occurred in the same way as that of ordinary people. So, although it occurs with all dhammas as its object, it nevertheless does so making those dhammas quite clearly defined, as though it had a single dhamma as its object. This is what is unthinkable here. ‘

                            There is as much knowledge as there is knowable, there is as much knowable as there is knowledge; the knowledge is limited by the knowable, the knowable is limited by the knowledge’ (Paṭis II l95). So he is Fully Enlightened because he has rightly and by himself discovered all dhammas together and separately, simultaneously and successively, according to his wish’ (\textbf{\cite{Vism-mhṭ}190–191}).}
                \subsection[\vismAlignedParas{§30–32}Endowed With Clear Vision and Virtuous Conduct]{Endowed With Clear Vision and Virtuous Conduct}

                    \vismParagraph{VII.30}{30}{}
                    He is endowed with [clear] vision and [virtuous] conduct: \emph{vijjācaraṇasampanno = vijjāhi caraṇena ca sampanno} (resolution of compound). \marginnote{\textcolor{teal}{\footnotesize\{252|194\}}}{}Herein, as to [\emph{clear] vision}: there are three kinds of clear vision and eight kinds of clear vision. The three kinds should be understood as stated in the Bhayabherava Sutta (\textbf{\cite{M}I 22f.}), and the eight kinds as stated in the Ambaṭṭha Sutta (\textbf{\cite{D}I 100}). For there eight kinds of clear vision are stated, made up of the six kinds of direct-knowledge together with insight and the supernormal power of the mind-made [body].

                    \vismParagraph{VII.31}{31}{}
                    \marginnote{\textcolor{teal}{\footnotesize\{253|195\}}}{}[\emph{Virtuous] conduct} should be understood as fifteen things, that is to say: restraint by virtue, guarding of the sense faculties, knowledge of the right amount in eating, devotion to wakefulness, the seven good states,\footnote{\vismAssertFootnoteCounter{8}\vismHypertarget{VII.n8}{}A possessor of “the seven” has faith, conscience, shame, learning, energy, mindfulness, and understanding (see \textbf{\cite{D}III 252}). PED traces \emph{saddhamma} (as “the true dhamma,” etc.) to \emph{sant + dhamma}; but it is as likely traceable to \emph{srad + dhamma} = (good ground) for the placing of faith (\emph{saddhā}).} and the four jhānas of the fine-material sphere. For it is precisely by means of these fifteen things that a noble disciple conducts himself, that he goes towards the deathless. That is why it is called “[\emph{virtuous] conduct},” according as it is said, “Here, Mahānāma, a noble disciple has virtue” (\textbf{\cite{M}I 355}), etc, the whole of which should be understood as given in the Middle Fifty [of the Majjhima Nikāya]. \marginnote{\textcolor{teal}{\footnotesize\{254|196\}}}{}\textcolor{brown}{\textit{[203]}} Now, the Blessed One is endowed with these kinds of clear vision and with this conduct as well; hence he is called “endowed with [clear] vision and [\emph{virtuous] conduct}.”

                    \vismParagraph{VII.32}{32}{}
                    Herein, the Blessed One’s possession of clear vision consists in the fulfilment of omniscience (\textbf{\cite{Paṭis}I 131}), while his possession of conduct consists in the fulfilment of the great compassion (\textbf{\cite{Paṭis}I 126}). He knows through omniscience what is good and harmful for all beings, and through compassion he warns them of harm and exhorts them to do good. That is how he is possessed of clear vision and conduct, which is why his disciples have entered upon the good way instead of entering upon the bad way as the self-mortifying disciples of those who are not possessed of clear vision and conduct have done.\footnote{\vismAssertFootnoteCounter{9}\vismHypertarget{VII.n9}{}“Here the Master’s possession of vision shows the greatness of understanding, and his possession of conduct the greatness of his compassion. It was through understanding that the Blessed One reached the kingdom of the Dhamma, and through compassion that he became the bestower of the Dhamma. It was through understanding that he felt revulsion for the round of rebirths, and through compassion that he bore it. It was through understanding that he fully understood others’ suffering, and through compassion that he undertook to counteract it. It was through understanding that he was brought face to face with Nibbāna, and through compassion that he attained it. It was through understanding that he himself crossed over, and through compassion that he brought others across. It was through understanding that he perfected the Enlightened One’s state, and through compassion that he perfected the Enlightened One’s task.

                            “Or it was through compassion that he faced the round of rebirths as a Bodhisatta, and through understanding that he took no delight in it. Likewise it was through compassion that he practiced non-cruelty to others, and through understanding that he was himself fearless of others. It was through compassion that he protected others to protect himself, and through understanding that he protected himself to protect others. Likewise it was through compassion that he did not torment others, and through understanding that he did not torment himself; so of the four types of persons beginning with the one who practices for his own welfare (\textbf{\cite{A}II 96}) he perfected the fourth and best type. Likewise it was through compassion that he became the world’s helper, and through understanding that he became his own helper. It was through compassion that he had humility [as a Bodhisatta], and through understanding that he had dignity [as a Buddha]. Likewise it was through compassion that he helped all beings as a father while owing to the understanding associated with it his mind remained detached from them all, and it was through understanding that his mind remained detached from all dhammas while owing to the compassion associated with it that he was helpful to all beings. For just as the Blessed One’s compassion was devoid of sentimental affection or sorrow, so his understanding was free from the thoughts of ‘I’ and ‘mine’” (\textbf{\cite{Vism-mhṭ}192–193}).}
                \subsection[\vismAlignedParas{§33–35}Sublime]{Sublime}

                    \vismParagraph{VII.33}{33}{}
                    He is called \emph{sublime} (\emph{sugata})\footnote{\vismAssertFootnoteCounter{10}\vismHypertarget{VII.n10}{}The following renderings have been adopted for the most widely-used epithets for the Buddha. \emph{Tathāgata,} (Perfect One—for definitions see \textbf{\cite{M-a}I 45f.}) \emph{Bhagavant} (Blessed

                            One), \emph{Sugata} (Sublime One). These renderings do not pretend to literalness. Attempts to be literal here are apt to produce a bizarre or quaint effect, and for that very reason fail to render what is in the Pali.} (i) because of a manner of going that is good (\emph{sobhaṇa-gamana}), (ii) because of being gone to an excellent place (\emph{sundaraṃ }\marginnote{\textcolor{teal}{\footnotesize\{255|197\}}}{}\emph{ṭhānaṃ gatattā}), (iii) because of having gone rightly (\emph{sammāgatattā}), and (iv) because of enunciating rightly (\emph{sammāgadattā}).

                    (i) A manner of going (\emph{gamana}) is called “gone” (\emph{gata}), and that in the Blessed One is good (\emph{sobhaṇa}), purified, blameless. But what is that? It is the noble path; for by means of that manner of going he has “gone” without attachment in the direction of safety—thus he is sublime (\emph{sugata}) because of a manner of going that is good.

                    (ii) And it is to the excellent (\emph{sundara}) place that he has gone (\emph{gata}), to the deathless Nibbāna—thus he is sublime (\emph{sugata}) also because of having gone to an excellent place.

                    \vismParagraph{VII.34}{34}{}
                    (iii) And he has rightly (\emph{sammā}) gone (\emph{gata}), without going back again to the defilements abandoned by each path. For this is said: “He does not again turn, return, go back, to the defilements abandoned by the stream entry path, thus he is sublime … he does not again turn, return, go back, to the defilements abandoned by the Arahant path, thus he is sublime” (old commentary?). Or alternatively, he has rightly gone from the time of [making his resolution] at the feet of Dīpaṅkara up till the Enlightenment Session, by working for the welfare and happiness of the whole world through the fulfilment of the thirty perfections and through following the right way without deviating towards either of the two extremes, that is to say, towards eternalism or annihilationism, towards indulgence in sense pleasures or self-mortification—thus he is sublime also because of having gone rightly.

                    \vismParagraph{VII.35}{35}{}
                    (iv) And he enunciates\footnote{\vismAssertFootnoteCounter{11}\vismHypertarget{VII.n11}{}\emph{Gadati—}“to enunciate”: only noun \emph{gada} in PED.} (\emph{gadati}) rightly (\emph{sammā}); he speaks only fitting speech in the fitting place—thus he is sublime also because of enunciating rightly.

                    Here is a sutta that confirms this: “Such speech as the Perfect One knows to be untrue and incorrect, conducive to harm, and displeasing and unwelcome to others, that he does not speak. And such speech as the Perfect One knows to be true and correct, but conducive to harm, and displeasing and unwelcome to others, that he does not speak. \textcolor{brown}{\textit{[204]}} And such speech as the Perfect One knows to be true and correct, conducive to good, but displeasing and unwelcome to others, that speech the Perfect One knows the time to expound. Such speech as the Perfect One knows to be untrue and incorrect, and conducive to harm, but pleasing and welcome to others, that he does not speak. And such speech as the Perfect One knows to be true and correct, but conducive to harm, though pleasing and welcome to others, that he does not speak. And such speech as the Perfect One knows to be true and correct, conducive to good, and pleasing and welcome to others, that speech the Perfect One knows the time to expound” (\textbf{\cite{M}I 395})—thus he is sublime also because of enunciating rightly.
                \subsection[\vismAlignedParas{§36–45}Knower of Worlds]{Knower of Worlds}

                    \vismParagraph{VII.36}{36}{}
                    \marginnote{\textcolor{teal}{\footnotesize\{256|198\}}}{}He is the \emph{knower of worlds} because he has known the world in all ways. For the Blessed One has experienced, known and penetrated the world in all ways to its individual essence, its arising, its cessation, and the means to its cessation, according as it is said: “Friend, that there is a world’s end where one neither is born nor ages nor dies nor passes away nor reappears, which is to be known or seen or reached by travel—that I do not say. Yet I do not say that there is ending of suffering without reaching the world’s end. Rather, it is in this fathom-long carcass with its perceptions and its consciousness that I make known the world, the arising of the world, the cessation of the world, and the way leading to the cessation of the world.
                    \begin{verse}
                        “Tis utterly impossible\\{}
                        To reach by travel the world’s end;\\{}
                        But there is no escape from pain\\{}
                        Until the world’s end has been reached.\\{}
                        It is a sage, a knower of the worlds,\\{}
                        Who gets to the world’s end, and it is he\\{}
                        Whose life divine is lived out to its term;\\{}
                        He is at peace who the world’s end has known\\{}
                        And hopes for neither this world nor the next” (\textbf{\cite{S}I 62}).
                    \end{verse}


                    \vismParagraph{VII.37}{37}{}
                    Moreover, there are three worlds: the world of formations, the world of beings, and the world of location. Herein, in the passage, “One world: all beings subsist by nutriment” (\textbf{\cite{Paṭis}I 122}), \textcolor{brown}{\textit{[205]}} the world of formations is to be understood. In the passage, “‘The world is eternal’ or ‘The world is not eternal’” (\textbf{\cite{M}I 426}) it is the world of beings. In the passage:
                    \begin{verse}
                        “As far as moon and sun do circulate\\{}
                        Shining\footnote{\vismAssertFootnoteCounter{12}\vismHypertarget{VII.n12}{}\emph{Bhanti—}“they shine”: this form is not given in PED under \emph{bhāti}.} and lighting up the [four] directions,\\{}
                        Over a thousand times as great a world\\{}
                        Your power holds unquestionable sway” (\textbf{\cite{M}I 328})—
                    \end{verse}


                    it is the world of location. The Blessed One has known that in all ways too.

                    \vismParagraph{VII.38}{38}{}
                    Likewise, because of the words: “One world: all beings subsist by nutriment. Two worlds: mentality and materiality. Three worlds: three kinds of feeling. Four worlds: four kinds of nutriment. Five worlds: five aggregates as objects of clinging. Six worlds: six internal bases. Seven worlds: seven stations of consciousness. Eight worlds: eight worldly states. Nine worlds: nine abodes of beings. Ten worlds: ten bases. Twelve worlds: twelve bases. Eighteen worlds: eighteen elements” (\textbf{\cite{Paṭis}I 122}),\footnote{\vismAssertFootnoteCounter{13}\vismHypertarget{VII.n13}{}To take what is not self-evident in this paragraph, \emph{three kinds of feeling} are pleasant, painful and neither-painful-nor-pleasant (see MN 59). \emph{Four kinds of nutriment }are physical nutriment, contact, mental volition, and consciousness (see \textbf{\cite{M}I 48}, and \textbf{\cite{M-a}I 207f.}). \emph{The seven stations of consciousness} are: (1) sense sphere, (2) Brahmā’s Retinue, (3) Ābhassara (Brahmā-world) Deities, (4) Subhakiṇṇa (Brahmā-world) Deities, (5) base consisting of boundless space, (6) base consisting of boundless consciousness, (7) base consisting of nothingness (see \textbf{\cite{D}III 253}). \emph{The eight worldly states} are gain, fame, praise, pleasure, and their opposites (see \textbf{\cite{D}III 260}). \emph{The nine abodes of beings}: (1–4) as in stations of consciousness, (5) unconscious beings, (6–9) the four immaterial states (see \textbf{\cite{D}III 263}). \emph{The ten bases }are eye, ear, nose, tongue, body, visible object, sound, odour, flavour, tangible object.} this world of formations was known to him in all ways.

                    \vismParagraph{VII.39}{39}{}
                    But he knows all beings’ habits, knows their inherent tendencies, knows their temperaments, knows their bents, knows them as with little dust on their eyes and with much dust on their eyes, with keen faculties and with dull faculties, with good behaviour and with bad behaviour, easy to teach and hard to teach, \marginnote{\textcolor{teal}{\footnotesize\{257|199\}}}{}capable and incapable [of achievement] (cf. \textbf{\cite{Paṭis}I 121}), therefore this world of beings was known to him in all ways.

                    \vismParagraph{VII.40}{40}{}
                    And as the world of beings so also the world of location. For accordingly this [world measures as follows]:

                    One world-sphere\footnote{\vismAssertFootnoteCounter{14}\vismHypertarget{VII.n14}{}\emph{Cakkavāḷa} (world-sphere or universe) is a term for the concept of a single complete universe as one of an infinite number of such universes. This concept of the cosmos, in its general form, is not peculiar to Buddhism, but appears to have been the already generally accepted one. The term \emph{loka-dhātu} (world-element), in its most restricted sense, is one world-sphere, but it can be extended to mean any number, for example, the set of world-spheres dominated by a particular Brahmā (see MN 120).

                            As thus conceived, a circle of “world-sphere mountains” “like the rim of a wheel” (\emph{cakka—}\textbf{\cite{Vism-mhṭ}198}) encloses the ocean. In the centre of the ocean stands Mount Sineru (or Meru), surrounded by seven concentric rings of mountains separated by rings of sea. In the ocean between the outermost of these seven rings and the enclosing “world-sphere mountain” ring are the “four continents.”

                            “Over forty-two thousand leagues away” (\textbf{\cite{Dhs-a}313}) the moon and the sun circulate above them inside the world-sphere mountain ring, and night is the effect of the sun’s going behind Sineru. The orbits of the moon and sun are in the sense-sphere heaven of the Four Kings (\emph{Catumahārājā}), the lowest heaven, which is a layer extending from the world-sphere mountains to the slopes of Sineru. The stars are on both sides of them (Dhs-a 318). Above that come the successive layers of the other five sense-sphere heavens—the four highest not touching the earth—and above them the fine-material Brahmā-worlds, the higher of which extend over more than one world-sphere (see \textbf{\cite{A}V 59}). The world-sphere rests on water, which rests on air, which rests on space. World-spheres “lie adjacent to each other in contact like bowls, leaving a triangular unlit space between each three” (Vism-mhṭ 199), called a “world-interspace” (see too \textbf{\cite{M-a}IV 178}). Their numbers extend thus in all four directions to infinity on the supporting water’s surface.

                            The southern continent of Jambudīpa is the known inhabited world (but see e.g. DN 26). Various hells (see e.g. MN 130; \textbf{\cite{A}V 173}; \textbf{\cite{Vin}III 107}) are below the earth’s surface. The lowest sensual-sphere heaven is that of the Deities of the Four Kings

                            (\emph{Cātumahārājika}). The four are Dhataraṭṭha Gandhabba-rāja (King of the East), Virūḷha Kumbhaṇḍa-rāja (King of the South), Virūpaka Nāga-rāja (King of the West), and Kuvera or Vessavaṇa Yakkha-rāja (King of the North—see DN 32). Here the moon and sun circulate. The deities of this heaven are often at war with the Asura demons (see e.g. \textbf{\cite{D}II 285}) for possession of the lower slopes of Sineru. The next higher is Tāvatiṃsa (the Heaven of the Thirty-three), governed by Sakka, Ruler of Gods (\emph{sakka-devinda}). Above this is the heaven of the Yāma Deities (Deities who have Gone to Bliss) ruled by King Suyāma (not to be confused with Yama King of the Underworld—see \textbf{\cite{M}III 179}). Higher still come the Deities of the Tusita (Contented) Heaven with King Santusita. The fifth of these heavens is that of the Nimmānarati Deities (Deities who Delight in Creating) ruled by King Sunimmita. The last and highest of the sensual-sphere heavens is the Paranimmitavasavatti Heaven (Deities who Wield Power over Others’ Creations). Their king is Vasavatti (see \textbf{\cite{A}I 227}; for details see Vibh-a 519f.). Māra (Death) lives in a remote part of this heaven with his hosts, like a rebel with a band of brigands (\textbf{\cite{M-a}I 33f.}). For destruction and renewal of all this at the end of the aeon, see \hyperlink{XIII}{Ch. XIII}{}.} is twelve hundred thousand leagues and thirty-four hundred and fifty leagues (1,203,450) in breadth and width. In circumference, however:
                    \begin{verse}
                        [The measure of it] all around\\{}
                        Is six and thirty hundred thousand\\{}
                        And then ten thousand in addition,\\{}
                        Four hundred too less half a hundred (3,610,350).
                    \end{verse}


                    \vismParagraph{VII.41}{41}{}
                    \marginnote{\textcolor{teal}{\footnotesize\{258|200\}}}{}Herein:
                    \begin{verse}
                        Two times a hundred thousand leagues\\{}
                        And then four \emph{nahutas} as well (240,000):\\{}
                        This earth, this “Bearer of All Wealth,”\\{}
                        Has that much thickness, as they tell.
                    \end{verse}


                    And its support:
                    \begin{verse}
                        Four times a hundred thousand leagues\\{}
                        And then eight \emph{nahutas} as well (480,000):\\{}
                        The water resting on the air\\{}
                        Has that much thickness, as they tell.
                    \end{verse}


                    And the support of that: \textcolor{brown}{\textit{[206]}}
                    \begin{verse}
                        Nine times a hundred thousand goes\\{}
                        The air out in the firmament\\{}
                        And sixty thousand more besides (960,000)\\{}
                        So this much is the world’s extent.
                    \end{verse}


                    \vismParagraph{VII.42}{42}{}
                    Such is its extent. And these features are to be found in it:

                    Sineru, tallest of all mountains, plunges down into the sea Full four and eighty thousand leagues, and towers up in like degree Seven concentric mountain rings surround Sineru in suchwise That each of them in depth and height is half its predecessor’s size: Vast ranges called Yugandhara, Īsadhara, Karavīka, Sudassana, Nemindhara, Vinataka, Assakaṇṇa. Heavenly [breezes fan] their cliffs agleam with gems, and here reside The Four Kings of the Cardinal Points, and other gods and sprites beside.\footnote{\vismAssertFootnoteCounter{15}\vismHypertarget{VII.n15}{}“Sineru is not only 84,000 leagues in height but measures the same in width and breadth. For this is said: ‘Bhikkhus, Sineru, king of mountains, is eighty-four thousand leagues in width and it is eighty-four thousand leagues in breadth’ (\textbf{\cite{A}IV 100}). Each of the seven surrounding mountains is half as high as that last mentioned, that is, Yugandhara is half as high as Sineru, and so on. The great ocean gradually slopes from the foot of the world-sphere mountains down as far as the foot of Sineru, where it measures in depth as much as Sineru’s height. And Yugandhara, which is half that height, rests on the earth as Īsadhara and the rest do; for it is said: ‘Bhikkhus, the great ocean gradually slopes, gradually tends, gradually inclines’ (\textbf{\cite{Ud}53}). Between Sineru and Yugandhara and so on, the oceans are called ‘bottomless’ (\emph{sīdanta}). Their widths correspond respectively to the heights of Sineru and the rest. The mountains stand all round Sineru, enclosing it, as it were. Yugandhara surrounds Sineru, then Īsadhara surrounds Yugandhara, and likewise with the others” (\textbf{\cite{Vism-mhṭ}199}).} Himālaya’s lofty mountain mass rises in height five hundred leagues And in its width and in its breadth it covers quite three thousand leagues, And then it is bedecked besides with four and eighty thousand peaks.\footnote{\vismAssertFootnoteCounter{16}\vismHypertarget{VII.n16}{}For the commentarial descriptions of Himavant (Himalaya) with its five peaks and seven great lakes, see \textbf{\cite{M-a}III 54}.}

                    \marginnote{\textcolor{teal}{\footnotesize\{259|201\}}}{}The Jambu Tree called Nāga lends the name, by its magnificence, To Jambudīpa’s land; its trunk, thrice five leagues in circumference, Soars fifty leagues, and bears all round branches of equal amplitude, So that a hundred leagues define diameter and altitude.

                    \vismParagraph{VII.43}{43}{}
                    The World-sphere Mountains’ line of summits plunges down into the sea

                    Just two and eighty thousand leagues, and towers up in like degree, Enringing one world-element all round in its entirety.

                    And the size of the Jambu (Rose-apple) Tree is the same as that of the Citrapāṭaliya Tree of the Asura demons, the Simbali Tree of the Garuḷa demons, the Kadamba Tree in [the western continent of] Aparagoyana, the Kappa Tree [in the northern continent] of the Uttarakurus, the Sirīsa Tree in [the eastern continent of] Pubbavideha, and the Pāricchattaka Tree [in the heaven] of the Deities of the Thirty-three (Tāvatiṃsa).\footnote{\vismAssertFootnoteCounter{17}\vismHypertarget{VII.n17}{}A-a commenting on \textbf{\cite{A}I 35} ascribes the Simbali Tree to the Supaṇṇas or winged demons. The commentary to \textbf{\cite{Ud}5.5}, incidentally, gives a further account of all these things, only a small portion of which are found in the Suttas.} Hence the Ancients said:

                    The Pāṭali, Simbali, and Jambu, the deities’ Pāricchattaka, The Kadamba, the Kappa Tree and the Sirīsa as the seventh.

                    \vismParagraph{VII.44}{44}{}
                    \textcolor{brown}{\textit{[207]}} Herein, the moon’s disk is forty-nine leagues [across] and the sun’s disk is fifty leagues. The realm of Tāvatiṃsa (the Thirty-three Gods) is ten thousand leagues. Likewise the realm of the Asura demons, the great Avīci (unremitting) Hell, and Jambudīpa (India). Aparagoyāna is seven thousand leagues. Likewise Pubbavideha. Uttarakurū is eight thousand leagues. And herein, each great continent is surrounded by five hundred small islands. And the whole of that constitutes a single world-sphere, a single world-element. Between [this and the adjacent world-spheres] are the Lokantarika (world-interspace) hells.\footnote{\vismAssertFootnoteCounter{18}\vismHypertarget{VII.n18}{}See note 14.} So the world-spheres are infinite in number, the world-elements are infinite, and the Blessed One has experienced, known and penetrated them with the infinite knowledge of the Enlightened Ones.

                    \vismParagraph{VII.45}{45}{}
                    Therefore this world of location was known to him in all ways too. So he is “knower of worlds” because he has seen the world in all ways.
                \subsection[\vismAlignedParas{§46–48}Incomparable Leader of Men to be Tamed]{Incomparable Leader of Men to be Tamed}

                    \vismParagraph{VII.46}{46}{}
                    \marginnote{\textcolor{teal}{\footnotesize\{260|202\}}}{}In the absence of anyone more distinguished for special qualities than himself, there is no one to compare with him, thus he is \emph{incomparable}. For in this way he surpasses the whole world in the special quality of virtue, and also in the special qualities of concentration, understanding, deliverance, and knowledge and vision of deliverance. In the special quality of virtue he is without equal, he is the equal only of those [other Enlightened Ones] without equal, he is without like, without double, without counterpart; … in the special quality of knowledge and vision of deliverance he is … without counterpart, according as it is said: “I do not see in the world with its deities, its Māras and its Brahmās, in this generation with its ascetics and brahmans, with its princes and men,\footnote{\vismAssertFootnoteCounter{19}\vismHypertarget{VII.n19}{}The rendering of \emph{sadevamanussānaṃ} by “with its princes and men” is supported by the commentary. See \textbf{\cite{M-a}II 20} and also \textbf{\cite{M-a}I 33} where the use of \emph{sammuti-deva} for a royal personage, not an actual god is explained. \emph{Deva} is the normal mode of addressing a king. Besides, the first half of the sentence deals with deities and it would be out of place to refer to them again in the clause related to mankind.} anyone more perfect in virtue than myself” (\textbf{\cite{S}I 139}), with the rest in detail, and likewise in the Aggappasāda Sutta (\textbf{\cite{A}II 34}; It 87), and so on, and in the stanzas beginning, “I have no teacher and my like does not exist in all the world” (\textbf{\cite{M}I 171}), all of which should be taken in detail.

                    \vismParagraph{VII.47}{47}{}
                    He guides (\emph{sāreti}) men to be tamed (\emph{purisa-damme}), thus he is \emph{leader of men to be tamed} (\emph{purisadammasārathī}); he tames, he disciplines, is what is meant. Herein, animal males (\emph{purisā}) and human males, and non-human males that are not tamed but fit to be tamed (\emph{dametuṃ yuttā}) are “men to be tamed” (\emph{purisadammā}). For the animal males, namely, the royal nāga (serpent) Apalāla, Cūḷodara, Mahodara, Aggisikha, Dhūmasikha, the royal nāga Āravāḷa, the elephant Dhanapālaka, and so on, were tamed by the Blessed One, freed from the poison [of defilement] and established in the refuges and the precepts of virtue; and also the human males, namely, Saccaka the Nigaṇṭhas’ (Jains’) son, the brahman student Ambaṭṭha, \textcolor{brown}{\textit{[208]}} Pokkharasāti, Soṇadaṇḍa, Kūṭadanta, and so on; and also the non-human males, namely, the spirits Āḷavaka, Sūciloma and Kharaloma, Sakka Ruler of Gods, etc.,\footnote{\vismAssertFootnoteCounter{20}\vismHypertarget{VII.n20}{}The references are these: Apalāla (\emph{Mahāvaṃsa}, p. 242), “Dwelling in the Himalayas” (\textbf{\cite{Vism-mhṭ}202}), Cūḷodara and Mahodara (\textbf{\cite{Mhv}pp. 7–8}; \textbf{\cite{Dīp}pp. 21–23}), Aggisikha and Dhūmasikha (“Inhabitant of Sri Lanka”—\textbf{\cite{Vism-mhṭ}202}), Āravāḷa and Dhanapālaka (\textbf{\cite{Vin}II 194–196}; \textbf{\cite{J-a}V 333–337}), Saccaka (MN 35 and 36), Ambaṭṭha (DN 3), Pokkharasāti (\textbf{\cite{D}I 109}), Soṇadaṇḍa (DN 4), Kūṭadanta (DN 5), Āḷavaka (Sn p. 31), Sūciloma and Kharaloma (Sn p. 47f.), Sakka (\textbf{\cite{D}I 263f.}).} were tamed and disciplined by various disciplinary means. And the following sutta should be given in full here: “I discipline men to be tamed sometimes gently, Kesi, and I discipline them sometimes roughly, and I discipline them sometimes gently and roughly” (\textbf{\cite{A}II 112}).

                    \vismParagraph{VII.48}{48}{}
                    Then the Blessed One moreover further tames those already tamed, doing so by announcing the first jhāna, etc., respectively to those whose virtue is purified, etc., and also the way to the higher path to stream enterers, and so on. \marginnote{\textcolor{teal}{\footnotesize\{261|203\}}}{}Or alternatively, the words \emph{incomparable leader of men} to be tamed can be taken together as one clause. For the Blessed One so guides men to be tamed that in a single session they may go in the eight directions [by the eight liberations] without hesitation. Thus he is called the \emph{incomparable leader of men to be tamed}. And the following sutta passage should be given in full here: “Guided by the elephant-tamer, bhikkhus, the elephant to be tamed goes in one direction …” (\textbf{\cite{M}III 222}).
                \subsection[\vismAlignedParas{§49–51}Teacher of Gods and Men]{Teacher of Gods and Men}

                    \vismParagraph{VII.49}{49}{}
                    He teaches (\emph{anusāsati}) by means of the here and now, of the life to come, and of the ultimate goal, according as befits the case, thus he is the Teacher (\emph{satthar}). And furthermore this meaning should be understood according to the Niddesa thus: “‘Teacher (\emph{satthar})’: the Blessed One is a caravan leader (\emph{satthar}) since he brings home caravans (\emph{sattha}). Just as one who brings a caravan home gets caravans across a wilderness, gets them across a robber-infested wilderness, gets them across a wild-beast-infested wilderness, gets them across a foodless wilderness, gets them across a waterless wilderness, gets them right across, gets them quite across, gets them properly across, gets them to reach a land of safety, so too the Blessed One is a caravan leader, one who brings home the caravans, he gets them across a wilderness, gets them across the wilderness of birth” (\textbf{\cite{Nidd}I 446}).

                    \vismParagraph{VII.50}{50}{}
                    \emph{Of gods and men: devamanussānaṃ = devānañ ca manussānañ ca} (resolution of compound). This is said in order to denote those who are the best and also to denote those persons capable of progress. For the Blessed One as a teacher bestowed his teaching upon animals as well. For when animals can, through listening to the Blessed One’s Dhamma, acquire the benefit of a [suitable rebirth as] support [for progress], and with the benefit of that same support they come, in their second or third rebirth, to partake of the path and its fruition.

                    \vismParagraph{VII.51}{51}{}
                    Maṇḍūka, the deity’s son, and others illustrate this. While the Blessed One was teaching the Dhamma to the inhabitants of the city of Campā on the banks of the Gaggarā Lake, it seems, a frog (\emph{maṇḍūka}) apprehended a sign in the Blessed One’s voice. \textcolor{brown}{\textit{[209]}} A cowherd who was standing leaning on a stick put his stick on the frog’s head and crushed it. He died and was straight away reborn in a gilded, divine palace, twelve leagues broad in the realm of the Thirty-three (\emph{Tāvatiṃsa}). He found himself there, as if waking up from sleep, amidst a host of celestial nymphs, and he exclaimed, “So I have actually been reborn here. What deed did I do?” When he sought for the reason, he found it was none other than his apprehension of the sign in the Blessed One’s voice. He went with his divine palace at once to the Blessed One and paid homage at his feet. Though the Blessed One knew about it, he asked him:
                    \begin{verse}
                        “Who now pays homage at my feet,\\{}
                        Shining with glory of success,\\{}
                        Illuminating all around\\{}
                        With beauty so outstanding?”
                    \end{verse}

                    \begin{verse}
                        \marginnote{\textcolor{teal}{\footnotesize\{262|204\}}}{}“In my last life I was a frog,\\{}
                        The waters of a pond my home;\\{}
                        A cowherd’s crook ended my life\\{}
                        While listening to your Dhamma” (\textbf{\cite{Vv}49}).
                    \end{verse}


                    The Blessed One taught him the Dhamma. Eighty-four thousand creatures gained penetration to the Dhamma. As soon as the deity’s son became established in the fruition of stream-entry he smiled and then vanished.
                \subsection[\vismAlignedParas{§52}Enlightened]{Enlightened}

                    \vismParagraph{VII.52}{52}{}
                    He is \emph{enlightened} (\emph{buddha}) with the knowledge that belongs to the fruit of liberation, since everything that can be known has been discovered (\emph{buddha}) by him.

                    Or alternatively, he discovered (\emph{bujjhi}) the four truths by himself and awakened (\emph{bodhesi}) others to them, thus and for other such reasons he is enlightened (\emph{buddha}). And in order to explain this meaning the whole passage in the Niddesa beginning thus: “He is the discoverer (\emph{bujjhitar}) of the truths, thus he is enlightened (\emph{buddha}). He is the awakened (\emph{bodhetar}) of the generation, thus he is enlightened (\emph{buddha})” (\textbf{\cite{Nidd}I 457}), or the same passage from the Paṭisambhidā (\textbf{\cite{Paṭis}I 174}), should be quoted in detail.
                \subsection[\vismAlignedParas{§53–67}Blessed]{Blessed}

                    \vismParagraph{VII.53}{53}{}
                    Blessed (\emph{bhagavant}) is a term signifying the respect and veneration accorded to him as the highest of all beings and distinguished by his special qualities.\footnote{\vismAssertFootnoteCounter{21}\vismHypertarget{VII.n21}{}For the breaking up of this compound cf. parallel passage at \textbf{\cite{M-a}I 10}.} Hence the Ancients said:
                    \begin{verse}
                        “Blessed” is the best of words,\\{}
                        “Blessed” is the finest word;\\{}
                        Deserving awe and veneration,\\{}
                        Blessed is the name therefore.
                    \end{verse}


                    \vismParagraph{VII.54}{54}{}
                    Or alternatively, names are of four kinds: denoting a period of life, describing a particular mark, signifying a particular acquirement, and fortuitously arisen,\footnote{\vismAssertFootnoteCounter{22}\vismHypertarget{VII.n22}{}\emph{Āvatthika—}“denoting a period in life” (from \emph{avatthā}, see \hyperlink{IV.167}{IV.167}{}); not in PED; the meaning given in the PED for \emph{liṅgika—}“describing a particular mark,” is hardly adequate for this ref.; \emph{nemittika—}“signifying a particular acquirement” is not in this sense in PED. For more on names see \textbf{\cite{Dhs-a}390}.} which last in the current usage of the world is called “capricious.” Herein, \textcolor{brown}{\textit{[210]}} names denoting a period of life are those such as “yearling calf” (\emph{vaccha}), “steer to be trained” (\emph{damma}), “yoke ox” (\emph{balivaddha}), and the like. \emph{Names describing a particular mark} are those such as “staff-bearer” (\emph{daṇḍin}), “umbrella-bearer” (\emph{chattin}), “topknot-wearer” (\emph{sikhin}), “hand possessor” (\emph{karin—}elephant), and the like. \emph{Names signifying a particular acquirement} are those such as “possessor of the threefold clear vision” (\emph{tevijja}), “possessor of the six direct-knowledges” (\emph{chaḷabhiñña}), and the like. Such names are Sirivaḍḍhaka (“Augmenter of \marginnote{\textcolor{teal}{\footnotesize\{263|205\}}}{}Lustre”), Dhanavaḍḍhaka (“Augmenter of Wealth”), etc., are \emph{fortuitously arisen names}; they have no reference to the word-meanings.

                    \vismParagraph{VII.55}{55}{}
                    This name, \emph{Blessed}, is one signifying a particular acquirement; it is not made by Mahā-Māyā, or by King Suddhodana, or by the eighty thousand kinsmen, or by distinguished deities like Sakka, Santusita, and others. And this is said by the General of the Law:\footnote{\vismAssertFootnoteCounter{23}\vismHypertarget{VII.n23}{}The commentarial name for the Elder Sāriputta to whom the authorship of the Paṭisambhidā is traditionally attributed. The Paṭisambhidā text has “Buddha,” not “Bhagavā.”} “‘Blessed’: this is not a name made by a mother … This [name] ‘Buddha,’ which signifies final liberation, is a realistic description of Buddhas (Enlightened Ones), the Blessed Ones, together with their obtainment of omniscient knowledge at the root of an Enlightenment [Tree]” (\textbf{\cite{Paṭis}I 174}; \textbf{\cite{Nidd}I 143}).

                    \vismParagraph{VII.56}{56}{}
                    Now, in order to explain also the special qualities signified by this name they cite the following stanza:
                    \begin{verse}
                        \emph{Bhagī bhajī bhāgī vibhattavā iti}\\{}
                        \emph{Akāsi bhaggan ti garū ti bhāgyavā}\\{}
                        \emph{Bahūhi ñāyehi subhāvitattano}\\{}
                        \emph{Bhavantago so bhagavā ti vuccati.}
                    \end{verse}


                    The reverend one (\emph{garu}) has blessings (\emph{bhagī}), is a frequenter (\emph{bhajī}), a partaker (\emph{bhāgī}), a possessor of what has been analyzed (\emph{vibhattavā});

                    He has caused abolishing (\emph{bhagga}), he is fortunate (\emph{bhāgyavā}),

                    He has fully developed himself (\emph{subhāvitattano}) in many ways;

                    He has gone to the end of becoming (\emph{bhavantago}); thus is called “Blessed” (\emph{bhagavā}).

                    The meaning of these words should be understood according to the method of explanation given in the Niddesa (\textbf{\cite{Nidd}I 142}).\footnote{\vismAssertFootnoteCounter{24}\vismHypertarget{VII.n24}{}“The Niddesa method is this: ‘The word Blessed (\emph{bhagavā}) is a term of respect. Moreover, he has abolished (\emph{bhagga}) greed, thus he is blessed (\emph{bhagavā}); he has abolished hate, … delusion, … views, … craving, … defilement, thus he is blessed.

                            “‘He divided (\emph{bhaji}), analyzed (\emph{vibhaji}), and classified (\emph{paṭivibhaji}) the Dhamma treasure, thus he is blessed (\emph{bhagavā}). He makes an end of the kinds of becoming (\emph{bhavānaṃ antakaroti}), thus he is blessed (\emph{bhagavā}). He has developed (\emph{bhāvita}) the body and virtue and the mind and understanding, thus he is blessed (\emph{bhagavā}).

                            “‘Or the Blessed One is a frequenter (\emph{bhajī}) of remote jungle-thicket resting places with little noise, with few voices, with a lonely atmosphere, where one can lie hidden from people, favourable to retreat, thus he is blessed (\emph{bhagavā}).

                            “‘Or the Blessed One is a partaker (\emph{bhāgī}) of robes, alms food, resting place, and the requisite of medicine as cure for the sick, thus he is blessed (\emph{bhagavā}). Or he is a partaker of the taste of meaning, the taste of the Law, the taste of deliverance, the higher virtue, the higher consciousness, the higher understanding, thus he is blessed (\emph{bhagavā}). Or he is a partaker of the four jhānas, the four measureless states, the four immaterial states, thus he is blessed. Or he is a partaker of the eight liberations, the eight bases of mastery, the nine successive attainments, thus he is blessed. Or he is a partaker of the ten developments of perception, the ten kasiṇa attainments, concentration due to mindfulness of breathing, the attainment due to foulness, thus he is blessed. Or he is a partaker of the ten powers of Perfect Ones (see MN 12), of the four kinds of perfect confidence (\emph{ibid}), of the four discriminations, of the six kinds of direct knowledge, of the six Enlightened Ones’ states [not shared by disciples (see note 7)], thus he is blessed. Blessed One (\emph{bhagavā}): this is not a name made by a mother … This name, Blessed One, is a designation based on realization”’ (\textbf{\cite{Vism-mhṭ}207}).}

                    \vismParagraph{VII.57}{57}{}
                    \marginnote{\textcolor{teal}{\footnotesize\{264|206\}}}{}But there is this other way:
                    \begin{verse}
                        \emph{Bhāgyavā bhaggavā yutto bhagehi ca vibhattavā.}\\{}
                        \emph{Bhattavā vanta-gamano bhavesu: bhagavā tato.}
                    \end{verse}


                    He is fortunate (\emph{bhāgyavā}), possessed of abolishment (\emph{bhaggavā}), associated with blessings (\emph{yutto bhagehi}), and a possessor of what has been analyzed (\emph{vibhattavā}).

                    He has frequented (\emph{bhattavā}), and he has rejected going in the kinds of becoming (\emph{VAnta-GAmano BHAvesu}), thus he is Blessed (\emph{Bhagavā}).

                    \vismParagraph{VII.58}{58}{}
                    Herein, by using the characteristic of language beginning with “vowel augmentation of syllable, elision of syllable” (see \emph{Kāśika} \hyperlink{VI.3}{VI.3}{}.109), or by using the characteristic of insertion beginning with [the example of] \emph{pisodara}, etc. (see Pāṇini, \emph{Gaṇapāṭha} 6, 3, 109), it may be known that he [can also] be called “blessed” (\emph{bhagavā}) when he can be called “fortunate” (\emph{bhāgyavā}) owing to the fortunateness (\emph{bhāgya}) to have reached the further shore [of the ocean of perfection] of giving, virtue, etc., which produce mundane and supramundane bliss (See Khp-a 108.).

                    \vismParagraph{VII.59}{59}{}
                    [Similarly], he [can also] be called “blessed” (\emph{bhagavā}) when he can be called “possessed of abolishment” (\emph{bhaggavā}) owing to the following menaces having been abolished; for he has abolished (\emph{abhañji}) all the hundred thousand kinds of trouble, anxiety and defilement classed as greed, as hate, as delusion, and as misdirected attention; as consciencelessness and shamelessness, as anger and enmity, as contempt and domineering, as envy and avarice, as deceit and fraud, as obduracy and presumption, as pride and haughtiness, as vanity and negligence, as craving and ignorance; as the three roots of the unprofitable, kinds of misconduct, defilement, stains, \textcolor{brown}{\textit{[211]}} fictitious perceptions, applied thoughts, and diversifications; as the four perversenesses, cankers, ties, floods, bonds, bad ways, cravings, and clingings; as the five wildernesses in the heart, shackles in the heart, hindrances, and kinds of delight; as the six roots of discord, and groups of craving; as the seven inherent tendencies; as the eight wrongnesses; as the nine things rooted in craving; as the ten courses of unprofitable action; as the sixty-two kinds of [false] view; as the hundred and eight ways of behaviour of craving\footnote{\vismAssertFootnoteCounter{25}\vismHypertarget{VII.n25}{}Here are explanations of those things in this list that cannot be discovered by reference to the index: The pairs, “anger and enmity” to “conceit and negligence (\textbf{\cite{M}I 16}). The “three roots” are greed, hate, and delusion (\textbf{\cite{D}III 214}). The “three kinds of misconduct” are that of body, speech, and mind (\textbf{\cite{S}V 75}). The “three defilements” are misconduct, craving and views (Ch. \hyperlink{I.9}{I.9}{},13). The “three erroneous perceptions” (\emph{visama-saññā}) are those connected with greed, hate, and delusion (Vibh 368). The three “applied thoughts” are thoughts of sense-desire, ill will, and cruelty (\textbf{\cite{M}I 114}). The “three diversifications” (\emph{papañca}) are those due to craving, conceit, and [false] views (XVI n. 17). “Four perversenesses”: seeing permanence, pleasure, self, and beauty, where there is none (\textbf{\cite{Vibh}376}). “Four cankers,” etc. (\hyperlink{XXII.47}{XXII.47ff.}{}). “Five wildernesses” and “shackles” (\textbf{\cite{M}I 101}). “Five kinds of delight”: delight in the five aggregates (\hyperlink{XVI.93}{XVI.93}{}). “Six roots of discord”: anger, contempt, envy, fraud, evilness of wishes, and adherence to one’s own view (\textbf{\cite{D}III 246}). “Nine things rooted in craving” (\textbf{\cite{D}III 288–289}). “Ten courses of unprofitable action”: killing, stealing, sexual misconduct, lying, slander, harsh speech, gossip, covetousness, ill will, wrong view (\textbf{\cite{M}I 47}, 286f.). “Sixty-two kinds of view”: (\textbf{\cite{D}I 12ff.}; MN 102). “The hundred and eight ways of behaviour of craving” (\textbf{\cite{Vibh}400}).}—or in brief, the five Māras, that is to say, the \marginnote{\textcolor{teal}{\footnotesize\{265|207\}}}{}Māras of defilement, of the aggregates, and of kamma-formations, Māra as a deity, and Māra as death.

                    And in this context it is said:
                    \begin{verse}
                        He has abolished (\emph{bhagga}) greed and hate,\\{}
                        Delusion too, he is canker-free;\\{}
                        Abolished every evil state,\\{}
                        “Blessed” his name may rightly be.
                    \end{verse}


                    \vismParagraph{VII.60}{60}{}
                    And by his fortunateness (\emph{bhāgyavatā}) is indicated the excellence of his material body which bears a hundred characteristics of merit; and by his having abolished defects (\emph{bhaggadosatā}) is indicated the excellence of his Dhamma body. Likewise, [by his fortunateness is indicated] the esteem of worldly [people; and by his having abolished defects, the esteem of] those who resemble him. [And by his fortunateness it is indicated] that he is fit to be relied on\footnote{\vismAssertFootnoteCounter{26}\vismHypertarget{VII.n26}{}\emph{Abhigamanīya—}“fit to be relied on”: \emph{abhigacchati} not in PED.} by laymen; and [by his having abolished defects that he is fit to be relied on by] those gone forth into homelessness; and when both have relied on him, they acquire relief from bodily and mental pain as well as help with both material and Dhamma gifts, and they are rendered capable of finding both mundane and supramundane bliss.

                    \vismParagraph{VII.61}{61}{}
                    He is also called “blessed” (\emph{bhagavā}) since he is “\emph{associated with blessings}” (\emph{bhagehi yuttattā}) such as those of the following kind, in the sense that he “has those blessings” (\emph{bhagā assa santi}). Now, in the world the word “blessing” is used for six things, namely, lordship, Dhamma, fame, glory, wish, and endeavour. He has supreme \emph{lordship} over his own mind, either of the kind reckoned as mundane and consisting in “minuteness, lightness,” etc.,\footnote{\vismAssertFootnoteCounter{27}\vismHypertarget{VII.n27}{}\textbf{\cite{Vism-mhṭ}} says the word “\emph{etc.}” includes the following six: \emph{mahimā, patti, pākamma, īsitā, vasitā}, and \emph{yatthakāmāvasāyitā}. “Herein, \emph{aṇimā} means making the body minute (the size of an atom—\emph{aṇu}). \emph{Laghimā} means lightness of body; walking on air, and so on. \emph{Mahimā} means enlargement producing hugeness of the body. \emph{Patti} means arriving where one wants to go. \emph{Pākamma} means producing what one wants by resolving, and so on. \emph{Isitā} means self-mastery, lordship. \emph{Vasitā} means mastery of miraculous powers. \emph{Yatthakāmāvasāyitā} means attainment of perfection in all ways in one who goes through the air or does anything else of the sort” (\textbf{\cite{Vism-mhṭ}210}). \emph{Yogabhāṣya} 3.45.} or that complete in all aspects, and likewise the supramundane \emph{Dhamma}. And he has exceedingly pure \emph{fame}, spread through the three worlds, acquired though the special quality of veracity. And he has \emph{glory} of all limbs, perfect in every aspect, which is capable of comforting the eyes of people eager to see his material body. And he has his \emph{wish}, in other words, the production of what is wanted, since whatever is wanted and \marginnote{\textcolor{teal}{\footnotesize\{266|208\}}}{}needed by him as beneficial to himself or others is then and there produced for him. And he has the \emph{endeavour}, in other words, the right effort, which is the reason why the whole world venerates him.

                    \vismParagraph{VII.62}{62}{}
                    [He can also] be called “blessed” (\emph{bhagavā}) when he can be called “\emph{a possessor of what has been analyzed}” (\emph{vibhattavā}) owing to his having analyzed [and clarified] all states into the [three] classes beginning with the profitable; or profitable, etc., states into such classes as aggregates, bases, elements, truths, faculties, dependent origination, etc.; \textcolor{brown}{\textit{[212]}} or the noble truth of suffering into the senses of oppressing, being formed, burning, and changing; and that of origin into the senses of accumulating, source, bond, and impediment; and that of cessation into the senses of escape, seclusion, being unformed, and deathless; and that of the path into the senses of outlet, cause, seeing, and predominance. Having analyzed, having revealed, having shown them, is what is meant.

                    \vismParagraph{VII.63}{63}{}
                    He [can also] be called “blessed” (\emph{bhagavā}) when he can be called one who “\emph{has frequented}” (\emph{bhattavā}) owing to his having frequented (\emph{bhaji}), cultivated, repeatedly practiced, such mundane and supramundane higher-than-human states as the heavenly, the divine, and the noble abidings,\footnote{\vismAssertFootnoteCounter{28}\vismHypertarget{VII.n28}{}The three “abidings” are these: heavenly abiding = kasiṇa jhāna, divine abiding = loving-kindness jhāna, etc., noble abiding = fruition attainment. For the three kinds of seclusion, see IV, note 23.} as bodily, mental, and existential seclusion, as the void, the desireless, and the signless liberations, and others as well.

                    \vismParagraph{VII.64}{64}{}
                    He [can also] be called “blessed” (\emph{bhagavā}) when he can be called one who “\emph{has rejected going in the kinds of becoming}” (\emph{vantagamano bhavesu}) because in the three kinds of becoming (\emph{bhava}), the going (\emph{gamana}), in other words, craving, has been rejected (\emph{vanta}) by him. And the syllables \emph{bha} from the word \emph{bhava}, and \emph{ga }from the word \emph{gamana}, and \emph{va} from the word \emph{vanta} with the letter a lengthened, make the word \emph{bhagavā}, just as is done in the world [of the grammarians outside the Dispensation] with the word \emph{mekhalā} (waist-girdle) since “garland for the private parts” (\emph{MEhanassa KHAssa māLĀ}) can be said.

                    \vismParagraph{VII.65}{65}{}
                    As long as [the meditator] recollects the special qualities of the Buddha in this way, “For this and this reason the Blessed One is accomplished, … for this and this reason he is blessed,” then: “On that occasion his mind is not obsessed by greed, or obsessed by hate, or obsessed by delusion; his mind has rectitude on that occasion, being inspired by the Perfect One” (\textbf{\cite{A}III 285}).\footnote{\vismAssertFootnoteCounter{29}\vismHypertarget{VII.n29}{}\textbf{\cite{Vism-mhṭ}} adds seven more plays on the word \emph{bhagavā}, which in brief are these: he is \emph{bhāgavā} (a possessor of parts) because he has the Dhamma aggregates of virtue, etc. (\emph{bhāgā} = part, \emph{vant} = possessor of). He is \emph{bhatavā} (possessor of what is borne) because he has borne (\emph{bhata}) the perfections to their full development. He has cultivated the parts (\emph{bhāge vani}), that is, he has developed the various classes of attainments. He has cultivated the blessings (\emph{bhage vani}), that is, the mundane and supramundane blessings. He is \emph{bhattavā} (possessor of devotees) because devoted (\emph{bhatta}) people show devotion (\emph{bhatti}) to him on account of his attainments. He has rejected blessings (\emph{bhage vami}) such as glory, lordship, fame and so on. He has rejected the parts (\emph{bhāge vami}) such as the five aggregates of experience, and so on (\textbf{\cite{Vism-mhṭ}241–246}).

                            As to the word “\emph{bhattavā}”: at \hyperlink{VII.63}{VII.63}{}, it is explained as “one who has frequented (\emph{bhaji}) attainments.” In this sense the attainments have been “frequented” (\emph{bhatta}) by him \textbf{\cite{Vism-mhṭ}} (214 f.). uses the same word in another sense as “possessor of devotees,” expanding it as \emph{bhattā daḷhabhattikā assa bahu atthi} (“he has many devoted firm devotees”—Skr. \emph{bhakta}). In PED under \emph{bhattavant} (citing also Vism 212) only the second meaning is given. \emph{Bhatta} is from the same root (\emph{bhaj}) in both cases.

                            For a short exposition of this recollection see commentary to AN 1:16.1.}

                    \vismParagraph{VII.66}{66}{}
                    \marginnote{\textcolor{teal}{\footnotesize\{267|209\}}}{}So when he has thus suppressed the hindrances by preventing obsession by greed, etc., and his mind faces the meditation subject with rectitude, then his applied thought and sustained thought occur with a tendency toward the Enlightened One’s special qualities. As he continues to exercise applied thought and sustained thought upon the Enlightened One’s special qualities, happiness arises in him. With his mind happy, with happiness as a proximate cause, his bodily and mental disturbances are tranquilized by tranquillity. When the disturbances have been tranquilized, bodily and mental bliss arise in him. When he is blissful, his mind, with the Enlightened One’s special qualities for its object, becomes concentrated, and so the jhāna factors eventually arise in a single moment. But owing to the profundity of the Enlightened One’s special qualities, or else owing to his being occupied in recollecting special qualities of many sorts, the jhāna is only access and does not reach absorption. And that access jhāna itself is known as “recollection of the Buddha” too, because it arises with the recollection of the Enlightened One’s special qualities as the means.

                    \vismParagraph{VII.67}{67}{}
                    When a bhikkhu is devoted to this recollection of the Buddha, he is respectful and deferential towards the Master. He attains fullness of faith, mindfulness, understanding and merit. He has much happiness and gladness. He conquers fear and dread. \textcolor{brown}{\textit{[213]}} He is able to endure pain. He comes to feel as if he were living in the Master’s presence. And his body, when the recollection of the Buddha’s special qualities dwells in it, becomes as worthy of veneration as a shrine room. His mind tends toward the plane of the Buddhas. When he encounters an opportunity for transgression, he has awareness of conscience and shame as vivid as though he were face to face with the Master. And if he penetrates no higher, he is at least headed for a happy destiny.
                    \begin{verse}
                        Now, when a man is truly wise,\\{}
                        His constant task will surely be\\{}
                        This recollection of the \emph{Buddha}\\{}
                        Blessed with such mighty potency.
                    \end{verse}


                    This, firstly, is the section dealing with the recollection of the Enlightened One in the detailed explanation.
            \section[\vismAlignedParas{§68–88}(2) Recollection of the Dhamma]{(2) Recollection of the Dhamma}

                \vismParagraph{VII.68}{68}{}
                One who wants to develop the recollection of the Dhamma (Law) should go into solitary retreat and recollect the special qualities of both the Dhamma (Law) of the scriptures and the ninefold supramundane Dhamma (state) as follows: \marginnote{\textcolor{teal}{\footnotesize\{268|210\}}}{}“The Dhamma is well proclaimed by the Blessed One, visible here and now, not delayed (timeless), inviting of inspection, onward-leading, and directly experienceable by the wise” (\textbf{\cite{M}I 37}; \textbf{\cite{A}III 285}).
                \subsection[\vismAlignedParas{§69–75}Well Proclaimed]{Well Proclaimed}

                    \vismParagraph{VII.69}{69}{}
                    \emph{Well proclaimed}: in this clause the Dhamma of the scriptures is included as well as the other; in the rest of the clauses only the supramundane Dhamma is included.

                    Herein, the Dhamma of the scriptures is well proclaimed because it is good in the beginning, the middle, and the end, and because it announces the life of purity that is utterly perfect and pure with meaning and with detail (see \textbf{\cite{M}I 179}).

                    Even a single stanza of the Blessed One’s teaching is good in the beginning with the first word, good in the middle with the second, third, etc., and good in the end with the last word, because the Dhamma is altogether admirable. A sutta with a single sequence of meaning\footnote{\vismAssertFootnoteCounter{30}\vismHypertarget{VII.n30}{}\emph{Anusandhi—}“sequence of meaning”: a technical commentarial term signifying both a particular subject treated in a discourse, and also the way of linking one subject with another in the same discourse. At \textbf{\cite{M-a}I 175} three kinds are distinguished: sequence of meaning in answer to a question (\emph{pucchānusandhi—}e.g. \textbf{\cite{M}I 36}), that to suit a personal idiosyncrasy, (\emph{ajjhāsayānusandhi—}e.g. \textbf{\cite{M}I 23}) and that due to the natural course of the teaching (\emph{yathānusandhi—}e.g. the whole development of MN 6).} is good in the beginning with the introduction, good in the end with the conclusion, and good in the middle with what is in between. A sutta with several sequences of meaning is good in the beginning with the first sequence of meaning, good in the end with the last sequence of meaning, and good in the middle with the sequences of meaning in between. Furthermore, it is good in the beginning with the introduction [giving the place of] and the origin [giving the reason for] its utterance. It is good in the middle because it suits those susceptible of being taught since it is unequivocal in meaning and reasoned with cause and example. It is good in the end with its conclusion that inspires faith in the hearers.

                    \vismParagraph{VII.70}{70}{}
                    Also the entire Dhamma of the Dispensation is good in the beginning with virtue as one’s own well-being. It is good in the middle with serenity and insight and with path and fruition. It is good in the end with Nibbāna. Or alternatively, it is good in the beginning with virtue and concentration. \textcolor{brown}{\textit{[214]}} It is good in the middle with insight and the path. It is good in the end with fruition and Nibbāna. Or alternatively, it is good in the beginning because it is the good discovery made by the Buddha. It is good in the middle because it is the well-regulatedness of the Dhamma. It is good in the end because it is the good way entered upon by the Saṅgha. Or alternatively, it is good in the beginning as the discovery of what can be attained by one who enters upon the way of practice in conformity after hearing about it. It is good in the middle as the unproclaimed enlightenment [of Paccekabuddhas]. It is good in the end as the enlightenment of disciples.

                    \vismParagraph{VII.71}{71}{}
                    And when listened to, it does good through hearing it because it suppresses the hindrances, thus it is good in the beginning. And when made the way of \marginnote{\textcolor{teal}{\footnotesize\{269|211\}}}{}practice it does good through the way being entered upon because it brings the bliss of serenity and insight, thus it is good in the middle. And when it has thus been made the way of practice and the fruit of the way is ready, it does good through the fruit of the way because it brings [unshakable] equipoise, thus it is good in the end.

                    So it is “well proclaimed” because of being good in the beginning, the middle and the end.

                    \vismParagraph{VII.72}{72}{}
                    Now, the life of purity, that is to say, the life of purity of the Dispensation and the life of purity of the path, which the Blessed One announces, which he shows in various ways when he teaches the Dhamma, is “with meaning” because of perfection of meaning, and it is “with detail” because of perfection of detail, as it is proper that it should be. It is “with meaning” because it conforms to the words declaring its meaning by pronouncing, clarifying, revealing, expounding, and explaining it. It is “with detail” because it has perfection of syllables, words, details, style, language, and descriptions. It is “with meaning” owing to profundity of meaning and profundity of penetration. It is “with detail” owing to profundity of law and profundity of teaching. It is “with meaning” because it is the province of the discriminations of meaning and of perspicuity. It is “with detail” because it is the province of the discriminations of law and of language (see \hyperlink{XIV.21}{XIV.21}{}). It is “with meaning” since it inspires confidence in persons of discretion, being experienceable by the wise. It is “with detail” since it inspires confidence in worldly persons, being a fit object of faith. It is “with meaning” because its intention is profound. It is “with detail” because its words are clear. It is “utterly perfect” with the complete perfection due to absence of anything that can be added. It is “pure” with the immaculateness due to absence of anything to be subtracted.

                    \vismParagraph{VII.73}{73}{}
                    Furthermore, it is “with meaning” because it provides the particular distinction\footnote{\vismAssertFootnoteCounter{31}\vismHypertarget{VII.n31}{}\emph{Vyatti }(\emph{byatti})—“particular distinction” (n. fm. \emph{vi + añj}); not so spelt in PED but see \emph{viyatti}. Glossed by \textbf{\cite{Vism-mhṭ}} with \emph{veyyatti}.} of achievement through practice of the way, and it is “with detail” because it provides the particular distinction of learning through mastery of scripture. It is “utterly perfect” because it is connected with the five aggregates of Dhamma beginning with virtue.\footnote{\vismAssertFootnoteCounter{32}\vismHypertarget{VII.n32}{}These “five aggregates” are those of virtue, concentration, understanding, deliverance, and knowledge and vision of deliverance.} It is “pure” because it has no imperfection, because it exists for the purpose of crossing over [the round of rebirths’ flood (see \textbf{\cite{M}I 134})], and because it is not concerned with worldly things.

                    So it is “well proclaimed” because it “announces the life of purity that is utterly perfect and pure with meaning and with detail.”

                    Or alternatively, it is \emph{well proclaimed} since it has been properly proclaimed with no perversion of meaning. For the meaning of other sectarians’ law suffers perversion since there is actually no obstruction in the \textcolor{brown}{\textit{[215]}} things described there as obstructive and actually no outlet in the things described there as outlets, \marginnote{\textcolor{teal}{\footnotesize\{270|212\}}}{}which is why their law is ill-proclaimed; but not so the Blessed One’s Law, whose meaning suffers no perversion since the things described there as obstructions and the things described there as outlets are so in actual fact.

                    So, in the first place, the Dhamma of the scriptures is “well proclaimed.”

                    \vismParagraph{VII.74}{74}{}
                    The supramundane Dhamma is \emph{well proclaimed} since both the way that accords with Nibbāna and the Nibbāna that accords with the way have been proclaimed, according as it is said: “The way leading to Nibbāna has been properly declared to the disciples by the Blessed One, and Nibbāna and the way meet. Just as the water of the Ganges meets and joins with the water of the Yamunā, so too the way leading to Nibbāna has been properly declared to the disciples by the Blessed One, and Nibbāna and the way meet” (\textbf{\cite{D}II 223}).

                    \vismParagraph{VII.75}{75}{}
                    And here the noble path, which is the middle way since it does not approach either extreme, is \emph{well proclaimed} in being proclaimed to be the middle way.

                    The fruits of asceticism, where defilements are tranquilized, are \emph{well proclaimed too in being proclaimed to have tranquilized defilement.}

                    Nibbāna, whose individual essence is eternal, deathless, the refuge, the shelter, etc., is \emph{well proclaimed} too in being proclaimed to have an individual essence that is eternal, and so on.

                    So the supramundane Dhamma is also “well proclaimed.”
                \subsection[\vismAlignedParas{§76–79}Visible Here and Now]{Visible Here and Now}

                    \vismParagraph{VII.76}{76}{}
                    \emph{Visible here and now}: firstly, the noble path is “visible here and now” since it can be seen by a noble person himself when he has done away with greed, etc., in his own continuity, according as it is said: “When a man is dyed with greed, brahman, and is overwhelmed and his mind is obsessed by greed, then he thinks for his own affliction, he thinks for others’ affliction, he thinks for the affliction of both, and he experiences mental suffering and grief. When greed has been abandoned, he neither thinks for his own affliction, nor thinks for others’ affliction, nor thinks for the affliction of both, and he does not experience mental suffering and grief. This, brahman, is how the Dhamma is visible here and now” (\textbf{\cite{A}I 156}). \textcolor{brown}{\textit{[216]}}

                    \vismParagraph{VII.77}{77}{}
                    Furthermore, the ninefold supramundane Dhamma is also \emph{visible here and now}, since when anyone has attained it, it is visible to him through reviewing knowledge without his having to rely on faith in another.

                    \vismParagraph{VII.78}{78}{}
                    Or alternatively, the view (\emph{diṭṭhi}) that is recommended (\emph{pasattha—}pp. of root \emph{saṃs}) is “proper view” (\emph{sandiṭṭhi}). It conquers by means of proper view, thus it “has proper view” (\emph{sandiṭṭhika—}“visible here and now”). For in this way the noble path conquers defilements by means of the proper view associated with it, and the noble fruition does so by means of the proper view that is its cause, and Nibbāna does so by means of the proper view that has Nibbāna as its objective field. So the ninefold supramundane Dhamma “has the proper view” (\emph{sandiṭṭhika—}“is visible here and now”) since it conquers by means of proper view, just as a charioteer (\emph{rathika}) is so called because he conquers by means of a chariot (\emph{ratha}).

                    \vismParagraph{VII.79}{79}{}
                    \marginnote{\textcolor{teal}{\footnotesize\{271|213\}}}{}Or alternatively, it is seeing (\emph{dassana}) that is called “the seen” (\emph{diṭṭha}); then \emph{diṭṭha} and \emph{sandiṭṭha} are identical in meaning as “seeing.” It is worthy of being seen (\emph{diṭṭha}), thus it is “visible here and now” (\emph{sandiṭṭhika}). For the supramundane Dhamma (law) arrests the fearful round [of kamma, etc.,] as soon as it is seen by means of penetration consisting in development [of the path] and by means of penetration consisting in realization [of Nibbāna]. So it is “visible here and now” (\emph{sandiṭṭhika}) since it is worthy of being seen (\emph{diṭṭha}), just as one who is clothable (\emph{vattihika})\footnote{\vismAssertFootnoteCounter{33}\vismHypertarget{VII.n33}{}\emph{Vatthika—}“clothable”; not in PED.} is so called because he is worthy of clothes (\emph{vattha}).
                \subsection[\vismAlignedParas{§80–81}Not Delayed]{Not Delayed}

                    \vismParagraph{VII.80}{80}{}
                    It has no delay (lit. “takes no time”—\emph{kāla}) in the matter of giving its own fruit, thus it is “without delay” (\emph{akāla}). “Without delay” is the same as “not delayed” (\emph{akālika}). What is meant is that instead of giving its fruit after creating a delay (using up time), say, five days, seven days, it gives its fruit immediately next to its own occurrence (see \textbf{\cite{Sn}226}).

                    \vismParagraph{VII.81}{81}{}
                    Or alternatively, what is delayed (\emph{kālika—}lit. “what takes time”) is what needs some distant\footnote{\vismAssertFootnoteCounter{34}\vismHypertarget{VII.n34}{}\emph{Pakaṭṭha—}“distant”; not in PED (= \emph{dura—}\textbf{\cite{Vism-mhṭ}297}).} time to be reached before it can give its fruit. What is that? It is the mundane law of profitable [kamma]. This, however, is undelayed (\emph{na kālika}) because its fruit comes immediately next to it, so it is “not delayed” (\emph{akālika}).

                    This is said with reference to the path.
                \subsection[\vismAlignedParas{§82}Inviting of Inspection]{Inviting of Inspection}

                    \vismParagraph{VII.82}{82}{}
                    It is worthy of an invitation to inspect (\emph{ehipassa-vidhi}) given thus: “Come and see this Dhamma” (\emph{ehi passa imaṃ dhammaṃ}), thus it is “inviting of inspection” (\emph{ehipassika}). But why is it worthy of this invitation? Because it is found and because of its purity. For if a man has said that there is money or gold in an empty fist, he cannot say, “Come and see it.” Why not? Because it is not found. And on the other hand, while dung or urine may well be found, a man cannot, for the purpose of cheering the mind by exhibiting beauty, say, “Come \marginnote{\textcolor{teal}{\footnotesize\{272|214\}}}{}and see this;” on the contrary, they have to be covered up with grass and leaves. Why? Because of their impurity. But this ninefold supramundane Dhamma is actually found as such in its individual essence, and it is as pure as the full moon’s disk in a cloudless sky, as a gem of pure water on bleached cloth. \textcolor{brown}{\textit{[217]}} Consequently, it is worthy of the invitation to inspect since it is found and pure, thus it is “inviting of inspection.”
                \subsection[\vismAlignedParas{§83–84}Onward-Leading]{Onward-Leading}

                    \vismParagraph{VII.83}{83}{}
                    The word \emph{opanayika} (“onward-leading”) is [equivalent to the gerund] \emph{upanetabba} (“ought to—can—be induced”). Here is an exposition. An inducing (\emph{upanayana}) is an inducement (\emph{upanaya}). [As the four paths and four fruitions] this [Dhamma] is worth inducing (\emph{upanayanaṃ arahati}) [that is, arousing] \emph{in }one’s own mind [subjectively] by means of development, without any question of whether or not one’s clothing or one’s head is on fire (see \textbf{\cite{A}IV 320}), thus it is “onward-leading” (\emph{opanayika}). This applies to the [above-mentioned eight] formed supramundane states (dhammas). But the unformed [dhamma] is worth inducing \emph{by} one’s own mind [to become the mind’s object], thus it is “onward-leading,” too; the meaning is that it is worth treating as one’s shelter by realizing it.

                    \vismParagraph{VII.84}{84}{}
                    Or alternatively, what induces (\emph{upaneti}) [the noble person] onwards to Nibbāna is the noble path, which is thus inductive (\emph{upaneyya}). Again, what can (ought to) be induced (\emph{upanetabba}) to realizability is the Dhamma consisting in fruition and Nibbāna, which is thus inductive (\emph{upaneyya}), too. The word \emph{upaneyya }is the same as the word \emph{opanayika}.\footnote{\vismAssertFootnoteCounter{35}\vismHypertarget{VII.n35}{}This passage is only loosely renderable because the exegesis here is based almost entirely on the substitution of one Pali grammatical form for another (\emph{padasiddhi}). The reading \emph{opaneyyiko} (for \emph{opanayiko}) does not appear in any Sinhalese text (generally the most reliable); consequently the sentence “\emph{opanayiko va opaneyyiko}” (see Harvard text) is absent in them, being superfluous. \textbf{\cite{Vism-mhṭ}}’s explanations are incorporated. This paragraph depends on the double sense of \emph{upaneti} (\emph{upa + neti}, to lead on or induce) and its derivatives as (i) an attractive inducement and (ii) a reliable guide, and so the word \emph{induce} is stretched a bit and \emph{inducive} coined on the analogy of conducive. \emph{Upanaya} (inducement) is not in PED, nor is \emph{upanayana} (inducing) in this sense (see also \hyperlink{XIV.68}{XIV.68}{}). \emph{Upanayana} means in logic “application,” “subsumption”; and also \emph{upanetabba }means “to be added”; see end of §72. For \emph{allīyana} (“treating as one’s shelter”) see references in Glossary.}
                \subsection[\vismAlignedParas{§85–88}Is Directly Experienceable by the Wise]{Is Directly Experienceable by the Wise}

                    \vismParagraph{VII.85}{85}{}
                    \emph{Is directly experienceable by the wise}: it can be experienced by all the kinds of wise men beginning with the “acutely wise” (see \textbf{\cite{A}II 135}) each in himself thus: “The path has been developed, fruition attained, and cessation realized, by me.” For it does not happen that when a preceptor has developed the path his co-resident abandons his defilements, nor does a co-resident dwell in comfort owing to the preceptor’s attainment of fruition, nor does he realize the Nibbāna realized by the preceptor. So this is not visible in the way that an ornament on another’s head is, but rather it is visible only in one’s own mind. What is meant is that it can be undergone by wise men, but it is not the province of fools.

                    \vismParagraph{VII.86}{86}{}
                    Now, in addition, this Dhamma is well proclaimed. Why? Because it is visible here and now. It is visible here and now because it is not delayed. It is not delayed because it invites inspection. And what invites inspection is onward-leading.

                    \vismParagraph{VII.87}{87}{}
                    As long as [the meditator] recollects the special qualities of the Dhamma in this way, then: “On that occasion his mind is not obsessed by greed, or obsessed by hate, or obsessed by delusion; his mind has rectitude on that occasion, being inspired by the Dhamma” (\textbf{\cite{A}III 285}).

                    So when he has suppressed the hindrances in the way already described (\hyperlink{VII.66}{§66}{}), the jhāna factors arise in a single conscious moment. But owing to the \marginnote{\textcolor{teal}{\footnotesize\{273|215\}}}{}profundity of the Dhamma’s special qualities, or else owing to his being occupied in recollecting special qualities of many sorts, the jhāna is only access and does not reach absorption. And that access jhāna itself is known as “recollection of the Dhamma” too because it arises with the recollection of the Dhamma’s special qualities as the means.

                    \vismParagraph{VII.88}{88}{}
                    \textcolor{brown}{\textit{[218]}} When a bhikkhu is devoted to this recollection of the Dhamma, he thinks: “I never in the past met a master who taught a law that led onward thus, who possessed this talent, nor do I now see any such a master other than the Blessed One.” Seeing the Dhamma’s special qualities in this way, he is respectful and deferential towards the Master. He entertains great reverence for the Dhamma and attains fullness of faith, and so on. He has much happiness and gladness. He conquers fear and dread. He is able to endure pain. He comes to feel as if he were living in the Dhamma’s presence. And his body, when the recollection of the Dhamma’s special qualities dwells in it, becomes as worthy of veneration as a shrine room. His mind tends towards the realization of the peerless Dhamma. When he encounters an opportunity for transgression, he has vivid awareness of conscience and shame on recollecting the well-regulatedness of the Dhamma. And if he penetrates no higher, he is at least headed for a happy destiny.
                    \begin{verse}
                        Now, when a man is truly wise,\\{}
                        His constant task will surely be\\{}
                        This recollection of the \emph{Dhamma}\\{}
                        Blessed with such mighty potency.
                    \end{verse}


                    This is the section dealing with the recollection of the Dhamma in the detailed explanation.
            \section[\vismAlignedParas{§89–100}(3) Recollection of the Saṅgha]{(3) Recollection of the Saṅgha}

                \vismParagraph{VII.89}{89}{}
                One who wants to develop the recollection of the Community should go into solitary retreat and recollect the special qualities of the community of Noble Ones as follows:

                “The community of the Blessed One’s disciples has entered on the good way, the community of the Blessed One’s disciples has entered on the straight way, the community of the Blessed One’s disciples has entered on the true way, the community of the Blessed One’s disciples has entered on the proper way, that is to say, the four pairs of men, the eight persons; this community of the Blessed One’s disciples is fit for gifts, fit for hospitality, fit for offerings, fit for reverential salutation, as an incomparable field of merit for the world” (\textbf{\cite{A}III 286}).
                \subsection[\vismAlignedParas{§90–93}Entered on the Good, Straight, True, Proper Way]{Entered on the Good, Straight, True, Proper Way}

                    \vismParagraph{VII.90}{90}{}
                    Herein, \emph{entered on the good way }(\emph{supaṭipanna}) is thoroughly entered on the way (\emph{suṭṭhu paṭipanna}). What is meant is that it has entered on a way (\emph{paṭipanna}) that is the right way (\emph{sammā-paṭipadā}), the way that is irreversible, the way that is in conformity [with truth], the way that has no opposition, the way that is regulated by the Dhamma. They hear (\emph{suṇanti}) attentively the Blessed One’s instruction, thus they are his disciples (\emph{sāvaka—}lit. “hearers”). \emph{The community of }\marginnote{\textcolor{teal}{\footnotesize\{274|216\}}}{}\emph{the disciples} is the community of those disciples. The meaning is that the total of disciples forms a communality because it possesses in common both virtue and [right] view. \textcolor{brown}{\textit{[219]}} That right way, being \emph{straight}, unbent, uncrooked, unwarped, is called noble and \emph{true} and is known as \emph{proper} owing to its becomingness, therefore the noble community that has entered on that is also said to have \emph{entered on the straight way, entered on the true way}, and \emph{entered on the proper way}.

                    \vismParagraph{VII.91}{91}{}
                    Those who stand on the path can be understood to have \emph{entered on the good way} since they possess the right way. And those who stand in fruition can be understood to have \emph{entered on the good way} with respect to the way that is now past since by means of the right way they have realized what should be realized.

                    \vismParagraph{VII.92}{92}{}
                    Furthermore, the Community \emph{has entered on the good way} because it has entered on the way according as instructed in the well-proclaimed Dhamma and Discipline (\emph{dhamma-vinaya}), and because it has entered on the immaculate way. It \emph{has entered on the straight way} because it has entered on the way avoiding the two extremes and taking the middle course, and because it has entered on the way of the abandonment of the faults of bodily and verbal crookedness, tortuousness and warpedness. It \emph{has entered on the true way} because Nibbāna is what is called “true” and it has entered on the way with that as its aim. It \emph{has entered on the proper way} because it has entered on the way of those who are worthy of proper acts [of veneration].

                    \vismParagraph{VII.93}{93}{}
                    The word \emph{yadidaṃ} (“that is to say”) = \emph{yāni imāni. The four pairs of men}: taking them pairwise, the one who stands on the first path and the one who stands in the first fruition as one pair, in this way there are four pairs. \emph{The eight persons}: taking them by persons, the one who stands on the first path as one and the one who stands in the first fruition as one, in this way there are eight persons. And there in the compound \emph{purisa-puggala} (persons) the words \emph{purisa} and \emph{puggala }have the same meaning, but it is expressed in this way to suit differing susceptibility to teaching.

                    \emph{This community of the Blessed One’s disciples}: this community of the Blessed One’s disciples taken by pairs as the four pairs of men (\emph{purisa}) and individually as the eight persons (\emph{purisa-puggala}).
                \subsection[\vismAlignedParas{§94–95}Fit for Gifts]{Fit for Gifts}

                    \vismParagraph{VII.94}{94}{}
                    As to \emph{fit for gifts}, etc.: what should be brought (\emph{ānetvā}) and given (\emph{hunitabba}) is a gift (\emph{āhuna—}lit. “sacrifice”); the meaning is, what is to be brought even from far away and donated to the virtuous. It is a term for the four requisites. The Community is fit to receive that gift (sacrifice) because it makes it bear great fruit, thus it is “fit for gifts” (\emph{āhuneyya}).

                    \vismParagraph{VII.95}{95}{}
                    Or alternatively, all kinds of property, even when the bringer comes (\emph{āgantvā}) from far away, can be given (\emph{hunitabba}) here, thus the Community “can be given to” (\emph{āhavanīya}); or it is fit to be given to by Sakka and others, thus it “can be given to.” And the brahmans’ fire is called “to be given (sacrificed) to” (\emph{āhavanīya}), for they believe that what is sacrificed to it brings great fruit. \textcolor{brown}{\textit{[220]}} But if something is to be sacrificed to for the sake of the great fruit brought by what is sacrificed to \marginnote{\textcolor{teal}{\footnotesize\{275|217\}}}{}it, then surely the Community should be sacrificed to; for what is sacrificed (given) to the Community has great fruit, according as it is said:
                    \begin{verse}
                        “Were anyone to serve the fire\\{}
                        Out in the woods a hundred years,\\{}
                        And pay one moment’s homage too\\{}
                        To men of self-development,\\{}
                        His homage would by far excel\\{}
                        His hundred years of sacrifice” (\textbf{\cite{Dhp}107}).
                    \end{verse}


                    And the words \emph{āhavanīya} (“to be sacrificed to”), which is used in the schools,\footnote{\vismAssertFootnoteCounter{36}\vismHypertarget{VII.n36}{}“In the Sarvāstivādin school and so on” (\textbf{\cite{Vism-mhṭ}230}).} is the same in meaning as this word \emph{āhuneyya} (“fit for gifts”) used here. There is only the mere trifling difference of syllables. So it is “fit for gifts.”
                \subsection[\vismAlignedParas{§96}Fit for Hospitality]{Fit for Hospitality}

                    \vismParagraph{VII.96}{96}{}
                    \emph{Fit for hospitality} (\emph{pāhuneyya}): “hospitality” (\emph{pāhuna}) is what a donation to visitors is called, prepared with all honours for the sake of dear and beloved relatives and friends who have come from all quarters. But even more than to such objects of hospitality, it is fitting that it should be given also to the Community; for there is no object of hospitality so fit to receive hospitality as the Community since it is encountered after an interval between Buddhas and possesses wholly endearing and lovable qualities. So it is “fit for hospitality” since the hospitality is fit to be given to it and it is fit to receive it.

                    But those who take the text to be \emph{pāhavanīya} (“fit to be given hospitality to”) have it that the Community is worthy to be placed first and so what is to be given should first of all be brought here and given (\emph{sabba-Paṭhamaṃ Ānetvā ettha HUNitabbaṃ}), and for that reason it is “fit to be given hospitality to” (\emph{pāhavanīya}) or since it is worthy to be given to in all aspects (\emph{sabba-Pakārena ĀHAVANAṃ arahati}), it is thus “fit to be given hospitality to” (\emph{pāhavanīya}). And here this is called \emph{pāhuneyya} in the same sense.
                \subsection[\vismAlignedParas{§97}Fit for Offering]{Fit for Offering}

                    \vismParagraph{VII.97}{97}{}
                    ”Offering” (\emph{dakkhiṇa}) is what a gift is called that is to be given out of faith in the world to come. The Community is worthy of that offering, or it is helpful to that offering because it purifies it by making it of great fruit, thus it is \emph{fit for offerings} (\emph{dakkhiṇeyya}).
                \subsection[\vismAlignedParas{§97}Fit for Salutation]{Fit for Salutation}

                    It is worthy of being accorded by the whole world the reverential salutation (\emph{añjali-kamma}) consisting in placing both hands [palms together] above the head, thus it is \emph{fit for reverential salutation} (\emph{añjalikaraṇīya}).
                \subsection[\vismAlignedParas{§98–100}As an Incomparable Field of Merit for the World]{As an Incomparable Field of Merit for the World}

                    \vismParagraph{VII.98}{98}{}
                    \emph{As an incomparable field of merit for the world}: as a place without equal in the world for growing merit; just as the place for growing the king’s or minister’s \marginnote{\textcolor{teal}{\footnotesize\{276|218\}}}{}rice or corn is the king’s rice-field or the king’s corn-field, so the Community is the place for growing the whole world’s merit. For the world’s various kinds of merit leading to welfare and happiness grow with the Community as their support. Therefore the Community is “an incomparable field of merit for the world.”

                    \vismParagraph{VII.99}{99}{}
                    As long as he recollects the special qualities of the Saṅgha in this way, classed as “having entered on the good way,” etc., \textcolor{brown}{\textit{[221]}} then: “On that occasion his mind is not obsessed by greed, or obsessed by hate, or obsessed by delusion; his mind has rectitude on that occasion, being inspired by the Saṅgha” (\textbf{\cite{A}III 286}).

                    So when he has suppressed the hindrances in the way already described (\hyperlink{VII.66}{§66}{}), the jhāna factors arise in a single conscious moment. But owing to the profundity of the Community’s special qualities, or else owing to his being occupied in recollecting special qualities of many sorts, the jhāna is only access and does not reach absorption. And that access jhāna itself is known as “recollection of the Saṅgha” too because it arises with the recollection of the Community’s special qualities as the means.

                    \vismParagraph{VII.100}{100}{}
                    When a bhikkhu is devoted to this recollection of the Community, he is respectful and deferential towards the Community. He attains fullness of faith, and so on. He has much happiness and bliss. He conquers fear and dread. He is able to endure pain. He comes to feel as if he were living in the Community’s presence. And his body, when the recollection of the Sangha’s special qualities dwells in it, becomes as worthy of veneration as an Uposatha house where the Community has met. His mind tends towards the attainment of the Community’s special qualities. When he encounters an opportunity for transgression, he has awareness of conscience and shame as vividly as if he were face to face with the Community. And if he penetrates no higher, he is at least headed for a happy destiny.
                    \begin{verse}
                        Now, when a man is truly wise,\\{}
                        His constant task will surely be\\{}
                        This recollection of the \emph{Saṅgha}\\{}
                        Blessed with such mighty potency.
                    \end{verse}


                    This is the section dealing with the recollection of the Community in the detailed explanation.
            \section[\vismAlignedParas{§101–106}(4) Recollection of Virtue]{(4) Recollection of Virtue}

                \vismParagraph{VII.101}{101}{}
                One who wants to develop the recollection of virtue should go into solitary retreat and recollect his own different kinds of virtue in their special qualities of being untorn, etc., as follows:

                Indeed, my various kinds of virtue are “untorn, unrent, unblotched, unmottled, liberating, praised by the wise, not adhered to, and conducive to concentration” (\textbf{\cite{A}III 286}). And a layman should recollect them in the form of laymen’s virtue while one gone forth into homelessness should recollect them in the form of the virtue of those gone forth.

                \vismParagraph{VII.102}{102}{}
                \marginnote{\textcolor{teal}{\footnotesize\{277|219\}}}{}Whether they are the virtues of laymen or of those gone forth, when no one of them is broken in the beginning or in the end, not being torn like a cloth ragged at the ends, then they are \emph{untorn}. \textcolor{brown}{\textit{[222]}} When no one of them is broken in the middle, not being rent like a cloth that is punctured in the middle, then they are \emph{unrent}. When they are not broken twice or thrice in succession, not being blotched like a cow whose body is some such colour as black or red with discrepant-coloured oblong or round patch appearing on her back or belly, then they are \emph{unblotched}. When they are not broken all over at intervals, not being mottled like a cow speckled with discrepant-coloured spots, then they are \emph{unmottled}.

                \vismParagraph{VII.103}{103}{}
                Or in general they are \emph{untorn, unrent, unblotched, unmottled} when they are undamaged by the seven bonds of sexuality (\hyperlink{I.144}{I.144}{}) and by anger and enmity and the other evil things (see \hyperlink{VII.59}{§59}{}).

                \vismParagraph{VII.104}{104}{}
                Those same virtues are \emph{liberating} since they liberate by freeing from the slavery of craving. They are \emph{praised by the wise} because they are praised by such wise men as Enlightened Ones. They are \emph{not adhered to} (\emph{aparāmaṭṭha}) since they are not adhered to (\emph{aparāmaṭṭhattā}) with craving and [false] view, or because of the impossibility of misapprehending (\emph{parāmaṭṭhuṃ}) that “There is this flaw in your virtues.” They are \emph{conducive to concentration} since they conduce to access concentration and absorption concentration, or to path concentration and fruition concentration.

                \vismParagraph{VII.105}{105}{}
                As long as he recollects his own virtues in their special qualities of being untorn, etc., in this way, then: “On that occasion his mind is not obsessed by greed, or obsessed by hate, or obsessed by delusion, his mind has rectitude on that occasion, being inspired by virtue” (\textbf{\cite{A}III 286}).

                So when he has suppressed the hindrances in the way already described (\hyperlink{VII.66}{§66}{}), the jhāna factors arise in a single conscious moment. But owing to the profundity of the virtues’ special qualities, or owing to his being occupied in recollecting special qualities of many sorts, the jhāna is only access and does not reach absorption. And that access jhāna itself is known as “recollection of virtue” too because it arises with the virtues’ special qualities as the means.

                \vismParagraph{VII.106}{106}{}
                And when a bhikkhu is devoted to this recollection of virtue, he has respect for the training. He lives in communion [with his fellows in the life of purity]. He is sedulous in welcoming. He is devoid of the fear of self-reproach and so on. He sees fear in the slightest fault. He attains fullness of faith, and so on. He has much happiness and gladness. And if he penetrates no higher, he is at least headed for a happy destiny.
                \begin{verse}
                    Now, when a man is truly wise,\\{}
                    His constant task will surely be\\{}
                    This recollection of his virtue\\{}
                    Blessed with such mighty potency.
                \end{verse}


                This is the section dealing with the recollection of virtue in the detailed explanation.\textcolor{brown}{\textit{[223]}}
            \section[\vismAlignedParas{§107–114}(5) Recollection of Generosity]{(5) Recollection of Generosity}

                \vismParagraph{VII.107}{107}{}
                \marginnote{\textcolor{teal}{\footnotesize\{278|220\}}}{}One who wants to develop the recollection of generosity should be naturally devoted to generosity and the constant practice of giving and sharing. Or alternatively, if he is one who is starting the development of it, he should make the resolution: “From now on, when there is anyone present to receive, I shall not eat even a single mouthful without having given a gift.” And that very day he should give a gift by sharing according to his means and his ability with those who have distinguished qualities. When he has apprehended the sign in that, he should go into solitary retreat and recollect his own generosity in its special qualities of being free from the stain of avarice, etc., as follows:

                “It is gain for me, it is great gain for me, that in a generation obsessed by the stain of avarice I abide with my heart free from stain by avarice, and am freely generous and open-handed, that I delight in relinquishing, expect to be asked, and rejoice in giving and sharing” (\textbf{\cite{A}III 287}).

                \vismParagraph{VII.108}{108}{}
                Herein, \emph{it is gain for me}: it is my gain, advantage. The intention is: I surely partake of those kinds of gain for a giver that have been commended by the Blessed One as follows: “A man who gives life [by giving food] shall have life either divine or human” (\textbf{\cite{A}III 42}), and: “A giver is loved and frequented by many” (\textbf{\cite{A}III 40}), and: “One who gives is ever loved, according to the wise man’s law” (\textbf{\cite{A}III 41}), and so on.

                \vismParagraph{VII.109}{109}{}
                \emph{It is great gain for me}: it is great gain for me that this Dispensation, or the human state, has been gained by me. Why? Because of the fact that “I abide \emph{with my mind free from stain by avarice … and rejoice in giving and sharing}.”

                \vismParagraph{VII.110}{110}{}
                Herein, \emph{obsessed by the stain of avarice} is overwhelmed by the stain of avarice. \emph{Generation}: beings, so called owing to the fact of their being generated. So the meaning here is this: among beings who are overwhelmed by the stain of avarice, which is one of the dark states that corrupt the [natural] transparency of consciousness (see \textbf{\cite{A}I 10}) and which has the characteristic of inability to bear sharing one’s own good fortune with others.

                \vismParagraph{VII.111}{111}{}
                \emph{Free from stain by avarice} because of being both free from avarice and from the other stains, greed, hate, and the rest. \emph{I abide with my heart}: I abide with my consciousness of the kind already stated, is the meaning. \textcolor{brown}{\textit{[224]}} But in the sutta, “I live the home life with my heart free” (\textbf{\cite{A}III 287}; V 331), etc., is said because it was taught there as a [mental] abiding to depend on [constantly] to Mahānāma the Sakyan, who was a stream-enterer asking about an abiding to depend on. There the meaning is “I live overcoming …”

                \vismParagraph{VII.112}{112}{}
                \emph{Freely generous}: liberally generous. \emph{Open-handed}: with hands that are purified. What is meant is: with hands that are always washed in order to give gifts carefully with one’s own hands. \emph{That I delight in relinquishing}: the act of relinquishing (\emph{vossajjana}) is relinquishing (\emph{vossagga}); the meaning is, giving up. To delight in relinquishing is to delight in constant devotion to that relinquishing. \emph{Expect to be asked} (\emph{yācayoga}): accustomed to being asked (\emph{yācana-yogga}) because of giving whatever others ask for, is the meaning. \emph{Yājayoga} is a reading, in which case the meaning is: devoted (\emph{yutta}) to sacrifice (\emph{yāja}), in other words, to \marginnote{\textcolor{teal}{\footnotesize\{279|221\}}}{}sacrificing (\emph{yajana}). \emph{And rejoice in sharing}: the meaning is, he recollects thus: “I give gifts and I share out what is to be used by myself, and I rejoice in both.”

                \vismParagraph{VII.113}{113}{}
                As long as he recollects his own generosity in its special qualities of freedom from stain by avarice, etc., in this way, then: “On that occasion his mind is not obsessed by greed, or obsessed by hate, or obsessed by delusion; his mind has rectitude on that occasion, being inspired by generosity” (\textbf{\cite{A}III 287}).

                So when he has suppressed the hindrances in the way already described (\hyperlink{VII.66}{§66}{}), the jhāna factors arise in a single conscious moment. But owing to the profundity of the generosity’s special qualities, or owing to his being occupied in recollecting the generosity’s special qualities of many sorts, the jhāna is only access and does not reach absorption. And that access jhāna is known as “recollection of generosity” too because it arises with the generosity’s special qualities as the means.

                \vismParagraph{VII.114}{114}{}
                And when a bhikkhu is devoted to this recollection of generosity, he becomes ever more intent on generosity, his preference is for non-greed, he acts in conformity with loving-kindness, he is fearless. He has much happiness and gladness. And if he penetrates no higher, he is at least headed for a happy destiny.
                \begin{verse}
                    Now, when a man is truly wise,\\{}
                    His constant task will surely be\\{}
                    This recollection of his giving\\{}
                    Blessed with such mighty potency.
                \end{verse}


                This is the section dealing with the recollection of generosity in the detailed explanation. \textcolor{brown}{\textit{[225]}}
            \section[\vismAlignedParas{§115–118}(6) Recollection of Deities]{(6) Recollection of Deities}

                \vismParagraph{VII.115}{115}{}
                One who wants to develop the recollection of deities should possess the special qualities of faith, etc., evoked by means of the noble path, and he should go into solitary retreat and recollect his own special qualities of faith, etc., with deities standing as witnesses, as follows:

                “There are deities of the Realm of the Four Kings (\emph{devā cātummahārājikā}), there are deities of the Realm of the Thirty-three (\emph{devā tāvatiṃsā}), there are deities who are Gone to Divine Bliss (\emph{yāmā}) … who are Contented (\emph{tusitā}) … who Delight in Creating (\emph{nimmānarati}) … who Wield Power Over Others’ Creations (\emph{paranimmitavasavatti}), there are deities of Brahmā’s Retinue (\emph{brahmakāyikā}), there are deities higher than that. And those deities were possessed of faith such that on dying here they were reborn there, and such faith is present in me too. And those deities were possessed of virtue … of learning … of generosity … of understanding such that when they died here they were reborn there, and such understanding is present in me too” (\textbf{\cite{A}III 287}).

                \vismParagraph{VII.116}{116}{}
                In the sutta, however, it is said: “On the occasion, Mahānāma, on which a noble disciple recollects the faith, the virtue, the learning, the generosity, and the understanding that are both his own and of those deities,” on that occasion his mind is not obsessed by greed …” (\textbf{\cite{A}III 287}). Although this is said, it should \marginnote{\textcolor{teal}{\footnotesize\{280|222\}}}{}nevertheless be understood as said for the purpose of showing that the special qualities of faith, etc., in oneself are those in the deities, making the deities stand as witnesses. For it is said definitely in the Commentary: “He recollects his own special qualities, making the deities stand as witnesses.”

                \vismParagraph{VII.117}{117}{}
                As long as in the prior stage he recollects the deities’ special qualities of faith, etc., and in the later stage he recollects the special qualities of faith, etc., existing in himself, then: “On that occasion his mind is not obsessed by greed, or obsessed by hate, or obsessed by delusion, his mind has rectitude on that occasion, being inspired by deities” (\textbf{\cite{A}III 288}).

                So when he has suppressed the hindrances in the way already stated (\hyperlink{VII.66}{§66}{}), the jhāna factors arise in a single conscious moment. But owing to the profundity of the special qualities of faith, etc., or owing to his being occupied in recollecting special qualities of many sorts, the jhāna is only access and does not reach absorption. And that access jhāna itself is known as “recollection of deities” too because it arises with the deities special qualities as the means. \textcolor{brown}{\textit{[226]}}

                \vismParagraph{VII.118}{118}{}
                And when a bhikkhu is devoted to this recollection of deities, he becomes dearly loved by deities. He obtains even greater fullness of faith. He has much happiness and gladness. And if he penetrates no higher, he is at least headed for a happy destiny.
                \begin{verse}
                    Now, when a man is truly wise,\\{}
                    His constant task will surely be\\{}
                    This recollection of deities\\{}
                    Blessed with such mighty potency.
                \end{verse}


                This is the section dealing with the recollection of deities in the detailed explanation.
            \section[\vismAlignedParas{§119–128}General]{General}

                \vismParagraph{VII.119}{119}{}
                Now, in setting forth the detail of these recollections, after the words, “His mind has rectitude on that occasion, being inspired by the Perfect One,” it is added: “When a noble disciple’s mind has rectitude, Mahānāma, the meaning inspires him, the law inspires him, and the application of the law makes him glad. When he is glad, happiness is born in him” (\textbf{\cite{A}III 285–288}). Herein, \emph{the meaning inspires him }should be understood as said of contentment inspired by the meaning beginning, “This Blessed One is such since he is …” (\hyperlink{VII.2}{§2}{}). \emph{The law inspires him} is said of contentment inspired by the text. The \emph{application of the law makes him glad} is said of both (cf. \textbf{\cite{M-a}I 173}).

                \vismParagraph{VII.120}{120}{}
                And when in the case of the recollection of deities \emph{inspired by deities} is said, this should be understood as said either of the consciousness that occurs in the prior stage inspired by deities or of the consciousness [that occurs in the later stage] inspired by the special qualities that are similar to those of the deities and are productive of the deities’ state (cf. \hyperlink{VII.117}{§117}{}).

                \vismParagraph{VII.121}{121}{}
                These six recollections succeed only in noble disciples. For the special qualities of the Enlightened One, the Law, and the Community, are evident to them; and they possess the virtue with the special qualities of untornness, etc., \marginnote{\textcolor{teal}{\footnotesize\{281|223\}}}{}the generosity that is free from stain by avarice, and the special qualities of faith, etc., similar to those of deities.

                \vismParagraph{VII.122}{122}{}
                And in the Mahānāma Sutta (\textbf{\cite{A}III 285} f.) they are expounded in detail by the Blessed One in order to show a stream-winner an abiding to depend upon when he asked for one.

                \vismParagraph{VII.123}{123}{}
                Also in the Gedha Sutta they are expounded in order that a noble disciple should purify his consciousness by means of the recollections and so attain further purification in the ultimate sense thus: “Here, bhikkhus, a noble disciple recollects the Perfect One in this way: That Blessed One is such since he is accomplished … His mind has rectitude on that occasion. He has renounced, \textcolor{brown}{\textit{[227]}} got free from, emerged from cupidity. Cupidity, bhikkhus, is a term for the five cords of sense desire. Some beings gain purity here by making this [recollection] their prop” (\textbf{\cite{A}III 312}).

                \vismParagraph{VII.124}{124}{}
                And in the Sambādhokāsa Sutta taught by the venerable Mahā-Kaccāna they are expounded as the realization of the wide-open through the susceptibility of purification that exists in the ultimate sense only in a noble disciple thus: “It is wonderful, friends, it is marvellous how the realization of the wide-open in the crowded [house life] has been discovered by the Blessed One who knows and sees, accomplished and fully enlightened, for the purification of beings, [for the surmounting of sorrow and lamentation, for the ending of pain and grief, for the attainment of the true way], for the realization of Nibbāna, that is to say, the six stations of recollection. What six? Here, friends, a noble disciple recollects the Perfect One … Some beings are susceptible to purification in this way” (\textbf{\cite{A}III 314–315}).

                \vismParagraph{VII.125}{125}{}
                Also in the Uposatha Sutta they are expounded in order to show the greatness of the fruit of the Uposatha, as a mind-purifying meditation subject for a noble disciple who is observing the Uposatha: “And what is the Noble Ones’ Uposatha, Visākhā? It is the gradual cleansing of the mind still sullied by imperfections. And what is the gradual cleansing of the mind still sullied by imperfections? Here, Visākhā, a noble disciple recollects the Perfect One …” (\textbf{\cite{A}I 206–211}).

                \vismParagraph{VII.126}{126}{}
                And in the Book of Elevens, when a noble disciple has asked, “Venerable sir, in what way should we abide who abide in various ways?” (\textbf{\cite{A}V 328}), they are expounded to him in order to show the way of abiding in this way: “One who has faith is successful, Mahānāma, not one who has no faith. One who is energetic … One whose mindfulness is established … One who is concentrated … One who has understanding is successful, Mahānāma, not one who has no understanding. Having established yourself in these five things, Mahānāma, you should develop six things. Here, Mahānāma, you should recollect the Perfect One: That Blessed One is such since …” (\textbf{\cite{A}V 329–332}).

                \vismParagraph{VII.127}{127}{}
                Still, though this is so, they can be brought to mind by an ordinary man too, if he possesses the special qualities of purified virtue, and the rest. \textcolor{brown}{\textit{[228]}} For when he is recollecting the special qualities of the Buddha, etc., even only according to hearsay, his consciousness settles down, by virtue of which the hindrances are suppressed. In his supreme gladness he initiates insight, and \marginnote{\textcolor{teal}{\footnotesize\{282|224\}}}{}he even attains to Arahantship, like the Elder Phussadeva who dwelt at Kaṭakandhakāra.

                \vismParagraph{VII.128}{128}{}
                That venerable one, it seems, saw a figure of the Enlightened One created by Māra. He thought, “How good this appears despite its having greed, hate and delusion! What can the Blessed One’s goodness have been like? For he was quite without greed, hate and delusion!” He acquired happiness with the Blessed One as object, and by augmenting his insight he reached Arahantship.

                The seventh chapter called “The Description of Six Recollections” in the Treatise on the Development of Concentration in the \emph{Path of Purification} composed for the purpose of gladdening good people.
        \chapter[Other Recollections as Meditation Subjects]{Other Recollections as Meditation Subjects\vismHypertarget{VIII}\newline{\textnormal{\emph{Anussati-kammaṭṭhāna-niddesa}}}}
            \label{VIII}

            \section[\vismAlignedParas{§1–41}(7) Mindfulness of Death]{(7) Mindfulness of Death}

                \vismParagraph{VIII.1}{1}{}
                \marginnote{\textcolor{teal}{\footnotesize\{283|225\}}}{}\textcolor{brown}{\textit{[229]}} Now comes the description of the development of mindfulness of death, which was listed next (\hyperlink{III.105}{III.105}{}).
                \subsection[\vismAlignedParas{§1–3}Definitions]{Definitions}

                    Herein, \emph{death }(\emph{maraṇa}) is the interruption of the life faculty included within [the limits of] a single becoming (existence). But death as termination (cutting off), in other words, the Arahant’s termination of the suffering of the round, is not intended here, nor is momentary death, in other words, the momentary dissolution of formations, nor the “death” of conventional (metaphorical) usage in such expressions as “dead tree,” “dead metal,” and so on.

                    \vismParagraph{VIII.2}{2}{}
                    As intended here it is of two kinds, that is to say, timely death and untimely death. Herein, \emph{timely death }comes about with the exhaustion of merit or with the exhaustion of a life span or with both. \emph{Untimely death }comes about through kamma that interrupts [other, life-producing] kamma.

                    \vismParagraph{VIII.3}{3}{}
                    Herein, death through \emph{exhaustion of merit }is a term for the kind of death that comes about owing to the result of [former] rebirth-producing kamma’s having finished ripening although favourable conditions for prolonging the continuity of a life span may be still present. Death through \emph{exhaustion of a life span }is a term for the kind of death that comes about owing to the exhaustion of the normal life span of men of today, which measures only a century owing to want of such excellence in destiny [as deities have] or in time [as there is at the beginning of an aeon] or in nutriment [as the Uttarakurus and so on have].\footnote{\vismAssertFootnoteCounter{1}\vismHypertarget{VIII.n1}{}Amplifications are from \textbf{\cite{Vism-mhṭ}, p. 236}.} \emph{Untimely death }is a term for the death of those whose continuity is interrupted by kamma capable of causing them to fall (\emph{cāvana}) from their place at that very moment, as in the case of Dūsi-Māra (see \textbf{\cite{M}I 337}), Kalāburājā (see \textbf{\cite{J-a}III 39}), etc.,\footnote{\vismAssertFootnoteCounter{2}\vismHypertarget{VIII.n2}{}“The word ‘etc.’ includes Nanda-yakkha, Nanda-māṇava, and others” (Vism-mhṭ 236). See \textbf{\cite{A-a}II 104}, and \textbf{\cite{M-a}IV 8}.} or for the death of those whose [life’s] continuity is interrupted by assaults with weapons, etc., due to previous kamma. \textcolor{brown}{\textit{[230]}} All these are included under the interruption of \marginnote{\textcolor{teal}{\footnotesize\{284|226\}}}{}the life faculty of the kinds already stated. So mindfulness of death is the remembering of death, in other words, of the interruption of the life faculty.
                \subsection[\vismAlignedParas{§4–7}Development]{Development}

                    \vismParagraph{VIII.4}{4}{}
                    One who wants to develop this should go into solitary retreat and exercise attention wisely in this way: “Death will take place; the life faculty will be interrupted,” or “Death, death.”

                    \vismParagraph{VIII.5}{5}{}
                    If he exercises his attention unwisely in recollecting the [possible] death of an agreeable person, sorrow arises, as in a mother on recollecting the death of her beloved child she bore; and gladness arises in recollecting the death of a disagreeable person, as in enemies on recollecting the death of their enemies; and no sense of urgency arises on recollecting the death of neutral people, as happens in a corpse-burner on seeing a dead body; and anxiety arises on recollecting one’s own death, as happens in a timid person on seeing a murderer with a poised dagger.

                    \vismParagraph{VIII.6}{6}{}
                    In all that there is neither mindfulness nor sense of urgency nor knowledge. So he should look here and there at beings that have been killed or have died, and advert to the death of beings already dead but formerly seen enjoying good things, doing so with mindfulness, with a sense of urgency and with knowledge, after which he can exercise his attention in the way beginning, “Death will take place.” By so doing he exercises it wisely. He exercises it as a [right] means, is the meaning.\footnote{\vismAssertFootnoteCounter{3}\vismHypertarget{VIII.n3}{}For the expression \emph{upāya-manasikāra—}“attention as a [right] means” see \textbf{\cite{M-a}I 64}.}

                    \vismParagraph{VIII.7}{7}{}
                    When some exercise it merely in this way, their hindrances get suppressed, their mindfulness becomes established with death as its object, and the meditation subject reaches access.
                \subsection[\vismAlignedParas{§8–39}Eight ways of recollecting death]{Eight ways of recollecting death}

                    \vismParagraph{VIII.8}{8}{}
                    But one who finds that it does not get so far should do his recollecting of death in eight ways, that is to say: (1) as having the appearance of a murderer, (2) as the ruin of success, (3) by comparison, (4) as to sharing the body with many, (5) as to the frailty of life, (6) as signless, (7) as to the limitedness of the extent, (8) as to the shortness of the moment.
                    \subsubsection[\vismAlignedParas{§9–13}Having the appearance of a murderer]{Having the appearance of a murderer}

                        \vismParagraph{VIII.9}{9}{}
                        \emph{1.} Herein, \emph{as having the appearance of a murderer}: he should do his recollecting thus, “Just as a murderer appears with a sword, thinking, ‘I shall cut this man’s head off,’ and applies it to his neck, so death appears.” Why? Because it comes with birth and it takes away life.

                        \vismParagraph{VIII.10}{10}{}
                        As budding toadstools always come up lifting dust on their tops, so beings are born along with aging and death. For accordingly their rebirth-linking consciousness reaches aging immediately next to its arising and then breaks up together with its associated aggregates, like a stone that falls from the summit of a rock. \textcolor{brown}{\textit{[231]}} So to begin with, momentary death comes along with birth. But death is inevitable for what is born; consequently the kind of death intended here also comes along with birth.

                        \vismParagraph{VIII.11}{11}{}
                        \marginnote{\textcolor{teal}{\footnotesize\{285|227\}}}{}Therefore, just as the risen sun moves on towards its setting and never turns back even for a little while from wherever it has got to, or just as a mountain torrent sweeps by with a rapid current, ever flowing and rushing on and never turning back even for a little while, so too this living being travels on towards death from the time when he is born, and he never turns back even for a little while. Hence it is said:
                        \begin{verse}
                            “Right from the very day a man\\{}
                            Has been conceived inside a womb\\{}
                            He cannot but go on and on,\\{}
                            Nor going can he once turn back” (\textbf{\cite{J-a}IV 494}).
                        \end{verse}


                        \vismParagraph{VIII.12}{12}{}
                        And whilst he goes on thus death is as near to him as drying up is to rivulets in the summer heat, as falling is to the fruits of trees when the sap reaches their attachments in the morning, as breaking is to clay pots tapped by a mallet, as vanishing is to dewdrops touched by the sun’s rays. Hence it is said:
                        \begin{verse}
                            “The nights and days go slipping by\\{}
                            As life keeps dwindling steadily\\{}
                            Till mortals’ span, like water pools\\{}
                            In failing rills, is all used up” (\textbf{\cite{S}I 109}).
                        \end{verse}

                        \begin{verse}
                            “As there is fear, when fruits are ripe,\\{}
                            That in the morning they will fall,\\{}
                            So mortals are in constant fear,\\{}
                            When they are born, that they will die.\\{}
                            And as the fate of pots of clay\\{}
                            Once fashioned by the potter’s hand,\\{}
                            Or small or big or baked or raw,\footnote{\vismAssertFootnoteCounter{4}\vismHypertarget{VIII.n4}{}This line is not in the Sutta-nipāta, but see \textbf{\cite{D}II 120}, note.}\\{}
                            Condemns them to be broken up,\\{}
                            So mortals’ life leads but to death” (\textbf{\cite{Sn}p. 576f.}).
                        \end{verse}

                        \begin{verse}
                            “The dewdrop on the blade of grass\\{}
                            Vanishes when the sun comes up;\\{}
                            Such is a human span of life;\\{}
                            So, mother, do not hinder me” (\textbf{\cite{J-a}IV 122}).
                        \end{verse}


                        \vismParagraph{VIII.13}{13}{}
                        So this death, which comes along with birth, is like a murderer with poised sword. And like the murderer who applies the sword to the neck, it carries off life and never returns to bring it back. \textcolor{brown}{\textit{[232]}} That is why, since death appears like a murderer with poised sword owing to its coming along with birth and carrying off life, it should be recollected as “having the appearance of a murderer.”
                    \subsubsection[\vismAlignedParas{§14–15}Ruin of success]{Ruin of success}

                        \vismParagraph{VIII.14}{14}{}
                        \emph{2. As the ruin of success}: here success shines as long as failure does not overcome it. And the success does not exist that might endure out of reach of failure. Accordingly:
                        \begin{verse}
                            “He gave with joy a hundred millions\\{}
                            After conquering all the earth,\\{}
                            Till in the end his realm came down
                        \end{verse}

                        \begin{verse}
                            \marginnote{\textcolor{teal}{\footnotesize\{286|228\}}}{}To less than half a gall-nut’s worth.\\{}
                            Yet when his merit was used up,\\{}
                            His body breathing its last breath,\\{}
                            The Sorrowless Asoka too\footnote{\vismAssertFootnoteCounter{5}\vismHypertarget{VIII.n5}{}The Emperor Asoka is referred to. His name Asoka means “Sorrowless.” This story is in the \emph{Asokāvadāna} and \emph{Divyāvadāna}, pp. 429–434.}\\{}
                            Felt sorrow face to face with death.”
                        \end{verse}


                        \vismParagraph{VIII.15}{15}{}
                        Furthermore, all health ends in sickness, all youth ends in aging, all life ends in death; all worldly existence is procured by birth, haunted by aging, surprised by sickness, and struck down by death. Hence it is said:
                        \begin{verse}
                            “As though huge mountains made of rock\\{}
                            So vast they reached up to the sky\\{}
                            Were to advance from every side,\\{}
                            Grinding beneath them all that lives,\\{}
                            So age and death roll over all,\\{}
                            Warriors, priests, merchants, and craftsmen,\\{}
                            The outcastes and the scavengers,\\{}
                            Crushing all beings, sparing none.\\{}
                            And here no troops of elephants,\\{}
                            No charioteers, no infantry,\\{}
                            No strategy in form of spells,\\{}
                            No riches, serve to beat them off” (\textbf{\cite{S}I 102}).
                        \end{verse}


                        This is how death should be recollected as the “ruin of success” by defining it as death’s final ruining of life’s success.
                    \subsubsection[\vismAlignedParas{§16–24}By comparison]{By comparison}

                        \vismParagraph{VIII.16}{16}{}
                        \emph{3. By comparison}: by comparing oneself to others. Herein, death should be recollected by comparison in seven ways, that is to say: with those of great fame, with those of great merit, with those of great strength, with those of great supernormal power, with those of great understanding, with Paccekabuddhas, with fully enlightened Buddhas. How?

                        \vismParagraph{VIII.17}{17}{}
                        \textcolor{brown}{\textit{[233]}} Although Mahāsammata, Mandhātu, Mahāsudassana, Daḷhanemi, Nimi,\footnote{\vismAssertFootnoteCounter{6}\vismHypertarget{VIII.n6}{}The references for the names here and in the following paragraphs are: Mahāsammata (\textbf{\cite{J-a}III 454}; II 311), Mandhātu (\textbf{\cite{J-a}II 311}), Mahāsudassana (\textbf{\cite{D}II 169f.}), Daḷhanemi (\textbf{\cite{D}III 59f.}), Nimi (\textbf{\cite{J-a}VI 96f.}), Jotika (Vism \hyperlink{XII.41}{XII.41}{}), Jaṭila (\hyperlink{XII.41}{XII.41}{}), Ugga (\textbf{\cite{A-a}I 394}), Meṇḍaka (\hyperlink{XII.41}{XII.41f.}{}), Puṇṇaka (\hyperlink{XII.42}{XII.42}{}), Vāsudeva (\textbf{\cite{J-a}IV 81f.}), Baladeva (\textbf{\cite{J-a}IV 81f.}), Bhīmasena (\textbf{\cite{J-a}V 426}), Yuddhiṭṭhila (\textbf{\cite{J-a}V 426}), Cāṇura (\textbf{\cite{J-a}IV 81}).} etc.,\footnote{\vismAssertFootnoteCounter{7}\vismHypertarget{VIII.n7}{}\emph{Pabhuti—}“etc.”: this meaning is not in PED; see §121.} were greatly famous and had a great following, and though they had amassed enormous wealth, yet death inevitably caught up with them at length, so how shall it not at length overtake me?
                        \begin{verse}
                            Great kings like Mahāsammata,\\{}
                            Whose fame did spread so mightily,\\{}
                            All fell into death’s power too;\\{}
                            What can be said of those like me?
                        \end{verse}


                        \marginnote{\textcolor{teal}{\footnotesize\{287|229\}}}{}It should be recollected in this way, firstly, by comparison \emph{with those of great fame}.

                        \vismParagraph{VIII.18}{18}{}
                        How by comparison with those of great merit?
                        \begin{verse}
                            Jotika, Jaṭila, Ugga,\\{}
                            And Meṇḍaka, and Puṇṇaka\\{}
                            These, the world said, and others too,\\{}
                            Did live most meritoriously;\\{}
                            Yet they came one and all to death;\\{}
                            What can be said of those like me?
                        \end{verse}


                        It should be recollected in this way by comparison with those of great merit.

                        \vismParagraph{VIII.19}{19}{}
                        How by comparison with those of great strength?
                        \begin{verse}
                            Vāsudeva, Baladeva,\\{}
                            Bhīmasena, Yuddhiṭṭhila,\\{}
                            And Cāṇura the wrestler,\\{}
                            Were in the Exterminator’s power.\\{}
                            Throughout the world they were renowned\\{}
                            As blessed with strength so mighty;\\{}
                            They too went to the realm of death;\\{}
                            What can be said of those like me?
                        \end{verse}


                        It should be recollected in this way by comparison with those of great strength.

                        \vismParagraph{VIII.20}{20}{}
                        How by comparison with those of great supernormal power?
                        \begin{verse}
                            The second of the chief disciples,\\{}
                            The foremost in miraculous powers,\\{}
                            Who with the point of his great toe\\{}
                            Did rock Vejayanta’s Palace towers,\\{}
                            Like a deer in a lion’s jaw, he too,\\{}
                            Despite miraculous potency,\\{}
                            Fell in the dreadful jaws of death;\\{}
                            What can be said of those like me?
                        \end{verse}


                        It should be recollected in this way by comparison with those of great supernormal power.

                        \vismParagraph{VIII.21}{21}{}
                        How by comparison with those of great understanding? \textcolor{brown}{\textit{[234]}}
                        \begin{verse}
                            The first of the two chief disciples\\{}
                            Did so excel in wisdom’s art\\{}
                            That, save the Helper of the World,\\{}
                            No being is worth his sixteenth part.\\{}
                            But though so great was Sāriputta’s\\{}
                            Understanding faculty,\\{}
                            He fell into death’s power too;\\{}
                            What can be said of those like me?
                        \end{verse}


                        It should be recollected in this way by comparison with those of great understanding.

                        \vismParagraph{VIII.22}{22}{}
                        \marginnote{\textcolor{teal}{\footnotesize\{288|230\}}}{}How by comparison \emph{with Paccekabuddhas}? Even those who by the strength of their own knowledge and energy crushed all the enemy defilements and reached enlightenment for themselves, who [stood alone] like the horn of the rhinoceros (see \textbf{\cite{Sn}p. 35f.}), who were self-perfected, were still not free from death. So how should I be free from it?
                        \begin{verse}
                            To help them in their search for truth\\{}
                            The Sages various signs employed,\\{}
                            Their knowledge brought them self-perfection,\\{}
                            Their cankers were at length destroyed.
                        \end{verse}

                        \begin{verse}
                            Like the rhinoceros’s horn\\{}
                            They lived alone in constancy,\\{}
                            But death they could no way evade;\\{}
                            What can be said of those like me?
                        \end{verse}


                        It should be recollected in this way by comparison with Paccekabuddhas.

                        \vismParagraph{VIII.23}{23}{}
                        How by comparison \emph{with fully enlightened Buddhas}? Even the Blessed One, whose material body was embellished with the eighty lesser details and adorned with the thirty-two marks of a great man (see MN 91; DN 30), whose Dhamma body brought to perfection the treasured qualities of the aggregates of virtue, etc.,\footnote{\vismAssertFootnoteCounter{8}\vismHypertarget{VIII.n8}{}Virtue, concentration, understanding, deliverance, knowledge, and vision of deliverance.} made pure in every aspect, who overpassed greatness of fame, greatness of merit, greatness of strength, greatness of supernormal power and greatness of understanding, who had no equal, who was the equal of those without equal, without double, accomplished and fully enlightened—even he was suddenly quenched by the downpour of death’s rain, as a great mass of fire is quenched by the downpour of a rain of water.
                        \begin{verse}
                            And so the Greatest Sage possessed\\{}
                            Such mighty power in every way,\\{}
                            And it was not through fear or guilt\\{}
                            That over him Death held his sway.
                        \end{verse}

                        \begin{verse}
                            No being, not even one without\\{}
                            Guilt or pusillanimity,\\{}
                            But will be smitten down; so how I\\{}
                            Will he not conquer those like me?
                        \end{verse}


                        It should be recollected in this way by comparison with fully enlightened Buddhas.

                        \vismParagraph{VIII.24}{24}{}
                        When he does his recollecting in this way by comparing himself with others possessed of such great fame, etc., in the light of the universality of death, thinking, “Death will come to me even as it did to those distinguished beings,” then his meditation subject reaches access. This is how death should be recollected by comparison. \textcolor{brown}{\textit{[235]}}
                    \subsubsection[\vismAlignedParas{§25–26}Sharing of the body with many]{Sharing of the body with many}

                        \vismParagraph{VIII.25}{25}{}
                        \emph{4. As to the sharing of the body with many}: this body is shared by many. Firstly, it is shared by the eighty families of worms. There too, creatures live in dependence \marginnote{\textcolor{teal}{\footnotesize\{289|231\}}}{}on the outer skin, feeding on the outer skin; creatures live in dependence on the inner skin, feeding on the inner skin; creatures live in dependence on the flesh, feeding on the flesh; creatures live in dependence on the sinews, feeding on the sinews; creatures live in dependence on the bones, feeding on the bones; and creatures live in dependence on the marrow, feeding on the marrow. And there they are born, grow old and die, evacuate, and make water; and the body is their maternity home, their hospital, their charnel-ground, their privy and their urinal. The body can also be brought to death with the upsetting of these worms. And just as it is shared with the eighty families of worms, so too it is shared by the several hundred internal diseases, as well as by such external causes of death as snakes, scorpions, and what not.

                        \vismParagraph{VIII.26}{26}{}
                        And just as when a target is set up at a crossroads and then arrows, spears, pikes, stones, etc., come from all directions and fall upon it, so too all kinds of accidents befall the body, and it also comes to death through these accidents befalling it. Hence the Blessed One said: “Here, bhikkhus, when day is departing and night is drawing on,\footnote{\vismAssertFootnoteCounter{9}\vismHypertarget{VIII.n9}{}\emph{Paṭihitāya—}“drawing on”: not in PED; \textbf{\cite{Vism-mhṭ}(p. 240)} reads \emph{paṇitāya} and explains by \emph{paccāgatāya} (come back).} a bhikkhu considers thus: ‘In many ways I can risk death. A snake may bite me, or a scorpion may sting me, or a centipede may sting me. I might die of that, and that would set me back. Or I might stumble and fall, or the food I have eaten might disagree with me, or my bile might get upset, or my phlegm might get upset [and sever my joints as it were] like knives. I might die of that, and that would set me back’” (\textbf{\cite{A}III 306}).

                        That is how death should be recollected as to sharing the body with many.
                    \subsubsection[\vismAlignedParas{§27–28}Fraility of life]{Fraility of life}

                        \vismParagraph{VIII.27}{27}{}
                        \emph{5. As to the frailty of life}: this life is impotent and frail. For the life of beings is bound up with breathing, it is bound up with the postures, it is bound up with cold and heat, it is bound up with the primary elements, and it is bound up with nutriment.

                        \vismParagraph{VIII.28}{28}{}
                        Life occurs only when the in-breaths and out-breaths occur evenly. But when the wind in the nostrils that has gone outside does not go in again, or when that which has gone inside does not come out again, then a man is reckoned to be dead.

                        And it occurs only when the four postures are found occurring evenly. \textcolor{brown}{\textit{[236]}}

                        But with the prevailing of anyone of them the life process is interrupted.

                        And it occurs only when cold and heat are found occurring evenly. But it fails when a man is overcome by excessive cold or heat.

                        And it occurs only when the four primary elements are found occurring evenly. But with the disturbance of the earth element even a strong man’s life can be terminated if his body becomes rigid, or with the disturbance of one of the elements beginning with water if his body becomes flaccid and petrified with a flux of the bowels, etc., or if he is consumed by a bad fever, or if he suffers a severing of his limb-joint ligatures (cf. \hyperlink{XI.102}{XI.102}{}).

                        \marginnote{\textcolor{teal}{\footnotesize\{290|232\}}}{}And life occurs only in one who gets physical nutriment at the proper time; but if he gets none, he uses his life up.

                        This is how death should be recollected as to the frailty of life.
                    \subsubsection[\vismAlignedParas{§29–34}Signless]{Signless}

                        \vismParagraph{VIII.29}{29}{}
                        \emph{6. As signless}: as indefinable. The meaning is that it is unpredictable. For in the case of all beings:
                        \begin{verse}
                            The span, the sickness, and the time, and where\\{}
                            The body will be laid, the destiny:\\{}
                            The living world can never know\footnote{\vismAssertFootnoteCounter{10}\vismHypertarget{VIII.n10}{}\emph{Nāyare—}“can know”: form not in PED; \textbf{\cite{Vism-mhṭ}} explains by \emph{ñāyanti}.} these things;\\{}
                            There is no sign foretells when they will be.
                        \end{verse}


                        \vismParagraph{VIII.30}{30}{}
                        Herein, firstly \emph{the span }has no sign because there is no definition such as: Just so much must be lived, no more than that. For beings [die in the various stages of the embryo, namely], at the time of the \emph{kalala}, of the \emph{abbuda}, of the \emph{pesi}, of the \emph{ghana}, at one month gone, two months gone, three months gone, four months gone, five months gone … ten months gone, and on the occasion of coming out of the womb. And after that they die this side or the other of the century.

                        \vismParagraph{VIII.31}{31}{}
                        And \emph{the sickness }has no sign because there is no definition such as: Beings die only of this sickness, not of any other. For beings die of eye disease or of any one among those beginning with ear disease (see \textbf{\cite{A}V 110}).

                        \vismParagraph{VIII.32}{32}{}
                        And \emph{the time }has no sign because there is no definition such as: One has to die only at this time, not at any other. For beings die in the morning and at any of the other times such as noon.

                        \vismParagraph{VIII.33}{33}{}
                        And \emph{where the body will be laid down }has no sign because there is no definition such as: When people die, they must drop their bodies only here, not anywhere else. For the person of those born inside a village is dropped outside the village, and that of those born outside the village is dropped inside it. Likewise that of those born in water is dropped on land, and that of those born on land in water. And this can be multiplied in many ways. \textcolor{brown}{\textit{[237]}}

                        \vismParagraph{VIII.34}{34}{}
                        And \emph{the destiny }has no sign because there is no definition such as: One who dies there must be reborn here. For there are some who die in a divine world and are reborn in the human world, and there are some who die in the human world and are reborn in a divine world, and so on. And in this way the world goes round and round the five kinds of destinies like an ox harnessed to a machine.

                        This is how death should be recollected as signless.
                    \subsubsection[\vismAlignedParas{§35–38}Limitedness of the extend]{Limitedness of the extend}

                        \vismParagraph{VIII.35}{35}{}
                        \emph{7. As to the limitedness of the extent}: the extent of human life is short now. One who lives long lives a hundred years, more or less. Hence the Blessed One said: “Bhikkhus, this human life span is short. There is a new life to be gone to, there are profitable [deeds] to be done, there is the life of purity to be led. There is no not dying for the born. He who lives long lives a hundred years, more or less …”
                        \begin{verse}
                            “The life of humankind is short;\\{}
                            A wise man holds it in contempt\\{}
                            And acts as one whose head is burning;\\{}
                            Death will never fail to come” (\textbf{\cite{S}I 108}).
                        \end{verse}


                        \marginnote{\textcolor{teal}{\footnotesize\{291|233\}}}{}And he said further: “Bhikkhus, there was once a teacher called Araka …” (\textbf{\cite{A}IV 136}), all of which sutta should be given in full, adorned as it is with seven similes.

                        \vismParagraph{VIII.36}{36}{}
                        And he said further: “Bhikkhus, when a bhikkhu develops mindfulness of death thus, ‘Oh, let me live a night and day that I may attend to the Blessed One’s teaching, surely much could be done by me,’ and when a bhikkhu develops mindfulness of death thus, ‘Oh, let me live a day that I may attend to the Blessed One’s teaching, surely much could be done by me,’ and when a bhikkhu develops mindfulness of death thus, ‘Oh, let me live as long as it takes to chew and swallow four or five mouthfuls that I may attend to the Blessed One’s teaching, surely much could be done by me’—these are called bhikkhus who dwell in negligence and slackly develop mindfulness of death for the destruction of cankers. \textcolor{brown}{\textit{[238]}}

                        \vismParagraph{VIII.37}{37}{}
                        “And, bhikkhus, when a bhikkhu develops mindfulness of death thus, ‘Oh, let me live for as long as it takes to chew and swallow a single mouthful that I may attend to the Blessed One’s teaching, surely much could be done by me,’ and when a bhikkhu develops mindfulness of death thus, ‘Oh, let me live as long as it takes to breathe in and breathe out, or as long as it takes to breathe out and breathe in, that I may attend to the Blessed One’s teaching, surely much could be done by me’—these are called bhikkhus who dwell in diligence and keenly develop mindfulness of death for the destruction of cankers” (\textbf{\cite{A}III 305–306}).

                        \vismParagraph{VIII.38}{38}{}
                        So short in fact is the extent of life that it is not certain even for as long as it takes to chew and swallow four or five mouthfuls.

                        This is how death should be recollected as to the limitedness of the extent.
                    \subsubsection[\vismAlignedParas{§39}Shortness of the moment]{Shortness of the moment}

                        \vismParagraph{VIII.39}{39}{}
                        \emph{8. As to the shortness of the moment}: in the ultimate sense the life-moment of living beings is extremely short, being only as much as the occurrence of a single conscious moment. Just as a chariot wheel, when it is rolling, rolls [that is, touches the ground] only on one point of [the circumference of] its tire, and, when it is at rest, rests only on one point, so too, the life of living beings lasts only for a single conscious moment. When that consciousness has ceased, the being is said to have ceased, according as it is said: “In a past conscious moment he did live, not he does live, not he will live. In a future conscious moment not he did live, not he does live, he will live. In the present conscious moment not he did live, he does live, not he will live.”
                        \begin{verse}
                            “Life, person, pleasure, pain—just these alone\\{}
                            Join in one conscious moment that flicks by.\\{}
                            Ceased aggregates of those dead or alive\\{}
                            Are all alike, gone never to return.
                        \end{verse}

                        \begin{verse}
                            No [world is] born if [consciousness is] not\\{}
                            Produced; when that is present, then it lives;\\{}
                            When consciousness dissolves, the world is dead:\\{}
                            The highest sense this concept will allow”\footnote{\vismAssertFootnoteCounter{11}\vismHypertarget{VIII.n11}{}“‘Person’ (\emph{atta-bhāva}) is the states other than the already-mentioned life, feeling and consciousness. The words ‘\emph{just these alone}’ mean that it is unmixed with self (\emph{attā}) or permanence” (\textbf{\cite{Vism-mhṭ}242}). \emph{Atta-bhāva} as used in the Suttas and in this work is more or less a synonym for \emph{sakkāya} in the sense of person (body and mind) or personality, or individual form. See Piṭaka refs. in PED and e.g. this chapter §35 and \hyperlink{XI.54}{XI.54}{}.

                                    “‘\emph{When consciousness dissolves, the world is dead}”: just as in the case of the death-consciousness, this world is also called ‘dead’ in the highest (ultimate) sense with the arrival of any consciousness whatever at its dissolution, since its cessation has no rebirth-linking (is ‘cessation never to return’). Nevertheless, though this is so, ‘\emph{the highest sense this concept will allow} (\emph{paññatti paramatthiyā})’—the ultimate sense will allow this concept of continuity, which is what the expression of common usage ‘Tissa lives, Phussa lives’ refers to, and which is based on consciousnesses [momentarily] existing along with a physical support; this belongs to the ultimate sense here, since, as they say, ‘It is not the name and surname that lives.’” (\textbf{\cite{Vism-mhṭ}242}, 801)

                                    Something may be said about the word \emph{paññatti} here. Twenty-four kinds are dealt with in the commentary to the Puggalapaññatti. The Puggalapaññatti Schedule (\emph{mātikā}) gives the following six \emph{paññatti} (here a making known, a setting out): of aggregates, bases, elements, truths, faculties, and persons. (Pug 1) The commentary explains the word in this sense as \emph{paññāpana} (making known) and \emph{ṭhapana} (placing), quoting “He announces, teaches, declares (\emph{paññāpeti}), establishes” (cf. \textbf{\cite{M}III 248}), and also “a well-appointed (\emph{supaññatta}) bed and chair” (?). It continues: “The making known of a name (\emph{nāma-paññatti}) shows such and such dhammas and places them in such and such compartments, while the making known of the aggregates (\emph{khandha-paññatti}) and the rest shows in brief the individual form of those making-known (\emph{paññatti}).”

                                    It then gives six kinds of \emph{paññatti} “according to the commentarial method but not in the texts”: (1) \emph{Concept of the existent} (\emph{vijjamāna-paññatti}), which is the conceptualizing of (making known) a dhamma that is existent, actual, become, in the true and ultimate sense (e.g. aggregates, etc.). (2) \emph{Concept of the non-existent}, which is, for example, the conceptualizing of “female,” “male,” “persons,” etc., which are non-existent by that standard and are only established by means of current speech in the world; similarly “such impossibilities as concepts of a fifth truth or the other sectarians’ Atom, Primordial Essence, World Soul, and the like.” (3) \emph{Concept of the non-existent based on the existent}, e.g. the expression, “One with the three clear-visions,” where the “person” (“one”) is nonexistent and the “clear-visions” are existent. (4)\emph{ Concept of the existent based on the non-existent}, e.g. the “female form,” “visible form” (= visible datum base) being existent and “female” non-existent. (5) \emph{Concept of the existent} based on the existent, e.g. “eye-contact,” both “eye” and “contact” being existent. (6) \emph{Concept of the non-existent based on the non-existent}, e.g. “banker’s son,” both being non-existent.

                                    Again two more sets of six are given as “according to the Teachers, but not in the Commentaries.” The first is: (1) \emph{Derivative concept} (\emph{upādā-paññatti}); this, for instance, is a “being,” which is a convention derived from the aggregates of materiality, feeling, etc., though it has no individual essence of its own apprehendable in the true ultimate sense, as materiality, say, has in its self-identity and its otherness from feeling, etc.; or a “house” or a “fist” or an “oven” as apart from its component parts, or a “pitcher” or a “garment,” which are all derived from those same aggregates; or “time” or “direction,” which are derived from the revolutions of the moon and sun; or the “learning sign” or “counterpart sign” founded on some aspect or other, which are a convention derived from some real sign as a benefit of meditative development: these are derived concepts, and this kind is a “concept” (\emph{paññatti}) in the sense of “ability to be set up” (\emph{paññāpetabba} = ability to be conceptualized), but not in the sense of “making known” (\emph{paññāpana}). Under the latter heading this would be a “concept of the nonexistent.” (2) \emph{Appositional concept} (\emph{upa-nidhā-p}.): many varieties are listed, namely, apposition of reference (“second” as against “first,” “third” as against “second,” “long” as against “short”); apposition of what is in the hand (“umbrella-in-hand,” “knife-in-hand”); apposition of association (“earring-wearer,” “topknot-wearer,” “crest-wearer”); apposition of contents (“corn-wagon,” “ghee-pot”); apposition of proximity (“Indasālā Cave,” “Piyaṅgu Cave”); apposition of comparison (“golden coloured,” “with a bull’s gait”); apposition of majority (“Padumassara-brahman Village”); apposition of distinction (“diamond ring”); and so on. (3) \emph{Collective concept }(\emph{samodhāna-p}.), e.g., “eight-footed,” “pile of riches.” (4) \emph{Additive concept} (\emph{upanikkhitta-p}.), e.g. “one,” “two,” “three.” (5) \emph{Verisimilar concept} (\emph{tajjā-p}.): refers to the individual essence of a given dhamma, e.g. “earth,” “fire,” “hardness,” “heat.” (6) \emph{Continuity concept} (\emph{santati-p}.): refers to the length of continuity of life, e.g. “octogenarian,” “nonagenarian.”

                                    In the second set there are: (i) \emph{Concept according to function} (\emph{kicca-p}.), e.g. “preacher,” “expounder of Dhamma.” (ii) \emph{Concept according to shape} (\emph{saṇṭhāna-p.}), e.g. “thin,” “stout,” “round,” “square.” (iii) \emph{Concept according to gender} (\emph{liṅga-p}.), e.g. “female,” “male.” (iv) \emph{Concept according to location} (\emph{bhūmi-p}.), e.g. “of the sense sphere,” “Kosalan.” (v) \emph{Concept as proper name} (\emph{paccatta-p}.), e.g. “Tissa,” “Nāga,” “Sumana,” which are making-known (appellations) by mere name-making. (vi) \emph{Concept of the unformed} (\emph{asaṅkhata-paññatti}), e.g. “cessation,” “Nibbāna,” etc., which make the unformed dhamma known—an existent concept. (From commentary to \emph{Puggalapaññatti}, condensed—see also \textbf{\cite{Dhs-a}390f.})

                                    All this shows that the word \emph{paññatti} carries the meanings of either appellation or concept or both together, and that no English word quite corresponds.} (\textbf{\cite{Nidd}I 42}).
                        \end{verse}


                        \marginnote{\textcolor{teal}{\footnotesize\{293|235\}}}{}This is how death should be recollected as to the shortness of the moment.
                \subsection[\vismAlignedParas{§40–41}Conclusion]{Conclusion}

                    \vismParagraph{VIII.40}{40}{}
                    So while he does his recollecting by means of one or other of these eight ways, his consciousness acquires [the support of] repetition owing to the reiterated attention, mindfulness settles down with death as its object, the hindrances are suppressed, and the jhāna factors make their appearance. But since the object is stated with individual essences,\footnote{\vismAssertFootnoteCounter{12}\vismHypertarget{VIII.n12}{}“‘But since the object is stated with individual essences’: the breakup of states with individual essences, their destruction, their fall—[all] that has to do only with states with individual essences. Hence the Blessed One said: ‘Bhikkhus, aging-and-death is impermanent, formed, dependently arisen’ (\textbf{\cite{S}II 26}). … If it cannot reach absorption because of [its object being] states with individual essences then what about the supramundane jhānas and certain of the immaterial jhānas? It was to answer this that he said ‘now with special development the supramundane jhāna’ and so on” (\textbf{\cite{Vism-mhṭ}243}). Kasiṇa jhāna, for example, has a concept (\emph{paññatti}) as its object (\hyperlink{IV.29}{IV.29}{}) and a concept is a dhamma without individual essence (\emph{asabhāva-dhamma}).} and since it awakens a sense of urgency, the jhāna does not reach absorption and is only access. \textcolor{brown}{\textit{[239]}} Now, with special development, the supramundane jhāna and the second and the fourth immaterial jhānas reach absorption even with respect to states with individual essences. For the supramundane reaches absorption by means of \marginnote{\textcolor{teal}{\footnotesize\{294|236\}}}{}progressive development of the purification and the immaterial jhānas do so by means of development consisting in the surmounting of the object (see \hyperlink{X}{Ch. X}{}) since there [in those two immaterial jhānas] there is merely the surmounting of the object of jhāna that had already reached absorption. But here [in mundane mindfulness of death] there is neither so the jhāna only reaches access. And that access is known as “mindfulness of death” too since it arises through its means.

                    \vismParagraph{VIII.41}{41}{}
                    A bhikkhu devoted to mindfulness of death is constantly diligent. He acquires perception of disenchantment with all kinds of becoming (existence). He conquers attachment to life. He condemns evil. He avoids much storing. He has no stain of avarice about requisites. Perception of impermanence grows in him, following upon which there appear the perceptions of pain and not-self. But while beings who have not developed [mindfulness of] death fall victims to fear, horror and confusion at the time of death as though suddenly seized by wild beasts, spirits, snakes, robbers, or murderers, he dies undeluded and fearless without falling into any such state. And if he does not attain the deathless here and now, he is at least headed for a happy destiny on the breakup of the body.
                    \begin{verse}
                        Now, when a man is truly wise,\\{}
                        His constant task will surely be\\{}
                        This recollection about death\\{}
                        Blessed with such mighty potency.
                    \end{verse}


                    This is the section dealing with the recollection of death in the detailed explanation.
            \section[\vismAlignedParas{§42–144}(8) Mindfulness Occupied with the Body]{(8) Mindfulness Occupied with the Body}

                \vismParagraph{VIII.42}{42}{}
                Now comes the description of the development of mindfulness occupied with the body as a meditation subject, which is never promulgated except after an Enlightened One’s arising, and is outside the province of any sectarians. It has been commended by the Blessed One in various ways in different suttas thus: “Bhikkhus, when one thing is developed and repeatedly practiced, it leads to a supreme sense of urgency, to supreme benefit, to supreme surcease of bondage, to supreme mindfulness and full awareness, to acquisition of knowledge and vision, to a happy life here and now, to realization of the fruit of clear vision and deliverance. What is that one thing? It is mindfulness occupied with the body” (\textbf{\cite{A}I 43}). And thus: “Bhikkhus, they savour the deathless who savour mindfulness occupied with the body; they do not savour the deathless who do not savour mindfulness occupied with the body.\footnote{\vismAssertFootnoteCounter{13}\vismHypertarget{VIII.n13}{}In the Aṅguttara text the negative and positive clauses are in the opposite order.} \textcolor{brown}{\textit{[240]}} They have savoured the deathless who have savoured mindfulness occupied with the body; they have not savoured … They have neglected … they have not neglected … They have missed … they have found the deathless who have found mindfulness occupied with the body” (\textbf{\cite{A}I 45}). And it has been described in fourteen sections in the passage beginning, “And how developed, bhikkhus, how repeatedly practiced is mindfulness occupied with the body of great fruit, of great benefit? Here, bhikkhus, a bhikkhu, gone to the forest …” (\textbf{\cite{M}III 89}), that is to say, the sections on breathing, on \marginnote{\textcolor{teal}{\footnotesize\{295|237\}}}{}postures, on the four kinds of full awareness, on attention directed to repulsiveness, on attention directed to elements, and on the nine charnel-ground contemplations.

                \vismParagraph{VIII.43}{43}{}
                Herein, the three, that is to say, the sections on postures, on the four kinds of full awareness (see \textbf{\cite{M-a}I 253f.}), and on attention directed to elements, as they are stated [in that sutta], deal with insight. Then the nine sections on the charnel-ground contemplations, as stated there, deal with that particular phase of insight knowledge called contemplation of danger. And any development of concentration in the bloated, etc., that might be implied there has already been explained in the Description of Foulness (\hyperlink{VI}{Ch. VI}{}). So there are only the two, that is, the sections on breathing and on directing attention to repulsiveness, that, as stated there, deal with concentration. Of these two, the section on breathing is a separate meditation subject, namely, mindfulness of breathing.
                \subsection[\vismAlignedParas{§44}Text]{Text}

                    \vismParagraph{VIII.44}{44}{}
                    What is intended here as mindfulness occupied with the body is the thirty-two aspects. This meditation subject is taught as the direction of attention to repulsiveness thus: “Again, bhikkhus, a bhikkhu reviews this body, up from the soles of the feet and down from the top of the hair and contained in the skin, as full of many kinds of filth thus: In this body there are head hairs, body hairs, nails, teeth, skin, flesh, sinews, bones, bone marrow, kidney, heart, liver, midriff, spleen, lungs, bowels, entrails, gorge, dung, bile, phlegm, pus, blood, sweat, fat, tears, grease, spittle, snot, oil of the joints, and urine” (\textbf{\cite{M}III 90}), the brain being included in the bone marrow in this version [with a total of only thirty-one aspects].
                \subsection[\vismAlignedParas{§45–47}Word Commentary]{Word Commentary}

                    \vismParagraph{VIII.45}{45}{}
                    Here is the description of the development introduced by a commentary on the text.

                    \emph{This body}: this filthy body constructed out of the four primary elements. \emph{Up from the soles of the feet}: from the soles of the feet upwards. \emph{Down from the top of the hair}: from the highest part of the hair downwards. \emph{Contained in the skin}: terminated all round by the skin. \emph{Reviews … as full of many kinds of filth}: \textcolor{brown}{\textit{[241]}} he sees that this body is packed with the filth of various kinds beginning with head hairs. How? “In this body there are head hairs … urine.”

                    \vismParagraph{VIII.46}{46}{}
                    Herein, \emph{there are }means, there are found. \emph{In this}: in this, which is expressed thus: “Up from the soles of the feet and down from the top of the hair and contained in the skin, as full of many kinds of filth.” \emph{Body}: the carcass; for it is the carcass that is called “body” (\emph{kāya}) because it is a conglomeration of filth, because such vile (\emph{kucchita}) things as the head hairs, etc., and the hundred diseases beginning with eye disease, have it as their origin (\emph{āya}).

                    \emph{Head hairs, body hairs}: these things beginning with head hairs are the thirty-two aspects. The construction here should be understood in this way: In this body there are head hairs, in this body there are body hairs.

                    \vismParagraph{VIII.47}{47}{}
                    \marginnote{\textcolor{teal}{\footnotesize\{296|238\}}}{}No one who searches throughout the whole of this fathom-long carcass, starting upwards from the soles of the feet, starting downwards from the top of the head, and starting from the skin all round, ever finds even the minutest atom at all beautiful in it, such as a pearl, or a gem, or beryl, or aloes,\footnote{\vismAssertFootnoteCounter{14}\vismHypertarget{VIII.n14}{}\emph{Agaru—}“aloes”: not so spelled in PED; but see \emph{agalu}.} or saffron, or camphor, or talcum powder; on the contrary he finds nothing but the various very malodorous, offensive, drab-looking sorts of filth consisting of the head hairs, body hairs, and the rest. Hence it is said: “In this body there are head hairs, body hairs … urine.”

                    This is the commentary on the word-construction here.
                \subsection[\vismAlignedParas{§48–144}Development]{Development}

                    \vismParagraph{VIII.48}{48}{}
                    Now, a clansman who, as a beginner, wants to develop this meditation subject should go to a good friend of the kind already described (\hyperlink{III.61}{III.61}{}–\hyperlink{III.73}{73}{}) and learn it. And the teacher who expounds it to him should tell him the sevenfold skill in learning and the tenfold skill in giving attention.
                    \subsubsection[\vismAlignedParas{§48–60}The Sevenfold Skill in Learning]{The Sevenfold Skill in Learning}

                        Herein, the sevenfold skill in learning should be told thus: (1) as verbal recitation, (2) as mental recitation, (3) as to colour, (4) as to shape, (5) as to direction, (6) as to location, (7) as to delimitation.

                        \vismParagraph{VIII.49}{49}{}
                        \emph{1.} This meditation subject consists in giving attention to repulsiveness. Even if one is master of the Tipiṭaka, the \emph{verbal recitation }should still be done at the time of first giving it attention. For the meditation subject only becomes evident to some through recitation, as it did to the two elders who learned the meditation subject from the Elder Mahā Deva of the Hill Country (Malaya). On being asked for the meditation subject, it seems, the elder \textcolor{brown}{\textit{[242]}} gave the text of the thirty-two aspects, saying, “Do only this recitation for four months.” Although they were familiar respectively with two and three Piṭakas, it was only at the end of four months of recitation of the meditation subject that they became stream-enterers, with right apprehension [of the text]. So the teacher who expounds the meditation subject should tell the pupil to do the recitation verbally first.

                        \vismParagraph{VIII.50}{50}{}
                        Now, when he does the recitation, he should divide it up into the “skin pentad,” etc., and do it forwards and backwards. After saying “Head hairs, body hairs, nails, teeth, skin,” he should repeat it backwards, “Skin, teeth, nails, body hairs, head hairs.”

                        \vismParagraph{VIII.51}{51}{}
                        Next to that, with the “kidney pentad,” after saying “Flesh, sinews, bones, bone marrow, kidney,” he should repeat it backwards, “Kidney, bone marrow, bones, sinews, flesh; skin, teeth, nails, body hairs, head hairs.”

                        \vismParagraph{VIII.52}{52}{}
                        Next, with the “lungs pentad,” after saying “Heart, liver, midriff, spleen, lungs,” he should repeat it backwards, “Lungs, spleen, midriff, liver, heart; kidney, bone marrow, bones, sinews, flesh; skin, teeth, nails, body hairs, head hairs.”

                        \vismParagraph{VIII.53}{53}{}
                        \marginnote{\textcolor{teal}{\footnotesize\{297|239\}}}{}Next, with the “brain pentad,” after saying “Bowels, entrails, gorge, dung, brain,” he should repeat it backwards, “Brain, dung, gorge, entrails, bowels; lungs, spleen, midriff, liver, heart; kidney, bone marrow, bones, sinews, flesh; skin, teeth, nails, body hairs, head hairs.”

                        \vismParagraph{VIII.54}{54}{}
                        Next, with the “fat sextad,” after saying “Bile, phlegm, pus, blood, sweat, fat,” he should repeat it backwards, “Fat, sweat, blood, pus, phlegm, bile; brain, dung, gorge, entrails, bowels; lungs, spleen, midriff, liver, heart; kidney, bone marrow, bones, sinews, flesh; skin, teeth, nails, body hairs, head hairs.”

                        \vismParagraph{VIII.55}{55}{}
                        Next, with the “urine sextad,” after saying “Tears, grease, spittle, snot, oil of the joints, urine,” he should repeat it backwards, “Urine, oil of the joints, snot, spittle, grease, tears; fat, sweat, blood, pus, phlegm, bile; brain, dung, gorge, entrails, bowels; lungs, spleen, midriff, liver, heart; kidney, bone marrow, bones, sinews, flesh; skin, teeth, nails, body hairs, head hairs.” \textcolor{brown}{\textit{[243]}}

                        \vismParagraph{VIII.56}{56}{}
                        The recitation should be done verbally in this way a hundred times, a thousand times, even a hundred thousand times. For it is through verbal recitation that the meditation subject becomes familiar, and the mind being thus prevented from running here and there, the parts become evident and seem like [the fingers of] a pair of clasped hands,\footnote{\vismAssertFootnoteCounter{15}\vismHypertarget{VIII.n15}{}\emph{Hatthasaṅkhalikā—}“the fingers of a pair of clasped hands,” “a row of fingers (\emph{aṅgulīpanti}) (\textbf{\cite{Vism-mhṭ}246}).} like a row of fence posts.

                        \vismParagraph{VIII.57}{57}{}
                        \emph{2. }The \emph{mental recitation }should be done just as it is done verbally. For the verbal recitation is a condition for the mental recitation, and the mental recitation is a condition for the penetration of the characteristic [of foulness].\footnote{\vismAssertFootnoteCounter{16}\vismHypertarget{VIII.n16}{}“For the penetration of the characteristic of foulness, for the observation of repulsiveness as the individual essence” (\textbf{\cite{Vism-mhṭ}246}).}

                        \vismParagraph{VIII.58}{58}{}
                        \emph{3. As to colour}: the colour of the head hairs, etc., should be defined.

                        \emph{4. As to shape}: their shape should be defined too.

                        \emph{5. As to direction}: in this body, upwards from the navel is the upward direction, and downwards from it is the downward direction. So the direction should be defined thus: “This part is in this direction.”

                        \emph{6. As to location}: the location of this or that part should be defined thus: “This part is established in this location.”

                        \vismParagraph{VIII.59}{59}{}
                        \emph{7. As to delimitation}: there are two kinds of delimitation, that is, delimitation of the similar and delimitation of the dissimilar. Herein, delimitation of the similar should be understood in this way: “This part is delimited above and below and around by this.” Delimitation of the dissimilar should be understood as non-intermixed-ness in this way: “Head hairs are not body hairs, and body hairs are not head hairs.”

                        \vismParagraph{VIII.60}{60}{}
                        When the teacher tells the skill in learning in seven ways thus, he should do so knowing that in certain suttas this meditation subject is expounded from the point of view of repulsiveness and in certain suttas from the point of view of elements. For in the Mahā Satipaṭṭhāna Sutta (DN 22) it is expounded only as repulsiveness. In the Mahā Hatthipadopama Sutta (MN 28), in the Mahā \marginnote{\textcolor{teal}{\footnotesize\{298|240\}}}{}Rāhulovāda Sutta (MN 62), and the Dhātuvibhaṅga (MN 140, also \textbf{\cite{Vibh}82}), it is expounded as elements. In the Kāyagatāsati Sutta (MN 119), however, four jhānas are expounded with reference to one to whom it has appeared as a colour [kasiṇa] (see \hyperlink{III.107}{III.107}{}). Herein, it is an insight meditation subject that is expounded as elements and a serenity meditation subject that is expounded as repulsiveness. Consequently it is only the serenity meditation subject [that is relevant] here.
                    \subsubsection[\vismAlignedParas{§61–79}The Tenfold Skill in Giving Attention]{The Tenfold Skill in Giving Attention}

                        \vismParagraph{VIII.61}{61}{}
                        Having thus told the sevenfold skill in learning, he should tell the tenfold skill in giving attention as follows: (1) as to following the order, (2) not too quickly, (3) not too slowly (4) as to warding off distraction, (5) as to surmounting the concept, (6) as to successive leaving, (7) as to absorption, (8)–(10) as to the three suttantas.

                        \vismParagraph{VIII.62}{62}{}
                        \emph{1.} Herein, \emph{as to following the order}: from the time of beginning the recitation \textcolor{brown}{\textit{[244]}} attention should be given following the serial order without skipping. For just as when someone who has no skill climbs a thirty-two-rung ladder using every other step, his body gets exhausted and he falls without completing the climb, so too, one who gives it attention skipping [parts] becomes exhausted in his mind and does not complete the development since he fails to get the satisfaction that ought to be got with successful development.

                        \vismParagraph{VIII.63}{63}{}
                        \emph{2.} Also when he gives attention to it following the serial order, he should do so \emph{not too quickly}. For just as when a man sets out on a three-league journey, even if he has already done the journey out and back a hundred times rapidly without taking note of [turnings] to be taken and avoided, though he may finish his journey, he still has to ask how to get there, so too, when the meditator gives his attention to the meditation subject too quickly, though he may reach the end of the meditation subject, it still does not become clear or bring about any distinction. So he should not give his attention to it too quickly.

                        \vismParagraph{VIII.64}{64}{}
                        \emph{3.} And as “not too quickly,” so also \emph{not too slowly}. For just as when a man wants to do a three-league journey in one day, if he loiters on the way among trees, rocks, pools, etc., he does not finish the journey in a day and needs two or three to complete it, so too, if the meditator gives his attention to the meditation subject too slowly, he does not get to the end and it does not become a condition for distinction.

                        \vismParagraph{VIII.65}{65}{}
                        \emph{4. As to warding off distraction}: he must ward off [temptation] to drop the meditation subject and to let his mind get distracted among the variety of external objects. For if not, just as when a man has entered on a one-foot-wide cliff path, if he looks about here and there without watching his step, he may miss his footing and fall down the cliff, which is perhaps as high as a hundred men, so too, when there is outward distraction, the meditation subject gets neglected and deteriorates. So he should give his attention to it warding off distraction.

                        \vismParagraph{VIII.66}{66}{}
                        \emph{5. As to surmounting the concept}: this [name-] concept beginning with “head hairs, body hairs” must be surmounted and consciousness established on [the aspect] “repulsive.” For just as when men find a water hole in a forest in a time \marginnote{\textcolor{teal}{\footnotesize\{299|241\}}}{}of drought, they hang up some kind of signal there such as a palm leaf, and people come to bathe and drink guided by the signal, \textcolor{brown}{\textit{[245]}} but when the way has become plain with their continual traffic, there is no further need of the signal and they go to bathe and drink there whenever they want, so too, when repulsiveness becomes evident to him as he is giving his attention to the meditation subject through the means of the [name-] concept “head hairs, body hairs,” he must surmount the concept “head hairs, body hairs” and establish consciousness on only the actual repulsiveness.

                        \vismParagraph{VIII.67}{67}{}
                        \emph{6. As to successive leaving}: in giving his attention he should eventually leave out any [parts] that do not appear to him. For when a beginner gives his attention to head hairs, his attention then carries on till it arrives at the last part, that is, urine and stops there; and when he gives his attention to urine, his attention then carries on till it arrives back at the first part, that is, head hairs, and stops there. As he persists in giving his attention thus, some parts appear to him and others do not. Then he should work on those that have appeared till one out of any two appears the clearer. He should arouse absorption by again and again giving attention to the one that has appeared thus.

                        \vismParagraph{VIII.68}{68}{}
                        Here is a simile. Suppose a hunter wanted to catch a monkey that lived in a grove of thirty-two palms, and he shot an arrow through a leaf of the palm that stood at the beginning and gave a shout; then the monkey went leaping successively from palm to palm till it reached the last palm; and when the hunter went there too and did as before, it came back in like manner to the first palm; and being followed thus again and again, after leaping from each place where a shout was given, it eventually jumped on to one palm, and firmly seizing the palm shoot’s leaf spike in the middle, would not leap any more even when shot—so it is with this.

                        \vismParagraph{VIII.69}{69}{}
                        The application of the simile is this. The thirty-two parts of the body are like the thirty-two palms in the grove. The monkey is like the mind. The meditator is like the hunter. The range of the meditator’s mind in the body with its thirty-two parts as object is like the monkey’s inhabiting the palm grove of thirty-two palms. The settling down of the meditator’s mind in the last part after going successively [from part to part] when he began by giving his attention to head hairs is like the monkey’s leaping from palm to palm and going to the last palm, \textcolor{brown}{\textit{[246]}} when the hunter shot an arrow through the leaf of the palm where it was and gave a shout. Likewise in the return to the beginning. His doing the preliminary work on those parts that have appeared, leaving behind those that did not appear while, as he gave his attention to them again and again, some appeared to him and some did not, is like the monkey’s being followed and leaping up from each place where a shout is given. The meditator’s repeated attention given to the part that in the end appears the more clearly of any two that have appeared to him and his finally reaching absorption, is like the monkey’s eventually stopping in one palm, firmly seizing the palm shoot’s leaf spike in the middle and not leaping up even when shot.

                        \vismParagraph{VIII.70}{70}{}
                        There is another simile too. Suppose an alms-food-eater bhikkhu went to live near a village of thirty-two families, and when he got two lots of alms at the first house he left out one [house] beyond it, and next day, when he got three lots \marginnote{\textcolor{teal}{\footnotesize\{300|242\}}}{}of [alms at the first house] he left out two [houses] beyond it, and on the third day he got his bowl full at the first [house], and went to the sitting hall and ate—so it is with this.

                        \vismParagraph{VIII.71}{71}{}
                        The thirty-two aspects are like the village with the thirty-two families. The meditator is like the alms-food eater. The meditator’s preliminary work is like the alms-food eater’s going to live near the village. The meditator’s continuing to give attention after leaving out those parts that do not appear and doing his preliminary work on the pair of parts that do appear is like the alms-food eater’s getting two lots of alms at the first house and leaving out one [house] beyond it, and like his next day getting three [lots of alms at the first house] and leaving out two [houses] beyond it. The arousing of absorption by giving attention again and again to that which has appeared the more clearly of two is like the alms-food eater’s getting his bowl full at the first [house] on the third day and then going to the sitting hall and eating.

                        \vismParagraph{VIII.72}{72}{}
                        \emph{7. As to absorption}: as to absorption part by part. The intention here is this: it should be understood that absorption is brought about in each one of the parts.

                        \vismParagraph{VIII.73}{73}{}
                        \emph{8–10. As to the three suttantas}: the intention here is this: it should be understood that the three suttantas, namely, those on higher consciousness,\footnote{\vismAssertFootnoteCounter{17}\vismHypertarget{VIII.n17}{}“The higher consciousness” is a term for jhāna.} on coolness, and on skill in the enlightenment factors, have as their purpose the linking of energy with concentration.

                        \vismParagraph{VIII.74}{74}{}
                        \emph{8. }Herein, this sutta should be understood to deal with higher consciousness: “Bhikkhus, there are three signs that should be given attention from time to time by a bhikkhu intent on higher consciousness. The sign of concentration should be given attention from time to time, the sign of exertion should be given attention from time to time, the sign of equanimity should be given attention from time to time. \textcolor{brown}{\textit{[247]}} If a bhikkhu intent on higher consciousness gives attention only to the sign of concentration, then his consciousness may conduce to idleness. If a bhikkhu intent on higher consciousness gives attention only to the sign of exertion, then his consciousness may conduce to agitation. If a bhikkhu intent on higher consciousness gives attention only to the sign of equanimity, then his consciousness may not become rightly concentrated for the destruction of cankers. But, bhikkhus, when a bhikkhu intent on higher consciousness gives attention from time to time to the sign of concentration … to the sign of exertion … to the sign of equanimity, then his consciousness becomes malleable, wieldy and bright, it is not brittle and becomes rightly concentrated for the destruction of cankers.

                        \vismParagraph{VIII.75}{75}{}
                        “Bhikkhus, just as a skilled goldsmith or goldsmith’s apprentice prepares his furnace and heats it up and puts crude gold into it with tongs; and he blows on it from time to time, sprinkles water on it from time to time, and looks on at it from time to time; and if the goldsmith or goldsmith’s apprentice only blew on the crude gold, it would burn and if he only sprinkled water on it, it would cool down, and if he only looked on at it, it would not get rightly refined; but, when \marginnote{\textcolor{teal}{\footnotesize\{301|243\}}}{}the goldsmith or goldsmith’s apprentice blows on the crude gold from time to time, sprinkles water on it from time to time, and looks on at it from time to time, then it becomes malleable, wieldy and bright, it is not brittle, and it submits rightly to being wrought; whatever kind of ornament he wants to work it into, whether a chain or a ring or a necklace or a gold fillet, it serves his purpose.

                        \vismParagraph{VIII.76}{76}{}
                        “So too, bhikkhus, there are three signs that should be given attention from time to time by a bhikkhu intent on higher consciousness … becomes rightly concentrated for the destruction of cankers. \textcolor{brown}{\textit{[248]}} He attains the ability to be a witness, through realization by direct-knowledge, of any state realizable by direct-knowledge to which he inclines his mind, whenever there is occasion” (\textbf{\cite{A}I 256–258}).\footnote{\vismAssertFootnoteCounter{18}\vismHypertarget{VIII.n18}{}\textbf{\cite{Vism-mhṭ}} explains “\emph{sati sati āyatane}” (rendered here by “whenever there is occasion” with “\emph{tasmiṃ tasmiṃ pubbahetu-ādi-kāraṇe sati}” (“when there is this or that reason consisting in a previous cause, etc.”); \textbf{\cite{M-a}IV 146} says: “\emph{Sati sati kāraṇe. Kim pan’ ettha kāraṇan’ti. Abhiññā’ va kāraṇaṃ} (‘Whenever there is a reason. But what is the reason here? The direct-knowledge itself is the reason’).”}

                        \vismParagraph{VIII.77}{77}{}
                        \emph{9.} This sutta deals with coolness: “Bhikkhus, when a bhikkhu possesses six things, he is able to realize the supreme coolness. What six? Here, bhikkhus, when consciousness should be restrained, he restrains it; when consciousness should be exerted, he exerts it; when consciousness should be encouraged, he encourages it; when consciousness should be looked on at with equanimity, he looks on at it with equanimity. He is resolute on the superior [state to be attained], he delights in Nibbāna. Possessing these six things a bhikkhu is able to realize the supreme coolness” (\textbf{\cite{A}III 435}).

                        \vismParagraph{VIII.78}{78}{}
                        \emph{10.} Skill in the enlightenment factors has already been dealt with in the explanation of skill in absorption (\hyperlink{IV.51}{IV.51}{}, \hyperlink{IV.57}{57}{}) in the passage beginning, “Bhikkhus, when the mind is slack, that is not the time for developing the tranquillity enlightenment factor …” (\textbf{\cite{S}V 113}).

                        \vismParagraph{VIII.79}{79}{}
                        So the meditator should make sure that he has apprehended this sevenfold skill in learning well and has properly defined this tenfold skill in giving attention, thus learning the meditation subject properly with both kinds of skill.
                    \subsubsection[\vismAlignedParas{§80}Starting the Practice]{Starting the Practice}

                        \vismParagraph{VIII.80}{80}{}
                        If it is convenient for him to live in the same monastery as the teacher, then he need not get it explained in detail thus [to begin with], but as he applies himself to the meditation subject after he has made quite sure about it he can have each successive stage explained as he reaches each distinction.

                        One who wants to live elsewhere, however, must get it explained to him in detail in the way already given, and he must turn it over and over, getting all the difficulties solved. He should leave an abode of an unsuitable kind as described in the Description of the Earth Kasiṇa, and go to live in a suitable one. Then he should sever the minor impediments (\hyperlink{IV.20}{IV.20}{}) and set about the preliminary work for giving attention to repulsiveness.
                    \subsubsection[\vismAlignedParas{§81–138}The Thirty-two Aspects in Detail]{The Thirty-two Aspects in Detail}

                        \vismParagraph{VIII.81}{81}{}
                        \marginnote{\textcolor{teal}{\footnotesize\{302|244\}}}{}When he sets about it, he should first apprehend the [learning] sign in head hairs. How? The \emph{colour }should be defined first by plucking out one or two head hairs and placing them on the palm of the hand. \textcolor{brown}{\textit{[249]}} He can also look at them in the hair-cutting place, or in a bowl of water or rice gruel. If the ones he sees are black when he sees them, they should be brought to mind as “black;” if white, as “white;” if mixed, they should be brought to mind in accordance with those most prevalent. And as in the case of head hairs, so too the sign should be apprehended visually with the whole of the “skin pentad.”

                        \vismParagraph{VIII.82}{82}{}
                        Having apprehended the sign thus and (a) defined all the other \emph{parts of the body }by colour, shape, direction, location, and delimitation (\hyperlink{VIII.58}{§58}{}), he should then (b) define \emph{repulsiveness }in five ways, that is, by colour, shape, odour, habitat, and location.

                        \vismParagraph{VIII.83}{83}{}
                        Here is the explanation of all the parts given in successive order.
                        \par\noindent[\textsc{\textbf{Head Hairs}}]

                            (a) Firstly head hairs are black in their normal \emph{colour}, the colour of fresh \emph{ariṭṭhaka }seeds.\footnote{\vismAssertFootnoteCounter{19}\vismHypertarget{VIII.n19}{}\emph{Ariṭṭhaka} as a plant is not in PED; see CPD—Sinh \emph{penela uṭa}.} As to \emph{shape}, they are the shape of long round measuring rods.\footnote{\vismAssertFootnoteCounter{20}\vismHypertarget{VIII.n20}{}There are various readings.} As to \emph{direction}, they lie in the upper direction. As to \emph{location}, their location is the wet inner skin that envelops the skull; it is bounded on both sides by the roots of the ears, in front by the forehead, and behind by the nape of the neck.\footnote{\vismAssertFootnoteCounter{21}\vismHypertarget{VIII.n21}{}“\emph{Galavāṭaka},” here rendered by “nape of the neck,” which the context demands. But elsewhere (e.g. \hyperlink{IV.47}{IV.47}{}, \hyperlink{VIII.110}{VIII.110}{}) “base of the neck” seems indicated, that is, where the neck fits on to the body, or “gullet.”} As to \emph{delimitation}, they are bounded below by the surface of their own roots, which are fixed by entering to the amount of the tip of a rice grain into the inner skin that envelops the head. They are bounded above by space, and all round by each other. There are no two hairs together. This is their delimitation by the similar. Head hairs are not body hairs, and body hairs are not head hairs; being likewise not intermixed with the remaining thirty-one parts, the head hairs are a separate part. This is their delimitation by the dissimilar. Such is the definition of head hairs as to colour and so on.

                            \vismParagraph{VIII.84}{84}{}
                            (b) Their definition \emph{as to repulsiveness }in the five ways, that is, by colour, etc., is as follows. Head hairs are repulsive in colour as well as in shape, odour, habitat, and location.

                            \vismParagraph{VIII.85}{85}{}
                            For on seeing the colour of a head hair in a bowl of inviting rice gruel or cooked rice, people are disgusted and say, “This has got hairs in it. Take it away.” So they are repulsive in \emph{colour}. Also when people are eating at night, they are likewise disgusted by the mere sensation of a hair-shaped \emph{akka}-bark or \emph{makaci}-bark fibre. So they are repulsive in \emph{shape}.

                            \vismParagraph{VIII.86}{86}{}
                            And the \emph{odour }of head hairs, unless dressed with a smearing of oil, scented with flowers, etc., is most offensive. And it is still worse when they are put in the \marginnote{\textcolor{teal}{\footnotesize\{303|245\}}}{}fire. \textcolor{brown}{\textit{[250]}} Even if head hairs are not directly repulsive in colour and shape, still their odour is directly repulsive. Just as a baby’s excrement, as to its colour, is the colour of turmeric and, as to its shape, is the shape of a piece of turmeric root, and just as the bloated carcass of a black dog thrown on a rubbish heap, as to its colour, is the colour of a ripe palmyra fruit and, as to its shape, is the shape of a [mandolin-shaped] drum left face down, and its fangs are like jasmine buds, and so even if both these are not directly repulsive in colour and shape, still their odour is directly repulsive, so too, even if head hairs are not directly repulsive in colour and shape, still their odour is directly repulsive.

                            \vismParagraph{VIII.87}{87}{}
                            But just as pot herbs that grow on village sewage in a filthy place are disgusting to civilized people and unusable, so also head hairs are disgusting since they grow on the sewage of pus, blood, urine, dung, bile, phlegm, and the like. This is the repulsive aspect of the \emph{habitat}.

                            \vismParagraph{VIII.88}{88}{}
                            And these head hairs grow on the heap of the [other] thirty-one parts as fungi do on a dung-hill. And owing to the filthy place they grow in they are quite as unappetizing as vegetables growing on a charnel-ground, on a midden, etc., as lotuses or water lilies growing in drains, and so on. This is the repulsive aspect of their \emph{location}.

                            \vismParagraph{VIII.89}{89}{}
                            And as in the case of head hairs, so also the repulsiveness of all the parts should be defined (b) in the same five ways by colour, shape, odour, habitat, and location. All, however, must be defined individually (a) by colour, shape, direction, location, and delimitation, as follows.
                        \par\noindent[\textsc{\textbf{Body Hairs}}]

                            \vismParagraph{VIII.90}{90}{}
                            Herein, firstly, as to natural \emph{colour}, body, hairs are not pure black like head hairs but blackish brown. As to \emph{shape}, they are the shape of palm roots with the tips bent down. As to \emph{direction}, they lie in the two directions. As to \emph{location}, except for the locations where the head hairs are established, and for the palms of the hands and soles of the feet, they grow in most of the rest of the inner skin that envelops the body. As to \emph{delimitation}, they are bounded below by the surface of their own roots, which are fixed by entering to the extent of a \emph{likhā}\footnote{\vismAssertFootnoteCounter{22}\vismHypertarget{VIII.n22}{}A measure of length, as much as a “louse’s head.”}\emph{ into the inner skin that envelops the body, above by space, and all round by each other. T}here are no two body hairs together. This is the delimitation by the similar. But their delimitation by the dissimilar is like that for the head hairs. [Note: These two last sentences are repeated verbatim at the end of the description of each part. They are not translated in the remaining thirty parts].
                        \par\noindent[\textsc{\textbf{Nails}}]

                            \vismParagraph{VIII.91}{91}{}
                            “Nails” is the name for the twenty nail plates. They are all white as to \emph{colour}. As to \emph{shape}, they are the shape of fish scales. As to \emph{direction}: the toenails are in the lower direction; the fingernails are in the upper direction. \textcolor{brown}{\textit{[251]}} So they grow in the two directions. As to \emph{location}, they are fixed on the tips of the backs of the fingers and toes. As to \emph{delimitation}, they are bounded in the two \marginnote{\textcolor{teal}{\footnotesize\{304|246\}}}{}directions by the flesh of the ends of the fingers and toes, and inside by the flesh of the backs of the fingers and toes, and externally and at the end by space, and all round by each other. There are no two nails together …
                        \par\noindent[\textsc{\textbf{Teeth}}]

                            \vismParagraph{VIII.92}{92}{}
                            There are thirty-two tooth bones in one whose teeth are complete. They are white in \emph{colour}. As to \emph{shape}, they are of various shapes; for firstly in the lower row, the four middle teeth are the shape of pumpkin seeds set in a row in a lump of clay; that on each side of them has one root and one point and is the shape of a jasmine bud; each one after that has two roots and two points and is the shape of a wagon prop; then two each side with three roots and three points, then two each side four-rooted and four-pointed. Likewise in the upper row. As to \emph{direction}, they lie in the upper direction. As to \emph{location}, they are fixed in the jawbones. As to \emph{delimitation}, they are bounded by the surface of their own roots which are fixed in the jawbones; they are bounded above by space, and all round by each other. There are no two teeth together …
                        \par\noindent[\textsc{\textbf{Skin (Taca)}}]

                            \vismParagraph{VIII.93}{93}{}
                            The inner skin envelops the whole body. Outside it is what is called the outer cuticle, which is black, brown or yellow in colour, and when that from the whole of the body is compressed together, it amounts to only as much as a jujube-fruit kernel. But as to \emph{colour}, the skin itself is white; and its whiteness becomes evident when the outer cuticle is destroyed by contact with the flame of a fire or the impact of a blow and so on.

                            \vismParagraph{VIII.94}{94}{}
                            As to \emph{shape}, it is the shape of the body in brief. But in detail, the skin of the toes is the shape of silkworms’ cocoons; the skin of the back of the foot is the shape of shoes with uppers; the skin of the calf is the shape of a palm leaf wrapping cooked rice; the skin of the thighs is the shape of a long sack full of paddy; the skin of the buttocks is the shape of a cloth strainer full of water; the skin of the back is the shape of hide streched over a plank; the skin of the belly is the shape of the hide stretched over the body of a lute; the skin of the chest is more or less square; the skin of both arms is the shape of the hide stretched over a quiver; the skin of the backs of the hands is the shape of a razor box, or the shape of a comb case; the skin of the fingers is the shape of a key box; the skin of the neck is the shape of a collar for the throat; the skin of the face \textcolor{brown}{\textit{[252]}} is the shape of an insects’ nest full of holes; the skin of the head is the shape of a bowl bag.

                            \vismParagraph{VIII.95}{95}{}
                            The meditator who is discerning the skin should first define the inner skin that covers the face, working his knowledge over the face beginning with the upper lip. Next, the inner skin of the frontal bone. Next, he should define the inner skin of the head, separating, as it were, the inner skin’s connection with the bone by inserting his knowledge in between the cranium bone and the inner skin of the head, as he might his hand in between the bag and the bowl put in the bag. Next, the inner skin of the shoulders. Next, the inner skin of the right arm forwards and backwards; and then in the same way the inner skin of the left \marginnote{\textcolor{teal}{\footnotesize\{305|247\}}}{}arm. Next, after defining the inner skin of the back, he should define the inner skin of the right leg forwards and backwards; then the inner skin of the left leg in the same way. Next, the inner skin of the groin, the paunch, the bosom and the neck should be successively defined. Then, after defining the inner skin of the lower jaw next after that of the neck, he should finish on arriving at the lower lip. When he discerns it in the gross in this way, it becomes evident to him more subtly too.

                            \vismParagraph{VIII.96}{96}{}
                            As to \emph{direction}, it lies in both directions. As to \emph{location}, it covers the whole body. As to \emph{delimitation}, it is bounded below by its fixed surface, and above by space …
                        \par\noindent[\textsc{\textbf{Flesh}}]

                            \vismParagraph{VIII.97}{97}{}
                            There are nine hundred pieces of flesh. As to \emph{colour}, it is all red, like \emph{kiṃsuka }flowers. As to \emph{shape}, the flesh of the calves is the shape of cooked rice in a palm-leaf bag. The flesh of the thighs is the shape of a rolling pin.\footnote{\vismAssertFootnoteCounter{23}\vismHypertarget{VIII.n23}{}\emph{Nisadapota—“rolling pin}”: (= \emph{silā-puttaka—}\textbf{\cite{Vism-mhṭ}250}) What is meant is probably the stone roller, thicker in the middle than at the ends, with which curry spices, etc., are normally rolled by hand on a small stone slab in Sri Lanka today.} The flesh of the buttocks is the shape of the end of an oven. The flesh of the back is the shape of a slab of palm sugar. The flesh between each two ribs is the shape of clay mortar squeezed thin in a flattened opening. The flesh of the breast is the shape of a lump of clay made into a ball and flung down. The flesh of the two upper arms is the shape of a large skinned rat and twice the size. When he discerns it grossly in this way, it becomes evident to him subtly too.

                            \vismParagraph{VIII.98}{98}{}
                            As to \emph{direction}, it lies in both directions. As to \emph{location}, it is plastered over the three hundred and odd bones. \textcolor{brown}{\textit{[253]}} As to \emph{delimitation}, it is bounded below by its surface, which is fixed on to the collection of bones, and above by the skin, and all round each by each other piece …
                        \par\noindent[\textsc{\textbf{Sinews}}]

                            \vismParagraph{VIII.99}{99}{}
                            There are nine hundred sinews. As to \emph{colour}, all the sinews are white. As to \emph{shape}, they have various shapes. For five of great sinews that bind the body together start out from the upper part of the neck and descend by the front, and five more by the back, and then five by the right and five by the left. And of those that bind the right hand, five descend by the front of the hand and five by the back; likewise those that bind the left hand. And of those that bind the right foot, five descend by the front and five by the back; likewise those that bind the left foot. So there are sixty great sinews called “body supporters” which descend [from the neck] and bind the body together; and they are also called “tendons.” They are all the shape of yam shoots. But there are others scattered over various parts of the body, which are finer than the last-named. They are the shape of strings and cords. There are others still finer, the shape of creepers. Others still finer are the shape of large lute strings. Yet others are the shape of coarse thread. The sinews in the backs of the hands and feet are the shape of a bird’s claw. The sinews in the head are the shape of children’s head nets. The sinews in the back are the shape of a \marginnote{\textcolor{teal}{\footnotesize\{306|248\}}}{}wet net spread out in the sun. The rest of the sinews, following the various limbs, are the shape of a net jacket fitted to the body.

                            \vismParagraph{VIII.100}{100}{}
                            As to \emph{direction}, they lie in the two directions. As to \emph{location}, they are to be found binding the bones of the whole body together. As to \emph{delimitation}, they are bounded below by their surface, which is fixed on to the three hundred bones, and above by the portions that are in contact with the flesh and the inner skin, and all round by each other …
                        \par\noindent[\textsc{\textbf{Bones}}]

                            \vismParagraph{VIII.101}{101}{}
                            Excepting the thirty-two teeth bones, these consist of the remaining sixty-four hand bones, sixty-four foot bones, sixty-four soft bones dependent on the flesh, two heel bones; then in each leg two ankle bones, two shin bones, one knee bone and one thigh bone; then two hip bones, eighteen spine bones, \textcolor{brown}{\textit{[254]}} twenty-four rib bones, fourteen breast bones, one heart bone (sternum), two collar bones, two shoulder blade bones,\footnote{\vismAssertFootnoteCounter{24}\vismHypertarget{VIII.n24}{}\emph{Koṭṭhaṭṭhīni—}“shoulder-blade bones”: for\emph{ koṭṭha} (= flat) cf. \emph{koṭṭhalika} §97; the meaning is demanded by the context, otherwise no mention would be made of these two bones, and the description fits. PED under this ref. has “stomach bone” (?). Should one read \emph{a-tikhiṇa} (blunt) or \emph{ati-khiṇa} (very sharp)?} two upper-arm bones, two pairs of forearm bones, two neck bones, two jaw bones, one nose bone, two eye bones, two ear bones, one frontal bone, one occipital bone, nine sincipital bones. So there are exactly three hundred bones. As to \emph{colour}, they are all white. As to \emph{shape}, they are of various shapes.

                            \vismParagraph{VIII.102}{102}{}
                            Herein, the end bones of the toes are the shape of \emph{kataka }seeds. Those next to them in the middle sections are the shape of jackfruit seeds. The bones of the base sections are the shape of small drums. The bones of the back of the foot are the shape of a bunch of bruised yarns. The heel bone is the shape of the seed of a single-stone palmyra fruit.

                            \vismParagraph{VIII.103}{103}{}
                            The ankle bones are the shape of [two] play balls bound together. The shin bones, in the place where they rest on the ankle bones, are the shape of a \emph{sindi }shoot without the skin removed. The small shin bone is the shape of a[toy] bow stick. The large one is the shape of a shrivelled snake’s back. The knee bone is the shape of a lump of froth melted on one side. Herein, the place where the shin bone rests on it is the shape of a blunt cow’s horn. The thigh bone is the shape of a badly-pared\footnote{\vismAssertFootnoteCounter{25}\vismHypertarget{VIII.n25}{}\emph{Duttacchita—}“badly pared”: \emph{tacchita}, pp. of \emph{tacchati} to pare (e.g. with an adze); not in PED; see \textbf{\cite{M}I 31},124; III 166.} handle for an axe or hatchet. The place where it fits into the hip bone is the shape of a play ball. The place in the hip bone where it is set is the shape of a big \emph{punnāga }fruit with the end cut off.

                            \vismParagraph{VIII.104}{104}{}
                            The two hip bones, when fastened together, are the shape of the ring-fastening of a smith’s hammer. The buttock bone on the end [of them] is the shape of an inverted snake’s hood. It is perforated in seven or eight places. The spine bones are internally the shape of lead-sheet pipes put one on top of the other; externally they are the shape of a string of beads. They have two or three rows of projections next to each other like the teeth of a saw.

                            \vismParagraph{VIII.105}{105}{}
                            \marginnote{\textcolor{teal}{\footnotesize\{307|249\}}}{}Of the twenty-four rib bones, the incomplete ones are the shape of incomplete sabres, \textcolor{brown}{\textit{[255]}} and the complete ones are the shape of complete sabres; all together they are like the outspread wings of a white cock. The fourteen breast bones are the shape of an old chariot frame.\footnote{\vismAssertFootnoteCounter{26}\vismHypertarget{VIII.n26}{}\emph{Pañjara—}“frame”: not quite in this sense in PED.} The heart bone (sternum) is the shape of the bowl of a spoon. The collar bones are the shape of small metal knife handles. The shoulder-blade bones are the shape of a Sinhalese hoe worn down on one side.

                            \vismParagraph{VIII.106}{106}{}
                            The upper-arm bones are the shape of looking glass handles. The forearm bones are the shape of a twin palm’s trunks. The wrist bones are the shape of lead-sheet pipes stuck together. The bones of the back of the hand are the shape of a bundle of bruised yams. As to the fingers, the bones of the base sections are the shape of small drums; those of the middle sections are the shape of immature jackfruit seeds; those of the end sections are the shape of \emph{kataka }seeds.

                            \vismParagraph{VIII.107}{107}{}
                            The seven neck bones are the shape of rings of bamboo stem threaded one after the other on a stick. The lower jawbone is the shape of a smith’s iron hammer ring-fastening. The upper one is the shape of a knife for scraping [the rind off sugarcanes]. The bones of the eye sockets and nostril sockets are the shape of young palmyra seeds with the kernels removed. The frontal bone is the shape of an inverted bowl made of a shell. The bones of the ear-holes are the shape of barbers’ razor boxes. The bone in the place where a cloth is tied [round the head] above the frontal bone and the ear holes is the shape of a piece of curled-up toffee flake.\footnote{\vismAssertFootnoteCounter{27}\vismHypertarget{VIII.n27}{}\emph{Saṅkuṭitaghaṭapuṇṇapaṭalakhaṇḍa—}“a piece of curled-up toffee flake.” The Sinhalese translation suggests the following readings and resolution: \emph{saṅkuthita} (thickened or boiled down (?), rather than \emph{saṅkuṭita}, curled up); \emph{ghata-puṇṇa} ([toffee?] “full of ghee”); \emph{paṭala} (flake or slab); \emph{khaṇḍa} (piece).} The occipital bone is the shape of a lopsided coconut with a hole cut in the end. The sincipital bones are the shape of a dish made of an old gourd held together with stitches.

                            \vismParagraph{VIII.108}{108}{}
                            As to \emph{direction}, they lie in both directions. As to \emph{location}, they are to be found indiscriminately throughout the whole body. But in particular here, the head bones rest on the neck bones, the neck bones on the spine bones, the spine bones on the hip bones, the hip bones on the thigh bones, the thigh bones on the knee bones, the knee bones on the shin bones, the shin bones on the ankle bones, the ankle bones on the bones of the back of the foot. As to \emph{delimitation}, they are bounded inside by the bone marrow, above by the flesh, at the ends and at the roots by each other …
                        \par\noindent[\textsc{\textbf{Bone Marrow}}]

                            \vismParagraph{VIII.109}{109}{}
                            This is the marrow inside the various bones. As to \emph{colour}, it is white. As to \emph{shape}, \textcolor{brown}{\textit{[256]}} that inside each large bone is the shape of a large cane shoot moistened and inserted into a bamboo tube. That inside each small bone is the shape of a slender cane shoot moistened and inserted in a section of bamboo twig. As to \emph{direction}, it lies in both directions. As to \emph{location}, it is set inside the bones. As to \emph{delimitation}, it is delimited by the inner surface of the bones …
                        \par\noindent[\textsc{\textbf{Kidney}}]

                            \vismParagraph{VIII.110}{110}{}
                            \marginnote{\textcolor{teal}{\footnotesize\{308|250\}}}{}This is two pieces of flesh with a single ligature. As to \emph{colour}, it is dull red, the colour of \emph{pālibhaddaka }(coral tree) seeds. As to \emph{shape}, it is the shape of a pair of child’s play balls; or it is the shape of a pair of mango fruits attached to a single stalk. As to \emph{direction}, it lies in the upper direction. As to \emph{location}, it is to be found on either side of the heart flesh, being fastened by a stout sinew that starts out with one root from the base of the neck and divides into two after going a short way. As to \emph{delimitation}, the kidney is bounded by what appertains to kidney …
                        \par\noindent[\textsc{\textbf{Heart}}]

                            \vismParagraph{VIII.111}{111}{}
                            This is the heart flesh. As to \emph{colour}, it is the colour of the back of a red-lotus petal. As to \emph{shape}, it is the shape of a lotus bud with the outer petals removed and turned upside down; it is smooth outside, and inside it is like the interior of a \emph{kosātakī }(loofah gourd). In those who possess understanding it is a little expanded; in those without understanding it is still only a bud. Inside it there is a hollow the size of a \emph{punnāga }seed’s bed where half a \emph{pasata }measure of blood is kept, with which as their support the mind element and mind-consciousness element occur.

                            \vismParagraph{VIII.112}{112}{}
                            That in one of greedy temperament is red; that in one of hating temperament is black; that in one of deluded temperament is like water that meat has been washed in; that in one of speculative temperament is like lentil soup in colour; that in one of faithful temperament is the colour of [yellow] \emph{kanikāra }flowers; that in one of understanding temperament is limpid, clear, unturbid, bright, pure, like a washed gem of pure water, and it seems to shine.

                            \vismParagraph{VIII.113}{113}{}
                            As to \emph{direction}, it lies in the upper direction. As to \emph{location}, it is to be found in the middle between the two breasts, inside the body. As to \emph{delimitation}, it is bounded by what appertains to heart … \textcolor{brown}{\textit{[257]}}
                        \par\noindent[\textsc{\textbf{Liver}}]

                            \vismParagraph{VIII.114}{114}{}
                            This is a twin slab of flesh. As to \emph{colour}, it is a brownish shade of red, the colour of the not-too-red backs of white water-lily petals. As to \emph{shape}, with its single root and twin ends, it is the shape of a \emph{koviḷāra} leaf. In sluggish people it is single and large; in those possessed of understanding there are two or three small ones. As to \emph{direction}, it lies in the upper direction. As to \emph{location}, it is to be found on the right side, inside from the two breasts. As to \emph{delimitation}, it is bounded by what appertains to liver …

                            \emph{[Midriff]}
                        \par\noindent[\textsc{\textbf{Midriff}}]

                            \vismParagraph{VIII.115}{115}{}
                            This\footnote{\vismAssertFootnoteCounter{28}\vismHypertarget{VIII.n28}{}\emph{Kilomaka}—“midriff”: the rendering is obviously quite inadequate for what is described here, but there is no appropriate English word.} is the covering of the flesh, which is of two kinds, namely, the concealed and the unconcealed. As to \emph{colour}, both kinds are white, the colour of \emph{dukūla }(muslin) rags. As to \emph{shape}, it is the shape of its location. As to \emph{direction}, the \marginnote{\textcolor{teal}{\footnotesize\{309|251\}}}{}concealed midriff lies in the upper direction, the other in both directions. As to \emph{location}, the concealed midriff is to be found concealing the heart and kidney; the unconcealed is to be found covering the flesh under the inner skin throughout the whole body. As to \emph{delimitation}, it is bounded below by the flesh, above by the inner skin, and all round by what appertains to midriff …
                        \par\noindent[\textsc{\textbf{Spleen}}]

                            \vismParagraph{VIII.116}{116}{}
                            This is the flesh of the belly’s “tongue.” As to \emph{colour}, it is blue, the colour of \emph{nigguṇḍi }flowers. As to \emph{shape}, it is seven fingers in size, without attachments, and the shape of a black calf’s tongue. As to \emph{direction}, it lies in the upper direction. As to \emph{location}, it is to be found near the upper side of the belly to the left of the heart. When it comes out through a wound a being’s life is terminated. As to \emph{delimitation}, it is bounded by what appertains to spleen …
                        \par\noindent[\textsc{\textbf{Lungs}}]

                            \vismParagraph{VIII.117}{117}{}
                            The flesh of the lungs is divided up into two or three pieces of flesh. As to \emph{colour}, it is red, the colour of not very ripe \emph{udumbara }fig fruits. As to \emph{shape}, it is the shape of an unevenly cut thick slice of cake. Inside, it is insipid and lacks nutritive essence, like a lump of chewed straw, because it is affected by the heat of the kamma-born fire [element] that springs up when there is need of something to eat and drink. As to \emph{direction}, it lies in the upper direction. As to \emph{location}, it is to be found inside the body between the two breasts, hanging above the heart \textcolor{brown}{\textit{[258]}} and liver and concealing them. As to \emph{delimitation}, it is bounded by what appertains to lungs …
                        \par\noindent[\textsc{\textbf{Bowel}}]

                            \vismParagraph{VIII.118}{118}{}
                            This is the bowel tube; it is looped\footnote{\vismAssertFootnoteCounter{29}\vismHypertarget{VIII.n29}{}\emph{Obhagga—}“looped”: not in this sense in PED; see \emph{obhañjati} (\hyperlink{XI.64}{XI.64}{} and PED).} in twenty-one places, and in a man it is thirty-two hands long, and in a woman, twenty-eight hands. As to \emph{colour}, it is white, the colour of lime [mixed] with sand. As to \emph{shape}, it is the shape of a beheaded snake coiled up and put in a trough of blood. As to \emph{direction}, it lies in the two directions. As to \emph{location}, it is fastened above at the gullet and below to the excrement passage (rectum), so it is to be found inside the body between the limits of the gullet and the excrement passage. As to \emph{delimitation}, it is bounded by what pertains to bowel …
                        \par\noindent[\textsc{\textbf{Entrails (Mesentery)}}]

                            \vismParagraph{VIII.119}{119}{}
                            This is the fastening in the places where the bowel is coiled. As to \emph{colour}, it is white, the colour of \emph{dakasītalika}\footnote{\vismAssertFootnoteCounter{30}\vismHypertarget{VIII.n30}{}\emph{Dakasītalika}: not in PED; rendered in Sinhalese translation by \emph{helmaeli} (white edible water lily).}\emph{ (white edible water lily) roots. As to shape}, it is the shape of those roots too. As to \emph{direction}, it lies in the two directions. As to \emph{location}, it is to be found inside the twenty-one coils of the bowel, like the strings \marginnote{\textcolor{teal}{\footnotesize\{310|252\}}}{}to be found inside rope-rings for wiping the feet on, sewing them together, and it fastens the bowel’s coils together so that they do not slip down in those working with hoes, axes, etc., as the marionette-strings do the marionette’s wooden [limbs] at the time of the marionette’s being pulled along. As to \emph{delimitation}, it is bounded by what appertains to entrails …
                        \par\noindent[\textsc{\textbf{Gorge}}]

                            \vismParagraph{VIII.120}{120}{}
                            This is what has been eaten, drunk, chewed and tasted, and is present in the stomach. As to \emph{colour}, it is the colour of swallowed food. As to \emph{shape}, it is the shape of rice loosely tied in a cloth strainer. As to \emph{direction}, it is in the upper direction. As to \emph{location}, it is in the stomach.

                            \vismParagraph{VIII.121}{121}{}
                            What is called the “stomach” is [a part of] the bowel-membrane, which is like the swelling [of air] produced in the middle of a length of wet cloth when it is being [twisted and] wrung out from the two ends. It is smooth outside. Inside, it is like a balloon of cloth\footnote{\vismAssertFootnoteCounter{31}\vismHypertarget{VIII.n31}{}\emph{Maṃsaka-sambupali-veṭhana-kiliṭṭha-pāvāra-pupphaka-sadisa}: this is rendered into Sinhalese by \emph{kuṇu mas kasaḷa velu porōnā kaḍek pup }(“an inflated piece (or bag) of cloth, which has wrapped rotten meat refuse”). In PED \emph{pāvāra} is given as “cloak, mantle” and (this ref.) as “the mango tree”; but there seems to be no authority for the rendering “mango tree,” which has nothing to do with this context. \emph{Pupphaka} (balloon) is not in PED (cf. common Burmese spelling of \emph{bubbuḷa} (bubble) as \emph{pupphuḷa}).} soiled by wrapping up meat refuse; or it can be said to be like the inside of the skin of a rotten jack fruit. It is the place where worms dwell seething in tangles: the thirty-two families of worms, such as round worms, boil-producing worms, “palm-splinter” worms, needle-mouthed worms, tape-worms, thread worms, and the rest.\footnote{\vismAssertFootnoteCounter{32}\vismHypertarget{VIII.n32}{}It would be a mistake to take the renderings of these worms’ names too literally. \emph{Gaṇḍuppada} (boil-producing worm?) appears only as “earth worm” in PED, which will not do here. The more generally accepted reading seems to take \emph{paṭatantuka} and \emph{suttaka} (tape-worm and thread-worm) as two kinds rather than \emph{paṭatantusuttaka}; neither is in PED.} When there is no food and drink, \textcolor{brown}{\textit{[259]}} etc., present, they leap up shrieking and pounce upon the heart’s flesh; and when food and drink, etc., are swallowed, they wait with uplifted mouths and scramble to snatch the first two or three lumps swallowed. It is these worms’ maternity home, privy, hospital and charnel ground. Just as when it has rained heavily in a time of drought and what has been carried by the water into the cesspit at the gate of an outcaste village—the various kinds of ordure\footnote{\vismAssertFootnoteCounter{33}\vismHypertarget{VIII.n33}{}\emph{Kuṇapa—}“ordure”; PED only gives the meaning “corpse,” which does not fit the meaning either here or, e.g., at \hyperlink{XI.21}{XI.21}{}, where the sense of a dead body is inappropriate.} such as urine, excrement, bits of hide and bones and sinews, as well as spittle, snot, blood, etc.—gets mixed up with the mud and water already collected there; and after two or three days the families of worms appear, and it ferments, warmed by the energy of the sun’s heat, frothing and bubbling on the top, quite black in colour, and so utterly stinking and loathsome that one can scarcely go near it or look at it, much less smell or taste it, so too, [the stomach is where] the assortment of food, drink, etc., falls after being pounded up by the tongue and stuck together with spittle and \marginnote{\textcolor{teal}{\footnotesize\{311|253\}}}{}saliva, losing at that moment its virtues of colour, smell, taste, etc., and taking on the appearance of weavers’ paste and dogs’ vomit, then to get soused in the bile and phlegm and wind that have collected there, where it ferments with the energy of the stomach-fire’s heat, seethes with the families of worms, frothing and bubbling on the top, till it turns into utterly stinking nauseating muck, even to hear about which takes away any appetite for food, drink, etc., let alone to see it with the eye of understanding. And when the food, drink, etc., fall into it, they get divided into five parts: the worms eat one part, the stomach-fire bums up another part, another part becomes urine, another part becomes excrement, and one part is turned into nourishment and sustains the blood, flesh and so on.

                            \vismParagraph{VIII.122}{122}{}
                            As to \emph{delimitation}, it is bounded by the stomach lining and by what appertains to gorge …
                        \par\noindent[\textsc{\textbf{Dung}}]

                            \vismParagraph{VIII.123}{123}{}
                            This is excrement. As to \emph{colour}, it is mostly the colour of eaten food. As to \emph{shape}, it is the shape of its location. As to \emph{direction}, it is in the lower direction. As to \emph{location}, it is to be found in the receptacle for digested food (rectum).

                            \vismParagraph{VIII.124}{124}{}
                            The receptacle for digested food is the lowest part at the end of the bowel, between the navel and the root of the spine. \textcolor{brown}{\textit{[260]}} It measures eight fingerbreadths in height and resembles a bamboo tube. Just as when rain water falls on a higher level it runs down to fill a lower level and stays there, so too, the receptacle for digested food is where any food, drink, etc., that have fallen into the receptacle for undigested food, have been continuously cooked and simmered by the stomach-fire, and have got as soft as though ground up on a stone, run down to through the cavities of the bowels, and it is pressed down there till it becomes impacted like brown clay pushed into a bamboo joint, and there it stays.

                            \vismParagraph{VIII.125}{125}{}
                            As to \emph{delimitation}, it is bounded by the receptacle for digested food and by what appertains to dung …
                        \par\noindent[\textsc{\textbf{Brain}}]

                            \vismParagraph{VIII.126}{126}{}
                            This is the lumps of marrow to be found inside the skull. As to \emph{colour}, it is white, the colour of the flesh of a toadstool; it can also be said that it is the colour of turned milk that has not yet become curd. As to \emph{shape}, it is the shape of its location. As to \emph{direction}, it belongs to the upper direction. As to \emph{location}, it is to be found inside the skull, like four lumps of dough put together to correspond with the [skull’s] four sutured sections. As to \emph{delimitation}, it is bounded by the skull’s inner surface and by what appertains to brain …
                        \par\noindent[\textsc{\textbf{Bile}}]

                            \vismParagraph{VIII.127}{127}{}
                            There are two kinds of bile: local bile and free bile. Herein as to \emph{colour}, the local bile is the colour of thick \emph{madhuka }oil; the free bile is the colour of faded \emph{ākulī }flowers. As to \emph{shape}, both are the shape of their location. As to \emph{direction}, the local bile belongs to the upper direction; the other belongs to both directions. As to \emph{location}, the free bile spreads, like a drop of oil on water, all over the body \marginnote{\textcolor{teal}{\footnotesize\{312|254\}}}{}except for the fleshless parts of the head hairs, body hairs, teeth, nails, and the hard dry skin. When it is disturbed, the eyes become yellow and twitch, and there is shivering and itching\footnote{\vismAssertFootnoteCounter{34}\vismHypertarget{VIII.n34}{}\emph{Kaṇḍūyati—}“to itch”: the verb is not in PED; see \emph{kaṇḍu}.} of the body. The local bile is situated near the flesh of the liver between the heart and the lungs. It is to be found in the bile container (gall bladder), which is like a large \emph{kosātakī }(loofah) gourd pip. When it is disturbed, beings go crazy and become demented, they throw off conscience and shame and do the undoable, speak the unspeakable, and think the unthinkable. As to \emph{delimitation}, it is bounded by what appertains to bile … \textcolor{brown}{\textit{[261]}}
                        \par\noindent[\textsc{\textbf{Phlegm}}]

                            \vismParagraph{VIII.128}{128}{}
                            The phlegm is inside the body and it measures a bowlful. As to \emph{colour}, it is white, the colour of the juice of \emph{nāgabalā }leaves. As to \emph{shape}, it is the shape of its location. As to \emph{direction}, it belongs to the upper direction. As to \emph{location}, it is to be found on the stomach’s surface. Just as duckweed and green scum on the surface of water divide when a stick or a potsherd is dropped into the water and then spread together again, so too, at the time of eating and drinking, etc., when the food, drink, etc., fall into the stomach, the phlegm divides and then spreads together again. And if it gets weak the stomach becomes utterly disgusting with a smell of ordure, like a ripe boil or a rotten hen’s egg, and then the belchings and the mouth reek with a stench like rotting ordure rising from the stomach, so that the man has to be told, “Go away, your breath smells.” But when it grows plentiful it holds the stench of ordure beneath the surface of the stomach, acting like the wooden lid of a privy. As to \emph{delimitation}, it is bounded by what appertains to phlegm …
                        \par\noindent[\textsc{\textbf{Pus}}]

                            \vismParagraph{VIII.129}{129}{}
                            Pus is produced by decaying blood. As to \emph{colour}, it is the colour of bleached leaves; but in a dead body it is the colour of stale thickened gruel. As to \emph{shape}, it is the shape of its location. As to \emph{direction}, it belongs to both directions. As to \emph{location}, however, there is no fixed location for pus where it could be found stored up. Wherever blood stagnates and goes bad in some part of the body damaged by wounds with stumps and thorns, by burns with fire, etc., or where boils, carbuncles, etc., appear, it can be found there. As to \emph{delimitation}, it is bounded by what appertains to pus …
                        \par\noindent[\textsc{\textbf{Blood}}]

                            \vismParagraph{VIII.130}{130}{}
                            There are two kinds of blood: stored blood and mobile blood. Herein, as to \emph{colour}, stored blood is the colour of cooked and thickened lac solution; mobile blood is the colour of clear lac solution. As to \emph{shape}, both are the shape of their locations. As to \emph{direction}, the stored blood belongs to the upper direction; the other belongs to both directions. As to \emph{location}, except for the fleshless parts of the head hairs, body hairs, teeth, nails, and the hard dry skin, the mobile blood permeates the whole of the clung-to (kammically-acquired)\footnote{\vismAssertFootnoteCounter{35}\vismHypertarget{VIII.n35}{}\emph{Upādiṇṇa—}“clung-to”: see \hyperlink{XIV.n23}{Ch. XIV, note 23}{}.} body by following the network of veins. The \marginnote{\textcolor{teal}{\footnotesize\{313|255\}}}{}stored blood fills the lower part of the liver’s site \textcolor{brown}{\textit{[262]}} to the extent of a bowlful, and by its splashing little by little over the heart, kidney and lungs, it keeps the kidney, heart, liver and lungs moist. For it is when it fails to moisten the kidney, heart, etc., that beings become thirsty. As to \emph{delimitation}, it is bounded by what appertains to blood …
                        \par\noindent[\textsc{\textbf{Sweat}}]

                            \vismParagraph{VIII.131}{131}{}
                            This is the water element that trickles from the pores of the body hairs, and so on. As to \emph{colour}, it is the colour of clear sesame oil. As to \emph{shape}, it is the shape of its location. As to \emph{direction}, it belongs to both directions. As to \emph{location}, there is no fixed location for sweat where it could always be found like blood. But if the body is heated by the heat of a fire, by the sun’s heat, by a change of temperature, etc., then it trickles from all the pore openings of the head hairs and body hairs, as water does from a bunch of unevenly cut lily-bud stems and lotus stalks pulled up from the water. So its shape should also be understood to correspond to the pore-openings of the head hairs and body hairs. And the meditator who discerns sweat should only give his attention to it as it is to be found filling the pore-openings of the head hairs and body hairs. As to \emph{delimitation}, it is bounded by what appertains to sweat …
                        \par\noindent[\textsc{\textbf{Fat}}]

                            \vismParagraph{VIII.132}{132}{}
                            This is a thick unguent. As to colour, it is the colour of sliced turmeric. As to shape, firstly in the body of a stout man it is the shape of turmeric-coloured \emph{dukūla} (muslin) rags placed between the inner skin and the flesh. In the body of a lean man it is the shape of turmeric-coloured \emph{dukūla} (muslin) rags placed in two or three thicknesses on the shank flesh, thigh flesh, back flesh near the spine, and belly-covering flesh. As to direction, it belongs to both directions. As to location, it permeates the whole of a stout man’s body; it is to be found on a lean man’s shank flesh, and so on. And though it was described as “unguent” above, still it is neither used as oil on the head nor as oil for the nose, etc., because of its utter disgustingness. As to delimitation, it is bounded below by the flesh, above by the inner skin, and all round by what appertains to fat …
                        \par\noindent[\textsc{\textbf{Tears}}]

                            \vismParagraph{VIII.133}{133}{}
                            These are the water element that trickles from the eye. As to \emph{colour}, they are the colour of clear sesame oil. As to \emph{shape}, they are the shape of their location. \textcolor{brown}{\textit{[263]}} As to \emph{direction}, they belong to the upper direction. As to \emph{location}, they are to be found in the eye sockets. But they are not stored in the eye sockets all the while as the bile is in the bile container. But when beings feel joy and laugh uproariously, or feel grief and weep and lament, or eat particular kinds of wrong food, or when their eyes are affected by smoke, dust, dirt, etc., then being originated by the joy, grief, wrong food, or temperature, they fill up the eye sockets or trickle out. And the meditator who discerns tears should discern them only as they are \marginnote{\textcolor{teal}{\footnotesize\{314|256\}}}{}to be found filling the eye sockets. As to \emph{delimitation}, they are bounded by what appertains to tears …
                        \par\noindent[\textsc{\textbf{Grease}}]

                            \vismParagraph{VIII.134}{134}{}
                            This is a melted unguent. As to \emph{colour}, it is the colour of coconut oil. Also it can be said to be the colour of oil sprinkled on gruel. As to \emph{shape}, it is a film the shape of a drop of unguent spread out over still water at the time of bathing. As to \emph{direction}, it belongs to both directions. As to \emph{location}, it is to be found mostly on the palms of the hands, backs of the hands, soles of the feet, backs of the feet, tip of the nose, forehead, and points of the shoulders. And it is not always to be found in the melted state in these locations, but when these parts get hot with the heat of a fire, the sun’s heat, upset of temperature or upset of elements, then it spreads here and there in those places like the film from the drop of unguent on the still water at the time of bathing. As to \emph{delimitation}, it is bounded by what appertains to grease …
                        \par\noindent[\textsc{\textbf{Spittle}}]

                            \vismParagraph{VIII.135}{135}{}
                            This is water element mixed with froth inside the mouth. As to \emph{colour}, it is white, the colour of the froth. As to \emph{shape}, it is the shape of its location, or it can be called “the shape of froth.” As to \emph{direction}, it belongs to the upper direction. As to \emph{location}, it is to be found on the tongue after it has descended from the cheeks on both sides. And it is not always to be found stored there; but when beings see particular kinds of food, or remember them, or put something hot or bitter or sharp or salty or sour into their mouths, or when their hearts are faint, or nausea arises on some account, then spittle appears and runs down from the cheeks on both sides to settle on the tongue. It is thin at the tip of the tongue, and thick at the root of the tongue. It is capable, without getting used up, of wetting unhusked rice or husked rice or anything else chewable that is put into the mouth, like the water in a pit scooped out in a river sand bank. \textcolor{brown}{\textit{[264]}} As to \emph{delimitation}, it is bounded by what appertains to spittle …
                        \par\noindent[\textsc{\textbf{Snot}}]

                            \vismParagraph{VIII.136}{136}{}
                            This is impurity that trickles out from the brain. As to \emph{colour}, it is the colour of a young palmyra kernel. As to \emph{shape}, it is the shape of its location. As to \emph{direction}, it belongs to the upper direction. As to \emph{location}, it is to be found filling the nostril cavities. And it is not always to be found stored there; but rather, just as though a man tied up curd in a lotus leaf, which he then pricked with a thorn underneath, and whey oozed out and dripped, so too, when beings weep or suffer a disturbance of elements produced by wrong food or temperature, then the brain inside the head turns into stale phlegm, and it oozes out and comes down by an opening in the palate, and it fills the nostrils and stays there or trickles out. And the meditator who discerns snot should discern it only as it is to be found filling the nostril cavities. As to \emph{delimitation}, it is bounded by what appertains to snot …
                        \par\noindent[\textsc{\textbf{Oil of the Joints}}]

                            \vismParagraph{VIII.137}{137}{}
                            \marginnote{\textcolor{teal}{\footnotesize\{315|257\}}}{}This is the slimy ordure inside the joints in the body. As to \emph{colour}, it is the colour of \emph{kaṇikāra }gum. As to \emph{shape}, it is the shape of its location. As to \emph{direction}, it belongs to both directions. As to \emph{location}, it is to be found inside the hundred and eighty joints, serving the function of lubricating the bones’ joints. If it is weak, when a man gets up or sits down, moves forward or backward, bends or stretches, then his bones creak, and he goes about making a noise like the snapping of fingers, and when he has walked only one or two leagues’ distance, his air element gets upset and his limbs pain him. But if a man has plenty of it, his bones do not creak when he gets up, sits down, etc., and even when he has walked a long distance, his air element does not get upset and his limbs do not pain him. As to \emph{delimitation}, it is bounded by what appertains to oil of the joints …
                        \par\noindent[\textsc{\textbf{Urine}}]

                            \vismParagraph{VIII.138}{138}{}
                            This is the urine solution. As to \emph{colour}, it is the colour of bean brine. As to \emph{shape}, it is the shape of water inside a water pot placed upside down. As to \emph{direction}, it belongs to the lower direction. As to \emph{location}, it is to be found inside the bladder. For the bladder sack is called the bladder. Just as when a porous pot with no mouth is put into a cesspool, \textcolor{brown}{\textit{[265]}} then the solution from the cesspool gets into the porous pot with no mouth even though no way of entry is evident, so too, while the urinary secretion from the body enters the bladder its way of entry is not evident. Its way of exit, however, is evident. And when the bladder is full of urine, beings feel the need to make water. As to \emph{delimitation}, it is delimited by the inside of the bladder and by what is similar to urine. This is the delimitation by the similar. But its delimitation by the dissimilar is like that for the head hairs (see note at end of \hyperlink{VIII.90}{§90}{}).
                    \subsubsection[\vismAlignedParas{§139–144}The Arising of Absorption]{The Arising of Absorption}

                        \vismParagraph{VIII.139}{139}{}
                        When the meditator has defined the parts beginning with the head hairs in this way by colour, shape, direction, location and delimitation (\hyperlink{VIII.58}{§58}{}), and he gives his attention in the ways beginning with “following the order, not too quickly” (\hyperlink{VIII.61}{§61}{}) to their repulsiveness in the five aspects of colour, shape, smell, habitat, and location (\hyperlink{VIII.84}{§84f.}{}), then at last he surmounts the concept (\hyperlink{VIII.66}{§66}{}). Then just as when a man with good sight is observing a garland of flowers of thirty-two colours knotted on a single string and all the flowers become evident to him simultaneously, so too, when the meditator observes this body thus, “There are in this body head hairs,” then all these things become evident to him, as it were, simultaneously. Hence it was said above in the explanation of skill in giving attention: “For when a beginner gives his attention to head hairs, his attention carries on till it arrives at the last part, that is, urine, and stops there” (\hyperlink{VIII.67}{§67}{}).

                        \vismParagraph{VIII.140}{140}{}
                        If he applies his attention externally as well when all the parts have become evident in this way, then human beings, animals, etc., as they go about are divested of their aspect of beings and appear as just assemblages of parts. And \marginnote{\textcolor{teal}{\footnotesize\{316|258\}}}{}when drink, food, etc., is being swallowed by them, it appears as though it were being put in among the assemblage of parts.

                        \vismParagraph{VIII.141}{141}{}
                        Then, as he gives his attention to them again and again as “Repulsive, repulsive,” employing the process of “successive leaving,” etc. (\hyperlink{VIII.67}{§67}{}), eventually absorption arises in him. Herein, the appearance of the head hairs, etc., as to colour, shape, direction, location, and delimitation is the learning sign; their appearance as repulsive in all aspects is the counterpart sign.

                        As he cultivates and develops that counterpart sign, absorption arises in him, but only of the first jhāna, in the same way as described under foulness as a meditation subject (\hyperlink{VI.64}{VI.64f.}{}). And it arises singly in one to whom only one part has become evident, or who has reached absorption in one part and makes no further effort about another.

                        \vismParagraph{VIII.142}{142}{}
                        But several first jhānas, according to the number of parts, are produced in one to whom several parts have become evident, or who has reached jhāna in one and also makes further effort about another. As in the case of the Elder Mallaka. \textcolor{brown}{\textit{[266]}}

                        The elder, it seems, took the Elder Abhaya, the Dīgha reciter, by the hand,\footnote{\vismAssertFootnoteCounter{36}\vismHypertarget{VIII.n36}{}Reference is sometimes made to the “hand-grasping question” (\emph{hattha-gahaka pañhā}). It may be to this; but there is another mentioned at the end of the commentary to the Dhātu-Vibhaṅga.} and after saying “Friend Abhaya, first learn this matter,” he went on: “The Elder Mallaka is an obtainer of thirty-two jhānas in the thirty-two parts. If he enters upon one by night and one by day, he goes on entering upon them for over a fortnight; but if he enters upon one each day, he goes on entering upon them for over a month.”

                        \vismParagraph{VIII.143}{143}{}
                        And although this meditation is successful in this way with the first jhāna, it is nevertheless called “mindfulness occupied with the body” because it is successful through the influence of the mindfulness of the colour, shape, and so on.

                        \vismParagraph{VIII.144}{144}{}
                        And the bhikkhu who is devoted to this mindfulness occupied with the body “is a conqueror of boredom and delight, and boredom does not conquer him; he dwells transcending boredom as it arises. He is a conqueror of fear and dread, and fear and dread do not conquer him; he dwells transcending fear and dread as they arise. He is one who bears cold and heat … who endures … arisen bodily feelings that are … menacing to life” (\textbf{\cite{M}III 97}); he becomes an obtainer of the four jhānas based on the colour aspect of the head hairs,\footnote{\vismAssertFootnoteCounter{37}\vismHypertarget{VIII.n37}{}The allusion seems to be to the bases of mastery (\emph{abhibhāyatana—}or better, bases for transcendence); see M II l3 and \textbf{\cite{M-a}III 257f.}; but see §60.} etc.; and he comes to penetrate the six kinds of direct-knowledge (see MN 6).
                        \begin{verse}
                            So let a man, if he is wise,\\{}
                            Untiringly devote his days\\{}
                            To mindfulness of body which\\{}
                            Rewards him in so many ways.
                        \end{verse}


                        \marginnote{\textcolor{teal}{\footnotesize\{317|259\}}}{}This is the section dealing with mindfulness occupied with the body in the detailed treatise.
            \section[\vismAlignedParas{§145–244}(9) Mindfulness of Breathing]{(9) Mindfulness of Breathing}

                \vismParagraph{VIII.145}{145}{}
                Now comes the description of the development of mindfulness of breathing as a meditation subject. It has been recommended by the Blessed One thus: “And, bhikkhus, this concentration through mindfulness of breathing, when developed and practiced much, is both peaceful and sublime, it is an unadulterated blissful abiding, and it banishes at once and stills evil unprofitable thoughts as soon as they arise” (\textbf{\cite{S}V 321}; \textbf{\cite{Vin}III 70}).
                \subsection[\vismAlignedParas{§145–146}Text]{Text}

                    It has been described by the Blessed One as having sixteen bases thus: “And how developed, bhikkhus, how practiced much, is concentration through mindfulness of breathing both peaceful and sublime, an unadulterated blissful abiding, banishing at once and stilling evil unprofitable thoughts as soon as they arise?

                    “Here, bhikkhus, a bhikkhu, gone to the forest or to the root of a tree or to an empty place, sits down; having folded his legs crosswise, set his body erect, established mindfulness in front of him, \textcolor{brown}{\textit{[267]}} ever mindful he breathes in, mindful he breathes out.

                    “(i) Breathing in long, he knows: ‘I breathe in long;’ or breathing out long, he knows: ‘I breathe out long.’ (ii) Breathing in short, he knows: ‘I breathe in short;’ or breathing out short, he knows: ‘I breathe out short.’ (iii) He trains thus: ‘I shall breathe in experiencing the whole body;’ he trains thus: ‘I shall breathe out experiencing the whole body.’ (iv) He trains thus: ‘I shall breathe in tranquilizing the bodily formation;’ he trains thus: ‘I shall breathe out tranquilizing the bodily formation.’

                    “(v) He trains thus: ‘I shall breathe in experiencing happiness;’ he trains thus: ‘I shall breathe out experiencing happiness.’ (vi) He trains thus: ‘I shall breathe in experiencing bliss;’ he trains thus: ‘I shall breathe out experiencing bliss.’ (vii) He trains thus: ‘I shall breathe in experiencing the mental formation;’ he trains thus: ‘I shall breathe out experiencing the mental formation.’ (viii) He trains thus: ‘I shall breathe in tranquilizing the mental formation;’ he trains thus: ‘I shall breathe out tranquilizing the mental formation.’

                    “(ix) He trains thus: ‘I shall breathe in experiencing the [manner of] consciousness;’ he trains thus: ‘I shall breathe out experiencing the [manner of] consciousness.’ (x) He trains thus: ‘I shall breathe in gladdening the [manner of] consciousness;’ he trains thus: ‘I shall breathe out gladdening the [manner of] consciousness.’ (xi) He trains thus: ‘I shall breathe in concentrating the [manner of] consciousness;’ he trains thus: ‘I shall breathe out concentrating the [manner of] consciousness.’ (xii) He trains thus: ‘I shall breathe in liberating the [manner of] consciousness;’ he trains thus: ‘I shall breathe out liberating the [manner of] consciousness.’ \marginnote{\textcolor{teal}{\footnotesize\{318|260\}}}{}“(xiii) He trains thus: ‘I shall breathe in contemplating impermanence;’ he trains thus: ‘I shall breathe out contemplating impermanence.’ (xiv) He trains thus: ‘I shall breathe in contemplating fading away;’ he trains thus: ‘I shall breathe out contemplating fading away.’ (xv) He trains thus: ‘I shall breathe in contemplating cessation;’ he trains thus: ‘I shall breathe out contemplating cessation.’ (xvi) He trains thus: ‘I shall breathe in contemplating relinquishment;’ he trains thus: ‘I shall breathe out contemplating relinquishment’ (\textbf{\cite{S}V 321–322}).

                    \vismParagraph{VIII.146}{146}{}
                    The description [of development] is complete in all respects, however, only if it is given in due course after a commentary on the text. So it is given here (\hyperlink{VIII.186}{§186}{}) introduced by a commentary on the [first part of the] text.
                \subsection[\vismAlignedParas{§146–162}Word Commentary]{Word Commentary}

                    \emph{And how developed, bhikkhus, how practiced much, is concentration through mindfulness of breathing}: here in the first place \emph{how} is a question showing desire to explain in detail the development of concentration through mindfulness of breathing in its various forms. \emph{Developed, bhikkhus, … is concentration through mindfulness of breathing}: this shows the thing that is being asked about out of desire to explain it in its various forms. \emph{How practiced much … as soon as they arise}?: here too the same explanation applies.

                    \vismParagraph{VIII.147}{147}{}
                    Herein, \emph{developed }means aroused or increased, \emph{concentration through mindfulness of breathing }(lit. “breathing-mindfulness concentration”) is either concentration associated with mindfulness that discerns breathing, or it is concentration on mindfulness of breathing. \emph{Practiced much}: practiced again and again.

                    \vismParagraph{VIII.148}{148}{}
                    \emph{Both peaceful and sublime }(\emph{santo c’ eva paṇīto ca}): it is peaceful in both ways and sublime in both ways; the two words should each be understood as governed by the word “both” (\emph{eva}). What is meant? Unlike foulness, which as a meditation subject is peaceful and sublime only by penetration, but is neither (\emph{n’ eva}) peaceful nor sublime in its object since its object [in the learning stage] is gross, and [after that] its object is repulsiveness—unlike that, this is not unpeaceful or unsublime in any way, but on the contrary it is peaceful, stilled and quiet both on account of the peacefulness of its object and on account of the peacefulness of that one of its factors called penetration. And it is sublime, something one cannot have enough of, both on account of the sublimeness of its object and on \textcolor{brown}{\textit{[268]}} account of the sublimeness of the aforesaid factor. Hence it is called “both peaceful and sublime.”

                    \vismParagraph{VIII.149}{149}{}
                    \emph{It is an unadulterated blissful abiding}: it has no adulteration, thus it is unadulterated; it is unalloyed, unmixed, particular, special. Here it is not a question of peacefulness to be reached through preliminary work [as with the kasiṇas] or through access [as with foulness, for instance]. It is peaceful and sublime in its own individual essence too starting with the very first attention given to it. But some\footnote{\vismAssertFootnoteCounter{38}\vismHypertarget{VIII.n38}{}“‘Some’ is said with reference to the inmates of the Uttara (Northern) monastery [in Anurādhapura]” (\textbf{\cite{Vism-mhṭ}256}).} say that it is “unadulterated” because it is unalloyed, \marginnote{\textcolor{teal}{\footnotesize\{319|261\}}}{}possessed of nutritive value and sweet in its individual essence too. So it should be understood to be “unadulterated” and a “blissful abiding” since it leads to the obtaining of bodily and mental bliss with every moment of absorption.

                    \vismParagraph{VIII.150}{150}{}
                    \emph{As soon as they arise}: whenever they are not suppressed. \emph{Evil}: bad. \emph{Unprofitable }(\emph{akusala}) \emph{thoughts}: thoughts produced by unskilfulness (\emph{akosalla}). \emph{It banishes at once}: it banishes, suppresses, at that very moment. \emph{Stills }(\emph{vūpasameti}): it thoroughly calms (\emph{suṭṭhu upasameti}); or else, when eventually brought to fulfilment by the noble path, it cuts off, because of partaking of penetration; it tranquilizes, is what is meant.

                    \vismParagraph{VIII.151}{151}{}
                    In brief, however, the meaning here is this: “Bhikkhus, in what way, in what manner, by what system, is concentration through mindfulness of breathing developed, in what way is it practiced much, that it is both peaceful … as soon as they arise?”

                    \vismParagraph{VIII.152}{152}{}
                    He now said, “Here, bhikkhus,” etc., giving the meaning of that in detail.

                    Herein, \emph{here, bhikkhus, a bhikkhu }means: bhikkhus, in this dispensation a bhikkhu. For this word \emph{here }signifies the [Buddha’s] dispensation as the prerequisite for a person to produce concentration through mindfulness of breathing in all its modes,\footnote{\vismAssertFootnoteCounter{39}\vismHypertarget{VIII.n39}{}“The words ‘in all its aspects’ refer to the sixteen bases; for these are only found in total in this dispensation. When outsiders know mindfulness of breathing they only know the first four modes” (\textbf{\cite{Vism-mhṭ}257}).} and it denies that such a state exists in any other dispensation. For this is said: “Bhikkhus, only here is there an ascetic, here a second ascetic, here a third ascetic, here a fourth ascetic; other dispensations are devoid of ascetics” (\textbf{\cite{M}I 63}; \textbf{\cite{A}II 238}).\footnote{\vismAssertFootnoteCounter{40}\vismHypertarget{VIII.n40}{}“‘The ascetic’ is a stream-enterer, the ‘second ascetic’ is a once-returner, the ‘third ascetic’ is a non-returner, the ‘fourth ascetic’ is an Arahant” (\textbf{\cite{M-a}II 4}).} That is why it was said above “in this dispensation a bhikkhu.”

                    \vismParagraph{VIII.153}{153}{}
                    \emph{Gone to the forest … or to an empty place}: this signifies that he has found an abode favourable to the development of concentration through mindfulness of breathing. For this bhikkhu’s mind has long been dissipated among visible data, etc., as its object, and it does not want to mount the object of concentration-through-mindfulness-of-breathing; it runs off the track like a chariot harnessed to a wild ox.\footnote{\vismAssertFootnoteCounter{41}\vismHypertarget{VIII.n41}{}\emph{Kūṭa—}“wild”: PED, this ref. gives “useless,” which misses the point. Cf. \textbf{\cite{M-a}II 82}; IV 198.} Now, suppose a cowherd \textcolor{brown}{\textit{[269]}} wanted to tame a wild calf that had been reared on a wild cow’s milk, he would take it away from the cow and tie it up apart with a rope to a stout post dug into the ground; then the calf might dash to and fro, but being unable to get away, it would eventually sit down or lie down by the post. So too, when a bhikkhu wants to tame his own mind which has long been spoilt by being reared on visible data, etc., as object for its food and drink, he should take it away from visible data, etc., as object and bring it into the forest or to the root of a tree or to an empty place and tie it up there to the post of in-breaths and out-breaths with the rope of mindfulness. And so his mind may then dash to and fro when it no longer gets the objects it was formerly \marginnote{\textcolor{teal}{\footnotesize\{320|262\}}}{}used to, but being unable to break the rope of mindfulness and get away, it sits down, lies down, by that object under the influence of access and absorption. Hence the Ancients said:

                    \vismParagraph{VIII.154}{154}{}
                    
                    \begin{verse}
                        “Just as a man who tames a calf\\{}
                        Would tie it to a post, so here\\{}
                        Should his own mind by mindfulness\\{}
                        Be firmly to the object tied.”
                    \end{verse}


                    This is how an abode is favourable to his development. Hence it was said above: “This signifies that he has found an abode favourable to the development of concentration through mindfulness of breathing.”

                    \vismParagraph{VIII.155}{155}{}
                    Or alternatively, this mindfulness of breathing as a meditation subject—which is foremost among the various meditation subjects of all Buddhas, [some] Paccekabuddhas and [some] Buddhas’ disciples as a basis for attaining distinction and abiding in bliss here and now—is not easy to develop without leaving the neighbourhood of villages, which resound with the noises of women, men, elephants, horses, etc., noise being a thorn to jhāna (see \textbf{\cite{A}V 135}), whereas in the forest away from a village a meditator can at his ease set about discerning this meditation subject and achieve the fourth jhāna in mindfulness of breathing; and then, by making that same jhāna the basis for comprehension of formations [with insight] (\hyperlink{XX.2}{XX.2f.}{}), he can reach Arahantship, the highest fruit. That is why the Blessed One said “gone to the forest,” etc., in pointing out a favourable abode for him.

                    \vismParagraph{VIII.156}{156}{}
                    For the Blessed One is like a master of the art of building sites (see \textbf{\cite{D}I 9}, 12; II 87). \textcolor{brown}{\textit{[270]}} As the master of the art of building sites surveys the proposed site for a town, thoroughly examines it, and then gives his directions, “Build the town here,” and when the town is safely finished, he receives great honour from the royal family, so the Blessed One examines an abode as to its suitability for the meditator, and he directs, “Devote yourself to the meditation subject here,” and later on, when the meditator has devoted himself to the meditation subject and has reached Arahantship and says, “The Blessed One is indeed fully enlightened,” the Blessed One receives great honour.

                    \vismParagraph{VIII.157}{157}{}
                    And this bhikkhu is compared to a leopard. For just as a great leopard king lurks in a grass wilderness or a jungle wilderness or a rock wilderness in the forest and seizes wild beasts—the wild buffalo, wild ox, boar, etc.—so too, the bhikkhu who devotes himself to his meditation subject in the forest, etc., should be understood to seize successively the paths of stream-entry, once-return, non-return, and Arahantship; and the noble fruitions as well. Hence the Ancients said:
                    \begin{verse}
                        “For as the leopard by his lurking [in the forest] seizes beasts\\{}
                        So also will this Buddhas’ son, with insight gifted, strenuous,\\{}
                        By his retreating to the forest seize the highest fruit of all” (\textbf{\cite{Mil}369}).
                    \end{verse}


                    \marginnote{\textcolor{teal}{\footnotesize\{321|263\}}}{}So the Blessed One said “gone to the forest,” etc., to point out a forest abode as a place likely to hasten his advancement.

                    \vismParagraph{VIII.158}{158}{}
                    Herein, \emph{gone to the forest }is gone to any kind of forest possessing the bliss of seclusion among the kinds of forests characterized thus: “Having gone out beyond the boundary post, all that is forest” (\textbf{\cite{Paṭis}I 176}; Vibh 251), and “A forest abode is five hundred bow lengths distant” (\textbf{\cite{Vin}IV 183}). \emph{To the root of a tree}: gone to the vicinity of a tree. \emph{To an empty place}: gone to an empty, secluded space. And here he can be said to have gone to an “empty place” if he has gone to any of the remaining seven kinds of abode (resting place).\footnote{\vismAssertFootnoteCounter{42}\vismHypertarget{VIII.n42}{}The nine kinds of abode (resting place) are the forest and the root of a tree already mentioned, and a rock, a hill cleft, a mountain cave, a charnel ground, a jungle thicket, an open space, a heap of straw (\textbf{\cite{M}I 181}).} \textcolor{brown}{\textit{[271]}}

                    \vismParagraph{VIII.159}{159}{}
                    Having thus indicated an abode that is suitable to the three seasons, suitable to humour and temperament,\footnote{\vismAssertFootnoteCounter{43}\vismHypertarget{VIII.n43}{}“In the hot season the forest is favourable, in the cold season the root of a tree, in the rainy season an empty place. For one of phlegmatic humour, phlegmatic by nature, the forest is favourable, for one of bilious humour the root of a tree, for one of windy humour an empty place. For one of deluded temperament the forest, for one of hating temperament the root of a tree, for one of greedy temperament an empty place” (\textbf{\cite{Vism-mhṭ}258}).} and favourable to the development of mindfulness of breathing, he then said sits down, etc., indicating a posture that is peaceful and tends neither to idleness nor to agitation. Then he said \emph{having folded his legs crosswise}, etc., to show firmness in the sitting position, easy occurrence of the in-breaths and out-breaths, and the means for discerning the object.

                    \vismParagraph{VIII.160}{160}{}
                    Herein, \emph{crosswise }is the sitting position with the thighs fully locked. \emph{Folded}: having locked. \emph{Set his body erect}: having placed the upper part of the body erect with the eighteen backbones resting end to end. For when he is seated like this, his skin, flesh and sinews are not twisted, and so the feelings that would arise moment by moment if they were twisted do not arise. That being so, his mind becomes unified, and the meditation subject, instead of collapsing, attains to growth and increase.

                    \vismParagraph{VIII.161}{161}{}
                    \emph{Established mindfulness in front of him }(\emph{parimukhaṃ satiṃ upaṭṭhapetvā}) = having placed (\emph{ṭhapayitvā}) mindfulness (\emph{satiṃ}) facing the meditation subject (\emph{kammaṭṭhānābhimukhaṃ}). Or alternatively, the meaning can be treated here too according to the method of explanation given in the Paṭisambhidā, which is this: \emph{Pari }has the sense of control (\emph{pariggaha}), \emph{mukhaṃ }(lit. mouth) has the sense of outlet (\emph{niyyāna}), \emph{sati }has the sense of establishment (\emph{upaṭṭhāna}); that is why \emph{parimukhaṃ satiṃ }(‘mindfulness as a controlled outlet’) is said” (\textbf{\cite{Paṭis}I 176}). The meaning of it in brief is: Having made mindfulness the outlet (from opposition, forgetfulness being thereby] controlled.\footnote{\vismAssertFootnoteCounter{44}\vismHypertarget{VIII.n44}{}The amplification is from \textbf{\cite{Vism-mhṭ}258}.}

                    \vismParagraph{VIII.162}{162}{}
                    \emph{Ever mindful he breathes in, mindful he breathes out}: having seated himself thus, having established mindfulness thus, the bhikkhu does not abandon that mindfulness; ever mindful he breathes in, mindful he breathes out; he is a mindful worker, is what is meant.
                \subsection[\vismAlignedParas{§163–225}Word Commentary Continued—First Tetrad]{Word Commentary Continued—First Tetrad}

                    \vismParagraph{VIII.163}{163}{}
                    \marginnote{\textcolor{teal}{\footnotesize\{322|264\}}}{}(i) \emph{Now, breathing in long}, etc., is said in order to show the different ways in which he is a mindful worker. For in the Paṭisambhidā, in the exposition of the clause, “Ever mindful he breathes in, mindful he breathes out,” this is said: “He is a mindful worker in thirty-two ways: (1) when he knows unification of mind and non-distraction by means of a long in-breath, mindfulness is established in him; owing to that mindfulness and that knowledge he is a mindful worker. (2) When he knows unification of mind and non-distraction by means of a long out-breath … (31) by means of breathing in contemplating relinquishment … (32) When he knows unification of mind and non-distraction by means of breathing out contemplating relinquishment, mindfulness is established in him; owing to that mindfulness and that knowledge he is a mindful worker” (\textbf{\cite{Paṭis}I 176}).

                    \vismParagraph{VIII.164}{164}{}
                    Herein, \emph{breathing in long }(\emph{assasanto}) is producing a long in-breath. \textcolor{brown}{\textit{[272]}} “\emph{Assāsa }is the wind issuing out; \emph{passāsa }is the wind entering in” is said in the Vinaya Commentary. But in the Suttanta Commentaries it is given in the opposite sense. Herein, when any infant comes out from the mother’s womb, first the wind from within goes out and subsequently the wind from without enters in with fine dust, strikes the palate and is extinguished [with the infant’s sneezing]. This, firstly, is how \emph{assāsa }and \emph{passāsa }should be understood.

                    \vismParagraph{VIII.165}{165}{}
                    But their length and shortness should be understood by extent (\emph{addhāna}). For just as water or sand that occupies an extent of space is called a “long water,” a “long sand,” a “short water,” a “short sand,” so in the case of elephants’ and snakes’ bodies the in-breaths and out-breaths regarded as particles\footnote{\vismAssertFootnoteCounter{45}\vismHypertarget{VIII.n45}{}“‘Regarded as particles’: as a number of groups (\emph{kalāpa})” (\textbf{\cite{Vism-mhṭ}259}). This conception of the occurrence of breaths is based on the theory of motion as “successive arisings in adjacent locations” (\emph{desantaruppatti}); see note 54 below. For “groups” see \hyperlink{XX.2}{XX.2f.}{}} slowly fill the long extent, in other words, their persons, and slowly go out again. That is why they are called “long.” They rapidly fill a short extent, in other words, the person of a dog, a hare, etc., and rapidly go out again. That is why they are called “short.”

                    \vismParagraph{VIII.166}{166}{}
                    And in the case of human beings some breathe in and breathe out long, by extent of time, as elephants, snakes, etc., do, while others breathe in and breathe out short in that way as dogs, hares, etc., do. Of these, therefore, the breaths that travel over a long extent in entering in and going out are to be understood as long in time; and the breaths that travel over a little extent in entering in and going out, as short in time.

                    \vismParagraph{VIII.167}{167}{}
                    Now, this bhikkhu knows “I breathe in, I breathe out, long” while breathing in and breathing out long in nine ways. And the development of the foundation of mindfulness consisting in contemplation of the body should be understood to be perfected in one aspect in him who knows thus, according as it is said in the Paṭisambhidā:

                    \vismParagraph{VIII.168}{168}{}
                    \marginnote{\textcolor{teal}{\footnotesize\{323|265\}}}{}“How, breathing in long, does he know: ‘I breathe in long,’ breathing out long, does he know: ‘I breathe out long?’ (1) He breathes in a long in-breath reckoned as an extent. (2) He breathes out a long out-breath reckoned as an extent. (3) He breathes in and breathes out long in-breaths and out-breaths reckoned as an extent. As he breathes in and breathes out long in-breaths and out-breaths reckoned as an extent, zeal arises.\footnote{\vismAssertFootnoteCounter{46}\vismHypertarget{VIII.n46}{}“‘Zeal arises’: additional zeal, which is profitable and has the characteristic of desire to act, arises due to the satisfaction obtained when the meditation has brought progressive improvement. ‘More subtle than before’: more subtle than before the already-described zeal arose; for the breaths occur more subtly owing to the meditation’s influence in tranquilizing the body’s distress and disturbance. ‘Gladness arises’: fresh happiness arises of the kinds classed as minor, etc., which is the gladness that accompanies the consciousness occupied with the meditation and is due to the fact that the peacefulness of the object increases with the growing subtlety of the breaths and to the fact that the meditation subject keeps to its course. ‘The mind turns away’: the mind turns away from the breaths, which have reached the point at which their manifestation needs investigating (see §177) owing to their gradually increasing subtlety. But some say (see Paṭis-a Ce, p. 351): ‘It is when the in-breaths and out-breaths have reached a subtler state owing to the influence of the meditation and the counterpart sign; for when that has arisen, the mind turns away from the normal breaths.’ ‘Equanimity is established’: when concentration, classed as access and absorption, has arisen in that counterpart sign, then, since there is no need for further interest to achieve jhāna, onlooking (equanimity) ensues, which is specific neutrality” (\textbf{\cite{Vism-mhṭ}260}).} (4) Through zeal he breathes in a long in-breath more subtle than before reckoned as an extent. (5) Through zeal he breathes out a long out-breath more subtle than before reckoned as an extent. (6) Through zeal he breathes in and breathes out long in-breaths and out-breaths more subtle than before reckoned as an extent. As, through zeal, he breathes in and breathes out long in-breaths and out-breaths more subtle than before reckoned as an extent, gladness arises. \textcolor{brown}{\textit{[273]}} (7) Through gladness he breathes in a long in-breath more subtle than before reckoned as an extent. (8) Through gladness he breathes out a long out-breath more subtle than before reckoned as an extent. (9) Through gladness he breathes in and breathes out long in-breaths and out-breaths more subtle than before reckoned as an extent. As, through gladness, he breathes in and breathes out long in-breaths and out-breaths more subtle than before reckoned as an extent, his mind turns away from the long in-breaths and out-breaths and equanimity is established.

                    “Long in-breaths and out-breaths in these nine ways are a body. The establishment (foundation)\footnote{\vismAssertFootnoteCounter{47}\vismHypertarget{VIII.n47}{}“‘In these nine ways’: that occur in the nine ways just described. ‘Long in-breaths and out-breaths are a body’: the in-breaths and out-breaths, which exist as particles though they have the aspect of length, constitute a ‘body’ in the sense of a mass. And here the sign that arises with the breaths as its support is also called ‘in-breath and out-breath.’ (cf. e.g. §206) ‘The establishment (foundation) is mindfulness’: mindfulness is called ‘establishment (foundation)—(\emph{upaṭṭhāna})’ since it approaches (\emph{upagantvā}) the object and remains (\emph{tiṭṭhati}) there. ‘The contemplation is knowledge’: contemplation of the sign by means of serenity, and contemplation of mentality-materiality by defining with insight the in-breaths and out-breaths and the body, which is their support, as materiality, and the consciousness and the states associated with it as the immaterial (mentality), are knowledge, in other words, awareness of what is actually there (has actually become). ‘The body is the establishment (foundation)’: there is that body, and mindfulness approaches it by making it its object and remains there, thus it is called ‘establishment.’ And the words ‘the body is the establishment’ include the other (the mental) kind of body too since the above-mentioned comprehension by insight is needed here too. ‘But it is not the mindfulness’: that body is not called ‘mindfulness’ [though it is called ‘the establishment’]. ‘Mindfulness is both the establishment (foundation) and the mindfulness,’ being so both in the sense of remembering (\emph{sarana}) and in the sense of establishing (\emph{upatiṭṭhana}). ‘By means of that mindfulness’: by means of that mindfulness already mentioned. ‘And that knowledge’: and the knowledge already mentioned. ‘That body’: that in-breath-and-out-breath body and that material body which is its support. ‘He contemplates (\emph{anupassati})’: he keeps re-seeing (\emph{anu anu passati}) with jhāna knowledge and with insight knowledge. ‘That is why “Development of the foundation (establishment) of mindfulness consisting in contemplation of the body as a body” is said’: in virtue of that contemplation this is said to be development of the foundation (establishment) of mindfulness consisting in contemplation of the body as a body of the kind already stated. What is meant is this: the contemplation of the body as an in-breath-and-out-breath body as stated and of the physical body that is its [material] support, which is not contemplation of permanence, etc., in a body whose individual essence is impermanent, etc.—like the contemplation of a waterless mirage as water—but which is rather contemplation of its essence as impermanent, painful, not-self, and foul, according as is appropriate, or alternatively, which is contemplation of it as a mere body only, by not contemplating it as containing anything that can be apprehended as ‘I’ or ‘mine’ or ‘woman’ or ‘man’—all this is ‘contemplation of the body.’ The mindfulness associated with that contemplation of the body, which mindfulness is itself the establishment, is the ‘establishment.’ The development, the increase, of that is the ‘development of the foundation (establishment) of mindfulness consisting in contemplation of the body.’” (\textbf{\cite{Vism-mhṭ}261})

                            The compound \emph{satipaṭṭhāna} is derived by the Paṭisambhidā from \emph{sati} (mindfulness) and \emph{upaṭṭhāna} (establishment—\textbf{\cite{Paṭis}I 182}), but in the Commentaries the resolution into \emph{sati} and \emph{paṭṭhāna} (foundation) is preferred. (\textbf{\cite{M-a}I 237–238}) In the 118th Sutta of the Majjhima Nikāya the first tetrad is called development of the first foundation of mindfulness, or contemplation of the body. (MN 10; DN 22) The object of the Paṭisambhidā passage quoted is to demonstrate this.} is mindfulness. The contemplation is knowledge. \marginnote{\textcolor{teal}{\footnotesize\{324|266\}}}{}The body is the establishment (foundation), but it is not the mindfulness. Mindfulness is both the establishment (foundation) and the mindfulness. By means of that mindfulness and that knowledge he contemplates that body. That is why ‘development of the foundation (establishment) of mindfulness consisting in contemplation of the body as a body’ (see \textbf{\cite{D}II 290}) is said” (\textbf{\cite{Paṭis}I 177}).

                    \vismParagraph{VIII.169}{169}{}
                    (ii) The same method of explanation applies also in the case of \emph{short }breaths. But there is this difference. While in the former case “a long in-breath reckoned as an extent” is said, here “a short in-breath reckoned as a little \marginnote{\textcolor{teal}{\footnotesize\{325|267\}}}{}[duration]” (\textbf{\cite{Paṭis}I 182}) is given. So it must be construed as “short” as far as the phrase “That is why ‘development of the foundation (establishment) of mindfulness consisting in contemplation of the body as a body’ is said” (\textbf{\cite{Paṭis}I 183}).

                    \vismParagraph{VIII.170}{170}{}
                    So it should be understood that it is when this bhikkhu knows in-breaths and out-breaths in these nine ways as “a [long] extent” and as “a little [duration]” that “breathing in long, he knows ‘I breathe in long;’ … breathing out short, he knows ‘I breathe out short’ is said of him. And when he knows thus:
                    \begin{verse}
                        “The long kind and the short as well,\\{}
                        The in-breath and the out-breath too,\\{}
                        Such then are the four kinds that happen\\{}
                        At the bhikkhu’s nose tip here.”
                    \end{verse}


                    \vismParagraph{VIII.171}{171}{}
                    (iii) \emph{He trains thus: “I shall breathe in … I shall breathe out experiencing the whole body}”: he trains thus: “I shall breathe in making known, making plain, the beginning, middle and end\footnote{\vismAssertFootnoteCounter{48}\vismHypertarget{VIII.n48}{}The beginning, middle and end are described in §197, and the way they should be treated is given in §199–201. What is meant is that the meditator should know what they are and be aware of them without his mindfulness leaving the tip of the nose to follow after the breaths inside the body or outside it, speculating on what becomes of them.} of the entire in-breath body. I shall breathe out making known, making plain, the beginning, middle and end of the entire out-breath body,” thus he trains. Making them known, making them plain, in this way he both breathes in and breathes out with consciousness associated with knowledge. That is why it is said, “He trains thus: ‘I shall breathe in … shall breathe out …’”

                    \vismParagraph{VIII.172}{172}{}
                    To one bhikkhu the beginning of the in-breath body or the out-breath body, distributed in particles, [that is to say, regarded as successive arisings (see note 45)] is plain, but not the middle or the end; he is only able to discern the beginning and has difficulty with the middle and the end. To another the middle is plain, not the beginning or the end; he is only able to discern the middle and has difficulty with the beginning and the end. To another the end is plain, not the beginning or the middle; he is only able to discern the end \textcolor{brown}{\textit{[274]}} and has difficulty with the beginning and the middle. To yet another all stages are plain; he is able to discern them all and has no difficulty with any of them. Pointing out that one should be like the last-mentioned bhikkhu, he said: “He trains thus: ‘I shall breathe in … shall breathe out experiencing the whole body.’”

                    \vismParagraph{VIII.173}{173}{}
                    Herein, \emph{he trains}: he strives, he endeavours in this way. Or else the restraint here in one such as this is training in the higher virtue, his consciousness is training in the higher consciousness, and his understanding is training in the higher understanding (see \textbf{\cite{Paṭis}I 184}). So he trains in, repeats, develops, repeatedly practices, these three kinds of training, on that object, by means of that mindfulness, by means of that attention. This is how the meaning should be regarded here.

                    \vismParagraph{VIII.174}{174}{}
                    \marginnote{\textcolor{teal}{\footnotesize\{326|268\}}}{}Herein, in the first part of the system (nos. i and ii)\footnote{\vismAssertFootnoteCounter{49}\vismHypertarget{VIII.n49}{}“‘In the first part of the system’: in the first part of the system of development; in the first two bases, is what is intended. Of course, arousing of knowledge must be admitted to take place here too because of the presence of awareness of the length and shortness of the breaths as they actually are (as they actually become); and it is not hard to do that, for it is merely the taking account of them as they occur. That is why it is put in the present tense here. But what follows is as hard as for a man to walk on a razor’s edge; which is why the future tense is used for the subsequent stages in order to indicate the need for exceptional prior effort” (\textbf{\cite{Vism-mhṭ}263}).} he should only breathe in and breathe out and not do anything else at all, and it is only afterwards that he should apply himself to the arousing of knowledge, and so on. Consequently the present tense is used here in the text, “He knows: ‘I breathe in’ … he knows: ‘I breathe out.’” But the future tense in the passage beginning “I shall breathe in experiencing the whole body” should be understood as used in order to show that the aspect of arousing knowledge, etc., has to be undertaken from then on.

                    \vismParagraph{VIII.175}{175}{}
                    (iv) He trains thus: “I shall breathe in … shall breathe out tranquilizing the bodily formation;” he trains thus: “I shall breathe in, shall breathe out tranquilizing, completely tranquilizing, stopping, stilling, the gross bodily formation\footnote{\vismAssertFootnoteCounter{50}\vismHypertarget{VIII.n50}{}“‘Bodily formation’: the in-breath and out-breath (see \textbf{\cite{M}I 301}). For although it is consciousness-originated, it is nevertheless called ‘bodily formation’ since its existence is bound up with the kamma-born body and it is formed with that as the means” (\textbf{\cite{Vism-mhṭ}263}).}”.

                    \vismParagraph{VIII.176}{176}{}
                    And here both the gross and subtle state and also [progressive] tranquilizing should be understood. For previously, at the time when the bhikkhu has still not discerned [the meditation subject], his body and his mind are disturbed and so they are gross. And while the grossness of the body and the mind has still not subsided the in-breaths and out-breaths are gross. They get stronger; his nostrils become inadequate, and he keeps breathing in and out through his mouth. But they become quiet and still when his body and mind have been discerned. When they are still then the in-breaths and out-breaths occur so subtly that he has to investigate whether they exist or not.

                    \vismParagraph{VIII.177}{177}{}
                    Suppose a man stands still after running, or descending from a hill, or putting down a big load from his head, then his in-breaths and out-breaths are gross, his nostrils become inadequate, and he keeps on breathing in and out through his mouth. But when he has rid himself of his fatigue and has bathed and drunk \textcolor{brown}{\textit{[275]}} and put a wet cloth on his heart, and is lying in the cool shade, then his in-breaths and out-breaths eventually occur so subtly that he has to investigate whether they exist or not; so too, previously, at the time when the bhikkhu has still not discerned, … he has to investigate whether they exist or not.

                    \vismParagraph{VIII.178}{178}{}
                    Why is that? Because previously, at the time when he has still not discerned, there is no concern in him, no reaction, no attention, no reviewing, to the effect that “I am [progressively] tranquilizing each grosser bodily formation.” But when he has discerned, there is. So his bodily formation at the time when he has \marginnote{\textcolor{teal}{\footnotesize\{327|269\}}}{}discerned is subtle in comparison with that at the time when he has not. Hence the Ancients said:
                    \begin{verse}
                        “The mind and body are disturbed,\\{}
                        And then in excess it occurs;\\{}
                        But when the body is undisturbed,\\{}
                        Then it with subtlety occurs.”
                    \end{verse}


                    \vismParagraph{VIII.179}{179}{}
                    In discerning [the meditation subject the formation] is gross, and it is subtle [by comparison] in the first-jhāna access; also it is gross in that, and subtle [by comparison] in the first jhāna; in the first jhāna and second-jhāna access it is gross, and in the second jhāna subtle; in the second jhāna and third-jhāna access it is gross, and in the third jhāna subtle; in the third jhāna and fourth-jhāna access it is gross, and in the fourth jhāna it is so exceedingly subtle that it even reaches cessation. This is the opinion of the Dīgha and Saṃyutta reciters. But the Majjhima reciters have it that it is subtler in each access than in the jhāna below too in this way: In the first jhāna it is gross, and in the second-jhāna access it is subtle [by comparison, and so on]. It is, however, the opinion of all that the bodily formation occurring before the time of discerning becomes tranquilized at the time of discerning, and the bodily formation at the time of discerning becomes tranquilized in the first-jhāna access … and the bodily formation occurring in the fourth-jhāna access becomes tranquilized in the fourth jhāna. This is the method of explanation in the case of serenity.

                    \vismParagraph{VIII.180}{180}{}
                    But in the case of insight, the bodily formation occurring at the time of not discerning is gross, and in discerning the primary elements it is [by comparison] subtle; that also is gross, and in discerning derived materiality it is subtle; that also is gross, and in discerning all materiality it is subtle; that also is gross, and in discerning the immaterial it is subtle; that also is gross, and in discerning the material and immaterial it is subtle; that also is gross, and in discerning conditions it is subtle; that also is gross, and in seeing mentality-materiality with its conditions it is subtle; that also is gross, and in insight that has the characteristics [of impermanence, etc.,] as its object it is subtle; that also is gross in weak insight while in strong insight it is subtle.

                    Herein, the tranquilizing should be understood as [the relative tranquillity] of the subsequent compared with the previous. Thus should the gross and subtle state, and the [progressive] tranquilizing, be understood here. \textcolor{brown}{\textit{[276]}}

                    \vismParagraph{VIII.181}{181}{}
                    But the meaning of this is given in the Paṭisambhidā together with the objection and clarification thus:

                    “How is it that he trains thus: ‘I shall breathe in … shall breathe out tranquilizing the bodily formation? What are the bodily formations? Long in-breaths … out-breaths [experiencing the whole body] belong to the body; these things, being bound up with the body, are bodily formations;’ he trains in tranquilizing, stopping, stilling, those bodily formations.

                    “When there are such bodily formations whereby there is bending backwards, sideways in all directions, and forwards, and perturbation, vacillation, moving and shaking of the body, he trains thus: ‘I shall breathe in tranquilizing the bodily formation;’ he trains thus: ‘I shall breathe out tranquilizing the bodily \marginnote{\textcolor{teal}{\footnotesize\{328|270\}}}{}formation.’ When there are such bodily formations whereby there is no bending backwards, sideways in all directions, and forwards, and no perturbation, vacillation, moving and shaking of the body, quietly, subtly, he trains thus: ‘I shall breathe in tranquilizing the bodily formation;’ he trains thus: ‘I shall breathe out tranquilizing the bodily formation.’

                    \vismParagraph{VIII.182}{182}{}
                    “[Objection:] So then, he trains thus: ‘I shall breathe in tranquilizing the bodily formation;’ he trains thus: ‘I shall breathe out tranquilizing the bodily formation’: that being so, there is no production of awareness of wind, and there is no production of in-breaths and out-breaths, and there is no production of mindfulness of breathing, and there is no production of concentration through mindfulness of breathing, and consequently the wise neither enter into nor emerge from that attainment.

                    \vismParagraph{VIII.183}{183}{}
                    “[Clarification:] So then, he trains thus: ‘I shall breathe in tranquilizing the bodily formation;’ he trains thus: ‘I shall breathe out tranquilizing the bodily formation’: that being so, there is production of awareness of wind, and there is production of in-breaths and out-breaths, and there is production of mindfulness of breathing, and there is production of concentration through mindfulness of breathing, and consequently the wise enter into and emerge from that attainment.

                    \vismParagraph{VIII.184}{184}{}
                    “Like what? Just as when a gong is struck. At first gross sounds occur and consciousness [occurs] because the sign of the gross sounds is well apprehended, well attended to, well observed; and when the gross sounds have ceased, then afterwards faint sounds occur and [consciousness occurs] because the sign of the faint sounds is well apprehended, well attended to, well observed; and when the faint sounds have ceased, then \textcolor{brown}{\textit{[277]}} afterwards consciousness occurs because it has the sign of the faint sounds as its object\footnote{\vismAssertFootnoteCounter{51}\vismHypertarget{VIII.n51}{}“The faint sound itself as a sign is the ‘sign of the faint sounds’; it has that as its object. What is meant? Of course, the faint sounds have ceased too then; but the sign of the sounds has been well apprehended and so consciousness occurs with the sign of fainter sounds as its object. For as from the outset he ascertains with undistracted consciousness the sign of each sound as it ceases, eventually his consciousness occurs in the end with the sign of ultra-subtle sounds too as its object” (\textbf{\cite{Vism-mhṭ}266}).}—so too, at first gross in-breaths and out-breaths occur and [consciousness does not become distracted] because the sign of the gross in-breaths and out-breaths is well apprehended, well attended to, well observed; and when the gross in-breaths and out-breaths have ceased, then afterwards faint in-breaths and out-breaths occur and [consciousness does not become distracted] because the sign of the faint in-breaths and out-breaths is well apprehended, well attended to, well observed; and when the faint in-breaths and out-breaths have ceased, then afterwards consciousness does not become distracted because it has the sign of the faint in-breaths and out-breaths as its object.

                    “That being so, there is production of awareness of wind, and there is production of in-breaths and out-breaths, and there is production of mindfulness of breathing, and there is production of concentration through mindfulness of breathing, and consequently the wise enter into and emerge from that attainment.

                    \vismParagraph{VIII.185}{185}{}
                    \marginnote{\textcolor{teal}{\footnotesize\{329|271\}}}{}“In-breaths and out-breaths tranquilizing the bodily formation are a body. The establishment (foundation) is mindfulness. The contemplation is knowledge. The body is the establishment (foundation), but it is not the mindfulness. Mindfulness is both the establishment (foundation) and the mindfulness. By means of that mindfulness and that knowledge he contemplates that body. That is why ‘development of the foundation (establishment) of mindfulness consisting in contemplation of the body as a body’ is said” (\textbf{\cite{Paṭis}I 184–186}).

                    This, in the first place, is the consecutive word commentary here on the first tetrad, which deals with contemplation of the body.
                    \subsubsection[\vismAlignedParas{§186–225}Method of Development]{Method of Development}

                        \vismParagraph{VIII.186}{186}{}
                        The first tetrad is set forth as a meditation subject for a beginner;\footnote{\vismAssertFootnoteCounter{52}\vismHypertarget{VIII.n52}{}“As a meditation subject for a beginner” is said with reference to the serenity (i.e. jhāna) meditation subject; but the insight meditation subject applies to the other tetrads too” (\textbf{\cite{Vism-mhṭ}266}).} but the other three tetrads are [respectively] set forth as the contemplations of feeling, of [the manner of] consciousness, and of mental objects, for one who has already attained jhāna in this tetrad. So if a clansman who is a beginner wants to develop this meditation subject, and through insight based on the fourth jhāna produced in breathing, to reach Arahantship together with the discriminations, he should first do all the work connected with the purification of virtue, etc., in the way already described, after which he should learn the meditation subject in five stages from a teacher of the kind already described.

                        \vismParagraph{VIII.187}{187}{}
                        Here are the five stages: learning, questioning, establishing, absorption, characteristic.

                        Herein, \emph{learning }is learning the meditation subject. \emph{Questioning }is questioning about the meditation subject. \emph{Establishing }is establishing the meditation subject. \emph{Absorption }\textcolor{brown}{\textit{[278]}} is the absorption of the meditation subject. \emph{Characteristic }is the characteristic of the meditation subject; what is meant is that it is the ascertaining of the meditation subject’s individual essence thus: “This meditation subject has such a characteristic.”

                        \vismParagraph{VIII.188}{188}{}
                        Learning the meditation subject in the five stages in this way, he neither tires himself nor worries the teacher. So in giving this meditation subject consisting in mindfulness of breathing attention, he can live either with the teacher or elsewhere in an abode of the kind already described, learning the meditation subject in the five stages thus, getting a little expounded at a time and taking a long time over reciting it. He should sever the minor impediments. After finishing the work connected with the meal and getting rid of any dizziness due to the meal, he should seat himself comfortably. Then, making sure he is not confused about even a single word of what he has learned from the teacher, he should cheer his mind by recollecting the special qualities of the Three Jewels.

                        \vismParagraph{VIII.189}{189}{}
                        \marginnote{\textcolor{teal}{\footnotesize\{330|272\}}}{}Here are the stages in giving attention to it: (1) counting, (2) connection, (3) touching, (4) fixing, (5) observing, (6) turning away, (7) purification, and (8) looking back on these.

                        Herein, \emph{counting }is just counting, \emph{connection }is carrying on, \emph{touching }is the place touched [by the breaths], \emph{fixing }is absorption, \emph{observing }is insight, \emph{turning away }is the path, \emph{purification }is fruition, \emph{looking back on these }is reviewing.
                        \par\noindent[\textsc{\textbf{Counting}}]

                            \vismParagraph{VIII.190}{190}{}
                            \emph{1.} Herein, this clansman who is a beginner should first give attention to this meditation subject by counting. And when counting, he should not stop short of five or go beyond ten or make any break in the series. By stopping short of five his thoughts get excited in the cramped space, like a herd of cattle shut in a cramped pen. By going beyond ten his thoughts take the number [rather than the breaths] for their support. By making a break in the series he wonders if the meditation subject has reached completion or not. So he should do his counting without those faults.

                            \vismParagraph{VIII.191}{191}{}
                            When counting, he should at first do it slowly [that is, late] as a grain measurer does. For a grain measurer, having filled his measure, says “One,” and empties it, and then refilling it, he goes on saying ‘”One, one” while removing any rubbish he may have noticed. And the same with “Two, two” and so on. So, taking the in-breath or the out-breath, whichever appears [most plainly], he should begin with “One, one” \textcolor{brown}{\textit{[279]}} and count up to “Ten, ten,” noting each as it occurs.

                            \vismParagraph{VIII.192}{192}{}
                            As he does his counting in this way, the in-breaths and out-breaths become evident to him as they enter in and issue out. Then he can leave off counting slowly (late), like a grain measurer, and he can count quickly [that is, early] as a cowherd does. For a skilled cowherd takes pebbles in his pocket and goes to the cow pen in the morning, whip in hand; sitting on the bar of the gate, prodding the cows in the back, he counts each one as it reaches the gate, saying “One, two,” dropping a pebble for each. And the cows of the herd, which have been spending the three watches of the night uncomfortably in the cramped space, come out quickly in parties, jostling each other as they escape. So he counts quickly (early) “Three, four, five” and so up to ten. In this way the in-breaths and out-breaths, which had already become evident to him while he counted them in the former way, now keep moving along quickly.

                            \vismParagraph{VIII.193}{193}{}
                            Then, knowing that they keep moving along quickly, not apprehending them either inside or outside [the body], but apprehending them just as they reach the [nostril] door, he can do his counting quickly (early): “One, two, three, four, five; one, two, three, four, five, six … seven … eight … nine … ten.” For as long as the meditation subject is connected with counting it is with the help of that very counting that the mind becomes unified, just as a boat in a swift current is steadied with the help of a rudder.

                            \vismParagraph{VIII.194}{194}{}
                            When he counts quickly, the meditation subject becomes apparent to him as an uninterrupted process. Then, knowing that it is proceeding uninterruptedly, he can count quickly (early) in the way just described, not discerning the wind either inside or outside [the body]. For by bringing his consciousness inside along with the incoming breath, it seems as if it were buffeted by the wind inside \marginnote{\textcolor{teal}{\footnotesize\{331|273\}}}{}or filled with fat.\footnote{\vismAssertFootnoteCounter{53}\vismHypertarget{VIII.n53}{}“‘Buffeted by wind’: if he gives much attention to the wind that has gone inside, that place seems to him as if it were buffeted by the wind, as if filled with fat” (Vism-mhṭ 268). No further explanation is given.} By taking his consciousness outside along with the outgoing breath, it gets distracted by the multiplicity of objects outside. However, his development is successful when he fixes his mindfulness on the place touched [by the breaths]. That is why it was said above: “He can count quickly (early) in the way just described, not discerning the wind either inside or outside.”

                            \vismParagraph{VIII.195}{195}{}
                            But how long is he to go on counting? Until, without counting, \textcolor{brown}{\textit{[280]}} mindfulness remains settled on the in-breaths and out-breaths as its object. For counting is simply a device for setting mindfulness on the in-breaths and out-breaths as object by cutting off the external dissipation of applied thoughts.
                        \par\noindent[\textsc{\textbf{Connection}}]

                            \vismParagraph{VIII.196}{196}{}
                            \emph{2.} Having given attention to it in this way by counting, he should now do so by \emph{connection. Connection }is the uninterrupted following of the in-breaths and out-breaths with mindfulness after counting has been given up. And that is not by following after the beginning, the middle and the end.\footnote{\vismAssertFootnoteCounter{54}\vismHypertarget{VIII.n54}{}“‘Following (\emph{anugamana})’ is occurring along with (\emph{anu anu pavattana}), going after (\emph{anugacchana}), by means of mindfulness through making the breaths the object as they occur, Hence he said, ‘And that is not by following after the beginning, middle and end.’ ‘The navel is the beginning’ because of their first arising there. For the notion of a beginning (\emph{ādi cintā}) is here in the sense of first arising, not in the sense of just arising [once only]. For they actually go on arising throughout [the whole length] from the navel to the nose-tip; and wherever they arise, there in that same place they dissolve, because there is no going (movement) of dhammas. The ordinary term ‘motion’ (\emph{gatisamaññā}) refers to successive arisings in adjacent locations (\emph{desantaruppatti}) according to conditions. ‘The heart is the middle’: near the heart, just above it is the middle. ‘The nose tip is the end’: the place where the nostrils are is the end; that is the limit of the application of the ordinary term ‘in-breaths and out-breaths,’ for it is accordingly that they are called ‘consciousness-originated,’ there being no production externally of what is consciousness-originated” (\textbf{\cite{Vism-mhṭ}268}).}

                            \vismParagraph{VIII.197}{197}{}
                            The navel is the beginning of the wind issuing out, the heart is its middle and the nose-tip is its end. The nose-tip is the beginning of the wind entering in, the heart is its middle and the navel is its end. And if he follows after that, his mind is distracted by disquiet and perturbation according as it is said: “When he goes in with mindfulness after the beginning, middle, and end of the in-breath, his mind being distracted internally, both his body and his mind are disquieted and perturbed and shaky. When he goes out with mindfulness after the beginning, middle and end of the out-breath, his mind being distracted externally, both his body and his mind are disquieted and perturbed and shaky” (\textbf{\cite{Paṭis}I 165}).
                        \par\noindent[\textsc{\textbf{Touching \& fixing}}]

                            \emph{3–4.} So when he gives his attention to it by connection, he should do so not by the beginning, middle and end, but rather by \emph{touching }and by \emph{fixing}.

                            \vismParagraph{VIII.198}{198}{}
                            There is no attention to be given to it by touching separate from fixing as there is by counting separate from connection. But when he is counting the breaths in the place touched by each, he is giving attention to them by counting \marginnote{\textcolor{teal}{\footnotesize\{332|274\}}}{}and touching. When he has given up counting and is connecting them by means of mindfulness in that same place and fixing consciousness by means of absorption, then he is said to be giving his attention to them by connection, touching and fixing. And the meaning of this may be understood through the similes of the man who cannot walk and the gatekeeper given in the commentaries, and through the simile of the saw given in the Paṭisambhidā.

                            \vismParagraph{VIII.199}{199}{}
                            Here is the simile of the man who cannot walk: Just as a man unable to walk, who is rocking a swing for the amusement of his children and their mother, sits at the foot of the swing post and sees both ends and the middle of the swing plank successively coming and going, \textcolor{brown}{\textit{[281]}} yet does not move from his place in order to see both ends and the middle, so too, when a bhikkhu places himself with mindfulness, as it were, at the foot of the post for anchoring [mindfulness] and rocks the swing of the in-breaths and out-breaths; he sits down with mindfulness on the sign at that same place, and follows with mindfulness the beginning, middle and end of the in-breaths and out-breaths at the place touched by them as they come and go; keeping his mind fixed there, he then sees them without moving from his place in order to see them. This is the simile of the man who cannot walk.

                            \vismParagraph{VIII.200}{200}{}
                            This is the simile of the gatekeeper: Just as a gatekeeper does not examine people inside and outside the town, asking, “Who are you? Where have you come from? Where are you going? What have you got in your hand?”—for those people are not his concern—but he does examine each man as he arrives at the gate, so too, the incoming breaths that have gone inside and the outgoing breaths that have gone outside are not this bhikkhu’s concern, but they are his concern each time they arrive at the [nostril] gate itself.

                            \vismParagraph{VIII.201}{201}{}
                            Then the simile of the saw should be understood from its beginning. For this is said:
                            \begin{verse}
                                “Sign, in-breath, out-breath, are not object\\{}
                                Of a single consciousness;\\{}
                                By one who knows not these three things\\{}
                                Development is not obtained.\\{}
                                “Sign, in-breath, out-breath, are not object\\{}
                                Of a single consciousness;\\{}
                                By one who does know these three things\\{}
                                Development can be obtained.”
                            \end{verse}


                            \vismParagraph{VIII.202}{202}{}
                            “How is it that these three things are not the object of a single consciousness, that they are nevertheless not unknown, that the mind does not become distracted, that he manifests effort, carries out a task, and achieves an effect?

                            “Suppose there were a tree trunk placed on a level piece of ground, and a man cut it with a saw. The man’s mindfulness is established by the saw’s teeth where they touch the tree trunk, without his giving attention to the saw’s teeth as they approach and recede, though they are not unknown to him as they do so; and he manifests effort, carries out a task, and achieves an effect. As the tree trunk placed on the level piece of ground, so the sign for the anchoring of mindfulness. \marginnote{\textcolor{teal}{\footnotesize\{333|275\}}}{}As the saw’s teeth, so the in-breaths and out-breaths. As the man’s mindfulness, established by the saw’s teeth where they touch the tree trunk, without his giving attention to the saw’s teeth as they approach and recede, though they are not unknown to him as they do so, and so he manifests effort, carries out a task, and achieves an effect, \textcolor{brown}{\textit{[282]}} so too, the bhikkhu sits, having established mindfulness at the nose tip or on the upper lip, without giving attention to the in-breaths and out-breaths as they approach and recede, though they are not unknown to him as they do so, and he manifests effort, carries out a task, and achieves an effect.

                            \vismParagraph{VIII.203}{203}{}
                            “‘Effort’: what is the effort? The body and the mind of one who is energetic become wieldy—this is the effort. What is the task? Imperfections come to be abandoned in one who is energetic, and his applied thoughts are stilled—this is the task. What is the effect? Fetters come to be abandoned in one who is energetic, and his inherent tendencies come to be done away with—this is the effect.

                            “So these three things are not the object of a single consciousness, and they are nevertheless not unknown, and the mind does not become distracted, and he manifests effort, carries out a task, and achieves an effect.
                            \begin{verse}
                                “Whose mindfulness of breathing in\\{}
                                And out is perfect, well developed,\\{}
                                And gradually brought to growth\\{}
                                According as the Buddha taught,\\{}
                                ’Tis he illuminates the world\\{}
                                Just like the full moon free from cloud”\footnote{\vismAssertFootnoteCounter{55}\vismHypertarget{VIII.n55}{}\textbf{\cite{Paṭis}I 170–172}; last line \textbf{\cite{Dhp}172}; whole verse \textbf{\cite{Th}548}.}
                            \end{verse}


                            This is the simile of the saw. But here it is precisely his not giving attention [to the breaths] as [yet to] come and [already] gone\footnote{\vismAssertFootnoteCounter{56}\vismHypertarget{VIII.n56}{}Reading \emph{āgata-gata-vasena} with \textbf{\cite{Vism-mhṭ}271}.} that should be understood as the purpose.

                            \vismParagraph{VIII.204}{204}{}
                            When someone gives his attention to this meditation subject, sometimes it is not long before the sign arises in him, and then the \emph{fixing}, in other words, absorption adorned with the rest of the jhāna factors, is achieved.

                            \vismParagraph{VIII.205}{205}{}
                            After someone has given his attention to counting, then just as when a body that is disturbed sits down on a bed or chair, the bed or chair sags down and creaks and the cover gets rumpled, but when a body that is not disturbed sits down, the bed or chair neither sags down nor creaks, the cover does not get rumpled, and it is as though filled with cotton wool—why? because a body that is not disturbed is light—so too, after he has given his attention to counting, when the bodily disturbance has been stilled by the gradual cessation of gross in-breaths and out-breaths, then both the body and the mind become light: the physical body is as though it were ready to leap up into the air. \textcolor{brown}{\textit{[283]}}

                            \vismParagraph{VIII.206}{206}{}
                            When his gross in-breaths and out breaths have ceased, his consciousness occurs with the sign of the subtle in-breaths and out-breaths as its object. And when that has ceased, it goes on occurring with the successively subtler signs as its object. How?

                            \vismParagraph{VIII.207}{207}{}
                            \marginnote{\textcolor{teal}{\footnotesize\{334|276\}}}{}Suppose a man stuck a bronze bell with a big iron bar and at once a loud sound arose, his consciousness would occur with the gross sound as its object; then, when the gross sound had ceased, it would occur afterwards with the sign of the subtle sound as its object; and when that had ceased, it would go on occurring with the sign of the successively subtler sounds as its object. This is how it should be understood. And this is given in detail in the passage beginning, “Just as when a metal gong is struck” (\hyperlink{VIII.184}{§184}{}).

                            \vismParagraph{VIII.208}{208}{}
                            For while other meditation subjects become clearer at each higher stage, this one does not: in fact, as he goes on developing it, it becomes more subtle for him at each higher stage, and it even comes to the point at which it is no longer manifest.

                            However, when it becomes unmanifest in this way, the bhikkhu should not get up from his seat, shake out his leather mat, and go away. What should be done? He should not get up with the idea “Shall I ask the teacher?” or “Is my meditation subject lost?”; for by going away, and so disturbing his posture, the meditation subject has to be started anew. So he should go on sitting as he was and [temporarily] substitute the place [normally touched for the actual breaths as the object of contemplation].\footnote{\vismAssertFootnoteCounter{57}\vismHypertarget{VIII.n57}{}The point made here is that if the breaths themselves get temporarily too faint to be observed, he should carry on by observing the tip of the nose where they normally touch until they become apparent again. He brings the meditation back to mind for the moment, “as the place (\emph{desato})” where they were last noticed, instead of “as breaths,” which have temporarily vanished.}

                            \vismParagraph{VIII.209}{209}{}
                            These are the means for doing it. The bhikkhu should recognize the unmanifest state of the meditation subject and consider thus: “Where do these in-breaths and out-breaths exist? Where do they not? In whom do they exist? In whom not?” Then, as he considers thus, he finds that they do not exist in one inside the mother’s womb, or in those drowned in water, or likewise in unconscious beings,\footnote{\vismAssertFootnoteCounter{58}\vismHypertarget{VIII.n58}{}Those born in the world of unconscious beings in the fine-material Brahmā world (\textbf{\cite{D}I 28}).} or in the dead, or in those attained to the fourth jhāna, or in those born into a fine-material or immaterial existence, or in those attained to cessation [of perception and feeling]. So he should apostrophize himself thus: “You with all your wisdom are certainly not inside a mother’s womb or drowned in water or in the unconscious existence or dead or attained to the fourth jhāna or born into the fine-material or immaterial existence or attained to cessation. Those in-breaths and out-breath are actually existent in you, only you are not able to discern them because your understanding is dull.” Then, fixing his mind on the place normally touched [by the breaths], he should proceed to give his attention to that.

                            \vismParagraph{VIII.210}{210}{}
                            These in-breaths and out-breaths occur striking the tip of the nose in a long-nosed man \textcolor{brown}{\textit{[284]}} and the upper lip in a short-nosed man. So he should fix the sign thus: “This is the place where they strike.” This was why the Blessed One said: “Bhikkhus, I do not say of one who is forgetful, who is not fully aware, [that he practices] development of mindfulness of breathing” (\textbf{\cite{M}III 84}).

                            \vismParagraph{VIII.211}{211}{}
                            \marginnote{\textcolor{teal}{\footnotesize\{335|277\}}}{}Although any meditation subject, no matter what, is successful only in one who is mindful and fully aware, yet any meditation subject other than this one gets more evident as he goes on giving it his attention. But this mindfulness of breathing is difficult, difficult to develop, a field in which only the minds of Buddhas, Paccekabuddhas, and Buddhas’ sons are at home. It is no trivial matter, nor can it be cultivated by trivial persons. In proportion as continued attention is given to it, it becomes more peaceful and more subtle. So strong mindfulness and understanding are necessary here.

                            \vismParagraph{VIII.212}{212}{}
                            Just as when doing needlework on a piece of fine cloth a fine needle is needed, and a still finer instrument for boring the needle’s eye, so too, when developing this meditation subject, which resembles fine cloth, both the mindfulness, which is the counterpart of the needle, and the understanding associated with it, which is the counterpart of the instrument for boring the needle’s eye, need to be strong. A bhikkhu must have the necessary mindfulness and understanding and must look for the in-breaths and out-breaths nowhere else than the place normally touched by them.

                            \vismParagraph{VIII.213}{213}{}
                            Suppose a ploughman, after doing some ploughing, sent his oxen free to graze and sat down to rest in the shade, then his oxen would soon go into the forest. Now, a skilled ploughman who wants to catch them and yoke them again does not wander through the forest following their tracks, but rather he takes his rope and goad and goes straight to the drinking place where they meet, and he sits or lies there. Then after the oxen have wandered about for a part of the day, they come to the drinking place where they meet and they bathe and drink, and when he sees that they have come out and are standing about, he secures them with the rope, and prodding them with the goad, he brings them back, yokes them, and goes on with his ploughing. So too, the bhikkhu should not look for the in-breaths and out-breaths anywhere else than the place normally touched by them. And he should take the rope of mindfulness and the goad of understanding, and fixing his mind on the place normally touched by them, he should go on giving his attention to that. \textcolor{brown}{\textit{[285]}} For as he gives his attention in this way they reappear after no long time, as the oxen did at the drinking place where they met. So he can secure them with the rope of mindfulness, and yoking them in that same place and prodding them with the goad of understanding, he can keep on applying himself to the meditation subject.

                            \vismParagraph{VIII.214}{214}{}
                            When he does so in this way, the sign\footnote{\vismAssertFootnoteCounter{59}\vismHypertarget{VIII.n59}{}“‘The sign’ is the learning sign and the counterpart sign, for both are stated here together. Herein, the three similes beginning with cotton are properly the learning sign, the rest are both. ‘Some’ are certain teachers. The similes beginning with the ‘cluster of gems’ are properly the counterpart sign” (\textbf{\cite{Vism-mhṭ}273}).} soon appears to him. But it is not the same for all; on the contrary, some say that when it appears it does so to certain people producing a light touch like cotton or silk-cotton or a draught.

                            \vismParagraph{VIII.215}{215}{}
                            But this is the exposition given in the commentaries: It appears to some like a star or a cluster of gems or a cluster of pearls, to others with a rough touch like that of silk-cotton seeds or a peg made of heartwood, to others like a long braid string or a wreath of flowers or a puff of smoke, to others like a stretched-out\marginnote{\textcolor{teal}{\footnotesize\{336|278\}}}{} cobweb or a film of cloud or a lotus flower or a chariot wheel or the moon’s disk or the sun’s disk.

                            \vismParagraph{VIII.216}{216}{}
                            In fact this resembles an occasion when a number of bhikkhus are sitting together reciting a suttanta. When a bhikkhu asks, “What does this sutta appear like to you?” one says, “It appears to me like a great mountain torrent,” another “To me it is like a line of forest trees,” another “To me it is like a spreading fruit tree giving cool shade.” For the one sutta appears to them differently because of the difference in their perception. Similarly this single meditation subject appears differently because of difference in perception.\footnote{\vismAssertFootnoteCounter{60}\vismHypertarget{VIII.n60}{}“‘Because of difference in perception’: because of the difference in the manner of perceiving that occurred before the arising of the sign” (\textbf{\cite{Vism-mhṭ}273}).} It is born of perception, its source is perception, it is produced by perception. Therefore it should be understood that when it appears differently it is because of difference in perception.

                            \vismParagraph{VIII.217}{217}{}
                            And here, the consciousness that has in-breath as its object is one, the consciousness that has out-breath as its object is another, and the consciousness that has the sign as its object is another. For the meditation subject reaches neither absorption nor even access in one who has not got these three things [clear]. But it reaches access and also absorption in one who has got these three things [clear]. For this is said:
                            \begin{verse}
                                “Sign, in-breath, out-breath, are not object\\{}
                                Of a single consciousness;\\{}
                                By one who knows not these three things\\{}
                                Development is not obtained.
                            \end{verse}

                            \begin{verse}
                                Sign, in-breath, out-breath, are not object\\{}
                                Of a single consciousness;\\{}
                                By one who does know these three things\\{}
                                Development can be obtained” (\textbf{\cite{Paṭis}I 170}). \textcolor{brown}{\textit{[286]}}
                            \end{verse}


                            \vismParagraph{VIII.218}{218}{}
                            And when the sign has appeared in this way, the bhikkhu should go to the teacher and tell him, “Venerable sir, such and such has appeared to me.” But [say the Dīgha reciters] the teacher should say neither “This is the sign” nor “This is not the sign”; after saying “It happens like this, friend,” he should tell him, “Go on giving it attention again and again;” for if he were told “It is the sign,” he might [become complacent and] stop short at that (see \textbf{\cite{M}I 193f.}), and if he were told “It is not the sign,” he might get discouraged and give up; so he should encourage him to keep giving it his attention without saying either. So the Dīgha reciters say, firstly. But the Majjhima reciters say that he should be told, “This is the sign, friend. Well done. Keep giving attention to it again and again.”

                            \vismParagraph{VIII.219}{219}{}
                            Then he should fix his mind on that same sign; and so from now on, his development proceeds by way of fixing. For the Ancients said this:
                            \begin{verse}
                                “Fixing his mind upon the sign\\{}
                                And putting away\footnote{\vismAssertFootnoteCounter{61}\vismHypertarget{VIII.n61}{}\emph{Vibhāvayaṃ} can mean “to do away with” or “to explain.” Either is applicable here according to \textbf{\cite{Vism-mhṭ}274}.} extraneous aspects,\\{}
                                \marginnote{\textcolor{teal}{\footnotesize\{337|279\}}}{}The clever man anchors his mind\\{}
                                Upon the breathings in and out.”
                            \end{verse}


                            \vismParagraph{VIII.220}{220}{}
                            So as soon as the sign appears, his hindrances are suppressed, his defilements subside, his mindfulness is established, and his consciousness is concentrated in access concentration.

                            \vismParagraph{VIII.221}{221}{}
                            Then he should not give attention to the sign as to its colour, or review it as to its [specific] characteristic. He should guard it as carefully as a king’s chief queen guards the child in her womb due to become a Wheel-turning Monarch,\footnote{\vismAssertFootnoteCounter{62}\vismHypertarget{VIII.n62}{}For the Wheel-turning Monarch (\emph{cakkavattin}) see DN 26 and MN 129.} or as a farmer guards the ripening crops; and he should avoid the seven unsuitable things beginning with the unsuitable abode and cultivate the seven suitable things. Then, guarding it thus, he should make it grow and improve with repeated attention, and he should practice the tenfold skill in absorption (\hyperlink{IV.42}{IV.42}{}) and bring about evenness of energy (\hyperlink{IV.66}{IV.66}{}).

                            \vismParagraph{VIII.222}{222}{}
                            As he strives thus, fourfold and fivefold jhāna is achieved by him on that same sign in the same way as described under the earth kasiṇa.
                        \par\noindent[\textsc{\textbf{Observing … turning away}}]

                            \emph{5–8.} (See \hyperlink{VIII.189}{§189}{}) However, when a bhikkhu has achieved the fourfold and fivefold jhāna and wants to reach purity by developing the meditation subject through \emph{observing }and through \emph{turning away}, he should make that jhāna familiar by attaining mastery in it in the five ways (\hyperlink{IV.131}{IV.131}{}), and then embark upon insight by defining mentality-materiality. How?

                            \vismParagraph{VIII.223}{223}{}
                            On emerging from the attainment, \textcolor{brown}{\textit{[287]}} he sees that the in-breaths and out-breaths have the physical body and the mind as their origin; and that just as, when a blacksmith’s bellows are being blown, the wind moves owing to the bag and to the man’s appropriate effort, so too, in-breaths and out-breaths are due to the body and the mind.

                            Next, he defines the in-breaths and out-breaths and the body as “materiality,” and the consciousness and the states associated with the consciousness as “the immaterial [mind].” This is in brief (cf. \textbf{\cite{M-a}I 249}); but the details will be explained later in the defining of mentality-materiality (\hyperlink{XVIII.3}{XVIII.3f.}{}).

                            \vismParagraph{VIII.224}{224}{}
                            Having defined mentality-materiality in this way, he seeks its condition. With search he finds it, and so overcomes his doubts about the way of mentality-materiality’s occurrence in the three divisions of time (\hyperlink{XIX}{Ch. XIX}{}).

                            His doubts being overcome, he attributes the three characteristics [beginning with that of suffering to mentality and materiality], comprehending [them] by groups (\hyperlink{XX.2}{XX.2f.}{}); he abandons the ten imperfections of insight beginning with illumination, which arise in the first stages of the contemplation of rise and fall (\hyperlink{XX.105}{XX.105f.}{}), and he defines as “the path” the knowledge of the way that is free from these imperfections (\hyperlink{XX.126}{XX.126f.}{}).

                            He reaches contemplation of dissolution by abandoning [attention to] arising. When all formations have appeared as terror owing to the contemplation of their incessant dissolution, he becomes dispassionate towards them (\hyperlink{XXI}{Ch. XXI}{}), his greed for them fades away, and he is liberated from them (Ch. XXII). \marginnote{\textcolor{teal}{\footnotesize\{338|280\}}}{}After he has [thus] reached the four noble paths in due succession and has become established in the fruition of Arahantship, he at last attains to the nineteen kinds of reviewing knowledge (\hyperlink{XXII.19}{XXII.19f.}{}), and he becomes fit to receive the highest gifts from the world with its deities.

                            \vismParagraph{VIII.225}{225}{}
                            At this point his development of concentration through mindfulness of breathing, beginning with \emph{counting }and ending with \emph{looking back }(\hyperlink{VIII.189}{§189}{}) is completed.

                            This is the commentary on the first tetrad in all aspects.
                \subsection[\vismAlignedParas{§226–230}Word Commentary Continued—Second Tetrad]{Word Commentary Continued—Second Tetrad}

                    \vismParagraph{VIII.226}{226}{}
                    Now, since there is no separate method for developing the meditation subject in the case of the other tetrads, their meaning therefore needs only to be understood according to the word commentary.

                    (v) He trains thus: “\emph{I shall breathe in … shall breathe out experiencing happiness},” that is, making happiness known, making it plain. Herein, the happiness is experienced in two ways: (a) with the object, and (b) with non-confusion.\footnote{\vismAssertFootnoteCounter{63}\vismHypertarget{VIII.n63}{}“‘With the object’: under the heading of the object. The happiness included in the jhāna that has that object is experienced ‘because of the experiencing of the object.’ What is meant? Just as, when a man who is looking for a snake discovers (experiences) its abode, the snake is, as it were, already discovered (experienced) and caught, owing to the ease with which he will then be able to catch it with charms and spells, so too, when the object, which is the abode of the happiness, is experienced (discovered), then the happiness itself is experienced (discovered) too, owing to the ease with which it will be apprehended in its specific and general characteristics. ‘By his penetration of its characteristics’: by penetration of the specific and general characteristics of happiness. For when the specific and general characteristics of anything are experienced then that thing is experienced according to reality” (\textbf{\cite{Vism-mhṭ}276}).}

                    \vismParagraph{VIII.227}{227}{}
                    (a) How is the happiness experienced with the object? He attains the two jhānas in which happiness is present. At the time when he has actually entered upon them the happiness is experienced with the object owing to the obtaining of the jhāna, because of the experiencing of the object. (b) How with non-confusion? When, after entering upon and emerging from one of the two jhānas accompanied by happiness, \textcolor{brown}{\textit{[288]}} he comprehends with insight that happiness associated with the jhāna as liable to destruction and to fall, then at the actual time of the insight the happiness is experienced with non-confusion owing to the penetration of its characteristics [of impermanence, and so on].

                    \vismParagraph{VIII.228}{228}{}
                    For this is said in the Paṭisambhidā: “When he knows unification of mind and non-distraction through long in-breaths, mindfulness is established in him. By means of that mindfulness and that knowledge that happiness is experienced. When he knows unification of mind and non-distraction through long out-breaths … through short in-breaths … through short out-breaths … through in-breaths … out-breaths experiencing the whole body … through in-breaths … out-breaths tranquilizing the bodily formation, mindfulness is established in \marginnote{\textcolor{teal}{\footnotesize\{339|281\}}}{}him. By means of that mindfulness and that knowledge that happiness is experienced.

                    “It is experienced by him when he adverts, when he knows, sees, reviews, steadies his mind, resolves with faith, exerts energy, establishes mindfulness, concentrates his mind, understands with understanding, directly knows what is to be directly known, fully understands what is to be fully understood, abandons what is to be abandoned, develops what is to be developed, realizes what is to be realized. It is in this way that that happiness is experienced” (\textbf{\cite{Paṭis}I 187}).

                    \vismParagraph{VIII.229}{229}{}
                    (vi–viii) The remaining [three] clauses should be understood in the same way as to meaning; but there is this difference here. The\emph{ experiencing }of bliss must be understood to be through three jhānas, and that of the \emph{mental formation }through four. The mental formation consists of the two aggregates of feeling and perception. And in the case of the clause, \emph{experiencing bliss}, it is said in the Paṭisambhidā in order to show the plane of insight here [as well]: “‘Bliss’: there are two kinds of bliss, bodily bliss and mental bliss” (\textbf{\cite{Paṭis}I 188}). \emph{Tranquilizing the mental formation}: tranquilizing the gross mental formation; stopping it, is the meaning. And this should be understood in detail in the same way as given under the bodily formation (see \hyperlink{VIII.176}{§§176}{}–\hyperlink{VIII.85}{85}{}).

                    \vismParagraph{VIII.230}{230}{}
                    Here, moreover, in the “happiness” clause feeling [which is actually being contemplated in this tetrad] is stated under the heading of “happiness” [which is a formation] but in the “bliss” clause feeling is stated in its own form. In the two “mental-formation” clauses the feeling is that [necessarily] associated with perception because of the words, “Perception and feeling belong to the mind, these things being bound up with the mind are mental formations” (\textbf{\cite{Paṭis}I 188}). \textcolor{brown}{\textit{[289]}}

                    So this tetrad should be understood to deal with contemplation of feeling.
                \subsection[\vismAlignedParas{§231–233}Word Commentary Continued—Third Tetrad]{Word Commentary Continued—Third Tetrad}

                    \vismParagraph{VIII.231}{231}{}
                    (ix) In the third tetrad the \emph{experiencing }of the [\emph{manner of] consciousness }must be understood to be through four jhānas.

                    (x) \emph{Gladdening the [manner of] consciousness}: he trains thus: “Making the mind glad, instilling gladness into it, cheering it, rejoicing it, I shall breathe in, shall breathe out.” Herein, there is gladdening in two ways, through concentration and through insight.

                    How through concentration? He attains the two jhānas in which happiness is present. At the time when he has actually entered upon them he inspires the mind with gladness, instils gladness into it, by means of the happiness associated with the jhāna. How through insight? After entering upon and emerging from one of the two jhānas accompanied by happiness, he comprehends with insight that happiness associated with the jhāna as liable to destruction and to fall; thus at the actual time of insight he inspires the mind with gladness, instils gladness into it, by making the happiness associated with the jhāna the object. It is of one \marginnote{\textcolor{teal}{\footnotesize\{340|282\}}}{}progressing in this way that the words, “He trains thus: ‘I shall breathe in … shall breathe out gladdening the [manner of] consciousness,’” are said.

                    \vismParagraph{VIII.232}{232}{}
                    (xi) \emph{Concentrating }(\emph{samādahaṃ}) \emph{the [manner of] consciousness}: evenly (\emph{samaṃ}) placing (\emph{ādahanto}) the mind, evenly putting it on its object by means of the first jhāna and so on. Or alternatively, when, having entered upon those jhānas and emerged from them, he comprehends with insight the consciousness associated with the jhāna as liable to destruction and to fall, then at the actual time of insight momentary unification of the mind\footnote{\vismAssertFootnoteCounter{64}\vismHypertarget{VIII.n64}{}“‘Momentary unification of the mind’: concentration lasting only for a moment. For that too, when it occurs uninterruptedly on its object in a single mode and is not overcome by opposition, fixes the mind immovably, as if in absorption” (\textbf{\cite{Vism-mhṭ}278}).} arises through the penetration of the characteristics [of impermanence, and so on]. Thus the words, “He trains thus: ‘I shall breathe in … shall breathe out concentrating the [manner of] consciousness,’” are said also of one who evenly places the mind, evenly puts it on its object by means of the momentary unification of the mind arisen thus.

                    \vismParagraph{VIII.233}{233}{}
                    (xii) \emph{Liberating the [manner of] consciousness}: he both breathes in and breathes out delivering, liberating, the mind from the hindrances by means of the first jhāna, from applied and sustained thought by means of the second, from happiness by means of the third, from pleasure and pain by means of the fourth. Or alternatively, when, having entered upon those jhānas and emerged from them, he comprehends with insight the consciousness associated with the jhāna as liable to destruction and to fall, then at the actual time of insight he delivers, liberates, the mind from the perception of permanence by means of the contemplation of impermanence, from the perception of pleasure by means of the contemplation of pain, from the perception of self by means of the contemplation of not self, from delight by means of the contemplation of dispassion, from greed by means of the contemplation of fading away, from arousing by means of the contemplation of cessation, from grasping by means of the contemplation of relinquishment. Hence it is said: \textcolor{brown}{\textit{[290]}} “He trains thus: ‘I shall breathe in … shall breathe out liberating the [manner of] consciousness.\footnote{\vismAssertFootnoteCounter{65}\vismHypertarget{VIII.n65}{}“‘Delivering’: secluding, separating, by means of deliverance consisting in suppression; abandoning the hindrances, is the meaning. ‘At the actual time of insight’: at the time of contemplation of dissolution. For dissolution is the furthest extreme of impermanence. So the meditator who is contemplating dissolution by its means sees under the heading of consciousness the whole field of formations as impermanent, not as permanent; and because of the suffering inherent in what is impermanent, and because of the absence of self in what is painful, he sees that same whole field of formations as painful, not as pleasant, and as not-self, not as self. But since what is impermanent, painful, and not-self is not something to delight in, and what is not something to delight in is not something to be greedy for, consequently he becomes dispassionate towards that whole field of formations when it is seen in the light of dissolution as impermanent, painful, not-self, he does not delight in it, and his greed for it fades away, does not dye him. Now, as he thus becomes dispassionate and his greed fades away, it is firstly by means of mundane knowledge only that he causes greed to cease and does not arouse it. The meaning is that he does not bring about its arising. Or alternatively, his greed having thus faded away, he causes by means of his own knowledge the cessation of the unseen field of formations just as that of the seen, he does not arouse it; the meaning is that he brings about only its cessation, he does not bring about its arising. Having entered on this way, he relinquishes, he does not grasp. What is meant? It is that this contemplation of impermanence, etc., is called relinquishment as giving up and relinquishment as entering into because it gives up defilements along with aggregate-producing kamma-formations and because, by seeing the flaws in what is formed and by inclining towards the opposite of what is formed, namely Nibbāna, it enters into that Nibbāna. Consequently the meditator who has that contemplation gives up defilements and enters into Nibbāna in the way stated. Herein, the contemplation of what is impermanent as only impermanent is ‘contemplation of impermanence’; this is a name for insight that occurs by taking formations of the three [mundane] planes [and leaving aside the supramundane] as impermanent. ‘From the perception of permanence’: from the wrong perception that occurs perceiving formed things as permanent, eternal; also the various views should be regarded as included under the heading of perception. Likewise with the perception of pleasure and so on. ‘By means of the contemplation of dispassion’: by means of the contemplation that occurs in the mode of dispassion for formations. ‘From delight’: from craving accompanied by happiness. ‘By means of the contemplation of fading away’: by means of the contemplation that occurs similarly in the mode of fading away; hence ‘delivering from greed’ is said. ‘By means of the contemplation of cessation’: by means of the successive seeing of formations’ cessation. Or contemplating cessation is contemplation such that formations cease only and do not arise with future renewal. For this is knowledge of desire for deliverance grown strong. Hence he said, ‘delivering from arousing.’ Contemplation that occurs in the mode of relinquishing is ‘contemplation of relinquishment.’ ‘From grasping’: from taking as permanent, etc.; or the meaning can also here be regarded as ‘from grasping rebirth-linking.’ (\textbf{\cite{Vism-mhṭ}279}) See Chapters XX and XXI.} ’” So this tetrad should be understood to deal with contemplation of mind.
                \subsection[\vismAlignedParas{§234–237}Word Commentary Continued—Fourth Tetrad]{Word Commentary Continued—Fourth Tetrad}

                    \vismParagraph{VIII.234}{234}{}
                    \marginnote{\textcolor{teal}{\footnotesize\{341|283\}}}{}(xiii) But in the fourth tetrad, as to \emph{contemplating impermanence}, here firstly, the impermanent should be understood, and impermanence, and the contemplation of impermanence, and one contemplating impermanence.

                    Herein, the five aggregates are \emph{the impermanent}. Why? Because their essence is rise and fall and change. \emph{Impermanence }is the rise and fall and change in those same aggregates, or it is their non-existence after having been; the meaning is, it is the breakup of produced aggregates through their momentary dissolution since they do not remain in the same mode. \emph{Contemplation of impermanence }is contemplation of materiality, etc., as “impermanent” in virtue of that impermanence. \emph{One contemplating impermanence }possesses that contemplation. So it is when one such as this is breathing in and breathing out that it can be understood of him: “He trains thus: ‘I shall breathe in … shall breathe out contemplating impermanence.’”\footnote{\vismAssertFootnoteCounter{66}\vismHypertarget{VIII.n66}{}“What is called ‘permanent’ is what is lasting, eternal, like Nibbāna. What is called ‘impermanent’ is what is not permanent, and is possessed of rise and fall. He said ‘The five aggregates are “the impermanent,’” signifying that they are formed dhammas as to meaning. Why? ‘Because their essence is rise and fall and change’: the meaning is that their individual essences have rise and fall and change. Herein, formed dhammas’ arising owing to cause and condition, their coming to be after non-existence, their acquisition of an individual self (\emph{attalābha}), is ‘rise.’ Their momentary cessation when arisen is ‘fall.’}

                    \vismParagraph{VIII.235}{235}{}
                    \marginnote{\textcolor{teal}{\footnotesize\{342|284\}}}{}(xiv) \emph{Contemplating fading away}: there are two kinds of fading away, that is, fading away as destruction, and absolute fading away.\footnote{\vismAssertFootnoteCounter{67}\vismHypertarget{VIII.n67}{}“‘Destruction’ is the vanishing of formations; it is the act of those formations’ fading away, their disintegration, that is ‘fading away.’ Destruction itself as fading away is ‘fading away as destruction’; this is momentary cessation. Formations fade away absolutely here when this has been reached, thus it is ‘absolute fading away;’ this is Nibbāna” (\textbf{\cite{Vism-mhṭ}280}).} Herein, “fading away as destruction” is the momentary dissolution of formations. “Absolute fading away” is Nibbāna. Contemplation of fading away is insight and it is the path, which occurs as the seeing of these two. It is when he possesses this twofold contemplation that it can be understood of him: “He trains thus: ‘I shall breathe in … shall breathe out contemplating fading away.’”

                    (xv) The same method of explanation applies to the clause, \emph{contemplating cessation}.

                    \vismParagraph{VIII.236}{236}{}
                    (xvi) \emph{Contemplating relinquishment}: relinquishment is of two kinds too, that is to say, relinquishment as giving up, and relinquishment as entering into. Relinquishment itself as [a way of] contemplation is “contemplation of relinquishment.” For insight is called both “relinquishment as giving up” and “relinquishment as entering into” since [firstly], through substitution of opposite qualities, it gives up defilements with their aggregate-producing kamma formations, and [secondly], through seeing the wretchedness of what is formed, it also enters into Nibbāna by inclining towards Nibbāna, which is the opposite of the formed (\hyperlink{XI.18}{XI.18}{}). Also the path is called both “relinquishment as giving up” and “relinquishment as entering into” since it gives up defilements with their aggregate-producing kamma-formations by cutting them off, and it enters into Nibbāna by making it its object. Also both [insight and path knowledge] are called contemplation (\emph{anupassanā}) because of their re-seeing successively (\emph{anu anu passanā}) each preceding kind of knowledge.\footnote{\vismAssertFootnoteCounter{68}\vismHypertarget{VIII.n68}{}“The act of relinquishing as the act of giving up by means of substituting for what should be abandoned its opposite quality or by cutting it off, is ‘relinquishment as giving up.’ Likewise the act of relinquishing of self that takes place in non-formation of kamma, which is the relinquishing of all substrata (circumstances) of becoming, being the entering into that [Nibbāna] either by inclination towards it [in insight] or by having it as object [in the path] is ‘relinquishment as entering into.’ ‘Through substitution of opposite qualities’: here contemplation of impermanence, firstly, gives up perception of permanence by abandoning through substitution of the opposite [e.g. substituting perception of impermanence for that of permanence in the case of all formed things]. And the giving up in this way is in the form of inducing non-occurrence. For all kamma-formations that are rooted in defilements due to apprehending (formations) as permanent, and the kamma-resultant aggregates rooted in both which might arise in the future, are abandoned by causing their non-occurrence. Likewise in the case of perception of pain, and so on. ‘Through seeing the wretchedness of what is formed’: through seeing the fault of impermanence, etc., in the formed three-plane field of formations. It is ‘the opposite of the formed’ owing to its permanence, and so on. When defilements are given up by the path, then kamma-formations are called ‘given up’ through producing (\emph{āpādana}) in them the nature of not causing result, and aggregates rooted in them are called ‘given up’ through their being rendered fit for non-arising. So the path gives up all these, is what is meant” (\textbf{\cite{Vism-mhṭ}281}). The word \emph{pakkhandana} (rendered by “entering into”) is used to define the act of faith, and can also be rendered by “launching out into” or by “leap.”} \textcolor{brown}{\textit{[291]}} It is when he possesses \marginnote{\textcolor{teal}{\footnotesize\{343|285\}}}{}this twofold contemplation that it can be understood of him: “He trains thus: ‘I shall breathe in … shall breathe out contemplating relinquishment.’”

                    \vismParagraph{VIII.237}{237}{}
                    This tetrad deals only with pure insight while the previous three deal with serenity and insight. This is how the development of mindfulness of breathing with its sixteen bases in four tetrads should be understood.
                \subsection[\vismAlignedParas{§237–244}Conclusion]{Conclusion}

                    This mindfulness of breathing with its sixteen bases thus is of great fruit, of great benefit.

                    \vismParagraph{VIII.238}{238}{}
                    Its great beneficialness should be understood here as peacefulness both because of the words, “And, bhikkhus, this concentration through mindfulness of breathing, when developed and much practiced, is both peaceful and sublime” (\textbf{\cite{S}V 321}), etc., and because of its ability to cut off applied thoughts; for it is because it is peaceful, sublime, and an unadulterated blissful abiding that it cuts off the mind’s running hither and thither with applied thoughts obstructive to concentration, and keeps the mind only on the breaths as object. Hence it is said: “Mindfulness of breathing should be developed in order to cut off applied thoughts” (\textbf{\cite{A}IV 353}).

                    \vismParagraph{VIII.239}{239}{}
                    Also its great beneficialness should be understood as the root condition for the perfecting of clear vision and deliverance; for this has been said by the Blessed One: “Bhikkhus, mindfulness of breathing, when developed and much practiced, perfects the four foundations of mindfulness. The four foundations of mindfulness, when developed and much practiced, perfect the seven enlightenment factors. The seven enlightenment factors, when developed and much practiced, perfect clear vision and deliverance” (\textbf{\cite{M}III 82}).

                    \vismParagraph{VIII.240}{240}{}
                    Again its great beneficialness should be understood to reside in the fact that it causes the final in-breaths and out-breaths to be known; for this is said by the Blessed One: “Rāhula, when mindfulness of breathing is thus developed, thus practiced much, the final in-breaths and out-breaths, too, are known as they cease, not unknown” (\textbf{\cite{M}I 425f.}).

                    \vismParagraph{VIII.241}{241}{}
                    Herein, there are three kinds of [breaths that are] final because of cessation, that is to say, final in becoming, final in jhāna, and final in death. For, among the various kinds of becoming (existence), in-breaths and out-breaths occur in the sensual-sphere becoming, not in the fine-material and immaterial kinds of becoming. That is why there are final ones in becoming. In the jhānas they occur \marginnote{\textcolor{teal}{\footnotesize\{344|286\}}}{}in the first three but not in the fourth. That is why there are final ones in jhāna. Those that arise along with the sixteenth consciousness preceding the death consciousness \textcolor{brown}{\textit{[292]}} cease together with the death consciousness. They are called “final in death.” It is these last that are meant here by “final.”

                    \vismParagraph{VIII.242}{242}{}
                    When a bhikkhu has devoted himself to this meditation subject, it seems, if he adverts, at the moment of arising of the sixteenth consciousness before the death consciousness, to their arising, then their arising is evident to him; if he adverts to their presence, then their presence is evident to him; if he adverts to their dissolution, then their dissolution is evident to him; and it is so because he has thoroughly discerned in-breaths and out-breaths as object.

                    \vismParagraph{VIII.243}{243}{}
                    When a bhikkhu has attained Arahantship by developing some other meditation subject than this one, he may be able to define his life term or not. But when he has reached Arahantship by developing this mindfulness of breathing with its sixteen bases, he can always define his life term. He knows, “My vital formations will continue now for so long and no more.” Automatically he performs all the functions of attending to the body, dressing and robing, etc., after which he closes his eyes, like the Elder Tissa who lived at the Koṭapabbata Monastery, like the Elder Mahā Tissa who lived at the Mahā Karañjiya Monastery, like the Elder Tissa the alms-food eater in the kingdom of Devaputta, like the elders who were brothers and lived at the Cittalapabbata monastery.

                    \vismParagraph{VIII.244}{244}{}
                    Here is one story as an illustration. After reciting the Pātimokkha, it seems, on the Uposatha day of the full moon, one of the two elders who were brothers went to his own dwelling place surrounded by the Community of Bhikkhus. As he stood on the walk looking at the moonlight he calculated his own vital formations, and he said to the Community of Bhikkhus, “In what way have you seen bhikkhus attaining Nibbāna up till now?” Some answered, “Till now we have seen them attain Nibbāna sitting in their seats.” Others answered, “We have seen them sitting cross-legged in the air.” The elder said, “I shall now show you one attaining Nibbāna while walking.” He then drew a line on the walk, saying, “I shall go from this end of the walk to the other end and return; when I reach this line I shall attain Nibbāna.” So saying, he stepped on to the walk and went to the far end. On his return he attained Nibbāna in the same moment in which he stepped on the line. \textcolor{brown}{\textit{[293]}}
                    \begin{verse}
                        So let a man, if he is wise,\\{}
                        Untiringly devote his days\\{}
                        To mindfulness of breathing, which\\{}
                        Rewards him always in these ways.
                    \end{verse}


                    This is the section dealing with mindfulness of breathing in the detailed explanation.
            \section[\vismAlignedParas{§245–251}(10) Recollection of Peace]{(10) Recollection of Peace}

                \vismParagraph{VIII.245}{245}{}
                One who wants to develop the recollection of peace mentioned next to mindfulness of breathing (\hyperlink{III.105}{III.105}{}) should go into solitary retreat and recollect the special qualities of Nibbāna, in other words, the stilling of all suffering, as follows: \marginnote{\textcolor{teal}{\footnotesize\{345|287\}}}{}“Bhikkhus, in so far as there are dhammas, whether formed or unformed, fading away is pronounced the best of them, that is to say, the disillusionment of vanity, the elimination of thirst, the abolition of reliance, the termination of the round, the destruction of craving, fading away, cessation, Nibbāna” (\textbf{\cite{A}II 34}).

                \vismParagraph{VIII.246}{246}{}
                Herein in \emph{so far as }means as many as. \emph{Dhammas }[means] individual essences.\footnote{\vismAssertFootnoteCounter{69}\vismHypertarget{VIII.n69}{}“In such passages as ‘Dhammas that are concepts’ (\textbf{\cite{Dhs}p. 1}, §1308) even a non-entity (\emph{abhāva}) is thus called a ‘dhamma’ since it is borne (\emph{dhārīyati}) and affirmed (\emph{avadhārīyati}) by knowledge. That kind of dhamma is excluded by his saying, \emph{‘Dhammas [means] individual essences.’} The act of becoming (\emph{bhavana}), which constitutes existingness (\emph{vijjamānatā}) in the ultimate sense, is essence (\emph{bhāva}); it is with essence (\emph{saha bhāvena}), thus it is an individual essence (\emph{sabhāva}); the meaning is that it is possible (\emph{labbhamānarūpa}) in the true sense, in the ultimate sense. For these are called ‘dhammas (bearers)’ because they bear (\emph{dhāraṇa}) their own individual essences (\emph{sabhāva}), and they are called ‘individual essences’ in the sense already explained” (\textbf{\cite{Vism-mhṭ}282}; cf. \hyperlink{VII.n1}{Ch. VII, n. 1}{}).

                        In the Piṭakas the word \emph{sabhāva} seems to appear only once (\textbf{\cite{Paṭis}II 178}). It next appears in the \textbf{\cite{Netti}(p. 79)}, the \emph{Milindapañhā} (pp. 90, 164, 212, 360). It is extensively used for exegetical purposes in the \emph{Visuddhimagga} and main commentaries and likewise in the subcommentaries. As has just been shown, it is narrower than dhamma (see also \hyperlink{XXIII}{Ch. XXIII}{}. n. 18). It often roughly corresponds to \emph{dhātu} (element—see e.g. \textbf{\cite{Dhs-a}263}) and to \emph{lakkhaṇa} (characteristic—see below), but less nearly to the vaguer and (in Pali) untechnical \emph{pakati} (nature), or to \emph{rasa} (function—see \hyperlink{I.21}{I.21}{}). The \emph{Atthasālinī }observes: “It is the individual essence, or the generality, of such and such dhammas that is called their characteristic” (\textbf{\cite{Dhs-a}63}); on which the \emph{Mūla Ṭīkā} comments: “The \emph{individual essence} consisting in, say, hardness as that of earth, or touching as that of contact, is not common to all dhammas. The \emph{generality} is the individual essence common to all consisting in impermanence, etc.; also in this context (i.e. \textbf{\cite{Dhs}§1}) the characteristic of being profitable may be regarded as general because it is the individual \emph{essence }common to all that is profitable; or alternatively it is their individual \emph{essence} because it is not common to the unprofitable and indeterminate [kinds of consciousness]” (\textbf{\cite{Dhs-a}63}). The individual \emph{essence} of any formed dhamma is manifested in the three instants of its existence (\emph{atthitā, vijjamānatā}), namely, arising, presence (= aging) and dissolution. It comes from nowhere and goes nowhere (\hyperlink{XV.15}{XV.15}{}) and is borne by the mind. Dhammas without individual essence (\emph{asabhāvadhamma}) include the attainment of cessation (see \hyperlink{XXIII.n18}{Ch. XXIII, n. 18}{}) and some concepts. Space and time belong to the last-mentioned. Of space (\emph{ākāsa}) the \emph{Majjhima Nikāya Ṭīkā} says: “Space, which is quite devoid of individual essence, is called empty” (commenting on MN 106), while of time (\emph{kāla}) the \emph{Mūla Ṭīkā }says: “Though time is determined by the kind of consciousness [e.g. as specified in the first paragraph of the Dhammasaṅgaṇī] and is non-existent (\emph{avijjamāna}) as to individual essence, yet as the non-entity (\emph{abhāva}) before and after the moment in which those [conascent and co-present] dhammas occur, it is called the ‘container (\emph{adhikaraṇa})’; it is perceived (symbolized) only as the state of a receptacle (\emph{ādhāra-bhāva}) (\textbf{\cite{Dhs-a}62}). Of Nibbāna (for which see \hyperlink{XVI.46}{XVI.46ff.}{}), which has its own individual essence, the \emph{Mūla Ṭīkā} says “Nibbāna is not like other dhammas; because of its extreme profundity it cannot be made an object of consciousness (\emph{ālambituṃ}) by one who has not realized it. That is why it has to be realized by change-of-lineage. It has profundity surpassing any individual essence belonging to the three periods of time” (\textbf{\cite{Vibh-a}38}).

                        \emph{Sabhāva} has not the extreme vagueness of its parent \emph{bhāva,} which can mean anything between “essence” (see e.g. \textbf{\cite{Dhs-a}61}) and “-ness” (e.g. \emph{natthibhāva} = non-existingness—\hyperlink{X.35}{X.35}{}). This may be remembered when \emph{sabhāva} is defined as above thus: “It is with essence (\emph{sahabhāvena}), thus it is individual essence (\emph{sabhāva})” (\textbf{\cite{Vism-mhṭ}282}), and when it is defined again thus: “A dhamma’s own essence or its existing essence (\emph{sako vā bhāvo samāno vā bhāvo}) is its individual essence (\emph{sabhāva})” (\textbf{\cite{Vism-mhṭ}433}). \emph{Sabhāva }can also be the basis of a wrong view, if regarded as the sole efficient cause or condition of any formed thing (\hyperlink{XVI.n23}{Ch. XVI, n.23}{}). The Sanskrit equivalent, \emph{svabhāva}, had a great vogue and checkered history in philosophical discussions on the Indian mainland.

                        This (unlike the word, \emph{dhamma,} which has many “referents”) is an instance in which it is of first importance to stick to one rendering. The word is a purely exegetical one; consequently vagueness is undesirable. “Individual essence” has been chosen principally on etymological grounds, and the word “essence” (an admittedly slippery customer) must be understood from the contexts in which it is used and not prejudged. Strictly it refers here to the triple moment of arising etc., of formed dhammas that can have such “existence” in their own right and be experienced as such; and it refers to the realizability of Nibbāna. We are here in the somewhat magical territory of ontology, a subject that is at present undergoing one of its periodical upheavals in Europe, this time in the hands of the existentialists. Consequently it is important to approach the subject with an open mind.} \emph{Whether formed or unformed}: whether made by conditions going \marginnote{\textcolor{teal}{\footnotesize\{346|288\}}}{}together, coming together, or not so made.\footnote{\vismAssertFootnoteCounter{70}\vismHypertarget{VIII.n70}{}“‘Made’ is generated. ‘Not so made’ is not made by any conditions at all.” (\textbf{\cite{Vism-mhṭ}281})} \emph{Fading away is pronounced the best of them}: of these formed and unformed dhammas, fading away is pronounced the best, is called the foremost, the highest.

                \vismParagraph{VIII.247}{247}{}
                Herein \emph{fading away }is not mere absence of greed, but rather it is that unformed dhamma which, while given the names “disillusionment of vanity,” etc., in the clause, “that is to say, the disillusionment of vanity, … Nibbāna,” is treated basically as \emph{fading away}.\footnote{\vismAssertFootnoteCounter{71}\vismHypertarget{VIII.n71}{}“That dhamma possessing individual essence and having the characteristic of being not formed is to be treated basically as ‘fading away,’ since it is there that the dhamma of defilement fades away” (\textbf{\cite{Vism-mhṭ}282}).} It is called \emph{disillusionment of vanity} because on coming to it all kinds of vanity (intoxication), such as the vanity of conceit, and vanity of manhood, are disillusioned, undone, done away with.\footnote{\vismAssertFootnoteCounter{72}\vismHypertarget{VIII.n72}{}“When they are being abandoned by the noble path, which occurs by making Nibbāna its object, it is said that they are abandoned by reaching that [Nibbāna] which is why he said, ‘Because on coming to it,’ and so on. Herein, ‘vanity of conceit (\emph{māna-mada})’ is conceit (\emph{māna}) that occurs as conceiving (\emph{maññanā}) ‘I am superior’ (\textbf{\cite{Vibh}353}). ‘Vanity of manhood’ is vanity about being of the male sex. The words ‘such as’ refer to vanity of birth, and so on (\textbf{\cite{Vibh}345})” (\textbf{\cite{Vism-mhṭ}282}).} And it is called \emph{elimination of thirst }because on coming to it all thirst for sense desires is eliminated and quenched. But it is called \emph{abolition of reliance }because on coming to its reliance on the five cords of sense desire is abolished. It is called \emph{termination of the round }because on coming to it the round of the three planes [of existence] is terminated. It is called \emph{destruction of craving }because on coming to it craving is \marginnote{\textcolor{teal}{\footnotesize\{347|289\}}}{}entirely destroyed, fades away and ceases. It is called \emph{Nibbāna }(extinction) because it has gone away from (\emph{nikkhanta}), has escaped from (\emph{nissaṭa}), is dissociated from craving, which has acquired in common usage the name “fastening” (\emph{vāna}) because, by ensuring successive becoming, craving serves as a joining together, a binding together, a lacing together, of the four kinds of generation, five destinies, seven stations of consciousness and nine abodes of beings.\footnote{\vismAssertFootnoteCounter{73}\vismHypertarget{VIII.n73}{}Modern etymology derives the word Nibbāna (Skr. \emph{nirvāṇa}) from the negative prefix \emph{nir} plus the root \emph{vā} (to blow). The original literal meaning was probably “extinction” of a fire by ceasing to blow on it with bellows (a smith’s fire for example). It seems to have been extended to extinction of fire by any means, for example, the going out of a lamp’s flame (\emph{nibbāyati—}\textbf{\cite{M}III 245}). By analogy it was extended to the extinction of the five-aggregate process on the Arahant’s death (see It 38). Nibbāna is not the “extinction of a self or of a living lasting being,” such a mistaken opinion being the annihilation view (see e.g. \textbf{\cite{M}I 140}, \textbf{\cite{S}III 109}).} \textcolor{brown}{\textit{[294]}}

                \vismParagraph{VIII.248}{248}{}
                This is how peace, in other words, Nibbāna, should be recollected according to its special qualities beginning with disillusionment of vanity. But it should also be recollected according to the other special qualities of peace stated by the Blessed One in the suttas beginning with: “Bhikkhus, I shall teach you the unformed … the truth … the other shore … the hard-to-see … the undecaying … the lasting … the undiversified … the deathless … the auspicious … the safe … the marvellous … the intact … the unafflicted … the purity … the island … the shelter ….” (\textbf{\cite{S}IV 360–372}).\footnote{\vismAssertFootnoteCounter{74}\vismHypertarget{VIII.n74}{}Some texts add \emph{leṇa} (another word for shelter). Still others are given in the Saṃyutta text.}

                \vismParagraph{VIII.249}{249}{}
                As he recollects peace in its special qualities of disillusionment of vanity, etc., in this way, then: “On that occasion his mind is not obsessed by greed or obsessed by hate or obsessed by delusion; his mind has rectitude on that occasion, being inspired by peace” (see \hyperlink{VII.65}{VII.65}{}, etc.). So when he has suppressed the hindrances in the way already described under the recollection of the Enlightened One, etc., the jhāna factors arise in a single moment. But owing to the profundity of the special qualities of peace, or owing to his being occupied in recollecting special qualities of various kinds, the jhāna is only access and does not reach absorption. And that jhāna itself is known as “recollection of peace” too because it arises by means of the special qualities of peace.

                \vismParagraph{VIII.250}{250}{}
                And as in the case of the six recollections, this also comes to success only in a noble disciple. Still, though this is so, it can nevertheless also be brought to mind by an ordinary person who values peace. For even by hearsay the mind has confidence in peace.

                \vismParagraph{VIII.251}{251}{}
                A bhikkhu who is devoted to this recollection of peace sleeps in bliss and wakes in bliss, his faculties are peaceful, his mind is peaceful, he has conscience and shame, he is confident, he is resolved [to attain] the superior [state], he is respected and honoured by his fellows in the life of purity. And even if he penetrates no higher, he is at least headed for a happy destiny.
                \begin{verse}
                    \marginnote{\textcolor{teal}{\footnotesize\{348|290\}}}{}So that is why a man of wit\\{}
                    Untiringly devotes his days\\{}
                    To mind the noble peace, which can\\{}
                    Reward him in so many ways.
                \end{verse}


                This is the section dealing with the recollection of peace in the detailed explanation.

                The eighth chapter called “The Description of Recollections as Meditation Subjects” in the Treatise on the Development of Concentration in the \emph{Path of Purification} composed for the purpose of gladdening good people.
        \chapter[The Divine Abidings]{The Divine Abidings\vismHypertarget{IX}\newline{\textnormal{\emph{Brahmavihāra-niddesa}}}}
            \label{IX}

            \section[\vismAlignedParas{§1–76}(1) Loving-Kindness]{(1) Loving-Kindness}

                \vismParagraph{IX.1}{1}{}
                \marginnote{\textcolor{teal}{\footnotesize\{349|291\}}}{}\textcolor{brown}{\textit{[295]}} The four divine abidings were mentioned next to the recollections as meditation subjects (\hyperlink{III.105}{III.105}{}). They are loving-kindness, compassion, gladness and equanimity. A meditator, who wants to develop firstly loving-kindness among these, if he is a beginner, should sever the impediments and learn the meditation subject. Then, when he has done the work connected with the meal and got rid of any dizziness due to it, he should seat himself comfortably on a well-prepared seat in a secluded place. To start with, he should review the danger in hate and the advantage in patience.

                \vismParagraph{IX.2}{2}{}
                Why? Because hate has to be abandoned and patience attained in the development of this meditation subject, and he cannot abandon unseen dangers and attain unknown advantages.

                Now, the danger in hate should be seen in accordance with such suttas as this: “Friends, when a man hates, is a prey to hate and his mind is obsessed by hate, he kills living things, and …” (\textbf{\cite{A}I 216}). And the advantage in patience should be understood according to such suttas as these:
                \begin{verse}
                    “No higher rule, the Buddhas say, than patience,\\{}
                    And no Nibbāna higher than forbearance” (\textbf{\cite{D}II 49}; \textbf{\cite{Dhp}184});
                \end{verse}

                \begin{verse}
                    “Patience in force, in strong array:\\{}
                    ’Tis him I call a brahman” (\textbf{\cite{Dhp}399});
                \end{verse}

                \begin{verse}
                    “No greater thing exists than patience” (\textbf{\cite{S}I 222}).
                \end{verse}


                \vismParagraph{IX.3}{3}{}
                Thereupon he should embark upon the development of loving-kindness for the purpose of secluding the mind from hate seen as a danger and introducing it to patience known as an advantage.

                But when he begins, he must know that some persons are of the wrong sort at the very beginning and that loving-kindness should be developed towards certain kinds of persons and not towards certain other kinds at first. \textcolor{brown}{\textit{[296]}}

                \vismParagraph{IX.4}{4}{}
                For loving-kindness should not be developed at first towards the following four kinds of persons: an antipathetic person, a very dearly loved friend, a neutral person, and a hostile person. Also it should not be developed specifically (see \hyperlink{IX.49}{§49}{}) towards the opposite sex, or towards a dead person.

                \vismParagraph{IX.5}{5}{}
                \marginnote{\textcolor{teal}{\footnotesize\{350|292\}}}{}What is the reason why it should not be developed at first towards an antipathetic person and the others? To put an antipathetic person in a dear one’s place is fatiguing. To put a very dearly loved friend in a neutral person’s place is fatiguing; and if the slightest mischance befalls the friend, he feels like weeping. To put a neutral person in a respected one’s or a dear one’s place is fatiguing. Anger springs up in him if he recollects a hostile person. That is why it should not be developed at first towards an antipathetic person and the rest.

                \vismParagraph{IX.6}{6}{}
                Then, if he develops it specifically towards the opposite sex, lust inspired by that person springs up in him. An elder supported by a family was asked, it seems, by a friend’s son, “Venerable sir, towards whom should loving-kindness be developed?” The elder told him, “Towards a person one loves.” He loved his own wife. Through developing loving-kindness towards her he was fighting against the wall all the night.\footnote{\vismAssertFootnoteCounter{1}\vismHypertarget{IX.n1}{}“‘Fighting against the wall’: having undertaken the precepts of virtue and sat down on a seat in his room with the door locked, he was developing loving-kindness. Blinded by lust arisen under cover of the loving-kindness, he wanted to go to his wife, and without noticing the door he beat on the wall in his desire to get out even by breaking the wall down” (\textbf{\cite{Vism-mhṭ}286}).} That is why it should not be developed specifically towards the opposite sex.

                \vismParagraph{IX.7}{7}{}
                But if he develops it towards a dead person, he reaches neither absorption nor access. A young bhikkhu, it seems, had started developing loving-kindness inspired by his teacher. His loving-kindness made no headway at all. He went to a senior elder and told him, “Venerable sir, I am quite familiar with attaining jhāna through loving-kindness, and yet I cannot attain it. What is the matter?” The elder said, “Seek the sign, friend, [the object of your meditation].” He did so. Finding that his teacher had died, he proceeded with developing loving-kindness inspired by another and attained absorption. That is why it should not be developed towards one who is dead.

                \vismParagraph{IX.8}{8}{}
                First of all it should be developed only towards oneself, doing it repeatedly thus: “May I be happy and free from suffering” or “May I keep myself free from enmity, affliction and anxiety and live happily.”

                \vismParagraph{IX.9}{9}{}
                If that is so, does it not conflict with what is said in the texts? For there is no mention of any development of it towards oneself in what is said in the Vibhaṅga: “And how does a bhikkhu dwell pervading one direction with his heart filled with loving-kindness? Just as he would feel loving-kindness on seeing a dearly loved person, so he pervades all beings with loving-kindness” (\textbf{\cite{Vibh}272}); and in what is said in the Paṭisambhidā: “In what five ways is the mind-deliverance of loving-kindness [practiced] with unspecified pervasion? May all beings be free from enmity, affliction and anxiety and live happily. May all breathing things \textcolor{brown}{\textit{[297]}} … all who are born … all persons … all those who have a personality be free from enmity, affliction and anxiety and live happily” (\textbf{\cite{Paṭis}II 130}); and in what is said in the Mettā Sutta: “In joy and safety may all beings be joyful at heart” (\textbf{\cite{Sn}145}). [Does it not conflict with those texts?]

                \vismParagraph{IX.10}{10}{}
                \marginnote{\textcolor{teal}{\footnotesize\{351|293\}}}{}It does not conflict. Why not? Because that refers to absorption. But this [initial development towards oneself] refers to [making oneself] an example. For even if he developed loving-kindness for a hundred or a thousand years in this way, “I am happy” and so on, absorption would never arise. But if he develops it in this way: “I am happy. Just as I want to be happy and dread pain, as I want to live and not to die, so do other beings, too,” making himself the example, then desire for other beings’ welfare and happiness arises in him. And this method is indicated by the Blessed One’s saying:
                \begin{verse}
                    I visited all quarters with my mind\\{}
                    Nor found I any dearer than myself;\\{}
                    Self is likewise to every other dear;\\{}
                    Who loves himself will never harm another (\textbf{\cite{S}I 75}; \textbf{\cite{Ud}47}).
                \end{verse}


                \vismParagraph{IX.11}{11}{}
                So he should first, as example, pervade himself with loving-kindness. Next after that, in order to proceed easily, he can recollect such gifts,\footnote{\vismAssertFootnoteCounter{2}\vismHypertarget{IX.n2}{}Reading \emph{dāna-piyavacanādīni} with Ce (see four \emph{saṅgahavatthūni—}\textbf{\cite{A}II 32}).} kind words, etc., as inspire love and endearment, such virtue, learning, etc., as inspire respect and reverence met with in a teacher or his equivalent or a preceptor or his equivalent, developing loving-kindness towards him in the way beginning, “May this good man be happy and free from suffering.” With such a person, of course, he attains absorption.

                \vismParagraph{IX.12}{12}{}
                But if this bhikkhu does not rest content with just that much and wants to break down the barriers, he should next, after that, develop loving-kindness towards a very dearly loved friend, then towards a neutral person as a very dearly loved friend, then towards a hostile person as neutral. And while he does so, he should make his mind malleable and wieldy in each instance before passing on to the next.

                \vismParagraph{IX.13}{13}{}
                But if he has no enemy, or he is of the type of a great man who does not perceive another as an enemy even when the other does him harm, he should not interest himself as follows: “Now that my consciousness of loving-kindness has become wieldy towards a neutral person, I shall apply it to a hostile one.” \textcolor{brown}{\textit{[298]}} Rather it was about one who actually has an enemy that it was said above that he should develop loving-kindness towards a hostile person as neutral.
                \subsection[\vismAlignedParas{§14–39}Getting Rid of Resentment]{Getting Rid of Resentment}

                    \vismParagraph{IX.14}{14}{}
                    If resentment arises in him when he applies his mind to a hostile person because he remembers wrongs done by that person, he should get rid of the resentment by entering repeatedly into loving-kindness [jhāna] towards any of the first-mentioned persons and then, after he has emerged each time, directing loving-kindness towards that person.

                    \vismParagraph{IX.15}{15}{}
                    But if it does not die out in spite of his efforts, then:
                    \begin{verse}
                        Let him reflect upon the saw\\{}
                        With other figures of such kind,\\{}
                        \marginnote{\textcolor{teal}{\footnotesize\{352|294\}}}{}And strive, and strive repeatedly,\\{}
                        To leave resentment far behind.
                    \end{verse}


                    He should admonish himself in this way: “Now, you who get angry, has not the Blessed One said this: ‘Bhikkhus, even if bandits brutally severed limb from limb with a two-handled saw, he who entertained hate in his heart on that account would not be one who carried out my teaching?’” (\textbf{\cite{M}I 129}). And this:
                    \begin{verse}
                        ”To repay angry men in kind\\{}
                        Is worse than to be angry first;\\{}
                        Repay not angry men in kind\\{}
                        And win a battle hard to win.
                    \end{verse}

                    \begin{verse}
                        The weal of both he does promote,\\{}
                        His own and then the other’s too,\\{}
                        Who shall another’s anger know\\{}
                        And mindfully maintain his peace” (\textbf{\cite{S}I 162}).
                    \end{verse}


                    And this: “Bhikkhus, there are seven things gratifying and helpful to an enemy that happen to one who is angry, whether woman or man. What seven? Here, bhikkhus, an enemy wishes thus for his enemy, ‘Let him be ugly!’ Why is that? An enemy does not delight in an enemy’s beauty. Now, this angry person is a prey to anger, ruled by anger; though well bathed, well anointed, with hair and beard trimmed and clothed in white, yet he is ugly, being a prey to anger. This is the first thing gratifying and helpful to an enemy that befalls one who is angry, whether woman or man. Furthermore, an enemy wishes thus for his enemy, ‘Let him lie in pain!’ … ‘Let him have no fortune!’ … ‘Let him not be wealthy!’ … ‘Let him not be famous!’ … ’Let him have no friends!’ \textcolor{brown}{\textit{[299]}} … ’Let him not on the breakup of the body, after death, reappear in a happy destiny in the heavenly world!’\footnote{\vismAssertFootnoteCounter{3}\vismHypertarget{IX.n3}{}The Aṅguttara text has “Let him … reappear in a state of loss” and so on.} Why is that? An enemy does not delight in an enemy’s going to a happy destiny. Now, this angry person is a prey to anger, ruled by anger; he misconducts himself in body, speech and mind. Misconducting himself thus in body, speech and mind, on the breakup of the body, after death, he reappears in a state of loss, in an unhappy destiny, in perdition, in hell, being a prey to anger” (\textbf{\cite{A}IV 94}).? And this: “As a log from a pyre, burnt at both ends and fouled in the middle, serves neither for timber in the village nor for timber in the forest, so is such a person as this I say” (\textbf{\cite{A}II 95}, \textbf{\cite{It}90})?. “If you are angry now, you will be one who does not carry out the Blessed One’s teaching; by repaying an angry man in kind you will be worse than the angry man and not win the battle hard to win; you will yourself do to yourself the things that help your enemy; and you will be like a pyre log.” (\emph{Source untraced})

                    \vismParagraph{IX.16}{16}{}
                    If his resentment subsides when he strives and makes effort in this way, it is good. If not, then he should remove irritation by remembering some controlled and purified state in that person, which inspires confidence when remembered.

                    \vismParagraph{IX.17}{17}{}
                    For one person may be controlled in his bodily behaviour with his control in doing an extensive course of duty known to all, though his verbal and mental \marginnote{\textcolor{teal}{\footnotesize\{353|295\}}}{}behaviour are not controlled. Then the latter should be ignored and the control in his bodily behaviour remembered.

                    \vismParagraph{IX.18}{18}{}
                    Another may be controlled in his verbal behaviour, and his control known to all—he may naturally be clever at welcoming kindly, easy to talk with, congenial, open-countenanced, deferential in speech, and he may expound the Dhamma with a sweet voice and give explanations of Dhamma with well-rounded phrases and details—though his bodily and mental behaviour are not controlled. Then the latter should be ignored and the control in his verbal behaviour remembered.

                    \vismParagraph{IX.19}{19}{}
                    Another may be controlled in his mental behaviour, and his control in worshiping at shrines, etc., evident to all. For when one who is uncontrolled in mind pays homage at a shrine or at an Enlightenment Tree or to elders, he does not do it carefully, \textcolor{brown}{\textit{[300]}} and he sits in the Dhamma-preaching pavilion with mind astray or nodding, while one whose mind is controlled pays homage carefully and deliberately, listens to the Dhamma attentively, remembering it, and evincing the confidence in his mind through his body or his speech. So another may be only controlled in his mental behaviour, though his bodily and verbal behaviour are not controlled. Then the latter should be ignored and the control in his mental behaviour remembered.

                    \vismParagraph{IX.20}{20}{}
                    But there may be another in whom not even one of these three things is controlled. Then compassion for that person should be aroused thus: “Though he is going about in the human world now, nevertheless after a certain number of days he will find himself in [one of] the eight great hells or the sixteen prominent hells.\footnote{\vismAssertFootnoteCounter{4}\vismHypertarget{IX.n4}{}“The eight great hells beginning with that of Sañjīva (see \textbf{\cite{J-a}V 266}, 270). At each of the four doors of the Great Unmitigated (\emph{Avīci}) Hell there are the four beginning with the Ember (\emph{Kukuḷa}) Hell (\textbf{\cite{M}III 185}), which make up the sixteen prominent hells” (\textbf{\cite{Vism-mhṭ}291}).}” For irritation subsides too through compassion. In yet another all three may be controlled. Then he can remember any of the three in that person, whichever he likes; for the development of loving-kindness towards such a person is easy.

                    \vismParagraph{IX.21}{21}{}
                    And in order to make the meaning of this clear the following sutta from the Book of Fives should be cited in full: “Bhikkhus, there are five ways of dispelling annoyance whereby annoyance arisen in a bhikkhu can be entirely dispelled” (\textbf{\cite{A}III 186–190}).

                    \vismParagraph{IX.22}{22}{}
                    But if irritation still arises in him in spite of his efforts, then he should admonish himself thus:
                    \begin{verse}
                        Suppose an enemy has hurt\\{}
                        You now in what is his domain,\\{}
                        Why try yourself as well to hurt\\{}
                        Your mind?—That is not his domain.
                    \end{verse}

                    \begin{verse}
                        In tears you left your family.\\{}
                        They had been kind and helpful too.\\{}
                        \marginnote{\textcolor{teal}{\footnotesize\{354|296\}}}{}So why not leave your enemy,\\{}
                        The anger that brings harm to you?
                    \end{verse}

                    \begin{verse}
                        This anger that you entertain\\{}
                        Is gnawing at the very roots\\{}
                        Of all the virtues that you guard—\\{}
                        Who is there such a fool as you?
                    \end{verse}

                    \begin{verse}
                        Another does ignoble deeds,\\{}
                        So you are angry—How is this?\\{}
                        Do you then want to copy too\\{}
                        The sort of acts that he commits?
                    \end{verse}

                    \begin{verse}
                        Suppose another, to annoy,\\{}
                        Provokes you with some odious act,\\{}
                        Why suffer anger to spring up,\\{}
                        And do as he would have you do?
                    \end{verse}

                    \begin{verse}
                        If you get angry, then maybe\\{}
                        You make \emph{him }suffer, maybe not;\\{}
                        Though with the hurt that anger brings\\{}
                        \emph{You }certainly are punished now.
                    \end{verse}

                    \begin{verse}
                        If anger-blinded enemies\\{}
                        Set out to tread the path of woe,\\{}
                        Do you by getting angry too\\{}
                        Intend to follow heel to toe?
                    \end{verse}

                    \begin{verse}
                        If hurt is done you by a foe\\{}
                        Because of anger on your part,\\{}
                        Then put your anger down, for why\\{}
                        Should you be harassed groundlessly? \textcolor{brown}{\textit{[301]}}
                    \end{verse}

                    \begin{verse}
                        Since states last but a moment’s time\\{}
                        Those aggregates, by which was done\\{}
                        The odious act, have ceased, so now\\{}
                        What is it you are angry with?
                    \end{verse}

                    \begin{verse}
                        Whom shall he hurt, who seeks to hurt\\{}
                        Another, in the other’s absence?\\{}
                        \emph{Your }presence is the cause of hurt;\\{}
                        Why are you angry, then, with \emph{him}?
                    \end{verse}


                    \vismParagraph{IX.23}{23}{}
                    But if resentment does not subside when he admonishes himself thus, then he should review the fact that he himself and the other are owners of their deeds (\emph{kamma}).

                    Herein, he should first review this in himself thus: “Now, what is the point of your getting angry with him? Will not this kamma of yours that has anger as its source lead to your own harm? For you are the owner of your deeds, heir of your deeds, having deeds as your parent, deeds as your kin, deeds as your refuge; you will become the heir of whatever deeds you do (see \textbf{\cite{A}III 186}). And this is not the kind of deed to bring you to full enlightenment, to undeclared enlightenment or \marginnote{\textcolor{teal}{\footnotesize\{355|297\}}}{}to the disciple’s grade, or to any such position as the status of Brahmā or Sakka, or the throne of a Wheel-turning Monarch or a regional king, etc.; but rather this is the kind of deed to lead to your fall from the Dispensation, even to the status of the eaters of scraps, etc., and to the manifold suffering in the hells, and so on. By doing this you are like a man who wants to hit another and picks up a burning ember or excrement in his hand and so first burns himself or makes himself stink.”

                    \vismParagraph{IX.24}{24}{}
                    Having reviewed ownership of deeds in himself in this way, he should review it in the other also: “And what is the point of his getting angry with you? Will it not lead to his own harm? For that venerable one is owner of his deeds, heir of his deeds … he will become the heir of whatever deeds he does. And this is not the kind of deed to bring him to full enlightenment, to undeclared enlightenment or to the disciple’s grade, or to any such position as the status of Brahmā or Sakka, or to the throne of a Wheel-turning Monarch or a regional king, etc.; but rather this is the kind of deed to lead to his fall from the Dispensation, even to the status of the eaters of scraps, etc., and to the manifold suffering in the hells, and so on. By doing this he is like a man who wants to throw dust at another against the wind and only covers himself with it.” For this is said by the Blessed One:
                    \begin{verse}
                        “When a fool hates a man that has no hate,\\{}
                        Is purified and free from every blemish, \textcolor{brown}{\textit{[302]}}\\{}
                        Such evil he will find comes back on him,\\{}
                        As does fine dust thrown up against the wind” (\textbf{\cite{Dhp}125}).
                    \end{verse}


                    \vismParagraph{IX.25}{25}{}
                    But if it still does not subside in him when he reviews ownership of deeds in this way, then he should review the special qualities of the Master’s former conduct.

                    \vismParagraph{IX.26}{26}{}
                    Here is the way of reviewing it: “Now you who have gone forth, is it not a fact that when your Master was a Bodhisatta before discovering full enlightenment, while he was still engaged in fulfilling the perfections during the four incalculable ages and a hundred thousand aeons, he did not allow hate to corrupt his mind even when his enemies tried to murder him on various occasions?

                    \vismParagraph{IX.27}{27}{}
                    “For example, in the Sīlavant Birth Story (\textbf{\cite{J-a}I 261}) when his friends rose to prevent his kingdom of three hundred leagues being seized by an enemy king who had been incited by a wicked minister in whose mind his own queen had sown hate for him, he did not allow them to lift a weapon. Again when he was buried, along with a thousand companions, up to the neck in a hole dug in the earth in a charnel ground, he had no thought of hate. And when, after saving his life by a heroic effort helped by jackals scraping away soil when they had come to devour the corpses, he went with the aid of a spirit to his own bedroom and saw his enemy lying on his own bed, he was not angry but treated him as a friend, undertaking a mutual pledge, and he then exclaimed:
                    \begin{verse}
                        “The brave aspire, the wise will not lose heart;\\{}
                        I see myself as I had wished to be” (\textbf{\cite{J-a}I 267}).
                    \end{verse}


                    \vismParagraph{IX.28}{28}{}
                    \marginnote{\textcolor{teal}{\footnotesize\{356|298\}}}{}“And in the Khantivādin Birth Story he was asked by the stupid king of Kāsi (Benares), ‘What do you preach, monk?’ and he replied, ‘I am a preacher of patience’; and when the king had him flogged with scourges of thorns and had his hands and feet cut off, he felt not the slightest anger (see \textbf{\cite{J-a}III 39}).

                    \vismParagraph{IX.29}{29}{}
                    “It is perhaps not so wonderful that an adult who had actually gone forth into homelessness should have acted in that way; but also as an infant he did so. For in the Cūḷa-Dhammapāla Birth Story his hands and feet were ordered to be lopped off like four bamboo shoots by his father, King Mahāpatāpa, and his mother lamented over him thus:
                    \begin{verse}
                        “Oh, Dhammapāla’s arms are severed\\{}
                        That had been bathed in sandalwood;\\{}
                        He was the heir to all the earth:\\{}
                        O king, my breath is choking me!” (\textbf{\cite{J-a}III 181}). \textcolor{brown}{\textit{[303]}}
                    \end{verse}


                    “Then his father, still not satisfied, commanded that his head be cut off as well. But even then he had not the least trace of hate, since he had firmly resolved thus: ‘Now is the time to restrain your mind; now, good Dhammapāla, be impartial towards these four persons, that is to say, towards your father who is having your head cut off, the man who is beheading you, your lamenting mother, and yourself.’

                    \vismParagraph{IX.30}{30}{}
                    “And it is perhaps not so wonderful that one who had become a human being should have acted in that way; but also as an animal he did so. For while the Bodhisatta was the elephant called Chaddanta he was pierced in the navel by a poisoned shaft. But even then he allowed no hate towards the hunter who had wounded him to corrupt his mind, according as it is said:
                    \begin{verse}
                        The elephant, when struck by the stout shaft,\\{}
                        Addressed the hunter with no hate in mind:\\{}
                        What is your aim? What is the reason why\\{}
                        You kill me thus? What can your purpose be? (\textbf{\cite{J-a}V 51}).
                    \end{verse}


                    “And when the elephant had spoken thus and was told, ‘Sir, I have been sent by the king of Kāsi’s queen to get your tusks,’ in order to fulfil her wish he cut off his own tusks whose gorgeous radiance glittered with the flashes of the six-coloured rays and gave them to him.

                    \vismParagraph{IX.31}{31}{}
                    “And when he was the Great Monkey, the man whom he had pulled out of a rocky chasm thought:
                    \begin{verse}
                        ‘Now, this is food for human kind\\{}
                        Like other forest animals,\\{}
                        So why then should a hungry man\\{}
                        Not kill the ape to eat? [I ask.]
                    \end{verse}

                    \begin{verse}
                        I’ll travel independently\\{}
                        Taking his meat as a provision;\\{}
                        Thus I shall cross the waste, and that\\{}
                        Will furnish my viaticum’ (\textbf{\cite{J-a}V 71}).
                    \end{verse}


                    Then he took up a stone and dashed it on his head. But the monkey looked at him with eyes full of tears and said:
                    \begin{verse}
                        \marginnote{\textcolor{teal}{\footnotesize\{357|299\}}}{}‘Oh, act not so, good sir, or else\\{}
                        The fate you reap will long deter\\{}
                        All others from such deeds as this\\{}
                        That you would do to me today’ (\textbf{\cite{J-a}V 71}).
                    \end{verse}


                    And with no hate in his mind and regardless of his own pain he saw to it that the man reached his journey’s end in safety.

                    \vismParagraph{IX.32}{32}{}
                    “And while he was the royal nāga (serpent) Bhūridatta, \textcolor{brown}{\textit{[304]}} when he had undertaken the Uposatha precepts and was lying on the top of a termite-mound, though he was [caught and] sprinkled with medicinal charms resembling the fire that ushers in the end of an aeon, and was put into a box and treated as a plaything throughout the whole of Jambudīpa, yet he had no trace of hate for that brahman, according as it is said:
                    \begin{verse}
                        ‘While being put into the coffer\\{}
                        And being crushed down with his hand,\\{}
                        I had no hate for Ālambāna\\{}
                        Lest I should break my precept vow’ (\textbf{\cite{Cp}85}).
                    \end{verse}


                    \vismParagraph{IX.33}{33}{}
                    “And when he was the royal nāga Campeyya he let no hate spring up in his mind while he was being cruelly treated by a snake charmer, according as it is said:
                    \begin{verse}
                        “While I was living in the Law\\{}
                        Observing the Uposatha\\{}
                        A snake charmer took me away\\{}
                        To play with at the royal gate.
                    \end{verse}

                    \begin{verse}
                        Whatever hue he might conceive,\\{}
                        Blue and yellow, and red as well,\\{}
                        So in accordance with his thought\\{}
                        I would become what he had wished;
                    \end{verse}

                    \begin{verse}
                        I would turn dry land into water,\\{}
                        And water into land likewise.\\{}
                        Now, had I given way to wrath\\{}
                        I could have seared him into ash,
                    \end{verse}

                    \begin{verse}
                        Had I relaxed mind-mastery\\{}
                        I should have let my virtue lapse;\\{}
                        And one who lets his virtue lapse\\{}
                        Cannot attain the highest goal” (\textbf{\cite{Cp}85}).
                    \end{verse}


                    \vismParagraph{IX.34}{34}{}
                    “And when he was the royal nāga Saṅkhapāla, while he was being carried along on a carrying pole by the sixteen village boys after they had wounded him in eight places with sharp spears and inserted thorn creepers into the wounds’ orifices, and while, after threading a strong rope through his nose, they were causing him great agony by dragging him along bumping his body on the surface of the ground, though he was capable of turning those village boys to cinders with a mere glance, yet he did not even show the least trace of hate on opening his eyes, according as it is said:
                    \begin{verse}
                        \marginnote{\textcolor{teal}{\footnotesize\{358|300\}}}{}‘On the fourteenth and the fifteenth too, Āḷāra,\\{}
                        I regularly kept the Holy Day,\\{}
                        Until there came those sixteen village boys\\{}
                        Bearing a rope and a stout spear as well.
                    \end{verse}

                    \begin{verse}
                        The hunters cleft my nose, and through the slit\\{}
                        They passed a rope and dragged me off like that.\\{}
                        But though I felt such poignant agony,\\{}
                        I let no hate disturb my Holy Day” (\textbf{\cite{J-a}V 172}). \textcolor{brown}{\textit{[305]}}
                    \end{verse}


                    \vismParagraph{IX.35}{35}{}
                    “And he performed not only these wonders but also many others too such as those told in the Mātuposaka Birth Story (\textbf{\cite{J-a}IV 90}). Now, it is in the highest degree improper and unbecoming to you to arouse thoughts of resentment, since you are emulating as your Master that Blessed One who reached omniscience and who has in the special quality of patience no equal in the world with its deities.”

                    \vismParagraph{IX.36}{36}{}
                    But if, as he reviews the special qualities of the Master’s former conduct, the resentment still does not subside in him, since he has long been used to the slavery of defilement, then he should review the suttas that deal with the beginninglessness [of the round of rebirths]. Here is what is said: “Bhikkhus, it is not easy to find a being who has not formerly been your mother … your father … your brother … your sister … your son … your daughter” (\textbf{\cite{S}II 189–190}). Consequently he should think about that person thus: “This person, it seems, as my mother in the past carried me in her womb for ten months and removed from me without disgust, as if it were yellow sandalwood, my urine, excrement, spittle, snot, etc., and played with me in her lap, and nourished me, carrying me about on her hip. And this person as my father went by goat paths and paths set on piles,\footnote{\vismAssertFootnoteCounter{5}\vismHypertarget{IX.n5}{}\emph{Saṅku-patha—}“set on piles”: \textbf{\cite{Vism-mhṭ}(p. 294)} says: “\emph{Saṅku laggāpetvā te ālambhitvā gamanamaggo saṅkupatho.}” This disagrees with PED for this ref.} etc., to pursue the trade of merchant, and he risked his life for me by going into battle in double array, by sailing on the great ocean in ships and doing other difficult things, and he nourished me by bringing back wealth by one means or another thinking to feed his children. And as my brother, sister, son, daughter, this person gave me such and such help. So it is unbecoming for me to harbour hate for him in my mind.”

                    \vismParagraph{IX.37}{37}{}
                    But if he is still unable to quench that thought in this way, then he should review the advantages of loving-kindness thus: “Now, you who have gone forth into homelessness, has it not been said by the Blessed One as follows: ‘Bhikkhus, when the mind-deliverance of loving-kindness is cultivated, developed, much practiced, made the vehicle, made the foundation, established, consolidated, and properly undertaken, eleven blessings can be expected. What are the eleven? A man sleeps in comfort, wakes in comfort, and dreams no evil dreams, he is dear to human beings, he is dear to non-human beings, deities guard him, fire and poison and weapons do not affect him, his mind is easily concentrated, the expression of his face is serene, he dies unconfused, if he penetrates no higher \marginnote{\textcolor{teal}{\footnotesize\{359|301\}}}{}he will be reborn in the Brahmā-world’ (\textbf{\cite{A}V 342}). \textcolor{brown}{\textit{[306]}} If you do not stop this thought, you will be denied these advantages.”

                    \vismParagraph{IX.38}{38}{}
                    But if he is still unable to stop it in this way, he should try resolution into elements. How? “Now, you who have gone forth into homelessness, when you are angry with him, what is it you are angry with? Is it head hairs you are angry with? Or body hairs? Or nails? … Or is it urine you are angry with? Or alternatively, is it the earth element in the head hairs, etc., you are angry with? Or the water element? Or the fire element? Or is it the air element you are angry with? Or among the five aggregates or the twelve bases or the eighteen elements with respect to which this venerable one is called by such and such a name, which then, is it the materiality aggregate you are angry with? Or the feeling aggregate, the perception aggregate, the formations aggregate, the consciousness aggregate you are angry with? Or is it the eye base you are angry with? Or the visible-object base you are angry with? … Or the mind base you are angry with? Or the mental-object base you are angry with? Or is it the eye element you are angry with? Or the visible-object element? Or the eye-consciousness element? … Or the mind element? Or the mental-object element? Or the mind-consciousness element you are angry with?” For when he tries the resolution into elements, his anger finds no foothold, like a mustard seed on the point of an awl or a painting on the air.

                    \vismParagraph{IX.39}{39}{}
                    But if he cannot effect the resolution into elements, he should try the giving of a gift. It can either be given by himself to the other or accepted by himself from the other. But if the other’s livelihood is not purified and his requisites are not proper to be used, it should be given by oneself. And in the one who does this the annoyance with that person entirely subsides. And in the other even anger that has been dogging him from a past birth subsides at the moment, as happened to the senior elder who received a bowl given to him at the Cittalapabbata Monastery by an almsfood-eater elder who had been three times made to move from his lodging by him, and who presented it with these words: “Venerable sir, this bowl worth eight ducats was given me by my mother who is a lay devotee, and it is rightly obtained; let the good lay devotee acquire merit.” So efficacious is this act of giving. And this is said:
                    \begin{verse}
                        A gift for taming the untamed,\\{}
                        A gift for every kind of good;\\{}
                        Through giving gifts they do unbend\\{}
                        And condescend to kindly speech. \textcolor{brown}{\textit{[307]}}
                    \end{verse}

                \subsection[\vismAlignedParas{§40–43}The Breaking Down of the Barriers—The Sign]{The Breaking Down of the Barriers—The Sign}

                    \vismParagraph{IX.40}{40}{}
                    When his resentment towards that hostile person has been thus allayed, then he can turn his mind with loving-kindness towards that person too, just as towards the one who is dear, the very dear friend, and the neutral person. Then he should break down the barriers by practicing loving-kindness over and over again, accomplishing mental impartiality towards the four persons, that is to say, himself, the dear person, the neutral person and the hostile person.

                    \vismParagraph{IX.41}{41}{}
                    \marginnote{\textcolor{teal}{\footnotesize\{360|302\}}}{}The characteristic of it is this. Suppose this person is sitting in a place with a dear, a neutral, and a hostile person, himself being the fourth; then bandits come to him and say, “Venerable sir, give us a bhikkhu,” and on being asked why, they answer, “So that we may kill him and use the blood of his throat as an offering;” then if that bhikkhu thinks, “Let them take this one, or this one,” he has not broken down the barriers. And also if he thinks, “Let them take me but not these three,” he has not broken down the barriers either. Why? Because he seeks the harm of him whom he wishes to be taken and seeks the welfare of the other only. But it is when he does not see a single one among the four people to be given to the bandits and he directs his mind impartially towards himself and towards those three people that he has broken down the barriers. Hence the Ancients said:

                    \vismParagraph{IX.42}{42}{}
                    When he discriminates between
                    \begin{verse}
                        The four, that is himself, the dear,\\{}
                        The neutral, and the hostile one,\\{}
                        Then “skilled” is not the name he gets,\\{}
                        Nor “having amity at will,”\\{}
                        But only “kindly towards beings.”\\{}
                        Now, when a bhikkhu’s barriers\\{}
                        Have all the four been broken down,\\{}
                        He treats with equal amity\\{}
                        The whole world with its deities;\\{}
                        Far more distinguished than the first\\{}
                        Is he who knows no barriers.
                    \end{verse}


                    \vismParagraph{IX.43}{43}{}
                    Thus the sign and access are obtained by this bhikkhu simultaneously with the breaking down of the barriers. But when breaking down of the barriers has been effected, he reaches absorption in the way described under the earth kasiṇa without trouble by cultivating, developing, and repeatedly practicing that same sign.

                    At this point he has attained the first jhāna, which abandons five factors, possesses five factors, is good in three ways, is endowed with ten characteristics, and is accompanied by loving-kindness. And when that has been obtained, then by cultivating, developing, and repeatedly practicing that same sign, he successively reaches the second and third jhānas in the fourfold system, and the second, third and fourth in the fivefold system. \textcolor{brown}{\textit{[308]}}
                \subsection[\vismAlignedParas{§44–76}Texts and Commentary]{Texts and Commentary}

                    \vismParagraph{IX.44}{44}{}
                    Now, it is by means of one of these jhānas beginning with the first that he “Dwells pervading (intent upon) one direction with his heart endued with loving-kindness, likewise the second direction, likewise the third direction, likewise the fourth direction, and so above, below, and around; everywhere and equally he dwells pervading the entire world with his heart endued with loving-kindness, abundant, exalted, measureless, free from enmity, and free from affliction” (Vibh 272; \textbf{\cite{D}I 250}). For this versatility comes about only in one whose consciousness has reached absorption in the first jhāna and the rest.

                    \vismParagraph{IX.45}{45}{}
                    \marginnote{\textcolor{teal}{\footnotesize\{361|303\}}}{}And here \emph{endued with loving-kindness }means possessing loving-kindness. \emph{With his heart }(\emph{cetasā}): with his mind (\emph{cittena}). \emph{One direction}: this refers to anyone direction in which a being is first discerned and means pervasion of the beings included in that one direction. \emph{Pervading}: touching, making his object. \emph{He dwells }(\emph{viharati}): he causes the occurrence of an abiding (\emph{vihāra—}dwelling or continuation) in postures that is devoted to the divine abidings (see \hyperlink{IV.103}{IV.103}{}). \emph{Likewise the second}: just as he dwells pervading anyone direction among those beginning with the eastern one, so he does with the next one, and the third and the fourth, is the meaning.

                    \vismParagraph{IX.46}{46}{}
                    \emph{So above}: in that same way in the upper direction is what is meant. \emph{Below, around}: so too the lower direction and the direction all round. Herein, \emph{below }is underneath, and \emph{around }is in the intermediate directions. So he sends his heart full of loving-kindness back and forth in all directions like a horse in a circus ground. Up to this point specified pervasion with loving-kindness is shown in the discernment of each direction separately.

                    \vismParagraph{IX.47}{47}{}
                    \emph{Everywhere}, etc., is said for the purpose of showing unspecified pervasion. Herein, \emph{everywhere }means in all places. \emph{Equally }(\emph{sabbattatāya}): to all classed as inferior, medium, superior, friendly, hostile, neutral, etc., just as to oneself (\emph{attatā}); equality with oneself (\emph{atta-samatā}) without making the distinction, “This is another being,” is what is meant. Or alternatively, \emph{equally }(\emph{sabbattatāya}) is with the whole state of the mind; not reserving even a little, is what is meant. \textcolor{brown}{\textit{[309]}} \emph{Entire }(\emph{sabbāvant}): possessing all beings (\emph{sabbasattavant}); associated with all beings, is the meaning. \emph{World }is the world of beings.

                    \vismParagraph{IX.48}{48}{}
                    \emph{Endued with loving-kindness }is said again here in order to introduce the synonyms beginning with \emph{abundant}. Or alternatively, \emph{endued with loving-kindness }is repeated because the word \emph{likewise }or the word \emph{so }is not repeated here as it was in the case of the [preceding] specified pervasion. Or alternatively, it is said as a way of concluding. And \emph{abundant }should be regarded here as abundance in pervading. But it is \emph{exalted }in plane [from the sensual-sphere plane to the fine-material-sphere plane], \emph{measureless }through familiarity and through having measureless beings as its object, \emph{free from enmity }through abandonment of ill will and hostility, and \emph{free from affliction }through abandonment of grief; without suffering, is what is meant. This is the meaning of the versatility described in the way beginning, “With his heart endued with loving-kindness.”

                    \vismParagraph{IX.49}{49}{}
                    And just as this versatility is successful only in one whose mind has reached absorption, so too that described in the Paṭisambhidā should be understood to be successful only in one whose mind has reached absorption, that is to say: “The mind-deliverance of loving-kindness is [practiced] with unspecified pervasion in five ways. The mind-deliverance of loving-kindness is [practiced] with specified pervasion in seven ways. The mind-deliverance of loving-kindness is [practiced] with directional pervasion in ten ways” (\textbf{\cite{Paṭis}II 130}).

                    \vismParagraph{IX.50}{50}{}
                    And herein, the mind-deliverance of loving-kindness is [practiced] with unspecified pervasion in these five ways: “May all beings be free from enmity, affliction and anxiety, and live happily. May all breathing things … all creatures \marginnote{\textcolor{teal}{\footnotesize\{362|304\}}}{}… all persons … all those who have a personality be free from enmity, affliction and anxiety, and live happily” (\textbf{\cite{Paṭis}II 130}).

                    \vismParagraph{IX.51}{51}{}
                    The mind-deliverance of loving-kindness is [practiced] with specified pervasion in these seven ways: “May all women be free from enmity, affliction and anxiety and live happily. May all men … all Noble Ones … all not Noble Ones … all deities … all human beings … all in states of loss be free from enmity, affliction and anxiety, and live happily” (\textbf{\cite{Paṭis}II 131}).

                    \vismParagraph{IX.52}{52}{}
                    The mind-deliverance of loving-kindness is [practiced] with directional pervasion in these ten ways: “May all beings in the eastern direction be free from enmity, affliction and anxiety, and live happily. May all beings in the western direction … northern direction … southern direction \textcolor{brown}{\textit{[310]}} … eastern intermediate direction … western intermediate direction … northern intermediate direction … southern intermediate direction … downward direction … upward direction be free from enmity, affliction and anxiety, and live happily. May all breathing things in the eastern direction … May all creatures in the eastern direction … May all persons in the eastern direction … May all who have a personality in the eastern direction … [etc.] … in the upward direction be free from enmity, affliction and anxiety, and live happily. May all women in the eastern direction … May all men in the eastern direction … May all Noble Ones in the eastern direction … May all not Noble Ones in the eastern direction … May all deities in the eastern direction … May all human beings in the eastern direction … May all those in states of loss in the eastern direction … [etc.] … be free from enmity, affliction and anxiety, and live happily” (\textbf{\cite{Paṭis}II 131}).

                    \vismParagraph{IX.53}{53}{}
                    Herein, \emph{all }signifies inclusion without exception. \emph{Beings }(\emph{satta}): they are held (\emph{satta}), gripped (\emph{visatta}) by desire and greed for the aggregates beginning with materiality, thus they are beings (\emph{satta}). For this is said by the Blessed One: “Any desire for matter, Rādha, any greed for it, any delight in it, any craving for it, has held (\emph{satta}) it, has gripped (\emph{visatta}) it, that is why ‘a being’ (\emph{satta}) is said” (\textbf{\cite{S}III 190}). But in ordinary speech this term of common usage is applied also to those who are without greed, just as the term of common usage “palm fan” (\emph{tālavaṇṭa}) is used for different sorts of fans [in general] even if made of split bamboo. However, [in the world] etymologists (\emph{akkhara-cintaka}) who do not consider meaning have it that it is a mere name, while those who do consider meaning have it that a “being” (\emph{satta}) is so called with reference to the “bright principle” (\emph{satta}).\footnote{\vismAssertFootnoteCounter{6}\vismHypertarget{IX.n6}{}\emph{Satta—}“the bright principle”: Skr. \emph{sattva}; one of the three principles in the Sāṅkhya system, the other two being \emph{rajas} (Pali: \emph{rajo}) or turbulence and \emph{tamas} (Pali: \emph{tamo}) or darkness. Not in PED.}

                    \vismParagraph{IX.54}{54}{}
                    \emph{Breathing things }(\emph{pāṇa}): so called because of their state of breathing (\emph{pāṇanatā}); the meaning is, because their existence depends on in-breaths and out-breaths. \emph{Creatures }(\emph{bhūta}): so called because of being (\emph{bhūtatta }= becomeness); the meaning is, because of their being fully become (\emph{sambhūtatta}), because of their being generated (\emph{abhinibbattatta}). \emph{Persons }(\emph{puggala}): “\emph{puṃ}” is what hell is called; they fall (\emph{galanti}) into that, is the meaning. \emph{Personality }(\emph{attabhāva}) is what the physical \marginnote{\textcolor{teal}{\footnotesize\{363|305\}}}{}body is called; or it is just the pentad of aggregates, since it is actually only a concept derived from that pentad of aggregates\footnote{\vismAssertFootnoteCounter{7}\vismHypertarget{IX.n7}{}“Here when the aggregates are not fully understood, there is naming (\emph{abhidhāna}) of them and of the consciousness of them as self (\emph{attā}), that is to say, the physical body or alternatively the five aggregates. ‘Derived from’: apprehending, gripping, making a support. ‘Since it is actually a mere concept’: because of presence (\emph{sabbhāvato}) as a mere concept in what is called a being, though in the highest sense the ‘being’ is non-existent” (\textbf{\cite{Vism-mhṭ}298}). See also \hyperlink{VIII.n11}{Ch. VIII, note 11}{}.} [What is referred to is] included (\emph{pariyāpanna}) in that personality, thus it “has a personality” (\emph{attabhāva-pariyāpanna}). “Included in” is delimited by; “gone into” is the meaning.

                    \vismParagraph{IX.55}{55}{}
                    And all the remaining [terms] should be understood as synonyms for “all beings” used in accordance with ordinary speech as in the case of the term “beings.” Of course, \textcolor{brown}{\textit{[311]}} there are other synonyms too for all “beings,” such as all “folks,” all “souls,” etc.; still it is for clarity’s sake that “The mind-deliverance of loving-kindness is [practiced] with unspecified pervasion in five ways” is said and that only these five are mentioned.

                    \vismParagraph{IX.56}{56}{}
                    Those who would have it that there is not only a mere verbal difference between “beings,” “breathing things,” etc., but also an actual difference in meaning, are contradicted by the mention of unspecified pervasion. So instead of taking the meaning in that way, the unspecified pervasion with loving-kindness is done in any one of these five ways.

                    And here, \emph{may all beings be free from enmity }is one absorption; \emph{free from affliction }is one absorption—free from affliction (\emph{abyābajjha}) is free from afflictedness (\emph{byābādha-rahita});\footnote{\vismAssertFootnoteCounter{8}\vismHypertarget{IX.n8}{}Harvard text reads \emph{byāpādarahita}, which would be renderable as “free from ill will.” \textbf{\cite{Vism-mhṭ}(p. 299)} supports a reading \emph{byābādha}, which seems better.} \emph{free from anxiety }is one absorption—free from anxiety is free from suffering; \emph{may they live happily }is one absorption. Consequently he should do his pervading with loving-kindness according to whichever of these phrases is clear to him. So with the four kinds of absorption in each of the five ways, there are twenty kinds of absorption in unspecified pervasion.

                    \vismParagraph{IX.57}{57}{}
                    In specified pervasion, with the four kinds of absorption in each of the seven ways, there are twenty-eight kinds of absorption. And here “woman” and “man” are stated according to sex; “Noble Ones” and “not Noble Ones” according to Noble Ones and ordinary people; “deities” and “human beings” and “those in states of loss” according to the kind of rebirth.

                    \vismParagraph{IX.58}{58}{}
                    In directional pervasion, with twenty kinds of absorption in each of the directions beginning with “all beings in the eastern direction,” there are two hundred kinds of absorption; and with twenty-eight kinds in each of the directions beginning with “all women in the eastern direction” there are two hundred and eighty kinds; so these make four hundred and eighty kinds of absorption. Consequently all the kinds of absorption mentioned in the Paṭisambhidā amount to five hundred and twenty-eight.

                    \vismParagraph{IX.59}{59}{}
                    \marginnote{\textcolor{teal}{\footnotesize\{364|306\}}}{}So when this meditator develops the mind-deliverance of loving-kindness through any one of these kinds of absorption, he obtains the eleven advantages described in the way beginning, “A man sleeps in comfort” (\hyperlink{IX.37}{§37}{}).

                    \vismParagraph{IX.60}{60}{}
                    Herein, \emph{sleeps in comfort }means that instead of sleeping uncomfortably, turning over and snoring as other people do, he sleeps comfortably, he falls asleep as though entering upon an attainment.

                    \vismParagraph{IX.61}{61}{}
                    He \emph{wakes in comfort}: instead of waking uncomfortably, groaning and yawning and turning over as others do, he wakes comfortably without contortions, like a lotus opening. \textcolor{brown}{\textit{[312]}}

                    \vismParagraph{IX.62}{62}{}
                    He \emph{dreams no evil dreams}: when he sees dreams, he sees only auspicious ones, as though he were worshipping a shrine, as though he were making an offering, as though he were hearing the Dhamma. But he does not see evil dreams as others do, as though being surrounded by bandits, as though being threatened by wild beasts, as though falling into chasms (see XIV, n. 45).

                    \vismParagraph{IX.63}{63}{}
                    \emph{He is dear to human beings}: he is as dear to and beloved by human beings as a necklace worn to hang on the chest, as a wreath adorning the head.

                    \vismParagraph{IX.64}{64}{}
                    \emph{He is dear to non-human beings}: he is just as dear to non-human beings as he is to human beings, as in the Elder Visākha’s case. He was a landowner, it seems, at Pāṭaliputta (Patna). While he was living there he heard this: “The Island of Tambapaṇṇi (Sri Lanka), apparently, is adorned with a diadem of shrines and gleams with the yellow cloth, and there a man can sit or lie wherever he likes; there the climate is favourable, the abodes are favourable, the people are favourable, the Dhamma to be heard is favourable, and all these favourable things are easily obtained there.”

                    \vismParagraph{IX.65}{65}{}
                    He made over his fortune to his wife and children and left his home with only a single ducat (\emph{kahāpaṇa}) sewn into the hem of his garment. He stopped for one month on the sea coast in expectation of a ship, and meanwhile by his skill in trading he made a thousand during the month by buying goods here and selling them there in lawful enterprise.

                    \vismParagraph{IX.66}{66}{}
                    Eventually he came to the Great Monastery [(Mahāvihāra) at Anurādhapura], and there he asked for the going forth into homelessness. When he was being conducted to the chapter house (\emph{sīmā}) for the going-forth ceremony, the purse containing the thousand pieces dropped out from under his belt. When asked “What is that?” he replied, “It is a thousand ducats, venerable sirs.” They told him, “Lay follower, it is not possible to distribute them after the going forth. Distribute them now.” Then he said, “Let none who have come to the scene of Visākha’s going forth depart empty-handed,” and opening [the purse] he strewed them over the chapter house yard, after which he received the going forth and the full admission.

                    \vismParagraph{IX.67}{67}{}
                    When he had acquired five years’ seniority and had become familiar with the two Codes (Pātimokkha; see \hyperlink{III.31}{III.31}{}) he celebrated the \emph{Pavāraṇā }at the end of the Rains, took a meditation subject that suited him, and set out to wander, living for four months in each monastery and doing the duties on a basis of equality with the residents. While he was wandering in this way:
                    \begin{verse}
                        \marginnote{\textcolor{teal}{\footnotesize\{365|307\}}}{}The elder halted in a wood\\{}
                        To scan the tenor of his way;\\{}
                        He thundered forth this roundelay\\{}
                        Proclaiming that he found it good:\\{}
                        So from your full-admission day\\{}
                        Till in this place you paused and stood\\{}
                        No stumbling mars your bhikkhuhood;\\{}
                        Be thankful for such grace, I say. \textcolor{brown}{\textit{[313]}}
                    \end{verse}


                    \vismParagraph{IX.68}{68}{}
                    On his way to Cittalapabbata he came to a road fork and stood wondering which turn to take. Then a deity living in a rock held out a hand pointing out the road to him.

                    \vismParagraph{IX.69}{69}{}
                    He came to the Cittalapabbata Monastery. After he had stayed there for four months he lay down thinking, “In the morning I depart.” Then a deity living in a \emph{maṇila }tree at the end of the walk sat down on a step of the stair and burst into tears. The elder asked, “Who is that?”—“It is I, Maṇiliyā, venerable sir.”—“What are you weeping for?”—“Because you are going away.”—“What good does my living here to you?”—“Venerable sir, as long as you live here non-human beings treat each other kindly. Now, when you are gone, they will start quarrels and loose talk.”\footnote{\vismAssertFootnoteCounter{9}\vismHypertarget{IX.n9}{}For \emph{duṭṭhulla} see \hyperlink{IV.n36}{Ch. IV, note 36}{}. Here the meaning is more likely to be “bad” or “lewd” than “inert.”} The elder said, “If my living here makes you live at peace, that is good,” and so he stayed there another four months. Then he again thought of leaving, but the deity wept as before. And so the elder lived on there, and it was there that he attained Nibbāna.

                    This is how a bhikkhu who abides in loving-kindness is dear to non-human beings.

                    \vismParagraph{IX.70}{70}{}
                    \emph{Deities guard him}: deities guard him as a mother and father guard their child.

                    \vismParagraph{IX.71}{71}{}
                    \emph{Fire, poison and weapons do not affect him}: they do not affect, do not enter into, the body of one who abides in loving-kindness, like the fire in the case of the lay woman devotee Uttarā (see \hyperlink{XII.34}{XII.34}{} and \textbf{\cite{Dhp-a}III 310}), like the poison in the case of the Saṃyutta reciter the Elder Cūḷa-Siva, like the knife in the case of the novice Saṅkicca (see \textbf{\cite{Dhp-a}II 249}); they do not disturb the body, is what is meant.

                    \vismParagraph{IX.72}{72}{}
                    And they tell the story of the cow here too. A cow was giving milk to her calf, it seems. A hunter, thinking “I shall shoot her,” flourished a long-handled spear in his hand and flung it. It struck her body and bounced off like a palm leaf—and that was owing neither to access nor to absorption, but simply to the strength of her consciousness of love for her calf. So mightily powerful is loving-kindness.

                    \vismParagraph{IX.73}{73}{}
                    \emph{His mind is easily concentrated}: the mind of one who abides in loving-kindness is quickly concentrated, there is no sluggishness about it. \textcolor{brown}{\textit{[314]}}

                    \vismParagraph{IX.74}{74}{}
                    \emph{The expression of his face is serene}: his face has a serene expression, like a palmyra fruit loosed from its stem.

                    \vismParagraph{IX.75}{75}{}
                    \marginnote{\textcolor{teal}{\footnotesize\{366|308\}}}{}\emph{He dies unconfused}: there is no dying deluded for one who abides in loving-kindness. He passes away undeluded as if falling asleep.

                    \vismParagraph{IX.76}{76}{}
                    \emph{If he penetrates no higher}: if he is unable to reach higher than the attainment of loving-kindness and attain Arahantship, then when he falls from this life, he reappears in the Brahmā-world as one who wakes up from sleep.

                    This is the detailed explanation of the development of loving-kindness.
            \section[\vismAlignedParas{§77–83}(2) Compassion]{(2) Compassion}

                \vismParagraph{IX.77}{77}{}
                One who wants to develop compassion should begin his task by reviewing the danger in lack of compassion and the advantage in compassion.

                And when he begins it, he should not direct it at first towards the dear, etc., persons; for one who is dear simply retains the position of one who is dear, a very dear companion retains the position of a very dear companion, one who is neutral retains the position of one who is neutral, one who is antipathetic retains the position of one who is antipathetic, and one who is hostile retains the position of one who is hostile. One of the opposite sex and one who is dead are also not the field for it.

                \vismParagraph{IX.78}{78}{}
                In the Vibhaṅga it is said: “And how does a bhikkhu dwell pervading one direction with his heart endued with compassion? Just as he would feel compassion on seeing an unlucky, unfortunate person, so he pervades all beings with compassion” (\textbf{\cite{Vibh}273}). Therefore first of all, on seeing a wretched man, unlucky, unfortunate, in every way a fit object for compassion, unsightly, reduced to utter misery, with hands and feet cut off, sitting in the shelter for the helpless with a pot placed before him, with a mass of maggots oozing from his arms and legs, and moaning, compassion should be felt for him in this way: “This being has indeed been reduced to misery; if only he could be freed from this suffering!”

                But if he does not encounter such a person, then he can arouse compassion for an evil-doing person, even though he is happy, by comparing him to one about to be executed. How?

                \vismParagraph{IX.79}{79}{}
                Suppose a robber has been caught with stolen goods, and in accordance with the king’s command to execute him, the king’s men bind him and lead him off to the place of execution, giving him a hundred blows in sets of four. Then people give him things to chew and eat and also garlands and perfumes, unguents and betel leaves. Although \textcolor{brown}{\textit{[315]}} he goes along eating and enjoying these things as though he were happy and well off, still no one fancies that he is really happy and well off. On the contrary people feel compassion for him, thinking, “This poor wretch is now about to die; every step he takes brings him nearer to the presence of death.” So too a bhikkhu whose meditation subject is compassion should arouse compassion for an [evil-doing] person even if he is happy: “Though this poor wretch is now happy, cheerful, enjoying his wealth, still for want of even one good deed done now in any one of the three doors [of body, speech and mind] he can come to experience untold suffering in the states of loss.”

                \vismParagraph{IX.80}{80}{}
                \marginnote{\textcolor{teal}{\footnotesize\{367|309\}}}{}Having aroused compassion for that person in that way, he should next arouse compassion for a dear person, next for a neutral person, and next for a hostile person, successively in the same way.

                \vismParagraph{IX.81}{81}{}
                But if resentment towards the hostile person arises in the way already described, he should make it subside in the way described under loving-kindness (\hyperlink{IX.14}{§§14}{}–\hyperlink{IX.39}{39}{}).

                And here too when someone has done profitable deeds and the meditator sees or hears that he has been overtaken by one of the kinds of ruin beginning with ruin of health, relatives, property, etc., he deserves the meditator’s compassion; and so he does too in any case, even with no such ruin, thus “In reality he is unhappy,” because he is not exempt from the suffering of the round [of becoming]. And in the way already described the meditator should break down the barriers between the four kinds of people, that is to say, himself, the dear person, the neutral person and the hostile person. Then cultivating that sign, developing it and repeatedly practicing it, he should increase the absorption by the triple and quadruple jhāna in the way already stated under loving-kindness.

                \vismParagraph{IX.82}{82}{}
                But the order given in the Aṅguttara Commentary is that a hostile person should first be made the object of compassion, and when the mind has been made malleable with respect to him, next the unlucky person, next the dear person, and next oneself. That does not agree with the text, “an unlucky, unfortunate person” (\hyperlink{IX.78}{§78}{}).Therefore he should begin the development, break down the barriers, and increase absorption only in the way stated here.

                \vismParagraph{IX.83}{83}{}
                After that, the versatility consisting in the unspecified pervasion in five ways, the specified pervasion in seven ways, and the directional pervasion in ten ways, and the advantages described as “He sleeps in comfort,” etc., should be understood in the same way as given under loving-kindness.

                This is the detailed explanation of the development of compassion. \textcolor{brown}{\textit{[316]}}
            \section[\vismAlignedParas{§84–87}(3) Gladness]{(3) Gladness}

                \vismParagraph{IX.84}{84}{}
                One who begins the development of gladness\footnote{\vismAssertFootnoteCounter{10}\vismHypertarget{IX.n10}{}\emph{Muditā—}“gladness” as one of the divine abidings is always in the sense of gladness at others’ success. Sometimes rendered as “altruistic joy” and “sympathetic gladness.”} should not start with the dear person and the rest; for a dear person is not the proximate cause of gladness merely in virtue of dearness, how much less the neutral and the hostile person. One of the opposite sex and one who is dead are also not the field for it.

                \vismParagraph{IX.85}{85}{}
                However, the very dear companion can be the proximate cause for it—one who in the commentaries is called a “boon companion,” for he is constantly glad: he laughs first and speaks afterwards. So he should be the first to be pervaded with gladness. Or on seeing or hearing about a dear person being happy, cheerful and glad, gladness can be aroused thus: “This being is indeed glad. How good, how excellent!” For this is what is referred to in the Vibhaṅga: “And how does a bhikkhu dwell pervading one direction with his heart endued \marginnote{\textcolor{teal}{\footnotesize\{368|310\}}}{}with gladness? Just as he would be glad on seeing a dear and beloved person, so he pervades all beings with gladness” (\textbf{\cite{Vibh}274}).

                \vismParagraph{IX.86}{86}{}
                But if his boon companion or the dear person was happy in the past but is now unlucky and unfortunate, then gladness can still be aroused by remembering his past happiness and apprehending the glad aspect in this way: “In the past he had great wealth, a great following and he was always glad.” Or gladness can be aroused by apprehending the future glad aspect in him in this way: “In the future he will again enjoy similar success and will go about in gold palanquins, on the backs of elephants or on horseback, and so on.”

                Having thus aroused gladness with respect to a dear person, he can then direct it successively towards a neutral one, and after that towards a hostile one.

                \vismParagraph{IX.87}{87}{}
                But if resentment towards the hostile one arises in him in the way already described, he should make it subside in the same way as described under loving-kindness (\hyperlink{IX.14}{§§14}{}–\hyperlink{IX.39}{39}{}).

                He should break down the barriers by means of mental impartiality towards the four, that is, towards these three and himself. And by cultivating that sign, developing and repeatedly practicing it, he should increase the absorption to triple and quadruple jhāna in the way already stated under loving-kindness.

                Next, the versatility consisting in unspecified pervasion in five ways, specified pervasion in seven ways, and directional pervasion in ten ways, and also the advantages described as “He sleeps in comfort,” etc., should be understood in the same way as stated under loving-kindness.

                This is the detailed explanation of the development of gladness.

                \textcolor{brown}{\textit{[317]}}
            \section[\vismAlignedParas{§88–90}(4) Equanimity]{(4) Equanimity}

                \vismParagraph{IX.88}{88}{}
                One who wants to develop equanimity must have already obtained the triple or quadruple jhāna in loving-kindness, and so on. He should emerge from the third jhāna [in the fourfold reckoning], after he has made it familiar, and he should see danger in the former [three divine abidings] because they are linked with attention given to beings’ enjoyment in the way beginning “May they be happy,” because resentment and approval are near, and because their association with joy is gross. And he should also see the advantage in equanimity because it is peaceful. Then he should arouse equanimity (\emph{upekkhā}) by looking on with equanimity (\emph{ajjhupekkhitvā}) at a person who is normally neutral; after that at a dear person, and the rest. For this is said: “And how does a bhikkhu dwell pervading one direction with his heart endued with equanimity? Just as he would feel equanimity on seeing a person who was neither beloved nor unloved, so he pervades all beings with equanimity” (\textbf{\cite{Vibh}275}).

                \vismParagraph{IX.89}{89}{}
                Therefore he should arouse equanimity towards the neutral person in the way already stated. Then, through the neutral one, he should break down the barriers in each case between the three people, that is, the dear person, then the \marginnote{\textcolor{teal}{\footnotesize\{369|311\}}}{}boon companion, and then the hostile one, and lastly himself. And he should cultivate that sign, develop and repeatedly practice it.

                \vismParagraph{IX.90}{90}{}
                As he does so the fourth jhāna arises in him in the way described under the earth kasiṇa.

                But how then? Does this arise in one in whom the third jhāna has already arisen on the basis of the earth kasiṇa, etc.? It does not. Why not? Because of the dissimilarity of the object. It arises only in one in whom the third jhāna has arisen on the basis of loving-kindness, etc., because the object is similar.

                But after that the versatility and the obtaining of advantages should be understood in the same way as described under loving-kindness.

                This is the detailed explanation of the development of equanimity.
            \section[\vismAlignedParas{§91–124}General]{General}

                \vismParagraph{IX.91}{91}{}
                Now, having thus known these divine abidings
                \begin{verse}
                    Told by the Divine One (\emph{brahmā}) supremely [wise],\\{}
                    There is this general explanation too\\{}
                    Concerning them that he should recognize.
                \end{verse}

                \subsection[\vismAlignedParas{§92}Meanings]{Meanings}

                    \vismParagraph{IX.92}{92}{}
                    Now, as to the meaning firstly of loving-kindness, compassion, gladness and equanimity: it fattens (\emph{mejjati}), thus it is loving-kindness (\emph{mettā}); it is solvent (\emph{siniyhati}) is the meaning. Also: it comes about with respect to a friend (\emph{mitta}), \textcolor{brown}{\textit{[318]}} or it is behaviour towards a friend, thus it is loving-kindness (\emph{mettā}).

                    When there is suffering in others it causes (\emph{karoti}) good people’s hearts to be moved (\emph{kampana}), thus it is compassion (\emph{karuṇā}). Or alternatively, it combats (\emph{kiṇāti})\footnote{\vismAssertFootnoteCounter{11}\vismHypertarget{IX.n11}{}\emph{Kiṇāti—}“it combats”: Skr. \emph{kṛnāti—}to injure or kill. PED gives this ref. under ordinary meaning “to buy,” which is wrong.} others’ suffering, attacks and demolishes it, thus it is compassion. Or alternatively, it is scattered (\emph{kiriyati}) upon those who suffer, it is extended to them by pervasion, thus it is compassion (\emph{karuṇā}).

                    Those endowed with it are glad (\emph{modanti}), or itself is glad (\emph{modati}), or it is the mere act of being glad (\emph{modana}), thus it is gladness (\emph{muditā}).

                    It looks on at (\emph{upekkhati}), abandoning such interestedness as thinking “May they be free from enmity” and having recourse to neutrality, thus it is equanimity (\emph{upekkhā}).
                \subsection[\vismAlignedParas{§93–96}Characteristic etc.]{Characteristic etc.}

                    \vismParagraph{IX.93}{93}{}
                    As to the characteristic, etc., \emph{loving-kindness }is characterized here as promoting the aspect of welfare. Its function is to prefer welfare. It is manifested as the removal of annoyance. Its proximate cause is seeing loveableness in beings. It succeeds when it makes ill will subside, and it fails when it produces (selfish) affection.

                    \vismParagraph{IX.94}{94}{}
                    \marginnote{\textcolor{teal}{\footnotesize\{370|312\}}}{}\emph{Compassion }is characterized as promoting the aspect of allaying suffering. Its function resides in not bearing others’ suffering. It is manifested as non-cruelty. Its proximate cause is to see helplessness in those overwhelmed by suffering. It succeeds when it makes cruelty subside and it fails when it produces sorrow.

                    \vismParagraph{IX.95}{95}{}
                    \emph{Gladness }is characterized as gladdening (produced by others’ success).\footnote{\vismAssertFootnoteCounter{12}\vismHypertarget{IX.n12}{}So \textbf{\cite{Vism-mhṭ}309}.} Its function resides in being unenvious. It is manifested as the elimination of aversion (boredom). Its proximate cause is seeing beings, success. It succeeds when it makes aversion (boredom) subside, and it fails when it produces merriment.

                    \vismParagraph{IX.96}{96}{}
                    \emph{Equanimity }is characterized as promoting the aspect of neutrality towards beings. Its function is to see equality in beings. It is manifested as the quieting of resentment and approval. Its proximate cause is seeing ownership of deeds (kamma) thus: “Beings are owners of their deeds. Whose\footnote{\vismAssertFootnoteCounter{13}\vismHypertarget{IX.n13}{}All texts read \emph{kassa} (whose), which is confirmed in the quotation translated in note 20. It is tempting, in view of the context, to read \emph{kammassa} (kamma’s), but there is no authority for it. The statement would then be an assertion instead of a question.} [if not theirs] is the choice by which they will become happy, or will get free from suffering, or will not fall away from the success they have reached?” It succeeds when it makes resentment and approval subside, and it fails when it produces the equanimity of unknowing, which is that [worldly-minded indifference of ignorance] based on the house life.
                \subsection[\vismAlignedParas{§97}Purpose]{Purpose}

                    \vismParagraph{IX.97}{97}{}
                    The general purpose of these four divine abidings is the bliss of insight and an excellent [form of future] existence. That peculiar to each is respectively the warding off of ill will, and so on. For here loving-kindness has the purpose of warding off ill will, while the others have the respective purposes of warding off cruelty, aversion (boredom), and greed or resentment. And this is said too: “For this is the escape from ill will, friends, that is to say, the mind-deliverance of loving-kindness … For this is the escape from cruelty, friends, that is to say, the mind-deliverance of compassion … For this is the escape from boredom, friends, that is to say, the mind-deliverance of gladness … For this is the escape from greed, friends, that is to say, the mind-deliverance of equanimity” (\textbf{\cite{D}III 248}).
                \subsection[\vismAlignedParas{§98–101}The Near and Far Enemies]{The Near and Far Enemies}

                    \vismParagraph{IX.98}{98}{}
                    And here each one has two enemies, one near and one far.

                    The divine abiding of \emph{loving-kindness }\textcolor{brown}{\textit{[319]}} has greed as its near enemy,\footnote{\vismAssertFootnoteCounter{14}\vismHypertarget{IX.n14}{}“Greed is the near enemy of loving-kindness since it is able to corrupt owing to its similarity, like an enemy masquerading as a friend” (\textbf{\cite{Vism-mhṭ}309}).} since both share in seeing virtues. Greed behaves like a foe who keeps close by a man, and it easily finds an opportunity. So loving-kindness should be well \marginnote{\textcolor{teal}{\footnotesize\{371|313\}}}{}protected from it. And ill will, which is dissimilar to the similar greed, is its far enemy like a foe ensconced in a rock wilderness. So loving-kindness must be practiced free from fear of that; for it is not possible to practice loving-kindness and feel anger simultaneously (see \textbf{\cite{D}III 247–248}).

                    \vismParagraph{IX.99}{99}{}
                    \emph{Compassion }has grief based on the home life as its near enemy, since both share in seeing failure. Such grief has been described in the way beginning, “When a man either regards as a privation failure to obtain visible objects cognizable by the eye that are sought after, desired, agreeable, gratifying and associated with worldliness, or when he recalls those formerly obtained that are past, ceased and changed, then grief arises in him. Such grief as this is called grief based on the home life” (\textbf{\cite{M}III 218}). And cruelty, which is dissimilar to the similar grief, is its far enemy. So compassion must be practiced free from fear of that; for it is not possible to practice compassion and be cruel to breathing things simultaneously.

                    \vismParagraph{IX.100}{100}{}
                    \emph{Gladness }has joy based on the home life as its near enemy, since both share in seeing success. Such joy has been described in the way beginning, “When a man either regards as gain the obtaining of visible objects cognizable by the eye that are sought … and associated with worldliness, or recalls those formerly obtained that are past, ceased, and changed, then joy arises in him. Such joy as this is called joy based on the home life” (\textbf{\cite{M}III 217}). And aversion (boredom), which is dissimilar to the similar joy, is its far enemy. So gladness should be practiced free from fear of that; for it is not possible to practice gladness and be discontented with remote abodes and things connected with the higher profitableness simultaneously.

                    \vismParagraph{IX.101}{101}{}
                    \emph{Equanimity }has the equanimity of unknowing based on the home life as its near enemy, since both share in ignoring faults and virtues. Such unknowing has been described in the way beginning, “On seeing a visible object with the eye equanimity arises in the foolish infatuated ordinary man, in the untaught ordinary man who has not conquered his limitations, who has not conquered future [kamma] result, who is unperceiving of danger. Such equanimity as this does not surmount the visible object. Such equanimity as this is called equanimity based on the home life” (\textbf{\cite{M}III 219}). And greed and resentment, which are dissimilar to the similar unknowing, are its far enemies. Therefore equanimity must be practiced free from fear of that; \textcolor{brown}{\textit{[320]}} for it is not possible to look on with equanimity and be inflamed with greed or be resentful\footnote{\vismAssertFootnoteCounter{15}\vismHypertarget{IX.n15}{}\emph{Paṭihaññati—}“to be resentful”: not in PED; the verb has been needed to correspond to “resentment” (\emph{paṭigha}), as the verb, “to be inflamed with greed” (\emph{rajjati}) corresponds with “greed” (\emph{rāga}).}simultaneously.
                \subsection[\vismAlignedParas{§102}The Beginning, Middle and End, Etc.]{The Beginning, Middle and End, Etc.}

                    \vismParagraph{IX.102}{102}{}
                    Now, zeal consisting in desire to act is the beginning of all these things. Suppression of the hindrances, etc., is the middle. Absorption is the end. Their \marginnote{\textcolor{teal}{\footnotesize\{372|314\}}}{}object is a single living being or many living beings, as a mental object consisting in a concept.
                \subsection[\vismAlignedParas{§103}The Order in Extension]{The Order in Extension}

                    \vismParagraph{IX.103}{103}{}
                    The extension of the object takes place either in access or in absorption. Here is the order of it. Just as a skilled ploughman first delimits an area and then does his ploughing, so first a single dwelling should be delimited and loving-kindness developed towards all beings there in the way beginning, “In this dwelling may all beings be free from enmity.” When his mind has become malleable and wieldy with respect to that, he can then delimit two dwellings. Next he can successively delimit three, four, five, six, seven, eight, nine, ten, one street, half the village, the whole village, the district, the kingdom, one direction, and so on up to one world-sphere, or even beyond that, and develop loving-kindness towards the beings in such areas. Likewise with compassion and so on. This is the order in extending here.
                \subsection[\vismAlignedParas{§104}The Outcome]{The Outcome}

                    \vismParagraph{IX.104}{104}{}
                    Just as the immaterial states are the outcome of the kasiṇas, and the base consisting of neither perception nor non-perception is the outcome of concentration, and fruition attainment is the outcome of insight, and the attainment of cessation is the outcome of serenity coupled with insight, so the divine abiding of equanimity is the outcome of the first three divine abidings. For just as the gable rafters cannot be placed in the air without having first set up the scaffolding and built the framework of beams, so it is not possible to develop the fourth (jhāna in the fourth divine abiding) without having already developed the third jhāna in the earlier (three divine abidings).
                \subsection[\vismAlignedParas{§105–110}Four Questions]{Four Questions}

                    \vismParagraph{IX.105}{105}{}
                    And here it may be asked: But why are loving-kindness, compassion, gladness, and equanimity, called divine abidings? And why are they only four? And what is their order? And why are they called measureless states in the Abhidhamma?

                    \vismParagraph{IX.106}{106}{}
                    It may be replied: The divineness of the abiding (\emph{brahmavihāratā}) should be understood here in the sense of best and in the sense of immaculate. For these abidings are the best in being the right attitude towards beings. And just as Brahmā gods abide with immaculate minds, so the meditators who associate themselves with these abidings abide on an equal footing with Brahmā gods. So they are called divine abidings in the sense of best and in the sense of immaculate. \textcolor{brown}{\textit{[321]}}

                    \vismParagraph{IX.107}{107}{}
                    Here are the answers to the questions beginning with “Why are they only four?”:
                    \begin{verse}
                        Their number four is due to paths to purity\\{}
                        And other sets of four; their order to their aim\\{}
                        As welfare and the rest. Their scope is found to be\\{}
                        Immeasurable, so “measureless states” their name.
                    \end{verse}


                    \vismParagraph{IX.108}{108}{}
                    \marginnote{\textcolor{teal}{\footnotesize\{373|315\}}}{}For among these, loving-kindness is the way to purity for one who has much ill will, compassion is that for one who has much cruelty, gladness is that for one who has much aversion (boredom), and equanimity is that for one who has much greed. Also attention given to beings is only fourfold, that is to say, as bringing welfare, as removing suffering, as being glad at their success, and as unconcern, [that is to say, impartial neutrality]. And one abiding in the measureless states should practice loving-kindness and the rest like a mother with four sons, namely, a child, an invalid, one in the flush of youth, and one busy with his own affairs; for she wants the child to grow up, wants the invalid to get well, wants the one in the flush of youth to enjoy for long the benefits of youth, and is not at all bothered about the one who is busy with his own affairs. That is why the measureless states are only four as “due to paths to purity and other sets of four.”

                    \vismParagraph{IX.109}{109}{}
                    One who wants to develop these four should practice them towards beings first as the promotion of the aspect of welfare—and loving-kindness has the promotion of the aspect of welfare as its characteristic; and next, on seeing or hearing or judging\footnote{\vismAssertFootnoteCounter{16}\vismHypertarget{IX.n16}{}\emph{Sambhāvetvā—}“judging”: not in this sense in PED. \textbf{\cite{Vism-mhṭ}(p. 313)} explains by \emph{parikappetvā} (conjecturing).} that beings whose welfare has been thus wished for are at the mercy of suffering, they should be practiced as the promotion of the aspect of the removal of suffering—and compassion has the promotion of the aspect of the removal of suffering as its characteristic; and then, on seeing the success of those whose welfare has been wished for and the removal of whose suffering has been wished for, they should be practiced as being glad—and gladness has the act of gladdening as its characteristic; but after that there is nothing to be done and so they should be practiced as the neutral aspect, in other words, the state of an onlooker—and equanimity has the promotion of the aspect of neutrality as its characteristic; therefore, since their respective aims are the aspect of welfare, etc., their order should be understood to correspond, with loving-kindness stated first, then compassion, gladness and equanimity.

                    \vismParagraph{IX.110}{110}{}
                    All of them, however, occur with a measureless scope, for their scope is measureless beings; and instead of assuming a measure such as “Loving-kindness, etc., should be developed only towards a single being, or in an area of such an extent,” they occur with universal pervasion.

                    That is why it was said: \textcolor{brown}{\textit{[322]}}
                    \begin{verse}
                        Their number four is due to paths to purity\\{}
                        And other sets of four; their order to their aim\\{}
                        As welfare and the rest. Their scope is found to be\\{}
                        Immeasurable, so “measureless states” their name.
                    \end{verse}

                \subsection[\vismAlignedParas{§111–118}As Producing Three Jhānas and Four Jhānas]{As Producing Three Jhānas and Four Jhānas}

                    \vismParagraph{IX.111}{111}{}
                    Though they have a single characteristic in having a measureless scope, yet the first three are only of triple and quadruple jhāna [respectively in the fourfold and fivefold reckonings]. Why? Because they are not dissociated from \marginnote{\textcolor{teal}{\footnotesize\{374|316\}}}{}joy. But why are their aims not dissociated from joy? Because they are the escape from ill will, etc., which are originated by grief. But the last one belongs only to the remaining single jhāna. Why? Because it is associated with equanimous feeling. For the divine abiding of equanimity that occurs in the aspect of neutrality towards beings does not exist apart from equanimous [that is to say, neither-painful-nor-pleasant] feeling.

                    \vismParagraph{IX.112}{112}{}
                    However, someone might say this: “It has been said by the Blessed One in the Book of Eights, speaking of the measureless states in general: ‘Next, bhikkhu, you should develop the concentration with applied thought and sustained thought, and you should develop it without applied thought and with sustained thought only, and you should develop it without applied thought and without sustained thought, and you should develop it with happiness, and you should develop it without happiness, and you should develop it accompanied by gratification, and you should develop it accompanied by equanimity’ (\textbf{\cite{A}IV 300}). Consequently all four measureless states have quadruple and quintuple jhāna.”

                    \vismParagraph{IX.113}{113}{}
                    He should be told: “Do not put it like that. For if that were so, then contemplation of the body, etc., would also have quadruple and quintuple jhāna. But there is not even the first jhāna in the contemplation of feeling or in the other two.\footnote{\vismAssertFootnoteCounter{17}\vismHypertarget{IX.n17}{}For which kinds of body contemplation give which kinds of concentration see 8.43 and \textbf{\cite{M-a}I 247}.} So do not misrepresent the Blessed One by adherence to the letter. The Enlightened One’s word is profound and should be taken as it is intended, giving due weight to the teachers.”

                    \vismParagraph{IX.114}{114}{}
                    And the intention here is this: The Blessed One, it seems, was asked to teach the Dhamma thus: “Venerable sir, it would be good if the Blessed One would teach me the Dhamma in brief, so that, having heard the Blessed One’s Dhamma, I may dwell alone, withdrawn, diligent, ardent and self-exerted” (\textbf{\cite{A}IV 299}). But the Blessed One had no confidence yet in that bhikkhu, since although he had already heard the Dhamma he had nevertheless gone on living there instead of going to do the ascetic’s duties, [and the Blessed One expressed his lack of confidence] thus: “So too, some misguided men merely question me, and when the Dhamma is expounded [to them], they still fancy that they need not follow me” (\textbf{\cite{A}IV 299}). However, the bhikkhu had the potentiality for the attainment of Arahantship, and so he advised him again, \textcolor{brown}{\textit{[323]}} saying: “Therefore, bhikkhu, you should train thus: ‘My mind shall be steadied, quite steadied internally, and arisen evil unprofitable things shall not obsess my mind and \marginnote{\textcolor{teal}{\footnotesize\{375|317\}}}{}remain.’ You should train thus” (\textbf{\cite{A}IV 299}). But what is stated in that advice is basic concentration consisting in mere unification of mind\footnote{\vismAssertFootnoteCounter{18}\vismHypertarget{IX.n18}{}\emph{“‘Mere unification of the mind’}: the kind of concentrating (\emph{samādhāna}) that is undeveloped and just obtained by one in pursuit of development. That is called ‘basic concentration,’ however, since it is the basic reason for the kinds of more distinguished concentration to be mentioned later in this connection. This ‘mere unification of the mind’ is intended as momentary concentration as in the passage beginning, ‘I internally settled, steadied, unified and concentrated my mind’ (\textbf{\cite{M}I 116}). For the first unification of the mind is recognized as momentary concentration here as it is in the first of the two successive descriptions: ‘Tireless energy was aroused in me … my mind was concentrated and unified’ followed by ‘Quite secluded from sense desires …’” (\textbf{\cite{M}I 21}) (\textbf{\cite{Vism-mhṭ}314}).} internally in the sense of in oneself (see \hyperlink{XIV.n75}{Ch. XIV, n. 75}{}).

                    \vismParagraph{IX.115}{115}{}
                    After that he told him about its development by means of loving-kindness in order to show that he should not rest content with just that much but should intensify his basic concentration in this way: “As soon as your mind has become steadied, quite steadied internally, bhikkhu, and arisen evil unprofitable things do not obsess your mind and remain, then you should train thus: ‘The mind-deliverance of loving-kindness will be developed by me, frequently practiced, made the vehicle, made the foundation, established, consolidated, and properly undertaken.’ You should train thus, bhikkhu” (\textbf{\cite{A}IV 299–300}), after which he said further: “As soon as this concentration has been thus developed by you, bhikkhu,\footnote{\vismAssertFootnoteCounter{19}\vismHypertarget{IX.n19}{}“‘\emph{Thus developed}’: just as a fire started with wood and banked up with cowdung, dust, etc., although it arrives at the state of a ‘cowdung fire,’ etc., (cf. \textbf{\cite{M}I 259}) is nevertheless called after the original fire that was started with the wood, so too it is the basic concentration that is spoken of here, taking it as banked up with loving-kindness, and so on. ‘In other objects’ means in such objects as the earth kasiṇa” (\textbf{\cite{Vism-mhṭ}315}).} and frequently practiced, then you should develop this concentration with applied thought and sustained thought … and you should develop it accompanied by equanimity” (\textbf{\cite{A}IV 300}).

                    \vismParagraph{IX.116}{116}{}
                    The meaning is this: “Bhikkhu, when this basic concentration has been developed by you by means of loving-kindness, then, instead of resting content with just that much, you should make this basic concentration reach quadruple and quintuple jhāna in other objects by [further] developing it in the way beginning ‘With applied thought.’”

                    \vismParagraph{IX.117}{117}{}
                    And having spoken thus, he further said: “As soon as this concentration has been thus developed by you, bhikkhu, and frequently practiced, then you should train thus: ‘The mind-deliverance of compassion will be developed by me …’ (\textbf{\cite{A}IV 300}), etc., pointing out that “you should effect its [further] development by means of quadruple and quintuple jhāna in other objects, this [further] development being preceded by the remaining divine abidings of compassion and the rest.”

                    \vismParagraph{IX.118}{118}{}
                    Having thus shown how its [further] development by means of quadruple and quintuple jhāna is preceded by loving-kindness, etc., and having told him, “As soon as this concentration has been developed by you, bhikkhu, and frequently practiced, then you should train thus: ‘I shall dwell contemplating the body as a body,’” etc., he concluded the discourse with Arahantship as its culmination thus: “As soon as this concentration has been developed by you, bhikkhu, completely developed, then wherever you go you will go in comfort, wherever you stand you will stand in comfort, wherever \textcolor{brown}{\textit{[324]}} you sit you will sit in comfort, wherever you make your couch you will do so in comfort” (\textbf{\cite{A}IV 301}). From that it must be understood that the [three] beginning with loving-kindness have only triple-quadruple jhāna, and that equanimity has only the single \marginnote{\textcolor{teal}{\footnotesize\{376|318\}}}{}remaining jhāna. And they are expounded in the same way in the Abhidhamma as well.
                \subsection[\vismAlignedParas{§119–124}The Highest Limit of Each]{The Highest Limit of Each}

                    \vismParagraph{IX.119}{119}{}
                    And while they are twofold by way of the triple-quadruple jhāna and the single remaining jhāna, still they should be understood to be distinguishable in each case by a different efficacy consisting in having “beauty as the highest,” etc. For they are so described in the Haliddavasana Sutta, according as it is said: “Bhikkhus, the mind-deliverance of loving-kindness has beauty as the highest, I say … The mind-deliverance of compassion has the base consisting of boundless space as the highest, I say … The mind-deliverance of gladness has the base consisting of boundless consciousness as the highest I say … The mind-deliverance of equanimity has the base consisting of nothingness as the highest, I say” (\textbf{\cite{S}V 119–121}).\footnote{\vismAssertFootnoteCounter{20}\vismHypertarget{IX.n20}{}“The beautiful” (\emph{subha}) is the third of the eight liberations (\emph{vimokkha—}see \textbf{\cite{M}II 12}; \textbf{\cite{M-a}III 255}).}

                    \vismParagraph{IX.120}{120}{}
                    But why are they described in this way? Because each is the respective basic support for each. For beings are unrepulsive to one who abides in loving-kindness. Being familiar with the unrepulsive aspect, when he applies his mind to unrepulsive pure colours such as blue-black, his mind enters into them without difficulty. So loving-kindness is the basic support for the liberation by the beautiful (see \textbf{\cite{M}II 12}; \textbf{\cite{M-a}III 256}), but not for what is beyond that. That is why it is called “having beauty as the highest.”

                    \vismParagraph{IX.121}{121}{}
                    One who abides in compassion has come to know thoroughly the danger in materiality, since compassion is aroused in him when he sees the suffering of beings that has as its material sign (cause) beating with sticks, and so on. So, well knowing the danger in materiality, when he removes whichever kasiṇa [concept he was contemplating], whether that of the earth kasiṇa or another, and applies his mind to the space [that remains (see \hyperlink{X.6}{X.6}{})], which is the escape from materiality, then his mind enters into that [space] without difficulty. So compassion is the basic support for the sphere of boundless space, but not for what is beyond that. That is why it is called “having the base consisting of boundless space as the highest.”

                    \vismParagraph{IX.122}{122}{}
                    When he abides in gladness, his mind becomes familiar with apprehending consciousness, since gladness is aroused in him when he sees beings’ consciousness arisen in the form of rejoicing over some reason for joy. Then when he surmounts the sphere of boundless space that he had already attained in due course and applies his mind to the consciousness that had as its object the sign of space, \textcolor{brown}{\textit{[325]}} his mind enters into it without difficulty. So gladness is the basic support for the base consisting of boundless consciousness, but not for what is beyond that. That is why it is called “having the sphere of boundless consciousness as the highest.”

                    \vismParagraph{IX.123}{123}{}
                    \marginnote{\textcolor{teal}{\footnotesize\{377|319\}}}{}When he abides in equanimity, his mind becomes skilled\footnote{\vismAssertFootnoteCounter{21}\vismHypertarget{IX.n21}{}Reading in both cases “\emph{avijjamāna-gahaṇa-dakkhaṃ cittaṃ},” not “-\emph{dukkhaṃ.” “‘Because it has no more concern} (\emph{ābhoga})’: because it has no further act of being concerned

                            (\emph{ābhujana}) by hoping (\emph{āsiṃsanā}) for their pleasure, etc., thus ‘May they be happy.’ The development of loving-kindness, etc., occurring as it does in the form of hope for beings’ pleasure, etc., makes them its object by directing [the mind] to apprehension of [what is existent in] the ultimate sense [i.e. pleasure, etc.]. But development of equanimity, instead of occurring like that, makes beings its object by simply looking on. But does not the divine abiding of equanimity itself too make beings its object by directing the mind to apprehension of [what is existent in] the ultimate sense, because of the words, ‘Beings are owners of their deeds. Whose [if not theirs] is the choice by which they will become happy …?’ (§96)—Certainly that is so. But that is in the prior stage of development of equanimity. When it has reached its culmination, it makes beings its object by simply looking on. So its occurrence is specially occupied with what is non-existent in the ultimate sense [i.e. beings, which are a concept]. And so skill in apprehending the non-existent should be understood as avoidance of bewilderment due to misrepresentation in apprehension of beings, which avoidance of bewilderment has reached absorption” (\textbf{\cite{Vism-mhṭ}}).} in apprehending what is (in the ultimate sense) non-existent, because his mind has been diverted from apprehension of (what is existent in) the ultimate sense, namely, pleasure, (release from) pain, etc., owing to having no further concern such as “May beings be happy” or “May they be released from pain” or “May they not lose the success they have obtained.” Now his mind has become used to being diverted from apprehension of [what is existent in] the ultimate sense, and his mind has become skilled in apprehending what is non-existent in the ultimate sense, (that is to say, living beings, which are a concept), and so when he surmounts the base consisting of boundless consciousness attained in due course and applies his mind to the absence, which is non-existent as to individual essence, of consciousness, which is a reality (is become—see \textbf{\cite{M}I 260}) in the ultimate sense, then his mind enters into that (nothingness, that non-existence) without difficulty (see \hyperlink{X.32}{X.32}{}). So equanimity is the basic support for the base consisting of nothingness, but not for what is beyond that. That is why it is called “having the base consisting of nothingness as the highest.”

                    \vismParagraph{IX.124}{124}{}
                    When he has understood thus that the special efficacy of each resides respectively in “having beauty as the highest,” etc., he should besides understand how they bring to perfection all the good states beginning with giving. For the Great Beings’ minds retain their balance by giving preference to beings’ welfare, by dislike of beings’ suffering, by desire for the various successes achieved by beings to last, and by impartiality towards all beings. And to all beings they give \emph{gifts}, which are a source a pleasure, without discriminating thus: “It must be given to this one; it must not be given to this one.” And in order to avoid doing harm to beings they undertake the precepts of \emph{virtue}. They practice \emph{renunciation }for the purpose of perfecting their virtue. They cleanse their \emph{understanding }for the purpose of non-confusion about what is good and bad for beings. They constantly arouse \emph{energy}, having beings’ welfare and happiness at heart. When they have acquired heroic fortitude through supreme energy, they become \emph{patient }with beings’ many kinds of faults. They \emph{do not deceive }when \marginnote{\textcolor{teal}{\footnotesize\{378|320\}}}{}promising “We shall give you this; we shall do this for you.” They are unshakably \emph{resolute }upon beings’ welfare and happiness. Through unshakable \emph{loving-kindness }they place them first [before themselves]. Through \emph{equanimity }they expect no reward. Having thus fulfilled the [ten] perfections, these [divine abidings] then perfect all the good states classed as the ten powers, the four kinds of fearlessness, the six kinds of knowledge not shared [by disciples], and the eighteen states of the Enlightened One.\footnote{\vismAssertFootnoteCounter{22}\vismHypertarget{IX.n22}{}For the “ten powers” and “four kinds of fearlessness” see MN 12. For the “six kinds of knowledge not shared by disciples” see \textbf{\cite{Paṭis}I 121f.} For the “eighteen states of the Enlightened One” see \textbf{\cite{Cp-a}}.} This is how they bring to perfection all the good states beginning with giving.

                    The ninth chapter called “The Description of the Divine Abidings” in the Treatise on the Development of Concentration in the \emph{Path of Purification }composed for the purpose of gladdening good people.
        \chapter[The Immaterial States]{The Immaterial States\vismHypertarget{X}\newline{\textnormal{\emph{Āruppa-niddesa}}}}
            \label{X}

            \section[\vismAlignedParas{§1–24}(1) The Base Consisting of Boundless Space]{(1) The Base Consisting of Boundless Space}

                \vismParagraph{X.1}{1}{}
                \marginnote{\textcolor{teal}{\footnotesize\{379|321\}}}{}\textcolor{brown}{\textit{[326]}} Now, as to the four immaterial states mentioned next to the divine abidings (\hyperlink{III.105}{III.105}{}), one who wants firstly to develop the base consisting of boundless space sees in gross physical matter danger through the wielding of sticks, etc., because of the words: “‘It is in virtue of matter that wielding of sticks, wielding of knives, quarrels, brawls and disputes takes place; but that does not exist at all in the immaterial state,’ and in this expectation he enters upon the way to dispassion for only material things, for the fading and cessation of only those” (\textbf{\cite{M}I 410}); and he sees danger in it too through the thousand afflictions beginning with eye disease. So, in order to surmount that, he enters upon the fourth jhāna in any one of the nine kasiṇas beginning with the earth kasiṇa and omitting the limited-space kasiṇa.

                \vismParagraph{X.2}{2}{}
                Now, although he has already surmounted gross physical matter by means of the fourth jhāna of the fine-material sphere, nevertheless he still wants also to surmount the kasiṇa materiality since it is the counterpart of the former. How does he do this?

                \vismParagraph{X.3}{3}{}
                Suppose a timid man is pursued by a snake in a forest and flees from it as fast as he can, then if he sees in the place he has fled to a palm leaf with a streak painted on it or a creeper or a rope or a crack in the ground, he is fearful, anxious and will not even look at it. Suppose again a man is living in the same village as a hostile man who ill-uses him and on being threatened by him with a flogging and the burning down of his house, he goes away to live in another village, then if he meets another man there of similar appearance, voice and manner, he is fearful, anxious and will not even look at him.

                \vismParagraph{X.4}{4}{}
                Here is the application of the similes. The time when the bhikkhu has the gross physical matter as his object is like the time when the men were respectively threatened by the snake and by the enemy. \textcolor{brown}{\textit{[327]}} The time when the bhikkhu surmounts the gross physical matter by means of the fourth jhāna of the fine-material sphere is like the first man’s fleeing as fast as he can and the other man’s going away to another village. The bhikkhu’s observing that even the matter of the kasiṇa is the counterpart of that gross physical matter and his wanting to surmount that also is like the first man’s seeing in the place he had \marginnote{\textcolor{teal}{\footnotesize\{380|322\}}}{}fled to the palm leaf with a streak painted on it, etc., and the other man’s seeing the man who resembled the enemy in the village he had left, and their unwillingness to look owing to fear and anxiety.

                And here the similes of the dog attacked by a boar and that of the \emph{pisāca }goblin and the timid man\footnote{\vismAssertFootnoteCounter{1}\vismHypertarget{X.n1}{}“A dog, it seems, was attacked in the forest by a boar and fled. When it was dusk he saw in the distance a cauldron for boiling rice, and perceiving it as a boar, he fled in fear and terror. Again, a man who was afraid of \emph{pisāca} goblins saw a decapitated palm stump at night in a place that was unfamiliar to him, and perceiving it as a \emph{pisāca }goblin, he fell down in his fear, horror and confusion” (\textbf{\cite{Vism-mhṭ}320}).}should be understood too.

                \vismParagraph{X.5}{5}{}
                So when he has thus become disgusted with (dispassionate towards) the kasiṇa materiality, the object of the fourth jhāna, and wants to get away from it, he achieves mastery in the five ways. Then on emerging from the now familiar fourth jhāna of the fine-material sphere, he sees the danger in that jhāna in this way: “This makes its object the materiality with which I have become disgusted,” and “It has joy as its near enemy,” and “It is grosser than the peaceful liberations.” There is, however, no [comparative] grossness of factors here [as in the case of the four fine-material jhānas]; for the immaterial states have the same two factors as this fine-material [jhāna].

                \vismParagraph{X.6}{6}{}
                When he has seen the danger in that [fine-material fourth jhāna] jhāna in this way and has ended his attachment to it, he gives his attention to the base consisting of boundless space as peaceful. Then, when he has spread out the kasiṇa to the limit of the world-sphere, or as far as he likes, he removes the kasiṇa [materiality] by giving his attention to the space touched by it, [regarding that] as “space” or “boundless space.”

                \vismParagraph{X.7}{7}{}
                When he is removing it, he neither folds it up like a mat nor withdraws it like a cake from a tin. It is simply that he does not advert to it or give attention to it or review it; it is when he neither adverts to it nor gives attention to it nor reviews it, but gives his attention exclusively to the space touched by it, [regarding that] as “space, space,” that he is said to “remove the kasiṇa.”

                \vismParagraph{X.8}{8}{}
                And when the kasiṇa is being removed, it does not roll up or roll away. It is simply that it is called “removed” on account of his non-attention to it, his attention being given to “space, space.” This is conceptualized as the mere space left by the removal of the kasiṇa [materiality]. Whether it is called “space left by the removal of the kasiṇa” or “space touched by the kasiṇa” or “space secluded from the kasiṇa,” it is all the same.

                \vismParagraph{X.9}{9}{}
                He adverts again and again to the sign of the space left by the removal of the kasiṇa \textcolor{brown}{\textit{[328]}} as “space, space,” and strikes at it with thought and applied thought. As he adverts to it again and again and strikes at it with thought and applied thought, the hindrances are suppressed, mindfulness is established and his mind becomes concentrated in access. He cultivates that sign again and again, develops and repeatedly practices it.

                \vismParagraph{X.10}{10}{}
                As he again and again adverts to it and gives attention to it in this way, consciousness belonging to the base consisting of boundless space arises in \marginnote{\textcolor{teal}{\footnotesize\{381|323\}}}{}absorption with the space [as its object], as the consciousness belonging to the fine-material sphere did in the case of the earth kasiṇa, and so on. And here too in the prior stage there are either three or four sensual-sphere impulsions associated with equanimous feeling, while the fourth or the fifth is of the immaterial sphere. The rest is the same as in the case of the earth kasiṇa (\hyperlink{IV.74}{IV.74}{}).

                \vismParagraph{X.11}{11}{}
                There is, however, this difference. When the immaterial-sphere consciousness has arisen in this way, the bhikkhu, who has been formerly looking at the kasiṇa disk with the jhāna eye finds himself looking at only space after that sign has been abruptly removed by the attention given in the preliminary work thus “space, space.” He is like a man who has plugged an opening in a [covered] vehicle, a sack or a pot\footnote{\vismAssertFootnoteCounter{2}\vismHypertarget{X.n2}{}PED, this ref. reads \emph{yānapuṭosā} for \emph{yānapatoḷi}, taking it as one compound (see under \emph{yāna} and \emph{mutoḷī}, but this does not fit the context happily. \textbf{\cite{Vism-mhṭ}(p. 321)} has: “‘\emph{Yānappatoḷikumbhimukhādīnan’ ti oguṇṭhana-sivikādi-yānaṃ mukhaṃ = yāna-mukhaṃ; patoḷiyā kuddakadvārassa mukhaṃ = patoḷi-mukhaṃ; kumbhi-mukhan ti paccekaṃ mukha-saddo sambandhitabbo}.” This necessitates taking \emph{yāna} separately.}with a piece of blue rag or with a piece of rag of some such colour as yellow, red or white and is looking at that, and then when the rag is removed by the force of the wind or by some other agency, he finds himself looking at space.
                \subsection[\vismAlignedParas{§12–24}Text and commentary]{Text and commentary}

                    \vismParagraph{X.12}{12}{}
                    And at this point it is said: “With the complete surmounting (\emph{samatikkamā}) of perceptions of matter, with the disappearance of perceptions of resistance, with non-attention to perceptions of variety, [aware of] ‘unbounded space,’ he enters upon and dwells in the base consisting of boundless space” (\textbf{\cite{Vibh}245}).

                    \vismParagraph{X.13}{13}{}
                    Herein, \emph{complete} is in all aspects or of all [perceptions]; without exception, is the meaning. \emph{Of perceptions of matter}: both (a) of the fine-material jhānas mentioned [here] under the heading of “perception,” and (b) of those things that are their object. For (a) the jhāna of the fine-material sphere is called “matter” in such passages as “Possessed of visible matter he sees instances of matter” (\textbf{\cite{D}II 70}; \textbf{\cite{M}II 12}), and (b) it is its object too [that is called “matter”] in such passages as “He sees instances of visible matter externally … fair and ugly” (\textbf{\cite{D}II 110}; \textbf{\cite{M}II 13}).\footnote{\vismAssertFootnoteCounter{3}\vismHypertarget{X.n3}{}These two quotations refer respectively to the first of the eight liberations and the first of the eight bases of mastery (See \textbf{\cite{M-a}III 255ff.}).}Consequently, here the words “perception of matter” (\emph{rūpa-saññā—}lit. “matter-perceptions”), in the sense of “perceptions about matter,” are used (a) for fine-material jhāna stated thus under the headings of “perceptions.” [Also] (b) it has the label (\emph{saññā}) “matter” (\emph{rūpa}), thus it (the jhāna’s object) is “labelled matter” (\emph{rūpa-saññā}); what is meant is that “matter” is its name. So it should be understood that this is also a term for (b) what is classed as the earth kasiṇa, etc., which is the object of that [jhāna].\footnote{\vismAssertFootnoteCounter{4}\vismHypertarget{X.n4}{}This explanation depends on a play on the word \emph{saññā} as the [subjective] perception and as the [objective] sign, signal or label perceived.} \textcolor{brown}{\textit{[329]}}

                    \vismParagraph{X.14}{14}{}
                    \marginnote{\textcolor{teal}{\footnotesize\{382|324\}}}{}\emph{With the surmounting}: with the fading away and with the cessation. What is meant? With the fading away and with the cessation, both because of the fading away and because of the cessation, either in all aspects or without exception, of these perceptions of matter, reckoned as jhāna, which number fifteen with the [five each of the] profitable, resultant and functional,\footnote{\vismAssertFootnoteCounter{5}\vismHypertarget{X.n5}{}See \hyperlink{XIV.129}{XIV.129}{}, description of perception aggregate, which is classified in the same way as the consciousness aggregate. Those referred to here are the fifteen fine-material kinds, corresponding to nos. (9–l3), (57–61) and (81–85) in Table III.} and also of these things labelled matter, reckoned as objects [of those perceptions], which number nine with the earth kasiṇa, etc., (\hyperlink{X.1}{§1}{}) he enters upon and dwells in the base consisting of boundless space. For he cannot enter upon and dwell in that without completely surmounting perceptions of matter.

                    \vismParagraph{X.15}{15}{}
                    Herein, there is no surmounting of these perceptions in one whose greed for the object [of those perceptions] has not faded away; and when the perceptions have been surmounted, their objects have been surmounted as well. That is why in the Vibhaṅga only the surmounting of the perceptions and not that of the objects is mentioned as follows: “Herein, what are perceptions of matter? They are the perception, perceiving, perceivedness, in one who has attained a fine-material-sphere attainment or in one who has been reborn there or in one who is abiding in bliss there in this present life. These are what are called perceptions of matter. These perceptions of matter are passed, surpassed, surmounted. Hence, ‘With the complete surmounting of perceptions of matter’ is said” (\textbf{\cite{Vibh}261}). But this commentary should be understood to deal also with the surmounting of the object because these attainments have to be reached by surmounting the object; they are not to be reached by retaining the same object as in the first and subsequent jhānas.

                    \vismParagraph{X.16}{16}{}
                    \emph{With the disappearance of perceptions of resistance}: perceptions of resistance are perceptions arisen through the impact of the physical base consisting of the eye, etc., and the respective objects consisting of visible objects etc.; and this is a term for perception of visible objects (\emph{rūpa}) and so on, according as it is said: “Here, what are perceptions of resistance? Perceptions of visible objects, perceptions of sounds, perceptions of odours, perceptions of flavours, perceptions of tangible objects—these are called ‘perceptions of resistance’” (\textbf{\cite{Vibh}261}); with the complete disappearance, the abandoning, the non-arising, of these ten kinds of perceptions of resistance, that is to say, of the five profitable-resultant and five unprofitable-resultant;\footnote{\vismAssertFootnoteCounter{6}\vismHypertarget{X.n6}{}See \hyperlink{XIV.96}{XIV.96f.}{} nos. (34–38) and (50–54) in Table III.} causing their non-occurrence, is what is meant.

                    \vismParagraph{X.17}{17}{}
                    Of course, these are not to be found in one who has entered upon the first jhāna, etc., either; for consciousness at that time does not occur by way of the five doors. Still \textcolor{brown}{\textit{[330]}} the mention of them here should be understood as a recommendation of this jhāna for the purpose of arousing interest in it, just as in the case of the fourth jhāna there is mention of the pleasure and pain already abandoned elsewhere, and in the case of the third path there is mention of the [false] view of personality, etc., already abandoned earlier.

                    \vismParagraph{X.18}{18}{}
                    \marginnote{\textcolor{teal}{\footnotesize\{383|325\}}}{}Or alternatively, though these are also not to be found in one who has attained the fine-material sphere, still their not being there is not due to their having been abandoned; for development of the fine-material sphere does not lead to fading of greed for materiality, and the occurrence of those [fine-material jhānas] is actually dependent on materiality. But this development [of the immaterial] does lead to the fading of greed for materiality. Therefore it is allowable to say that they are actually abandoned here; and not only to say it, but to maintain it absolutely.

                    \vismParagraph{X.19}{19}{}
                    In fact it is because they have not been abandoned already before this that it was said by the Blessed One that sound is a thorn to one who has the first jhāna (\textbf{\cite{A}V 135}). And it is precisely because they are abandoned here that the imperturbability (see Vibh 135) of the immaterial attainments and their state of peaceful liberation are mentioned (\textbf{\cite{M}I 33}), and that Āḷāra Kālāma neither saw the five hundred carts that passed close by him nor heard the sound of them while he was in an immaterial attainment (\textbf{\cite{D}II 130}).

                    \vismParagraph{X.20}{20}{}
                    \emph{With non-attention to perceptions of variety}: either to perceptions occurring with variety as their domain or to perceptions themselves various. For “perceptions of variety” are so called [for two reasons]: firstly, because the kinds of perception included along with the mind element and mind-consciousness element in one who has not attained—which kinds are intended here as described in the Vibhaṅga thus: “Herein, what are perceptions of variety? The perception, perceiving, perceivedness, in one who has not attained and possesses either mind element or mind-consciousness element in one who has not attained and possesses either mind element or mind-consciousness element: these are called ‘perceptions of variety’” (\textbf{\cite{Vibh}261})—occur with respect to a domain that is varied in individual essence with the variety classed as visible-object, sound, etc.; and secondly, because the forty-four kinds of perception—that is to say, eight kinds of sense-sphere profitable perception, twelve kinds of unprofitable perception, eleven kinds of sense-sphere profitable resultant perception, two kinds of unprofitable-resultant perception, and eleven kinds of sense-sphere functional perception—themselves have variety, have various individual essences, and are dissimilar from each other. With the complete non-attention to, non-adverting to, non-reaction to, non-reviewing of, these perceptions of variety; what is meant is that because he does not advert to them, give them attention or review them, therefore …

                    \vismParagraph{X.21}{21}{}
                    And [two things] should be understood: firstly, that their absence is stated here in two ways as “surmounting” and “disappearance” because the earlier perceptions of matter and perceptions of resistance do not exist even in the kind of existence produced by this jhāna on rebirth, let alone when this jhāna is entered upon and dwelt in that existence; \textcolor{brown}{\textit{[331]}} and secondly, in the case of perceptions of variety, “non-attention” to them is said because twenty-seven kinds of perception—that is to say, eight kinds of sense-sphere profitable perception, nine kinds of functional perception, and ten kinds of unprofitable perception—still exist in the kind of existence produced by this jhāna. For when he enters upon and dwells in this jhāna there too, he does so by non-attention to them also, but he has not attained when he does give attention to them.

                    \vismParagraph{X.22}{22}{}
                    \marginnote{\textcolor{teal}{\footnotesize\{384|326\}}}{}And here briefly it should be understood that the abandoning of all fine-material-sphere states is signified by the words \emph{with the surmounting of perceptions of matter}, and the abandoning of and non-attention to all sense-sphere consciousness and its concomitants is signified by the words \emph{with the disappearance of perceptions of resistance, with non-attention to perceptions of variety}.

                    \vismParagraph{X.23}{23}{}
                    \emph{Unbounded space}: here it is called “unbounded” (\emph{ananta}, lit. endless) because neither its end as its arising nor its end as its fall are made known.\footnote{\vismAssertFootnoteCounter{7}\vismHypertarget{X.n7}{}“A [formed] dhamma with an individual essence is delimited by rise and fall because it is produced after having not been, and because after having been it vanishes. But space is called boundless since it has neither rise nor fall because it is a dhamma without individual essence” (\textbf{\cite{Vism-mhṭ}323}).} It is the space left by the removal of the kasiṇa that is called “space.” And here unboundedness (endlessness) should be understood as [referring to] the attention also, which is why it is said in the Vibhaṅga: “He places, settles his consciousness in that space, he pervades unboundedly (\emph{anantaṃ)}, hence ‘unbounded (\emph{ananto}) space’ is said” (\textbf{\cite{Vibh}262}).

                    \vismParagraph{X.24}{24}{}
                    \emph{He enters upon and dwells in the base consisting of boundless space}: it has no bound (\emph{anta}), and thus it is unbounded (\emph{ananta}). What is spatially unbounded (\emph{ākāsaṃ anantaṃ}) is unbounded space (\emph{ākāsānantaṃ}). Unbounded space is the same as boundless space (\emph{ākāsānañcaṃ—}lit. space-boundlessness). That “boundless space” is a “base” (\emph{āyatana}) in the sense of habitat for the jhāna whose nature it is to be associated with it, as the “deities’ base” is for deities, thus it is the “base consisting of boundless space” (\emph{ākāsānañcāyatana}). \emph{He enters and dwells in}: having reached that base consisting of boundless space, having caused it to be produced, he dwells (\emph{viharati}) with an abiding (\emph{vihāra}) consisting in postures that are in conformity with it.

                    This is the detailed explanation of the base consisting of boundless space as a meditation subject.
            \section[\vismAlignedParas{§25–31}(2) The Base Consisting of Boundless Consciousness]{(2) The Base Consisting of Boundless Consciousness}

                \vismParagraph{X.25}{25}{}
                When he wants to develop the base consisting of boundless consciousness, he must first achieve mastery in the five ways in the attainment of the base consisting of boundless space. Then he should see the danger in the base consisting of boundless space in this way: “This attainment has fine-material jhāna as its near enemy, and it is not as peaceful as the base consisting of boundless consciousness.” So having ended his attachment to that, he should give his attention to the base consisting of boundless consciousness as peaceful, adverting again and again as “consciousness, consciousness” to the consciousness that occurred pervading that space [as its object]. He should give it attention, review it, and strike at it with applied and sustained thought; \textcolor{brown}{\textit{[332]}} but he should not give attention [simply] in this way “boundless, boundless.”\footnote{\vismAssertFootnoteCounter{8}\vismHypertarget{X.n8}{}“He should not give attention to it only as ‘Boundless, boundless;’ instead of developing it thus, he should give attention to it as ‘Boundless consciousness, boundless consciousness’ or as ‘Consciousness, consciousness’” (\textbf{\cite{Vism-mhṭ}324}).}

                \vismParagraph{X.26}{26}{}
                \marginnote{\textcolor{teal}{\footnotesize\{385|327\}}}{}As he directs his mind again and again on to that sign in this way, the hindrances are suppressed, mindfulness is established, and his mind becomes concentrated in access. He cultivates that sign again and again, develops and repeatedly practices it. As he does so, consciousness belonging to the base consisting of boundless consciousness arises in absorption with the [past] consciousness that pervaded the space [as its object], just as that belonging to the base consisting of boundless space did with the space [as its object]. But the method of explaining the process of absorption should be understood in the way already described.
                \subsection[\vismAlignedParas{§27–31}Text and commentary]{Text and commentary}

                    \vismParagraph{X.27}{27}{}
                    And at this point it is said: “By completely surmounting (\emph{samatikkamma}) the base consisting of boundless space, [aware of] ‘unbounded consciousness,’ he enters upon and dwells in the base consisting of boundless consciousness” (\textbf{\cite{Vibh}245}).

                    \vismParagraph{X.28}{28}{}
                    Herein, \emph{completely} is as already explained. By … \emph{surmounting the base consisting of boundless space}: the jhāna is called the “base consisting of boundless space” in the way already stated (\hyperlink{X.24}{§24}{}), and its object is so called too. For the object, too, is “boundless space” (\emph{ākāsānañcaṃ}) in the way already stated (\hyperlink{X.24}{§24}{}), and then, because it is the object of the first immaterial jhāna, it is its “base” in the sense of habitat, as the “deities’ base” is for deities, thus it is the “base consisting of boundless space.” Likewise: it is “boundless space,” and then, because it is the cause of the jhāna’s being of that species, it is its “base” in the sense of locality of the species, as Kambojā is the “base” of horses, thus it is the “base consisting of boundless space” in this way also. So it should be understood that the words, “By … surmounting the base consisting of boundless space” include both [the jhāna and its object] together, since this base consisting of boundless consciousness is to be entered upon and dwelt in precisely by surmounting, by causing the non-occurrence of, and by not giving attention to, both the jhāna and its object.

                    \vismParagraph{X.29}{29}{}
                    \emph{Unbounded consciousness}: What is meant is that he gives his attention thus “unbounded consciousness” to that same consciousness that occurred in pervading [as its object the space] as “unbounded space.” Or “unbounded” refers to the attention. For when he gives attention without reserve to the consciousness that had the space as its object, then the attention he gives to it is “unbounded.”

                    \vismParagraph{X.30}{30}{}
                    For it is said in the Vibhaṅga: “‘Unbounded consciousness’: he gives attention to that same space pervaded by consciousness, he pervades boundlessly, hence ‘unbounded consciousness’ is said” (\textbf{\cite{Vibh}262}). But in that passage (\emph{taṃ yeva ākāsaṃ viññāṇena phuṭaṃ}) the instrumental case “by consciousness” must be understood in the sense of accusative; for the teachers of the commentary explain its meaning in that way. What is meant by “He pervades boundlessly” is that “he gives attention to that same consciousness which had pervaded that space” (\emph{taṃ yeva ākāsaṃ phuṭaṃ viññāṇaṃ}).

                    \vismParagraph{X.31}{31}{}
                    \marginnote{\textcolor{teal}{\footnotesize\{386|328\}}}{}\emph{He enters upon and dwells in the base consisting of boundless consciousness}: \textcolor{brown}{\textit{[333]}} it has no bound (\emph{anta}, lit. end), thus it is unbounded (\emph{ananta}). What is unbounded is boundless (\emph{ānañca} lit. unboundedness), and unbounded consciousness is called “boundless consciousness,” that is “\emph{viññāṇañcaṃ}” [in the contracted form] instead of “\emph{viññāṇānañcaṃ}” [which is the full number of syllables]. This is an idiomatic form. That boundless consciousness (\emph{viññāṇañca}) is the base (\emph{āyatana}) in the sense of foundation for the jhāna whose nature it is to be associated with it, as the “deities’ base” is for deities, thus it is the “base consisting of boundless consciousness” (\emph{viññāṇañcāyatana}). The rest is the same as before.

                    This is the detailed explanation of the base consisting of boundless consciousness as a meditation subject.
            \section[\vismAlignedParas{§32–39}(3) The Base Consisting of Nothingness]{(3) The Base Consisting of Nothingness}

                \vismParagraph{X.32}{32}{}
                When he wants to develop the base consisting of nothingness, he must first achieve mastery in the five ways in the attainment in the base consisting of boundless consciousness. Then he should see the danger in the base consisting of boundless consciousness in this way: “This attainment has the base consisting of boundless space as its near enemy, and it is not as peaceful as the base consisting of nothingness.” So having ended his attachment to that, he should give his attention to the base consisting of nothingness as peaceful. He should give attention to the [present] non-existence, voidness, secluded aspect, of that same [past] consciousness belonging to the base consisting of boundless space, which became the object of [the consciousness belonging to] the base consisting of boundless consciousness. How does he do this?

                \vismParagraph{X.33}{33}{}
                Without giving [further] attention to that consciousness, he should [now] advert again and again in this way, “there is not, there is not,” or “void, void,” or “secluded, secluded,” and give his attention to it, review it, and strike at it with thought and applied thought.

                \vismParagraph{X.34}{34}{}
                As he directs his mind on to that sign thus, the hindrances are suppressed, mindfulness is established, and his mind becomes concentrated in access. He cultivates that sign again and again, develops and repeatedly practices it. As he does so, consciousness belonging to the base consisting of nothingness arises in absorption, making its object the void, secluded, non-existent state of that same [past] exalted consciousness that occurred in pervading the space, just as the [consciousness belonging to the] base consisting of boundless consciousness did the [then past] exalted consciousness that had pervaded the space. And here too the method of explaining the absorption should be understood in the way already described.

                \vismParagraph{X.35}{35}{}
                But there is this difference. Suppose a man sees a community of bhikkhus gathered together in a meeting hall or some such place and then goes elsewhere; then after the bhikkhus have risen at the conclusion of the business for which they had met and have departed, the man comes back, and as he stands in the doorway looking at that place again, he sees it only as void, he sees it only as secluded, he does not think, “So many bhikkhus have died, so many have left the \marginnote{\textcolor{teal}{\footnotesize\{387|329\}}}{}district,” but rather \textcolor{brown}{\textit{[334]}} he sees only the non-existence thus, “This is void, secluded”—so too, having formerly dwelt seeing with the jhāna eye belonging to the base consisting of boundless consciousness the [earlier] consciousness that had occurred making the space its object, [now] when that consciousness has disappeared owing to his giving attention to the preliminary work in the way beginning, “There is not, there is not,” he dwells seeing only its non-existence, in other words, its departedness when this consciousness has arisen in absorption.
                \subsection[\vismAlignedParas{§36–39}Text and commentary]{Text and commentary}

                    \vismParagraph{X.36}{36}{}
                    And at this point it is said: “By completely surmounting the base consisting of boundless consciousness, [aware that] ‘There is nothing,’ he enters upon and dwells in the base consisting of nothingness” (\textbf{\cite{Vibh}245}).

                    \vismParagraph{X.37}{37}{}
                    Herein, \emph{completely} is as already explained. By … \emph{surmounting the base consisting of boundless consciousness}: here too the jhāna is called the “base consisting of boundless consciousness” in the way already stated, and its object is so-called too. For the object too is “boundless consciousness” (\emph{viññāṇañcaṃ}) in the way already stated, and then, because it is the object of the second immaterial jhāna, it is its “base” in the sense of habitat, as the “deities’ base” is for deities, thus it is the “base consisting of boundless consciousness.” Likewise it is “boundless consciousness,” and then because it is the cause of the jhāna’s being of that species, it is its “base” in the sense of locality of the species, as Kambojā is the “base” of horses, thus it is the “base consisting of boundless consciousness” in this way also. So it should be understood that the words, “By … surmounting the base consisting of boundless consciousness” include both [the jhāna and its object] together, since this base consisting of nothingness is to be entered upon and dwelt in precisely by surmounting, by causing the non-occurrence of, by not giving attention to, both jhāna and its object.

                    \vismParagraph{X.38}{38}{}
                    \emph{There is nothing }(\emph{natthi kiñci}): what is meant is that he gives his attention thus, “there is not, there is not,” or “void, void,” or “secluded, secluded.” It is said in the Vibhaṅga: “‘There is nothing’: he makes that same consciousness non-existent, makes it absent, makes it disappear, sees that ‘there is nothing’, hence ‘there is nothing’ is said” (\textbf{\cite{Vibh}262}), which is expressed in a way that resembles comprehension [by insight] of liability to destruction, nevertheless the meaning should be understood in the way described above. For the words “He makes that same consciousness non-existent, makes it absent, makes it disappear” are said of one who does not advert to it or gives attention to it or review it, and only gives attention to its non-existence, its voidness, its secludedness; they are not meant in the other way (Cf. \hyperlink{XXI.17}{XXI.17}{}).

                    \vismParagraph{X.39}{39}{}
                    \emph{He enters upon and dwells in the base consisting of nothingness}: it has no owning (\emph{kiñcana}),\footnote{\vismAssertFootnoteCounter{9}\vismHypertarget{X.n9}{}There is a play on the words \emph{natthi kiñci} (“there is nothing”) and \emph{akiñcana} (“non-owning”). At \textbf{\cite{M}I 298} there occurs the expression “\emph{Rāgo kho āvuso kiñcano} (greed, friend, is an owning),” which is used in connection with this attainment. The commentary (\textbf{\cite{M-a}II 354}) says “\emph{Rāgo uppajjitvā puggalaṃ kiñcati, maddati, palibujjhati, tasmā kiñcano ti vutto} (greed having arisen owns, presses, impedes, a person, that is why it is called an owning)” (Cf. \textbf{\cite{M-a}I 27}; also \hyperlink{XXI.53}{XXI.53}{} and note 19). \textbf{\cite{Vism-mhṭ}(p. 327)} here says “\emph{Kiñcanan ti kiñci pi}.” The word \emph{kiñcati} is not in PED.} this it is non-owning (\emph{akiñcana}); what is meant is that it has not even \marginnote{\textcolor{teal}{\footnotesize\{388|330\}}}{}the mere act of its dissolution remaining. The state (essence) of non-owning is nothingness (\emph{ākiñcañña}). This is a term for the disappearance of the consciousness belonging to the base consisting of boundless space. \textcolor{brown}{\textit{[335]}} That nothingness is the “base” in the sense of foundation for that jhāna, as the “deities’ base” is for deities, thus it is the “base consisting of nothingness.” The rest is as before.

                    This is the detailed explanation of the base consisting of nothingness as a meditation subject.
            \section[\vismAlignedParas{§40–55}(4) The Base Consisting of Neither Perception nor Non-Perception]{(4) The Base Consisting of Neither Perception nor Non-Perception}

                \vismParagraph{X.40}{40}{}
                When, however, he wants to develop the base consisting of neither perception nor non-perception, he must first achieve mastery in the five ways in the attainment of the base consisting of nothingness. Then he should see the danger in the base consisting of nothingness and the advantage in what is superior to it in this way: “This attainment has the base consisting of boundless consciousness as its near enemy, and it is not as peaceful as the base consisting of neither perception nor non-perception,” or in this way: “Perception is a boil, perception is a dart … this is peaceful, this is sublime, that is to say, neither perception nor non-perception” (\textbf{\cite{M}II 231}). So having ended his attachment to the base consisting of nothingness, he should give attention to the base consisting of neither perception non non-perception as peaceful. He should advert again and again to that attainment of the base consisting of nothingness that has occurred making non-existence its object, adverting to it as “peaceful, peaceful,” and he should give his attention to it, review it and strike at it with thought and applied thought.

                \vismParagraph{X.41}{41}{}
                As he directs his mind again and again on to that sign in this way, the hindrances are suppressed, mindfulness is established, and his mind becomes concentrated in access. He cultivates that sign again and again, develops and repeatedly practices it. As he does so, consciousness belonging to the base consisting of neither perception nor non-perception arises in absorption making its object the four [mental] aggregates that constitute the attainment of the base consisting of nothingness, just as the [consciousness belonging to the] base consisting of nothingness did the disappearance of the [previous] consciousness. And here too the method of explaining the absorption should be understood in the way already described.
                \subsection[\vismAlignedParas{§42–55}Text and commentary]{Text and commentary}

                    \vismParagraph{X.42}{42}{}
                    And at this point it is said: “By completely surmounting the base consisting of nothingness he enters upon and dwells in the base consisting of neither perception nor non-perception” (\textbf{\cite{Vibh}245}).

                    \vismParagraph{X.43}{43}{}
                    \marginnote{\textcolor{teal}{\footnotesize\{389|331\}}}{}Herein, \emph{completely} is already explained. \emph{By … surmounting the base consisting of nothingness}: here too the jhāna is called the “base consisting of nothingness” in the way already stated, and its object is so called too. For the object too is “nothingness” (\emph{ākiñcaññaṃ}) in the way already stated, and then because it is the object of the third immaterial jhāna, it is its “base” in the sense of habitat, as the “deities’ base” is for deities, thus it is the “base consisting of nothingness.” Likewise: it is “nothingness,” and then, because it is the cause of the jhāna’s being of that species, it is its “base” in the sense of locality of the species, as Kambojā is the “base” of horses, thus it is the “base consisting of nothingness” in this way also. \textcolor{brown}{\textit{[336]}} So it should be understood that the words, “By … surmounting the base consisting of nothingness” include both [the jhāna and its object] together, since the base consisting of neither perception nor non-perception is to be entered upon and dwelt in precisely by surmounting, by causing the non-occurrence of, by not giving attention to, both the jhāna and its object.

                    \vismParagraph{X.44}{44}{}
                    \emph{Base consisting of neither perception nor non-perception}: then there is he who so practices that there is in him the perception on account of the presence of which this [attainment] is called the “the base consisting of neither perception nor non-perception,” and in the Vibhaṅga, in order to point out that [person], firstly one specified as “neither percipient nor non-percipient,” it is said, “gives attention to that same base consisting of nothingness as peaceful, he develops the attainment with residual formations, hence ‘neither percipient nor non-percipient’ is said” (\textbf{\cite{Vibh}263}).

                    \vismParagraph{X.45}{45}{}
                    Herein, \emph{he gives attention … as peaceful}, means that he gives attention to it as “peaceful” because of the peacefulness of the object thus: “How peaceful this attainment is; for it can make even non-existence its object and still subsist!”

                    If he brings it to mind as “peaceful” then how does there come to be surmounting? Because there is no actual desire to attain it. For although he gives attention to it as “peaceful,” yet there is no concern in him or reaction or attention such as “I shall advert to this” or “I shall attain this” or “I shall resolve upon [the duration of] this.” Why not? Because the base consisting of neither perception nor non-perception is more peaceful and better than the base consisting of nothingness.

                    \vismParagraph{X.46}{46}{}
                    Suppose a king is proceeding along a city street with the great pomp of royalty,\footnote{\vismAssertFootnoteCounter{10}\vismHypertarget{X.n10}{}\emph{Mahacca} (see \textbf{\cite{D}I 49} and D-a I 148); the form is not given in PED; probably a form of \emph{mahatiya}.} splendidly mounted on the back of an elephant, and he sees craftsmen wearing one cloth tightly as a loin-cloth and another tied round their heads, working at the various crafts such as ivory carving, etc., their limbs covered with ivory dust, etc.; now while he is pleased with their skill, thinking, “How skilled these craft-masters are, and what crafts they practice!” he does not, however, think, “Oh that I might abandon royalty and become a craftsman like that!” Why not? Because of the great benefits in the majesty of kings; he leaves the craftsmen behind and proceeds on his way. So too, though this [meditator] gives \marginnote{\textcolor{teal}{\footnotesize\{390|332\}}}{}attention to that attainment as “peaceful,” yet there is no concern in him or reaction or attention such as “I shall advert to this attainment” or “I shall attain this” or “I shall resolve upon [the duration of] it” or “I shall emerge from it” or “I shall review it.”

                    \vismParagraph{X.47}{47}{}
                    As he gives attention to it as “peaceful” in the way already described, \textcolor{brown}{\textit{[337]}} he reaches the ultra-subtle absorbed perception in virtue of which he is called “neither percipient nor non-percipient,” and it is said of him that “He develops the attainment with residual formations.”

                    \emph{The attainment with residual formations} is the fourth immaterial attainment whose formations have reached a state of extreme subtlety.

                    \vismParagraph{X.48}{48}{}
                    Now, in order to show the meaning of the kind of perception that has been reached, on account of which [this jhāna] is called the “base consisting of neither perception nor non-perception,” it is said: “‘Base consisting of neither perception nor non-perception’: states of consciousness or its concomitants in one who has attained the base consisting of neither perception nor non-perception or in one who has been reborn there or in one who is abiding in bliss there in this present life” (\textbf{\cite{Vibh}263}). Of these, what is intended here is the states of consciousness and its concomitants in one who has attained.

                    \vismParagraph{X.49}{49}{}
                    The word meaning here is this: that jhāna with its associated states neither has perception nor has no perception because of the absence of gross perception and the presence of subtle perception, thus it is “neither perception nor non-perception” (\emph{n’ eva-saññā-nāsaññaṃ}). It is “neither perception nor non-perception” and it is a “base” (\emph{āyatana}) because it is included in the mind-base (\emph{manāyatana}) and the mental-object base (\emph{dhammāyatana}), thus it is the “base consisting of neither perception nor non-perception” (\emph{nevasaññānāsaññāyatana}).

                    \vismParagraph{X.50}{50}{}
                    Or alternatively: the perception here is neither perception, since it is incapable of performing the decisive function of perception, nor yet non-perception, since it is present in a subtle state as a residual formation, thus it is “neither perception nor non-perception.” It is “neither perception nor non-perception” and it is a “base” in the sense of a foundation for the other states, thus it is the “base consisting of neither perception nor non-perception.”

                    And here it is not only perception that is like this, but feeling as well is neither-feeling-nor-non-feeling, consciousness is neither-consciousness-nor-non-consciousness, and contact is neither-contact-nor-non-contact, and the same description applies to the rest of the associated states; but it should be understood that this presentation is given in terms of perception.

                    \vismParagraph{X.51}{51}{}
                    And the meaning should be illustrated by the similes beginning with the smearing of oil on the bowl. A novice smeared a bowl with oil, it seems, and laid it aside. When it was time to drink gruel, an elder told him to bring the bowl. He said, “Venerable sir, there is oil in the bowl.” But then when he was told, “Bring the oil, novice, I shall fill the oil tube,” he replied, “There is no oil, venerable sir.” Herein, just as “There is oil” is in the sense of incompatibility with the gruel because it has been poured into [the bowl] and just as “There is no oil” is in the sense of filling the oil tube, etc., so too this perception is “neither perception” since \marginnote{\textcolor{teal}{\footnotesize\{391|333\}}}{}it is incapable of performing the decisive function of perception and it is “nor non-perception” because it is present in a subtle form as a residual formation. \textcolor{brown}{\textit{[338]}}

                    \vismParagraph{X.52}{52}{}
                    But in this context what is perception’s function? It is the perceiving of the object, and it is the production of dispassion if [that attainment and its object are] made the objective field of insight. But it is not able to make the function of perceiving decisive, as the heat element in tepid\footnote{\vismAssertFootnoteCounter{11}\vismHypertarget{X.n11}{}\emph{Sukhodaka—}“tepid water”: see Monier Williams’ \emph{Sanskrit Dictionary}; this meaning of \emph{sukha} not given in PED.} water is not able to make the function of burning decisive; and it is not able to produce dispassion by treatment of its objective field with insight in the way that perception is in the case of the other attainments.

                    \vismParagraph{X.53}{53}{}
                    There is in fact no bhikkhu capable of reaching dispassion by comprehension of aggregates connected with the base consisting of neither perception nor non-perception unless he has already done his interpreting with other aggregates (see \hyperlink{XX.2}{XX.2f.}{} and \hyperlink{XXI.23}{XXI.23}{}). And furthermore, when the venerable Sāriputta, or someone very wise and naturally gifted with insight as he was, is able to do so, even he has to do it by means of comprehension of groups (\hyperlink{XX.2}{XX.2}{}) in this way, “So it seems, these states, not having been, come to be; having come to be, they vanish” (\textbf{\cite{M}III 28}), and not by means of [actual direct] insight into states one by one as they arise. Such is the subtlety that this attainment reaches.

                    \vismParagraph{X.54}{54}{}
                    And this meaning should be illustrated by the simile of the water on the road, as it was by the simile of the oil-smearing on the bowl. A novice was walking in front of an elder, it seems, who had set out on a journey. He saw a little water and said, “There is water, venerable sir, remove your sandals.” Then the elder said, “If there is water, bring me the bathing cloth and let us bathe,” but the novice said, “There is none, venerable sir.” Herein, just as “There is water” is in the sense of mere wetting of the sandals, and “There is none” is in the sense of bathing, so too, this perception is “neither perception” since it is incapable of performing the decisive function of perception, and it is “nor non-perception” because it is present in a subtle form as a residual formation.

                    \vismParagraph{X.55}{55}{}
                    And this meaning should be illustrated not only by these similes but by other appropriate ones as well.

                    \emph{Enters upon and dwells in} is already explained.

                    This is the detailed explanation of the base consisting of neither perception nor non-perception as a meditation subject.
            \section[\vismAlignedParas{§56–66}General]{General}

                \vismParagraph{X.56}{56}{}
                
                \begin{verse}
                    Thus has the Peerless Helper told\\{}
                    The fourfold immaterial state;\\{}
                    To know these general matters too\\{}
                    Will not be inappropriate.
                \end{verse}


                \vismParagraph{X.57}{57}{}
                
                \begin{verse}
                    For these immaterial states:\\{}
                    While reckoned by the surmounting of\\{}
                    \marginnote{\textcolor{teal}{\footnotesize\{392|334\}}}{}The object they are four, the wise\\{}
                    Do not admit surmounting of\\{}
                    Factors that one can recognize.
                \end{verse}


                \vismParagraph{X.58}{58}{}
                Of these [four], the first is due to surmounting signs of materiality, the second is due to surmounting space, the third is due to surmounting the consciousness that occurred with that space as its object, and the fourth is due to surmounting the disappearance of the consciousness that occurred with that space as its object. So they should be understood as four in number with the surmounting of the object in each case. \textcolor{brown}{\textit{[339]}} But the wise do not admit any surmounting of [jhāna] factors; for there is no surmounting of factors in them as there is in the case of the fine-material-sphere attainments. Each one has just the two factors, namely equanimity and unification of mind.

                \vismParagraph{X.59}{59}{}
                That being so:
                \begin{verse}
                    They progress in refinement; each\\{}
                    Is finer than the one before.\\{}
                    Two figures help to make them known;\\{}
                    The cloth lengths, and each palace floor.
                \end{verse}


                \vismParagraph{X.60}{60}{}
                Suppose there were a four-storied palace: on its first floor the five objects of sense pleasure were provided in a very fine form as divine dancing, singing and music, and perfumes, scents, garlands, food, couches, clothing, etc., and on the second they were finer than that, and on the third finer still, and on the fourth they were finest of all; yet they are still only palace floors, and there is no difference between them in the matter of their state (essence) as palace floors; it is with the progressive refinement of the five objects of sense pleasure that each one is finer than the one below;—again suppose there were lengths of cloth of quadruple, triple, double and single thickness, and [made] of thick, thin, thinner, and very thin thread spun by one woman, all the same measure in width and breadth; now although these lengths of cloth are four in number, yet they measure the same in width and breadth, there is no difference in their measurement; but in softness to the touch, fineness, and costliness each is finer than the one before;—so too, although there are only the two factors in all four [immaterial states], that is to say, equanimity and unification of mind, still each one should be understood as finer than the one before with the progressive refinement of factors due to successful development.

                \vismParagraph{X.61}{61}{}
                And for the fact that each one of them is finer than the last [there is this figure:]
                \begin{verse}
                    One hangs upon a tent that stands\\{}
                    On filth; on him another leans.\\{}
                    Outside a third not leaning stands,\\{}
                    Against the last another leans.\\{}
                    Between the four men and these states\\{}
                    The correspondence then is shown,\\{}
                    And so how each to each relates\\{}
                    Can by a man of wit be known.
                \end{verse}


                \vismParagraph{X.62}{62}{}
                \marginnote{\textcolor{teal}{\footnotesize\{393|335\}}}{}This is how the meaning should be construed. There was a tent in a dirty place, it seems. Then a man arrived, and being disgusted with the dirt, he rested himself on the tent with his hands and remained as if hung or hanging on to it. Then another man came and leant upon the man hanging on to the tent. Then another man came and thought, “The one who is hanging on to the tent and the one who is leaning upon him are both badly off, and if the tent falls they will certainly fall. I think I shall stand outside.” \textcolor{brown}{\textit{[340]}} So instead of leaning upon the one leaning upon the first, he remained outside. Then another arrived, and taking account of the insecurity of the one hanging on to the tent and the one leaning upon him, and fancying that the one standing outside was well placed, he stood leaning upon him.

                \vismParagraph{X.63}{63}{}
                Herein, this is how it should be regarded. The space from which the kasiṇa has been removed is like the tent in the dirty place. The [consciousness of the] base consisting of boundless space, which makes space its object owing to disgust with the sign of the fine-material, is like the man who hangs on to the tent owing to disgust with the dirt. The [consciousness of the] base consisting of boundless consciousness, the occurrence of which is contingent upon [the consciousness of] the base consisting of boundless space whose object is space, is like the man who leans upon the man who hangs on to the tent. The [consciousness of the] base consisting of nothingness, which instead of making [the consciousness of the] base consisting of boundless space its object has the non-existence of that as its object, is like the man who, after considering the insecurity of those two, does not lean upon the one hanging on to the tent, but stands outside. The [consciousness of the] base consisting of neither perception nor non-perception, the occurrence of which is contingent upon [the consciousness of] the base consisting of nothingness, which stands in a place outside, in other words, in the non-existence of [the past] consciousness, is like the man who stands leaning upon the last-named, having considered the insecurity of the one hanging on to the tent and the one leaning upon him, and fancying that the one standing outside is well placed.

                \vismParagraph{X.64}{64}{}
                And while occurring in this way:
                \begin{verse}
                    It takes this for its object since\\{}
                    There is no other one as good,\\{}
                    As men depend upon a king,\\{}
                    Whose fault they see, for livelihood.
                \end{verse}


                \vismParagraph{X.65}{65}{}
                For although this [consciousness of the] base consisting of neither perception nor non-perception has seen the flaw in the base consisting of nothingness in this way, “This attainment has the base consisting of boundless consciousness as its near enemy,” notwithstanding that fact it takes it as its object in the absence of any other. Like what? As men for the sake of livelihood depend on kings whose faults they see. For just as, for the sake of livelihood and because they cannot get a livelihood elsewhere, people put up with some king, ruler of all quarters, who is unrestrained, and harsh in bodily, verbal, and mental behaviour, though they see his faults thus, “He is harshly behaved,” so too the [consciousness of the] base consisting of neither perception nor non-perception \marginnote{\textcolor{teal}{\footnotesize\{394|336\}}}{}takes that base consisting of nothingness as its object in spite of seeing its faults in this way, and it does so since it cannot find another [better] object.

                \vismParagraph{X.66}{66}{}
                As one who mounts a lofty stair Leans on its railings for a prop, As one who climbs an airy peak Leans on the mountain’s very top, As one who stands on a crag’s edge Leans for support on his own knees—Each jhāna rests on that below; For so it is with each of these.

                The tenth chapter called “The Description of the Immaterial States” in the treatise on the Development of Concentration in the \emph{Path of Purification} composed for the purpose of gladdening good people.
        \chapter[Concentration—Conclusion: Nutriment and the Elements]{Concentration—Conclusion: Nutriment and the Elements\vismHypertarget{XI}\newline{\textnormal{\emph{Samādhi-niddesa}}}}
            \label{XI}

            \section[\vismAlignedParas{§1–26}Perception of Repulsiveness in Nutriment]{Perception of Repulsiveness in Nutriment}

                \vismParagraph{XI.1}{1}{}
                \marginnote{\textcolor{teal}{\footnotesize\{395|337\}}}{}\textcolor{brown}{\textit{[341]}} Now comes the description of the development of the perception of repulsiveness in nutriment, which was listed as the “one perception”\footnote{\vismAssertFootnoteCounter{1}\vismHypertarget{XI.n1}{}“The word ‘perception’ (\emph{saññā}) is used for the \emph{dhamma} with the characteristic of perceiving (\emph{sañjānana}), as in the case of ‘perception of visible objects,’ ‘perception of sound,’ etc.; and it is used for insight, as in the case of ‘perception of impermanence,’ ‘perception of suffering,’ etc.; and it is used for serenity, as in the passage, ‘Perception of the bloated and perception of visible objects, have these one meaning or different meanings, Sopāka?’ (\emph{Source untraced. }Cf. \hyperlink{III.111}{III.111}{}), and so on. Here, however, it should be understood as the preliminary work for serenity; for it is the apprehending of the repulsive aspect in nutriment, or the access jhāna produced by means of that, that is intended here by, ‘perception of repulsiveness in nutriment’”(\textbf{\cite{Vism-mhṭ}334–335}).} next to the immaterial states (\hyperlink{III.105}{III.105}{}).

                Herein, it nourishes (\emph{āharati}, lit. “brings on”), thus it is nutriment (\emph{āhāra}, lit. “bringing on”). That is of four kinds as: physical nutriment, nutriment consisting of contact, nutriment consisting of mental volition, and nutriment consisting of consciousness.\footnote{\vismAssertFootnoteCounter{2}\vismHypertarget{XI.n2}{}A more detailed exposition of nutriment is given at \textbf{\cite{M-a}I 107ff.} “‘It nourishes’ (\emph{āharati})”: the meaning is that it leads up, fetches, produces, its own fruit through its state as a condition for the fruit’s arising or presence, which state is called “nutriment condition.” It is made into a mouthful (\emph{kabalaṃ karīyati}), thus it is physical (\emph{kabaliṅkāra}). In this way it gets its designation from the concrete object; but as to characteristic, it should be understood to have the characteristic of nutritive essence (\emph{ojā}). It is physical and it is nutriment in the sense stated, thus it is physical nutriment; so with the rest. It touches (\emph{phusati}), thus it is contact (\emph{phassa}); for although this is an immaterial state, it occurs also as the aspect of touching on an object (\emph{ārammaṇa—}lit. “what is to be leaned on”), which is why it is said to have the characteristic of touching. It wills (\emph{cetayati}), thus it is volition (\emph{cetanā}); the meaning is that it arranges (collects) itself together with associated states upon the object. Mental volition is volition occupied with the mind. It cognizes (\emph{vijānāti}) by conjecturing about rebirth (see \hyperlink{XVII.303}{XVII.303}{}), thus it is consciousness (\emph{viññāṇa} = cognition) (\textbf{\cite{Vism-mhṭ}335}).}

                \vismParagraph{XI.2}{2}{}
                \marginnote{\textcolor{teal}{\footnotesize\{396|338\}}}{}But what is it here that nourishes (brings on) what? Physical nutriment (\emph{kabaliṅkārāhāra}) nourishes (brings on) the materiality of the octad that has nutritive essence as eighth:\footnote{\vismAssertFootnoteCounter{3}\vismHypertarget{XI.n3}{}For the “octad with nutritive essence as eighth” (\emph{ojaṭṭhamaka), }see \hyperlink{XVIII.5}{XVIII.5ff.}{} and \hyperlink{XX.27}{XX.27ff.}{}} contact as nutriment nourishes (brings on) the three kinds of feeling; mental volition as nutriment nourishes (brings on) rebirth-linking in the three kinds of becoming; consciousness as nutriment nourishes (brings on) mentality-materiality at the moment of rebirth-linking.

                \vismParagraph{XI.3}{3}{}
                Now, when there is physical nutriment there is attachment, which brings peril; when there is nutriment as contact there is approaching, which brings peril; when there is nutriment as mental volition there is rebirth-linking, which brings peril.\footnote{\vismAssertFootnoteCounter{4}\vismHypertarget{XI.n4}{}\textbf{\cite{Vism-mhṭ}(p. 355)} explains \emph{attachment }here as craving which is “perilous because it brings harm” (see e.g. \textbf{\cite{D}II 58–59}), or in other words, “greed for the five aggregates (lust after five-aggregate experience).” It cites the following: “Bhikkhus, when there is physical nutriment, there is greed (lust), there is delighting, there is craving; consciousness being planted therein grows. Wherever consciousness being planted grows, there is the combination of mind-and-matter. Wherever there is the combination of mind-and-matter, there is ramification of formations. Wherever there is ramification of formations, there is production of further becoming in the future. Wherever there is production of further becoming in the future, there is future birth, aging and death. Wherever there is future birth, aging and death, bhikkhus, the end is sorrow, I say, with woe and despair” (\textbf{\cite{S}II 101}; cf. \textbf{\cite{S}II 66}). \emph{Approaching }is explained as “meeting, coinciding, with unabandoned perversions [of perception] due to an object [being perceived as permanent, etc., when it is not].” That is, “perilous since it is not free from the three kinds of suffering.” The quotation given is: “Bhikkhus, due to contact of the kind to be felt as pleasant, pleasant feeling arises. With that feeling as condition there is craving, … thus there is the arising of this whole mass of suffering” (cf. \textbf{\cite{S}IV 215}). \emph{Reappearance }is “rebirth in some kind of becoming or other. Being flung into a new becoming is perilous because there is no immunity from the risks rooted in reappearance.” The following is quoted: “Not knowing, bhikkhus, a man forms the formation of merit, and his [rebirth] consciousness accords with the merit [tie performed]; he forms the formation of demerit; … he forms the formation of the imperturbable …” (\textbf{\cite{S}II 82}). \emph{Rebirth-linking }is the actual linking with the next becoming, which “is perilous since it is not immune from the suffering due to the signs of [the impending] rebirth-linking.” The quotation given is: “Bhikkhus, when there is consciousness as nutriment there is greed (lust), there is delighting …” (\textbf{\cite{S}II 102}—complete as above).} And to show how they bring fear thus, physical nutriment should be illustrated by the simile of the child’s flesh (\textbf{\cite{S}II 98}), contact as nutriment by the simile of the hideless cow (\textbf{\cite{S}II 99}), mental volition as nutriment by the simile of the pit of live coals (\textbf{\cite{S}II 99}), and consciousness as nutriment by the simile of the hundred spears (\textbf{\cite{S}II 100}).

                \vismParagraph{XI.4}{4}{}
                But of these four kinds of nutriment it is only physical nutriment, classed as what is eaten, drunk, chewed, and tasted, that is intended here as “nutriment” in this sense. The perception arisen as the apprehension of the repulsive aspect in that nutriment is, “perception of repulsiveness in nutriment.”

                \vismParagraph{XI.5}{5}{}
                \marginnote{\textcolor{teal}{\footnotesize\{397|339\}}}{}One who wants to develop that perception of repulsiveness in nutriment should learn the meditation subject and see that he has no uncertainty about even a single word of what he has learnt. Then he should go into solitary retreat and \textcolor{brown}{\textit{[342]}} review repulsiveness in ten aspects in the physical nutriment classified as what is eaten, drunk, chewed, and tasted, that is to say, as to going, seeking, using, secretion, receptacle, what is uncooked (undigested), what is cooked (digested), fruit, outflow, and smearing.

                \vismParagraph{XI.6}{6}{}
                \emph{1.} Herein, \emph{as to going}: even when a man has gone forth in so mighty a dispensation, still after he has perhaps spent all night reciting the Enlightened One’s word or doing the ascetic’ s work, after he has risen early to do the duties connected with the shrine terrace and the Enlightenment-tree terrace, to set out the water for drinking and washing, to sweep the grounds and to see to the needs of the body, after he has sat down on his seat and given attention to his meditation subject twenty or thirty times\footnote{\vismAssertFootnoteCounter{5}\vismHypertarget{XI.n5}{}“‘\emph{Twenty or thirty times’: }here some say that the definition of the number of times is according to what is present-by-continuity (see \hyperlink{XIV.188}{XIV.188}{}). But others say that it is by way of “warming up the seat” (see \textbf{\cite{M-a}I 255}); for development that has not reached suppression of hindrances does not remove the bodily discomfort in the act of sitting, because of the lack of pervading happiness. So there is inconstancy of posture too. Then ‘twenty or thirty’ is taken as the number already observed by the time of setting out on the alms round. Or alternatively, from ‘going’ up to ‘smearing’ is one turn; then it is after giving attention to the meditation subject by twenty or thirty turns in this way” (\textbf{\cite{Vism-mhṭ}339}).} and got up again, then he must take his bowl and [outer] robe, he must leave behind the ascetics’ woods that are not crowded with people, offer the bliss of seclusion, possess shade and water, and are clean, cool, delightful places, he must disregard the Noble Ones’ delight in seclusion, and he must set out for the village in order to get nutriment, as a jackal for the charnel ground.

                \vismParagraph{XI.7}{7}{}
                And as he goes thus, from the time when he steps down from his bed or chair he has to tread on a carpet\footnote{\vismAssertFootnoteCounter{6}\vismHypertarget{XI.n6}{}\emph{Paccattharaṇa—“}carpet”: the word normally means a coverlet, but here, according to \textbf{\cite{Vism-mhṭ}}, (p. 339) it is, “a spread \emph{(attharaṇa) }consisting of a rug \emph{(cilimika) }to be spread on the ground for protecting the skin.”} covered with the dust of his feet, geckos’ droppings, and so on. Next he has to see the doorstep,\footnote{\vismAssertFootnoteCounter{7}\vismHypertarget{XI.n7}{}For \emph{pamukha—“}doorstep,” perhaps an open upper floor gallery here, see \hyperlink{XIII.6}{XIII.6}{}.} which is more repulsive than the inside of the room since it is often fouled with the droppings of rats, bats,\footnote{\vismAssertFootnoteCounter{8}\vismHypertarget{XI.n8}{}\emph{Jatukā—“}bat” = \emph{khuddaka-vaggulī }(\textbf{\cite{Vism-mhṭ}339}): not in PED; see \hyperlink{XIII.97}{XIII.97}{}.} and so on. Next the lower terrace, which is more repulsive than the terrace above since it is all smeared with the droppings of owls, pigeons,\footnote{\vismAssertFootnoteCounter{9}\vismHypertarget{XI.n9}{}\emph{Pārāvata—“}pigeon”: only spelling \emph{pārāpata }given in PED.} and so on. Next the grounds,\footnote{\vismAssertFootnoteCounter{10}\vismHypertarget{XI.n10}{}For this meaning of \emph{pariveṇa }see \hyperlink{IV.n37}{Ch. IV, note 37}{}.} which are more repulsive than the lower floor since they are defiled by old grass and leaves blown about by the wind, by sick novices’ urine, excrement, spittle and snot, and in the rainy season by water, mud, and so on. And he has to see the road to the monastery, which is more repulsive than the grounds.

                \vismParagraph{XI.8}{8}{}
                \marginnote{\textcolor{teal}{\footnotesize\{398|340\}}}{}In due course, after standing in the debating lodge\footnote{\vismAssertFootnoteCounter{11}\vismHypertarget{XI.n11}{}\emph{Vitakka-māḷaka—“}debating lodge”: \textbf{\cite{Vism-mhṭ}(p. 339)} says: “‘\emph{Kattha nu kho ajja bhikkhāya caritabban’ ti ādinā vitakkamāḷake” }(“in a lodge for thinking in the way beginning ‘Where must I go for alms today?’”).} when he has finished paying homage at the Enlightenment Tree and the shrine, he sets out thinking, “Instead of looking at the shrine that is like a cluster of pearls, and the Enlightenment Tree that is as lovely as a bouquet of peacock’s tail feathers, and the abode that is as fair as a god’s palace, I must now turn my back on such a charming place and go abroad for the sake of food;” and on the way to the village, the view of a road of stumps and thorns and an uneven road broken up by the force of water awaits him.

                \vismParagraph{XI.9}{9}{}
                Next, after he has put on his waist cloth as one who hides an abscess, and tied his waist band as one who ties a bandage on a wound, and robed himself in his upper robes as one who hides a skeleton, and taken out his bowl as one who takes out a pan for medicine, \textcolor{brown}{\textit{[343]}} when he reaches the vicinity of the village gate, perhaps the sight of an elephant’s carcass, a horse’s carcass, a buffalo’s carcass, a human carcass, a snake’s carcass, or a dog’s carcass awaits him, and not only that, but he has to suffer his nose to be assailed by the smell of them.

                Next, as he stands in the village gateway, he must scan the village streets in order to avoid danger from savage elephants, horses, and so on.

                \vismParagraph{XI.10}{10}{}
                So this repulsive [experience] beginning with the carpet that has to be trodden on and ending with the various kinds of carcasses that have to be seen and smelled, [has to be undergone] for the sake of nutriment: “Oh, nutriment is indeed a repulsive thing!”

                This is how repulsiveness should be reviewed as to going.

                \vismParagraph{XI.11}{11}{}
                \emph{2. }How \emph{as to seeking}? When he has endured the repulsiveness of going in this way, and has gone into the village, and is clothed in his cloak of patches, he has to wander in the village streets from house to house like a beggar with a dish in his hand. And in the rainy season wherever he treads his feet sink into water and mire up to the flesh of the calves.\footnote{\vismAssertFootnoteCounter{12}\vismHypertarget{XI.n12}{}\emph{Piṇḍika-maṃsa—“}flesh of the calves” \emph{= jaṅghapiṇḍikaṃamsapadesa. }(\textbf{\cite{Vism-mhṭ}340}) Cf. \hyperlink{VIII.97}{VIII.97}{}; also \textbf{\cite{A-a}} 417. Not in this sense in PED.} He has to hold the bowl in one hand and his robe up with the other. In the hot season he has to go about with his body covered with the dirt, grass, and dust blown about by the wind. On reaching such and such a house door he has to see and even to tread in gutters and cesspools covered with blue-bottles and seething with all the species of worms, all mixed up with fish washings, meat washings, rice washings, spittle, snot, dogs’ and pigs’ excrement, and what not, from which flies come up and settle on his outer cloak of patches and on his bowl and on his head.

                \vismParagraph{XI.12}{12}{}
                And when he enters a house, some give and some do not. And when they give, some give yesterday’s cooked rice and stale cakes and rancid jelly, sauce and so on.\footnote{\vismAssertFootnoteCounter{13}\vismHypertarget{XI.n13}{}\emph{Kummāsa—“}jelly”: usually rendered “junket,” but the Vinaya commentaries give it as made of corn \emph{(yava).}} Some, not giving, say, “Please pass on, venerable sir,” others keep \marginnote{\textcolor{teal}{\footnotesize\{399|341\}}}{}silent as if they did not see him. Some avert their faces. Others treat him with harsh words such as: “Go away, you bald-head.” When he has wandered for alms in the village in this way like a beggar, he has to depart from it.

                \vismParagraph{XI.13}{13}{}
                So this [experience] beginning with the entry into the village and ending with the departure from it, which is repulsive owing to the water, mud, etc., that has to be trodden in and seen and endured, [has to be undergone] for the sake of nutriment: “Oh, nutriment is indeed a repulsive thing!”

                This is how repulsiveness should be reviewed as to seeking. \textcolor{brown}{\textit{[344]}}

                \vismParagraph{XI.14}{14}{}
                \emph{3.} How \emph{as to using}? After he has sought the nutriment in this way and is sitting at ease in a comfortable place outside the village, then so long as he has not dipped his hand into it he would be able to invite a respected bhikkhu or a decent person, if he saw one, [to share it]; but as soon as he has dipped his hand into it out of desire to eat he would be ashamed to say, “Take some.” And when he has dipped his hand in and is squeezing it up, the sweat trickling down his five fingers wets any dry crisp food there may be and makes it sodden.

                \vismParagraph{XI.15}{15}{}
                And when its good appearance has been spoilt by his squeezing it up, and it has been made into a ball and put into his mouth, then the lower teeth function as a mortar, the upper teeth as a pestle, and the tongue as a hand. It gets pounded there with the pestle of the teeth like a dog’s dinner in a dog’s trough, while he turns it over and over with his tongue; then the thin spittle at the tip of the tongue smears it, and the thick spittle behind the middle of the tongue smears it, and the filth from the teeth in the parts where a tooth-stick cannot reach smears it.

                \vismParagraph{XI.16}{16}{}
                When thus mashed up and besmeared, this peculiar compound now destitute of the [original] colour and smell is reduced to a condition as utterly nauseating as a dog’s vomit in a dog’s trough. Yet, notwithstanding that it is like this, it can still be swallowed because it is no longer in range of the eye’s focus.

                This is how repulsiveness should be reviewed as to using.

                \vismParagraph{XI.17}{17}{}
                \emph{4.} How \emph{as to secretion}? Buddhas and Paccekabuddhas and Wheel-turning Monarchs have only one of the four secretions consisting of bile, phlegm, pus and blood, but those with weak merit have all four. So when [the food] has arrived at the stage of being eaten and it enters inside, then in one whose secretion of bile is in excess it becomes as utterly nauseating as if smeared with thick \emph{madhuka }oil; in one whose secretion of phlegm in excess it is as if smeared with the juice of \emph{nāgabalā} leaves;\footnote{\vismAssertFootnoteCounter{14}\vismHypertarget{XI.n14}{}\emph{Nāgabalā—}a kind of plant; not in PED.} in one whose secretion of pus is in excess it is as if smeared with rancid buttermilk; and in one whose secretion of blood is in excess it is as utterly nauseating as if smeared with dye. This is how repulsiveness should be reviewed as to secretion.

                \vismParagraph{XI.18}{18}{}
                \emph{5.} How \emph{as to receptacle}? When it has gone inside the belly and is smeared with one of these secretions, then the receptacle it goes into is no gold dish or crystal or silver dish and so on. On the contrary, if it is swallowed by one ten years old, it finds itself in a place like a cesspit unwashed for ten years. \textcolor{brown}{\textit{[345]}} If it is swallowed by one twenty years old, thirty, forty, fifty, sixty, seventy, eighty, ninety \marginnote{\textcolor{teal}{\footnotesize\{400|342\}}}{}years old, if it is swallowed by one a hundred years old, it finds itself in a place like a cesspit unwashed for a hundred years. This is how repulsiveness should be reviewed as to receptacle.

                \vismParagraph{XI.19}{19}{}
                \emph{6.} How \emph{as to what is uncooked }(\emph{undigested})? After this nutriment has arrived at such a place for its receptacle, then for as long as it remains uncooked it stays in that same place just described, which is shrouded in absolute darkness, pervaded by draughts,\footnote{\vismAssertFootnoteCounter{15}\vismHypertarget{XI.n15}{}\emph{Pavana—}“draught”: not in this sense in PED; see \hyperlink{XVI.37}{XVI.37}{}.} tainted by various smells of ordure and utterly fetid and loathsome. And just as when a cloud out of season has rained during a drought and bits of grass and leaves and rushes and the carcasses of snakes, dogs and human beings that have collected in a pit at the gate of an outcaste village remain there warmed by the sun’s heat until the pit becomes covered with froth and bubbles, so too, what has been swallowed that day and yesterday and the day before remains there together, and being smothered by the layer of phlegm and covered with froth and bubbles produced by digestion through being fermented by the heat of the bodily fires, it becomes quite loathsome. This is how repulsiveness should be reviewed as to what is uncooked.

                \vismParagraph{XI.20}{20}{}
                \emph{7.} How \emph{as to what is cooked}? When it has been completely cooked there by the bodily fires, it does not turn into gold, silver, etc., as the ores\footnote{\vismAssertFootnoteCounter{16}\vismHypertarget{XI.n16}{}\emph{Dhātu—“}ore”: not in this sense in PED. See also \hyperlink{XV.20}{XV.20}{}.} of gold, silver, etc., do [through smelting]. Instead, giving off froth and bubbles, it turns into excrement and fills the receptacle for digested food, like brown clay squeezed with a smoothing trowel and packed into a tube, and it turns into urine and fills the bladder. This is how repulsiveness should be reviewed as to what is cooked.

                \vismParagraph{XI.21}{21}{}
                \emph{8.} How \emph{as to fruit}? When it has been rightly cooked, it produces the various kinds of ordure consisting of head hairs, body hairs, nails, teeth, and the rest. When wrongly cooked it produces the hundred diseases beginning with itch, ring-worm, smallpox, leprosy, plague, consumption, coughs, flux, and so on. Such is its fruit. This is how repulsiveness should be reviewed as to fruit.

                \vismParagraph{XI.22}{22}{}
                9. How \emph{as to outflow}? On being swallowed, it enters by one door, after which it flows out by several doors in the way beginning, “Eye-dirt from the eye, ear-dirt from the ear” (\textbf{\cite{Sn}197}). And on being swallowed it is swallowed even in the company of large gatherings. But on flowing out, now converted into excrement, urine, etc., it is excreted only in solitude. \textcolor{brown}{\textit{[346]}} On the first day one is delighted to eat it, elated and full of happiness and joy. On the second day one stops one’s nose to void it, with a wry face, disgusted and dismayed. And on the first day one swallows it lustfully, greedily, gluttonously, infatuatedly. But on the second day, after a single night has passed, one excretes it with distaste, ashamed, humiliated and disgusted. Hence the Ancients said:

                \vismParagraph{XI.23}{23}{}
                
                \begin{verse}
                    The food and drink so greatly prized—\\{}
                    The crisp to chew, the soft to suck—\\{}
                    Go in all by a single door,\\{}
                    But by nine doors come oozing out.
                \end{verse}

                \begin{verse}
                    \marginnote{\textcolor{teal}{\footnotesize\{401|343\}}}{}The food and drink so greatly prized—\\{}
                    The crisp to chew, the soft to suck—\\{}
                    Men like to eat in company,\\{}
                    But to excrete in secrecy.
                \end{verse}

                \begin{verse}
                    The food and drink so greatly prized—\\{}
                    The crisp to chew, the soft to suck—\\{}
                    These a man eats with high delight,\\{}
                    And then excretes with dumb disgust.
                \end{verse}

                \begin{verse}
                    The food and drink so greatly prized—\\{}
                    The crisp to chew, the soft to suck—\\{}
                    A single night will be enough\\{}
                    To bring them to putridity.
                \end{verse}


                This is how repulsiveness should be reviewed as to outflow.

                \vismParagraph{XI.24}{24}{}
                \emph{10.} How \emph{as to smearing}? At the time of using it he smears his hands, lips, tongue and palate, and they become repulsive by being smeared with it. And even when washed, they have to be washed again and again in order to remove the smell. And, just as, when rice is being boiled, the husks, the red powder covering the grain, etc., rise up and smear the mouth, rim and lid of the cauldron, so too, when eaten it rises up during its cooking and simmering by the bodily fire that pervades the whole body, it turns into tartar, which smears the teeth, and it turns into spittle, phlegm, etc., which respectively smear the tongue, palate, etc.; and it turns into eye-dirt, ear-dirt, snot, urine, excrement, etc., which respectively smear the eyes, ears, nose and nether passages. And when these doors are smeared by it, they never become either clean or pleasing even though washed every day. And after one has washed a certain one of these, the hand has to be washed again.\footnote{\vismAssertFootnoteCounter{17}\vismHypertarget{XI.n17}{}“‘\emph{A certain one’ }is said with reference to the anal orifice. But those who are scrupulously clean by nature wash their hands again after washing the mouth, and so on” (\textbf{\cite{Vism-mhṭ}342}).} And after one has washed a certain one of these, the repulsiveness does not depart from it even after two or three washings with cow dung and clay and scented powder. This is how repulsiveness should be reviewed as to smearing.

                \vismParagraph{XI.25}{25}{}
                As he reviews repulsiveness in this way in ten aspects and strikes at it with thought and applied thought, physical nutriment \textcolor{brown}{\textit{[347]}} becomes evident to him in its repulsive aspect. He cultivates that sign\footnote{\vismAssertFootnoteCounter{18}\vismHypertarget{XI.n18}{}“‘\emph{That sign’: }that object as the sign for development, which sign is called \emph{physical nutriment} and has appeared in the repulsive aspect to one who gives his attention to it repeatedly in the ways already described. And there, while development occurs through the repulsive aspect, it is only the dhammas on account of which there comes to be the concept of physical nutriment that are repulsive, not the concept. But it is because the occurrence of development is contingent only upon dhammas with an individual essence, and because the profundity is due to that actual individual essence of dhammas that have individual essences, that the jhāna cannot reach absorption in it through apprehension of the repulsive aspect. For it is owing to profundity that the first pair of truths is hard to see” (\textbf{\cite{Vism-mhṭ}342–343}).} again and again, develops and \marginnote{\textcolor{teal}{\footnotesize\{402|344\}}}{}repeatedly practices it. As he does so, the hindrances are suppressed, and his mind is concentrated in access concentration, but without reaching absorption because of the profundity of physical nutriment as a state with an individual essence. But perception is evident here in the apprehension of the repulsive aspect, which is why this meditation subject goes by the name of “perception of repulsiveness in nutriment.”

                \vismParagraph{XI.26}{26}{}
                When a bhikkhu devotes himself to this perception of repulsiveness in nutriment, his mind retreats, retracts and recoils from craving for flavours. He nourishes himself with nutriment without vanity and only for the purpose of crossing over suffering, as one who seeks to cross over the desert eats his own dead child’s flesh (\textbf{\cite{S}II 98}). Then his greed for the five cords of sense desire comes to be fully understood without difficulty by means of the full understanding of the physical nutriment. He fully understands the materiality aggregate by means of the full-understanding of the five cords of sense desire. Development of mindfulness occupied with the body comes to perfection in him through the repulsiveness of “what is uncooked” and the rest. He has entered upon a way that is in conformity with the perception of foulness. And by keeping to this way, even if he does not experience the deathless goal in this life, he is at least bound for a happy destiny.

                This is the detailed explanation of the development of the perception of repulsiveness in nutriment.
            \section[\vismAlignedParas{§27–117}Defining of The Elements]{Defining of The Elements}
                \subsection[\vismAlignedParas{§27}Word Definitions]{Word Definitions}

                    \vismParagraph{XI.27}{27}{}
                    Now comes the description of the development of the definition of the four elements, which was listed as the “one defining” next to the perception of repulsiveness in nutriment (\hyperlink{III.105}{III.105}{}).

                    Herein, “defining” (\emph{vavatthāna}) is determining by characterizing individual essences.\footnote{\vismAssertFootnoteCounter{19}\vismHypertarget{XI.n19}{}“‘\emph{By characterizing individual essences’: }by making certain \emph{(upadhāraṇa) }of the specific characteristics of hardness, and so on. For this meditation subject does not consist in the observing of a mere concept, as in the case of the earth kasiṇa as a meditation subject, neither does it consist in the observing of the colour blue, etc., as in the case of the blue kasiṇa as a meditation subject, nor in the observing of the general characteristics of impermanence, etc., in formations, as in the case of insight as a meditation subject; but rather it consists in the observing of the individual essences of earth, and so on. That is why ‘by characterizing individual essences’ is said, which means, ‘by making certain of the specific characteristics of hardness, and so on”(Vism-mhṭ 344).} [The compound] \emph{catudhātuvavatthāna} (“four-element defining”) is [resolvable into] \emph{catunnaṃ dhātūnaṃ vavatthānaṃ} (“defining of the four elements”). “Attention given to elements,” “the meditation subject consisting of elements” and “defining of the four elements” all mean the same thing.

                    This is given in two ways: in brief and in detail. It is given in brief in the Mahāsatipaṭṭhāna Sutta (\textbf{\cite{D}II 294}), and in detail in the Mahāhatthipadopama Sutta (\textbf{\cite{M}I 185}), the Rāhulovāda Sutta (\textbf{\cite{M}I 421}), and the Dhātuvibhaṅga Sutta (\textbf{\cite{M}III 240}).
                \subsection[\vismAlignedParas{§28–30}Texts and Commentary in Brief]{Texts and Commentary in Brief}

                    \vismParagraph{XI.28}{28}{}
                    \marginnote{\textcolor{teal}{\footnotesize\{403|345\}}}{}It is given in brief in the Mahāsatipaṭṭhāna Sutta, for one of quick understanding whose meditation subject is elements, as follows: “Bhikkhus, just as though a skilled butcher or butcher’s apprentice had killed a cow and were seated at the crossroads \textcolor{brown}{\textit{[348]}} with it cut up into pieces, so too, bhikkhus, a bhikkhu reviews this body however placed, however disposed, as consisting of elements: In this body there are the earth element, the water element, the fire element, and the air element” \footnote{\vismAssertFootnoteCounter{20}\vismHypertarget{XI.n20}{}“Herein, as regards \emph{‘earth element,’ }etc., the meaning of element is the meaning of individual essence, the meaning of individual essence is the meaning of voidness, the meaning of voidness is the meaning of not-a-living-being. So it is just earth in the sense of individual essence, voidness and not-a-living-being that is the element; hence it is earth element; so too in the case of the water element, and the rest. The earth element is the element that is the foothold for the conascent material states. Likewise the water element is the element of their cohesion; the fire element is the element of their ripening; and the air element is the element of their conveyance and distension” (\textbf{\cite{Vism-mhṭ}345}).

                            To avoid confusion, it might be mentioned here that in “physical” earth, fire, water, and air, it would be held that all four elements are present in each equally, but that in “physical” earth the earth element is dominant in efficacy as the mode of hardness; and correspondingly with water and the rest. See e.g. \hyperlink{XIV.45}{XIV.45}{}.}(\textbf{\cite{D}II 294}).

                    \vismParagraph{XI.29}{29}{}
                    The meaning is this: \emph{just as though a clever butcher}, or his \emph{apprentice }who worked for his keep, \emph{had killed a cow} and divided it up \emph{and were seated at the crossroads}, reckoned as the intersection of the main roads going in the four directions, having laid it out part by part, \emph{so too a bhikkhu reviews the body, however placed }because it is in some one of the four postures and \emph{however disposed }because it is so placed, thus: \emph{In this body there are the earth element, the water element, the fire element, and the air element}.

                    \vismParagraph{XI.30}{30}{}
                    What is meant? Just as the butcher, while feeding the cow, bringing it to the shambles, keeping it tied up after bringing it there, slaughtering it, and seeing it slaughtered and dead, does not lose the perception “cow” so long as he has not carved it up and divided it into parts; but when he has divided it up and is sitting there, he loses the perception “cow” and the perception “meat” occurs; he does not think “I am selling cow” or “They are carrying cow away,” but rather he thinks “I am selling meat” or “They are carrying meat away”; so too this bhikkhu, while still a foolish ordinary person—both formerly as a layman and as one gone forth into homelessness—does not lose the perception “living being” or “man” or “person” so long as he does not, by resolution of the compact into elements, review this body, however placed, however disposed, as consisting of elements. But when he does review it as consisting of elements, he loses the perception “living being” and his mind establishes itself upon elements. That is why the Blessed One said: “Bhikkhus, just as though a skilled butcher … were seated at the crossroads … so too, bhikkhus, a bhikkhu … air element.”
                \subsection[\vismAlignedParas{§31–38}In Detail]{In Detail}

                    \vismParagraph{XI.31}{31}{}
                    \marginnote{\textcolor{teal}{\footnotesize\{404|346\}}}{}In the Mahāhatthipadopama Sutta it is given in detail for one of not over-quick understanding whose meditation subject is elements—and as here so also in the Rāhulovāda and Dhātuvibhaṅga Suttas—as follows:

                    “And what is the internal earth element, friends? Whatever there is internally in oneself that is hard, harsh,\footnote{\vismAssertFootnoteCounter{21}\vismHypertarget{XI.n21}{}\emph{Kharigata—}“harsh”: not in PED, but see \emph{khara.}} and clung to (acquired through kamma), that is to say, head hairs, body hairs, teeth, nails, skin, flesh, sinews, bones, bone marrow, kidney, heart, liver, midriff, spleen, lungs, bowels, entrails, gorge, dung, or whatever else there is internally in oneself that is hard, harsh, and clung to—this is called the internal earth element” (\textbf{\cite{M}I 185}). \textcolor{brown}{\textit{[349]}}

                    And: “What is the internal water element, friends? Whatever there is internally in oneself that is water, watery, and clung to, that is to say, bile, phlegm, pus, blood, sweat, fat, tears, grease, spittle, snot, oil of the joints, and urine, or whatever else there is internally in oneself that is water, watery, and clung to—this is called the internal water element” (\textbf{\cite{M}I 187}).

                    And: “What is the internal fire element, friends? Whatever there is internally in oneself that is fire, fiery, and clung to, that is to say, that whereby one is warmed, ages, and burns up, and whereby what is eaten, drunk, chewed and tasted gets completely digested, or whatever else there is internally in oneself that is fire, fiery, and clung to—this is called the internal fire element” (\textbf{\cite{M}I 188}).

                    And: “What is the internal air element, friends? Whatever there is internally in oneself that is air, airy, and clung to, that is to say, up-going winds, down-going winds, winds in the belly, winds in the bowels, winds that course through all the limbs, in-breath and out-breath, or whatever else there is internally in oneself that is air, airy, and clung to—this is called the internal air element” (\textbf{\cite{M}I 188}).

                    \vismParagraph{XI.32}{32}{}
                    Here is the commentary on the words that are not clear. \emph{Internally in oneself }(\emph{ajjhattaṃ paccattaṃ}): both these words are terms for what is one’s own (\emph{niyaka}), since what is one’s own is what is produced in one’s own self (\emph{attani jātaṃ}); the meaning is, included in one’s continuity (\emph{sasantati-pariyāpanna}). This is called “internal” (\emph{ajjhanaṃ = adhi + attā}, lit. “belonging-to-self”) because it occurs in self (\emph{attani—}locative case) just as in the world, speech among women (\emph{itthīsu—}loc. case) is called “[speech] belonging-to-women” (\emph{adhitthi}). And it is called, “in oneself” (\emph{paccattaṃ}) because it occurs owing to self (\emph{attānaṃ paṭicca}).\footnote{\vismAssertFootnoteCounter{22}\vismHypertarget{XI.n22}{}“What occurs in attendance \emph{(adhikicca) }upon self \emph{(attā) }by its pertaining to the state that may be taken as self because it is included in one’s own continuity as internal \emph{(ajjhatta)” }(\textbf{\cite{Vism-mhṭ}347}).}

                    \vismParagraph{XI.33}{33}{}
                    \emph{Hard}: rigid. \emph{Harsh}: rough. Herein, the first is a word for the characteristic, while the second is a word for the mode; for the earth element is characterized as hard, but its mode is rough, which is why it is called “harsh.” \emph{Clung to}: taken firmly [by kamma]; the meaning is, firmly taken, seized, adhered to, as “I,” “mine” (see \hyperlink{XI.89}{§89f.}{}).

                    \vismParagraph{XI.34}{34}{}
                    \emph{That is to say}: the word \emph{seyyathidaṃ }(“that is to say”) is a particle; its meaning is, “What is that?” Next, showing what that is, “head hairs, body hairs,” etc., is \marginnote{\textcolor{teal}{\footnotesize\{405|347\}}}{}said. And here the \emph{brain }must be added since it has to be understood that the earth element needs to be described in twenty modes. Or \emph{whatever else}: the earth element included in the remaining three portions.

                    \vismParagraph{XI.35}{35}{}
                    \textcolor{brown}{\textit{[350]}} It flows (\emph{appoti}), flows on (\emph{pappoti}), to such and such a place as a state of streaming, thus it is water (\emph{āpo}). The \emph{watery }(\emph{āpo-gata}) is what is gone (\emph{gata}) among such various kinds of water (\emph{āpo}) as the kamma-originated, and so on. What is that? It is what has the water element’s characteristic of cohesion.

                    \vismParagraph{XI.36}{36}{}
                    \emph{Fire }(\emph{tejo}) [is definable] as heating (\emph{tejana}). The \emph{fiery }(\emph{tejo-gata}) is what is gone (\emph{gata}), in the way already described, among the kinds of fire (\emph{tejo}). What is that? It is what has the characteristic of heat. \emph{Whereby}: by means of which the fire element, when excited, this body is \emph{warmed}, becomes heated by the state of one-day fever,\footnote{\vismAssertFootnoteCounter{23}\vismHypertarget{XI.n23}{}\emph{Jara—“fever”}: not in PED; see \textbf{\cite{A}V 100}; \textbf{\cite{Nidd}I 17}.} and so on. \emph{Ages}: whereby this body grows old, reaches the decline of the faculties, loss of strength, wrinkles, grayness, and so on. \emph{Burns up}: whereby, when excited, it causes this body to burn, and the person cries out, “I am burning, I am burning!” and longs for ghee a hundred times washed and for \emph{gosīsa }sandalwood ointment, etc., and for the breeze of a fan. \emph{And whereby what is eaten, drunk, chewed and tasted gets completely digested}: whereby the boiled rice, etc., that is eaten, or the beverage, etc., that is drunk, or the hard food consisting of flour biscuits, etc., that is chewed, the mango fruit, honey, molasses, etc., that is tasted, gets completely cooked; gets its juice, etc., extracted, is the meaning. And here the first three kinds of fire element [that is to say, “is warmed,” “ages,” and “burns up”] are of fourfold origination (\hyperlink{XX.27}{XX.27ff.}{}), while the last is only kamma-originated.

                    \vismParagraph{XI.37}{37}{}
                    \emph{Air }(\emph{vāyo}) [is definable] as blowing (\emph{vāyana}). The \emph{airy }(\emph{vāyo-gata}) is what is gone (\emph{gata}), in the way already described, among the kinds of air. What is that? It is what has the characteristic of distension.\footnote{\vismAssertFootnoteCounter{24}\vismHypertarget{XI.n24}{}\emph{Vitthambhana—}“distension”: the word most usually employed to describe the air element. It is often rendered by “supporting,” a word earmarked here for \emph{nissaya. }The twofold function of the air element is (a) to uphold \emph{(sandhārana) }by distending \emph{(vitthambhana) }and preventing collapse (§92), and (b) to move \emph{(samudīraṇa), }or more strictly, cause the appearance of motion \emph{(calana, }see n. 37). In \hyperlink{XIV.61}{XIV.61}{} it is said to cause \emph{thambhana, }rendered by “stiffening”; but there is the description of the earth element as \emph{thaddha }(e.g. §39; pp. of \emph{thambhati, }from which the noun \emph{thambhana }comes), rendered by “stiffenedness.” It may also be noted that the word \emph{sandhāraṇa }(upholding) is used to describe both the earth element (\hyperlink{XIV.47}{XIV.47}{}) and the air element (\hyperlink{XIV.61}{XIV.61}{}).} \emph{Upgoing winds}: winds (forces) mounting upwards that cause the occurrence of vomiting, belching, and so on. \emph{Down-going winds}: winds (forces) descending downwards that expel excrement and urine. \emph{Winds in the belly}: winds (forces) outside the bowels. \emph{Winds in the bowels}: winds (forces) inside the bowels. \emph{Winds that course through all the limbs}: winds (forces) that produce flexing, extending, etc., and are distributed over the limbs and the whole body by means of the network of veins (nerves). \emph{In-breath}: wind in the nostrils entering in. \emph{Out-breath}: wind in the nostrils issuing out. And here the first five are of fourfold origination. In-breath and out-breath are consciousness-originated. \textcolor{brown}{\textit{[351]}} \marginnote{\textcolor{teal}{\footnotesize\{406|348\}}}{}In each instance the phrase \emph{or whatever else }comprises respectively the water element, the fire element, or the air element included in the other three portions.

                    \vismParagraph{XI.38}{38}{}
                    So the four elements have been detailed in forty-two aspects, that is to say, the earth element in twenty aspects, the water element in twelve, the fire element in four, and the air element in six.

                    This, firstly, is the commentary on the texts here.
                \subsection[\vismAlignedParas{§39–44}Method of Development in Brief]{Method of Development in Brief}

                    \vismParagraph{XI.39}{39}{}
                    As regards the method of development here, however, to discern the elements in detail in this way, “The head hairs are the earth element, the body hairs are the earth element,” appears redundant to a bhikkhu of quick understanding, though the meditation subject becomes clear to him if he gives his attention to it in this way: “What has the characteristic of stiffenedness is the earth element, what has the characteristic of cohesion is the water element, what has the characteristic of ripening (maturing) is the fire element, what has the characteristic of distending (supporting) is the air element.” But when one of not over-quick understanding gives his attention to it in this way, it appears obscure and unevident, and it only becomes plain to him if he gives his attention to it in the first-mentioned way. Why?

                    \vismParagraph{XI.40}{40}{}
                    Suppose two bhikkhus are reciting a text with many elided repetitions, then the bhikkhu with the quicker understanding fills out the elided repetitions once or twice, after which he goes on doing the recital with only the two end parts of the elisions. Here the one of less quick understanding says, “What is he reciting? Why, he does not even give one time to move one’s lips! If the recitation is done like this, when shall we ever get familiar with the text?” and so he does his recitation filling out each elision as it comes. Then the other says, “What is he reciting? Why, he never lets one get to the end of it! If the recitation is done like this; when shall we ever get to the end of it?” So too, the detailed discerning of the elements by head hairs, etc., appears redundant to one of quick understanding, though the meditation subject becomes clear to him if he gives his attention to it in brief in this way, “What has the characteristic of stiffenedness is the earth element,” and so on. But when the other gives his attention to it in this way, it appears obscure and unevident, and it only becomes plain to him if he gives his attention in detail by head hairs and so on.

                    \vismParagraph{XI.41}{41}{}
                    So firstly, one of quick understanding who wants to develop this meditation subject should go into solitary retreat. Then he should advert to his own entire material body and discern the elements in brief in this way: “In this body what is stiffenedness or harshness is the earth element, what is cohesion or fluidity\footnote{\vismAssertFootnoteCounter{25}\vismHypertarget{XI.n25}{}\emph{Drava-bhāva—}“fluidity”: not in PED.} \textcolor{brown}{\textit{[352]}} is the water element, what is maturing (ripening) or heat is the fire element, what is distension or movement is the air element.” And he should advert and give attention to it and review it again and again as “earth element, water element,” that is to say, as mere elements, not a living being, and soulless.

                    \vismParagraph{XI.42}{42}{}
                    As he makes effort in this way it is not long before concentration arises in him, which is reinforced by understanding that illuminates the classification of \marginnote{\textcolor{teal}{\footnotesize\{407|349\}}}{}the elements, and which is only access and does not reach absorption because it has states with individual essences as its object.

                    \vismParagraph{XI.43}{43}{}
                    Or alternatively, there are these four [bodily] parts mentioned by the General of the Dhamma [the Elder Sāriputta] for the purpose of showing the absence of any living being in the four great primary elements thus: “When a space is enclosed with bones and sinews and flesh and skin, there comes to be the term ‘material form’ (\emph{rūpa})” (\textbf{\cite{M}I 190}). And he should resolve each of these [as a separate entity], separating them out by the hand of knowledge, and then discern them in the way already stated thus: “In these what is stiffenedness or harshness is the earth element.” And he should again and again advert to them, give attention to them and review them as mere elements, not a living being, not a soul.

                    \vismParagraph{XI.44}{44}{}
                    As he makes effort in this way, it is not long before concentration arises in him, which is reinforced by understanding that illuminates the classification of the elements, and which is only access and does not reach absorption because it has states with individual essences as its object.

                    This is the method of development when the definition of the elements is given in brief.
                \subsection[\vismAlignedParas{§45–85}Method of Development in Detail]{Method of Development in Detail}

                    \vismParagraph{XI.45}{45}{}
                    The method given in detail should be understood in this way. A meditator of not over-quick understanding who wants to develop this meditation subject should learn the elements in detail in the forty-two aspects from a teacher, and he should live in an abode of the kind already described. Then, when he has done all the duties, he should go into solitary retreat and develop the meditation subject in four ways thus: (1) with constituents in brief, (2) with constituents by analysis, (3) with characteristics in brief, and (4) with characteristics by analysis.
                    \subsubsection[\vismAlignedParas{§46}(1) With Constituents in Brief]{(1) With Constituents in Brief}

                        \vismParagraph{XI.46}{46}{}
                        Herein, how does he develop it \emph{with constituents in brief}? Here a bhikkhu does his defining in this way, “In twenty of the parts what has the stiffened mode is the earth element,” and he does his defining thus, “In twelve parts the liquid called water with the mode of cohesion is the water element,” \textcolor{brown}{\textit{[353]}} and he does his defining thus, “In four parts what matures (what has the mode of ripening) is the fire element,” and he does his defining thus, “In six parts what has the mode of distending is the air element.” As he defines them in this way they become evident to him. As he again and again adverts to them and gives his attention to them, concentration arises as access only.
                    \subsubsection[\vismAlignedParas{§47–83}(2) With Constituents by Analysis]{(2) With Constituents by Analysis}

                        \vismParagraph{XI.47}{47}{}
                        However, if his meditation subject is not successful while he develops it in this way, then he should develop it \emph{with constituents by analysis}. How? Firstly, the bhikkhu should carry out all the directions given for the thirty-two-fold aspect in the description of mindfulness occupied with the body as a meditation subject (\hyperlink{VIII.48}{VIII.48}{}–\hyperlink{VIII.78}{78}{}), namely, the sevenfold skill in learning and the tenfold skill in giving \marginnote{\textcolor{teal}{\footnotesize\{408|350\}}}{}attention, and he should start with the verbal recitation, in direct and reverse order, of the skin pentad and so on, without omitting any of it. The only difference is this: there, after giving attention to the head hairs, etc., as to colour, shape, direction, location, and delimitation, the mind had to be fixed by means of repulsiveness (\hyperlink{VIII.83}{VIII.83}{}), but here it is done by means of elements. Therefore at the end of each part after giving attention to head hairs, etc., each in the five ways beginning with colour (\hyperlink{VIII.83}{VIII.83}{}), attention should be given as follows.

                        \vismParagraph{XI.48}{48}{}
                        These things called \emph{head hairs }grow on the inner skin that envelops the skull. Herein, just as when \emph{kuṇṭha }grasses grow on the top of an anthill, the top of the termite-mound does not know, “\emph{Kuṇṭha }grasses are growing on me,” nor do the \emph{kuṇṭha }grasses know, “We are growing on the top of a termite-mound,” so too, the inner skin that covers the skull does not know, “Head hairs grow on me,” nor do the head hairs know, “We grow on inner skin that envelops a skull.” These things are devoid of mutual concern and reviewing. So what are called \emph{head hairs} are a particular component of this body, without thought, [morally] indeterminate, void, not a living being, rigid (stiffened) earth element.

                        \vismParagraph{XI.49}{49}{}
                        \emph{Body hairs }grow on the inner skin that envelops the body. Herein, just as, when \emph{dabba }grasses grow on the square in an empty village, the square in the empty village does not know\emph{, “Dabba }grasses grow on me,” nor do the \emph{dabba }grasses know, “We grow on the square in an empty village,” so too, the inner skin that envelops the body does not know, “Body hairs grow on me,” nor do the body hairs know, “We grow on inner skin that envelops a body.” These things are devoid of mutual concern and reviewing. So what are called \emph{body hairs} are a particular component of this body, without thought, indeterminate, void, not a living being, rigid earth element.

                        \vismParagraph{XI.50}{50}{}
                        \emph{Nails} grow on the tips of the fingers and toes. Herein, just as, when children play a game by piercing \emph{madhuka}-fruit kernels with sticks, the sticks \textcolor{brown}{\textit{[354]}} do not know, \emph{“Madhuka}-fruit kernels are put on us,” nor do the \emph{madhuka}-fruit kernels know, “We are put on sticks,” so too, the fingers and toes do not know, “Nails grow on our tips,” nor do the nails know, “We grow on the tips of fingers and toes.” These things are devoid of mutual concern and reviewing. So what are called \emph{nails} are a particular component of this body, without thought, indeterminate, void, not a living being, rigid earth element.

                        \vismParagraph{XI.51}{51}{}
                        \emph{Teeth }grow in the jaw bones. Herein, just as, when posts are placed by builders in stone sockets and fastened with some kind of cement,\footnote{\vismAssertFootnoteCounter{26}\vismHypertarget{XI.n26}{}\emph{Silesa—}”cement”: not in this meaning in PED; \textbf{\cite{M-a}I 37} \emph{saṃsilesa.}} the sockets do not know, “Posts are placed in us,” nor do the posts know, “We are placed in sockets,” so too, the jaw bones do not know, “Teeth grow in us,” nor do the teeth know, “We grow in jaw bones’.” These things are devoid of mutual concern and reviewing. So what are called \emph{teeth} are a particular component of this body, without thought, indeterminate, void, not a living being, rigid earth element.

                        \vismParagraph{XI.52}{52}{}
                        \emph{Skin }is to be found covering the whole body. Herein, just as, when a big lute is covered with damp ox-hide, the lute does not know, “I am covered with damp ox-hide,” nor does the damp ox-hide know, “A lute is covered by me,” so too, the \marginnote{\textcolor{teal}{\footnotesize\{409|351\}}}{}body does not know, “I am covered by skin,” nor does the skin know, “A body is covered by me.” These things are devoid of mutual concern and reviewing. So what is called \emph{skin} is a particular component of this body, without thought, indeterminate, void, not a living being, rigid earth element.

                        \vismParagraph{XI.53}{53}{}
                        \emph{Flesh }is to be found plastered over the framework of bones. Herein, just as, when a wall is plastered with thick clay, the wall does not know, “I am plastered with thick clay,” nor does the thick clay know, “A wall is plastered with me,” so too, the framework of bones does not know, “I am plastered with flesh consisting of nine hundred pieces of flesh,” nor does the flesh know, “A framework of bones is plastered with me.” These things are devoid of mutual concern and reviewing. So what is called \emph{flesh} is a particular component of this body, without thought, indeterminate, void, not a living being, rigid earth element.

                        \vismParagraph{XI.54}{54}{}
                        \emph{Sinews }are to be found in the interior of the body binding the bones together. Herein, just as, when withies and sticks are bound together with creepers, the withies and sticks do not know \textcolor{brown}{\textit{[355]}} “We are bound together with creepers,” nor do the creepers know, “Withies and sticks are bound together by us,” so too, the bones do not know, “We are bound by sinews,” nor do the sinews know, “Bones are bound together by us.” These things are devoid of mutual concern and reviewing. So what are called \emph{sinews} are a particular component of this body, without thought, indeterminate, void, not a living being, rigid earth element.

                        \vismParagraph{XI.55}{55}{}
                        As to the \emph{bones}, the heel bone is to be found holding up the ankle bone, the ankle bone holding up the shin bone, the shin bone the thigh bone, the thigh bone the hip bone, the hip bone the backbone, the backbone the neck bone, and the neck bone is to be found holding up the cranium bone. The cranium bone rests on the neck bone, the neck bone on the backbone, the backbone on the hip bone, the hip bone on the thigh bone, the thigh bone on the shin bone, the shin bone on the ankle bone, the ankle bone on the heel bone.

                        \vismParagraph{XI.56}{56}{}
                        Herein, just as, when bricks, timber or [blocks of dried] cow dung are built up, those below do not know, “We each stand holding up those above us,” nor do those above know, “We each rest on those below us,” so too, the heel bone does not know, “I stand holding up the ankle bone,” nor does the ankle bone know, “I stand holding up the shin bone,” nor does the shin bone know, “I stand holding up the thigh bone,” nor does the thigh bone know, “I stand holding up the hip bone,” nor does the hip bone know, “I stand holding up the backbone,” nor does the backbone know, “I stand holding up the neck bone,” nor does the neck bone know, “I stand holding up the cranium bone,” nor does the cranium bone know, “I rest on the neck bone,” nor does the neck bone know, “I rest on the backbone,” nor does the backbone know, “I rest on the hip bone,” nor does the hip bone know, “I rest on the thigh bone,” nor does the thigh bone know, “I rest on the shin bone,” nor does the shin bone know, “I rest on the ankle bone,” nor does the ankle bone know, “I rest on the heel bone.” These things are devoid of mutual concern and reviewing. So what are called\emph{ bones} \textcolor{brown}{\textit{[356]}} are a particular component of this body, without thought, indeterminate, void, not a living being, rigid earth element.

                        \vismParagraph{XI.57}{57}{}
                        \marginnote{\textcolor{teal}{\footnotesize\{410|352\}}}{}\emph{Bone marrow }is to be found inside the various bones. Herein, just as, when boiled bamboo sprouts, etc., are put inside bamboo joints, etc., the bamboo joints, etc., do not know, “Bamboo sprouts, etc., are put in us,” nor do the bamboo sprouts, etc., know, “We are inside bamboo joints, etc.,” so too, the bones do not know, “Marrow is inside us,” nor does the bone marrow know, “I am inside bones.” These things are devoid of mutual concern and reviewing. So what is called \emph{bone marrow} is a particular component of this body, without thought, indeterminate, void, not a living being, rigid earth element.

                        \vismParagraph{XI.58}{58}{}
                        \emph{Kidney }is to be found on each side of the heart flesh, being fastened by the stout sinew that starts out with a single root from the base of the neck and divides into two after going a short way. Herein, just as, when a pair of mango fruits are bound together by their stalk, the stalk does not know, “A pair of mango fruits is bound together by me,” nor do the pair of mango fruits know, “We are bound together by a stalk,” so too, the stout sinew does not know, “Kidneys are bound together by me,” nor does the kidney know, “I am bound together by a stout sinew.” These things are devoid of mutual concern and reviewing. So what is called \emph{kidney} is a particular component of this body, without thought, indeterminate, void, not a living being, rigid earth element.

                        \vismParagraph{XI.59}{59}{}
                        \emph{Heart }is to be found in the inside of the body near the middle of the frame of the ribs. Herein, just as, when a piece of meat is placed near the framework of an old cart, the inside of the framework of the old cart does not know, “A piece of meat is placed near the middle of me,” nor does the piece of meat know, “I am near the middle of the inside of the framework of an old cart,” so too, the inside of the framework of the ribs does not know, “A heart is near the middle of me,” nor does the heart know, “I am near the middle of the inside of a framework of ribs.” These things are devoid of mutual concern and reviewing. So what is called \emph{heart} is a particular component of this body, without thought, indeterminate, void, not a living being, rigid earth element.

                        \vismParagraph{XI.60}{60}{}
                        \emph{Liver }is to be found inside the body, near the right side between the two breasts. Herein, just as, when a twin lump of meat is stuck on the side of a cooking pot, the side of the cooking pot does not know, “A twin lump of meat is stuck on me,” nor does the twin lump of meat know, \textcolor{brown}{\textit{[357]}} “I am stuck on the side of a cooking pot,” so too, the right side between the breasts does not know, “Liver is near me,” nor does the liver know, “I am near a right side between two breasts.” These things are devoid of mutual concern and reviewing. So what is called \emph{liver} is a particular component of this body, without thought, indeterminate, void, not a living being, rigid earth element.

                        \vismParagraph{XI.61}{61}{}
                        As to the \emph{midriff}, the concealed midriff is to be found surrounding the heart and kidney, while the unconcealed midriff is to be found covering the flesh under the skin in the whole body. Herein, just as, when meat is wrapped in a rag, the meat does not know, “I am wrapped in a rag,” nor does the rag know, “Meat is wrapped in me,” so too, the heart and kidney, and the flesh in the whole body, do not know, “I am concealed by midriff,” nor does the midriff know, “Heart and kidney, and flesh in a whole body, are concealed by me.” These things are devoid of mutual concern and reviewing. So what is called \emph{midriff }is a particular \marginnote{\textcolor{teal}{\footnotesize\{411|353\}}}{}component of this body, without thought, indeterminate, void, not a living being, rigid earth element.

                        \vismParagraph{XI.62}{62}{}
                        \emph{Spleen }is to be found near the upper side of the belly lining on the left side of the heart. Herein, just as, when a lump of cow dung is near the upper side of a barn, the upper side of the barn does not know, “A lump of cow dung is near me,” nor does the lump of cow dung know, “I am near the upper side of a barn,” so too, the upper side of the belly lining does not know, “Spleen is near me,” nor does the spleen know, “I am near the upper side of a belly lining.” These things are devoid of mutual concern and reviewing. So what is called \emph{spleen} is a particular component of this body, without thought, indeterminate, void, not a living being, rigid earth element.

                        \vismParagraph{XI.63}{63}{}
                        \emph{Lungs }are to be found inside the body between the two breasts, hanging over the heart and liver and concealing them. Herein, just as when a bird’s nest is hanging inside an old barn, the inside of the old barn does not know, “A bird’s nest is hanging in me,” nor does the bird’s nest know, “I am hanging inside an old barn,” so too, \textcolor{brown}{\textit{[358]}} the inside of the body does not know, “Lungs are hanging in me,” nor do the lungs know, “We are hanging inside such a body.” These things are devoid of mutual concern and reviewing. So what is called \emph{lungs} is a particular component of this body, without thought, indeterminate, void, not a living being, rigid earth element.

                        \vismParagraph{XI.64}{64}{}
                        \emph{Bowel }is to be found inside the body extending from the base of the neck to the excrement passage. Herein, just as, when the carcass of a large beheaded rat snake\footnote{\vismAssertFootnoteCounter{27}\vismHypertarget{XI.n27}{}\emph{Dhammani—}“rat snake”: not in this sense in PED; see \textbf{\cite{A-a}} 459.} is coiled up and put into a trough of blood, the red trough does not know, “A rat snake’s carcass has been put in me,” nor does the rat snake’s carcass know, “I am in a red trough,” so too, the inside of the body does not know, “A bowel is in me,” nor does the bowel know, “I am in a body.” These things are devoid of mutual concern and reviewing. So what is called the \emph{bowel} is a particular component of this body, without thought, indeterminate, void, not a living being, rigid earth element.

                        \vismParagraph{XI.65}{65}{}
                        \emph{Entrails }are to be found in the interspaces between the twenty-one coils of the bowel, binding them together. Herein, just as, when ropes are found sewing together a rope ring for wiping the feet, the rope ring for wiping the feet does not know, “Ropes are to be found sewing me together,” nor do the ropes know, “We are to be found sewing together a rope ring,” so too, the bowel does not know, “Entrails are to be found binding me together,” nor do the entrails know, “We are to be found binding a bowel together.” These things are devoid of mutual concern and reviewing. So what is called \emph{entrails} is a particular component of this body, without thought, indeterminate, void, not a living being, rigid earth element.

                        \vismParagraph{XI.66}{66}{}
                        \emph{Gorge }is what is eaten, drunk, chewed and tasted that lies in the stomach. Herein, just as, when a dog’s vomit lies in a dog’s bowl, the dog’s bowl does not know, “Dog’s vomit is lying in me,” nor does the dog’s vomit know, “I am lying \marginnote{\textcolor{teal}{\footnotesize\{412|354\}}}{}in a dog’s bowl,” so too, the stomach does not know, “Gorge is lying in me,” nor does the gorge know, “I am lying in a stomach.” These things are devoid of mutual concern and reviewing. So what is called \emph{gorge} is a particular component of this body, without thought, indeterminate, void, not a living being, rigid earth element.

                        \vismParagraph{XI.67}{67}{}
                        \emph{Dung }is to be found at the end of the bowel, which resembles a bamboo joint eight fingerbreadths long and is called the “receptacle for digested food.” \textcolor{brown}{\textit{[359]}} Herein, just as, when soft brown clay is impacted in a bamboo joint, the bamboo joint does not know, “Brown clay is in me,” nor does brown clay know, “I am in a bamboo joint,” so too, the receptacle for digested food does not know, “Dung is in me,” nor does the dung know, “I am in a receptacle for digested food.” These things are devoid of mutual concern and reviewing. So what is called \emph{dung} is a particular component of this body, without thought, indeterminate, void, not a living being, rigid earth element.

                        \vismParagraph{XI.68}{68}{}
                        \emph{Brain }is to be found in the interior of the skull. Herein, just as, when a lump of dough is put inside an old gourd rind, the gourd rind does not know, “A lump of dough is in me,” nor does the lump of dough know, “I am inside a gourd rind,” so too, the inside of the skull does not know, “Brain is in me,” nor does the brain know, “I am inside a skull.” These things are devoid of mutual concern and reviewing. So what is called \emph{brain} is a particular component of this body, without thought, indeterminate, void, not a living being, rigid earth element.

                        \vismParagraph{XI.69}{69}{}
                        As to \emph{bile}, the free bile, which is bound up with the life faculty, is to be found soaking the whole body, while the local bile is to be found in the bile container (gall-bladder). Herein, just as, when oil has soaked a cake, the cake does not know, “Oil soaks me,” nor does the oil know, “I soak a cake,” so too, the body does not know, “Free bile soaks me,” nor does the free bile know, “I soak a body.” And just as, when a\emph{ kosāṭakī }(loofah) creeper bladder is filled with rain water, the \emph{kosāṭakī }creeper bladder does not know, “Rain water is in me,” nor does the rain water know, “I am in a \emph{kosāṭakī }creeper bladder,” so too, the bile bladder does not know, “Local bile is in me,” nor does the local bile know, “I am in a bile bladder.” These things are devoid of mutual concern and reviewing. So what is called \emph{bile }is a particular component of this body, without thought, indeterminate, void, not a living being, liquid water element in the mode of cohesion.

                        \vismParagraph{XI.70}{70}{}
                        \emph{Phlegm }is to be found on the surface of the stomach and measures a bowlful. Herein, just as, when a cesspool has a surface of froth, the cesspool does not know, “A surface of froth is on me,” nor does the surface of froth \textcolor{brown}{\textit{[360]}} know, “I am on a cesspool,” so too, the surface of the stomach does not know, “Phlegm is on me” nor does the phlegm know, “I am on the surface of a stomach.” These things are devoid of mutual concern and reviewing. So what is called \emph{phlegm} is a particular component of this body, without thought, indeterminate, void, not a living being, liquid water element in the mode of cohesion.

                        \vismParagraph{XI.71}{71}{}
                        \emph{Pus }has no fixed location. It is to be found wherever the blood stagnates and goes bad in a part of the body damaged by wounds caused by splinters and thorns, and by burns due to fire, or where boils, carbuncles, etc., appear. Herein, just as, when a tree oozes gum through being hit by, say, an axe, the parts of the \marginnote{\textcolor{teal}{\footnotesize\{413|355\}}}{}tree that have been hit do not know, “Gum is in us,” nor does the gum know, “I am in a part of a tree that has been hit,” so too, the parts of the body wounded by splinters, thorns, etc., do not know, “Pus is in us,” nor does the pus know, “I am in such places.” These things are devoid of mutual concern and reviewing. So what is called \emph{pus} is a particular component of this body, without thought, indeterminate, void, not a living being, liquid water element in the mode of cohesion.

                        \vismParagraph{XI.72}{72}{}
                        As to \emph{blood}, the mobile blood is to be found, like the bile, soaking the whole body. The stored blood, is to be found filling the lower part of the liver’s site to the extent of a bowlful, wetting the kidney, heart, liver and lungs. Herein, the definition of the mobile blood is similar to that of the free bile. But as to the other, just as, when rain water seeps through an old pot and wets clods and stumps below, the clods and stumps do not know, “We are being wetted with water,” nor does the water know, “I am wetting clods and stumps,” so too, the lower part of the liver’s site, or the kidneys, etc., respectively do not know, “Blood is in me,” or “We are being wetted,” nor does the blood know, “I fill the lower part of a liver’s site, am wetting a kidney, and so on.” These things are devoid of mutual concern and reviewing. So what is called \emph{blood} is a particular component of this body, without thought, indeterminate, void, not a living being, liquid water element in the mode of cohesion.

                        \vismParagraph{XI.73}{73}{}
                        \emph{Sweat }is to be found filling the openings of the pores of the head hairs and body hairs when there is heat due to fires, etc., and it trickles out of them. Herein, just as, when \textcolor{brown}{\textit{[361]}} bunches of lily bud stems and lotus stalks are pulled up out of water, the openings in the bunches of lilies, etc., do not know, “Water trickles from us,” nor does the water trickling from the openings in the bunches of lilies, etc., know, “I am trickling from openings in bunches of lilies, etc.,” so too, the openings of the pores of the head hairs and body hairs do not know, “Sweat trickles from us,” nor does the sweat know, “I trickle from openings of pores of head hairs and body hairs.” These things are devoid of mutual concern and reviewing. So what is called \emph{sweat} is a particular component of this body, without thought, indeterminate, void, not a living being, liquid water element in the mode of cohesion.

                        \vismParagraph{XI.74}{74}{}
                        \emph{Fat }is the thick unguent to be found pervading the whole body of one who is stout, and on the shank flesh, etc., of one who is lean. Herein, just as, when a heap of meat is covered by a yellow rag, the heap of meat does not know, “A yellow rag is next to me,” nor does the yellow rag know, “I am next to a heap of meat,” so too, the flesh to be found on the whole body, or on the shanks, etc., does not know, “Fat is next to me,” nor does the fat know,”I am next to flesh on a whole body, or on the shanks, and so on.” These things are devoid of mutual concern and reviewing. So what is called \emph{fat} is a particular component of this body, without thought, indeterminate, void, not a living being, thick-liquid water element in the mode of cohesion.

                        \vismParagraph{XI.75}{75}{}
                        \emph{Tears}, when produced, are to be found filling the eye sockets or trickling out of them. Herein, just as, when the sockets of young palm kernels are filled with water, the sockets of the young palm kernels do not know, “Water is in us,” nor \marginnote{\textcolor{teal}{\footnotesize\{414|356\}}}{}does the water in the sockets of the young palm kernels know, “I am in sockets of young palm kernels,” so too, the eye sockets do not know, “Tears are in us,” nor do the tears know, “We are in eye sockets.” These things are devoid of mutual concern and reviewing. So what is called \emph{tears} is a particular component of this body, without thought, indeterminate, void, not a living being, liquid water element in the mode of cohesion.

                        \vismParagraph{XI.76}{76}{}
                        \emph{Grease }is the melted unguent to be found on the palms and backs of the hands, on the soles and backs of the feet, on the nose and forehead and on the points of the shoulders, when heated by fire, and so on. Herein, just as, when rice gruel has oil put on it, the rice gruel does not know, “Oil is spread over me,” nor does the oil know, “I am spread over rice gruel,” so too, the place consisting of the palm of the hand, etc., \textcolor{brown}{\textit{[362]}} does not know, “Grease is spread over me,” nor does the grease know, “I am spread over places consisting of the palm of the hand, and so on.” These things are devoid of mutual concern and reviewing. So what is called \emph{grease} is a particular component of this body, without thought, indeterminate, void, not a living being, liquid water element in the mode of cohesion.

                        \vismParagraph{XI.77}{77}{}
                        \emph{Spittle }is to be found on the surface of the tongue after it has descended from the cheeks on both sides, when there is a condition for the arising of spittle. Herein, just as, when a hollow in a river bank is constantly oozing with water, the surface of the hollow does not know, “Water lies on me,” nor does the water know, “I lie on the surface of a hollow,” so too, the surface of the tongue does not know, “Spittle that has descended from cheeks on both sides is on me,” nor does the spittle know, “I have descended from cheeks on both sides and am on the surface of a tongue.” These things are devoid of mutual concern and reviewing. So what is called \emph{spittle} is a particular component of this body, without thought, indeterminate, void, not a living being, liquid water element in the mode of cohesion.

                        \vismParagraph{XI.78}{78}{}
                        \emph{Snot}, when produced, is to be found filling the nostrils or trickling out of them. Herein, just as, when a bag\footnote{\vismAssertFootnoteCounter{28}\vismHypertarget{XI.n28}{}\emph{Sippikā—}“bag” (?): not in this sense in PED.} is loaded with rotting curd, the bag does not know, “Rotting curd is in me,” nor does the rotting curd know, “I am in a bag,” so too, the nostrils do not know, “Snot is in us,” nor does the snot know, “I am in nostrils.” These things are devoid of mutual concern and reviewing. So what is called \emph{snot} is a particular component of this body, without thought, indeterminate, void, not a living being, liquid water element in the mode of cohesion.

                        \vismParagraph{XI.79}{79}{}
                        \emph{Oil of the joints }is to be found in the hundred and eighty joints serving the function of lubricating the joints of the bones. Herein, just as, when an axle is lubricated with oil, the axle does not know, “Oil lubricates me,” nor does the oil know, “I lubricate an axle,” so too, the hundred and eighty joints do not know, “Oil of the joints lubricates us,” nor does the oil of the joints know, “I lubricate a hundred and eighty joints.” These things are devoid of mutual concern and reviewing. So what is called \emph{oil of the joints} is a particular component of this body, without thought, indeterminate, void, not a living being, liquid water element in the mode of cohesion.

                        \vismParagraph{XI.80}{80}{}
                        \marginnote{\textcolor{teal}{\footnotesize\{415|357\}}}{}\emph{Urine }is to be found inside the bladder. Herein, just as, when a porous pot is put upside down in a cesspool, the porous pot does not know, “Cesspool filtrate is in me,” nor does the cesspool filtrate know, “I am in a porous pot,” so too, the bladder does not know, \textcolor{brown}{\textit{[363]}} “Urine is in me,” nor does the urine know, “I am in a bladder.” These things are devoid of mutual concern and reviewing. So what is called \emph{urine} is a particular component of this body, without thought, indeterminate, void, not a living being, liquid water element in the mode of cohesion.

                        \vismParagraph{XI.81}{81}{}
                        When he has given his attention in this way to the body hairs, etc., he should then give his attention to the [four] fire components thus: \emph{That whereby one is warmed—}this is a particular component of this body, without thought, indeterminate, void, not a living being; it is fire element in the mode of maturing (ripening).

                        \emph{That whereby one ages …}

                        \emph{That whereby one burns up …}

                        \emph{That whereby what is eaten, drunk, chewed and tasted becomes completely digested—}this is a particular component of this body, without thought, indeterminate, void, not a living being; it is fire element in the mode of maturing (ripening).

                        \vismParagraph{XI.82}{82}{}
                        After that, having discovered the \emph{up-going winds }(\emph{forces}) as upgoing, the \emph{down-going winds }(\emph{forces}) as down-going, the \emph{winds }(\emph{forces}) \emph{in the belly }as in the belly, the \emph{winds }(\emph{forces})\emph{ in the bowels }as in the bowels, the \emph{winds }(\emph{forces}) \emph{that course through all the limbs }as coursing through all the limbs, and \emph{in-breath and out-breath }as in-breath and out-breath, he should give his attention to these [six] air components in this way: What is called \emph{up-going winds }(\emph{forces}) is a particular component of this body, without thought, indeterminate, void, not a living being; it is air element in the mode of distending.

                        What is called \emph{down-going winds (forces) }…

                        What is called \emph{winds (forces) in the belly }…

                        What is called \emph{winds (forces) in the bowels }…

                        What is called\emph{ winds (forces) that course through all the limbs }…

                        What is called \emph{in-breath and out-breath} is a particular component of this body, without thought, indeterminate, void, not a living being; it is air element in the mode of distending.

                        \vismParagraph{XI.83}{83}{}
                        As he gives his attention in this way the elements become evident to him. As he adverts and gives attention to them again and again access concentration arises in him in the way already described.
                    \subsubsection[\vismAlignedParas{§84}(3) With Characteristics in Brief]{(3) With Characteristics in Brief}

                        \vismParagraph{XI.84}{84}{}
                        But if his meditation subject is still not successful when he gives his attention to it in this way, then he should develop it \emph{with characteristics in brief}. How? In the \emph{twenty components }the characteristic of stiffenedness should be defined as the earth element, and the characteristic of cohesion, which is there too, as the water element, and the characteristic of maturing (ripening), which is there too, as the fire element, and the characteristic of distension, which is there too, as the air element. In the \emph{twelve components }the characteristic of cohesion should be defined as the water \marginnote{\textcolor{teal}{\footnotesize\{416|358\}}}{}element, the characteristic of maturing (ripening), which is there too, as the fire element, the characteristic of distension, which is there too, as the air element, and the characteristic of stiffenedness, which is there too, as the earth element. In the \emph{four components }the characteristic of maturing (ripening) should be defined as the fire element, the characteristic of distension unresolvable (inseparable) from it as the air element, \textcolor{brown}{\textit{[364]}} the characteristic of stiffenedness as the earth element, and the characteristic of cohesion as the water element. In the \emph{six components }the characteristic of distension should be defined as the air element, the characteristic of stiffenedness there too as the earth element, the characteristic of cohesion as the water element, and the characteristic of maturing (ripening) as the fire element.

                        As he defines them in this way the elements become evident to him. As he adverts to them and gives attention to them again and again access concentration arises in him in the way already stated.
                    \subsubsection[\vismAlignedParas{§85}(4) With Characteristics by Analysis]{(4) With Characteristics by Analysis}

                        \vismParagraph{XI.85}{85}{}
                        However, if he still does not succeed with his meditation subject when he gives his attention to it in this way, then he should develop it \emph{with characteristics by analysis}. How? After discerning head hairs, etc., in the way already described, the characteristic of stiffenedness in head hairs should be defined as the earth element, the characteristic of cohesion there too as the water element, the characteristic of maturing (ripening) as the fire element, and the characteristic of distension as the air element. The four elements should be defined in this way in the case of each component.

                        As he defines them in this way the elements become evident to him. As he adverts and gives attention to them again and again access concentration arises in him in the way already described.
                \subsection[\vismAlignedParas{§86–117}Additional Ways of Giving Attention]{Additional Ways of Giving Attention}

                    \vismParagraph{XI.86}{86}{}
                    In addition, attention should be given to the elements in the following ways: (1) as to word meaning, (2) by groups, (3) by particles, (4) by characteristic, etc., (5) as to how originated, (6) as to variety and unity, (7) as to resolution (separability) and non-resolution (inseparability), (8) as to the similar and the dissimilar, (9) as to distinction between internal and external, (10) as to inclusion, (11) as to condition, (12) as to lack of conscious reaction, (13) as to analysis of conditions.

                    \vismParagraph{XI.87}{87}{}
                    \emph{1.} Herein, one who gives his attention to them \emph{as to word meaning }should do so separately and generally thus: [separately] it is earth (\emph{pathavī}) because it is spread out (\emph{patthaṭa}); it flows (\emph{appoti}) or it glides (\emph{āpiyati}) or it satisfies (\emph{appāyati}), thus it is water (\emph{āpo}); it heats (\emph{tejati}), thus it is fire (\emph{tejo}); it blows (\emph{vāyati}), thus it is air (\emph{vāyo}). But without differentiation they are elements (\emph{dhātu}) because of bearing (\emph{dhāraṇa}) their own characteristics, because of grasping (\emph{ādāna}) suffering, and because of sorting out (\emph{ādhāna}) suffering (see \hyperlink{XV.19}{XV.19}{}).\footnote{\vismAssertFootnoteCounter{29}\vismHypertarget{XI.n29}{}\emph{“‘Because of bearing their own characteristics’}: these are not like the Primordial Essence \emph{(pakati—Skr. prakṛti) }and the self \emph{(attā) }imagined by the theorists which are non-existent as to individual essence. On the contrary, these do bear their own characteristics, which is why they are elements” (\textbf{\cite{Vism-mhṭ}359}). Capitals have been used here and elsewhere though Indian alphabets do not justify it. \emph{Appāyati—“}to satisfy” is not in PED; see \textbf{\cite{Vibh-a}9}.} This is how they should be given attention as to word meaning.

                    \vismParagraph{XI.88}{88}{}
                    \marginnote{\textcolor{teal}{\footnotesize\{417|359\}}}{}\emph{2. By groups}: there is the earth element described under the twenty aspects (modes) beginning with head hairs, body hairs, and also the water element described under the twelve (modes) aspects beginning with bile, phlegm, etc. Now, as to these:
                    \begin{verse}
                        Colour, odour, taste, and nutritive\\{}
                        Essence, and the four elements—\\{}
                        From combination of these eight\\{}
                        There comes the common usage head hairs;\\{}
                        And separately from these eight\footnote{\vismAssertFootnoteCounter{30}\vismHypertarget{XI.n30}{}\emph{“‘From resolution of these eight’}: the eight dhammas beginning with colour when resolved by means of understanding, are apprehendable \emph{(upalabbhanti) }in the ultimate sense through mutual negation \emph{(aññam-añña-vyatirekena); }but head hairs are not apprehendable in the ultimate sense through negation of colour and so on. Consequently, the term of common usage ‘head hairs’ is applied to these dhammas in their co-arisen state; but if they are each taken separately, \emph{‘There is no common-usage head hairs.’ }The meaning is that it is a mere conventional term. \emph{‘Only a mere group of eight states’ }is said, taking the colour, etc., which are real \emph{(bhūta—lit. }‘become’), as a unity by means of the concept \emph{(paññatti) }‘a head hair,’ not only because they are merely the eight states” (\textbf{\cite{Vism-mhṭ}360}).}\\{}
                        There is no common usage head hairs.
                    \end{verse}


                    Consequently, head hairs are only a mere group of eight states. Likewise, body hairs, \textcolor{brown}{\textit{[365]}} and the rest. A component here that is kamma-originated is a group of ten states, [that is to say, the former eight] together with the life faculty and sex. But it is on account of respective prominence [of stiffenedness or cohesion] that it comes to be styled “earth element” or “water element.” This is how they should be given attention to “by groups.”

                    \vismParagraph{XI.89}{89}{}
                    \emph{3. By particles}: in this body the earth element taken as reduced to fine dust and powdered to the size of the smallest atom\footnote{\vismAssertFootnoteCounter{31}\vismHypertarget{XI.n31}{}\emph{Paramaṇu—}“the smallest atom”; see \textbf{\cite{Vibh-a}343}. According to \textbf{\cite{Vibh-a}}, the size of a \emph{paramaṇu} works out at 1/581,147,136th part of an \emph{aṅgula} (fingerbreadth or inch). \textbf{\cite{Vism-mhṭ}} remarks (p. 361): “Therefore … a \emph{paramaṇu} as a particle of space is not the province of the physical eye, it is the province of the divine eye.”} might amount to an average \emph{doṇa }measure full; and that is held together\footnote{\vismAssertFootnoteCounter{32}\vismHypertarget{XI.n32}{}\emph{Saṅgahita—}“held together”: not quite in this sense in PED. “Held \emph{(gahita) }by conjoining through cohesion and prevented from being scattered” (\textbf{\cite{Vism-mhṭ}361}).} by the water element measuring half as much. Being maintained\footnote{\vismAssertFootnoteCounter{33}\vismHypertarget{XI.n33}{}“Kept guarded \emph{(anurakkhita) }so that it may not lapse into a wet and slippery state through the water element, which has trickling as its essence” (\textbf{\cite{Vism-mhṭ}361}).} by the fire element, and distended by the air element, it does not get scattered or dissipated. Instead of getting scattered or dissipated, it arrives at the alternative states of the female and male sex, etc., and manifests smallness, bigness, length, shortness, toughness, rigidity, and so on.

                    \vismParagraph{XI.90}{90}{}
                    \marginnote{\textcolor{teal}{\footnotesize\{418|360\}}}{}The liquid water element that is the mode of cohesion, being founded on earth, maintained by fire, and distended by air, does not trickle or run away.\footnote{\vismAssertFootnoteCounter{34}\vismHypertarget{XI.n34}{}\emph{Parissavati—}“to run away”: not in PED;—\emph{vissarati }(\textbf{\cite{Vism-mhṭ}361}).} Instead of trickling or running away it provides continued refreshments.\footnote{\vismAssertFootnoteCounter{35}\vismHypertarget{XI.n35}{}“This is said with reference to the water element as a juice that helps growth” (\textbf{\cite{Vism-mhṭ}361}).}

                    \vismParagraph{XI.91}{91}{}
                    And here the fire element that cooks what is eaten, drunk, etc., and is the mode of warming and has the characteristic of heat, being established on earth, held together by water, and distended by air, maintains this body and ensures its proper appearance. And this body, being maintained by it, shows no putrefaction.

                    \vismParagraph{XI.92}{92}{}
                    The air element that courses through all the limbs and has the characteristic of moving and distending, being founded upon earth, held together by water, and maintained by fire, distends this body. And this body, being distended by the latter kind of air, does not collapse, but stands erect, and being propelled\footnote{\vismAssertFootnoteCounter{36}\vismHypertarget{XI.n36}{}\emph{Samabbhāhata—}“propelled”: see \hyperlink{IV.n38}{Ch. IV, note 38}{}.} by the other [motile] air, it shows intimation and it flexes and extends and it wriggles the hands and feet, doing so in the postures comprising of walking, standing, sitting and lying down. So this mechanism of elements carries on like a magic trick, deceiving foolish people with the male and female sex and so on.

                    This is how they should be given attention by particles.

                    \vismParagraph{XI.93}{93}{}
                    \emph{4. As to characteristic, etc}.: he should advert to the four elements in this way: “The earth element—what are its characteristic, function, manifestation?” [defining them in this way]: The earth element has the characteristic of hardness. Its function is to act as a foundation. It is manifested as receiving. The water element has the characteristic of trickling. Its function is to intensify. It is manifested as holding together. The fire element has the characteristic of heat. Its function is to mature (maintain). It is manifested as a continued supply of softness. The air element has the characteristic of distending. Its function is to cause motion. It is manifested as conveying.\footnote{\vismAssertFootnoteCounter{37}\vismHypertarget{XI.n37}{}\emph{Abhinīhāra—}“conveying”: not in this sense in PED. “\emph{‘Conveying’ }is acting as cause for the successive arising at adjacent locations \emph{(desantaruppatti) }of the conglomeration of elements \emph{(bhūta-saṅghāta)” }(\textbf{\cite{Vism-mhṭ}363}). Elsewhere \textbf{\cite{Vism-mhṭ}(p. 359)} says of the air element: “‘\emph{It blows’ }(§87): it is stirred; the meaning is that the conglomeration of elements is made to move (go) by its action as cause for successive arising at adjacent locations (points),” and “Propelling \emph{(samabbhāhana) }is the act of causing the successive arising at adjacent locations of material groups \emph{(rūpa-kalāpa)” }(p. 362).} This is how they should be given attention to by characteristic, and so on. \textcolor{brown}{\textit{[366]}}

                    \vismParagraph{XI.94}{94}{}
                    \emph{5. As to how originated}: among the forty-two components beginning with head hairs shown in the detailed treatment of the earth element, etc., the four consisting of gorge, dung, pus, and urine are temperature-originated only; the four consisting of tears, sweat, spittle, and snot are temperature-originated and consciousness-originated only. The fire that cooks what is eaten, etc., is kamma-originated only; in-breath and out-breath are consciousness-originated only; all \marginnote{\textcolor{teal}{\footnotesize\{419|361\}}}{}the rest are of fourfold origination. This is how they should be given attention as to how originated.

                    \vismParagraph{XI.95}{95}{}
                    \emph{6. As to variety and unity}: there is variety in the specific characteristics, etc., of all the elements; for the characteristic, function, and manifestation of the earth element is one, and those of the water element, etc., are different. But there is unity in them as materiality, great primary, element, state (\emph{dhamma}), impermanence, etc., notwithstanding the fact that they are various according to [specific] characteristic, etc., and according to origination by kamma and so on.

                    \vismParagraph{XI.96}{96}{}
                    All these elements are “instances of materiality” (\emph{rūpāni}) because they do not exceed the characteristic of “being molested” (\emph{ruppana}). They are “great primaries” (\emph{mahābhūta}) by reason of “great manifestation,” and so on. “By reason of ‘great manifestation,’ and so on” means that these elements are called “great primaries” for the following reasons, namely: (a) manifestation of greatness; (b) likeness to great creatures; (c) great maintenance; (d) great alteration; and (e) because they are great and because they are entities.

                    \vismParagraph{XI.97}{97}{}
                    Herein, (a) \emph{manifestation of greatness}: they are manifested as great both in a continuity that is not clung to (acquired through kamma) and in a continuity that is clung to. For their manifestation of greatness in a continuity that is not clung to is given in the description of the recollection of the Buddha in the way beginning:
                    \begin{verse}
                        Two times a hundred thousand [leagues]\\{}
                        And then four \emph{nahutas} as well:\\{}
                        This earth, this “bearer of all wealth,”\\{}
                        Has that much thickness, as they tell (\hyperlink{VII.41}{VII.41}{}).
                    \end{verse}


                    And they are manifested on a great scale also in a continuity that is clung to, for instance, in the bodies of fishes, turtles, deities, Dānava demons, and so on. For this is said: “Bhikkhus, there are individual creatures of a hundred leagues in the great ocean” (\textbf{\cite{A}IV 207}), and so on.

                    \vismParagraph{XI.98}{98}{}
                    (b) \emph{Likeness to great creatures}: just as a magician turns water that is not crystal into crystal, and turns a clod that is not gold into gold, and shows them, and being himself neither a spirit nor a bird, shows himself as a spirit or a bird, so too, being themselves not blue-black, they turn themselves into blue-black derived materiality, being themselves not yellow … not red … not white, \textcolor{brown}{\textit{[367]}} they turn themselves into white derived materiality and show that. In this way they are “great primaries” (\emph{mahābhūta}) in being like the great creatures (\emph{mahābhūta}) of a magician.\footnote{\vismAssertFootnoteCounter{38}\vismHypertarget{XI.n38}{}“A great primary \emph{(mahābhūta) }is a great wonder \emph{(mahanto abbhuto) }because it shows various unreal things \emph{(abhūta), }various wonders \emph{(abbhuta), }and various marvels \emph{(acchariya). }Or alternatively: there are great wonders \emph{(abbhuta) }here, thus there are magicians. And spirits, etc., are huge \emph{(mahant) }creatures \emph{(bhūta) }owing to being born from them, thus they are great primaries. Or alternatively: this term ‘great primary’ can be regarded as a generic term for all of them. But earth, etc., are great primaries because they deceive, and because, like the huge creatures, their standing place cannot be pointed to. The deception lies in causing the apparent individual essences of blue-black, etc., and it lies in causing the appearance of what has the aspect of woman and man, and so on. Likewise their undemonstrability, since they are not found inside or outside each other though they rely upon each other for support. For if these elements were found inside each other, they would not each perform their particular functions, owing to mutual frustration. And if they were found outside each other, they would be already resolved (separate), and that being so, any description of them as unresolved (inseparable) would be meaningless. So although their standing place is undemonstrable, still each one assists the other by its particular function—the functions of establishing, etc., whereby each becomes a condition for the others as conascence condition and so on” (\textbf{\cite{Vism-mhṭ}363}).}

                    \vismParagraph{XI.99}{99}{}
                    \marginnote{\textcolor{teal}{\footnotesize\{420|362\}}}{}And just as, whomsoever the great creatures such as the spirits (\emph{yakkha}) grasp hold of (possess), they have no standing place either inside him or outside him and yet they have no standing independently of him, so too, these elements are not found to stand either inside or outside each other yet they have no standing independently of one another. Thus they are also great primaries (\emph{mahābhūta}) in being equal to the great creatures (\emph{mahābhūta}) such as the spirits because they have no thinkable standing place [relative to each other].

                    \vismParagraph{XI.100}{100}{}
                    And just as the great creatures known as female spirits (\emph{yakkhinī}) conceal their own fearfulness with a pleasing colour, shape and gesture to deceive beings, so too, these elements conceal each their own characteristic and function classed as hardness, etc., by means of a pleasing skin colour of women’s and men’s bodies, etc., and pleasing shapes of limbs and pleasing gestures of fingers, toes and eyebrows, and they deceive simple people by concealing their own functions and characteristics beginning with hardness and do not allow their individual essences to be seen. Thus they are great primaries (\emph{mahābhūta}) in being equal to the great creatures (\emph{mahābhūta}), the female spirits, since they are deceivers.

                    \vismParagraph{XI.101}{101}{}
                    (c) \emph{Great maintenance}: this is because they have to be sustained by the great requisites. For these elements are great primaries (\emph{mahābhūta}) since they have become (\emph{bhūta}), have occurred, by means of the food, clothing, etc., which are great (\emph{mahant}) [in importance] because they have to be found every day. Or alternatively, they are great primaries (\emph{mahābhūta}) since they are primaries whose maintenance is great.

                    \vismParagraph{XI.102}{102}{}
                    (d) \emph{Great alteration}: the unclung-to and the clung-to are the [basis of] great alterations. Herein, the great alteration of the unclung-to evidences itself in the emergence of an aeon (see \hyperlink{XIII.34}{XIII.34}{}), and that of the clung-to in the disturbance of the elements [in the body]. For accordingly:
                    \begin{verse}
                        The conflagration’s flame bursts up\\{}
                        Out of the ground and races higher\\{}
                        And higher, right to the Brahmā heaven,\\{}
                        When the world is burnt up by fire.
                    \end{verse}

                    \begin{verse}
                        A whole world system measuring\\{}
                        One hundred thousand millions wide\\{}
                        Subsides, as with its furious waters\\{}
                        The flood dissolves the world beside.
                    \end{verse}

                    \begin{verse}
                        \marginnote{\textcolor{teal}{\footnotesize\{421|363\}}}{}One hundred thousand million leagues,\\{}
                        A whole world system’s broad extent\\{}
                        Is rent and scattered, when the world\\{}
                        Succumbs to the air element.
                    \end{verse}

                    \begin{verse}
                        The bite of wooden-mouths can make\\{}
                        The body stiff; to all intent,\\{}
                        When roused is its earth element,\\{}
                        It might be gripped by such a snake.
                    \end{verse}

                    \begin{verse}
                        The bite of rotten-mouths can make\\{}
                        The body rot; to all intent,\\{}
                        When roused its water element,\\{}
                        It might be gripped by such a snake. \textcolor{brown}{\textit{[368]}}
                    \end{verse}

                    \begin{verse}
                        The bite of fiery-mouths can make\\{}
                        The body burn; to all intent,\\{}
                        When roused is its fire element,\\{}
                        It might be gripped by such a snake.
                    \end{verse}

                    \begin{verse}
                        The bite of dagger-mouths can make\\{}
                        The body burst; to all intent,\\{}
                        When roused is its air element,\\{}
                        It might be gripped by such a snake.
                    \end{verse}


                    So they are great primaries (\emph{mahābhūta}) because they have become (\emph{bhūta}) [the basis of] great (\emph{mahant}) alteration.

                    \vismParagraph{XI.103}{103}{}
                    (e) \emph{Because they are great and because they are entities}: “great” (\emph{mahant}) because they need great effort to discern them, and “entities” (\emph{bhūta }= become) because they are existent; thus they are great primaries (\emph{mahābhūta}) because they are great (\emph{mahā}) and because they are entities (\emph{bhūta}).

                    This is how all these elements are “great primaries” by reason of “great manifestation,” and so on.

                    \vismParagraph{XI.104}{104}{}
                    Again, they are elements (\emph{dhātu}) because of bearing (\emph{dhāraṇa}) their own characteristics, because of grasping (\emph{ādāna}) suffering, and because of sorting out (\emph{ādhāna}) suffering (see \hyperlink{XV.19}{XV.19}{}), and because none of them are exempt from the characteristic of being elements.

                    They are states (\emph{dhamma}) owing to bearing (\emph{dhāraṇa}) their own characteristics and owing to their so bearing (\emph{dhāraṇa}) for the length of the moment appropriate to them.\footnote{\vismAssertFootnoteCounter{39}\vismHypertarget{XI.n39}{}This alludes to the length of duration of a moment of matter’s existence, which is described as seventeen times as long as that of consciousness (see \textbf{\cite{Vibh-a}25f.}).} They are impermanent in the sense of [liability to] destruction; they are painful in the sense of [causing] terror; they are not self in the sense of having no core [of permanence, and so on]. Thus there is unity of all since all are materiality, great primaries, elements, states, impermanent, and so on.

                    This is how they should be given attention “as to variety and unity.”

                    \vismParagraph{XI.105}{105}{}
                    \emph{7. As to resolution }(\emph{separability}) \emph{and non-resolution }(\emph{inseparability}): they are positionally unresolvable (inseparable) since they always arise together in every \marginnote{\textcolor{teal}{\footnotesize\{422|364\}}}{}single minimal material group consisting of the bare octad and the others; but they are resolvable (separable) by characteristic. This is how they should be given attention “as to resolution (separability) and non-resolution (inseparability).”

                    \vismParagraph{XI.106}{106}{}
                    \emph{8. As to the similar and dissimilar}: and although they are unresolved (inseparable) in this way, yet the first two similar in heaviness, and so are the last two in lightness; but [for this reason] the first two are dissimilar to the last two and the last two to the first two. This is how they should be given attention “as to the similar and dissimilar.”

                    \vismParagraph{XI.107}{107}{}
                    \emph{9. As to distinction between internal and external}: the internal elements are the [material] support for the physical bases of consciousness, for the kinds of intimation, and for the material faculties. They are associated with postures, and they are of fourfold origination. The external elements are of the opposite kind. This is how they should be given attention “as to distinction between internal and external.”

                    \vismParagraph{XI.108}{108}{}
                    \emph{10. As to inclusion}: kamma-originated earth element is included together with the other kamma-originated elements because there is no difference in their origination. Likewise the consciousness-originated is included together with other consciousness-originated elements. This is how they should be given attention “as to inclusion.”

                    \vismParagraph{XI.109}{109}{}
                    \emph{11. As to condition}: the earth element, which is held together by water, maintained by fire and distended by air, is a condition for the other three great primaries by acting as their foundation. The water element, which is founded on earth, maintained by fire and distended by air, is a condition for the other three great primaries by acting as their cohesion. The fire element, which is founded on earth, held together by water \textcolor{brown}{\textit{[369]}} and distended by air, is a condition for the other three great primaries by acting as their maintaining. The air element, which is founded on earth, held together by water, and maintained by fire, is a condition for the other three great primaries by acting as their distension. This is how they should be given attention “as to condition.”

                    \vismParagraph{XI.110}{110}{}
                    \emph{12. As to lack of conscious reaction}: here too the earth element does not know, “I am the earth element” or “I am a condition by acting as a foundation for three great primaries.” And the other three do not know, “The earth element is a condition for us by acting as a foundation for three great primaries.” And the other three do not know, “The earth element is a condition for us by acting as our foundation.” And similarly in each instance. This is how they should be given attention “as to lack of conscious reaction.”

                    \vismParagraph{XI.111}{111}{}
                    \emph{13. As to analysis of conditions}: there are four conditions for the elements, that is to say, kamma, consciousness, nutriment, and temperature.

                    Herein, \emph{kamma} only is the condition for the kamma-originated [elements]; \emph{consciousness} (\emph{citta}), etc. [i.e. nutriment and temperature] are not. Consciousness, etc., only are the conditions for the consciousness-originated [elements]; the others are not.\emph{ Kamma} is the producing condition\footnote{\vismAssertFootnoteCounter{40}\vismHypertarget{XI.n40}{}“The term \emph{‘producing condition’ }refers to causing origination, though as a condition it is actually kamma condition. For this is said: ‘Profitable and unprofitable volition is a condition, as kamma condition, for resultant aggregates and for materiality due to kamma performed’ (\textbf{\cite{Paṭṭh}I 5})” (\textbf{\cite{Vism-mhṭ}368}).} for the kamma-originated \marginnote{\textcolor{teal}{\footnotesize\{423|365\}}}{}elements; for the rest it is indirectly the decisive-supportive condition.\footnote{\vismAssertFootnoteCounter{41}\vismHypertarget{XI.n41}{}“‘\emph{For the rest’: }for consciousness-originated, and so on. It is the indirectly decisive-support condition because in the Paṭṭhāna the decisive-support condition has only been given for immaterial dhammas, so there is, directly, no decisive-support condition [in kamma] for material dhammas. However, because of the words, ‘With a person as decisive support’ (\textbf{\cite{M}I 107}) and ‘With a grove as decisive support’ (\textbf{\cite{M}I 106}) in the Suttas, the decisive-support condition can be indirectly understood according to the Suttas in the sense of ‘absence without’” (\textbf{\cite{Vism-mhṭ}368}).} \emph{Consciousness} is the producing condition for the consciousness-originated elements; for the rest it is the post-nascence condition, presence condition and non-disappearance condition. \emph{Nutriment} is the producing condition for the nutriment-originated elements; for the rest it is the nutriment condition, presence condition and non-disappearance condition. \emph{Temperature} is the productive condition for the temperature-originated elements; for the rest it is the presence condition and non-disappearance condition.

                    Herein, the kamma-originated great primary is the condition for the kamma-originated great primaries, and also for the consciousness-originated [great primaries]. Likewise are the consciousness-originated [great primary] and the nutriment originated [great primary]. The temperature-originated great primary is the condition for the temperature-originated great primaries, and also for the kamma-originated [great primaries], and so on (cf. \hyperlink{XX.27}{XX.27f.}{}).

                    \vismParagraph{XI.112}{112}{}
                    Herein, the kamma-originated earth element is a condition for the other kamma-originated elements both as conascence, mutuality, support, presence, and non-disappearance conditions and as foundation, but not as producing condition. It is a condition for the other [three] great primaries in a triple continuity (see \hyperlink{XX.22}{XX.22}{}) as support, presence and non-disappearance conditions, but not as foundation or producing condition. And here the water element is a condition for the remaining three elements both as conascence, etc., conditions and as cohesion, but not as producing condition. And for the others in a triple continuity it is a condition as support, presence, and non-disappearance conditions too, but not as cohesion or producing condition. And the fire element here is a condition for the other three elements both as conascence, etc., conditions and as maintaining but not as producing condition. And for the others in a triple continuity it is a condition as support, presence, and non-disappearance conditions too, but not as maintaining or producing condition. And the air element here is a condition for the other three elements \textcolor{brown}{\textit{[370]}} both as conascence, etc., conditions and as distension, but not as producing condition. And for the others in a triple continuity it is a condition as support, presence, and non-disappearance conditions too, but not as distension or producing condition.

                    \vismParagraph{XI.113}{113}{}
                    The same method applies in the case of the consciousness-originated, the nutriment-originated, and the temperature-originated earth element, and the rest. And when these elements have been made to occur through the influence of the conascence, etc., conditions:
                    \begin{verse}
                        \marginnote{\textcolor{teal}{\footnotesize\{424|366\}}}{}With three in four ways to one due,\\{}
                        And likewise with one due to three;\\{}
                        With two in six ways due to two—\\{}
                        Thus their occurrence comes to be.
                    \end{verse}


                    \vismParagraph{XI.114}{114}{}
                    Taking each one, beginning with earth, there are three others whose occurrence is due to that one, thus with three due to one their occurrence takes place in four ways. Likewise each one, beginning with earth, occurs in dependence on the other three, thus with one due to three their occurrence takes place in four ways. But with the last two dependent on the first two, with the first two dependent on the last two, with the second and fourth dependent on the first and third, with the first and third dependent on the second and fourth, with the second and third dependent on the first and fourth, and with the first and fourth dependent on the second and third, they occur in six ways with two elements due to two.

                    \vismParagraph{XI.115}{115}{}
                    At the time of moving forward and moving backward (\textbf{\cite{M}I 57}), the earth element among these is a condition for pressing. That, seconded by the water element, is a condition for establishing on a foundation. But the water element seconded by the earth element is a condition for lowering down. The fire element seconded by the air element is a condition for lifting up. The air element seconded by the fire element is a condition for shifting forwards and shifting sideways (see \hyperlink{XX.62}{XX.62f.}{} and \textbf{\cite{M-a}I 160}).

                    This is how they should be given attention “as to analysis of conditions.”

                    \vismParagraph{XI.116}{116}{}
                    As he gives his attention to them “as to word meaning,” etc., in this way, the elements become evident to him under each heading. As he again and again adverts and gives attention to them access concentration arises in the way already described. And this concentration too is called “definition of the four elements” because it arises in one who defines the four elements owing to the influence of his knowledge.

                    \vismParagraph{XI.117}{117}{}
                    This bhikkhu who is devoted to the defining of the four elements immerses himself in voidness and eliminates the perception of living beings. Since he does not entertain false notions about wild beasts, spirits, ogres, etc., because he has abolished the perception of living beings, he conquers fear and dread and conquers delight and aversion (boredom); he is not exhilarated or depressed\footnote{\vismAssertFootnoteCounter{42}\vismHypertarget{XI.n42}{}\emph{Ugghāta—}“exhilarated” and \emph{nigghāta—}“depressed”: neither word is in PED; Vism-mhṭ glosses with \emph{ubbilāvitatta }and \emph{dīnabhāvappatti }respectively.} by agreeable and disagreeable things; and as one of great understanding, he either ends in the deathless or he is bound for a happy destiny.
                    \begin{verse}
                        Defining the four elements\\{}
                        Is ever the wise man’s resort;\\{}
                        The noble meditator lion\footnote{\vismAssertFootnoteCounter{43}\vismHypertarget{XI.n43}{}Reading \emph{yogivarasīhassa kīlitaṃ.} Cf. \emph{Nettippakaraṇa }“Sīha-kīlana.”}\\{}
                        Will make this mighty theme his sport.
                    \end{verse}


                    This is the description of the development of the defining of the four elements. \textcolor{brown}{\textit{[371]}}
            \section[\vismAlignedParas{§118–126}Development of Concentration—Conclusion]{Development of Concentration—Conclusion}

                \vismParagraph{XI.118}{118}{}
                \marginnote{\textcolor{teal}{\footnotesize\{425|367\}}}{}This completes in all its aspects the commentary on the meaning of the clause, “How should it be developed?” in the set of questions beginning with “What is concentration?” which was formulated in order to show the method of development of concentration in detail (see \hyperlink{III.1}{III.1}{}).

                \vismParagraph{XI.119}{119}{}
                This concentration as intended here is twofold, that is to say, access concentration and absorption concentration. Herein, the unification [of mind] in the case of ten meditation subjects and in the consciousness preceding absorption [in the case of the remaining meditation subjects]\footnote{\vismAssertFootnoteCounter{44}\vismHypertarget{XI.n44}{}The sense demands reading with \textbf{\cite{Vism-mhṭ}} \emph{appanāpubba-bhāgacittesu }as a single compound.} is access concentration. The unification of mind in the case of the remaining meditation subjects is absorption concentration. And so it is developed in two forms with the development of these meditation subjects. Hence it was said above: “This completes in all its aspects the commentary on the meaning of the clause, ‘How should it be developed?’”
                \subsection[\vismAlignedParas{§120–126}The Benefits of Developing Concentration]{The Benefits of Developing Concentration}

                    \vismParagraph{XI.120}{120}{}
                    The question, (viii) \emph{what are the benefits of the development of concentration? }was also asked, however (\hyperlink{III.1}{III.1}{}). Herein, the benefits of the development of concentration are fivefold, as a blissful abiding here and now, and so on. For the development of absorption concentration provides the benefit of a blissful abiding here and now for the Arahants with cankers destroyed who develop concentration, thinking, “We shall attain and dwell with unified mind for a whole day.” Hence the Blessed One said: “But, Cunda, it is not these that are called effacement in the Noble Ones’ discipline; these are called blissful abidings in the Noble Ones’ discipline” (\textbf{\cite{M}I 40}).

                    \vismParagraph{XI.121}{121}{}
                    When ordinary people and trainers develop it, thinking, “After emerging, we shall exercise insight with concentrated consciousness,” the development of absorption concentration provides them with the benefit of insight by serving as the proximate cause for insight, and so too does access concentration as a method of arriving at wide open [conditions] in crowded [circumstances].\footnote{\vismAssertFootnoteCounter{45}\vismHypertarget{XI.n45}{}This is an allusion to \textbf{\cite{M}I 179}, etc. “The process of existence in the round of rebirths, which is a very cramped place, is crowded by the defilements of craving and so on” (\textbf{\cite{Vism-mhṭ}371}).} Hence the Blessed One said: “Bhikkhus, develop concentration; a bhikkhu who is concentrated understands correctly” (\textbf{\cite{S}III 13}).

                    \vismParagraph{XI.122}{122}{}
                    But when they have already produced the eight attainments and then, aspiring to the kinds of direct-knowledge described in the way beginning, “Having been one, he becomes many” (\hyperlink{XII.2}{XII.2}{}), they produce them by entering upon jhāna as the basis for direct-knowledge and emerging from it, then the development of absorption concentration provides for them the benefit of the kinds of direct-knowledge, since it becomes the proximate cause for the kinds of direct-knowledge whenever there is an occasion. Hence the Blessed One said: \marginnote{\textcolor{teal}{\footnotesize\{426|368\}}}{}“He attains the ability to be a witness, through realization by direct-knowledge, of any state realizable by direct-knowledge to which his mind inclines, whenever there is an occasion” (\textbf{\cite{M}III 96}; \textbf{\cite{A}I 254}). \textcolor{brown}{\textit{[372]}}

                    \vismParagraph{XI.123}{123}{}
                    When ordinary people have not lost their jhāna, and they aspire to rebirth in the Brahmā-world thus, “Let us be reborn in the Brahmā-world,” or even though they do not make the actual aspiration, then the development of absorption concentration provides them with the benefits of an improved form of existence since it ensures that for them. Hence the Blessed One said: “Where do they reappear after developing the first jhāna limitedly? They reappear in the company of the deities of Brahmā’s Retinue” (\textbf{\cite{Vibh}424}), and so on. And even the development of access concentration ensures an improved form of existence in the happy destinies of the sensual sphere.

                    \vismParagraph{XI.124}{124}{}
                    But when Noble Ones who have already produced the eight attainments develop concentration, thinking, “We shall enter upon the attainment of cessation, and by being without consciousness for seven days we shall abide in bliss here and now by reaching the cessation that is Nibbāna,” then the development of absorption concentration provides for them the benefit of cessation. Hence it is said: “Understanding as mastery owing to … sixteen kinds of behaviour of knowledge, and to nine kinds of behaviour of concentration, is knowledge of the attainment of cessation” (\textbf{\cite{Paṭis}I 97}; see \hyperlink{XXIII.18}{Ch. XXIII, 18f.}{}).

                    \vismParagraph{XI.125}{125}{}
                    That is how this benefit of the development of concentration is fivefold as a blissful abiding here and now, and so on.
                    \begin{verse}
                        So wise men fail not in devotion\\{}
                        To the pursuit of concentration:\\{}
                        It cleans defiling stains’ pollution,\footnote{\vismAssertFootnoteCounter{46}\vismHypertarget{XI.n46}{}\emph{Sūdana—}“cleaning”: not in PED. See title of Majjhima Nikāya Commentary. Another reading here is \emph{sodhana.}}\\{}
                        And brings rewards past calculation.
                    \end{verse}


                    \vismParagraph{XI.126}{126}{}
                    And at this point in the \emph{Path of Purification}, which is taught under the headings of virtue, concentration and understanding in the stanza, “When a wise man, established well in virtue …,” concentration has been fully explained.

                    The eleventh chapter concluding “The Description of Concentration” in the \emph{Path of Purification} composed for the purpose of gladdening good people.
        \chapter[The Supernormal Powers]{The Supernormal Powers\vismHypertarget{XII}\newline{\textnormal{\emph{Iddhividha-niddesa}}}}
            \label{XII}

            \section[\vismAlignedParas{§1}The Benefits of Concentration (Continued)]{The Benefits of Concentration (Continued)}

                \vismParagraph{XII.1}{1}{}
                \marginnote{\textcolor{teal}{\footnotesize\{427|369\}}}{}\textcolor{brown}{\textit{[373]}} It was said above with reference to the mundane kinds of direct-knowledge that this development of concentration “provides … the benefit of the kinds of direct-knowledge” (\hyperlink{XI.122}{XI.122}{}). Now, in order to perfect those kinds of direct-knowledge the task must be undertaken by a meditator who has reached the fourth jhāna in the earth kasiṇa, and so on. And in doing this, not only will this development of concentration have provided benefits in this way, it will also have become more advanced; and when he thus possesses concentration so developed as to have both provided benefits and become more advanced, he will then more easily perfect the development of understanding. So meanwhile we shall deal with the explanation of the kinds of direct-knowledge now.
            \section[\vismAlignedParas{§2}The five kinds of direct-knowledge]{The five kinds of direct-knowledge}

                \vismParagraph{XII.2}{2}{}
                In order to show the benefits of developing concentration to clansmen whose concentration has reached the fourth jhāna, and in order to teach progressively refined Dhamma, five kinds of mundane direct-knowledge have been described by the Blessed One. They are: (1) the kinds of supernormal power, described in the way beginning, “When his concentrated mind is thus purified, bright, unblemished, rid of defilement, and has become malleable, wieldy, steady, and attained to imperturbability,\footnote{\vismAssertFootnoteCounter{1}\vismHypertarget{XII.n1}{}\emph{Āneñja—}“imperturbability”: a term normally used for the four immaterial states, together with the fourth jhāna. See also §16f., and MN 106.} he directs, he inclines, his mind to the kinds of supernormal power. He wields the various kinds of supernormal power. Having been one, he becomes many …” (\textbf{\cite{D}I 77}); (2) the knowledge of the divine ear element; (3) the knowledge of penetration of minds; (4) the knowledge of recollection of past lives; and (5) the knowledge of the passing away and reappearance of beings.
            \section[\vismAlignedParas{§2–139}(1) The Kinds of Supernormal Power]{(1) The Kinds of Supernormal Power}

                If a meditator wants to begin performing the transformation by supernormal power described as, “Having been one, he becomes many,” etc., he must achieve the eight attainments in each of the eight kasiṇas ending with the white kasiṇa. He must also have complete control of his mind in the following fourteen ways: \textcolor{brown}{\textit{[374]}} (i) in the order of the kasiṇa, (ii) in the reverse order of the kasiṇa, (iii) in the \marginnote{\textcolor{teal}{\footnotesize\{428|370\}}}{}order and reverse order of the kasiṇa, (iv) in the order of the jhāna, (v) in the reverse order of the jhāna (vi) in the order and reverse order of the jhāna, (vii) skipping jhāna, (viii) skipping kasiṇa, (ix) skipping jhāna and kasiṇa, (x) transposition of factors, (xi) transposition of object, (xii) transposition of factors and object, (xiii) definition of factors, and (xiv) definition of object.

                \vismParagraph{XII.3}{3}{}
                But what is “in the order of the kasiṇa” here? What is “definition of object”?

                (i) Here a bhikkhu attains jhāna in the earth kasiṇa, after that in the water kasiṇa, and so progressing through the eight kasiṇas, doing so even a hundred times, even a thousand times, in each one. This is called \emph{in the order of the kasiṇas}. (ii) Attaining them in like manner in reverse order, starting with the white kasiṇa, is called \emph{in the reverse order of the kasiṇas}. (iii) Attaining them again and again in forward and reverse order, from the earth kasiṇa up to the white kasiṇa and from the white kasiṇa back to the earth kasiṇa, is called \emph{in the order and reverse order of the kasiṇas}.

                \vismParagraph{XII.4}{4}{}
                (iv) Attaining again and again from the first jhāna up to the base consisting of neither perception nor non-perception is called \emph{in the order of the jhānas.} (v) Attaining again and again from the base consisting of neither perception nor non-perception back to the first jhāna is called \emph{in the reverse order of the jhānas}. (vi) Attaining in forward and reverse order, from the first jhāna up to the base consisting of neither perception nor non-perception and from the base consisting of neither perception nor non-perception back to the first jhāna, is called \emph{in the order and reverse order of the jhānas}.

                \vismParagraph{XII.5}{5}{}
                (vii) He skips alternate jhānas without skipping the kasiṇas in the following way: having first attained the first jhāna in the earth kasiṇa, he attains the third jhāna in that same kasiṇa, and after that, having removed [the kasiṇa (\hyperlink{X.6}{X.6}{}), he attains] the base consisting of boundless space, after that the base consisting of nothingness. This is called \emph{skipping jhānas}. And that based on the water kasiṇa, etc., should be construed similarly. (viii) When he skips alternate kasiṇas without skipping jhānas in the following way: having attained the first jhāna in the earth kasiṇa, he again attains that same jhāna in the fire kasiṇa and then in the blue kasiṇa and then in the red kasiṇa, this is called \emph{skipping kasiṇas}. (ix) When he skips both jhānas and kasiṇas in the following way: having attained the first jhāna in the earth kasiṇa, he next attains the third in the fire kasiṇa, next the base consisting of boundless space after removing the blue kasiṇa, next the base consisting of nothingness [arrived at] from the red kasiṇa, this is called \emph{skipping jhānas and kasiṇas}.

                \vismParagraph{XII.6}{6}{}
                (x) Attaining the first jhāna in the earth kasiṇa \textcolor{brown}{\textit{[375]}} and then attaining the others in that same kasiṇa is called \emph{transposition of factors}. (xi) Attaining the first jhāna in the earth kasiṇa and then that same jhāna in the water kasiṇa … in the white kasiṇa is called \emph{transposition of object}. (xii) Transposition of object and factors together takes place in the following way: he attains the first jhāna in the earth kasiṇa, the second jhāna in the water kasiṇa, the third in the fire kasiṇa, the fourth in the air kasiṇa, the base consisting of boundless space by removing the blue kasiṇa, the base consisting of boundless consciousness [arrived at] from the yellow kasiṇa, the base consisting of nothingness from the red kasiṇa, \marginnote{\textcolor{teal}{\footnotesize\{429|371\}}}{}and the base consisting of neither perception nor non-perception from the white kasiṇa. This is called \emph{transposition of factors and object}.

                \vismParagraph{XII.7}{7}{}
                (xiii) The defining of only the jhāna factors by defining the first jhāna as five-factored, the second as three-factored, the third as two-factored, and likewise the fourth, the base consisting of boundless space, … and the base consisting of neither perception nor non-perception, is called \emph{definition of factors}. (xiv) Likewise, the defining of only the object as “This is the earth kasiṇa,” “This is the water kasiṇa” … “This is the white kasiṇa,” is called \emph{definition of object}. Some would also have “defining of factors and object”; but since that is not given in the commentaries it is certainly not a heading in the development.

                \vismParagraph{XII.8}{8}{}
                It is not possible for a meditator to begin to accomplish transformation by supernormal powers unless he has previously completed his development by controlling his mind in these fourteen ways. Now, the kasiṇa preliminary work is difficult for a beginner and only one in a hundred or a thousand can do it. The arousing of the sign is difficult for one who has done the preliminary work and only one in a hundred or a thousand can do it. To extend the sign when it has arisen and to reach absorption is difficult and only one in a hundred or a thousand can do it. To tame one’s mind in the fourteen ways after reaching absorption is difficult and only one in a hundred or a thousand can do it. The transformation by supernormal power after training one’s mind in the fourteen ways is difficult and only one in a hundred or a thousand can do it. Rapid response after attaining transformation is difficult and only one in a hundred or a thousand can do it.

                \vismParagraph{XII.9}{9}{}
                Like the Elder Rakkhita who, eight years after his full admission to the Order, was in the midst of thirty thousand bhikkhus possessing supernormal powers who had come to attend upon the sickness of the Elder Mahā-Rohaṇa-Gutta at Therambatthala. \textcolor{brown}{\textit{[376]}} His feat is mentioned under the earth kasiṇa (\hyperlink{IV.135}{IV.135}{}). Seeing his feat, an elder said, “Friends, if Rakkhita had not been there, we should have been put to shame. [It could have been said], ‘They were unable to protect the royal nāga.’ So we ourselves ought to go about [with our abilities perfected], just as it is proper (for soldiers) to go about with weapons cleaned of stains.” The thirty thousand bhikkhus heeded the elder’s advice and achieved rapid response.

                \vismParagraph{XII.10}{10}{}
                And helping another after acquiring rapidity in responding is difficult and only one in a hundred or a thousand can do it. Like the elder who gave protection against the rain of embers by creating earth in the sky, when the rain of embers was produced by Māra at the \emph{Giribhaṇḍavahana} offering. \footnote{\vismAssertFootnoteCounter{2}\vismHypertarget{XII.n2}{}\emph{Giribhaṇḍavahanapūjā}: \textbf{\cite{Vism-mhṭ}(p. 375)} says: “\emph{Giribhaṇḍa-vahanapūjā nāma Cetiyagiriṃ ādiṃ katvā sakaladīpe samudde ca yāva yojanā mahatī dīpapūjā }(‘it is a name for a great island-offering starting with the Cetiyagiri (Mihintale) and extending over the whole island and up to a league into the sea ’).” Mentioned in A-a to AN 1:1; \textbf{\cite{M-a}II 398}; and \emph{Mahāvaṃsa }\hyperlink{XXXIV.81}{XXXIV.81}{}.}

                \vismParagraph{XII.11}{11}{}
                It is only in Buddhas, Paccekabuddhas, chief disciples, etc., who have vast previous endeavour behind them, that this transformation by supernormal power \marginnote{\textcolor{teal}{\footnotesize\{430|372\}}}{}and other such special qualities as the discriminations are brought to success simply with the attainment of Arahantship and without the progressive course of development of the kind just described.

                \vismParagraph{XII.12}{12}{}
                So just as when a goldsmith wants to make some kind of ornament, he does so only after making the gold malleable and wieldy by smelting it, etc., and just as when a potter wants to make some kind of vessel, he does so only after making the clay well kneaded and malleable, a beginner too must likewise prepare for the kinds of supernormal powers by controlling his mind in these fourteen ways; and he must do so also by making his mind malleable and wieldy both by attaining under the headings of zeal, consciousness, energy, and inquiry,\footnote{\vismAssertFootnoteCounter{3}\vismHypertarget{XII.n3}{}These are the four headings of the roads to power (see §50).} and by mastery in adverting, and so on. But one who already has the required condition for it owing to practice in previous lives needs only prepare himself by acquiring mastery in the fourth jhāna in the kasiṇas.

                \vismParagraph{XII.13}{13}{}
                Now, the Blessed One showed how the preparation should be done in saying, “When his concentrated mind,” and so on. Here is the explanation, which follows the text (see \hyperlink{XII.2}{§2}{}). Herein, \emph{he} is a meditator who has attained the fourth jhāna. \emph{Thus} signifies the order in which the fourth jhāna comes; having obtained the fourth jhāna in this order beginning with attaining the first jhāna, is what is meant. \emph{Concentrated}: concentrated by means of the fourth jhāna. \emph{Mind}: fine-material-sphere consciousness.

                \vismParagraph{XII.14}{14}{}
                But as to the words “purified,” etc., it is \emph{purified} by means of the state of mindfulness purified by equanimity. \textcolor{brown}{\textit{[377]}} It is \emph{bright} precisely because it is \emph{purified}; it is limpid (see \textbf{\cite{A}I 10}), is what is meant. It is \emph{unblemished} since the blemishes consisting of greed, etc., are eliminated by the removal of their conditions consisting of bliss, and the rest. It is \emph{rid of defilement} precisely because it is unblemished; for it is by the blemish that the consciousness becomes defiled. It has \emph{become malleable} because it is well developed; it suffers mastery, is what is meant, for consciousness that suffers mastery is called “malleable.” It is \emph{wieldy }(\emph{kammanīya}) precisely because it is malleable; it suffers being worked (\emph{kammakkhama}), is fit to be worked (\emph{kammayogga}), is what is meant.

                \vismParagraph{XII.15}{15}{}
                For a malleable consciousness is wieldy, like well-smelted gold; and it is both of these because it is well developed, according as it is said: “Bhikkhus, I do not see anyone thing that, when developed and cultivated, becomes so malleable and wieldy as does the mind” (\textbf{\cite{A}I 9}).

                \vismParagraph{XII.16}{16}{}
                It is \emph{steady} because it is steadied in this purifiedness, and the rest. It is \emph{attained to imperturbability} (\emph{āneñjappatta}) precisely because it is steady; it is motionless, without perturbation (\emph{niriñjana}), is what is meant. Or alternatively, it is steady because steady in its own masterability through malleability and wieldiness, and it is \emph{attained to imperturbability} because it is reinforced by faith, and so on.

                \vismParagraph{XII.17}{17}{}
                For consciousness reinforced by faith is not perturbed by faithlessness; when reinforced by energy, it is not perturbed by idleness; when reinforced by mindfulness, it is not perturbed by negligence; when reinforced by concentration, \marginnote{\textcolor{teal}{\footnotesize\{431|373\}}}{}it is not perturbed by agitation; when reinforced by understanding, it is not perturbed by ignorance; and when illuminated, it is not perturbed by the darkness of defilement. So when it is reinforced by these six states, it is attained to imperturbability.

                \vismParagraph{XII.18}{18}{}
                Consciousness possessing these eight factors in this way is susceptible of being directed to the realization by direct-knowledge of states realizable by direct-knowledge.

                \vismParagraph{XII.19}{19}{}
                Another method: It is \emph{concentrated} by means of fourth-jhāna concentration. It is \emph{purified} by separation from the hindrances. It is \emph{bright} owing to the surmounting of applied thought and the rest. It is \emph{unblemished} owing to absence of evil wishes based on the obtainment of jhāna.\footnote{\vismAssertFootnoteCounter{4}\vismHypertarget{XII.n4}{}I.e. one wants it to be known that he can practice jhāna.}It is \emph{rid of defilement} owing to the disappearance of the defilements of the mind consisting in covetousness, etc.; and both of these should be understood according to the Anaṅgaṇa Sutta (MN 5) and the Vattha Sutta (MN 7). It is \emph{become malleable} by masterability. It is \emph{wieldy} by reaching the state of a road to power (\hyperlink{XII.50}{§50}{}). It is \emph{steady and attained to imperturbability} by reaching the refinement of completed development; the meaning is that according as it has attained imperturbability so it is steady. And the consciousness possessing these eight factors in this way \textcolor{brown}{\textit{[378]}} is susceptible of being directed to the realization by direct-knowledge of states realizable by direct-knowledge, since it is the basis, the proximate cause, for them.

                \vismParagraph{XII.20}{20}{}
                \emph{He directs, he inclines, his mind to the kinds of supernormal powers} (\emph{iddhi-vidha—}lit. “kinds of success”): here “success” (\emph{iddhi}) is the success of succeeding (\emph{ijjhana}); in the sense of production, in the sense of obtainment, is what is meant. For what is produced and obtained is called “successful,” according as it is said, “When a mortal desires, if his desire is fulfilled” (\emph{samijjhati}) (\textbf{\cite{Sn}766}), and likewise: “Renunciation succeeds (\emph{ijjhati}), thus it is a success (\emph{iddhi}) … It metamorphoses (\emph{paṭiharati}) [lust], thus it is a metamorphosis (\emph{pāṭihāriya}) \footnote{\vismAssertFootnoteCounter{5}\vismHypertarget{XII.n5}{}“It counter-strikes (\emph{paṭiharati)}, thus it is a counter-stroke (\emph{pāṭihāriya—}metamorphosis = miracle). What strikes out (\emph{harati)}, removes, what is counter to it (\emph{paṭipakkha)} is therefore called counter-striking (\emph{paṭihāriya)}, since what is counter-striking strikes out anything counter (\emph{paṭipakkha)} to itself. \emph{Paṭihāriya }(counter-striking) is the same as \emph{pāṭihāriya }(counter-stroke = metamorphosis = miracle)” (\textbf{\cite{Vism-mhṭ}379}).} … The Arahant path succeeds, thus it is a success … It metamorphoses [all defilements], thus it is a metamorphosis” (\textbf{\cite{Paṭis}II 229}).

                \vismParagraph{XII.21}{21}{}
                Another method: success is in the sense of succeeding. That is a term for the effectiveness of the means; for effectiveness of the means succeeds with the production of the result intended, according as it is said: “This householder Citta is virtuous and magnanimous. If he should aspire, ‘Let me in the future become a Wheel-turning Monarch,’ being virtuous, he will succeed in his aspiration, because it is purified” (\textbf{\cite{S}IV 303}).

                \vismParagraph{XII.22}{22}{}
                Another method: beings succeed by its means, thus it is success. They succeed, thus they are successful; they are enriched, promoted, is what is meant. \marginnote{\textcolor{teal}{\footnotesize\{432|374\}}}{}That [success (power)] is of ten kinds, according as it is said, “Kinds of success: ten kinds of success,” after which it is said further, “What ten kinds of success? Success by resolve, success as transformation, success as the mind-made [body], success by intervention of knowledge, success by intervention of concentration, Noble Ones’ success, success born of kamma result, success of the meritorious, success through the sciences, success in the sense of succeeding due to right exertion applied here or there” (\textbf{\cite{Paṭis}II 205}).

                \vismParagraph{XII.23}{23}{}
                (i) Herein, the success shown in the exposition [of the above summary] thus, “Normally one, he adverts to [himself as] many or a hundred or a thousand or a hundred thousand; having adverted, he resolves with knowledge, “Let me be many” (\textbf{\cite{Paṭis}II 207}), is called \emph{success by resolve} because it is produced by resolving.

                \vismParagraph{XII.24}{24}{}
                (ii) That given as follows, “Having abandoned his normal form, he shows [himself in] the form of a boy or the form of a serpent … or he shows a manifold military array” (\textbf{\cite{Paṭis}II 210}), is called \emph{success as transformation} because of the abandoning and alteration of the normal form. \textcolor{brown}{\textit{[379]}}

                \vismParagraph{XII.25}{25}{}
                (iii) That given in this way, “Here a bhikkhu creates out of this body another body possessing visible form, mind-made” (\textbf{\cite{Paṭis}II 210}), is called \emph{success as the mind-made} (body) because it occurs as the production of another, mind-made, body inside the body.

                \vismParagraph{XII.26}{26}{}
                (iv) A distinction brought about by the influence of knowledge either before the arising of the knowledge or after it or at that moment is called \emph{success by intervention of knowledge}; for this is said: “The meaning (purpose) as abandoning perception of permanence succeeds through contemplation of impermanence, thus it is success by intervention of knowledge … The meaning (purpose) as abandoning all defilements succeeds through the Arahant path, thus it is success by intervention of knowledge. There was success by intervention of knowledge in the venerable Bakkula. There was success by intervention of knowledge in the venerable Saṅkicca. There was success by intervention of knowledge in the venerable Bhūtapāla” (\textbf{\cite{Paṭis}II 211}).

                \vismParagraph{XII.27}{27}{}
                Herein, when the venerable Bakkula as an infant was being bathed in the river on an auspicious day, he fell into the stream through the negligence of his nurse. A fish swallowed him and eventually came to the bathing place at Benares. There it was caught by a fisherman and sold to a rich man’s wife. The fish interested her, and thinking to cook it herself, she slit it open. When she did so, she saw the child like a golden image in the fish’s stomach. She was overjoyed, thinking, “At last I have got a son.” So the venerable Bakkula’s safe survival in a fish’s stomach in his last existence is called “success by intervention of knowledge” because it was brought about by the influence of the Arahant-path knowledge due to be obtained by [him in] that life. But the story should be told in detail (see \textbf{\cite{M-a}IV 190}).

                \vismParagraph{XII.28}{28}{}
                The Elder Saṅkicca’s mother died while he was still in her womb. At the time of her cremation she was pierced by stakes and placed on a pyre. The infant received a wound on the corner of his eye from the point of a stake and made a sound. Then, thinking that the child must be alive, they took down the body and \marginnote{\textcolor{teal}{\footnotesize\{433|375\}}}{}opened its belly. They gave the child to the grandmother. Under her care he grew up, and eventually he went forth and reached Arahantship together with the discriminations. So the venerable Saṅkicca’s safe survival on the pyre is called, “success by intervention of knowledge” in the way just stated (see \textbf{\cite{Dhp-a}II 240}).

                \vismParagraph{XII.29}{29}{}
                The boy Bhūtapāla’s father was a poor man in Rājagaha. \textcolor{brown}{\textit{[380]}} He went into the forest with a cart to get a load of wood. It was evening when he returned to the city gate. Then his oxen slipped the yoke and escaped into the city. He seated the child beside the cart and went into the city after the oxen. Before he could come out again, the gate was closed. The child’s safe survival through the three watches of the night outside the city in a place infested by wild beasts and spirits is called, “success by intervention of knowledge” in the way just stated. But the story should be told in detail.

                \vismParagraph{XII.30}{30}{}
                (v) A distinction brought about by the influence of serenity either before the concentration or after it or at that moment is called \emph{success by intervention of concentration} for this is said: “The meaning (purpose) as abandoning the hindrances succeeds by means of the first jhāna, thus it is success by intervention of concentration … The meaning (purpose) as abandoning the base consisting of nothingness succeeds by means of the attainment of the base consisting of neither perception nor non-perception, thus it is success by intervention of concentration. There was success by intervention of concentration in the venerable Sāriputta … in the venerable Sañjīva … in the venerable Khāṇu-Kondañña … in the laywoman devotee Uttarā … in the lay-woman devotee Sāmāvatī” (\textbf{\cite{Paṭis}II 211–212}).

                \vismParagraph{XII.31}{31}{}
                Herein, while the venerable Sāriputta was living with the Elder Mahā Moggallāna at Kapotakandarā he was sitting in the open on a moonlit night with his hair newly cut. Then a wicked spirit, though warned by his companion, gave him a blow on the head, the noise of which was like a thunder clap. At the time the blow was given the elder was absorbed in an attainment; consequently he suffered no harm from the blow. This was success by intervention of concentration in that venerable one. The story is given in the Udāna too (\textbf{\cite{Ud}39}).

                \vismParagraph{XII.32}{32}{}
                While the Elder Sañjīva was in the attainment of cessation, cowherds, etc., who noticed him thought he was dead. They brought grass and sticks and cow-dung and set fire to them. Not even a corner of the elder’s robe was burnt. This was success by intervention of concentration in him because it was brought about by the influence of the serenity occurring in his successive attainment [of each of the eight jhānas preceding cessation]. But the story is given in the Suttas too (\textbf{\cite{M}I 333}).

                \vismParagraph{XII.33}{33}{}
                The Elder Khāṇu Kondañña was naturally gifted in attainments. He was sitting absorbed in attainment one night in a certain forest. \textcolor{brown}{\textit{[381]}} Five hundred robbers came by with stolen booty. Thinking that no one was following them and needing rest, they put the booty down. Believing the elder was a tree stump (\emph{khāṇuka}), they piled all the booty on him. The elder emerged at the predetermined time just as they were about to depart after resting, at the very time in fact when the one who had put his booty down first was picking it up. When they saw the elder move, they cried out in fear. The elder said, “Do not be afraid, lay followers; \marginnote{\textcolor{teal}{\footnotesize\{434|376\}}}{}I am a bhikkhu.” They came and paid homage. Such was their confidence in the elder that they went forth into homelessness, and they eventually reached Arahantship together with the discriminations. The absence here of harm to the elder, covered as he was by five hundred bundles of goods, was success by intervention of concentration (see \textbf{\cite{Dhp-a}II 254}).

                \vismParagraph{XII.34}{34}{}
                The laywoman devotee Uttarā was the daughter of a rich man called Puṇṇaka. A harlot called Sirimā who was envious of her, poured a basin of hot oil over her head. At that moment Uttarā had attained [jhāna in], loving-kindness. The oil ran off her like water on a lotus leaf. This was success by intervention of concentration in her. But the story should be given in detail (see \textbf{\cite{Dhp-a}III 310}; A-a I 451).

                \vismParagraph{XII.35}{35}{}
                King Udena’s chief queen was called Sāmāvatī. The brahman Māgaṇḍiya, who aspired to elevate his own daughter to the position of chief queen, put a poisonous snake into Sāmāvatī’s lute. Then he told the king, “Sāmāvatī wants to kill you, sire. She is carrying a poisonous snake about in her lute.” When the king found it, he was furious. Intending to kill her, he took his bow and aimed a poisoned arrow. Sāmāvatī with her retinue pervaded the king with loving-kindness. The king stood trembling, unable either to shoot the arrow or to put it away. Then the queen said to him, “What is it, sire, are you tired?”—“Yes, I am tired.”—“Then put down the bow.” The arrow fell at the king’s feet. Then the queen advised him, “Sire, one should not hate one who has no hate.” So the king’s not daring to release the arrow was success by intervention of concentration in the laywoman Sāmāvatī (see \textbf{\cite{Dhp-a}I 216}; \textbf{\cite{A-a}I 443}).

                \vismParagraph{XII.36}{36}{}
                (vi) That which consists in dwelling perceiving the unrepulsive in the repulsive, etc., is called \emph{Noble Ones’ success}, according as it is said: “What is Noble Ones’ success? Here, if a bhikkhu should wish, “May I dwell perceiving the unrepulsive in the repulsive,” he dwells perceiving the unrepulsive in that … he dwells in equanimity towards that, mindful and fully aware” (\textbf{\cite{Paṭis}II 212}). \textcolor{brown}{\textit{[382]}} This is called “Noble Ones’ success” because it is only produced in Noble Ones who have reached mind mastery.

                \vismParagraph{XII.37}{37}{}
                For if a bhikkhu with cankers destroyed possesses this kind of success, then when in the case of a disagreeable object he is practicing pervasion with loving-kindness or giving attention to it as elements, he dwells perceiving the unrepulsive; or when in the case of an agreeable object he is practicing pervasion with foulness or giving attention to it as impermanent, he dwells perceiving the repulsive. Likewise, when in the case of the repulsive and unrepulsive he is practicing that same pervasion with loving-kindness or giving attention to it as elements, he dwells perceiving the unrepulsive; and when in the case of the unrepulsive and repulsive he is practicing that same pervasion with foulness or giving attention to it as impermanent, he dwells perceiving the repulsive. But when he is exercising the six-factored equanimity in the following way, “On seeing a visible object with the eye, he is neither glad nor …” (\textbf{\cite{Paṭis}II 213}), etc., then rejecting both the repulsive and the unrepulsive, he dwells in equanimity, mindful and fully aware.

                \vismParagraph{XII.38}{38}{}
                \marginnote{\textcolor{teal}{\footnotesize\{435|377\}}}{}For the meaning of this is expounded in the Paṭisambhidā in the way beginning: “How does he dwell perceiving the unrepulsive in the repulsive? In the case of a disagreeable object he pervades it with loving-kindness or he treats it as elements” (\textbf{\cite{Paṭis}II 212}). Thus it is called, “Noble Ones’ success” because it is only produced in Noble Ones who have reached mind mastery.

                \vismParagraph{XII.39}{39}{}
                (vii) That consisting in travelling through the air in the case of winged birds, etc., is called \emph{success born of kamma result}, according as it is said: “What is success born of kamma result? That in all winged birds, in all deities, in some human beings, in some inhabitants of states of loss, is success born of kamma result” (\textbf{\cite{Paṭis}II 213}). For here it is the capacity in all winged birds to travel through the air without jhāna or insight that is success born of kamma result; and likewise that in all deities, and some human beings, at the beginning of the aeon, and likewise that in some inhabitants of states of loss such as the female spirit Piyaṅkara’s mother (see \textbf{\cite{S-a}II 509}), Uttara’s mother (\textbf{\cite{Pv}140}), Phussamittā, Dhammaguttā, and so on.

                \vismParagraph{XII.40}{40}{}
                (viii) That consisting in travelling through the air, etc., in the case of Wheel-turning Monarchs, etc., is called \emph{success of the meritorious}, according as it is said: “What is success of the meritorious? The Wheel-turning Monarch travels through the air with his fourfold army, even with his grooms and shepherds. The householder Jotika had the success of the meritorious. The householder Jaṭilaka had the success of the meritorious. \textcolor{brown}{\textit{[383]}} The householder Ghosita had the success of the meritorious. The householder Meṇḍaka had the success of the meritorious. That of the five very meritorious is success of the meritorious” (\textbf{\cite{Paṭis}II 213}). In brief, however, it is the distinction that consists in succeeding when the accumulated merit comes to ripen that is success of the meritorious.

                \vismParagraph{XII.41}{41}{}
                A crystal palace and sixty-four wishing trees cleft the earth and sprang into existence for the householder Jotika. That was success of the meritorious in his case (\textbf{\cite{Dhp-a}IV 207}). A golden rock of eighty cubits [high] was made for Jaṭilaka (\textbf{\cite{Dhp-a}IV 216}). Ghosita’s safe survival when attempts were made in seven places to kill him was success of the meritorious (\textbf{\cite{Dhp-a}I 174}). The appearance to Meṇḍaka (= Ram) of rams (\emph{meṇḍaka}) made of the seven gems in a place the size of one \emph{sītā }\footnote{\vismAssertFootnoteCounter{6}\vismHypertarget{XII.n6}{}\emph{Sītā:} not in this sense in PED. \textbf{\cite{Vism-mhṭ}(p. 383)} says, “It is the path traversed by a ploughshare in ploughing that is called a \emph{sītā}.” Another reading is \emph{karīsa }(an area of land).}was success of the meritorious in Meṇḍaka (\textbf{\cite{Dhp-a}III 364}).

                \vismParagraph{XII.42}{42}{}
                The “five very meritorious” are the rich man Meṇḍaka, his wife Candapadumasiri, his son the rich man Dhanañjaya, his daughter-in-law Sumanadevī, and his slave Puṇṇa. When the rich man [Meṇḍaka] washed his head and looked up at the sky, twelve thousand five hundred measures were filled for him with red rice from the sky. When his wife took a \emph{nāḷi} measure of cooked rice, the food was not used up though she served the whole of Jambudīpa with it. When his son took a purse containing a thousand [ducats (\emph{kahāpaṇa})], the ducats were not exhausted even though he made gifts to all the inhabitants \marginnote{\textcolor{teal}{\footnotesize\{436|378\}}}{}of Jambudīpa. When his daughter-in-law took a pint (\emph{tumba}) measure of paddy, the grain was not used up even when she shared it out among all the inhabitants of Jambudīpa. When the slave ploughed with a single ploughshare, there were fourteen furrows, seven on each side (see \textbf{\cite{Vin}I 240}; \textbf{\cite{Dhp-a}I 384}). This was success of the meritorious in them.

                \vismParagraph{XII.43}{43}{}
                (ix) That beginning with travelling through the air in the case of masters of the sciences is \emph{success through the sciences}, according as it is said: “What is success through the sciences? Masters of the sciences, having pronounced their scientific spells, travel through the air, and they show an elephant in space, in the sky … and they show a manifold military array” (\textbf{\cite{Paṭis}II 213}).

                \vismParagraph{XII.44}{44}{}
                (x) But the succeeding of such and such work through such and such right exertion is \emph{success in the sense of succeeding due to right exertion applied here or there}, according as it is said: “The meaning (purpose) of abandoning lust succeeds through renunciation, thus it is success in the sense of succeeding due to right exertion applied here or there … The meaning (purpose) of abandoning all defilements succeeds through the Arahant path, thus it is success in the sense of succeeding due to right exertion applied here or there” (\textbf{\cite{Paṭis}II 213}). \textcolor{brown}{\textit{[384]}} And the text here is similar to the previous text in the illustration of right exertion, in other words, the way. But in the Commentary it is given as follows: “Any work belonging to a trade such as making a cart assemblage, etc., any medical work, the learning of the Three Vedas, the learning of the Three Piṭakas, even any work connected with ploughing, sowing, etc.—the distinction produced by doing such work is success in the sense of succeeding due to right exertion applied here or there.”

                \vismParagraph{XII.45}{45}{}
                So, among these ten kinds of success, only (i) success by resolve is actually mentioned in the clause “kinds of supernormal power (success),” but (ii) success as transformation and (iii) success as the mind-made [body] are needed in this sense as well.
                \subsection[\vismAlignedParas{§46–137}Supernormal power as resolve]{Supernormal power as resolve}

                    \vismParagraph{XII.46}{46}{}
                    (i) \emph{To the kinds of supernormal power} (see \hyperlink{XII.20}{§20}{}): to the components of supernormal power, or to the departments of supernormal power. \emph{He directs, he inclines, his mind}: when that bhikkhu’s consciousness has become the basis for direct-knowledge in the way already described, he directs the preliminary-work consciousness with the purpose of attaining the kinds of supernormal power, he sends it in the direction of the kinds of supernormal power, leading it away from the kasiṇa as its object. Inclines: makes it tend and lean towards the supernormal power to be attained.

                    \vismParagraph{XII.47}{47}{}
                    \emph{He}: the bhikkhu who has done the directing of his mind in this way. \emph{The various}: varied, of different sorts. \emph{Kinds of supernormal power}: departments of supernormal power. \emph{Wields: paccanubhoti = paccanu-bhavati }(alternative form); the meaning is that he makes contact with, realizes, reaches.

                    \vismParagraph{XII.48}{48}{}
                    Now, in order to show that variousness, it is said: “Having been one, [he becomes many; having been many, he becomes one. He appears and vanishes. He goes unhindered through walls, through enclosures, through mountains, as though in open space. He dives in and out of the earth as though in water. He goes on unbroken water as though on earth. Seated cross-legged he travels in space like a winged bird. With his hand he \marginnote{\textcolor{teal}{\footnotesize\{437|379\}}}{}touches and strokes the moon and sun so mighty and powerful. He wields bodily mastery even as far as the Brahmā-world]” (\textbf{\cite{D}I 77}).

                    Herein, \emph{having been one}: having been normally one before giving effect to the supernormal power. \emph{He becomes many}: wanting to walk with many or wanting to do a recital or wanting to ask questions with many, he becomes a hundred or a thousand. But how does he do this? He accomplishes, (1) the four planes, (2) the four bases (roads), (3) the eight steps, and (4) the sixteen roots of supernormal power, and then he (5) resolves with knowledge.

                    \vismParagraph{XII.49}{49}{}
                    \emph{1.} Herein, the \emph{four planes} should be understood as the four jhānas; for this has been said by the General of the Dhamma [the Elder Sāriputta]: “What are the four planes of supernormal power? They are the first jhāna as the plane born of seclusion, the second jhāna as the plane of happiness and bliss, the third jhāna as the plane of equanimity and bliss, the fourth jhāna as the plane of neither pain nor pleasure. These four planes of supernormal power lead to the attaining of supernormal power, to the obtaining of supernormal power, to the transformation due to supernormal power, to the majesty\footnote{\vismAssertFootnoteCounter{7}\vismHypertarget{XII.n7}{}\emph{Visavitā—}“majesty”: not in PED; cf. \emph{passavati.} \textbf{\cite{Vism-mhṭ}(p. 385)} glosses with \emph{iddhiyā vividhānisaṃsa-pasavanāya.} Cf. \textbf{\cite{Dhs-a}109}; \textbf{\cite{Dhs-ṭ}(p. 84)} glosses thus \emph{visavitāyā ti arahatāya.}} of supernormal power, to the mastery of supernormal power, to fearlessness in supernormal power” (\textbf{\cite{Paṭis}II 205}). And he reaches supernormal power by becoming light, malleable and wieldy in the body after steeping himself in blissful perception and light perception due to the pervasion of happiness and pervasion of bliss, \textcolor{brown}{\textit{[385]}} which is why the first three jhānas should be understood as the accessory plane since they lead to the obtaining of supernormal power in this manner. But the fourth is the natural plane for obtaining supernormal power.

                    \vismParagraph{XII.50}{50}{}
                    \emph{2.} The \emph{four bases (roads)} should be understood as the four bases of success (\emph{iddhi-pāda—}roads to power); for this is said: “What are the four bases (\emph{pāda—}roads) for success (\emph{iddhi—}power)? Here a bhikkhu develops the basis for success (road to power) that possesses both concentration due to zeal and the will to strive (endeavour); he develops the basis for success (road to power) that possesses both concentration due to energy and the will to strive; he develops the basis for success (road to power) that possesses both concentration due to [natural purity of] consciousness and the will to strive; he develops the basis for success (road to power) that possesses both concentration due to inquiry and the will to strive. These four bases (roads) for success (power) lead to the obtaining of supernormal power (success) … to the fearlessness due to supernormal power (success)” (\textbf{\cite{Paṭis}II 205}).

                    \vismParagraph{XII.51}{51}{}
                    And here the concentration that has zeal as its cause, or has zeal outstanding, is \emph{concentration due to zeal}; this is a term for concentration obtained by giving precedence to zeal consisting in the desire to act. Will (formation) as endeavour is \emph{will to strive}; this is a term for the energy of right endeavour accomplishing its fourfold function (see \hyperlink{XII.53}{§53}{}). \emph{Possesses}: is furnished with concentration due to zeal and with the [four] instances of the will to strive.

                    \vismParagraph{XII.52}{52}{}
                    \marginnote{\textcolor{teal}{\footnotesize\{438|380\}}}{}\emph{Road to power (basis for success)}: the meaning is, the total of consciousness and its remaining concomitants [except the concentration and the will], which are, in the sense of resolve, the road to (basis for) the concentration due to zeal and will to strive associated with the direct-knowledge consciousness, which latter are themselves termed “power (success)” either by treatment as “production” (\hyperlink{XII.20}{§20}{}) or in the sense of “succeeding” (\hyperlink{XII.21}{§21}{}) or by treatment in this way, “beings succeed by its means, thus they are successful; they are enriched, promoted” (\hyperlink{XII.22}{§22}{}). For this is said: “Basis for success (road to power): it is the feeling aggregate, [perception aggregate, formations aggregate, and] consciousness aggregate, in one so become” (\textbf{\cite{Vibh}217}).

                    \vismParagraph{XII.53}{53}{}
                    Or alternatively: it is arrived at (\emph{pajjate}) by means of that, thus that is a road (\emph{pāda—}basis); it is reached, is the meaning. \emph{Iddhi-pāda = iddhiyā pāda} (resolution of compound): this is a term for zeal, etc., according as it is said: “Bhikkhus, if a bhikkhu obtains concentration, obtains unification of mind supported by zeal, this is called concentration due to zeal. He [awakens zeal] for the non-arising of unarisen evil, unprofitable states, [strives, puts forth energy, strains his mind and] struggles. [He awakens zeal for the abandoning of arisen evil, unprofitable states … He awakens zeal for the arousing of unarisen profitable states … He awakens zeal for the maintenance, non-disappearance, increase, growth, development and perfection of arisen profitable states, strives, puts forth energy, strains his mind and struggles]. These are called instances of the will to strive. So this zeal and this concentration due to zeal and these [four] instances of will to strive are called the road to power (basis for success) that possesses concentration due to zeal and the will to strive” (\textbf{\cite{S}V 268}). And the meaning should be understood in this way in the case of the other roads to power (bases for success).\footnote{\vismAssertFootnoteCounter{8}\vismHypertarget{XII.n8}{}Further explanatory details are given in the commentary to the Iddhipāda Vibhaṅga.}

                    \vismParagraph{XII.54}{54}{}
                    \emph{3.} The \emph{eight steps} should be understood as the eight beginning with zeal; for this is said: “What are the eight steps? If a bhikkhu obtains concentration, obtains unification of mind supported by zeal, then the zeal is not the concentration; the concentration is not the zeal. \textcolor{brown}{\textit{[386]}} The zeal is one, the concentration is another. If a bhikkhu … supported by energy … supported by [natural purity of] consciousness … supported by inquiry … then the inquiry is not the concentration; the concentration is not the inquiry. The inquiry is one, the concentration is another. These eight steps to power lead to the obtaining of supernormal power (success) … to fearlessness due to supernormal power (success)” (\textbf{\cite{Paṭis}II 205}). For here it is the zeal consisting in the desire to arouse supernormal power (success), which zeal is joined with concentration, that leads to the obtaining of the supernormal power. Similarly in the case of energy, and so on. That should be understood as the reason why they are called the “eight steps.”

                    \vismParagraph{XII.55}{55}{}
                    \emph{4.} The \emph{sixteen roots}: the mind’s unperturbedness\footnote{\vismAssertFootnoteCounter{9}\vismHypertarget{XII.n9}{}\emph{Aneja} (or \emph{aneñja})—“unperturbed”: form not in PED.} should be understood in sixteen modes, for this is said: “What are the sixteen roots of success (power)? \marginnote{\textcolor{teal}{\footnotesize\{439|381\}}}{}Undejected consciousness is not perturbed by indolence, thus it is unperturbed. Unelated consciousness is not perturbed by agitation, thus it is unperturbed. Unattracted consciousness is not perturbed by greed, thus it is unperturbed. Unrepelled consciousness is not perturbed by ill will, thus it is unperturbed. Independent consciousness is not perturbed by [false] view, thus it is unperturbed. Untrammelled consciousness is not perturbed by greed accompanied by zeal, thus it is unperturbed. Liberated consciousness is not perturbed by greed for sense desires, thus it is unperturbed. Unassociated consciousness is not perturbed by defilement, thus it is unperturbed. Consciousness rid of barriers is not perturbed by the barrier of defilement, thus it is unperturbed. Unified consciousness is not perturbed by the defilement of variety, thus it is unperturbed. Consciousness reinforced by faith is not perturbed by faithlessness, thus it is unperturbed. Consciousness reinforced by energy is not perturbed by indolence, thus it is unperturbed. Consciousness reinforced by mindfulness is not perturbed by negligence, thus it is unperturbed. Consciousness reinforced by concentration is not perturbed by agitation, thus it is unperturbed. Consciousness reinforced by understanding is not perturbed by ignorance, thus it is unperturbed. Illuminated consciousness is not perturbed by the darkness of ignorance, thus it is unperturbed. These sixteen roots of success (power) lead to the obtaining of supernormal power (success) … to fearlessness due to supernormal power (success)” (\textbf{\cite{Paṭis}II 206}).

                    \vismParagraph{XII.56}{56}{}
                    Of course, this meaning is already established by the words, “When his concentrated mind,” etc., too, but it is stated again for the purpose of showing that the first jhāna, etc., are the three planes, bases (roads), steps, and roots, of success (to supernormal powers). And the first-mentioned method is the one given in the Suttas, but this is how it is given in the Paṭisambhidā. So it is stated again for the purpose of avoiding confusion in each of the two instances.

                    \vismParagraph{XII.57}{57}{}
                    \emph{5. He resolves with knowledge} (\hyperlink{XII.48}{§48}{}): when he has accomplished these things consisting of the planes, bases (roads), steps, and roots, of success (to supernormal power), \textcolor{brown}{\textit{[387]}} then he attains jhāna as the basis for direct-knowledge and emerges from it. Then if he wants to become a hundred, he does the preliminary work thus, “Let me become a hundred, let me become a hundred,” after which he again attains jhāna as basis for direct-knowledge, emerges, and resolves. He becomes a hundred simultaneously with the resolving consciousness. The same method applies in the case of a thousand, and so on. If he does not succeed in this way, he should do the preliminary work again, and attain, emerge, and resolve a second time. For it is said in the Saṃyutta Commentary that it is allowable to attain once, or twice.

                    \vismParagraph{XII.58}{58}{}
                    Herein, the basic-jhāna consciousness has the sign as its object; but the preliminary-work consciousnesses have the hundred as their object or the thousand as their object. And these latter are objects as appearances, not as concepts. The resolving consciousness has likewise the hundred as its object or the thousand as its object. That arises once only, next to change-of-lineage [consciousness], as in the case of absorption consciousness already described (\hyperlink{IV.78}{IV.78}{}), and it is fine-material-sphere consciousness belonging to the fourth jhāna.

                    \vismParagraph{XII.59}{59}{}
                    \marginnote{\textcolor{teal}{\footnotesize\{440|382\}}}{}Now, it is said in the Paṭisambhidā: “Normally one, he adverts to [himself as] many or a hundred or a thousand or a hundred thousand; having adverted, he resolves with knowledge, ‘Let me be many.’ He becomes many, like the venerable Cūḷa-Panthaka” (\textbf{\cite{Paṭis}II 207}). Here \emph{he adverts} is said with respect only to the preliminary work. \emph{Having adverted, he resolves with knowledge} is said with respect to the knowledge of the direct-knowledge. Consequently, he adverts to many. After that he attains with the last one of the preliminary-work consciousnesses. After emerging from the attainment, he again adverts thus, “Let me be many,” after which he resolves by means of the single [consciousness] belonging to the knowledge of direct-knowledge, which has arisen next to the three, or four, preparatory consciousnesses that have occurred, and which has the name “resolve” owing to its making the decision. This is how the meaning should be understood here.

                    \vismParagraph{XII.60}{60}{}
                    \emph{Like the venerable Cūḷa-Panthaka} is said in order to point to a bodily witness of this multiple state; but that must be illustrated by the story. There were two brothers, it seems, who were called, “Panthaka (Roadling)” because they were born on a road. The senior of the two was called Mahā-Panthaka. He went forth into homelessness and reached Arahantship together with the discriminations. When he had become an Arahant, he made Cūḷa-Panthaka go forth too, and he set him this stanza: \textcolor{brown}{\textit{[388]}}
                    \begin{verse}
                        As a scented \emph{kokanada} lotus\\{}
                        Opens in the morning with its perfume,\\{}
                        See the One with Radiant Limbs who glitters\footnote{\vismAssertFootnoteCounter{10}\vismHypertarget{XII.n10}{}\emph{Aṅgīrasa—}“the One with Radiant Limbs”: one of the epithets for the Buddha. Not in PED; see \textbf{\cite{A}III 239}.}\\{}
                        Like the sun’s orb blazing in the heavens (\textbf{\cite{A}III 239}; \textbf{\cite{S}I 81}).
                    \end{verse}


                    Four months went by, but he could not get it by heart. Then the elder said, “You are useless in this dispensation,” and he expelled him from the monastery.

                    \vismParagraph{XII.61}{61}{}
                    At that time the elder had charge of the allocation of meal [invitations]. Jīvaka approached the elder, saying, “Take alms at our house, venerable sir, together with the Blessed One and five hundred bhikkhus.” The elder consented, saying, “I accept for all but Cūḷa-Panthaka.” Cūḷa-Panthaka stood weeping at the gate. The Blessed One saw him with the divine eye, and he went to him. “Why are you weeping?” he asked, and he was told what had happened.

                    \vismParagraph{XII.62}{62}{}
                    The Blessed One said, “No one in my dispensation is called useless for being unable to do a recitation. Do not grieve, bhikkhu.” Taking him by the arm, he led him into the monastery. He created a piece of cloth by supernormal power and gave it to him, saying, “Now, bhikkhu, keep rubbing this and recite over and over again: ‘Removal of dirt, removal of dirt.’” While doing as he had been told, the cloth became black in colour. What he came to perceive was this: “The cloth is clean; there is nothing wrong there. It is this selfhood that is wrong.” He brought his knowledge to bear on the five aggregates, and by increasing insight he reached the neighbourhood of conformity [knowledge] and change-of-lineage [knowledge].

                    \vismParagraph{XII.63}{63}{}
                    \marginnote{\textcolor{teal}{\footnotesize\{441|383\}}}{}Then the Blessed One uttered these illuminative stanzas:
                    \begin{verse}
                        Now greed it is, not dust, that we call “dirt,”\\{}
                        And “dirt” is just a term in use for greed;\\{}
                        This greed the wise reject, and they abide\\{}
                        Keeping the Law of him that has no greed.\\{}
                        Now, hate it is, not dust, that we call “dirt,”\\{}
                        … … …\\{}
                        Delusion too, it is not dust, that we call “dirt,”\\{}
                        And “dirt” is just a term used for delusion;\\{}
                        Delusion the wise reject, and they abide\\{}
                        Keeping the Dhamma of him without delusion (\textbf{\cite{Nidd}I 505}).\\{}
                        \textcolor{brown}{\textit{[389]}}
                    \end{verse}


                    When the stanzas were finished, the venerable Cūḷa-Panthaka had at his command the nine supramundane states attended by the four discriminations and six kinds of direct-knowledge.

                    \vismParagraph{XII.64}{64}{}
                    On the following day the Master went to Jīvaka’s house together with the Community of Bhikkhus. Then when the gruel was being given out at the end of the water-offering ceremony,\footnote{\vismAssertFootnoteCounter{11}\vismHypertarget{XII.n11}{}Dedication of what is to be given accompanied by pouring water over the hand.} he covered his bowl. Jīvaka asked, “What is it, venerable sir?”—“There is a bhikkhu at the monastery.” He sent a man, telling him, “Go, and return quickly with the lord.”

                    \vismParagraph{XII.65}{65}{}
                    When the Blessed One had left the monastery:
                    \begin{verse}
                        Now, having multiplied himself\\{}
                        Up to a thousand, Panthaka\\{}
                        Sat in the pleasant mango wood\\{}
                        until the time should be announced (\textbf{\cite{Th}563}).
                    \end{verse}


                    \vismParagraph{XII.66}{66}{}
                    When the man went and saw the monastery all glowing with yellow, he returned and said, “Venerable sir, the monastery is crowded with bhikkhus. I do not know which of them the lord is.” Then the Blessed One said, “Go and catch hold of the hem of the robe of the first one you see, tell him, ‘The Master calls you’ and bring him here.” He went and caught hold of the elder’s robe. At once all the creations vanished. The elder dismissed him, saying, “You may go,” and when he had finished attending to his bodily needs such as mouth washing, he arrived first and sat down on the seat prepared.

                    It was with reference to this that it was said, “like the venerable Cūḷa-Panthaka.”

                    \vismParagraph{XII.67}{67}{}
                    The many who were created there were just like the possessor of the supernormal power because they were created without particular specification. Then whatever the possessor of the supernormal powers does, whether he stands, sits, etc., or speaks, keeps silent, etc., they do the same. But if he wants to make them different in appearance, some in the first phase of life, some in the middle phase, and some in the last phase, and similarly some long-haired, some half-shaved, some shaved, some grey-haired, some with lightly dyed robes, some with heavily dyed robes, or expounding phrases, explaining Dhamma, intoning, asking questions, answering \marginnote{\textcolor{teal}{\footnotesize\{442|384\}}}{}questions, cooking dye, sewing and washing robes, etc., or if he wants to make still others of different kinds, he should emerge from the basic jhāna, do the preliminary work in the way beginning ‘Let there be so many bhikkhus in the first phase of life’, etc.; then he should once more attain and emerge, and then resolve. They become of the kinds desired simultaneously with the resolving consciousness.\footnote{\vismAssertFootnoteCounter{12}\vismHypertarget{XII.n12}{}“‘\emph{They become of the kinds desired}’: they become whatever the kinds that were desired: for they come to possess as many varieties in appearance, etc., as it was wished they should have. But although they become manifold in this way by being made the object in different modes of appearance, nevertheless it is only a single resolution consciousness that occurs. This is its power. For it is like the single volition that produces a personality possessed of many different facets (see \hyperlink{XIV.n14}{Ch. XIV, n. 14}{}). And there it is the aspiration to become that is a condition for the differentiation in the kamma; and kamma-result is imponderable. And here too it is the preliminary-work consciousness that should be taken as a condition for the difference. And the field of supernormal power is imponderable too.” (\textbf{\cite{Vism-mhṭ}390})}

                    \vismParagraph{XII.68}{68}{}
                    The same method of explanation applies to the clause \emph{having been many, he becomes one}: but there is this difference. After this bhikkhu thus created a manifold state, then he again thinks, “As one only I will walk about, do a recital, \textcolor{brown}{\textit{[390]}} ask a question,” or out of fewness of wishes he thinks, “This is a monastery with few bhikkhus. If someone comes, he will wonder, ‘Where have all these bhikkhus who are all alike come from? Surely it will be one of the elder’s feats?’ and so he might get to know about me.” Meanwhile, wishing, “Let me be one only,” he should attain the basic jhāna and emerge. Then, after doing the preliminary work thus, “Let me be one,” he should again attain and emerge and then resolve thus, ‘Let me be one’. He becomes one simultaneously with the resolving consciousness. But instead of doing this, he can automatically become one again with the lapse of the predetermined time.

                    \vismParagraph{XII.69}{69}{}
                    \emph{He appears and vanishes}: the meaning here is that he causes appearance, causes vanishing. For it is said in the Paṭisambhidā with reference to this: “‘He appears’: he is not veiled by something, he is not hidden, he is revealed, he is evident. ‘Vanishes’: he is veiled by something, he is hidden, he is shut away, he is enclosed” (\textbf{\cite{Paṭis}II 207}).\footnote{\vismAssertFootnoteCounter{13}\vismHypertarget{XII.n13}{}Certain grammatical problems arise about the case of the words \emph{āvibhāvaṃ}, etc., both in the sutta passage and (more so) in the Paṭisambhidā passage; they are examined by \textbf{\cite{Vism-mhṭ}(p. 390)} but are not renderable into English.}

                    Now, this possessor of supernormal power who wants to make an appearance, makes darkness into light, or he makes revealed what is hidden, or he makes what has not come into the visual field come into the visual field.

                    \vismParagraph{XII.70}{70}{}
                    How? If he wants to make himself or another visible even though hidden or at a distance, he emerges from the basic jhāna and adverts thus, “Let this that is dark become light” or “Let this that is hidden be revealed” or “Let this that has not come into the visual field come into the visual field.” Then he does the preliminary work and resolves in the way already described. It becomes as resolved simultaneously with the resolve. Others then see even when at a distance; and he himself sees too, if he wants to see.

                    \vismParagraph{XII.71}{71}{}
                    \marginnote{\textcolor{teal}{\footnotesize\{443|385\}}}{}But by whom was this miracle formerly performed? By the Blessed One. For when the Blessed One had been invited by Cūḷa-Subhaddā and was traversing the seven-league journey between Sāvatthī and Sāketa with five hundred palanquins\footnote{\vismAssertFootnoteCounter{14}\vismHypertarget{XII.n14}{}\emph{Kūṭāgāra—}“palanquin”: not in this sense in PED. See story at \textbf{\cite{M-a}V 90}, where it is told how 500 of these were made by Sakka’s architect Vissakamma for the Buddha to journey through the air in. The same word is also commonly used in the Commentaries for the portable structure (catafalque) in which a bier is carried to the pyre. This, built often in the form of a house, is still used now in Sri Lanka and called \emph{ransivi-ge}. See A-a commentary to AN 3:42, and to AN 1:38; also \textbf{\cite{Dhp-a}III 470}. Not in this sense in PED.}created by Vissakamma (see \textbf{\cite{Dhp-a}III 470}), he resolved in suchwise that citizens of Sāketa saw the inhabitants of Sāvatthī and citizens of Sāvatthī saw the inhabitants of Sāketa. And when he had alighted in the centre of the city, he split the earth in two and showed Avīci, and he parted the sky in two and showed the Brahmā-world.

                    \vismParagraph{XII.72}{72}{}
                    And this meaning should also be explained by means of the Descent of the Gods (\emph{devorohaṇa}). When the Blessed One, it seems, had performed the Twin Miracle\footnote{\vismAssertFootnoteCounter{15}\vismHypertarget{XII.n15}{}The only book in the Tipiṭaka to mention the Twin Miracle is the Paṭisambhidāmagga (\textbf{\cite{Paṭis}I 53}).16 Anāthapiṇḍika’s younger brother (\textbf{\cite{Vism-mhṭ}391}).}and had liberated eighty-four thousand beings from bonds, he wondered, “Where did the past Enlightened Ones go to when they had finished the Twin Miracle?” He saw that they had gone to the heaven of the Thirty-three. \textcolor{brown}{\textit{[391]}} Then he stood with one foot on the surface of the earth, and placed the second on Mount Yugandhara. Then again he lifted his first foot and set it on the summit of Mount Sineru. He took up the residence for the Rains there on the Red Marble Terrace, and he began his exposition of the Abhidhamma, starting from the beginning, to the deities of ten thousand world-spheres. At the time for wandering for alms he created an artificial Buddha to teach the Dhamma.

                    \vismParagraph{XII.73}{73}{}
                    Meanwhile the Blessed One himself would chew a tooth-stick of \emph{nāgalatā }wood and wash his mouth in Lake Anotatta. Then, after collecting alms food among the Uttarakurus, he would eat it on the shores of that lake. [Each day] the Elder Sāriputta went there and paid homage to the Blessed One, who told him, “Today I taught this much Dhamma,” and he gave him the method. In this way he gave an uninterrupted exposition of the Abhidhamma for three months. Eighty million deities penetrated the Dhamma on hearing it.

                    \vismParagraph{XII.74}{74}{}
                    At the time of the Twin Miracle an assembly gathered that was twelve leagues across. Then, saying, “We will disperse when we have seen the Blessed One,” they made an encampment and waited there. Anāthapiṇḍika the Lesser\footnote{\vismAssertFootnoteCounter{16}\vismHypertarget{XII.n16}{}Anāthapiṇḍika’s younger brother (\textbf{\cite{Vism-mhṭ}391}).} supplied all their needs. People asked the Elder Anuruddha to find out where the Blessed One was. The elder extended light, and with the divine eye he saw where the Blessed One had taken up residence for the Rains. As soon as he saw this, he announced it.

                    \vismParagraph{XII.75}{75}{}
                    They asked the venerable Mahā Moggallāna to pay homage to the Blessed One. In the midst of the assembly the elder dived into the earth. Then cleaving \marginnote{\textcolor{teal}{\footnotesize\{444|386\}}}{}Mount Sineru, he emerged at the Perfect One’s feet, and he paid homage at the Blessed One’s feet. This is what he told the Blessed One: “Venerable sir, the inhabitants of Jambudīpa pay homage at the Blessed One’s feet, and they say, ‘We will disperse when we have seen the Blessed One.’” The Blessed One said, “But, Moggallāna, where is your elder brother, the General of the Dhamma?”—“At the city of Saṅkassa, venerable sir.”—“Moggallāna, those who wish to see me should come tomorrow to the city of Saṅkassa. Tomorrow being the Uposatha day of the full moon, I shall descend to the city of Saṅkassa for the Mahāpavāraṇā ceremony.”

                    \vismParagraph{XII.76}{76}{}
                    Saying, “Good, venerable sir,” the elder paid homage to Him of the Ten Powers, and descending by the way he came, he reached the human neighbourhood. And at the time of his going and coming he resolved that people should see it. This, firstly, is the miracle of becoming apparent that the Elder Mahā Moggallāna performed here. Having arrived thus, he related what had happened, and he said, “Come forth after the morning meal and pay no heed to distance” [thus promising that they would be able to see in spite of the distance].

                    \vismParagraph{XII.77}{77}{}
                    The Blessed One informed Sakka, Ruler of Gods, “Tomorrow, O King, I am going to the human world.” The Ruler of Gods \textcolor{brown}{\textit{[392]}} commanded Vissakamma, “Good friend, the Blessed One wishes to go to the human world tomorrow. Build three flights of stairs, one of gold, one of silver and one of crystal.” He did so.

                    \vismParagraph{XII.78}{78}{}
                    On the following day the Blessed One stood on the summit of Sineru and surveyed the eastward world element. Many thousands of world-spheres were visible to him as clearly as a single plain. And as the eastward world element, so too he saw the westward, the northward and the southward world elements all clearly visible. And he saw right down to Avīci, and up to the Realm of the Highest Gods. That day, it seems, was called the day of the Revelation of Worlds (\emph{loka-vivaraṇa}). Human beings saw deities, and deities saw human beings. And in doing so the human beings did not have to look up or the deities down. They all saw each other face to face.

                    \vismParagraph{XII.79}{79}{}
                    The Blessed One descended by the middle flight of stairs made of crystal; the deities of the six sense-sphere heavens by that on the left side made of gold; and the deities of the Pure Abodes, and the Great Brahmā, by that on the right side made of silver. The Ruler of Gods held the bowl and robe. The Great Brahmā held a three-league-wide white parasol. Suyāma held a yak-tail fan. Five-crest (\emph{Pañcasikha}), the son of the gandhabba, descended doing honour to the Blessed One with his bael-wood lute measuring three quarters of a league. On that day there was no living being present who saw the Blessed One but yearned for enlightenment. This is the miracle of becoming apparent that the Blessed One performed here.

                    \vismParagraph{XII.80}{80}{}
                    Furthermore, in Tambapaṇṇi Island (Sri Lanka), while the Elder Dhammadinna, resident of Taḷaṅgara, was sitting on the shrine terrace in the Great Monastery of Tissa (\emph{Tissamahāvihāra}) expounding the Apaṇṇaka Sutta, “Bhikkhus, when a bhikkhu possesses three things he enters upon the untarnished way” (\textbf{\cite{A}I 113}), he turned his fan face downwards and an opening right down to Avīci appeared. Then he turned it face upwards and an opening \marginnote{\textcolor{teal}{\footnotesize\{445|387\}}}{}right up to the Brahmā-world appeared. Having thus aroused fear of hell and longing for the bliss of heaven, the elder taught the Dhamma. Some became stream-enterers, some once-returners, some non-returners, some Arahants.

                    \vismParagraph{XII.81}{81}{}
                    But one who wants to cause a vanishing makes light into darkness, or he hides what is unbidden, or he makes what has come into the visual field come no more into the visual field. How? If he wants to make himself or another invisible even though unconcealed or nearby, he emerges from the basic jhāna and adverts thus, “Let this light become darkness” or \textcolor{brown}{\textit{[393]}} “Let this that is unhidden be hidden” or “Let this that has come into the visual field not come into the visual field.” Then he does the preliminary work and resolves in the way already described. It becomes as he has resolved simultaneously with the resolution. Others do not see even when they are nearby. He too does not see, if he does not want to see.

                    \vismParagraph{XII.82}{82}{}
                    But by whom was this miracle formerly performed? By the Blessed One. For the Blessed One so acted that when the clansman Yasa was sitting beside him, his father did not see him (\textbf{\cite{Vin}I 16}). Likewise, after travelling two thousand leagues to meet [King] Mahā Kappina and establishing him in the fruition of non-return and his thousand ministers in the fruition of stream-entry, he so acted that Queen Anojā, who had followed the king with a thousand women attendants and was sitting nearby, did not see the king and his retinue. And when he was asked, “Have you seen the king, venerable sir?,” he asked, But which is better for you, to seek the king or to seek [your] self?” (cf. \textbf{\cite{Vin}I 23}). She replied, “[My] self, venerable sir.” Then he likewise taught her the Dhamma as she sat there, so that, together with the thousand women attendants, she became established in the fruition of stream-entry, while the ministers reached the fruition of non-return, and the king that of Arahantship (see \textbf{\cite{A-a}I 322}; \textbf{\cite{Dhp-a}II 124}).

                    \vismParagraph{XII.83}{83}{}
                    Furthermore, this was performed by the Elder Mahinda, who so acted on the day of his arrival in Tambapaṇṇi Island that the king did not see the others who had come with him (see \emph{Mahāvaṃsa} I 103).

                    \vismParagraph{XII.84}{84}{}
                    Furthermore, all miracles of making evident are called an appearance, and all miracles of making unevident are called a vanishing. Herein, in the miracle of making evident, both the supernormal power and the possessor of the supernormal power are displayed. That can be illustrated with the Twin Miracle; for in that both are displayed thus: “Here the Perfect One performs the Twin Miracle, which is not shared by disciples. He produces a mass of fire from the upper part of his body and a shower of water from the lower part of his body …” (\textbf{\cite{Paṭis}I 125}). In the case of the miracle of making unevident, only the supernormal power is displayed, not the possessor of the supernormal power. That can be illustrated by means of the Mahaka Sutta (\textbf{\cite{S}IV 200}), and the Brahmanimantanika Sutta (\textbf{\cite{M}I 330}). For there it was only the supernormal power of the venerable Mahaka and of the Blessed One respectively that was displayed, not the possessors of the supernormal power, according as it is said:

                    \vismParagraph{XII.85}{85}{}
                    “When he had sat down at one side, the householder Citta said to the venerable Mahaka, ‘Venerable sir, it would be good if the lord would show me a miracle of supernormal power belonging to the higher than human state.’—\marginnote{\textcolor{teal}{\footnotesize\{446|388\}}}{}‘Then, householder, spread your upper robe out on the terrace \textcolor{brown}{\textit{[394]}} and scatter\footnote{\vismAssertFootnoteCounter{17}\vismHypertarget{XII.n17}{}\emph{Okāseti—}“to scatter”: PED, this ref., gives “to show,” which does not fit the context. \textbf{\cite{Vism-mhṭ}} glosses with \emph{pakirati}.} a bundle of hay on it.’—‘Yes, venerable sir,’ the householder replied to the venerable Mahaka, and he spread out his upper robe on the terrace and scattered a bundle of hay on it. Then the venerable Mahaka went into his dwelling and fastened the latch, after which he performed a feat of supernormal power such that flames came out from the keyhole and from the gaps in the fastenings and burned the hay without burning the upper robe” (\textbf{\cite{S}IV 290}).

                    \vismParagraph{XII.86}{86}{}
                    Also according as it is said: “Then, bhikkhus, I performed a feat of supernormal power such that Brahmā and Brahmā’s retinue, and those attached to Brahmā’s retinue might hear my voice and yet not see me, and having vanished in this way, I spoke this stanza:
                    \begin{verse}
                        I saw the fear in [all kinds of] becoming,\\{}
                        Including becoming that seeks non-becoming;\\{}
                        And no becoming do I recommend;\\{}
                        I cling to no delight therein at all (\textbf{\cite{M}I 330}).
                    \end{verse}


                    \vismParagraph{XII.87}{87}{}
                    \emph{He goes unhindered through walls, through enclosures, through mountains, as though in open space}: here \emph{through walls} is beyond walls; the yonder side of a wall, is what is meant. So with the rest. And wall is a term for the wall of a house; enclosure is a wall surrounding a house, monastery (park), village, etc.; \emph{mountain }is a mountain of soil or a mountain of stone. \emph{Unhindered}: not sticking. \emph{As though in open space}: just as if he were in open space.

                    \vismParagraph{XII.88}{88}{}
                    One who wants to go in this way should attain the space-kasiṇa [jhāna] and emerge, and then do the preliminary work by adverting to the wall or the enclosure or some such mountain as Sineru or the World-sphere Mountains, and he should resolve, “Let there be space.” It becomes space only; it becomes hollow for him if he wants to go down or up; it becomes cleft for him if he wants to penetrate it. He goes through it unhindered.

                    \vismParagraph{XII.89}{89}{}
                    But here the Elder Tipiṭaka Cūḷa-Abhaya said: “Friends, what is the use of attaining the space-kasiṇa [jhāna]? Does one who wants to create elephants, horses, etc., attain an elephant-kasiṇa jhāna or horse-kasiṇa jhāna, and so on? Surely the only standard is mastery in the eight attainments, and after the preliminary work has been done on any kasiṇa, it then becomes whatever he wishes.” The bhikkhus said, “Venerable sir, only the space kasiṇa has been given in the text, so it should certainly be mentioned.”

                    \vismParagraph{XII.90}{90}{}
                    Here is the text: “He is normally an obtainer of the space-kasiṇa attainment. He adverts: “Through the wall, through the enclosure, through the mountain.” \textcolor{brown}{\textit{[395]}} Having adverted, he resolves with knowledge: “Let there be space.” There is space. He goes unhindered through the wall, through the enclosure, through the mountain. Just as men normally not possessed of supernormal power go unhindered where there is no obstruction or enclosure, so too this possessor of supernormal power, by his attaining mental mastery, goes unhindered through \marginnote{\textcolor{teal}{\footnotesize\{447|389\}}}{}the wall, through the enclosure, through the mountain, as though in open space” (\textbf{\cite{Paṭis}II 208}).

                    \vismParagraph{XII.91}{91}{}
                    What if a mountain or a tree is raised in this bhikkhu’s way while he is travelling along after resolving; should he attain and resolve again?—There is no harm in that. For attaining and resolving again is like taking the dependence (see \textbf{\cite{Vin}I 58}; II 274) in the preceptor’s presence. And because this bhikkhu has resolved, “Let there be space,” there will be only space there, and because of the power of his first resolve it is impossible that another mountain or tree can have sprung up meanwhile made by temperature. However, if it has been created by another possessor of supernormal power and created first, it prevails; the former must go above or below it.

                    \vismParagraph{XII.92}{92}{}
                    \emph{He dives in and out of the ground} (\emph{pathaviyā pi ummujjanimmujjaṃ}): here it is rising up that is called “diving out” (\emph{ummujja}) and it is sinking down that is called “diving in” (\emph{nimmujja}). \emph{Ummujjanimmujjaṃ = ummujjañ ca nimmujjañ ca }(resolution of compound).

                    One who wants to do this should attain the water-kasiṇa [jhāna] and emerge. Then he should do the preliminary work, determining thus, “Let the earth in such an area be water,” and he should resolve in the way already described. Simultaneously with the resolve, that much extent of earth according as determined becomes water only. It is there he does the diving in and out.

                    \vismParagraph{XII.93}{93}{}
                    Here is the text: “He is normally an obtainer of the water-kasiṇa attainment. He adverts to earth. Having adverted, he resolves with knowledge: “Let there be water.” There is water. He does the diving in and out of the earth. Just as men normally not possessed of supernormal power do diving in and out of water, so this possessor of supernormal power, by his attaining mental mastery, does the diving in and out of the earth as though in water” (\textbf{\cite{Paṭis}II 208}).

                    \vismParagraph{XII.94}{94}{}
                    And he does not only dive in and out, but whatever else he wants, such as bathing, drinking, mouth washing, washing of chattels, and so on. And not only water, but there is whatever else (liquid that) he wants, such as ghee, oil, honey, molasses, and so on. When he does the preliminary work, after adverting thus, “Let there be so much of this and this” and resolves, \textcolor{brown}{\textit{[396]}} it becomes as he resolved. If he takes them and fills dishes with them, the ghee is only ghee, the oil, etc., only oil, etc., the water only water. If he wants to be wetted by it, he is wetted, if he does not want to be wetted by it, he is not wetted. And it is only for him that that earth becomes water, not for anyone else. People go on it on foot and in vehicles, etc., and they do their ploughing, etc., there. But if he wishes, “Let it be water for them too,” it becomes water for them too. When the time determined has elapsed, all the extent determined, except for water originally present in water pots, ponds, etc., becomes earth again.

                    \vismParagraph{XII.95}{95}{}
                    \emph{On unbroken water}: here water that one sinks into when trodden on is called “broken,” the opposite is called “unbroken.” But one who wants to go in this way should attain the earth-kasiṇa [jhāna] and emerge. Then he should do the preliminary work, determining thus, “Let the water in such an area become earth,” and he should resolve in the way already described. Simultaneously with the resolve, the water in that place becomes earth. He goes on that.

                    \vismParagraph{XII.96}{96}{}
                    \marginnote{\textcolor{teal}{\footnotesize\{448|390\}}}{}Here is the text: “He is normally an obtainer of the earth-kasiṇa attainment. He adverts to water. Having adverted, he resolves with knowledge: ‘Let there be earth.’ There is earth. He goes on unbroken water. Just as men normally not possessed of supernormal power go on unbroken earth, so this possessor of supernormal power, by his attaining of mental mastery, goes on unbroken water as if on earth” (\textbf{\cite{Paṭis}II 208}).

                    \vismParagraph{XII.97}{97}{}
                    And he not only goes, but he adopts whatever posture he wishes. And not only earth, but whatever else [solid that] he wants such as gems, gold, rocks, trees, etc. he adverts to that and resolves, and it becomes as he resolves. And that water becomes earth only for him; it is water for anyone else. And fishes and turtles and water birds go about there as they like. But if he wishes to make it earth for other people, he does so too. When the time determined has elapsed, it becomes water again.

                    \vismParagraph{XII.98}{98}{}
                    \emph{Seated cross-legged he travels}: he goes seated cross-legged. \emph{Like a winged bird}: like a bird furnished with wings. One who wants to do this should attain the earth kasiṇa and emerge. \textcolor{brown}{\textit{[397]}} Then if he wants to go cross-legged, he should do the preliminary work and determine an area the size of a seat for sitting cross-legged on, and he should resolve in the way already described. If he wants to go lying down, he determines an area the size of a bed. If he wants to go on foot, he determines a suitable area the size of a path, and he resolves in the way already described: “Let it be earth.” Simultaneously with the resolve it becomes earth.

                    \vismParagraph{XII.99}{99}{}
                    Here is the text: “‘Seated cross-legged he travels in space like a winged bird’: he is normally an obtainer of the earth-kasiṇa attainment. He adverts to space. Having adverted, he resolves with knowledge: ‘Let there be earth.’ There is earth. He travels (walks), stands, sits, and lies down in space, in the sky. Just as men normally not possessed of supernormal power travel (walk), stand, sit, and lie down on earth, so this possessor of supernormal power, by his attaining of mental mastery, travels (walks), stands, sits, and lies down in space, in the sky” (\textbf{\cite{Paṭis}II 208}).

                    \vismParagraph{XII.100}{100}{}
                    And a bhikkhu who wants to travel in space should be an obtainer of the divine eye. Why? On the way there may be mountains, trees, etc., that are temperature-originated, or jealous nāgas, supaṇṇas, etc., may create them. He will need to be able to see these. But what should be done on seeing them? He should attain the basic jhāna and emerge, and then he should do the preliminary work thus, “Let there be space,” and resolve.

                    \vismParagraph{XII.101}{101}{}
                    But the Elder [Tipiṭaka Cūḷa-Abhaya] said: “Friends, what is the use of attaining the attainment? Is not his mind concentrated? Hence any area that he has resolved thus, ‘Let it be space’ is space.” Though he spoke thus, nevertheless the matter should be treated as described under the miracle of going unhindered through walls. Moreover, he should be an obtainer of the divine eye for the purpose of descending in a secluded place, for if he descends in a public place, in a bathing place, or at a village gate, he is exposed to the multitude. So, seeing with the divine eye, he should avoid a place where there is no open space and descend in an open space.

                    \vismParagraph{XII.102}{102}{}
                    \marginnote{\textcolor{teal}{\footnotesize\{449|391\}}}{}\emph{With his hand he touches and strokes the moon and sun so mighty and powerful}: here the “might” of the moon and sun should be understood to consist in the fact that they travel at an altitude of forty-two thousand leagues, and their “power” to consist in their simultaneous illuminating of three [of the four] continents. \textcolor{brown}{\textit{[398]}} Or they are “mighty” because they travel overhead and give light as they do, and they are “powerful” because of that same might. \emph{He touches}: he seizes, or he touches in one place. \emph{Strokes}: he strokes all over, as if it were the surface of a looking-glass.

                    \vismParagraph{XII.103}{103}{}
                    This supernormal power is successful simply through the jhāna that is made the basis for direct-knowledge; there is no special kasiṇa attainment here. For this is said in the Paṭisambhidā: “‘With his hand … so mighty and powerful’: here this possessor of supernormal power who has attained mind mastery … adverts to the moon and sun. Having adverted, he resolves with knowledge: ‘Let it be within hand’s reach.’ It is within hand’s reach. Sitting or lying down, with his hand he touches, makes contact with, strokes the moon and sun. Just as men normally not possessed of supernormal power touch, make contact with, stroke, some material object within hand’s reach, so this possessor of supernormal power, by his attaining of mental mastery, sitting or lying down, with his hands touches, makes contact with, strokes the moon and sun” (\textbf{\cite{Paṭis}II 298}).

                    \vismParagraph{XII.104}{104}{}
                    If he wants to go and touch them, he goes and touches them. But if he wants to touch them here sitting or lying down, he resolves: “Let them be within hand’s reach. Then he either touches them as they stand within hand’s reach when they have come by the power of the resolve like palmyra fruits loosened from their stalk, or he does so by enlarging his hand. But when he enlarges his hand, does he enlarge what is clung to or what is not clung to? He enlarges what is not clung to supported by what is clung to.

                    \vismParagraph{XII.105}{105}{}
                    Here the Elder Tipiṭaka Cūḷa-Nāga said: “But, friends, why does what is clung to not become small and big too? When a bhikkhu comes out through a keyhole, does not what is clung to become small? And when he makes his body big, does it not then become big, as in the case of the Elder Mahā Moggallāna?”

                    \vismParagraph{XII.106}{106}{}
                    At one time, it seems, when the householder Anāthapiṇḍika had heard the Blessed One preaching the Dhamma, he invited him thus, Venerable sir, take alms at our house together with five hundred bhikkhus,” and then he departed. The Blessed One consented. When the rest of that day and part of the night had passed, he surveyed the ten-thousandfold world element in the early morning. Then the royal nāga (serpent) called Nandopananda came within the range of his knowledge.

                    \vismParagraph{XII.107}{107}{}
                    The Blessed One considered him thus: “This royal nāga has come into the range of my knowledge. Has he the potentiality for development?” Then he saw that he had wrong view and no confidence in the Three Jewels. \textcolor{brown}{\textit{[399]}} He considered thus, “Who is there that can cure him of his wrong view?” He saw that the Elder Mahā Moggallāna could. Then when the night had turned to dawn, after he had seen to the needs of the body, he addressed the venerable Ānanda: “Ānanda, tell five hundred bhikkhus that the Perfect One is going on a visit to the gods.”

                    \vismParagraph{XII.108}{108}{}
                    \marginnote{\textcolor{teal}{\footnotesize\{450|392\}}}{}It was on that day that they had got a banqueting place ready for Nandopananda. He was sitting on a divine couch with a divine white parasol held aloft, surrounded by the three kinds of dancers\footnote{\vismAssertFootnoteCounter{18}\vismHypertarget{XII.n18}{}\textbf{\cite{Vism-mhṭ}(p. 394)}: “\emph{Vadhūkumārikaññā-vatthāhi tividhāhi nāṭakitthīhi}.”}and a retinue of nāgas, and surveying the various kinds of food and drink served up in divine vessels. Then the Blessed One so acted that the royal nāga saw him as he proceeded directly above his canopy in the direction of the divine world of the Thirty-three, accompanied by the five hundred bhikkhus.

                    \vismParagraph{XII.109}{109}{}
                    Then this evil view arose in Nandopananda the royal nāga: “There go these bald-headed monks in and out of the realm of the Thirty-three directly over my realm. I will not have them scattering the dirt off their feet on our heads.” He got up, and he went to the foot of Sineru. Changing his form, he surrounded it seven times with his coils. Then he spread his hood over the realm of the Thirty-three and made everything there invisible.

                    \vismParagraph{XII.110}{110}{}
                    The venerable Raṭṭhapāla said to the Blessed One: “Venerable sir, standing in this place formerly I used to see Sineru and the ramparts of Sineru,\footnote{\vismAssertFootnoteCounter{19}\vismHypertarget{XII.n19}{}“‘\emph{The ramparts of Sineru}’: the girdle of Sineru. There are, it seems, four ramparts that encircle Sineru, measuring 5,000 leagues in breadth and width. They were built to protect the realm of the Thirty-three against nāgas, garudas, kumbhaṇḍas and yakkhas. They enclose half of Sineru, it seems” (\textbf{\cite{Vism-mhṭ}394}).}and the Thirty-three, and the Vejayanta Palace, and the flag over the Vejayanta Palace. Venerable sir, what is the cause, what is the reason, why I now see neither Sineru nor … the flag over the Vejayanta Palace?”—“This royal nāga called Nandopananda is angry with us, Raṭṭhapāla. He has surrounded Sineru seven times with his coils, and he stands there covering us with his raised hood, making it dark.”—“I will tame him, venerable sir.” But the Blessed One would not allow it. Then the venerable Bhaddiya and the venerable Rāhula and all the bhikkhus in turn offered to do so, but the Blessed One would not allow it.

                    \vismParagraph{XII.111}{111}{}
                    Last of all the venerable Mahā Moggallāna said, “I will tame him, venerable sir.” The Blessed One allowed it, saying, “Tame him, Moggallāna.” The elder abandoned that form and assumed the form of a huge royal nāga, and he surrounded Nandopananda fourteen times with his coils and raised his hood above the other’s hood, and he squeezed him against Sineru. The royal nāga produced smoke. \textcolor{brown}{\textit{[400]}} The elder said, “There is smoke not only in your body but also in mine,” and he produced smoke. The royal nāga’s smoke did not distress the elder, but the elder’s smoke distressed the royal nāga. Then the royal nāga produced flames. The elder said, “There is fire not only in your body but also in mine,” and he produced flames. The royal nāga’s fire did not distress the elder, but the elder’s fire distressed the royal nāga.

                    \vismParagraph{XII.112}{112}{}
                    The royal nāga thought, “He has squeezed me against Sineru, and he has produced both smoke and flames.” Then he asked, “Sir, who are you?”—“I am Moggallāna, Nanda.”—“Venerable sir, resume your proper bhikkhu’s state.” The elder abandoned that form, and he went into his right ear and came out from his left ear; then he went into his left ear and came out from his right ear. Likewise he went \marginnote{\textcolor{teal}{\footnotesize\{451|393\}}}{}into his right nostril and came out from his left nostril; then he went into his left nostril and came out from his right nostril. Then the royal nāga opened his mouth. The elder went inside it, and he walked up and down, east and west, inside his belly.

                    \vismParagraph{XII.113}{113}{}
                    The Blessed One said, “Moggallāna, Moggallāna, beware; this is a mighty nāga.” The elder said, “Venerable sir, the four roads to power have been developed by me, repeatedly practiced, made the vehicle, made the basis, established, consolidated, and properly undertaken. I can tame not only Nandopananda, venerable sir, but a hundred, a thousand, a hundred thousand royal nāgas like Nandopananda.”

                    \vismParagraph{XII.114}{114}{}
                    The royal nāga thought, “When he went in the first place I did not see him. But now when he comes out I shall catch him between my fangs and chew him up.” Then he said, “Venerable sir, come out. Do not keep troubling me by walking up and down inside my belly.” The elder came out and stood outside. The royal nāga recognized him, and blew a blast from his nose. The elder attained the fourth jhāna, and the blast failed to move even a single hair on his body. The other bhikkhus would, it seems, have been able to perform all the miracles up to now, but at this point they could not have attained with so rapid a response, which is why the Blessed One would not allow them to tame the royal nāga.

                    \vismParagraph{XII.115}{115}{}
                    The royal nāga thought, “I have been unable to move even a single hair on this monk’s body with the blast from my nose. He is a mighty monk.” The elder abandoned that form, and having assumed the form of a supaṇṇa, he pursued the royal nāga demonstrating the supaṇṇa’s blast. \textcolor{brown}{\textit{[401]}} The royal nāga abandoned that form, and having assumed the form of a young brahman, he said, “Venerable sir, I go for refuge to you,” and he paid homage at the elder’s feet. The elder said, “The Master has come, Nanda; come, let us go to him.” So having tamed the royal nāga and deprived him of his poison, he went with him to the Blessed One’s presence.

                    \vismParagraph{XII.116}{116}{}
                    The royal nāga paid homage to the Blessed One and said, “Venerable sir, I go for refuge to you.” The Blessed One said, “May you be happy, royal nāga.” Then he went, followed by the Community of Bhikkhus, to Anāthapiṇḍika’s house. Anāthapiṇḍika said, “Venerable sir, why have you come so late?”—“There was a battle between Moggallāna and Nandopananda.”—“Who won, venerable sir? Who was defeated?”—“Moggallāna won; Nanda was defeated.” Anāthapiṇḍika said, “Venerable sir, let the Blessed One consent to my providing meals for seven days in a single series, and to my honouring the elder for seven days.” Then for seven days he accorded great honour to the five hundred bhikkhus with the Enlightened One at their head.

                    \vismParagraph{XII.117}{117}{}
                    So it was with reference to this enlarged form created during this taming of Nandopananda that it was said: “When he makes his body big, does it not then become big, as in the case of the Elder Mahā Moggallāna?” (\hyperlink{XII.105}{§105}{}). Although this was said, the bhikkhus observed, “He enlarges only what is not clung to supported by what is clung to.” And only this is correct here.\footnote{\vismAssertFootnoteCounter{20}\vismHypertarget{XII.n20}{}“Only this is correct because instances of clung-to (kammically acquired) materiality do not arise owing to consciousness or to temperature. Or alternatively, ‘clung-to’ is intended as all matter that is bound up with faculties (i.e. ‘sentient’), too. And so to take it as enlargement of that is likewise not correct. Consequently, enlargement should be understood only in the way stated. Though the clung-to and the unclung-to occur, as it were, mixed up in a single continuity, they are nevertheless not mixed up in meaning. Herein, just as when a pint measure (\emph{āḷhaka}) of milk is poured into a number of pints of water, though the milk becomes completely mixed up with the water, and is present appreciably in all, it is nevertheless not the milk that has increased there, but only the water. And so too, although the clung-to and unclung-to occur mixed up together, it is nevertheless not the clung-to that is enlarged. It should be taken that it is the consciousness-born matter that is enlarged by the influence of the supernormal power, and the temperature-born is enlarged \emph{pari passu}” (\textbf{\cite{Vism-mhṭ}395}).}

                    \vismParagraph{XII.118}{118}{}
                    \marginnote{\textcolor{teal}{\footnotesize\{452|394\}}}{}And when he has done this, he not only touches the moon and sun, but if he wishes, he makes a footstool [of them] and puts his feet on it, he makes a chair [of them] and sits on it, he makes a bed [of them] and lies on it, he makes a leaning-plank [of them] and leans on it. And as one does, so do others. For even when several hundred thousand bhikkhus do this and each one succeeds, still the motions of the moon and sun and their radiance remain the same. For just as when a thousand saucers are full of water and moon disks are seen in all the saucers, still the moon’s motion is normal and so is its radiance. And this miracle resembles that.

                    \vismParagraph{XII.119}{119}{}
                    \emph{Even as far as the Brahmā-world}: having made even the Brahmā-world the limit. \emph{He wields bodily mastery}: herein, he wields self-mastery in the Brahmā-world by means of the body. The meaning of this should be understood according to the text.

                    Here is the text: “‘He wields bodily mastery even as far as the Brahmā-world’: if this possessor of supernormal power, having reached mental mastery, wants to go to the Brahmā-world, though far, he resolves upon nearness, ‘Let it be near.’ \textcolor{brown}{\textit{[402]}} It is near. Though near, he resolves upon farness, ‘Let it be far.’ It is far. Though many, he resolves upon few, ‘Let there be few.’ There are few. Though few, he resolves upon many, ‘Let there be many.’ There are many. With the divine eye he sees the [fine-material] visible form of that Brahmā. With the divine ear element he hears the voice of that Brahmā. With the knowledge of penetration of minds he understands that Brahmā’s mind. If this possessor of supernormal power, having reached mental mastery, wants to go to the Brahmā-world with a visible body, he converts his mind to accord with his body, he resolves his mind to accord with his body. Having converted his mind to accord with his body, resolved his mind to accord with his body, he arrives at blissful (easy) perception and light (quick) perception, and he goes to the Brahmā-world with a visible body. If this possessor of supernormal power, having reached mental mastery, wants to go to the Brahmā-world with an invisible body, he converts his body to accord with his mind, he resolves his body to accord with his mind. Having converted his body to accord with his mind, resolved his body to accord with his mind, he arrives at blissful (easy) perception and light (quick) perception, and he goes to the Brahmā-world with an invisible body. He creates a [fine-material] visible form before that Brahmā, mind-made with all its limbs, lacking no faculty. If that possessor of supernormal power walks up and down, \marginnote{\textcolor{teal}{\footnotesize\{453|395\}}}{}the creation walks up and down there too. If that possessor of supernormal power stands … sits … lies down, the creation lies down there too. If that possessor of supernormal power produces smoke … produces flames … preaches Dhamma … asks a question … being asked a question, answers, the creation, being asked a question, answers there too. If that possessor of supernormal power stands with that Brahmā, converses, enters into communication with that Brahmā, the creation stands with that Brahmā there too, converses, enters into communication with that Brahmā there too. Whatever that possessor of supernormal power does, the creation does the same thing’” (\textbf{\cite{Paṭis}II 209}).

                    \vismParagraph{XII.120}{120}{}
                    Herein, \emph{though far, he resolves upon nearness}: having emerged from the basic jhāna, he adverts to a far-off world of the gods or to the Brahmā-world thus, “Let it be near.” Having adverted and done the preliminary work, he attains again, and then resolves with knowledge: “Let it be near.” It becomes near. The same method of explanation applies to the other clauses too.

                    \vismParagraph{XII.121}{121}{}
                    Herein, who has taken what was far and made it near? The Blessed One. For when the Blessed One was going to the divine world after the Twin Miracle, he made Yugandhara and Sineru near, and from the earth’s surface he set one foot \textcolor{brown}{\textit{[403]}} on Yugandhara, and then he set the other on the summit of Sineru.

                    \vismParagraph{XII.122}{122}{}
                    Who else has done it? The Elder Mahā Moggallāna. For when the elder was leaving Sāvatthī after completing his meal, he abridged the twelve-league crowd and the thirty-league road to the city of Saṅkassa, and he arrived at the same moment.

                    \vismParagraph{XII.123}{123}{}
                    Furthermore, the Elder Cūḷa Samudda did it as well in Tambapaṇṇi Island. During a time of scarcity, it seems, seven hundred bhikkhus came to the elder one morning. The elder thought, “Where can a large community of bhikkhus wander for alms?” He saw nowhere at all in Tambapaṇṇi Island, but he saw that it would be possible on the other shore at Pāṭaliputta (Patna). He got the bhikkhus to take their bowls and [outer] robes, and he said, “Come friends, let us go wandering for alms.” Then he abridged the earth and went to Pāṭaliputta. The bhikkhus asked, “What is the city, venerable sir?”—“It is Pāṭaliputta, friends.”—“Pāṭaliputta is far away, venerable sir.”—“Friends, experienced elders make what is far near.”—“Where is the ocean (\emph{mahā-samudda}), venerable sir?”—“Friends, did you not cross a blue stream on the way as you came?”—“Yes, venerable sir, but the ocean is vast.”—“Friends, experienced elders also make what is vast small.”

                    \vismParagraph{XII.124}{124}{}
                    And the Elder Tissadatta did likewise, when he had put on his upper robes after bathing in the evening, and the thought of paying homage at the Great Enlightenment Tree arose in him.

                    \vismParagraph{XII.125}{125}{}
                    Who has taken what was near and made it far? The Blessed One. For although Aṅgulimāla was near to the Blessed One, yet he made him far (see \textbf{\cite{M}II 99}).

                    \vismParagraph{XII.126}{126}{}
                    Who has made much little? The Elder Mahā Kassapa. One feast day at Rājagaha, it seems, there were five hundred girls on their way to enjoy the festival, and they had taken moon cakes with them. They saw the Blessed One but gave \marginnote{\textcolor{teal}{\footnotesize\{454|396\}}}{}him nothing. On their way back, however, they saw the elder. Thinking, “He is our elder,” they each took a cake and approached the elder. The elder took out his bowl and made a single bowlful of them all. The Blessed One had sat down first to await the elder. The elder brought them and gave them to the Blessed One.

                    \vismParagraph{XII.127}{127}{}
                    In the story of the rich man Illīsa, however, (\textbf{\cite{J-a}I 348}; \textbf{\cite{Dhp-a}I 372}) the Elder Mahā Moggallāna made little much. And in the story of Kākavaḷiya the Blessed One did so. The Elder Mahā Kassapa, it seems, after spending seven days in attainment, stood at the house door of a man in poor circumstances called Kākavaḷiya in order to show favour to the poor. \textcolor{brown}{\textit{[404]}} His wife saw the elder, and she poured into his bowl the unsalted sour gruel that she had cooked for her husband. The elder took it and placed it in the Blessed One’s hand. The Blessed One resolved to make it enough for the Greater Community of Bhikkhus. What was brought in a single bowl became enough for all. And on the seventh day Kākavaḷiya became a rich man.

                    \vismParagraph{XII.128}{128}{}
                    And not only in the case of making little much, but whatever the possessor of supernormal power wishes, whether to make the sweet unsweet, etc., it is successful for him. For so it was that when the Elder Mahā Anula saw many bhikkhus sitting on the banks of the Gaṅgā River [in Sri Lanka] eating plain rice, which was all that they had got after doing their alms round, he resolved, “Let the Gaṅgā River water be cream of ghee,” and he gave a sign to the novices. They fetched it in their vessels and gave it to the Community of Bhikkhus. All of them ate their meal with sweet cream of ghee.

                    \vismParagraph{XII.129}{129}{}
                    \emph{With the divine eye}: remaining here and extending light, he sees the visible form of that Brahmā. And remaining here he also hears the sound of his speech and he understands his mind.

                    \vismParagraph{XII.130}{130}{}
                    \emph{He converts his mind according to his body}: he converts the mind to accord with the material body; taking the consciousness of the basic jhāna, he mounts it upon the body, he makes its going slow to coincide with that of the body; for the body’s mode of going is slow.

                    \vismParagraph{XII.131}{131}{}
                    \emph{He arrives at blissful perception and light perception}: he arrives at, enters, makes contact with, reaches, the perception of bliss and perception of lightness that are conascent with the consciousness whose object is the basic jhāna. And it is perception associated with equanimity that is called “perception of bliss”; for equanimity is called “bliss” since it is peaceful. And that same perception should be understood to be called “perception of lightness” too because it is liberated from hindrances and from the things that oppose it beginning with applied thought. But when he arrives at that state, his physical body too becomes as light as a tuft of cotton. He goes to the Brahmā-world thus with a visible body as light as a tuft of cotton wafted by the wind.

                    \vismParagraph{XII.132}{132}{}
                    As he goes thus, if he wishes, he creates a path in space by means of the earth kasiṇa and goes on foot. If he wishes, he resolves by means of the air kasiṇa that there shall be air, and he goes by air like a tuft of cotton. Moreover, the desire to go is the measure here. When there is the desire to go, one who has made his mental resolve in this way goes visibly, carried by the force of the resolution like an arrow shot by an archer. \textcolor{brown}{\textit{[405]}}

                    \vismParagraph{XII.133}{133}{}
                    \marginnote{\textcolor{teal}{\footnotesize\{455|397\}}}{}\emph{He converts his body to accord with his mind}: he takes the body and mounts it on the mind. He makes its going swift to coincide with that of the mind; for the mind’s mode of going is swift.

                    \emph{He arrives at blissful perception and light perception}: he arrives at perception of bliss and perception of lightness that are conascent with the supernormal-power consciousness whose object is the material body. The rest should be understood in the way already described. But here there is only the going of consciousness.\footnote{\vismAssertFootnoteCounter{21}\vismHypertarget{XII.n21}{}“‘\emph{There is only the going of consciousness}’: there is only a going that is the same as that of the mind. But how does the body, whose going [being that of matter] is slow, come to have the same going as the mind, which quickly passes? Its going is not the same in all respects; for in the case of converting the mind to conform with the body, the mind does not come to have the same going as the body in all respects. For it is not that the mind then occurs with the moment of a material state, which passes slowly, instead of passing with its own kind of moment, which is what establishes its individual essence. But rather the mind is called ‘converted to accord with the going of the body’ as long as it goes on occurring in a continuity that conforms with the body until the desired place is arrived at. This is because its passing occurs parallel with that of the body, whose going is slow, owing to the resolution, ‘Let the mind be like this body.’ And likewise, it is while the body keeps occurring in suchwise that its arrival at the desired place comes about in only a few quick passes of the mind instead of passing slowly, as in those who have not developed the roads to power—and this mode of occurrence is due to the possession of the perception of lightness, to say nothing of the resolve, ‘Let this body be like this mind’—that the body is called ‘converted to accord with the going of the mind,’ not because it arrives at the desired place in a single consciousness moment. And when taken thus the simile, ‘Just as a strong man might stretch out his bent arm, or bend his outstretched arm’ (\textbf{\cite{Vin}I 5}) can be taken literally. And this must be accepted in this way without reserve, otherwise there is conflict with the Suttas, the Abhidhamma and the Commentary, as well as contradiction of natural law (\emph{dhammatā}). ‘Bhikkhus, I see no other one thing that is so quickly transformed as the mind’ (\textbf{\cite{A}I 10})—here it is material states that are referred to by the word ‘other’ because they do not pass quickly. And in the Abhidhamma only matter is called prenascence condition and only consciousness postnascence condition. And wherever states (\emph{dhamma}) arise, there they dissolve. There is no transmigration to an adjacent location (\emph{desantara-saṅkamana}), nor does the individual essence become other. For it is not possible to effect any alteration of the characteristics of dhammas by force of the roads to power. But it is possible to effect alteration of the mode in which they are present (\emph{bhāva})” (\textbf{\cite{Vism-mhṭ}397}).}

                    \vismParagraph{XII.134}{134}{}
                    When it was asked, “As he goes with an invisible body thus, does he go at the moment of the resolution-consciousness’s arising or at the moment of its presence or at the moment of its dissolution?”, an elder replied, “He goes in all three moments.”—“But does he go himself, or does he send a creation?”—“He does as he pleases. But here it is only the going himself that has been given [in the text].”

                    \vismParagraph{XII.135}{135}{}
                    \emph{Mind-made}: mind-made because created by the mind in resolution. \emph{Lacking no faculty}: this refers to the shape of the eye, ear, etc.; but there is no sensitivity in \marginnote{\textcolor{teal}{\footnotesize\{456|398\}}}{}a created visible form.\footnote{\vismAssertFootnoteCounter{22}\vismHypertarget{XII.n22}{}“This should be regarded as implying that there is no sex or life faculty in it either.” (\textbf{\cite{Vism-mhṭ}398}).} \emph{If the possessor of supernormal power walks up and down, the creation walks up and down there too}, etc., all refers to what a disciple creates; but what the Blessed One creates does whatever the Blessed One does, and it also does other things according to the Blessed One’s pleasure.

                    \vismParagraph{XII.136}{136}{}
                    When this possessor of supernormal power, while remaining here sees a visible object with the divine eye, hears a sound with the divine ear element, knows consciousness with the penetration of minds, he does not wield bodily power in doing that. And when, while remaining here, he stands with that Brahmā, converses, enters into communication with that Brahmā, he does not wield bodily power in doing that. And when he makes his resolve described in the way beginning “though far, he resolves upon nearness,” he does not wield bodily power in doing that. And when he goes to the Brahmā-world with a visible or an invisible body, he does not wield bodily power in doing that. But when he enters upon the process described in the way beginning, “He creates a visible form before that Brahmā, mind-made,” then he wields bodily power in doing that. The rest, however, is said here for the purpose of showing the stage prior to the wielding of the bodily power. This, firstly, is (i) success by resolve (\hyperlink{XII.45}{§45}{}).

                    \vismParagraph{XII.137}{137}{}
                    The difference between (i) success as transformation and (ii) success as the mind-made [body], is as follows (see \hyperlink{XII.22}{§22}{}, 24, 25, 45).
                \subsection[\vismAlignedParas{§137–138}Supernormal power as transformation]{Supernormal power as transformation}

                    (i) One, firstly, who performs a transformation \textcolor{brown}{\textit{[406]}} should resolve upon whatever he chooses from among the things beginning with the appearance of a boy, described as follows: “He abandons his normal appearance and shows the appearance of a boy or the appearance of a nāga (serpent), or the appearance of a supaṇṇa (winged demon), or the appearance of an asura (demon), or the appearance of the Ruler [of Gods] (Indra), or the appearance of some [other sensual-sphere] deity, or the appearance of a Brahmā, or the appearance of the sea, or the appearance of a rock, or the appearance of a lion, or the appearance of a tiger, or the appearance of a leopard, or he shows an elephant, or he shows a horse, or he shows a chariot, or he shows a foot soldier, or he shows a manifold military array” (\textbf{\cite{Paṭis}II 210}).

                    \vismParagraph{XII.138}{138}{}
                    And when he resolves he should emerge from the fourth jhāna that is the basis for direct-knowledge and has one of the things beginning with the earth kasiṇa as its object, and he should advert to his own appearance as a boy. After adverting and finishing the preliminary work, he should attain again and emerge, and he should resolve thus: “Let me be a boy of such and such a type.” Simultaneously with the resolve consciousness he becomes the boy, just as Devadatta did (\textbf{\cite{Vin}I 185}; \textbf{\cite{Dhp-a}I 139}). This is the method in all instances. But \emph{he shows an elephant}, etc., is said here with respect to showing an elephant, etc., externally. Herein, instead of resolving, “Let me be an elephant,” he resolves, “Let there be an elephant.” The same method applies in the case of the horse and the rest.

                    \marginnote{\textcolor{teal}{\footnotesize\{457|399\}}}{}This is success as transformation.
                \subsection[\vismAlignedParas{§139}Supernormal power as mind-made body]{Supernormal power as mind-made body}

                    \vismParagraph{XII.139}{139}{}
                    (ii) One who wants to make the mind-made [body] should emerge from the basic jhāna and first advert to the body in the way already described, and then he should resolve, “Let it be hollow.” It becomes hollow. Then he adverts to another body inside it, and having done the preliminary work in the way already described, he resolves, “Let there be another body inside it.” Then he draws it out like a reed from its sheath, like a sword from its scabbard, like a snake from its slough. Hence it is said: “Here a bhikkhu creates from this body another body possessing visible form, mind-made, with all its limbs, lacking no faculty. Just as though a man pulled out a reed from its sheath and thought thus: ‘This is the sheath; this is the reed; the sheath is one, the reed is another, it was from the sheath that the reed was pulled out’” (\textbf{\cite{Paṭis}II 210}), and so on. And here, just as the reed, etc., are similar to the sheath, etc., so too the mind-made visible form is similar to the possessor of supernormal power, and this simile is given in order to show that.

                    This is success as the mind-made [body].

                    The twelfth chapter called “The Description of the Supernormal Powers” in the \emph{Path of Purification }composed for the purpose of gladdening good people.
        \chapter[Other Direct-knowledges]{Other Direct-knowledges\vismHypertarget{XIII}\newline{\textnormal{\emph{Abhiññā-niddesa}}}}
            \label{XIII}

            \section[\vismAlignedParas{§1–7}(2) The Divine Ear Element]{(2) The Divine Ear Element}

                \vismParagraph{XIII.1}{1}{}
                \marginnote{\textcolor{teal}{\footnotesize\{458|400\}}}{}\textcolor{brown}{\textit{[407]}} It is now the turn for the description of the divine ear element. Herein, and also in the case of the remaining three kinds of direct-knowledge, the meaning of the passage beginning, “When his concentrated mind …” (\textbf{\cite{D}I 79}) should be understood in the way already stated (\hyperlink{XII.13}{XII.13f.}{}); and in each case we shall only comment on what is different. [The text is as follows: “He directs, he inclines, his mind to the divine ear element. With the divine ear element, which is purified and surpasses the human, he hears both kinds of sounds, the divine and the human, those that are far as well as near”(\textbf{\cite{D}I 79}).]

                \vismParagraph{XIII.2}{2}{}
                Herein, \emph{with the divine ear element}: it is \emph{divine} here because of its similarity to the divine; for deities have as the divine ear element the sensitivity that is produced by kamma consisting in good conduct and is unimpeded by bile, phlegm, blood, etc., and capable of receiving an object even though far off because it is liberated from imperfections. And this ear element consisting in knowledge, which is produced by the power of this bhikkhu’s energy in development, is similar to that, so it is “divine” because it is similar to the divine. Furthermore, it is “divine” because it is obtained by means of divine abiding and because it has divine abiding as its support. And it is an “ear element” (\emph{sota-dhātu}) in the sense of hearing (\emph{savana}) and in the sense of being a soulless [element]. Also it is an “ear element” because it is like the ear element in its performance of an ear element’s function. With that divine ear element … he hears …

                \emph{Which is purified}: which is quite pure through having no imperfection. \emph{And surpasses the human}: which in the hearing of sounds surpasses, stands beyond, the human ear element by surpassing the human environment.

                \vismParagraph{XIII.3}{3}{}
                \emph{He hears both kinds of sounds}: he hears the two kinds of sounds. What two? The divine and the human: the sounds of deities and of human beings, is what is meant. This should be understood as partially inclusive. \emph{Those that are far as well as near}: what is meant is that he hears sounds that are far off, even in another world-sphere, and those that are near, even the sounds of the creatures living in his own body. This should be understood as completely inclusive.

                \vismParagraph{XIII.4}{4}{}
                But how is this [divine ear element] aroused? The bhikkhu \textcolor{brown}{\textit{[408]}} should attain jhāna as basis for direct-knowledge and emerge. Then, with the \marginnote{\textcolor{teal}{\footnotesize\{459|401\}}}{}consciousness belonging to the preliminary-work concentration,\footnote{\vismAssertFootnoteCounter{1}\vismHypertarget{XIII.n1}{}“With the consciousness belonging to the particular concentration that constitutes the preliminary work. The meaning is: by means of consciousness concentrated with the momentary concentration that occurs in the form of the preliminary work for knowledge of the divine ear element. The occasion of access for the divine ear element is called preliminary-work consciousness, but that as stated refers to multiple advertings” (\textbf{\cite{Vism-mhṭ}401}).} he should advert first to the gross sounds in the distance normally within range of hearing: the sound in the forest of lions, etc., or in the monastery the sound of a gong, the sound of a drum, the sound of a conch, the sound of recitation by novices and young bhikkhus reciting with full vigour, the sound of their ordinary talk such as “What, venerable sir?”, “What, friend?”, etc., the sound of birds, the sound of wind, the sound of footsteps, the fizzing sound of boiling water, the sound of palm leaves drying in the sun, the sound of ants, and so on. Beginning in this way with quite gross sounds, he should successively advert to more and more subtle sounds. He should give attention to the sound sign of the sounds in the eastern direction, in the western direction, in the northern direction, in the southern direction, in the upper direction, in the lower direction, in the eastern intermediate direction, in the western intermediate direction, in the northern intermediate direction, and in the southern intermediate direction. He should give attention to the sound sign of gross and of subtle sounds.\footnote{\vismAssertFootnoteCounter{2}\vismHypertarget{XIII.n2}{}“The sound sign is the sound itself since it is the cause for the arising of the knowledge. Or the gross-subtle aspect acquired in the way stated is the sound sign” (\textbf{\cite{Vism-mhṭ}402}).}

                \vismParagraph{XIII.5}{5}{}
                These sounds are evident even to his normal consciousness; but they are especially evident to his preliminary-work-concentration consciousness. \footnote{\vismAssertFootnoteCounter{3}\vismHypertarget{XIII.n3}{}“This is momentary-concentration consciousness, which owing to the fact that the preliminary work contingent upon the sound has been performed, occurs in one who has attained the basic jhāna and emerged for the purpose of arousing the divine ear element” (\textbf{\cite{Vism-mhṭ}402}).} As he gives his attention to the sound sign in this way, [thinking] “Now the divine ear element will arise,” mind-door adverting arises making one of these sounds its object. When that has ceased, then either four or five impulsions impel, the first three, or four, of which are of the sense sphere and are called preliminary-work, access, conformity, and change-of-lineage, while the fourth, or the fifth, is fine-material-sphere absorption consciousness belonging to the fourth jhāna.

                \vismParagraph{XIII.6}{6}{}
                Herein, it is knowledge arisen together with the absorption consciousness that is called the divine ear element. After that [absorption has been reached, the divine ear element] becomes merged in that ear [of knowledge]. \footnote{\vismAssertFootnoteCounter{4}\vismHypertarget{XIII.n4}{}“‘\emph{Becomes merged}’ is amalgamated with the divine ear element. He is called an obtainer of divine-ear knowledge as soon as the absorption consciousness has arisen. The meaning is that there is now no further need of development for the purpose” (\textbf{\cite{Vism-mhṭ}403}).} When consolidating it, he should extend it by delimiting a single finger-breadth thus, “I will hear sounds within this area,” then two finger-breadths, four finger-breadths, eight finger-breadths, a span, a\emph{ ratana} (= 24 finger-breadths), the interior \marginnote{\textcolor{teal}{\footnotesize\{460|402\}}}{}of the room, the veranda, the building, the surrounding walk, the park belonging to the community, the alms-resort village, the district, and so on up to the [limit of the] world-sphere, or even more. This is how he should extend it by delimited stages.

                \vismParagraph{XIII.7}{7}{}
                One who has reached direct-knowledge in this way hears also by means of direct-knowledge without re-entering the basic jhāna any sound that has come within the space touched by the basic jhāna’s object. And in hearing in this way, even if there is an uproar with sounds of conches, drums, cymbals, etc., right up to the Brahmā-world \textcolor{brown}{\textit{[409]}} he can, if he wants to, still define each one thus, “This is the sound of conches, this is the sound of drums.”

                The explanation of the divine ear element is ended.
            \section[\vismAlignedParas{§8–12}(3) Penetration of Minds]{(3) Penetration of Minds}

                \vismParagraph{XIII.8}{8}{}
                As to the explanation of knowledge of penetration of minds, [the text is as follows: “He directs, he inclines, his mind to the knowledge of penetration of minds. He penetrates with his mind the minds of other beings, of other persons, and understands them thus: he understands [the manner of] consciousness affected by greed as affected by greed, and understands [the manner of] consciousness unaffected by greed as unaffected by greed; he understands consciousness affected by hate as affected by hate, and consciousness unaffected by hate as unaffected by hate; he understands consciousness affected by delusion as affected by delusion, and consciousness unaffected by delusion as unaffected by delusion; he understands cramped consciousness as cramped, and distracted consciousness as distracted; he understands exalted consciousness as exalted, and unexalted consciousness as unexalted; he understands surpassed consciousness as surpassed and unsurpassed consciousness as unsurpassed; he understands concentrated consciousness as concentrated and unconcentrated consciousness as unconcentrated; he understands the liberated [manner of] consciousness as liberated, and the unliberated [manner of] consciousness as unliberated” (\textbf{\cite{D}I 79}). Here, it goes all round (\emph{pariyāti}), thus it is penetration (\emph{pariya}); the meaning is that it delimits (\emph{paricchindati}). The penetration of the heart (\emph{cetaso pariyaṃ}) is “penetration of minds” (\emph{cetopariya}). It is penetration of hearts and that is knowledge, thus it is knowledge of penetration of minds (\emph{cetopariyañāṇa}). [He directs his consciousness] to that, is what is meant.

                \emph{Of other beings}: of the rest of beings, himself excluded. \emph{Of other persons}: this has the same meaning as the last, the wording being varied to suit those susceptible of teaching [in another way], and for the sake of elegance of exposition. \emph{With his mind the minds}: with his [manner of] consciousness the [manner of] consciousness of other beings. \emph{Having penetrated} (\emph{paricca}): having delimited all round. \emph{He understands}: he understands them to be of various sorts beginning with that affected by greed.

                \vismParagraph{XIII.9}{9}{}
                But how is this knowledge to be aroused? That is successfully done through the divine eye, which constitutes its preliminary work. Therefore the bhikkhu should extend light, and he should seek out (\emph{pariyesitabba}) another’s [manner of] consciousness by keeping under observation with the divine eye the colour \marginnote{\textcolor{teal}{\footnotesize\{461|403\}}}{}of the blood present with the matter of the physical heart as its support.\footnote{\vismAssertFootnoteCounter{5}\vismHypertarget{XIII.n5}{}The “matter of the heart” is not the heart-basis, but rather it is the heart as the piece of flesh described as resembling a lotus bud in shape outside and like a \emph{kosātakī }fruit inside (\hyperlink{VIII.111}{VIII.111}{}). For the blood mentioned here is to be found with that as its support. But the heart-basis occurs with this blood as its support” (\textbf{\cite{Vism-mhṭ}403}).} For when [a manner of] consciousness accompanied by joy is present, the blood is red like a banyan-fig fruit; when [a manner of] consciousness accompanied by grief is present, it is blackish like a rose-apple fruit; when [a manner of] consciousness accompanied by serenity is present, it is clear like sesame oil. So he should seek out another’s [manner of] consciousness by keeping under observation the colour of the blood in the physical heart thus, “This matter is originated by the joy faculty; this is originated by the grief faculty; this is originated by the equanimity faculty,” and so consolidate his knowledge of penetration of hearts.

                \vismParagraph{XIII.10}{10}{}
                It is when it has been consolidated in this way that he can gradually get to understand not only all manner of sense-sphere consciousness but those of fine-material and immaterial consciousness as well by tracing one [manner of] consciousness from another without any more seeing the physical heart’s matter. For this is said in the Commentary: “When he wants to know another’s [manner of] consciousness in the immaterial modes, whose physical-heart matter can he observe? Whose material alteration [originated] by the faculties can he look at? No one’s. The province of a possessor of supernormal power is [simply] this, namely, wherever the [manner of] consciousness he adverts to is, there he knows it according to these sixteen classes.” But this explanation [by means of the physical heart] is for one who has not [yet] done any interpreting.\footnote{\vismAssertFootnoteCounter{6}\vismHypertarget{XIII.n6}{}“Of one who has not done any interpreting (\emph{abhinivesa}) reckoned as study for direct-knowledge” (\textbf{\cite{Vism-mhṭ}407}). A rather special use of the word \emph{abhinivesa}, perhaps more freely renderable here as “practice.”}

                \vismParagraph{XIII.11}{11}{}
                As regards [\emph{the manner of] consciousness affected by greed}, etc., the eight [manners of] consciousness accompanied by greed (see Table III, nos. (22)–(29)) \textcolor{brown}{\textit{[410]}} should be understood as [\emph{the manner of}] consciousness affected by greed. The remaining profitable and indeterminate [manners of] consciousness in the four planes are \emph{unaffected by greed}. The four, namely, the two consciousnesses accompanied by grief (nos. (30) and (31)), and the two consciousnesses [accompanied respectively by] uncertainty (32) and agitation (33) are not included in this dyad, though some elders include them too. It is the two consciousnesses accompanied by grief that are called \emph{consciousness affected by hate}. And all profitable and indeterminate consciousnesses in the four planes are unaffected by hate. The remaining ten kinds of unprofitable consciousnesses (nos. (22)–(29) and (32) and (33)) are not included in this dyad, though some elders include them too. \emph{Affected by delusion … unaffected by delusion}: here only the two, namely, that accompanied by uncertainty and that accompanied by agitation, are affected by delusion alone [without being accompanied by the other two unprofitable roots]. But [all] the twelve kinds of unprofitable consciousnesses (nos. (22)–(33)) can also be understood as [the manner of] \marginnote{\textcolor{teal}{\footnotesize\{462|404\}}}{}consciousness affected by delusion since delusion is present in all kinds of unprofitable consciousnesses. The rest are \emph{unaffected by delusion}.

                \vismParagraph{XIII.12}{12}{}
                \emph{Cramped} is that attended by stiffness and torpor. Distracted is that attended by agitation. \emph{Exalted} is that of the fine-material and immaterial spheres. \emph{Unexalted }is the rest. \emph{Surpassed} is all that in the three [mundane] planes. \emph{Unsurpassed} is the supramundane. \emph{Concentrated} is that attained to access and that attained to absorption. \emph{Unconcentrated} is that not attained to either. \emph{Liberated} is that attained to any [of the five kinds of] deliverance, that is to say, deliverance by substitution of opposites [through insight], by suppression [through concentration], by cutting off [by means of the path], by tranquillization [by means of fruition], and by renunciation [as Nibbāna] (see \textbf{\cite{Paṭis}I 26} under “abandoning”). \emph{Unliberated }is that which has not attained to any of the five kinds of liberation.

                So the bhikkhu who has acquired the knowledge of penetration of hearts understands all these [manners of consciousness, namely, the manner of] consciousness affected by greed as affected by greed … [the unliberated manner of] consciousness as unliberated.
            \section[\vismAlignedParas{§13–71}(4) Recollection of Past Lives]{(4) Recollection of Past Lives}

                \vismParagraph{XIII.13}{13}{}
                As to the explanation of knowledge of recollection of past lives, [the text is as follows:] He directs, he inclines, his mind to the knowledge of recollection of past lives. He recollects his manifold past lives, that is to say, one birth, two births, three births, four births, five births, ten births, twenty births, thirty births, forty births, fifty births, a hundred births, a thousand births, a hundred thousand births, many eons of world contraction, many eons of world expansion: many eons of world contraction and expansion: “There I was so named, of such a race, with such an appearance, such was my food, such my experience of pleasure and pain, such the end of my life span; and passing away from there, I reappeared elsewhere; and there too I was so named, of such a race, with such an appearance, such was my food, such my experience of pleasure and pain, such the end of my life span; and passing away from there, I reappeared here.” Thus with its aspects and particulars he recollects his manifold past lives” (\textbf{\cite{D}I 81}). [Herein,] to \emph{the knowledge of recollection of past lives} [means] for knowledge concerning recollection of past lives. \emph{Past lives} is aggregates lived in the past in former births. “Lived” [in that case means] lived out, undergone, arisen and ceased in one’s own [subjective] continuity. Or alternatively, [past lives] is mental objects lived [in the past in one’s former births]; and “lived” in that case means lived by living in one’s [objective] resort, which has been cognized and delimited by one’s own consciousness, or cognized by another’s consciousness, too. In the case of recollection of those [past Enlightened Ones] who have broken the cycle, and so on,\footnote{\vismAssertFootnoteCounter{7}\vismHypertarget{XIII.n7}{}For the term \emph{chinna-vaṭumaka} (“one who has broken the cycle of rebirths”) as an epithet of former Buddhas, see \textbf{\cite{M}III 118}.} these last are only accessible to Enlightened Ones. \emph{Recollection of past lives}: the mindfulness (memory) by means of which he recollects the past lives is the recollection of past lives. Knowledge is the knowledge associated with that mindfulness. \textcolor{brown}{\textit{[411]}} \emph{To the knowledge of recollection of past lives}: for the purpose of \marginnote{\textcolor{teal}{\footnotesize\{463|405\}}}{}the knowledge of the recollection of past lives in this way; for the attaining, for the reaching, of that knowledge, is what is meant.

                \vismParagraph{XIII.14}{14}{}
                \emph{Manifold}: of many kinds: or that has occurred in many ways. Given in detail, is the meaning.\footnote{\vismAssertFootnoteCounter{8}\vismHypertarget{XIII.n8}{}\emph{Saṃvaṇṇita—}“given in detail”; \textbf{\cite{Vism-mhṭ}} glosses by \emph{vitthāritan ti attho}. Not in this meaning in PED. See prologue verses to the four Nikāyas.} \emph{Past lives }is the continuity lived here and there, taking the immediately previous existence as the beginning [and working backwards]. \emph{He recollects}: he recalls it, following it out by the succession of aggregates, or by death and rebirth-linking.

                \vismParagraph{XIII.15}{15}{}
                There are six kinds of people who recollect these past lives. They are: other sectarians, ordinary disciples, great disciples, chief disciples, Paccekabuddhas, and Buddhas.

                \vismParagraph{XIII.16}{16}{}
                Herein, other sectarians recollect only as far back as forty eons, but not beyond that. Why? Because their understanding is weak for lack of delimitation of mind and matter (see \hyperlink{XVIII}{Ch. XVIII}{}). Ordinary disciples recollect as far back as a hundred eons and as far back as a thousand eons because their understanding is strong. The eighty great disciples recollect as far back as a hundred thousand eons. The two chief disciples recollect as far back as an incalculable age and a hundred thousand eons. Paccekabuddhas recollect as far back as two incalculable ages and a hundred thousand eons. For such is the extent to which they can convey [their minds back respectively]. But there is no limit in the case of Buddhas.

                \vismParagraph{XIII.17}{17}{}
                Again, other sectarians only recollect the succession of aggregates; they are unable to recollect according [only] to death and rebirth-linking, letting go of the succession of aggregates. They are like the blind in that they are unable to descend upon any place they choose; they go as the blind do without letting go of their sticks. So they recollect without letting go of the succession of aggregates. Ordinary disciples both recollect by means of the succession of aggregates and trace by means of death and rebirth-linking. Likewise, the eighty great disciples. But the chief disciples have nothing to do with the succession of aggregates. When they see the death of one person, they see the rebirth-linking, and again when they see the death of another, they see the rebirth-linking. So they go by tracing through death and rebirth-thinking. Likewise, Paccekabuddhas.

                \vismParagraph{XIII.18}{18}{}
                Buddhas, however, have nothing to do either with succession of aggregates or with tracing through death and rebirth-linking; for whatever instance they choose in many millions of eons, or more or less, is evident to them. So they go, and so they descend with the lion’s descent\footnote{\vismAssertFootnoteCounter{9}\vismHypertarget{XIII.n9}{}A commentarial account of the behaviour of lions will be found in the \emph{Manorathapurāṇī}, commentary to AN 4:33. \textbf{\cite{Vism-mhṭ}} says: \emph{Sīh-okkamana-vasena sīhātipatanavasena ñāṇagatiyā gacchati} (p. 408).} wherever they want, even skipping over many millions of eons as though they were an elision in a text. And just as an arrow shot by such a master of archery expert in hair-splitting as Sarabhaṅga (see \textbf{\cite{J-a}V 129}) always hits the target without getting held up among trees, creepers, etc., on its way, and so neither gets held up nor misses, so too, since Buddhas go in this way their knowledge does not get held up in intermediate \marginnote{\textcolor{teal}{\footnotesize\{464|406\}}}{}births \textcolor{brown}{\textit{[412]}} or miss; without getting held up or missing, it seizes any instance required.

                \vismParagraph{XIII.19}{19}{}
                Among these beings with recollection of past lives, the sectarians’ vision of past lives seems like the light of a glow-worm, that of ordinary disciples like the light of a candle, that of the great disciples like the light of a torch, that of the chief disciples like the light of the morning star, that of Paccekabuddhas like the light of the moon, and that of Buddhas like the glorious autumn sun’s disk with its thousand rays.

                \vismParagraph{XIII.20}{20}{}
                Other sectarians see past lives as blind men go [tapping] with the point of a stick. Ordinary disciples do so as men who go on a log bridge. The great disciples do so as men who go on a foot bridge. The chief disciples do so as men who go on a cart bridge. Paccekabuddhas do so as men who go on a main foot-path. And Buddhas do so as men who go on a high road for carts.

                \vismParagraph{XIII.21}{21}{}
                In this connection it is the disciples’ recollection of past lives that is intended. Hence it was said above: “‘He recollects’: he recollects it following it out by the succession of aggregates, or by death and rebirth-linking” (\hyperlink{XIII.14}{§14}{}).

                \vismParagraph{XIII.22}{22}{}
                So a bhikkhu who is a beginner and wants to recollect in this way should go into solitary retreat on return from his alms round after his meal. Then he should attain the four jhānas in succession and emerge from the fourth jhāna as basis for direct-knowledge. He should then advert to his most recent act of sitting down [for this purpose], next, to the preparation of the seat, to the entry into the lodging, to the putting away of the bowl and [outer] robe, to the time of eating, to the time of returning from the village, to the time of wandering for alms in the village, to the time of entering the village, to the time of setting out from the monastery, to the time of paying homage at the shrine terrace and the Enlightenment-tree terrace, to the time of washing the bowl, to the time of picking up the bowl, to the things done from the time of picking up the bowl back to the mouth washing, to the things done in the early morning, to the things done in the middle watch, in the first watch. In this way he should advert to all the things done during the whole night and day in reverse order.

                \vismParagraph{XIII.23}{23}{}
                While this much, however, is evident even to his normal consciousness, it is especially evident to his preliminary-work consciousness. But if anything there is not evident, he should again attain the basic jhāna, emerge and advert. By so doing it becomes as evident as when a lamp is lit. And so, in reverse order too, he should advert to the things done on the second day back, and on the third, fourth and fifth day, and in the ten days, and in the fortnight, and as far back as a year.

                \vismParagraph{XIII.24}{24}{}
                When by these means he adverts to ten years, twenty years, and so on as far back as his own rebirth-linking in this existence, \textcolor{brown}{\textit{[413]}} he should advert to the mentality-materiality occurring at the moment of death in the preceding existence; for a wise bhikkhu is able at the first attempt to remove\footnote{\vismAssertFootnoteCounter{10}\vismHypertarget{XIII.n10}{}\emph{Ugghaṭetvā}: see \hyperlink{X.6}{X.6}{}; the word is obviously used here in the same sense.} the rebirth-linking and make the mentality-materiality at the death moment his object.

                \vismParagraph{XIII.25}{25}{}
                \marginnote{\textcolor{teal}{\footnotesize\{465|407\}}}{}But the mentality-materiality in the previous existence has ceased without remainder and another has arisen, and consequently that instance is, as it were, shut away in darkness, and it is hard for one of little understanding to see it. Still he should not give up the task, thinking, “I am unable to remove the rebirth-linking and make the mentality-materiality that occurred at the death moment my object.” On the contrary, he should again and again attain that same basic jhāna, and each time he emerges he should advert to that instance.

                \vismParagraph{XIII.26}{26}{}
                Just as when a strong man is felling a big tree for the purpose of making the peak of a gable, but is unable to fell the big tree with an axe blade blunted by lopping the branches and foliage, still he does not give up the task; on the contrary, he goes to a smithy and has his axe sharpened, after which he returns and continues chopping the tree; and when the axe again gets blunt, he does as before and continues chopping it; and as he goes on chopping it in this way, the tree falls at length, because each time there is no need to chop again what has already been chopped and what has not yet been chopped gets chopped; so too, when he emerges from the basic jhāna, instead of adverting to what he has already adverted to, he should advert only to the rebirth-linking, and at length he removes the rebirth-linking and makes the mentality-materiality that occurred at the death moment his object. And this meaning should also be illustrated by means of the wood cutter and the hair-cutter as well.

                \vismParagraph{XIII.27}{27}{}
                Herein, the knowledge that occurs making its object the period from the last sitting down for this purpose back to the rebirth-linking is not called knowledge of recollection of past lives; but it is called preliminary-work-concentration knowledge; and some call it “knowledge of the past” (\emph{atītaṃsa-ñāṇa}), but that is inappropriate to the fine-material sphere.

                However, when this bhikkhu has got back beyond the rebirth-linking, there arises in him mind-door adverting making its object the mentality-materiality that occurred at the death moment. And when that has ceased, then either four or five impulsions impel making that their object too. The first of these, called “preliminary-work,” etc., in the way already described (\hyperlink{XIII.5}{§5}{}), are of the sense sphere. The last is a fine-material absorption consciousness of the fourth jhāna. The knowledge that arises in him then together with that consciousness is what is called, “knowledge of recollection of past lives.” It is with the mindfulness (memory) associated with that knowledge that he “recollects his manifold past lives, that is to say, one birth, two births, …”\textcolor{brown}{\textit{[414]}} thus with details and particulars he recollects his manifold past lives (\textbf{\cite{D}I 81}).

                \vismParagraph{XIII.28}{28}{}
                Herein, \emph{one birth} is the continuity of aggregates included in a single becoming starting with rebirth-linking and ending with death. So too with two births, and the rest.

                But in the case of \emph{many eons of world contraction}, etc., it should be understood that the aeon of world contraction is an aeon of diminution and the aeon of world expansion is an aeon of increase.

                \vismParagraph{XIII.29}{29}{}
                Herein, what supersedes the contraction is included in the contraction since it is rooted in it; and so too what supersedes the expansion is included in the expansion. This being so, it includes what is stated thus: “Bhikkhus, there \marginnote{\textcolor{teal}{\footnotesize\{466|408\}}}{}are four incalculables of the aeon. What four? The contraction, what supersedes the contraction, the expansion, and what supersedes the expansion” (\textbf{\cite{A}II 142} abbreviated).

                \vismParagraph{XIII.30}{30}{}
                Herein, there are three kinds of contraction: contraction due to water, contraction due to fire, and contraction due to air (see MN 28). Also there are three limits to the contraction; the Ābhassara (Streaming-radiance) Brahmā-world, that of the Subhakiṇha (Refulgent-glory), and that of the Vehapphala (Great-fruit). When the aeon contracts owing to fire, all below the Ābhassara [Brahmā-world] is burnt up by fire. When it contracts owing to water, it is all dissolved by water up to the Subhakiṇha [Brahmā-world]. When it contracts owing to air, it is all demolished by wind up to the Vehapphala [Brahmā-world].

                \vismParagraph{XIII.31}{31}{}
                In breadth it is always one of the Buddha-fields that is destroyed. For the Buddha-fields are of three kinds, that is, the field of birth, the field of authority, and the field of scope.

                Herein, the field of birth is limited by the ten thousand world-spheres that quaked on the Perfect One’s taking rebirth-linking, and so on. The field of authority is limited by the hundred thousand million world-spheres where the following safeguards (\emph{paritta}) are efficacious, that is, the Ratana Sutta (Sn p.39), the Khandha Paritta (\textbf{\cite{Vin}II 109}; \textbf{\cite{A}II 72}), the Dhajagga Paritta (\textbf{\cite{S}I 218}), the Āṭānāṭiya Paritta (\textbf{\cite{D}III 194}), and the Mora Paritta (\textbf{\cite{J-a}II 33}). The field of scope is boundless, immeasurable: “As far as he wishes” (\textbf{\cite{A}I 228}), it is said. The Perfect One knows anything anywhere that he wishes. So one of these three Buddha-fields, that is to say, the field of authority is destroyed. But when that is being destroyed, the field of birth also gets destroyed. And that happens simultaneously; and when it is reconstituted, that happens simultaneously (cf. \textbf{\cite{M-a}IV 114}).

                \vismParagraph{XIII.32}{32}{}
                Now, it should be understood how its destruction and reconstitution come about thus. On the occasion when the aeon is destroyed by fire \textcolor{brown}{\textit{[415]}} first of all a great cloud heralding the aeon’s destruction appears, and there is a great downpour all over the hundred thousand million world-spheres. People are delighted, and they bring out all their seeds and sow them. But when the sprouts have grown enough for an ox to graze, then not a drop of rain falls any more even when the asses bray. Rain is withheld from then on. This is what the Blessed One referred to when he said: “Bhikkhus, an occasion comes when for many years, for many hundreds of years, for many thousands of years, for many hundreds of thousands of years, there is no rain” (\textbf{\cite{A}IV 100}). Beings that live by rain die and are reborn in the Brahmā-world, and so are the deities that live on flowers and fruits.

                \vismParagraph{XIII.33}{33}{}
                When a long period has passed in this way, the water gives out here and there. Then in due course the fishes and turtles die and are reborn in the Brahmā-world, and so are the beings in hell. Some say that the denizens of hell perish there with the appearance of the seventh sun (\hyperlink{XIII.41}{§41}{}).

                Now, there is no rebirth in the Brahmā-world without jhāna; and some of them, being obsessed with the scarcity of food, are unable to attain jhāna, so how are they reborn there? By means of jhāna obtained in the [sense-sphere] divine world.

                \vismParagraph{XIII.34}{34}{}
                \marginnote{\textcolor{teal}{\footnotesize\{467|409\}}}{}For then the sense-sphere deities called world-marshal (\emph{loka-byūha}) deities come to know that at the end of a hundred thousand years there will be the emergence of an aeon, and they travel up and down the haunts of men, their heads bared, their hair dishevelled, with piteous faces, mopping their tears with their hands, clothed in dyed cloth, and wearing their dress in great disorder. They make this announcement: “Good sirs, good sirs, at the end of a hundred thousand years from now there will be the emergence of an aeon. This world will be destroyed. Even the ocean will dry up. This great earth, and the Sineru King of Mountains, will be consumed and destroyed. The destruction of the earth will extend as far as the Brahmā-world. Develop loving-kindness, good sirs, develop compassion, gladness, equanimity, good sirs. Care for your mothers, care for your fathers, honour the elders of your clans.”

                \vismParagraph{XIII.35}{35}{}
                When human beings and earth deities hear their words, they mostly are filled with a sense of urgency. They become kind to each other and make merit with loving-kindness, etc., and so they are reborn in the divine world. There they eat divine food, and they do the preliminary work on the air kasiṇa and acquire jhāna. Others, however, are reborn in a [sense-sphere] divine world through kamma to be experienced in a future life. For there is no being traversing the round of rebirths who is destitute of kamma to be experienced in a future life. They too acquire jhāna there in the same way. \textcolor{brown}{\textit{[416]}} All are eventually reborn in the Brahmā-world by acquiring jhāna in a [sense-sphere] divine world in this way.

                \vismParagraph{XIII.36}{36}{}
                However, at the end of a long period after the withholding of the rain, a second sun appears. And this is described by the Blessed One in the way beginning, “Bhikkhus, there is the occasion when …” (\textbf{\cite{A}IV 100}), and the Sattasuriya Sutta should be given in full. Now, when that has appeared, there is no more telling night from day; as one sun sets, the other rises. The world is uninterruptedly scorched by the suns. But there is no sun deity in the aeon-destruction sun as there is in the ordinary sun.\footnote{\vismAssertFootnoteCounter{11}\vismHypertarget{XIII.n11}{}“The ‘ordinary sun’ is the sun’s divine palace that arose before the emergence of the aeon. But like the other sense-sphere deities at the time of the emergence of the aeon, the sun deity too produces jhāna and reappears in the Brahmā-world. But the actual sun’s disk becomes brighter and more fiery. Others say that it disappears and another appears in its place” (\textbf{\cite{Vism-mhṭ}412}).} Now, when the ordinary sun is present, thunder clouds and mare’s-tail vapours cross the skies. But when the aeon-destruction sun is present, the sky is as blank as the disk of a looking-glass and destitute of clouds and vapour. Beginning with the rivulet, the water in all the rivers except the five great rivers\footnote{\vismAssertFootnoteCounter{12}\vismHypertarget{XIII.n12}{}The five are the Ganges, Yamunā (Jumma), Sarabhu, Sarassatī, and Mahī (\textbf{\cite{Vism-mhṭ}412}).} dries up.

                \vismParagraph{XIII.37}{37}{}
                After that, at the end of a long period, a third sun appears. And when that has appeared, the great rivers dry up too.

                \vismParagraph{XIII.38}{38}{}
                After that, at the end of a long period, a fourth sun appears. And when that has appeared, the seven great lakes in Himalaya, the sources of the great rivers, \marginnote{\textcolor{teal}{\footnotesize\{468|410\}}}{}dry up, that is to say: Sīhapapāta, Haṃsapātana,\footnote{\vismAssertFootnoteCounter{13}\vismHypertarget{XIII.n13}{}\emph{Haṃsapātana} is another name for \emph{Maṇḍākinī. }(Vism-mhṭ) For seven Great Lakes see \textbf{\cite{A}IV 101}.} Kaṇṇamuṇḍaka, Rathakāra, Anotatta, Chaddanta, and Kuṇāla.

                \vismParagraph{XIII.39}{39}{}
                After that, at the end of a long period, a fifth sun appears, and when that has appeared, there eventually comes to be not enough water left in the great ocean to wet one finger joint.

                \vismParagraph{XIII.40}{40}{}
                After that, at the end of a long period, a sixth sun appears, and when that has appeared, the whole world-sphere becomes nothing but vapour, all its moisture being evaporated.

                And the hundred thousand million world-spheres are the same as this one.

                \vismParagraph{XIII.41}{41}{}
                After that, at the end of a long period, a seventh sun appears. And when that has appeared, the whole world-sphere together with the hundred thousand million other world-spheres catches fire. Even the summits of Sineru, a hundred leagues and more high, crumble and vanish into space. The conflagration mounts up and invades the realm of the Four Kings. When it has burnt up all the golden palaces, the jewelled palaces and the crystal palaces there, it invades the Realm of the Thirty-three. And so it goes right on up to the plane of the first jhāna. When it has burnt three [lower] Brahmā-worlds, it stops there at the Ābhassara-world. \textcolor{brown}{\textit{[417]}} As long as any formed thing (formation) the size of an atom still exists it does not go out; but it goes out when all formed things have been consumed. And like the flame that burns ghee and oil, it leaves no ash.

                \vismParagraph{XIII.42}{42}{}
                The upper space is now all one with the lower space in a vast gloomy darkness. Then at the end of a long period a great cloud arises, and at first it rains gently, and then it rains with ever heavier deluges, like lotus stems, like rods, like pestles, like palm trunks, more and more. And so it pours down upon all burnt areas in the hundred thousand million world-spheres till they disappear. Then the winds (forces) beneath and all around that water rise up and compact it and round it, like water drops on a lotus leaf. How do they compact the great mass of water? By making gaps; for the wind makes gaps in it here and there.

                \vismParagraph{XIII.43}{43}{}
                Being thus compressed by the air, compacted and reduced, it gradually subsides. As it sinks, the [lower] Brahmā-world reappears in its place, and worlds divine reappear in the places of the four upper divine worlds of the sensual sphere.\footnote{\vismAssertFootnoteCounter{14}\vismHypertarget{XIII.n14}{}“At the place where the Yāma Deities are established. The places where the Cātumahārājika and Tāvatiṃsa heavens become established do not reappear at first because they are connected with the earth” (\textbf{\cite{Vism-mhṭ}412}).} But when it has sunk to the former earth’s level, strong winds (forces) arise and they stop it and hold it stationary, like the water in a water pot when the outlet is plugged. As the fresh water gets used up, the essential humus makes its appearance on it. That possesses colour, smell and taste, like the surface film on milk rice when it dries up.

                \vismParagraph{XIII.44}{44}{}
                Then the beings that were reborn first in the Brahmā-world of Streaming-radiance (Ābhassara) fall from there with the exhaustion of their life span, or \marginnote{\textcolor{teal}{\footnotesize\{469|411\}}}{}when their merit is exhausted, and they reappear here. They are self-luminous and wander in the sky. On eating the essential humus, as is told in the Aggañña Sutta (\textbf{\cite{D}III 85}), they are overcome by craving, and they busy themselves in making lumps of it to eat. Then their self-luminosity vanishes, and it is dark. They are frightened when they see the darkness.

                \vismParagraph{XIII.45}{45}{}
                Then in order to remove their fears and give them courage, the sun’s disk appears full fifty leagues across. They are delighted to see it, thinking, “We have light,” and they say, “It has appeared in order to allay our fears and give us courage (\emph{sūrabhāva}), so let it be called ‘sun’ (\emph{suriya}).” So they give it the name “sun” (\emph{suriya}). Now, when the sun has given light for a day, it sets. Then they are frightened again, thinking, “We have lost the light we had,” and they think, “How good if we had another light!” \textcolor{brown}{\textit{[418]}}

                \vismParagraph{XIII.46}{46}{}
                As if knowing their thought, the moon’s disk appears, forty-nine leagues across. On seeing it they are still more delighted, and they say, “It has appeared, seeming as if it knew our desire (\emph{chanda}), so let it be called ‘moon’ (\emph{canda}).” So they give it the name “moon” (\emph{canda}).

                \vismParagraph{XIII.47}{47}{}
                After the appearance of the moon and sun in this way, the stars appear in their constellations. After that, night and day are made known, and in due course, the month and half month, the season, and the year.

                \vismParagraph{XIII.48}{48}{}
                On the day the moon and sun appear, the mountains of Sineru, of the World-sphere and of Himalaya appear too. And they appear on the full-moon day of the month of Phagguna (March), neither before nor after. How? Just as, when millet is cooking and bubbles arise, then simultaneously, some parts are domes, some hollow, and some flat, so too, there are mountains in the domed places, seas in the hollow places, and continents (islands) in the flat places.

                \vismParagraph{XIII.49}{49}{}
                Then, as these beings make use of the essential humus, gradually some become handsome and some ugly. The handsome ones despise the ugly ones. Owing to their contempt the essential humus vanishes and an outgrowth from the soil appears. Then that vanishes in the same way and the \emph{badālatā} creeper appears. That too vanishes in the same way and the rice without red powder or husk that ripens without tilling appears, a clean sweet-smelling rice fruit.

                \vismParagraph{XIII.50}{50}{}
                Then vessels appear. These beings put the rice into the vessels, which they put on the tops of stones. A flame appears spontaneously and cooks it. The cooked rice resembles jasmine flowers. It has no need of sauces and curries, since it has whatever flavour they want to taste.

                \vismParagraph{XIII.51}{51}{}
                As soon as they eat this gross food, urine and excrement appear in them. Then wound orifices break open in them to let these things out. The male sex appears in the male, and the female sex in the female. Then the females brood over the males, and the males over the females for a long time. Owing to this long period of brooding, the fever of sense desires arises. After that they practice sexual intercourse.

                \vismParagraph{XIII.52}{52}{}
                \textcolor{brown}{\textit{[419]}} For their [overt] practice of evil they are censured and punished by the wise, and so they build houses for the purpose of concealing the evil. When they live in houses, they eventually fall in with the views of the more lazy, and \marginnote{\textcolor{teal}{\footnotesize\{470|412\}}}{}they make stores of food. As soon as they do that, the rice becomes enclosed in red powder and husks and no longer grows again of itself in the place where it was reaped. They meet together and bemoan the fact, “Evil has surely made its appearance among beings; for formerly we were mind-made …” (\textbf{\cite{D}III 90}), and all this should be given in full in the way described in the Aggañña Sutta.

                \vismParagraph{XIII.53}{53}{}
                After that, they set up boundaries. Then some being takes a portion given to another. After he has been twice rebuked, at the third time they come to blows with fists, clods, sticks, and so on. When stealing, censuring, lying, resorting to sticks, etc., have appeared in this way, they meet together, thinking, “Suppose we elect a being who would reprove those who should be reproved, censure those who should be censured, and banish those who should be banished, and suppose we keep him supplied with a portion of the rice?” (\textbf{\cite{D}III 92}).

                \vismParagraph{XIII.54}{54}{}
                When beings had come to an agreement in this way in this aeon, firstly this Blessed One himself, who was then the Bodhisatta (Being due to be Enlightened), was the handsomest, the most comely, the most honourable, and was clever and capable of exercising the effort of restraint. They approached him, asked him, and elected him. Since he was recognized (\emph{sammata}) by the majority (\emph{mahā-jana}) he was called Mahā Sammata. Since he was lord of the fields (\emph{khetta}) he was called khattiya (warrior noble). Since he promoted others’ good (\emph{rañjeti}) righteously and equitably he was a king (\emph{rājā}). This is how he came to be known by these names. For the Bodhisatta himself is the first man concerned in any wonderful innovation in the world. So after the khattiya circle had been established by making the Bodhisatta the first in this way, the brahmans and the other castes were founded in due succession.

                \vismParagraph{XIII.55}{55}{}
                Herein, the period from the time of the great cloud heralding the aeon’s destruction up till the ceasing of the flames constitutes one incalculable, and that is called the “contraction.” That from the ceasing of the flames of the aeon destruction up till the great cloud of rehabilitation, which rains down upon the hundred thousand million world-spheres, constitutes the second incalculable, and that is called, “what supersedes the contraction.” That from the time of the great cloud of rehabilitation up till the appearance of the moon and sun constitutes the third incalculable, and that is called the “expansion.” That from the appearance of the moon and sun up till \textcolor{brown}{\textit{[420]}} the reappearance of the great cloud of the aeon destruction is the fourth incalculable, and that is called, “what supersedes the expansion.” These four incalculables make up one great aeon. This, firstly, is how the destruction by fire and reconstitution should be understood.

                \vismParagraph{XIII.56}{56}{}
                The occasion when the aeon is destroyed by water should be treated in the way already described beginning, “First of all a great cloud heralding the aeon’s destruction appears …” (\hyperlink{XIII.32}{§32}{}).

                \vismParagraph{XIII.57}{57}{}
                There is this difference, however. While in the former case a second sun appeared, in this case a great cloud of caustic waters\footnote{\vismAssertFootnoteCounter{15}\vismHypertarget{XIII.n15}{}\emph{Khārudaka—}“caustic waters”: the name given to the waters on which the world-spheres rest (see \textbf{\cite{M-a}IV 178}).} appears. At first it rains \marginnote{\textcolor{teal}{\footnotesize\{471|413\}}}{}very gently, but it goes on to rain with gradually greater deluges, pouring down upon the hundred thousand million world-spheres. As soon as they are touched by the caustic waters, the earth, the mountains, etc., melt away, and the waters are supported all round by winds. The waters take possession from the earth up to the plane of the second jhāna. When they have dissolved away the three Brahmā-worlds there, they stop at the Subhakiṇha-world. As long as any formed thing the size of an atom exists they do not subside; but they suddenly subside and vanish away when all formed things have been overwhelmed by them. All beginning with: “The upper space is all one with the lower space in a vast gloomy darkness …” (\hyperlink{XIII.42}{§42}{}) is as already described, except that here the world begins its reappearance with the Ābhassara Brahmā-world. And beings falling from the Subhakiṇha Brahmā-world are reborn in the places beginning with the Ābhassara Brahmā-world.

                \vismParagraph{XIII.58}{58}{}
                Herein, the period from the time of the great cloud heralding the aeon’s destruction up till the ceasing of the aeon-destroying waters constitutes one incalculable. That from the ceasing of the waters up till the great cloud of rehabilitation constitutes the second incalculable. That from the great cloud of rehabilitation … These four incalculables make up one great aeon. This is how the destruction by water and reconstitution should be understood.

                \vismParagraph{XIII.59}{59}{}
                The occasion when the aeon is destroyed by air should be treated in the way already described beginning with the “first of all a great cloud heralding the aeon’s destruction appears …” (\hyperlink{XIII.32}{§32}{}).

                \vismParagraph{XIII.60}{60}{}
                There is this difference, however. While in the first case there was a second sun, here a wind arises in order to destroy the aeon. First of all it lifts up the coarse flue, then the fine flue, then the fine sand, coarse sand, gravel, stones, etc., \textcolor{brown}{\textit{[421]}} until it lifts up stones as big as a catafalque,\footnote{\vismAssertFootnoteCounter{16}\vismHypertarget{XIII.n16}{}\emph{Kūṭāgāra: }see \hyperlink{XII.n14}{Ch. XII, n.14}{}; here this seems the most likely of the various meanings of the word.} and great trees standing in uneven places. They are swept from the earth up into the sky, and instead of falling down again they are broken to bits there and cease to exist.

                \vismParagraph{XIII.61}{61}{}
                Then eventually wind arises from underneath the great earth and overturns the earth, flinging it into space. The earth splits into fragments measuring a hundred leagues, measuring two, three, four, five hundred leagues, and they are hurled into space too, and there they are broken to bits and cease to exist. The world-sphere mountains and Mount Sineru are wrenched up and cast into space, where they crash against each other till they are broken to bits and disappear. In this way it destroys the divine palaces built on the earth [of Mount Sineru] and those built in space, it destroys the six sensual-sphere divine worlds, and it destroys the hundred thousand million world-spheres. Then world-sphere collides with world-sphere, Himalaya Mountain with Himalaya Mountain, Sineru with Sineru, till they are broken to bits and disappear.

                \vismParagraph{XIII.62}{62}{}
                The wind takes possession from the earth up to the plane of the third jhāna. There, after destroying three Brahmā-worlds, it stops at the Vehapphala-world. When it has destroyed all formed things in this way, it spends itself too. \marginnote{\textcolor{teal}{\footnotesize\{472|414\}}}{}Then all happens as already described in the way beginning, “The upper space is all one with the lower space in a vast gloomy darkness …” (\hyperlink{XIII.42}{§42}{}). But here the world begins its reappearance with the Subhakiṇha Brahmā-world. And beings falling from the Vehapphala Brahmā-world are reborn in the places beginning with the Subhakiṇha Brahmā-world.

                \vismParagraph{XIII.63}{63}{}
                Herein, the period from the time of the great cloud heralding the aeon’s destruction up till the ceasing of the aeon-destroying wind is one incalculable. That from the ceasing of the wind up till the great cloud of rehabilitation is the second incalculable … These four incalculables make up one great aeon. This is how the destruction by wind and reconstitution should be understood.

                \vismParagraph{XIII.64}{64}{}
                What is the reason for the world’s destruction in this way? The [three] roots of the unprofitable are the reason. When any one of the roots of the unprofitable becomes conspicuous, the world is destroyed accordingly. When greed is more conspicuous, it is destroyed by fire. When hate is more conspicuous, it is destroyed by water—though some say that it is destroyed by fire when hate is more conspicuous and by water when greed is more conspicuous. And when delusion is more conspicuous, it is destroyed by wind.

                \vismParagraph{XIII.65}{65}{}
                Destroyed as it is in this way, it is destroyed for seven turns in succession by fire and the eighth turn by water; then again seven turns by fire and the eighth turn by water; then, when it has been seven times destroyed by water at each eighth \textcolor{brown}{\textit{[422]}} turn, it is again destroyed for seven turns by fire. Sixty-three eons pass in this way. And now the air takes the opportunity to usurp the water’s turn for destruction, and in destroying the world it demolishes the Subhakiṇha Brahmā-world where the life span is the full sixty-four eons.

                \vismParagraph{XIII.66}{66}{}
                Now, when a bhikkhu capable of recollecting eons is recollecting his former lives, then of such eons as these he recollects \emph{many eons of world contraction, many eons of world expansion, many eons of world contraction and expansion}. How? In the way beginning, \emph{There I was }…

                Herein, \emph{There I was}: in that eon of contraction I was in that kind of becoming or generation or destiny or station of consciousness or abode of beings or order of beings.

                \vismParagraph{XIII.67}{67}{}
                \emph{So named}: [such forenames as] Tissa, say, or Phussa. \emph{Of such a race}: [such family names as] Kaccāna, say, or Kassapa. This is said of the recollection of his own name and race (surname) in his past existence. But if he wants to recollect his own appearance at that time, or whether his life was a rough or refined one, or whether pleasure or pain was prevalent, or whether his life span was short or long, he recollects that too. Hence he said with \emph{such an appearance … such the end of my life span}.

                \vismParagraph{XIII.68}{68}{}
                Here, \emph{with such an appearance} means fair or dark. \emph{Such was my food}: with white rice and meat dishes as food or with windfall fruits as food. \emph{Such my experience of pleasure and pain}: with varied experience of bodily and mental pleasure and pain classed as worldly and unworldly, and so on. \emph{Such the end of my life span}: with such a life span of a century or life span of eighty-four thousand eons.

                \vismParagraph{XIII.69}{69}{}
                \marginnote{\textcolor{teal}{\footnotesize\{473|415\}}}{}\emph{And passing away from there, I reappeared elsewhere}: having passed away from that becoming, generation, destiny, station of consciousness, abode of beings or order of beings, I again appeared in that other becoming, generation, destiny, station of consciousness, abode of beings or order of beings. \emph{And there too I was}: then again I was there in that becoming, generation, destiny, station of consciousness, abode of beings or order of beings. \emph{So named}, etc., are as already stated.

                \vismParagraph{XIII.70}{70}{}
                Furthermore, the words \emph{there I was }refer to the recollection of one who has cast back retrospectively as far as he wishes, and the words \emph{and passing away from there} refer to his reviewing after turning forward again; consequently, the words \emph{I appeared elsewhere} can be understood to be said with reference to the place of his reappearance next before his appearance here, which is referred to by the words \emph{I appeared here}. But the words \emph{there too I was}, etc., \textcolor{brown}{\textit{[423]}} are said in order to show the recollection of his name, race, etc., there in the place of his reappearance next before this appearance. \emph{And passing away from there, I reappeared here}: having passed away from that next place of reappearance, I was reborn here in this khattiya clan or brahman clan.

                \vismParagraph{XIII.71}{71}{}
                \emph{Thus}: so. \emph{With its aspects and particulars}: with its particulars consisting in name and race; with its aspects consisting in appearance, and so on. For it is by means of name and race that a being is particularized as, say Tissa Kassapa; but his distinctive personality is made known by means of appearance, etc., as dark or fair. So the name and race are the particulars, while the others are the aspects. \emph{He recollects his manifold past lives}: the meaning of this is clear.

                The explanation of the knowledge of recollection of past lives is ended.
            \section[\vismAlignedParas{§72–101}(5) The Divine Eye—Knowledge of Passing Away and Reappearance of Beings]{(5) The Divine Eye—Knowledge of Passing Away and Reappearance of Beings}

                \vismParagraph{XIII.72}{72}{}
                As to the explanation of the knowledge of passing away and reappearance of beings, [here is the text: “He directs, he inclines, his mind to the knowledge of the passing away and reappearance of beings. With the divine eye, which is purified and surpasses the human, he sees beings passing away and reappearing, inferior and superior, fair and ugly, happy or unhappy in their destiny; he understands beings as faring according to their deeds: ‘These worthy beings who were ill-conducted in body, speech and mind, revilers of Noble Ones, wrong in their views, acquirers of kamma due to wrong view, have, on the breakup of the body, after death, appeared in a state of loss, in an unhappy destiny, in perdition in hell; but these worthy beings, who are well conducted in body, speech and mind, not revilers of Noble Ones, right in their views, acquirers of kamma due to right view, have, on the breakup of the body, after death, appeared in a happy destiny, in the heavenly world.’ Thus with the divine eye, which is purified and surpasses the human, he sees beings passing away and reappearing, inferior and superior, fair and ugly, happy or unhappy in their destiny; he understands beings as faring according to their deeds” (\textbf{\cite{D}I 82}). Herein,] \emph{to the knowledge of the passing away and reappearance: cutūpapātañāṇāya= cutiyā ca upapāte ca ñāṇāya} (resolution of compound); [the meaning is,] for the \marginnote{\textcolor{teal}{\footnotesize\{474|416\}}}{}kind of knowledge by means of which beings’ passing away and reappearance is known; for knowledge of the divine eye, is what is meant. \emph{He directs, he inclines his mind}: he both directs and inclines preliminary-work consciousness. He is the bhikkhu who does the directing of his mind.

                \vismParagraph{XIII.73}{73}{}
                But as regards \emph{with the divine eye}, etc., it is \emph{divine} because of its similarity to the divine; for deities have as divine eye the sensitivity that is produced by kamma consisting in good conduct and is unimpeded by bile, phlegm, blood, etc., and capable of receiving an object even though far off because it is liberated from imperfections. And this eye, consisting in knowledge, which is produced by the power of this bhikkhu’s energy in development, is similar to that, so it is “divine” because it is similar to the divine. Also it is “divine” because it is obtained by means of divine abiding, and because it has divine abiding as its support. And it is “divine” because it greatly illuminates by discerning light. And it is “divine” because it has a great range through seeing visible objects that are behind walls, and so on. All that should be understood according to the science of grammar. It is an \emph{eye} in the sense of seeing. Also it is an eye since it is like an eye in its performance of an eye’s function. It is \emph{purified} since it is a cause of purification of view, owing to seeing passing away and reappearance.

                \vismParagraph{XIII.74}{74}{}
                One who sees only passing away and not reappearance assumes the annihilation view; and one who sees only reappearance and not passing away assumes the view that a new being appears. But since one who sees both outstrips that twofold [false] view, that vision of his is therefore a cause for purification of view. And the Buddhas’ sons see both of these. Hence it was said above: \textcolor{brown}{\textit{[424]}} “It is ‘purified’ since it is a cause of purification of view, owing to seeing passing away and reappearance.”

                \vismParagraph{XIII.75}{75}{}
                \emph{It surpasses the human} in the seeing of visible objects by surpassing the human environment. Or it can be understood that it \emph{surpasses the human} in surpassing the human fleshly eye. With that \emph{divine eye, which is purified and superhuman, he sees beings}, he watches beings as men do with the fleshly eye.

                \vismParagraph{XIII.76}{76}{}
                \emph{Passing away and reappearing}: he cannot see them with the divine eye actually at the death moment of reappearance.\footnote{\vismAssertFootnoteCounter{17}\vismHypertarget{XIII.n17}{}\emph{“‘He cannot see them with the divine }eye’—with the knowledge of the divine eye—because of the extreme brevity and extreme subtlety of the material moment in anyone. Moreover, it is present materiality that is the object of the divine eye, and that is by prenascence condition. And there is no occurrence of exalted consciousness without adverting and preliminary work. Nor is materiality that is only arising able to serve as object condition, nor that which is dissolving. Therefore, it is rightly said that he cannot see with the divine eye materiality at the moments of death and reappearance. If the knowledge of the divine eye has only materiality as its object, then why is it said that he ‘sees beings’? It is said in this way since it is mainly concerned with instances of materiality in a being’s continuity, or because that materiality is a reason for apprehending beings. Some say that this is said according to conventional usage” (\textbf{\cite{Vism-mhṭ}417}).} But it is those who, being on the verge of death, will die now that are intended as “passing away” and those who have taken rebirth-linking and have just reappeared that are intended by \marginnote{\textcolor{teal}{\footnotesize\{475|417\}}}{}“reappearing.” What is pointed out is that he sees them as such passing away and reappearing.

                \vismParagraph{XIII.77}{77}{}
                \emph{Inferior}: despised, disdained, looked down upon, scorned, on account of birth, clan, wealth, etc., because of reaping the outcome of delusion. \emph{Superior}: the opposite of that because of reaping the outcome of non-delusion. \emph{Fair}: having a desirable, agreeable, pleasing appearance because of reaping the outcome of non-hate. \emph{Ugly}: having undesirable, disagreeable, unpleasing appearance because of reaping the outcome of hate; unsightly, ill-favoured, is the meaning. \emph{Happy in their destiny}: gone to a happy destiny; or rich, very wealthy, because of reaping the outcome of non-greed. Unhappy in their destiny: gone to an unhappy destiny; or poor with little food and drink because of reaping the outcome of greed.

                \vismParagraph{XIII.78}{78}{}
                \emph{Faring according to their deeds}: moving on in accordance with whatever deeds (\emph{kamma}) may have been accumulated. Herein, the function of the divine eye is described by the first expressions beginning with “passing away.” But the function of knowledge of faring according to deeds is described by this last expression.

                \vismParagraph{XIII.79}{79}{}
                The order in which that knowledge arises is this. Here a bhikkhu extends light downwards in the direction of hell, and he sees beings in hell undergoing great suffering. That vision is only the divine eye’s function. He gives it attention in this way, “After doing what deeds do these beings undergo this suffering?” Then knowledge that has those deeds as its object arises in him in this way, “It was after doing this.” Likewise he extends light upwards in the direction of the [sensual-sphere] divine world, and he sees beings in the Nandana Grove, the Missaka Grove, the Phārusaka Grove, etc., enjoying great good fortune. That vision also is only the divine eye’s function. He gives attention to it in this way, “After doing what deeds do these beings enjoy this good fortune?” Then knowledge that has those deeds as its object arises in him in this way, “It was after doing this.” This is what is called knowledge of faring according to deeds.

                \vismParagraph{XIII.80}{80}{}
                There is no special preliminary work for this. And as in this case, so too in the case of knowledge of the future; for these have the divine eye as their basis and their success is dependent on that of the divine eye. \textcolor{brown}{\textit{[425]}}

                \vismParagraph{XIII.81}{81}{}
                As to \emph{ill-conducted in body}, etc., it is bad conduct (\emph{duṭṭhu caritaṃ}), or it is corrupted conduct (\emph{duṭṭhaṃ caritaṃ}) because it is rotten with defilements, thus it is ill-conduct (\emph{duccarita}). The ill-conduct comes about by means of the body, or the ill-conduct has arisen due to the body, thus it is ill-conduct in body; so too with the rest. \emph{Ill-conducted} is endowed with ill-conduct.

                \vismParagraph{XIII.82}{82}{}
                \emph{Revilers of Noble Ones}: being desirous of harm for Noble Ones consisting of Buddhas, Paccekabuddhas, and disciples, and also of householders who are stream-enterers, they revile them with the worst accusations or with denial of their special qualities (see \textbf{\cite{Ud}44} and MN 12); they abuse and upbraid them, is what is meant.

                \vismParagraph{XIII.83}{83}{}
                Herein, it should be understood that when they say, “They have no asceticism, they are not ascetics,” they revile them with the worst accusation; \marginnote{\textcolor{teal}{\footnotesize\{476|418\}}}{}and when they say, “They have no jhāna or liberation or path of fruition, etc.,” they revile them with denial of their special qualities. And whether done knowingly or unknowingly it is in either case reviling of Noble Ones; it is weighty kamma resembling that of immediate result, and it is an obstacle both to heaven and to the path. But it is remediable.

                \vismParagraph{XIII.84}{84}{}
                The following story should be understood in order to make this clear. An elder and a young bhikkhu, it seems, wandered for alms in a certain village. At the first house they got only a spoonful of hot gruel. The elder’s stomach was paining him with wind. He thought, “This gruel is good for me; I shall drink it before it gets cold.” People brought a wooden stool to the doorstep, and he sat down and drank it. The other was disgusted and remarked, “The old man has let his hunger get the better of him and has done what he should be ashamed to do.” The elder wandered for alms, and on returning to the monastery he asked the young bhikkhu, “Have you any footing in this Dispensation, friend?”—“Yes, venerable sir, I am a stream-enterer.”—“Then, friend, do not try for the higher paths; one whose cankers are destroyed has been reviled by you.” The young bhikkhu asked for the elder’s forgiveness and was thereby restored to his former state.

                \vismParagraph{XIII.85}{85}{}
                So one who reviles a Noble One, even if he is one himself, should go to him; if he himself is senior, \textcolor{brown}{\textit{[426]}} he should sit down in the squatting position and get his forgiveness in this way, “I have said such and such to the venerable one; may he forgive me.” If he himself is junior, he should pay homage, and sitting in the squatting position and holding out his hand palms together, he should get his forgiveness in this way, “I have said such and such to you, venerable sir; forgive me.” If the other has gone away, he should get his forgiveness either by going to him himself or by sending someone such as a co-resident.

                \vismParagraph{XIII.86}{86}{}
                If he can neither go nor send, he should go to the bhikkhus who live in that monastery, and, sitting down in the squatting position if they are junior, or acting in the way already described if they are senior, he should get forgiveness by saying, “Venerable sirs, I have said such and such to the venerable one named so and so; may that venerable one forgive me.” And this should also be done when he fails to get forgiveness in his presence.

                \vismParagraph{XIII.87}{87}{}
                If it is a bhikkhu who wanders alone and it cannot be discovered where he is living or where he has gone, he should go to a wise bhikkhu and say, “Venerable sir, I have said such and such to the venerable one named so and so. When I remember it, I am remorseful. What shall I do?” He should be told, “Think no more about it; the elder forgives you. Set your mind at rest.” Then he should extend his hands palms together in the direction taken by the Noble One and say, “Forgive me.”

                \vismParagraph{XIII.88}{88}{}
                If the Noble One has attained the final Nibbāna, he should go to the place where the bed is, on which he attained the final Nibbāna, and should go as far as the charnel ground to ask forgiveness. When this has been done, there is no obstruction either to heaven or to the path. He becomes as he was before.

                \vismParagraph{XIII.89}{89}{}
                \emph{Wrong in their views}: having distorted vision. \emph{Acquirers of kamma due to wrong view}: those who have kamma of the various kinds acquired through wrong \marginnote{\textcolor{teal}{\footnotesize\{477|419\}}}{}view, and also those who incite others to bodily kamma, etc., rooted in wrong view. And here, though reviling of Noble Ones has already been included by the mention of verbal misconduct, and though wrong view has already been included by the mention of mental misconduct, it may be understood, nevertheless, that the two are mentioned again in order to emphasize their great reprehensibility.

                \vismParagraph{XIII.90}{90}{}
                Reviling Noble Ones is greatly reprehensible because of its resemblance to kamma with immediate result. For this is said: “Sāriputta, just as a bhikkhu possessing virtuous conduct, concentration and understanding could here and now attain final knowledge, so it is in this case, I say; if he does not abandon such talk and such thoughts and renounce such views, he will find himself in hell as surely as if he had been carried off and put there” (\textbf{\cite{M}I 71}).\footnote{\vismAssertFootnoteCounter{18}\vismHypertarget{XIII.n18}{}In rendering \emph{yathābhataṃ }here in this very idiomatic passage \textbf{\cite{M-a}II 32} has been consulted.} \textcolor{brown}{\textit{[427]}} And there is nothing more reprehensible than wrong view, according as it is said: “Bhikkhus, I do not see any one thing so reprehensible as wrong view” (\textbf{\cite{A}I 33}).

                \vismParagraph{XIII.91}{91}{}
                \emph{On the breakup of the body}: on the giving up of the clung-to aggregates. After death: in the taking up of the aggregates generated next after that. Or alternatively, \emph{on the breakup of the body} is on the interruption of the life faculty, and \emph{after death} is beyond the death consciousness.

                \vismParagraph{XIII.92}{92}{}
                \emph{A state of loss} and the rest are all only synonyms for hell. Hell is a \emph{state of loss }(\emph{apāya}) because it is removed (\emph{apeta}) from the reason (\emph{aya})\footnote{\vismAssertFootnoteCounter{19}\vismHypertarget{XIII.n19}{}For the word \emph{aya }see \hyperlink{XVI.17}{XVI.17}{}.} known as merit, which is the cause of [attaining] heaven and deliverance; or because of the absence (\emph{abhāva}) of any origin (\emph{āya}) of pleasures. The destiny (\emph{gati}, going), the refuge, of suffering (\emph{dukkha}) is the \emph{unhappy destiny} (\emph{duggati}); or the destiny (\emph{gati}) produced by kamma that is corrupted (\emph{duṭṭha}) by much hate (\emph{dosa}) is an \emph{unhappy destiny }(\emph{duggati}). Those who commit wrongdoings, being separated out (\emph{vivasa}) fall (\emph{nipatanti}) in here, thus it is \emph{perdition} (\emph{vinipāta}); or alternatively, when they are destroyed (\emph{vinassanto}), they fall (\emph{patanti}) in here, all their limbs being broken up, thus it is \emph{perdition} (\emph{vinipāta}). There is no reason (\emph{aya}) reckoned as satisfying here, thus it is hell (\emph{niraya}).

                \vismParagraph{XIII.93}{93}{}
                Or alternatively, the animal generation is indicated by the mention of \emph{states of loss}; for the animal generation is a state of loss because it is removed from the happy destiny; but it is not an unhappy destiny because it allows the existence of royal nāgas (serpents), who are greatly honoured. The realm of ghosts is indicated by the mention of the \emph{unhappy destiny}; for that is both a state of loss and an unhappy destiny because it is removed from the happy destiny and because it is the destiny of suffering; but it is not perdition because it is not a state of perdition such as that of the asura demons. The race of asura demons is indicated by the mention of \emph{perdition}; for that is both a state of loss and an unhappy destiny in the way already described, and it is called “perdition” (deprivation) from all opportunities. Hell itself in the various aspects of Avīci, etc., is indicated by the mention of \emph{hell}.

                \marginnote{\textcolor{teal}{\footnotesize\{478|420\}}}{}\emph{Have … appeared}: have gone to; have been reborn there, is the intention.

                \vismParagraph{XIII.94}{94}{}
                The bright side should be understood in the opposite way. But there is this difference. Here the mention of the happy destiny includes the human destiny, and only the divine destiny is included by the mention of \emph{heavenly}. Herein, a good (\emph{sundara}) destiny (\emph{gati}) is a \emph{happy destiny} (\emph{sugati}). It is the very highest (\emph{suṭṭhu aggo}) in such things as the objective fields comprising visible objects, etc., thus it is \emph{heavenly} (\emph{sagga}). All that is a world (\emph{loka}) in the sense of crumbling and disintegrating (\emph{lujjana-palujjana}). This is the word meaning.

                \emph{Thus with the divine eye}, etc., is all a summing-up phrase; the meaning here in brief is this: so with the divine eye … he sees.

                \vismParagraph{XIII.95}{95}{}
                Now, a clansman who is a beginner and wants to see in this way should make sure that the jhāna, which has a kasiṇa as its object and is the basis for direct-knowledge, is made in all ways susceptible of his guidance. Then one of these three kasiṇas, that is to say, the fire kasiṇa, white kasiṇa, \textcolor{brown}{\textit{[428]}} or light kasiṇa, should be brought to the neighbourhood [of the arising of divine-eye knowledge]. He should make this access jhāna his resort and stop there to extend [the kasiṇa]; the intention is that absorption should not be aroused here; for if he does induce absorption, the [kasiṇa] will become the support for basic jhāna, but not for the [direct-knowledge] preliminary work. The light kasiṇa is the best of the three. So either that, or one of the others, should be worked up in the way stated in the Description of the Kasiṇas, and it should be stopped at the level of access and extended there. And the method for extending it should be understood in the way already described there too. It is only what is visible within the area to which the kasiṇa has been extended that can be seen.

                \vismParagraph{XIII.96}{96}{}
                However, while he is seeing what is visible, the turn of the preliminary work runs out. Thereupon the light disappears. When that has disappeared, he no longer sees what is visible (cf. \textbf{\cite{M}III 158}). Then he should again and again attain the basic jhāna, emerge and pervade with light. In this way the light gradually gets consolidated till at length it remains in whatever sized area has been delimited by him in this way, “Let there be light here.” Even if he sits watching all day he can still see visible objects.

                \vismParagraph{XIII.97}{97}{}
                And here there is the simile of the man who set out on a journey by night with a grass torch. Someone set out on a journey by night, it seems, with a grass torch. His torch stopped flaming. Then the even and uneven places were no more evident to him. He stubbed the torch on the ground and it again blazed up. In doing so it gave more light than before. As it went on dying out and flaring up again, eventually the sun rose. When the sun had risen, he thought, “There is no further need of the torch,” and he threw it away and went on by daylight.

                \vismParagraph{XIII.98}{98}{}
                Herein, the kasiṇa light at the time of the preliminary work is like the light of the torch. His no more seeing what is visible when the light has disappeared owing to the turn of the preliminary work running out while he is seeing what is visible is like the man’s not seeing the even and uneven places owing to the torch’s stopping flaming. His repeated attaining is like the stubbing of the torch. His more powerful pervasion with light by repeating the preliminary work is like the torch’s giving more light than before. The strong light’s \marginnote{\textcolor{teal}{\footnotesize\{479|421\}}}{}remaining in as large an area as he delimits is like the sun’s rising. His seeing even during a whole day what is visible in the strong light after throwing the limited light away is like the man’s going on by day after throwing the torch away.

                \vismParagraph{XIII.99}{99}{}
                Herein, when visible objects that are not within the focus of the bhikkhu’s fleshly eye come into the focus of his eye of knowledge—that is to say, visible objects that are inside his belly, belonging to the heart basis, belonging to what is below the earth’s surface, behind walls, mountains and enclosures, or in another world-sphere—\textcolor{brown}{\textit{[429]}} and are as if seen with the fleshly eye, then it should be understood that the divine eye has arisen. And only that is capable of seeing the visible objects here, not the preliminary-work consciousnesses.

                \vismParagraph{XIII.100}{100}{}
                But this is an obstacle for an ordinary man. Why? Because wherever he determines, “Let there be light,” it becomes all light, even after penetrating through earth, sea and mountains. Then fear arises in him when he sees the fearful forms of spirits, ogres, etc., that are there, owing to which his mind is distracted and he loses his jhāna. So he needs to be careful in seeing what is visible (see \textbf{\cite{M}III 158}).

                \vismParagraph{XIII.101}{101}{}
                Here is the order of arising of the divine eye: when mind-door adverting, which has made its object that visible datum of the kind already described, has arisen and ceased, then, making that same visible datum the object, all should be understood in the way already described beginning, “Either four or five impulsions impel …” (\hyperlink{XIII.5}{§5}{}) Here also the [three or four] prior consciousnesses are of the sense sphere and have applied and sustained thought. The last of these consciousnesses, which accomplishes the aim, is of the fine-material sphere belonging to the fourth jhāna. Knowledge conascent with that is called “knowledge of the passing away and reappearance of beings” and “knowledge of the divine eye.”

                The explanation of knowledge of passing away and reappearance is ended.
            \section[\vismAlignedParas{§102–129}General]{General}

                \vismParagraph{XIII.102}{102}{}
                
                \begin{verse}
                    The Helper, knower of five aggregates,\\{}
                    Had these five direct-knowledges to tell;\\{}
                    When they are known, there are concerning them\\{}
                    These general matters to be known as well.
                \end{verse}


                \vismParagraph{XIII.103}{103}{}
                Among these, the divine eye, called knowledge of passing away and reappearance, has two accessory kinds of knowledge, that is to say, “knowledge of the future” and “knowledge of faring according to deeds.” So these two along with the five beginning with the kinds of supernormal power make seven kinds of direct-knowledge given here.

                \vismParagraph{XIII.104}{104}{}
                Now, in order to avoid confusion about the classification of their objects:
                \begin{verse}
                    The Sage has told four object triads\\{}
                    By means of which one can infer\\{}
                    Just how these seven different kinds\\{}
                    Of direct-knowledges occur.
                \end{verse}


                \vismParagraph{XIII.105}{105}{}
                \marginnote{\textcolor{teal}{\footnotesize\{480|422\}}}{}Here is the explanation. Four object triads have been told by the Greatest of the Sages. What four? The limited-object triad, the path-object triad, the past-object triad, and the internal-object triad.\footnote{\vismAssertFootnoteCounter{20}\vismHypertarget{XIII.n20}{}See \emph{Abhidhamma Mātikā} (“schedule”), \textbf{\cite{Dhs}1f.} This consists of 22 sets of triple classifications \emph{(tika) }and 100 sets of double ones \emph{(duka). }The first triad “profitable, unprofitable, and [morally] indeterminate,” and the first dyad is “root-cause, not-root-cause.” The \emph{Mātikā} is used in the Dhammasaṅgaṇī (for which it serves as the basic structure), in the Vibhaṅga (in some of the “Abhidhamma Sections” and in the “Questionnaires”) and in the Paṭṭhāna. All dhammas are either classifiable according to these triads and dyads, under one of the headings, if the triad or dyad is all-embracing, or are called “not-so-classifiable” \emph{(na-vattabba), }if the triad or dyad is not. The four triads mentioned here are: no. 13, “dhammas with a limited object, with an exalted object, with a measureless object”; no. 16, “dhammas with a path as object, with a path as root-cause, with path as predominance”; no. 19, “dhammas with a past object, with a future object, with a present object”; and no. 21, “dhammas with an internal object, with an external object, with an internal-external object.”}

                \vismParagraph{XIII.106}{106}{}
                (1) Herein, \emph{knowledge of supernormal power} \textcolor{brown}{\textit{[430]}} occurs with respect to seven kinds of object, that is to say, as having a limited or exalted, a past, future or present, and an internal or external object. How?

                When he wants to go with an invisible body after making the body dependent on the mind, and he converts the body to accord with the mind (\hyperlink{XII.119}{XII.119}{}), and he sets it, mounts it, on the exalted consciousness, then taking it that the [word in the] accusative case is the proper object,\footnote{\vismAssertFootnoteCounter{21}\vismHypertarget{XIII.n21}{}The “word in the accusative case” is in the first instance “body,” governed by the verb “converts” \emph{(kāyaṃ pariṇāmeti); }see \textbf{\cite{Vism-mhṭ}}.} it \emph{has a limited object }because its object is the material body. When he wants to go with a visible body after making the mind dependent on the body and he converts the mind to accord with the body and sets it, mounts it, on the material body, then taking it that the [word in the] accusative case is the proper object, it \emph{has an exalted object }because its object is the exalted consciousness.

                \vismParagraph{XIII.107}{107}{}
                But that same consciousness takes what has passed, has ceased, as its object, therefore it \emph{has a past object}. In those who resolve about the future, as in the case of the Elder Mahā Kassapa in the Great Storing of the Relics, and others, it has a future object. When the Elder Mahā Kassapa was making the great relic store, it seems, he resolved thus, “During the next two hundred and eighteen years in the future let not these perfumes dry up or these flowers wither or these lamps go out,” and so it all happened. When the Elder Assagutta saw the Community of Bhikkhus eating dry food in the Vattaniya Lodging he resolved thus, “Let the water pool become cream of curd every day before the meal,” and when the water was taken before the meal it was cream of curd; but after the meal there was only the normal water.\footnote{\vismAssertFootnoteCounter{22}\vismHypertarget{XIII.n22}{}\textbf{\cite{Vism-mhṭ}} comments: “Although with the words: \emph{‘These perfumes,’ }etc., he apprehends present perfumes, etc., nevertheless the object of his resolving consciousness is actually their future materiality that is to be associated with the distinction of not drying up. This is because the resolve concerns the future … \emph{‘Cream of curd’: }when resolving, his object is the future appearance of curd.”

                        Vattanīyasenāsana was apparently a monastery in the Vindhya Hills (\emph{Viñjaṭavī}): see \textbf{\cite{Mhv}XIX.6}; \textbf{\cite{Dhs-a}419}. The Elders Assagutta and Rohaṇa instructed Kajaṅgala who was sent to convert Menander (Lamotte, \emph{Histoire de la Bouddhisme Indien,} p. 440).}

                \vismParagraph{XIII.108}{108}{}
                \marginnote{\textcolor{teal}{\footnotesize\{481|423\}}}{}At the time of going with an invisible body after making the body dependent on the mind it \emph{has a present object}.

                At the time of converting the mind to accord with the body, or the body to accord with the mind, and at the time of creating one’s own appearance as a boy, etc., it \emph{has an internal object} because it makes one’s own body and mind its object. But at the time of showing elephants, horses, etc., externally it has \emph{an external object}.

                This is how, firstly, the kinds of supernormal power should be understood to occur with respect to the seven kinds of object.

                \vismParagraph{XIII.109}{109}{}
                (2) \emph{Knowledge of the divine ear element} occurs with respect to four kinds of object, that is to say, as having a limited, and a present, and an internal or external object. How?

                Since it makes sound its object and since sound is limited (see \textbf{\cite{Vibh}74}), it therefore has a limited object.\footnote{\vismAssertFootnoteCounter{23}\vismHypertarget{XIII.n23}{}Cf. also \textbf{\cite{Vibh}62} and 91.} But since it occurs only by making existing sound its object, it \emph{has a present object}. At the time of hearing sounds in one’s own belly it has an internal object. At the time of hearing the sounds of others it \emph{has an external object}. \textcolor{brown}{\textit{[431]}} This is how the knowledge of the divine ear element should be understood to occur with respect to the four kinds of object.

                \vismParagraph{XIII.110}{110}{}
                (3) \emph{Knowledge of penetration of minds} occurs with respect to eight kinds of object, that is to say, as having a limited, exalted or measureless object, path as object, and a past, future or present object, and an external object. How?

                At the time of knowing others’ sense-sphere consciousness it \emph{has a limited object}. At the time of knowing their fine-material-sphere or immaterial-sphere consciousness it \emph{has an exalted object}. At the time of knowing path and fruition it has a measureless object. And here an ordinary man does not know a stream-enterer’s consciousness, nor does a stream-enterer know a once-returner’s, and so up to the Arahant’s consciousness. But an Arahant knows the consciousness of all the others. And each higher one knows the consciousnesses of all those below him. This is the difference to be understood. At the time when it has path consciousness as its object it \emph{has path as object}. But when one knows another’s consciousness within the past seven days, or within the future seven days, then it \emph{has a past object} and \emph{has a future object} respectively.

                \vismParagraph{XIII.111}{111}{}
                How does it have a present object? “Present” (\emph{paccuppanna}) is of three kinds, that is to say, present by moment, present by continuity, and present by extent. Herein, what has reached arising (\emph{uppāda}), presence (\emph{ṭhiti}), and dissolution (\emph{bhaṅga}) is \emph{present by moment}. What is included in one or two rounds of continuity is \emph{present by continuity}.

                \vismParagraph{XIII.112}{112}{}
                Herein, when someone goes to a well-lit place after sitting in the dark, an object is not clear at first; until it becomes clear, one or two rounds of continuity \marginnote{\textcolor{teal}{\footnotesize\{482|424\}}}{}should be understood [to pass] meanwhile. And when he goes into an inner closet after going about in a well-lit place, a visible object is not immediately evident at first; until it becomes clear, one or two rounds of continuity should be understood [to pass] meanwhile. When he stands at a distance, although he sees the alterations (movements) of the hands of washer-men and the alterations (movements) of the striking of gongs, drums, etc., yet he does not hear the sound at first (see \hyperlink{XIV.n22}{Ch. XIV n. 22}{}); until he hears it, one or two rounds of continuity should be understood [to pass] meanwhile. This, firstly, is according to the Majjhima reciters.

                \vismParagraph{XIII.113}{113}{}
                The Saṃyutta reciters, however, say that there are two kinds of continuity, that is to say, material continuity and immaterial continuity: that a material continuity lasts as long as the [muddy] line of water touching the bank when one treads in the water takes to clear,\footnote{\vismAssertFootnoteCounter{24}\vismHypertarget{XIII.n24}{}\textbf{\cite{Vism-mhṭ}} adds: “Some however explain the meaning in this way: It is as long as, when one has stepped on the dry bank with a wet foot, the water line on the foot does not disappear.”} as long as the heat of the body in one who has walked a certain extent takes to die down, as long as the blindness in one who has come from the sunshine into a room does not depart, as long as when, after someone has been giving attention to his meditation subject in a room and then opens the shutters by day and looks out, the dazzling in his eyes does not die down; and that an immaterial continuity consists in two or three rounds of impulsions. Both of these are [according to them] called “present by continuity.” \textcolor{brown}{\textit{[432]}}

                \vismParagraph{XIII.114}{114}{}
                What is delimited by a single becoming (existence) is called \emph{present by extent}, with reference to which it is said in the Bhaddekaratta Sutta: “Friends, the mind and mental objects are both what is present. Consciousness is bound by desire and greed for what is present. Because consciousness is bound by desire and greed he delights in that. When he delights in that, then he is vanquished with respect to present states” (\textbf{\cite{M}III 197}).

                And here, “present by continuity” is used in the Commentaries while “present

                \vismParagraph{XIII.115}{115}{}
                by extent” is used in the Suttas. Herein, some\footnote{\vismAssertFootnoteCounter{25}\vismHypertarget{XIII.n25}{}The residents of the Abhayagiri Monastery in Anurādhapura (\textbf{\cite{Vism-mhṭ}}).} say that consciousness “present by moment” is the object of knowledge of penetration of minds. What reason do they give? It is that the consciousness of the possessor of supernormal power and that of the other arise in a single moment. Their simile is this: just as when a handful of flowers is thrown into the air, the stalk of one flower is probably struck by the stalk of another, and so too, when with the thought, “I will know another’s mind,” the mind of a multitude is adverted to as a mass, then the mind of one is probably penetrated by the mind of the other either at the moment of arising or at the moment of presence or at the moment of dissolution.

                \vismParagraph{XIII.116}{116}{}
                That, however, is rejected in the Commentaries as erroneous, because even if one went on adverting for a hundred or a thousand years, there is never co-presence of the two consciousnesses, that is to say, of that with which he adverts \marginnote{\textcolor{teal}{\footnotesize\{483|425\}}}{}and that [of impulsion] with which he knows, and because the flaw of plurality of objects follows if presence [of the same object] to both adverting and impulsion is not insisted on. What should be understood is that the object is present by continuity and present by extent.

                \vismParagraph{XIII.117}{117}{}
                Herein, another’s consciousness during a time measuring two or three cognitive series with impulsions extending before and after the [strictly] currently existing cognitive series with impulsions, is all called “present by continuity.” But in the Saṃyutta Commentary it is said that “present by extent” should be illustrated by a round of impulsions.

                \vismParagraph{XIII.118}{118}{}
                That is rightly said. Here is the illustration. The possessor of supernormal-power who wants to know another’s mind adverts. The adverting [consciousness] makes [the other’s consciousness that is] present by moment its object and ceases together with it. After that there are four or five impulsions, of which the last is the supernormal-power consciousness, the rest being of the sense sphere. That same [other’s] consciousness, which has ceased, is the object of all these too, and so they do not have different objects because they have an object that is “present by extent.” And while they have a single object it is only the supernormal-power consciousness that actually knows another’s consciousness, not the others, just as in the eye-door it is only eye-consciousness that actually sees the visible datum, not the others.

                \vismParagraph{XIII.119}{119}{}
                So this has a present object in what is present by continuity and what is present by extent. \textcolor{brown}{\textit{[433]}} Or since what is present by continuity falls within what is present by extent, it can therefore be understood that it has a present object simply in what is present by extent.

                It \emph{has an external object} because it has only another’s mind as its object.

                This is how knowledge of penetration of minds should be understood to occur with respect to the eight kinds of objects.

                \vismParagraph{XIII.120}{120}{}
                (4) \emph{Knowledge of past lives} occurs with respect to eight kinds of object, that is to say, as having a limited, exalted, or measureless object, path as object, a past object, and an internal, external, or not-so-classifiable object. How?

                At the time of recollecting sense-sphere aggregates it \emph{has a limited object}. At the time of recollecting fine-material-sphere or immaterial-sphere aggregates it \emph{has an exalted object}. At the time of recollecting a path developed, or a fruition realized, in the past either by oneself or by others, it \emph{has a measureless object}. At the time of recollecting a path developed it \emph{has a path as object}. But it invariably \emph{has a past object}.

                \vismParagraph{XIII.121}{121}{}
                Herein, although knowledge of penetration of minds and knowledge of faring according to deeds also have a past object, still, of these two, the object of the knowledge of penetration of minds is only consciousness within the past seven days. It knows neither other aggregates nor what is bound up with aggregates [that is, name, surname, and so on]. It is said indirectly that it has a path as object since it has the consciousness associated with the path as its object. Also, the object of knowledge of faring according to deeds is simply past volition. But there is nothing, whether past aggregates or what is bound up \marginnote{\textcolor{teal}{\footnotesize\{484|426\}}}{}with aggregates, that is not the object of knowledge of past lives; for that is on a par with omniscient knowledge with respect to past aggregates and states bound up with aggregates. This is the difference to be understood here.

                \vismParagraph{XIII.122}{122}{}
                This is the method according to the Commentaries here. But it is said in the Paṭṭhāna: “Profitable aggregates are a condition, as object condition, for knowledge of supernormal power, for knowledge of penetration of minds, for knowledge of past lives, for knowledge of faring according to deeds, and for knowledge of the future” (\textbf{\cite{Paṭṭh}I 154}), and therefore four aggregates are also the objects of knowledge of penetration of minds and of knowledge of faring according to deeds. And there too profitable and unprofitable [aggregates are the object] of knowledge of faring according to deeds.

                \vismParagraph{XIII.123}{123}{}
                At the time of recollecting one’s own aggregates it \emph{has an internal object}. At the time of recollecting another’s aggregates it \emph{has an external object}. At the time of recollecting [the concepts consisting in] name, race (surname) in the way beginning, “In the past there was the Blessed One Vipassin. His mother was Bhandumatī. His father was Bhandumant” (see \textbf{\cite{D}II 6–7}), and [the concept consisting in] the sign of the earth, etc., it \emph{has a not-so-classifiable object}. And here the name and race (surname, lineage) must be regarded not as the actual words but as the meaning of the words, which is established by convention and bound up with aggregates. For the actual words \textcolor{brown}{\textit{[434]}} are “limited” since they are included by the sound base, according as it is said: “The discrimination of language has a limited object” (\textbf{\cite{Vibh}304}). Our preference here is this.

                This is how the knowledge of past lives should be understood to occur with respect to the eight kinds of object.

                \vismParagraph{XIII.124}{124}{}
                (5) \emph{Knowledge of the divine eye} occurs with respect to four kinds of object, that is to say, as having a limited, a present, and an internal or external object. How? Since it makes materiality its object and materiality is limited (see \textbf{\cite{Vibh}62}) it therefore \emph{has a limited object}. Since it occurs only with respect to existing materiality it \emph{has a present object}. At the time of seeing materiality inside one’s own belly, etc., it \emph{has an internal object}. At the time of seeing another’s materiality it \emph{has an external object}. This is how the knowledge of the divine eye should be understood to occur with respect to the four kinds of object.

                \vismParagraph{XIII.125}{125}{}
                (6) \emph{Knowledge of the future} occurs with respect to eight kinds of object, that is to say, as having a limited or exalted or immeasurable object, a path as object, a future object, and an internal, external, or not-so classifiable object. How? At the time of knowing this, “This one will be reborn in the future in the sense sphere,” it \emph{has a limited object}. At the time of knowing, “He will be reborn in the fine-material or immaterial sphere,” it \emph{has an exalted object}. At the time of knowing, “He will develop the path, he will realize fruition,” it \emph{has an immeasurable object}. At the time of knowing, “He will develop the path,” it \emph{has a path as object} too. But it invariably \emph{has a future object}.

                \vismParagraph{XIII.126}{126}{}
                Herein, although knowledge of penetration of minds has a future object too, nevertheless its object is then only future consciousness that is within seven days; for it knows neither any other aggregate nor what is bound up with \marginnote{\textcolor{teal}{\footnotesize\{485|427\}}}{}aggregates. But there is nothing in the future, as described under the knowledge of past lives (\hyperlink{XIII.121}{§121}{}), that is not an object of knowledge of the future.

                \vismParagraph{XIII.127}{127}{}
                At the time of knowing, “I shall be reborn there,” it \emph{has an internal object}. At the time of knowing, “So-and-so will be reborn there,” it \emph{has an external object}. But at the time of knowing name and race (surname) in the way beginning, “In the future the Blessed One Metteyya will arise. His father will be the brahman Subrahmā. His mother will be the brahmani Brahmavatī” (see \textbf{\cite{D}III 76}), it has a not-so-classifiable object in the way described under knowledge of past lives (\hyperlink{XIII.123}{§123}{}).

                This is how the knowledge of the future should be understood.

                \vismParagraph{XIII.128}{128}{}
                (7) \emph{Knowledge of faring according to deeds} occurs with respect to five kinds of object, that is to say, as having a limited or exalted, a past, and an internal or external object. How? At the time of knowing sense-sphere kamma (deeds) it \emph{has a limited object}. \textcolor{brown}{\textit{[435]}} At the time of knowing fine-material-sphere or immaterial-sphere kamma it \emph{has an exalted object}. Since it knows only what is past it \emph{has a past object}. At the time of knowing one’s own kamma it \emph{has an internal object}. At the time of knowing another’s kamma it \emph{has an external object}. This is how the knowledge of faring according to deeds should be understood to occur with respect to the five kinds of object.

                \vismParagraph{XIII.129}{129}{}
                And when [the knowledge] described here both as “having an internal object” and “having an external object” knows [these objects] now internally and now externally, it is then said that it \emph{has an internal-external object} as well.

                The thirteenth chapter concluding “The Description of Direct-knowledge” in the \emph{Path of Purification} composed for the purpose of gladdening good people.
    
    \part[Understanding (\emph{Paññā})]{Understanding (\emph{Paññā})\vismHypertarget{pIII}}
        \label{pIII}


        \chapter[The Aggregates]{The Aggregates\vismHypertarget{XIV}\newline{\textnormal{\emph{Khandha-niddesa}}}}
            \label{XIV}

            \section[\vismAlignedParas{§1–32}A. Understanding]{A. Understanding}

                \vismParagraph{XIV.1}{1}{}
                \marginnote{\textcolor{teal}{\footnotesize\{489|431\}}}{}\textcolor{brown}{\textit{[436]}} Now, concentration was described under the heading of \emph{consciousness }in the stanza:
                \begin{verse}
                    When a wise man, established well in virtue,\\{}
                    Develops consciousness and understanding (\hyperlink{I.1}{I.1}{}).
                \end{verse}


                And that has been developed in all its aspects by the bhikkhu who is thus possessed of the more advanced development of concentration that has acquired with direct-knowledge the benefits [described in Chs. XII and XIII]. But \emph{understanding} comes next and that has still to be developed. Now, that is not easy, firstly even to know about, let alone to develop, when it is taught very briefly. In order, therefore, to deal with the detailed method of its development there is the following set of questions:

                
                    \begin{enumerate}[(i),nosep]
                        \item What is understanding?
                        \item In what sense is it understanding?
                        \item What are its characteristic, function, manifestation, and proximate cause?
                        \item How many kinds of understanding are there?
                        \item How is it developed?
                        \item What are the benefits of developing understanding?
                    \end{enumerate}

                \vismParagraph{XIV.2}{2}{}
                Here are the answers:
                \subsection[\vismAlignedParas{§2}(i) What is understanding?]{(i) What is understanding?}

                    Understanding (\emph{paññā}) is of many sorts and has various aspects. An answer that attempted to explain it all would accomplish neither its intention nor its purpose, and would, besides, lead to distraction; so we shall confine ourselves to the kind intended here, which is understanding consisting in insight knowledge associated with profitable consciousness.
                \subsection[\vismAlignedParas{§3–6}(ii) In what sense is it understanding?]{(ii) In what sense is it understanding?}

                    \vismParagraph{XIV.3}{3}{}
                    (ii) \textsc{In what sense is it understanding?} It is understanding (\emph{paññā}) in the sense of act of understanding (\emph{pajānana}).\footnote{\vismAssertFootnoteCounter{1}\vismHypertarget{XIV.n1}{}Cf. \textbf{\cite{Paṭis}I 42}, etc.; Abhidhamma definitions very commonly make use of the Pali forms of verbal nouns, here instanced by \emph{paññā }(understanding = state of understanding) and \emph{pajānana }(understanding = act of understanding), both from the verb \emph{pajānāti }(he understands). English does not always, as in this case, distinguish between the two. Similarly, for example, from the verb \emph{socati }(he sorrows) we find \emph{soka }(sorrow, state of sorrowing) and \emph{socana }(sorrowing, act of sorrowing), and here the English differentiates. Cf. parallel treatment of\emph{ paññā }at \textbf{\cite{M-a}II 343f.}} What is this act of understanding? It is \marginnote{\textcolor{teal}{\footnotesize\{490|432\}}}{}knowing (\emph{jānana}) in a particular mode separate from the modes of perceiving (\emph{sañjānana}) and cognizing (\emph{vijānana}). \textcolor{brown}{\textit{[437]}} For though the state of knowing (\emph{jānana-bhāva}) is equally present in perception (\emph{saññā}), in consciousness (\emph{viññāṇa}), and in understanding (\emph{paññā}), nevertheless perception is only the mere perceiving of an object as, say, blue or yellow; it cannot bring about the penetration of its characteristics as impermanent, painful, and not-self. Consciousness knows the objects as blue or yellow, and it brings about the penetration of its characteristics, but it cannot bring about, by endeavouring, the manifestation of the [supramundane] path. Understanding knows the object in the way already stated, it brings about the penetration of the characteristics, and it brings about, by endeavouring, the manifestation of the path.

                    \vismParagraph{XIV.4}{4}{}
                    Suppose there were three people, a child without discretion, a villager, and a money-changer, who saw a heap of coins lying on a money-changer’s counter. The child without discretion knows merely that the coins are figured and ornamented, long, square or round; he does not know that they are reckoned as valuable for human use and enjoyment. And the villager knows that they are figured and ornamented, etc., and that they are reckoned as valuable for human use and enjoyment; but he does not know such distinctions as, “This one is genuine, this is false, this is half-value.” The money-changer knows all those kinds, and he does so by looking at the coin, and by listening to the sound of it when struck, and by smelling its smell, tasting its taste, and weighing it in his hand, and he knows that it was made in a certain village or town or city or on a certain mountain or by a certain master. And this may be understood as an illustration.

                    \vismParagraph{XIV.5}{5}{}
                    Perception is like the child without discretion seeing the coin, because it apprehends the mere mode of appearance of the object as blue and so on. Consciousness is like the villager seeing the coin, because it apprehends the mode of the object as blue, etc., and because it extends further, reaching the penetration of its characteristics. Understanding is like the money-changer seeing the coin, because, after apprehending the mode of the object as blue, etc., and extending to the penetration of the characteristics, it extends still further, reaching the manifestation of the path.

                    \vismParagraph{XIV.6}{6}{}
                    That is why this act of understanding should be understood as “knowing in a particular mode separate from the modes of perceiving and cognizing.” For that is what the words “it is understanding in the sense of act of understanding” refer to. However, it is not always to be found where perception and consciousness are.\footnote{\vismAssertFootnoteCounter{2}\vismHypertarget{XIV.n2}{}“In arisings of consciousness with two root-causes [i.e. with non-greed and non-hate but without non-delusion], or without root-cause, understanding does not occur” (\textbf{\cite{Vism-mhṭ}432}). “Just as pleasure is not invariably inseparable from happiness, so perception and consciousness are not invariably inseparable from understanding. But just as happiness is invariably inseparable from pleasure, so understanding is invariably inseparable from perception and consciousness” (\textbf{\cite{Vism-mhṭ}432}).} \textcolor{brown}{\textit{[438]}} But when it is, it is not disconnected from those states. And because it cannot be taken as disconnected thus: “This is perception, this is consciousness, this is understanding,” its difference is consequently subtle and hard to see. Hence the venerable Nāgasena said: “A difficult thing, O King, has been done by the Blessed One.”—“What, venerable Nāgasena, is the difficult thing that has been done by the Blessed One?”—“The difficult thing, O King, done by the Blessed One was the \marginnote{\textcolor{teal}{\footnotesize\{491|433\}}}{}defining of the immaterial states of consciousness and its concomitants, which occur with a single object, and which he declared thus: ‘This is contact, this is feeling, this is perception, this is volition, this is consciousness’” (\textbf{\cite{Mil}87}).
                \subsection[\vismAlignedParas{§7}(iii) What are its characteristic, etc.?]{(iii) What are its characteristic, etc.?}

                    \vismParagraph{XIV.7}{7}{}
                    (iii) \textsc{What are its characteristic, function, manifestation and proximate cause? }Understanding has the characteristic of penetrating the individual essences of states.\footnote{\vismAssertFootnoteCounter{3}\vismHypertarget{XIV.n3}{}“A phenomenon’s own essence (\emph{sako bhāvo}) or existing essence \emph{(samāno vā bhāva) }is its individual essence \emph{(sabhāva})” (\textbf{\cite{Vism-mhṭ}433}). Cf. \hyperlink{VIII.n68}{Ch. VIII, note 68}{}, where Vism-mhṭ gives the definition from \emph{saha-bhāva }(with essence).} Its function is to abolish the darkness of delusion, which conceals the individual essences of states. It is manifested as non-delusion. Because of the words, “One who is concentrated knows and sees correctly” (\textbf{\cite{A}V 3}), its proximate cause is concentration.
                \subsection[\vismAlignedParas{§8–31}(iv) How many kinds of understanding are there?]{(iv) How many kinds of understanding are there?}

                    \vismParagraph{XIV.8}{8}{}
                    (iv) \textsc{How many kinds of understanding are there?}

                    
                        \begin{enumerate}[1.,nosep]
                            \item Firstly, as having the characteristic of penetrating the individual essences of states, it is of one kind.
                            \item As mundane and supramundane it is of two kinds.
                            \item Likewise as subject to cankers and free from cankers, and so on,
                            \item As the defining of mentality and of materiality,
                            \item As accompanied by joy or by equanimity,
                            \item As the planes of seeing and of development.
                            \item It is of three kinds as consisting in what is reasoned, consisting in what is learnt (heard), and consisting in development.
                            \item Likewise as having a limited, exalted, or measureless object,
                            \item As skill in improvement, detriment, and means,
                            \item As interpreting the internal, and so on.
                            \item It is of four kinds as knowledge of the four truths,
                            \item And as the four discriminations.\footnote{\vismAssertFootnoteCounter{4}\vismHypertarget{XIV.n4}{}\emph{Paṭisambhidā }is usually rendered by “analysis” (see e.g. \emph{Points of Controversy—Kathāvatthu }translation—pp. 377ff). But the Tipiṭaka explanations of the four \emph{paṭisambhidā }suggest no emphasis on \emph{analysis }rather than \emph{synthesis. }\textbf{\cite{Vism-mhṭ}} gives the following definition of the term: “Knowledge that is classified (\emph{pabheda-gata }= put into a division) under meaning \emph{(attha) }as capable of effecting the explanation and definition of specific characteristics of the meaning class (meaning division) is called \emph{attha-paṭisambhidā; }and so with the other three” (\textbf{\cite{Vism-mhṭ}436}). “Discrimination” has been chosen for \emph{paṭisambhidā }because, while it has the sense of “division,” it does not imply an opposite process as “analysis” does. Also it may be questioned whether the four are well described as “entirely logical”: “entirely epistemological” might perhaps be both less rigid and nearer; for they seem to cover four interlocking fields, namely: meanings of statements and effects of causes (etc.), statements of meanings and causes of effects (etc.), language as restricted to etymological rules of verbal expression, and clarity (or perspicuous inspiration) in marshalling the other three.}
                        \end{enumerate}

                    \vismParagraph{XIV.9}{9}{}
                    \emph{1.} Herein, the singlefold section is obvious in meaning.

                    \emph{2.} As regards the twofold section, the \emph{mundane} is that associated with the mundane path and the \emph{supramundane} is that associated with the supramundane path. So it is of two kinds as mundane and supramundane.

                    \vismParagraph{XIV.10}{10}{}
                    \marginnote{\textcolor{teal}{\footnotesize\{492|434\}}}{}\emph{3.} In the second dyad, that \emph{subject to cankers} is that which is the object of cankers. That \emph{free from cankers} is not their object. This dyad is the same in meaning as the mundane and supramundane. The same method applies to the dyads subject to cankers and free from cankers, associated with cankers and dissociated from cankers (\textbf{\cite{Dhs}3}), and so on. So it is of two kinds as subject to cankers and free from cankers, and so on.

                    \vismParagraph{XIV.11}{11}{}
                    \emph{4.} In the third dyad, when a man wants to begin insight, his understanding of the defining of the four immaterial aggregates is understanding as \emph{defining of mentality}, \textcolor{brown}{\textit{[439]}} and his understanding of the defining of the material aggregate is understanding as \emph{defining of materiality}. So it is of two kinds as the defining of mentality and of materiality.

                    \vismParagraph{XIV.12}{12}{}
                    \emph{5.} In the fourth dyad, understanding belonging to two of the kinds of sense-sphere profitable consciousness, and belonging to sixteen\footnote{\vismAssertFootnoteCounter{5}\vismHypertarget{XIV.n5}{}I.e. the four paths with the first jhāna and those with the second, third, and fourth, out of the five (Vism-mh 434).} of the kinds of path consciousness with four of the jhānas in the fivefold method, is \emph{accompanied by joy}. Understanding belonging to two of the kinds of sense-sphere profitable consciousness, and belonging to (the remaining) four kinds of path consciousness with the fifth jhānas is \emph{accompanied by equanimity}. So it is of two kinds as accompanied by joy or by equanimity.

                    \vismParagraph{XIV.13}{13}{}
                    \emph{6.} In the fifth dyad, understanding belonging to the first path is the \emph{plane of seeing}. Understanding belonging to the remaining three paths is the \emph{plane of development} (see \hyperlink{XXII.127}{XXII.127}{}). So it is of two kinds as the planes of seeing and of development.

                    \vismParagraph{XIV.14}{14}{}
                    \emph{7.} As regards the triads, understanding acquired without hearing from another is that \emph{consisting in what is reasoned} because it is produced by one’s own reasoning. Understanding acquired by hearing from another is that \emph{consisting in what is heard}, because it is produced by hearing. Understanding that has reached absorption, having been somehow produced by (meditative) development, is that consisting in development. And this is said: Herein, what is understanding consisting in what is reasoned? In the spheres of work invented by ingenuity, or in the spheres of craft invented by ingenuity, or in the sorts of science invented by ingenuity, any preference, view, choice, opinion, judgment, liking for pondering over things, that concerns ownership of deeds (\emph{kamma}) or is in conformity with truth or is of such kind as to conform with (the axioms) ‘Materiality is impermanent’ or ‘Feeling … perception … formations … \marginnote{\textcolor{teal}{\footnotesize\{493|435\}}}{}consciousness is impermanent’ that one acquires without hearing it from another—that is called understanding consisting in what is reasoned.

                    (In the spheres ) that one acquires by hearing it from another—that is called understanding consisting in what is learnt (heard).

                    And all understanding in anyone who has attained (an attainment) is understanding consisting in development (\textbf{\cite{Vibh}324–325}).

                    So it is of three kinds as consisting in what is thought out, in what is heard, and in development.

                    \vismParagraph{XIV.15}{15}{}
                    \emph{8. }In the second triad, the understanding that occurs contingent upon sense-sphere states has a \emph{limited} object. That which occurs contingent upon fine-material-sphere states or immaterial-sphere states has an \emph{exalted} object. That is mundane insight. That which occurs contingent upon Nibbāna has a \emph{measureless} object. That is supramundane insight. So it is of three kinds as having a limited, an exalted, or a measureless object.

                    \vismParagraph{XIV.16}{16}{}
                    \emph{9.} In the third triad, it is increase that is called improvement. That is twofold as the elimination of harm and the arousing of good. Skill in improvement is skill in these, according as it is said: Herein, what is skill in improvement? When a man brings these things to mind both unarisen unprofitable things do not arise and arisen unprofitable things are abandoned in him; or when he brings these things to mind \textcolor{brown}{\textit{[440]}} both unarisen profitable things arise and arisen profitable things advance to growth, increase, development, and perfection in him. Whatever here is understanding, act of understanding [for words elided see \textbf{\cite{Dhs}16}] non-delusion, investigation of states, right view, is called skill in improvement (\textbf{\cite{Vibh}325–326}).

                    \vismParagraph{XIV.17}{17}{}
                    Non-increase is what is called\emph{ detriment}. That also is twofold as the diminution of good and the arousing of harm. Skill in detriment is skill in these, according as it is said: “Herein, what is skill in detriment? When a man brings these things to mind, both unarisen profitable things do not arise …” (\textbf{\cite{Vibh}326}), and so on.

                    \vismParagraph{XIV.18}{18}{}
                    But in either of these cases any skill in means to cause the production of such and such things, which skill occurs at that moment and is aroused on that occasion, is what is called \emph{skill in means}, according as it is said: “And all understanding of means thereto is \emph{skill in means}” (\textbf{\cite{Vibh}326}).

                    So it is of three kinds as skill in improvement, in detriment, and in means.

                    \vismParagraph{XIV.19}{19}{}
                    \emph{10.} In the fourth triad, insight-understanding initiated by apprehending one’s own aggregates is \emph{interpreting the internal}.\footnote{\vismAssertFootnoteCounter{6}\vismHypertarget{XIV.n6}{}The word \emph{abhinivisati }with its noun \emph{abhinivesa }means literally “to dwell on,” and so “to adhere,” or “insist.” In the Tipiṭaka it always appears in a bad sense and always appears in contexts with wrong view and clinging (see e.g. \textbf{\cite{M}III 30–31}, \textbf{\cite{Nidd}I 436}, and also \textbf{\cite{Vism-mhṭ}} quoted above at I. 140). However, in the Commentaries, the word appears also in a good sense as at \hyperlink{XIV.130}{XIV.130}{}, \hyperlink{XXI.73}{XXI.73}{} and 83f., and at \textbf{\cite{M-a}I 250} (cf. \emph{saddhaṃ nivisati, }\textbf{\cite{M}II 173}). In this good sense it is synonymous with \emph{right }interpretation of experience. All the bare experience of perception is interpreted by the mind either in the sense of permanence, pleasure, self, which is wrong because it is not confirmed by experience, or in the sense of impermanence, etc., which is right because it is confirmed by experience (see XIV. 130). There is no not interpreting experience, and it is a function of the mind that the interpretation adopted is “dwelt upon,” i.e. insisted upon. And so it is this insistence or interpretation in accordance with reality as confirmed by experience that is the \emph{abhinivesa }of the Commentaries in the good sense. For these reasons the words \emph{interpretation, misinterpretation }and \emph{insistence }have been chosen here as renderings.} That initiated by apprehending another’s aggregates or external materiality not bound up with the faculties, [that is, inanimate matter], is \emph{interpreting the external}. That initiated by apprehending both is \emph{interpreting the internal and external}. So it is of three kinds as interpreting the internal, and so on.

                    \vismParagraph{XIV.20}{20}{}
                    \marginnote{\textcolor{teal}{\footnotesize\{494|436\}}}{}\emph{11.} As regards the tetrads, in the first tetrad, knowledge that occurs contingent upon the truth of suffering is \emph{knowledge of suffering}; knowledge that occurs contingent upon the origin of suffering is \emph{knowledge of the origin of suffering}; knowledge that occurs contingent upon the cessation of suffering is \emph{knowledge of the cessation of suffering}; and knowledge that occurs contingent upon the way leading to the cessation of suffering is \emph{knowledge of the way leading to the cessation of suffering}. So it is of four kinds as knowledge of the four truths.

                    \vismParagraph{XIV.21}{21}{}
                    \emph{12.} In the second tetrad, the four kinds of knowledge classed as that concerned with meaning, etc., are called the \emph{four discriminations}. For this is said: “Knowledge about meaning is the discrimination of meaning (\emph{attha-paṭisambhidā}). Knowledge about law is the discrimination of law (\emph{dhamma-paṭisambhidā}). Knowledge about enunciation of language dealing with meaning and law is the discrimination of language (\emph{nirutti-paṭisambhidā}). Knowledge about kinds of knowledge is discrimination of perspicuity (\emph{paṭibhāna-paṭisambhidā})” (\textbf{\cite{Vibh}293}).

                    \vismParagraph{XIV.22}{22}{}
                    Herein, \emph{meaning} (\emph{attha}) is briefly a term for the fruit of a cause (\emph{hetu}). For in accordance with the cause it is served\footnote{\vismAssertFootnoteCounter{7}\vismHypertarget{XIV.n7}{}\emph{Arīyati—}“to honour, to serve.” Not in PED. Cf. ger. \emph{araṇīya }(\textbf{\cite{M-a}I 21},173), also not in PED, explained by the \emph{Majjhima Nidāya ṭīkā} as “to be honoured” (\emph{payirūpasitabba}).}arrived at, reached, therefore it is called “meaning” (or “purpose”). But in particular the five things, namely, (i) anything conditionally produced, \textcolor{brown}{\textit{[441]}} (ii) Nibbāna, (iii) the meaning of what is spoken, (iv) (kamma-) result, and (v) functional (consciousness), should be understood as \emph{meaning}. When anyone reviews that meaning, any knowledge of his, falling within the category (\emph{pabheda}) concerned with meaning, is the \emph{discrimination of meaning}.

                    \vismParagraph{XIV.23}{23}{}
                    \emph{Law} (\emph{dhamma}) is briefly a term for a condition (\emph{paccaya}). For since a condition necessitates (\emph{dahati}) whatever it may be, makes it occur or allows it to happen, it is therefore called “law” (\emph{dhamma}). But in particular the five things, namely, (i) any cause that produces fruit, (ii) the noble path, (iii) what is spoken, (iv) what is profitable, and (v) what is unprofitable, should be understood as \emph{law}. When anyone reviews that law, any knowledge of his, falling within the category concerned with law, is the \emph{discrimination of law}.

                    \vismParagraph{XIV.24}{24}{}
                    \marginnote{\textcolor{teal}{\footnotesize\{495|437\}}}{}This same meaning is shown in the Abhidhamma by the following analysis:

                    (a) “Knowledge about suffering is the \emph{discrimination of meaning}. Knowledge about the origin of suffering is the \emph{discrimination of law}. [Knowledge about the cessation of suffering is the \emph{discrimination of meaning}. Knowledge about the way leading to the cessation of suffering is the \emph{discrimination of law}] …

                    (b) “Knowledge about cause is the \emph{discrimination of law}. Knowledge about the fruit of a cause is the \emph{discrimination of meaning} …

                    (c) “Knowledge about whatever things are born, become, brought to birth, produced, completed, made manifest, is the \emph{discrimination of meaning}. Knowledge about the things from which those things were born, became, were brought to birth, produced, completed, made manifest, is the \emph{discrimination of law} …

                    (d) “Knowledge about ageing and death is the \emph{discrimination of meaning}. Knowledge about the origin of ageing and death is the \emph{discrimination of law}. [Knowledge about the cessation of ageing and death is the \emph{discrimination of meaning}. Knowledge about the way leading to the cessation of ageing and death is the \emph{discrimination of law}. Knowledge about birth … becoming … clinging … craving … feeling … contact … the sixfold base … mentality-materiality … consciousness … knowledge about formations is the \emph{discrimination of meaning}. Knowledge about the origin of formations is the \emph{discrimination of law}.] Knowledge about the cessation of formations is the \emph{discrimination of meaning}. Knowledge about the way leading to the cessation of formations is the \emph{discrimination of law} …

                    (e) “Here a bhikkhu knows the Dhamma (Law)—the Discourses, Songs, [Expositions, Stanzas, Exclamations, Sayings, Birth Stories, Marvels, and] Answers to Questions—this is called the \emph{discrimination of law}. He knows the meaning of whatever is said thus: ‘This is the meaning of this that was said; this is the meaning of this that was said’—this is called the \emph{discrimination of meaning} …

                    (f) “What states are profitable? On an occasion when profitable consciousness of the sense sphere has arisen [that is accompanied by joy and associated with knowledge, having a visible datum as its object … or a mental datum as its object, or contingent upon whatever it may be, on that occasion there is contact … (for elision see \textbf{\cite{Dhs}§1}) … there is non-wavering]—these things are profitable. Knowledge about these things is the discrimination of law. Knowledge about their result is the discrimination of meaning” … (\textbf{\cite{Vibh}293–295}).\footnote{\vismAssertFootnoteCounter{8}\vismHypertarget{XIV.n8}{}This quotation has been filled out from the Vibhaṅga text for clarity.}

                    \vismParagraph{XIV.25}{25}{}
                    \emph{Knowledge about enunciation of language dealing with meaning and law }(\hyperlink{XIV.21}{§21}{}): there is the language that is individual essence, the usage that has no exceptions,\footnote{\vismAssertFootnoteCounter{9}\vismHypertarget{XIV.n9}{}\emph{Byabhicāra (vyabhicāra): }not in PED; normal grammarian’s term for an “exception.”} and deals with that meaning and that law. Any knowledge falling within the category concerned with the enunciation of that, with the speaking, with the utterance of that, concerned with the root-speech of all beings, the Magadhan \marginnote{\textcolor{teal}{\footnotesize\{496|438\}}}{}language that is individual essence, in other words, the language of law (\emph{dhamma}), [any knowledge that] as soon as it hears it spoken, pronounced, uttered, knows, “This is the individual-essence language; this is not the individual-essence language”—[such knowledge] is \emph{discrimination of language}.\footnote{\vismAssertFootnoteCounter{10}\vismHypertarget{XIV.n10}{}The idea behind the term “individual-essence language” (\emph{sabhāvanirutti}), that is to say, that there is a real name for each thing that is part of that thing’s individual essence, is dealt with at \textbf{\cite{Dhs-a}391–392}. Magadhan as “the root speech of all beings” and the “individual-essence language” is dealt with in greater detail at \textbf{\cite{Vibh-a}387}.} \textcolor{brown}{\textit{[442]}} One who has reached the discrimination of language knows, on hearing the words “\emph{phasso, vedanā},” etc., that that is the individual-essence language, and on hearing “\emph{phassā, vedano},” etc., he knows that that is not the individual-essence language.

                    \vismParagraph{XIV.26}{26}{}
                    \emph{Knowledge about kinds of knowledge} (\hyperlink{XIV.21}{§21}{}): when a man is reviewing and makes any of the foregoing kinds of knowledge the object [of his knowledge], then any knowledge in him that has knowledge as its object is \emph{discrimination of perspicuity}, and so is any knowledge about these aforesaid kinds of knowledge, which is concerned with details of their individual domains, functions, and so on.

                    \vismParagraph{XIV.27}{27}{}
                    And these four kinds of discrimination can be placed in two categories: the plane of the trainer and the plane of the non-trainer. Herein, those of the chief disciples and great disciples come into the category of the non-trainer’s plane. Those of the Elder Ānanda, the householder Citta, the layman Dhammika, the householder Upāli, the laywoman Khujjuttarā, etc., come into the category of the trainer’s plane.

                    \vismParagraph{XIV.28}{28}{}
                    And though they come into the categories of the two planes thus, they are nevertheless distinguishable in five aspects, that is to say, as achievement, mastery of scriptures, hearing, questioning, and prior effort. Herein, \emph{achievement} is the reaching of Arahantship. \emph{Mastery of scriptures} is mastery of the Buddha’s word. \emph{Hearing} is learning the Dhamma carefully and attentively. \emph{Questioning} is discussion of knotty passages and explanatory passages in the texts, commentaries, and so on. \emph{Prior effort} is devotion to insight in the dispensation of former Buddhas, up to the vicinity of [the stages of] conformity and change-of-lineage by one who has practiced [the duty of] going [with the meditation subject on alms round] and coming back [with it].\footnote{\vismAssertFootnoteCounter{11}\vismHypertarget{XIV.n11}{}The expression \emph{garapaccāgatikabhāva }refers to the practice of “carrying the meditation subject to and from the alms round,” which is described at \textbf{\cite{M-a}I 257} in detail. The same expression is also used of a certain kind of refuse-rag (see II. 17).}

                    \vismParagraph{XIV.29}{29}{}
                    Others have said:
                    \begin{verse}
                        A prior effort, and great knowledge,\\{}
                        [Knowledge of] dialects, of scriptures,\\{}
                        And questioning, and then achievement,\\{}
                        And likewise waiting on a teacher,\\{}
                        Success in friends—these are conditions\\{}
                        Productive of discriminations.
                    \end{verse}


                    \vismParagraph{XIV.30}{30}{}
                    \marginnote{\textcolor{teal}{\footnotesize\{497|439\}}}{}Herein, \emph{prior effort} is the same as that already stated. \emph{Great learning} is skill in some science or sphere of craft. \emph{Dialects} means skill in the hundred-and-one tongues, particularly in that of Magadha. \emph{Scriptures} means mastery of the Buddha’s word, even if only of the Chapter of Similes.\footnote{\vismAssertFootnoteCounter{12}\vismHypertarget{XIV.n12}{}“The ‘Chapter of Similes’ is the Chapter of Twin Verses in the Dhammapada (\textbf{\cite{Dhp}1–20}), they say. Others say that it is the Book of Pairs in the First Fifty (MN 31–40)” (\textbf{\cite{Vism-mhṭ}436}).}\emph{Questioning} is questioning about defining the meaning of even a single stanza. \emph{Achievement} is stream-entry … or Arahantship. \emph{Waiting on a teacher} is living with very learned intelligent teachers. \emph{Success in friends} is acquisition of friends such as that. \textcolor{brown}{\textit{[443]}}

                    \vismParagraph{XIV.31}{31}{}
                    Herein, Buddhas and Paccekabuddhas reach the discriminations through prior effort and through achievement. Disciples do so through all these means. And there is no special way of developing a meditation subject in order to attain discriminations. But in trainers the attaining of the discriminations comes about next upon the liberation consisting in trainers’ fruition, and in non-trainers it does so next upon the liberation consisting in non-trainers’ fruition. For the discriminations come to success in Noble Ones only through the noble fruition as the ten powers do in Perfect Ones.

                    So these were the discriminations referred to when it was said above: “It is of four kinds … as the four discriminations” (\hyperlink{XIV.8}{§8}{}).
                \subsection[\vismAlignedParas{§32}(v) How is it developed?]{(v) How is it developed?}

                    \vismParagraph{XIV.32}{32}{}
                    (v) \textsc{How is it developed?} Now, the things classed as aggregates, bases, elements, faculties, truths, dependent origination, etc., are the soil of this understanding, and the [first] two purifications, namely, purification of virtue and purification of consciousness, are its roots, while the five purifications, namely, purification of view, purification by overcoming doubt, purification by knowledge and vision of what is the path and what is not the path, purification by knowledge and vision of the way, and purification by knowledge and vision, are the trunk. Consequently, one who is perfecting these should first fortify his knowledge by learning and questioning about those things that are the “soil” after he has perfected the two purifications that are the “roots,” then he can develop the five purifications that are the “trunk.” This is in brief. The detail is as follows.
            \section[\vismAlignedParas{§33–184}B. Description of the Five Aggregates]{B. Description of the Five Aggregates}

                \vismParagraph{XIV.33}{33}{}
                When it was said above “the things classed as aggregates, bases, elements, faculties, truths, dependent origination, etc., are the soil,” the aggregates here are the five aggregates, that is to say, the materiality aggregate, the feeling aggregate, the perception aggregate, the formations aggregate, and the consciousness aggregate.
                \subsection[\vismAlignedParas{§34–80}The Materiality Aggregate]{The Materiality Aggregate}

                    \vismParagraph{XIV.34}{34}{}
                    Herein, all kinds of states whatsoever that have the characteristic of “being molested” (\emph{ruppana}) by cold, etc., taken all together should be understood as the materiality (\emph{rūpa}) aggregate.

                    \marginnote{\textcolor{teal}{\footnotesize\{498|440\}}}{}\emph{1.} That is of one kind with the characteristic of “being molested.”

                    \emph{2.} It is also of two kinds when classed as (a) primary entity (\emph{bhūta}) and (b) derived [by clinging] (\emph{upādāya}).

                    \vismParagraph{XIV.35}{35}{}
                    Herein (a) \emph{primary materiality} is of four kinds as the earth element, water element, fire element, and air element. Their characteristic, function, and manifestation have been given under the definition of the four elements (\hyperlink{XI.87}{XI.87}{}, \hyperlink{XI.93}{93}{}); but as to the proximate cause, each has the other three as its proximate cause. \textcolor{brown}{\textit{[444]}}

                    \vismParagraph{XIV.36}{36}{}
                    (b) \emph{Derived materiality} is of twenty-four kinds as eye, ear, nose, tongue, body, visible datum, sound, odour, flavour;\footnote{\vismAssertFootnoteCounter{13}\vismHypertarget{XIV.n13}{}“Tangible data are omitted from this list because, not being derived matter, they are included in the primaries” (\textbf{\cite{Vism-mhṭ}442}). They are described as consisting of three of the four primaries, excluding the water (cohesion) element. “What is the materiality of the great primaries? It is the tangible-data base and the water-element” (\textbf{\cite{Dhs}663}). For the whole list see \textbf{\cite{Dhs}596}, in which (N.B.) the heart-basis does not appear. See also note 32 and \hyperlink{XV.n15}{Ch. XV, n. 15}{}.} femininity faculty, masculinity faculty, life faculty, heart-basis; bodily intimation, verbal intimation; space element; lightness of matter, malleability of matter, wieldiness of matter, growth of matter, continuity of matter, ageing of matter, impermanence of matter, and physical nutriment.

                    \vismParagraph{XIV.37}{37}{}
                    \emph{1.} Herein, the \emph{eye’s} characteristic is sensitivity of primary elements that is ready for the impact of visible data; or its characteristic is sensitivity of primary elements originated by kamma sourcing from desire to see.\footnote{\vismAssertFootnoteCounter{14}\vismHypertarget{XIV.n14}{}“Here the first-mentioned characteristic of the eye is described according to the kamma that produces a selfhood, and is common to all of it, and this without touching on differentiation is the cause. The second is according to the specialized kamma generated thus, “Let my eye be thus.” This is what they say. But it can be taken that the first-mentioned characteristic is stated as sensitivity’s interest in lighting up its own objective fields, the five senses’ state of sensitivity being taken as a generality; and that the second is stated as the seeing that is due to the particular division of its own cause, the sensitivities’ cause as the state of kamma being taken as a generality or as a unity. The same method applies to the ear and so on.

                            “Here it may be asked, ‘Is the arising of the faculties of the eye, etc., due to kamma that is one or to kamma that is different?’ Now, the Ancients say, “In both ways.” Herein, firstly, in the case of the arising of an eye, etc., due to kamma that is different there is nothing to be explained since the cause is divided up. But when their arising is due to kamma that is one, how does there come to be differentiation among them? It is due to dividedness in the cause too. For it is craving, in the form of longing for this or that kind of becoming that, itself having specific forms owing to hankering after the sense-bases included in some kind of becoming or other, contrives, acting as decisive-support, the specific divisions in the kamma that generates such a kind of becoming. As soon as the kamma has acquired the differentiation induced by that [hankering] it generates through effort consisting in appropriate ability a multiple fruit with differentiated individual essences, as though it had itself taken on a multiple form. And the ability here need not be understood as anything other than the able state; for it is simply the effort of producing fruit that is differentiated by the differentiation due to the differentiation in its cause. And the fact of this differentiating effort on the part of kamma that is one being the cause of the multiple faculties will be dealt with below as to logic and texts (note 21). Besides, it is told how one kind of consciousness only is the cause of the generation of the sixteen kinds of resultant consciousness and so on; and in the world it is also found that a single paddy seed is the cause of the generation of the ripe, the unripe, the husked, and the unhusked fruit. But what is the use of logical thinking? For the eye, etc., are the fruit of kamma; and kamma-result is exclusively the province of a Buddha’s knowledge” (\textbf{\cite{Vism-mhṭ}444}).}Its function is to \marginnote{\textcolor{teal}{\footnotesize\{499|441\}}}{}pick up [an object]\footnote{\vismAssertFootnoteCounter{15}\vismHypertarget{XIV.n15}{}\emph{Āviñjana—}“picking up”: see \emph{āvijjhati }in PED.} among visible data. It is manifested as the footing of eye-consciousness. Its proximate cause is primary elements born of kamma sourcing from desire to see.

                    \vismParagraph{XIV.38}{38}{}
                    \emph{2.} The \emph{ear’s} characteristic is sensitivity of primary elements that is ready for the impact of sounds; or its characteristic is sensitivity of primary elements originated by kamma sourcing from desire to hear. Its function is to pick up [an object] among sounds. It is manifested as the footing of ear-consciousness. Its proximate cause is primary elements born of kamma sourcing from desire to hear.

                    \vismParagraph{XIV.39}{39}{}
                    \emph{3. }The \emph{nose’s} characteristic is sensitivity of primary elements that is ready for the impact of odours; or its characteristic is sensitivity of primary elements originated by kamma sourcing from desire to smell. Its function is to pick up [an object] among odours. It is manifested as the footing of nose-consciousness. Its proximate cause is primary elements born of kamma sourcing from desire to smell.

                    \vismParagraph{XIV.40}{40}{}
                    \emph{4. }The \emph{tongue’s} characteristic is sensitivity of primary elements that is ready for the impact of flavours; or its characteristic is sensitivity of primary elements originated by kamma sourcing from desire to taste. Its function is to pick up [an object] among flavours. It is manifested as the footing of tongue-consciousness. Its proximate cause is primary elements born of kamma sourcing from desire to taste.

                    \vismParagraph{XIV.41}{41}{}
                    \emph{5. }The body’s characteristic is sensitivity of primary elements that is ready for the impact of tangible data; or its characteristic is sensitivity of primary elements originated by kamma sourcing from desire to touch. Its function is to pick up [an object] among tangible data. It is manifested as the footing of body-consciousness. Its proximate cause is primary elements born of kamma sourcing from desire to touch.

                    \vismParagraph{XIV.42}{42}{}
                    Some,\footnote{\vismAssertFootnoteCounter{16}\vismHypertarget{XIV.n16}{}\emph{“‘Some’ }are certain Mahāsaṅghikas; for among these Vasudhamma says this: ‘In the eye fire is in excess; in the ear, air; in the nose, earth; in the tongue, water; in the body all are equal’” (\textbf{\cite{Vism-mhṭ}444}).} however, say that the eye is sensitivity of primary elements that have fire in excess, and that the ear, nose, and tongue are sensitivity of primary elements that have [respectively] air, earth, and water in excess, and that the body is that of all [four equally]. Others say that the eye is sensitivity of those that have fire in excess, and that the ear, nose, tongue, and body are [sensitivity] of those that have [respectively] aperture, air, water, and earth in excess. They should be asked to quote a sutta. They will certainly not find one.

                    \vismParagraph{XIV.43}{43}{}
                    But some give as their reason that it is because these [several sensitivities] are [respectively] aided by visible data, etc., as qualities of fire, and so on.\footnote{\vismAssertFootnoteCounter{17}\vismHypertarget{XIV.n17}{}\emph{“‘As qualities of fire, and so on’: }[aided] by visible data as the illuminating [quality] of heat, which is called lighting up; by sound [as a quality] of air, by odour [as a quality] of earth, by flavour [as a quality] of the water called spittle—so according to the first theory [that of ‘some’]; and it can be suitably adjusted to accord with the second [that of ‘others’] because they need to be assisted by such and such qualities of primaries what is meant is that they have to be helped in apprehending visible data and so on. This theory holds that the quality is the ability of the eye, etc., to light up [respectively] visible data, etc., only when associated with the reasons that are their accessories consisting of light, etc., and aperture’s state of decisive support for ear-consciousness. Aperture is taken in due order, as are fire, etc., since it is absence of primaries. Or alternatively, when others intend that aperture is a quality of primaries, as visible data, etc., are, then the qualities of primaries are construable in their order thus: [aided] by visible data and light [as a quality] of fire, by sound [as a quality] of aperture called space, by odour [as a quality] of air, by flavour [as a quality] of water, by tangible data [as a quality] of earth” (\textbf{\cite{Vism-mhṭ}445}).} They \marginnote{\textcolor{teal}{\footnotesize\{500|442\}}}{}should be asked, “But who has said that visible data, etc., are qualities of fire and so on? \textcolor{brown}{\textit{[445]}} For it is not possible to say of primary elements, which remain always inseparable,\footnote{\vismAssertFootnoteCounter{18}\vismHypertarget{XIV.n18}{}The four primaries are held to be inseparable and not to exist separate from each other; cf. quotation from the “Ancients” in §45. Vism-mhṭ says: “Excess is in capability, not in quantity, otherwise their inseparability would be illogical” (\textbf{\cite{Vism-mhṭ}451}).} that ‘This is a quality of this one, that is a quality of that one.’”

                    \vismParagraph{XIV.44}{44}{}
                    Then they may say: “Just as you assume, from excess of some primary element in such and such material things, the [respective] functions of upholding (\emph{sandhāraṇa}), etc., for earth, etc., so from finding visibility, etc., [respectively] in a state of excess\footnote{\vismAssertFootnoteCounter{19}\vismHypertarget{XIV.n19}{}\emph{“‘From finding visibility, etc., [respectively] in a state of excess’: }from finding them associated with these differences, namely, the bright visible datum in fire, sound audible through its individual essence in air, the odour beginning with \emph{surabhi }perfume in earth, and the sweet taste in water, thus \emph{‘visible data, etc., are the [respective] qualities of these.’ }This is according to the first theory, and he has stated the conclusion \emph{(uttara) }that follows, beginning with \emph{‘we might assume’ }in terms of that. The second is confuted in the same way. Or alternatively, \emph{‘Then they may say,’ }etc., can be taken as said emphasizing, in order to confute it, the theory of Kaṇāda, which asserts that the eye, etc., are respectively made by fire, space, earth, water, and air, that have visible data, etc., as their respective qualities” (\textbf{\cite{Vism-mhṭ}445}).} in material things that have fire in excess, one may assume that visible data, etc., are [respectively] qualities of these.” They should be told: “We might assume it if there were more odour in cotton, which has earth in excess, than in fermented liquor, which has water in excess, and if the colour of cold water were weaker than the colour of hot water, which has heat in excess.

                    \vismParagraph{XIV.45}{45}{}
                    “But since neither of these is a fact, you should therefore give up conjecturing the difference to be in the supporting primary elements. Just as the natures of visible objects, etc., are dissimilar from each other though there is no difference in the primaries that form a single group, so too are eye-sensitivity, etc., though no other cause of their difference exists.”\footnote{\vismAssertFootnoteCounter{20}\vismHypertarget{XIV.n20}{}In the P.T.S. text and Sinhalese Hewavitarne text the word \emph{ekakalāpe, }“that form a single group,” occurs in this sentence but is not in the Harvard text.} This is how it should be taken. \marginnote{\textcolor{teal}{\footnotesize\{501|443\}}}{}But what is it that is not common to them all?\footnote{\vismAssertFootnoteCounter{21}\vismHypertarget{XIV.n21}{}“If there is no differentiation according to primaries, what then is the reason for the differentiation of the eye, and so on? Though the kamma that is produced by the longing for a selfhood (individual personality) with five sense-bases is one only, still it should be taken as called \emph{‘not common to them all’ }and \emph{‘difference of kamma’ }because it is the cause of the differentiation of the eye, and so on. For it is not a condition for the ear through the same particular difference through which it is a condition for the eye, since, if it were, it would then follow that there was no distinction between the faculties. Because of the words, ‘At the moment of rebirth-linking, exalted volition is a condition, as kamma condition, for the kinds of materiality due to kamma performed’ (Paṭṭh) it must be recognized that a single volition is kamma condition for all the kinds of materiality due to kamma performed that come into existence at the moment of rebirth-linking. For if the volition were different, then, when there came to be the arising of the faculties, it would follow that the materiality due to kamma performed was generated by limited and exalted kamma. And rebirth-linking that is one is not generated by a plurality of kinds of kamma. Thus it is established that the arising of the plurality of the faculties is due to a single kamma” (\textbf{\cite{Vism-mhṭ}446}).} It is the kamma itself that is the reason for their difference. Therefore their difference is due to difference of kamma, not to difference of primary elements; for if there were difference of primary elements, sensitivity itself would not arise, since the Ancients have said: “Sensitivity is of those that are equal, not of those that are unequal.”

                    \vismParagraph{XIV.46}{46}{}
                    Now, among these [sensitivities thus] possessed of difference due to difference of kamma, the eye and the ear apprehend non-contiguous objective fields, since consciousness is caused even if the supporting [primaries] of the objective fields do not adhere to the [faculties’] own supporting primaries.\footnote{\vismAssertFootnoteCounter{22}\vismHypertarget{XIV.n22}{}See also §134 and notes 60, 61. The amplification in this paragraph is from Vism-mhṭ, which continues: “There is another method: the eye and the ear have non-contiguous objective fields because arising of consciousness is caused while their objective fields are separated by an interval and apart \emph{(adhika). }Some say that the ear has a contiguous objective field. If it did, then sound born of consciousness would not be the object of ear-consciousness, for there is no arising externally of what is consciousness-originated. And in the texts sound as object is spoken of as being the object of ear-consciousness without making any distinction. Besides, there would be no defining the direction and position of the sound because it would then have to be apprehended in the place occupied by the possessor of the objective field, as happens in the case of an odour. Consequently it remains in the same place where it arose, if it comes into focus in the ear avenue (so the Burmese ed.). Are not the sounds of washermen [beating their washing on stones] heard later by those who stand at a distance? No; because there is a difference in the way of apprehending a sound according to the ways in which it becomes evident to one nearby and to one at a distance. For just as, because of difference in the way of apprehending the sound of words according to the ways in which it becomes evident to one at a distance and to one nearby, there comes to be [respectively] not apprehending, and apprehending of the differences in the syllables, so also, when the sound of washermen (a) becomes [an occurrence] that is evident throughout from beginning to end to one who is nearby, and (b) becomes an occurrence that is evident in compressed form in the end or in the middle to one who is at a distance, it is because there is a difference in the apprehending and definition, which occur later in the cognitive series of ear-consciousness, that there comes to be the assumption \emph{(abhimāna) }‘Heard faintly is heard later.’ But that sound comes into the ear’s focus at the moment of its own existence and in dependence on the place where it arises (see XIII. 112; \textbf{\cite{Dhs-a}313}). If there is absolutely no successive becoming of sound, how does an echo arise? The sound, though it remains at a distance, is a condition for the arising of an echo and for the vibration of vessels, etc., elsewhere as a magnet \emph{(ayo-kanta) }is for the movement of iron” (\textbf{\cite{Vism-mhṭ}446–447}).} The nose, \marginnote{\textcolor{teal}{\footnotesize\{502|444\}}}{}tongue and body apprehend contiguous objective fields, because consciousness is caused only if their objective fields’ [primaries] adhere to their own supporting [primaries], [that is to say, if the objective fields’ primaries adhere] as support [in the case of odours and flavours], and themselves [directly in the case of tangible data, which are identical with the three primaries excluding water].

                    \vismParagraph{XIV.47}{47}{}
                    \emph{1.} There is what is called the “eye” in the world. That looks like a blue lotus petal and is surrounded by black eyelashes and varied with dark and light circles. The eye [sensitivity as meant] here is to be found in the place in the middle of the black circle surrounded by the white circle in that [feature of the] eye with its accessories where there appears the image of the bodies of those who stand in front of it. It pervades the eye’s seven layers like oil sprinkled on seven layers of cotton. It is assisted by the four primary elements whose [respective] functions are upholding, cohering, maturing, and moving, as a warrior prince is by four nurses whose functions are holding, bathing, dressing, and fanning. It is consolidated by temperature, consciousness, and nutriment; it is maintained by life; it is fu rnished with colour, odour, flavour, etc. (see \hyperlink{XVIII.5}{Ch. XVIII, §5}{}); it is the size of a mere louse’s head; and it duly serves both as physical basis and as door for eye-consciousness, and the rest [of the consciousness of the cognitive series]. \textcolor{brown}{\textit{[446]}}

                    \vismParagraph{XIV.48}{48}{}
                    And this is said by the General of the Dhamma:

                    “The sensitivity with which he sees a visible object Is small and it is subtle, too, no bigger, than a louse’s head.”(?)

                    \vismParagraph{XIV.49}{49}{}
                    \emph{2. }The \emph{ear} [sensitivity] is to be found inside the [feature of the] ear-hole with its accessories in the place that is shaped like a finger-stall and surrounded by fine brown hairs. It is assisted by the elements in the way aforesaid. It is consolidated by temperature, consciousness, and nutriment; it is maintained by life; it is equipped with colour, etc.; and it duly serves both as physical basis and as door for ear-consciousness, and the rest.

                    \vismParagraph{XIV.50}{50}{}
                    \emph{3.} The \emph{nose} [sensitivity] is to be found inside the [feature of the] nose-hole with its accessories in the place shaped like a goat’s hoof. It has assistance, consolidation and maintenance in the way aforesaid; and it duly serves both as physical basis and as door for nose-consciousness, and the rest.

                    \vismParagraph{XIV.51}{51}{}
                    \emph{4.} The \emph{tongue} [sensitivity] is to be found in the middle of the [feature of the] tongue with its accessories in the place shaped like a lotus petal tip. It has assistance, consolidation and maintenance in the way aforesaid; and it duly serves both as physical basis and as door for tongue-consciousness, and the rest.

                    \vismParagraph{XIV.52}{52}{}
                    \marginnote{\textcolor{teal}{\footnotesize\{503|445\}}}{}\emph{5.} The \emph{body} [sensitivity] is to be found everywhere, like a liquid that soaks a layer of cotton, in this physical body where there is matter that is clung to.\footnote{\vismAssertFootnoteCounter{23}\vismHypertarget{XIV.n23}{}\emph{Upādiṇṇa }(also \emph{upādiṇṇaka) }is pp. of \emph{upādiyati }(he clings), from which the noun \emph{upādāna }(clinging) also comes. \emph{Upādiṇṇa-(ka-)rūpa }(clung-to matter) = \emph{kammaja-rūpa }(kamma-born matter): see \textbf{\cite{Dhs}§653}. It is vaguely renderable by “organic or sentient or living matter”; technically, it is matter of the four primaries that is “clung to” \emph{(upādiṇṇa) }or “derived” \emph{(upādāya) }by kamma. Generally taken as a purely Abhidhamma term (Dhs 1), it nevertheless occurs in the Suttas at \textbf{\cite{M}I 185} in the same sense.} It has assistance, consolidation and maintenance in the way aforesaid too; and it duly serves both as physical basis and as door for body-consciousness, and the rest.

                    \vismParagraph{XIV.53}{53}{}
                    Like snakes, crocodiles, birds, dogs, and jackals that gravitate to their own respective resorts, that is to say, termite-mounds, water, space, villages, and charnel grounds, so the eye, etc., should be regarded as gravitating to their own respective resorts, that is to say, visible data, and so on (cf. \textbf{\cite{Dhs-a}314}).

                    \vismParagraph{XIV.54}{54}{}
                    \emph{6. }As regards visible data, etc., which come next, a \emph{visible datum} has the characteristic of impinging on the eye. Its function is to be the objective field of eye-consciousness. It is manifested as the resort of that too. Its proximate cause is the four great primaries. And all the [following] kinds of derived materiality are the same as this. Where there is a difference we shall mention it. This [visible datum] is of various kinds as “blue, yellow” (\textbf{\cite{Dhs}§617}) and so on.

                    \vismParagraph{XIV.55}{55}{}
                    \emph{7.} Sound has the characteristic of impinging on the ear. Its function is to be the object of ear-consciousness. It is manifested as the resort of that too. It is of various kinds as “drum sound, tabour sound” (\textbf{\cite{Dhs}§621}) and so on. \textcolor{brown}{\textit{[447]}}

                    \vismParagraph{XIV.56}{56}{}
                    \emph{8. Odour} has the characteristic of impinging on the nose. Its function is to be the object of nose-consciousness. It is manifested as the resort of that too. It is of various kinds as “root odour, heartwood odour” (\textbf{\cite{Dhs}§625}) and so on.

                    \vismParagraph{XIV.57}{57}{}
                    \emph{9. Flavour} has the characteristic of impinging on the tongue. Its function is to be the object of tongue-consciousness. It is manifested as the resort of that too. It is of various kinds as “root flavour, trunk flavour” (\textbf{\cite{Dhs}§629}) and so on.

                    \vismParagraph{XIV.58}{58}{}
                    \emph{10.} The \emph{femininity faculty} has the female sex as its characteristic. Its function is to show that “this is a female.” It is manifested as the reason for the mark, sign, work, and ways of the female (cf. \textbf{\cite{Dhs}§633}).

                    \emph{11.} The \emph{masculinity faculty} has the male sex as its characteristic. Its function is to show that “this is a male.” It is manifested as the reason for the mark, sign, work, and ways of the male (cf. \textbf{\cite{Dhs}§634}).

                    Both these last are coextensive with the whole body, as body-sensitivity is. But it does not follow that they have to be called either “located in the space where body-sensitivity is located” or “located in the space where that is not located.” Like the natures of visible data, etc., these are not confoundable one with the other.\footnote{\vismAssertFootnoteCounter{24}\vismHypertarget{XIV.n24}{}Ee reads \emph{añnamaññaṃ saṅkaro natthi. }Ae omits \emph{saṅkaro natthi. }The word \emph{saṅkara }in the sense of “confounding” or “error” is not in PED; see Vism concluding verses, PTS ed., p. 711:

                            “Though these things, that is to say, the ‘mark … of the female,’ etc., arise each due to its own condition consisting in kamma, etc., they mostly only do so as modes in a continuity accompanied by the femininity faculty. And so ‘it is manifested as the reason for the mark,’ etc., is said making the femininity faculty their cause.

                            “As regards the ‘mark of the female,’ etc., too, its ‘facultiness’ is stated as predominance, in other words, as a state of cause, because the conditions for the modal matter (\emph{ākāra-rūpa}) consisting of the mark of the female, etc., in a continuity accompanied by faculties do not arise otherwise, and because these kinds of materiality are a condition for apprehending the female. But because the femininity faculty does not generate even the material instances in its own group or maintain or consolidate them, and because it does not so act for the material instances of other groups, it is therefore not called in the text faculty, presence, and non-disappearance conditions, as the life faculty is for the material instances of its group, and as nutriment is for the material instances in succeeding groups. And it is because the mark, etc., are dependent on other conditions that wherever they have predominance its shape is encountered, even in dead and sculptured matter that resembles it. And so too with the masculinity faculty.

                            “And since these two do not occur together in a single continuity, because of the words, ‘Does the masculinity faculty arise in one in whom the femininity faculty arises?—No’ (Yamaka), etc., therefore even in a hermaphrodite there is only one of them at a given moment (see also \textbf{\cite{Dhs-a}323})” (\textbf{\cite{Vism-mhṭ}448}).}

                    \vismParagraph{XIV.59}{59}{}
                    \marginnote{\textcolor{teal}{\footnotesize\{504|446\}}}{}\emph{12.} The \emph{life faculty} has the characteristic of maintaining conascent kinds of matter. Its function is to make them occur. It is manifested in the establishing of their presence. Its proximate cause is primary elements that are to be sustained. And although it has the capacity consisting in the characteristic of maintaining, etc., yet it only maintains conascent kinds of matter at the moment of presence, as water does lotuses and so on. Though states (\emph{dhamma}) arise due to their own conditions, it maintains them, as a wet-nurse does a prince. And it occurs itself only through its connection with the states that occur, like a pilot; it does not cause occurrence after dissolution, because of its own absence and that of what has to be made to occur. It does not prolong presence at the moment of dissolution because it is itself dissolving, like the flame of a lamp when the wick and the oil are getting used up. But it must not be regarded as destitute of power to maintain, make occur, and make present, because it does accomplish each of these functions at the moment stated (cf. \textbf{\cite{Dhs}§635}).\footnote{\vismAssertFootnoteCounter{25}\vismHypertarget{XIV.n25}{}“Since the life faculty is itself entirely kamma-born it is established, by taking them as conascent, that the things to be protected by it are kamma-born too; this is why there is no inclusion of the term ‘kamma-born.’ It maintains as if it were its own that kamma-born matter by being the cause of its occurrence even though only lasting for a moment; that is why it has the characteristic of maintaining conascent kinds of matter. For kamma alone is not competent to be the cause of kamma-born things” presence, as nutriment, etc., are of the nutriment-born. Why? Because it is no longer existent at that moment.

                            “‘Because it does accomplish each of those functions’: it does so because it is a condition for distinguishing what is living. For it is the life faculty that distinguishes matter that is bound up with faculties from dead matter, and kamma-born matter and what is bound up with that from matter that is temperature originated, and so on.

                            “And the life faculty must be regarded as the reason not only for presence during a moment but also for non-interruption of connection; otherwise death as the termination of a life span would be illogical” (\textbf{\cite{Vism-mhṭ}448}).}

                    \vismParagraph{XIV.60}{60}{}
                    \marginnote{\textcolor{teal}{\footnotesize\{505|447\}}}{}\emph{13.} The heart-basis has the characteristic of being the (material) support for the mind-element and for the mind-consciousness-element. Its function is to observe them. It is manifested as the carrying of them. It is to be found in dependence on the blood, of the kind described in the treatise on mindfulness of the body (\hyperlink{VIII.111}{VIII.111}{}), inside the heart. It is assisted by the primaries with their functions of upholding, etc.; it is consolidated by temperature, consciousness, and nutriment; it is maintained by life; and it serves as physical basis for the mind-element and mind-consciousness-element, and for the states associated with them.\footnote{\vismAssertFootnoteCounter{26}\vismHypertarget{XIV.n26}{}“‘The heart-basis … the support for the mind-element and for the mind-consciousness-element’: how is that to be known? (i) From scriptures and (ii) from logical reasoning.

                            “The scripture is this: ‘The materiality dependent on which the mind-element and mind-consciousness-element occur is a condition, as a support condition, for the mind-element and the mind-consciousness-element and what is associated therewith’ (\textbf{\cite{Paṭṭh}} \hyperlink{I.4}{I.4}{}). If that is so, why is it not mentioned in the Rūpakaṇḍa of the Dhammasaṅgaṇi? (\textbf{\cite{Dhs}§583ff.}). Its not being mentioned there is for another reason. What is that? Non-inconsistency of the teaching. For while eye consciousness, etc., have the eye, etc., as their respective supports absolutely, mind-consciousness does not in the same way have the heart-basis as its support absolutely. And the teaching in the physical-basis dyad \emph{(vatthu-duka) }is given by way of the material support thus, ‘There is matter that is the physical basis of eye-consciousness, there is matter that is not the physical basis of eye-consciousness’ (\textbf{\cite{Dhs}§585}) and so on; and if the dyads were stated by way of what had the heart-basis absolutely as its support thus, ‘There is matter that is the physical basis of mind-consciousness’ and so on, then the object dyads \emph{(ārammaṇa-duka) }do not fall into line: for one cannot say: ‘There is matter that is the object of mind-consciousness, there is matter that is not the object of mind-consciousness.’ So the physical-basis dyads and object dyads being thus made inconsistent, the teaching would lack unity, and the Master’s wish was to give the teaching here in a form that has unity. That is why the heart-basis is not mentioned, not because it is unapprehendable.

                            “(ii) But the logical reasoning should be understood in this way. In the five constituent becoming, [that is, in the sense sphere and fine-material sphere,] these two elements have as their support produced \emph{(nipphanna) }derived matter. Herein, since the visible-data base, etc., and nutritive essence, are found to occur apart from what is bound up with faculties, to make them the support would be illogical. And since these two elements are found in a continuity that is devoid of the femininity and masculinity faculties [i.e. in the Brahmā-world], to make them the support would be illogical too. And in the case of the life faculty that would have to have another function, so to make it the support would be illogical too. So it is the heart-basis that remains to be recognized as their support. For it is possible to say that these two elements have as their support produced derived matter, since existence is bound up with matter in the five-constituent becoming. Whatever has its existence bound up with matter is found to have as its support produced derived matter, as the eye-consciousness-element does. And the distinction ‘in the five-constituent becoming’ is made on account of the mind-consciousness-element; in the four-constituent becoming, [that is, the immaterial sphere,] there is no mind-element. Does there not follow contradiction of the middle term \emph{(hetu) }because of establishing faculties as their support? No; because that is disproved by what is seen. For these two elements are not, as in the case of eye-consciousness, controlled by the slackness and keenness, etc., of their physical basis; and accordingly it is not said in the texts that they have the faculties as their condition. Hence their having faculties as their support, in other words, their being controlled by them, is disproved.

                            “Granted that these two elements have as their support the derived matter consisting of the heart-basis, how is it to be known that it is kamma-originated, has an invariable function, and is to be found located in the heart? It may be said to be kamma-originated because, like the eye, it is the materiality of a physical basis; and because of that it has an invariable function; because it is the materiality of a physical basis and because it is a support for consciousness, is the meaning. It is known that its location is there because of the heart’s exhaustion \emph{(khijjana) }in one who thinks of anything, bringing it to mind intently and directing his whole mind to it” (\textbf{\cite{Vism-mhṭ}449–450}).

                            The word \emph{hadaya }(heart), used in a purely mental and not physical sense, occurs in the definitions of the mind-element and mind-consciousness-element in the Vibhaṅga (\textbf{\cite{Vibh}88–89}). The brain \emph{(matthaluṅga), }which seems to have been first added as the 32nd part of the body in the Paṭisambhidā (\textbf{\cite{Paṭis}I 7}), was ignored, and the \emph{Visuddhimagga }is hard put to it to find a use for it. The Piṭakas (e.g. \textbf{\cite{Paṭṭh}} 1,4 quoted above) connect the mind with the matter of the body without specifying.}

                    \vismParagraph{XIV.61}{61}{}
                    \marginnote{\textcolor{teal}{\footnotesize\{506|448\}}}{}\emph{14. Bodily intimation} is the mode (conformation) and the alteration (deformation) in the consciousness-originated air element that causes the occurrence of moving forward, etc., which mode and alteration are a condition for the stiffening, upholding, and moving of the conascent material body. \textcolor{brown}{\textit{[448]}} Its function is to display intention. It is manifested as the cause of bodily excitement. Its proximate cause is the consciousness-originated air element. But it is called “bodily intimation” (\emph{kāya-viññatti}) because it is the cause of the intimating (\emph{viññāpana}) of intention by means of bodily excitement, and because it is itself intimatable through the body, in other words, through that bodily excitement. Moving forward, etc., should be understood to occur owing to the movement of the [kinds of matter] that are temperature-born, etc., which are interlocked with the consciousness-born kinds moved by that [intimation]\footnote{\vismAssertFootnoteCounter{27}\vismHypertarget{XIV.n27}{}“It is the mode and the alteration of what? Of consciousness-originated primary elements that have the air-element in excess of capability. What is that capability? It is the state of being consciousness-born and the state of being derived matter. Or alternatively, it can be taken as the mode alteration of the air element. If that is so, then intimation is illogical as derived matter, for there is no derived matter with a single primary as its support, since ‘matter derived from the four great primaries’ (\textbf{\cite{M}I 53}) is said. That is not wrong. Alteration of one of the four is that of all four, as with wealth shared among four. And excess of air element in a material group \emph{(kalāpa) }does not contradict the words ‘of the air element’; and excess is in capability, not in quantity, otherwise their inseparability would be illogical. According to some it is that of the air element only. In their opinion the state of derived matter is inapplicable \emph{(durupapāda) }to intimation, since the alteration of one is not that of all. But this [air element] is apprehended by mind-door impulsion that is next to the non-intimating [apprehension] that is next to the apprehension of the appearance of motion in the movement of the hands, and so on. There is a certain kind of alteration that is separate from the appearance of motion. And the apprehension of the former is next to the apprehension of the latter. How is that to be known? By the apprehension of intention. For no apprehension of intention such as, ‘He is getting this done, it seems’ is met with in the case of trees’ movements, etc., which are devoid of intention. But it is met with in the case of hand movements and so on. Therefore there is a certain kind of alteration that is separate from the appearance of motion, and it is known as the ‘intimator of the intention.’ Also it is known by inference that the apprehension of the alteration is next to the apprehension of the appearance thus: The intimator intimates the meaning to be intimated only when it is apprehended as a cause, not merely as present. For they say accordingly:

                            Sounds that have entered no objective field Do not awaken any kind of meaning; And also beings merely recognized As such communicate no meanings either.

                            “If just the apprehension of the alteration is the reason for the apprehension of the intention, why is there no apprehension of intention in unapprehended communication \emph{(saṅketa)ī }It is not only just the apprehension of the alteration that is the reason for the apprehension of the intention; but rather it should be taken that the apprehension of the previously-established connection is the decisive support for this. The stiffening, upholding, and movement are due to the air-element associated with the alteration belonging to the intimation, is what is said. What, is it all the air-element that does all those things? It is not like that. For it is the air element given rise to by the seventh impulsion that, by acquiring as its reinforcing conditions the air elements given rise to by the preceding impulsions, moves consciousness-originated matter by acting as cause for its successive arisings in adjacent locations, \emph{(desantaruppatti—}cf. \hyperlink{VIII}{Ch. VIII}{}, n. 54) not the others. The others, however, help it by doing the stiffening and upholding, the successive arising in adjacent locations being itself the movement. So the instrumentality should be taken as attributed when there is the sign [of movement]; otherwise there would not be uninterestedness and momentariness of dhammas. And here the cart to be drawn by seven yokes is given as simile in the Commentary. But when consciousness-born matter moves, the kinds of matter born of temperature, kamma, and nutriment move too because they are bound up with it, like a piece of dry cow-dung thrown into a river’s current.

                            “Since it has been said that the apprehension of intimation is next to the apprehension of the appearance of motion, how then, is the air element itself as the maker of the movement accompanied by the alteration consisting in the intimation? It is not like that. It is the air elements given rise to by the first impulsion, etc., and which are unable to cause movement in that way and perform only the stiffening and upholding, that should be taken as only accompanied by the alteration belonging to intimation. For it is the alteration coexistent with the intention that is the intimation, because of giving rise to alteration in whatever direction it wishes to cause the occurrence of moving forward and so on. Taking it in this way, it is perfectly logical to say that the origination of intimation belongs to mind-door adverting. Since the intention possessed of the aforesaid alteration is intimated through the apprehension of that alteration, it is said that ‘Its function is to display intention.’ The air element being the cause of the motion of the bodily intimation, is figuratively said, as a state of alteration, to be ‘manifested as the cause of bodily motion.’ ‘Its proximate cause is the consciousness-originated air-element’ is said since the air element’s excessive function is the cause of intimating intention by movement of the body” (\textbf{\cite{Vism-mhṭ}450–452}). Cf. \textbf{\cite{Dhs-a}83f.}}(See \textbf{\cite{Dhs}§636}).

                    \vismParagraph{XIV.62}{62}{}
                    \marginnote{\textcolor{teal}{\footnotesize\{508|450\}}}{}\emph{15.} Verbal intimation is the mode (conformation) and the alteration (deformation) in the consciousness-originated earth element that causes that occurrence of speech utterance which mode and alteration are a condition for the knocking together of clung-to matter.\footnote{\vismAssertFootnoteCounter{28}\vismHypertarget{XIV.n28}{}\emph{Vacībheda—}”speech utterance” is not in PED, which does not give this use of \emph{bheda. }\textbf{\cite{Vism-mhṭ}(p. 452)} explains: “The function (—‘knocking together’) of the vocal apparatus (—‘clung-to matter’).”} Its function is to display intention. It is manifested as the cause of the voice in speech. Its proximate cause is the consciousness-originated earth element. But it is called “verbal intimation” because it is the cause of the intimating of intention by means of the voice in speech, and because it is itself intimatable through speech, in other words, through that voice in speech. For, just as, on seeing a sign for water consisting of an ox skull, etc., hung up in the forest, it is intimated that “there is water here,” so too, on noticing either the bodily shaking or the voice in speech thus, they intimate.\footnote{\vismAssertFootnoteCounter{29}\vismHypertarget{XIV.n29}{}“The question, ‘It is the mode and the alteration of what?,’ should be handled in the same way as for bodily intimation, with this difference: for ‘next to the apprehension of the appearance of movement’ substitute ‘next to the hearing of an audible sound.’ And here, because of the absence of stiffening, etc., the argument beginning, ‘For it is the air element given rise to by the seventh impulsion’ does not apply; for the sound arises together with the knocking together, and the knocking together only applies in the case of the first impulsion, and so on. The knocking together is the arising of groups of primaries \emph{(bhūta-kalāpa) }in proximity to each other due to conditions. The movement is the progression of the successive arising in adjacent locations. This is the difference. The earth element’s knocking together is parallel to the air element’s moving as regards function” (\textbf{\cite{Vism-mhṭ}452}).} (See \textbf{\cite{Dhs}§637}.)

                    \vismParagraph{XIV.63}{63}{}
                    \emph{16.} The \emph{space element} has the characteristic of delimiting matter. Its function is to display the boundaries of matter. It is manifested as the confines of matter; or it is manifested as untouchedness, as the state of gaps and apertures (cf. \textbf{\cite{Dhs}§638}). Its proximate cause is the matter delimited. And it is on account of it that one can say of material things delimited that “this is above, below, around, that.”

                    \vismParagraph{XIV.64}{64}{}
                    \emph{17. Lightness of matter} has the characteristic of non-slowness. Its function is to dispel heaviness of matter. It is manifested as light transformability. Its proximate cause is light matter (cf. \textbf{\cite{Dhs}§639}).

                    \emph{18. Malleability of matter} has the characteristic of non-stiffenedness. Its function is to dispel stiffness of matter. It is manifested as non-opposition to any kind of action. Its proximate cause is malleable matter (cf. \textbf{\cite{Dhs}§640}).

                    \emph{19. Wieldiness of matter} has the characteristic of wieldiness that is favourable to bodily action. Its function is to dispel unwieldiness. It is manifested as non-weakness. Its proximate cause is wieldy matter (cf. \textbf{\cite{Dhs}§641}).

                    \vismParagraph{XIV.65}{65}{}
                    These three, however, are not found apart from each other. Still their difference may be understood as follows. \emph{Lightness of matter} is alteration of matter such as any light (agile) state in material instances, as in one who is healthy, any \marginnote{\textcolor{teal}{\footnotesize\{509|451\}}}{}non-slowness, any manner of light transformability in them, which is originated by conditions that prevent any disturbance of elements capable of creating sluggishness of matter. \emph{Malleability of matter} is alteration of matter such as any malleable state in material instances, as in a well-pounded hide, any pliable manner consisting in amenableness to exercise of power over them in all kinds of work without distinction, which \textcolor{brown}{\textit{[449]}} is originated by conditions that prevent any disturbance of elements capable of creating stiffness of matter. \emph{Wieldiness of matter} is alteration of matter such as any wieldy state in material instances, as in well-refined gold, any manner in them consisting in favourableness to the work of the body, which is originated by conditions that prevent any disturbance of elements capable of creating unfavourableness to the work of the body.

                    \vismParagraph{XIV.66}{66}{}
                    \emph{20. Growth of matter} has the characteristic of setting up. Its function is to make material instances emerge in the first instance. It is manifested as launching; or it is manifested as the completed state. Its proximate cause is grown matter.

                    \emph{21. Continuity of matter} has the characteristic of occurrence. Its function is to anchor. It is manifested as non-interruption. Its proximate cause is matter that is to be anchored.

                    Both of these are terms for matter at its birth; but owing to difference of mode, and according to [different persons’] susceptibility to instruction, the teaching in the summary (\emph{uddesa}) in the Dhammasaṅgaṇī is given as “growth and continuity” (cf. \textbf{\cite{Dhs}§596}); but since there is here no difference in meaning, consequently in the description (\emph{niddesa}) of these words, “the setting up of the sense-bases is the growth of matter” and “the growth of matter is the continuity of matter” is said (\textbf{\cite{Dhs}§642}, 732, 865).

                    \vismParagraph{XIV.67}{67}{}
                    And in the Commentary, after saying, “It is genesis that is called ‘setting up,’ increase that is called ‘growth,’ occurrence that is called ‘continuity,’” this simile is given: “Genesis as setting up is like the time when water comes up in a hole dug in a river bank; increase as growth is like the time when it fills [the hole]; occurrence as continuity is like the time when it overflows.” And at the end of the simile it is said: “So what is stated? Setting up is stated by sense-base; sense-base is stated by setting up.” Consequently, it is the first genesis of material instances that is their \emph{setting up}; the genesis also of others that are generated in addition to those is \emph{growth} since it appears in the aspect of increase; the repeated genesis also of others that are generated in addition to those is \emph{continuity} since it appears in the aspect of anchoring. This is how it should be understood to have been declared thus.

                    \vismParagraph{XIV.68}{68}{}
                    \emph{22. Ageing} has the characteristic of maturing (ripening) material instances. Its function is to lead on towards [their termination]. It is manifested as the loss of newness without the loss of individual essence, like oldness in paddy. Its proximate cause is matter that is maturing (ripening). This is said with reference to the kind of ageing that is evident through seeing alteration in teeth, etc., as their brokenness, and so on (cf. \textbf{\cite{Dhs}§644}). But that of immaterial states, which has no such [visible] alteration, is called hidden ageing. And that in earth, water, rocks, the moon, the sun, etc., is called incessant ageing. \textcolor{brown}{\textit{[450]}}

                    \vismParagraph{XIV.69}{69}{}
                    \marginnote{\textcolor{teal}{\footnotesize\{510|452\}}}{}\emph{23. Impermanence of matter} has the characteristic of complete breaking up. Its function is to make material instances subside. It is manifested as destruction and fall (cf. \textbf{\cite{Dhs}§645}). Its proximate cause is matter that is completely breaking up.

                    \vismParagraph{XIV.70}{70}{}
                    \emph{24. Physical nutriment} has the characteristic of nutritive essence. Its function is to feed kinds of matter. It is manifested as consolidating. Its proximate cause is a physical basis that must be fed with physical food. It is a term for the nutritive essence by means of which living beings sustain themselves (cf. \textbf{\cite{Dhs}§646}).

                    \vismParagraph{XIV.71}{71}{}
                    These, firstly, are the material instances that have been handed down in the texts.\footnote{\vismAssertFootnoteCounter{30}\vismHypertarget{XIV.n30}{}In actual fact the \emph{heart-basis }is not in the Piṭakas as such.} But in the Commentary, others have been added as follows: matter as power, matter as procreation, matter as birth, matter as sickness; and, in the opinion of some, matter as torpor.\footnote{\vismAssertFootnoteCounter{31}\vismHypertarget{XIV.n31}{}\emph{“‘Some’ }are the inmates of the Abhayagiri Monastery at Anurādhapura” (Vism-mhṭ 455). A long discussion on this follows in \textbf{\cite{Vism-mhṭ}}, not given here.}

                    In the first place, \emph{matter as torpor} is rejected as non-existent by the words:
                    \begin{verse}
                        Surely thou art a sage enlightened,\\{}
                        There are no hindrances in thee (\textbf{\cite{Sn}541}).
                    \end{verse}


                    As to the rest, \emph{matter as sickness} is included by ageing and by impermanence; \emph{matter as birth} by growth and continuity; \emph{matter as procreation}, by the water element; and \emph{matter as power} by the air element. So taken separately not even one of these exists: this was the agreement reached.

                    So this derived matter of twenty-four sorts and the aforesaid matter of the primary elements, which is of four sorts, together amount to twenty-eight sorts, neither more nor less.

                    \vismParagraph{XIV.72}{72}{}
                    And all that [matter of twenty-eight sorts] is of one kind as “not-root-cause, root-causeless, dissociated from root-cause, with conditions, mundane, subject to cankers” (\textbf{\cite{Dhs}§584}), and so on.

                    It is of two kinds as internal and external, gross and subtle, far and near, produced (\emph{nipphanna}) and unproduced, sensitive matter and insensitive matter, faculty and non-faculty, clung to and not-clung to, and so on.

                    \vismParagraph{XIV.73}{73}{}
                    Herein, the five kinds beginning with the eye are \emph{internal} because they occur as an integral part of the selfhood (in oneself); the rest are \emph{external }because they are external to that selfhood (personality). The nine beginning with the eye and the three elements excepting the water element, making twelve kinds in all, are to be taken as \emph{gross} because of impinging; the rest are \emph{subtle} because they are the opposite of that. What is subtle is \emph{far} because it is difficult to penetrate, the other is \emph{near} because it is easy to penetrate. The eighteen kinds of matter, that is to say, the four elements, the thirteen beginning with the eye, and physical nutriment, are produced because they can be discerned through their own individual essences, having exceeded the [purely conceptual] states of [matter as] delimitation, [matter as] alteration, and [matter as] characteristic (see \hyperlink{XIV.77}{§77}{}); the rest, being the opposite, are \emph{unproduced}. The five kinds beginning with the \marginnote{\textcolor{teal}{\footnotesize\{511|453\}}}{}eye are \emph{sensitive matter} through their being conditions for the apprehension of visible data, etc., because they are, as it were, bright like the surface of a looking glass; the rest are \emph{insensitive matter} because they are the opposite of that. \textcolor{brown}{\textit{[451]}} Sensitive matter itself, together with the three beginning with the femininity faculty, is \emph{faculty} in the sense of predominance; the rest are \emph{not-faculty} because they are the opposite of that. What we shall later describe as “kamma-born” (\hyperlink{XIV.75}{§75}{} and \hyperlink{XX.27}{XX.27}{}) is \emph{clung} to because that is “clung to,” [that is, acquired] by kamma. The rest are \emph{not-clung} to because they are the opposite of that.

                    \vismParagraph{XIV.74}{74}{}
                    Again, all matter is of three kinds according to the visible (\emph{sanidassana}) triad, the kamma-born triad, etc. (see \textbf{\cite{Dhs}2}). Herein, as regards the gross, a visible datum is visible with impact; the rest are invisible with impact; all the subtle kinds are invisible without impact. So firstly it is of three kinds according to the visible triad.

                    \vismParagraph{XIV.75}{75}{}
                    According to the kamma-born triad, etc., however, that born from kamma is \emph{kamma-born}; that born from a condition other than that is \emph{not-kamma-born}; that not born from anything is \emph{neither-kamma-born-nor-not-kamma-born}.

                    That born from consciousness is \emph{consciousness-born}; that born from a condition other than consciousness is \emph{not-consciousness-born}; that not born from anything is \emph{neither-consciousness-born-nor-not-consciousness-born}.

                    That born from nutriment is \emph{nutriment-born}; that born from a condition other than that is \emph{not-nutriment-born}; that not born from anything is \emph{neither-nutriment-born-nor-not-nutriment-born}.

                    That born from temperature is \emph{temperature-born}; that born from a condition other than that is \emph{not-temperature-born}; that not born from anything is \emph{neither-temperature-born-nor-not-temperature-born}.

                    So it is of three kinds according to the kamma-born triad, and so on.

                    \vismParagraph{XIV.76}{76}{}
                    Again, it is of four kinds as seen, etc., as concrete matter, etc., and as the physical basis tetrads, and so on.

                    Herein, the visible-data base is \emph{seen} because it is the objective field of seeing. The sound base is \emph{heard} because it is the objective field of hearing. The three, that is to say, odours, flavours, and tangible data, are \emph{sensed} (lit. contacted) because they are the objective fields of faculties that take contiguous [objective fields]. The rest are \emph{cognized} because they are the objective field of consciousness (cognition) only. So firstly it is of four kinds according to the seen, etc., tetrad.\footnote{\vismAssertFootnoteCounter{32}\vismHypertarget{XIV.n32}{}\emph{“‘Sensed (muta)’ }means apprehendable by sensing \emph{(mutvā), }by reaching; hence he said \emph{‘because they are the objective fields of faculties that take contiguous [objective fields]’ }(cf. §46). But what is it that is called a tangible datum? It is the three elements, earth, heat, and air. But why is the water element not included here? Is not cold apprehended by touching; and that is the water element? Certainly it is apprehended but it is not the water element. What is it then? It is just the fire element. For there is the sensation \emph{(buddhi) }of cold when heat is sluggish. There is no quality that is called cold; there is only the assumption \emph{(abhimāna) }of coldness due to the sluggishness of the state of heat. How is that to be known? Because of the unreliability of the sensation of cold, like ‘beyond and not beyond.’ For in hot weather, while those who stand in the sun and go into the shade have the sensation of cold, yet those who go to the same place from an underground cave have the sensation of heat. And if coldness were the water element it would be found in a single group \emph{(kalāpa) }along with heat; but it is not so found. That is why it may be known that coldness is not the water element. And that is conclusive \emph{(uttara) }for those who agree to the inseparable existence of the primary elements; and it is conclusive too even for those who do not agree because it is disproved by associate existence through seeing the functions of the four primaries in a single group. It is conclusive too for those who say that coldness is the characteristic of the air element; for if coldness were the air element, coldness would be found in a single group along with heat, and it is not so found. That is why it may be known that coldness is not the air element either. But those who hold the opinion that fluidity \emph{(dravatā) }is the water element and that that is apprehended by touching should be told: ‘That fluidity touched is merely the venerable ones’ assumption as is the case with shape.’ For this is said by the Ancients:

                            ‘Three elements coexisting with fluidity Together form what constitutes a tangible; That “I succeed in touching this fluidity” Is a common misconception in the world. And as a man who touches elements, And apprehends a shape then with his mind, Fancies “I really have been touching shape,” So too fluidity is recognized’” (\textbf{\cite{Vism-mhṭ}459}).}

                    \vismParagraph{XIV.77}{77}{}
                    \marginnote{\textcolor{teal}{\footnotesize\{512|454\}}}{}Here, however, “produced matter” is \emph{concrete matter}; the space-element is \emph{delimiting matter}; those from “bodily intimation” up to “wieldiness” are \emph{matter as alteration}; birth, ageing and dissolution are \emph{matter as characteristic}. So it is of four kinds as concrete matter and so on.

                    \vismParagraph{XIV.78}{78}{}
                    Here, however, what is called the materiality of the heart is \emph{physical basis, not door} (see \textbf{\cite{Dhs-a}82f.}); the two intimations are \emph{door, not physical basis}; sensitive matter is \emph{both physical basis and door}; the rest are \emph{neither physical basis nor door}. So it is four kinds according to the physical basis tetrad.

                    \vismParagraph{XIV.79}{79}{}
                    Again, it is of five kinds as born of one, born of two, born of three, born of four, and not born of anything.

                    Herein, what is kamma-born only or consciousness-born only is called \emph{born of one}. Of these, materiality of the faculties, together with the heart-basis, is kamma-born only; the two intimations are consciousness-born only. But what is born [now] of consciousness and [now] of temperature is called \emph{born of two}. That is the sound base only.\footnote{\vismAssertFootnoteCounter{33}\vismHypertarget{XIV.n33}{}“‘\emph{The sound base only’: }here some say, ‘The consciousness-born is always intimative \emph{(saviññattika).’ }The Ancients say, ‘There is sound due to the intervention \emph{(vipphāra) }of applied thought that does not intimate.’ While depending on the word of the Great Commentary that puts it thus, ‘Intimatable (cognizable) through the ear by means of the sound due to applied thought’s intervention,’ still there is also need of the arising of consciousness-originated sound without intimation (cognition) for because of the words, ‘For the intimation (cognition) is not due to intimating speech’ (?), it arises together with sound not intimatable (cognizable) through the ear. That being so, there would have to be a consciousness-born sound-ennead. And that theory is rejected by Saṅghakaras who imagine that it is self-contradictory to say that there is sound not intimatable (cognizable) through the ear. Others, however, do not reject the Great Commentary’s statement and they comment on its intention. How? [They say that] the non-intimation (non-cognition) through the ear of the sound activated due to applied thought’s intervention is stated in the Suttas with this intention, ‘He tells by hearing with the divine ear the subtle sound that is conascent with the intimation, originated by applied thought, and consisting in movement of the tongue and palate, and so on’ (cf. \textbf{\cite{A}I 171}), and that in the Paṭṭhāna (Paṭṭh 1, 7) the state of object condition for ear-consciousness is stated with reference to gross sound” (\textbf{\cite{Vism-mhṭ}460}).} What is born of temperature, consciousness, and nutriment \marginnote{\textcolor{teal}{\footnotesize\{513|455\}}}{}\textcolor{brown}{\textit{[452]}} is called \emph{born of three}. But that is the three beginning with “lightness” only. What is born from the four beginning with kamma is called \emph{born of four}. That is all the rest except “matter as characteristic.”

                    \vismParagraph{XIV.80}{80}{}
                    But “matter as characteristic” is called \emph{not born of anything}. Why? Because there is no arising of arising, and the other two are the mere maturing and breakup of what has arisen. Though in the passage, “The visible-data base, the sound base, the odour base, the flavour base, the tangible-data base, the space element, the water element, lightness of matter, malleability of matter, wieldiness of matter, growth of matter, continuity of matter, and physical food—these states are consciousness-originated” (cf. \textbf{\cite{Dhs}§667}) and so on, a state of birth [that is, growth] being born from somewhere can be understood as allowable since the point of view here is the moment when the conditions that are giving birth to the kinds of materiality are exercising their function.

                    This, firstly, is the section of the detailed explanation dealing with the materiality aggregate.
                \subsection[\vismAlignedParas{§81–124}The Consciousness Aggregate]{The Consciousness Aggregate}

                    \vismParagraph{XIV.81}{81}{}
                    Among the remaining aggregates, however, whatever has the characteristic of being felt\footnote{\vismAssertFootnoteCounter{34}\vismHypertarget{XIV.n34}{}“‘\emph{Has the characteristic of being felt’ }means that it has as its characteristic what is felt, what is experienced as the ‘taste (stimulus)’ of the object. ‘\emph{Characteristic of perceiving’ }means that it has as its characteristic the perceiving of an object classed as blue, etc., and the knowing, the apprehending, of it by arousing the perception of it as blue, yellow, long, short, and so on. Forming \emph{(abhisaṅkharaṇa) }is accumulating, or it is contriving by becoming interested. And it is because volition is basic in both of these ways that the formations aggregate is said thus to have the \emph{characteristic of forming. }For in expounding the formations aggregate in the Suttanta-Bhājaniya of the Vibhaṅga, volition was expounded by the Blessed One thus, ‘Eye-contact-born volition’ (\textbf{\cite{Vibh}8}) and so on. ‘\emph{Has the characteristic of cognizing’ }means that it has as its characteristic that kind of knowing called apprehension of an object in a mode in which the objective field is apprehended differently from the mode of perceiving” (\textbf{\cite{Vism-mhṭ}462}).} should be understood, all taken together, as the feeling aggregate; and whatever has the characteristic of perceiving, all taken together, as the perception aggregate; and whatever has the characteristic of forming, all taken together, as the formations aggregate; and whatever has the characteristic of cognizing, all taken together, as the consciousness aggregate. Herein, since the rest are easy to understand when the consciousness aggregate has been \marginnote{\textcolor{teal}{\footnotesize\{514|456\}}}{}understood, we shall therefore begin with the commentary on the consciousness aggregate.

                    \vismParagraph{XIV.82}{82}{}
                    “Whatever has the characteristic of cognizing should be understood, all taken together, as the consciousness aggregate” was said above. And what has the characteristic of cognizing (\emph{vijānana})? Consciousness (\emph{viññāṇa}); according as it is said, “It cognizes, friend, that is why ‘consciousness’ is said” (\textbf{\cite{M}I 292}). The words \emph{viññāṇa} (consciousness), \emph{citta} (mind, consciousness), and \emph{mano} (mind) are one in meaning.
                    \subsubsection[\vismAlignedParas{§82–110}The 89 Kinds of Consciousness—see Table III]{The 89 Kinds of Consciousness—see Table III}

                        That same [consciousness], though one in its individual essence with the characteristic of cognizing, is threefold according to kind, namely, (I) profitable, (II) unprofitable, and (III) indeterminate.\footnote{\vismAssertFootnoteCounter{35}\vismHypertarget{XIV.n35}{}\emph{Profitable }in the sense of health, blamelessness, and pleasant result (see Vism-mhṭ 463). \emph{Unprofitable }in the opposite sense. \emph{Indeterminate }because not describable as either profitable or unprofitable (see \textbf{\cite{Vism-mhṭ}464}). This is the first of the twenty-two triads in the Abhidhamma Mātikā (\textbf{\cite{Dhs}1}).

                                Pali has five principal words, \emph{nāma, viññāṇa, mano, citta,} and \emph{ceto,} against the normal English \emph{consciousness }and \emph{mind.} While their etymology can be looked up in the dictionary, one or two points need noting here. \emph{Nāma} (rendered by “mentality” when not used to refer to a \emph{name}) is almost confined in the sense considered to the expression \emph{nāma-rūpa} (“mentality-materiality”) as the fourth member of the dependent origination, where it comprises the three mental aggregates of feeling, perception and formations, but not that of consciousness (\emph{viññāṇa}). \emph{Viññāṇa} (rendered by “consciousness”) is, loosely, more or less a synonym for \emph{mano} and \emph{citta}; technically, it is bare cognition considered apart from feeling, perception or formations. \emph{Mano} (rendered by “mind”), when used technically, is confined to the sixth internal base for contact (\hyperlink{XV}{Ch. XV}{}). \emph{Citta }(rendered by “mind” and “consciousness” or “[manner of] consciousness”), when used technically, refers to a momentary type-situation considered as \emph{viññāṇa} in relation to the tone of its concomitant feeling, perception and formations. Possibly, a better rendering would have been “cognizance” throughout. It carries a flavour of its etymological relative, \emph{cetanā} (“volition”). \emph{Ceto} (another etymological relative, rendered by “heart”—i.e. “seat of the emotions,”—“will” or “mind”), when used loosely is very near to \emph{citta;} but technically it is restricted to one or two such expressions as \emph{ceto-vimutti} (“mind-deliverance” or “heart-deliverance”).}
                        \par\noindent[\textsc{\textbf{I. Profitable}}]

                            \vismParagraph{XIV.83}{83}{}
                            I. Herein, the \emph{profitable} is fourfold according to plane, namely, (A) of the sense sphere, (B) of the fine-material sphere, (C) of the immaterial sphere, and (D) supramundane.\footnote{\vismAssertFootnoteCounter{36}\vismHypertarget{XIV.n36}{}“‘Sense sphere’ \emph{(kāmāvacara): }here there are the two kinds of sense desire \emph{(kāma), }sense desire as basis \emph{(vatthu-kāma) }and sense desire as defilement \emph{(kilesakāma). }Of these, sense desire as [objective] basis particularized as the five cords of sense desire \emph{(pañca-kāma-guṇa }= dimensions of sensual desires), is desired \emph{(kāmiyati). }Sense desire as defilement, which is craving, desires \emph{(kāmeti). }The sense sphere \emph{(kāmāvacara) }is where these two operate \emph{(avacaranti) }together. But what is that? It is the elevenfold sense-desire becoming, i.e. hell, asura demons, ghosts, animals, human beings, and six sensual-sphere heavens. So too with the fine-material sphere and the immaterial sphere, taking “fine-material” as craving for the fine-material too, and “immaterial” as craving for the immaterial too. It crosses over \emph{(uttarati) }from the world \emph{(loka), }thus it is supramundane \emph{(lokuttara)” }(\textbf{\cite{Vism-mhṭ}464}).} \marginnote{\textcolor{teal}{\footnotesize\{515|457\}}}{}I. A. Herein, (1)–(8) that of the \emph{sense sphere} is eightfold, being classified according to joy, equanimity, knowledge, and prompting, that is to say: (1) when accompanied-by-joy it is either associated-with-knowledge and unprompted, or (2) prompted; or (3) it is dissociated-from-knowledge and likewise [unprompted, or (4) prompted]; and (5) when accompanied-by-equanimity it is either associated-with-knowledge and prompted, or (6) unprompted; or (7) it is dissociated-from-knowledge \textcolor{brown}{\textit{[453]}} and likewise [unprompted, or (8) prompted].

                            \vismParagraph{XIV.84}{84}{}
                            (1) When a man is happy on encountering an excellent gift to be given, or recipient, etc., or some such cause for joy, and by placing right view foremost that occurs in the way beginning “There is [merit in] giving” (\textbf{\cite{M}I 288}), he unhesitatingly and unurged by others performs such merit as giving, etc., then his consciousness is \emph{accompanied by joy, associated with knowledge, and unprompted}. (2) But when a man is happy and content in the way aforesaid, and, while placing right view foremost, yet he does it hesitantly through lack of free generosity, etc., or urged on by others, then his consciousness is of the same kind as the last but \emph{prompted}; for in this sense “prompting” is a term for a prior effort exerted by himself or others

                            \vismParagraph{XIV.85}{85}{}
                            (3) But when young children have a natural habit due to seeing the behaviour of relatives and are joyful on seeing bhikkhus and at once give them whatever they have in their hands or pay homage, then the third kind of consciousness arises. (4) But when they behave like this on being urged by their relatives, “Give; pay homage,” then the fourth kind of consciousness arises. (5)–(8) But when the consciousnesses are devoid of joy in these four instances through encountering no excellence in the gift to be given, or in the recipient, etc., or through want of any such cause for joy, then the remaining four, which are \emph{accompanied by equanimity}, arise.

                            So sense-sphere profitable [consciousness] should be understood as of eight kinds, being classed according to joy, equanimity, knowledge, and prompting.

                            \vismParagraph{XIV.86}{86}{}
                            I. B. The consciousness of the fine-material sphere is fivefold, being classed according to association with the jhāna factors. That is to say, (9) the first is associated with applied thought, sustained thought, happiness, bliss, and concentration, (10) the second leaves out applied thought from that, (11) the third leaves out sustained thought from that, (12) the fourth makes happiness fade away from that, (13) the fifth is associated with equanimity and concentration, bliss having subsided.

                            \vismParagraph{XIV.87}{87}{}
                            I. C. That of the \emph{immaterial sphere} is fourfold by association with the four immaterial states; for (14) the first is associated with the jhāna of the base consisting of boundless space in the way aforesaid, while (15)–(17) the second, third, and fourth, are [respectively] associated with those of the base consisting of boundless consciousness, and so on.

                            \vismParagraph{XIV.88}{88}{}
                            I. D. The \emph{supramundane} is fourfold (18)–(21) by association with the four paths.

                            \marginnote{\textcolor{teal}{\footnotesize\{516|458\}}}{}So firstly, profitable consciousness itself is of twenty-one kinds. \textcolor{brown}{\textit{[454]}}
                        \par\noindent[\textsc{\textbf{II. Unprofitable}}]

                            \vismParagraph{XIV.89}{89}{}
                            II. The \emph{unprofitable} is one kind according to plane, being only of the sense sphere. It is of three kinds according to root, as (a) rooted in greed, (b) rooted in hate, and (c) rooted in delusion.

                            \vismParagraph{XIV.90}{90}{}
                            II. (a) Herein, (22)–(29) that \emph{rooted in greed} is of eight kinds, being classed according to joy, equanimity, [false] view, and prompting, that is to say: (22) when accompanied by joy it is either associated-with-[false-]view and unprompted, or (23) prompted; or (24) it is dissociated-from-[false-]view and likewise [unprompted or (25) prompted]; and (26) when accompanied-by-equanimity it is either associated-with-[false-]view and unprompted, or (27) prompted; or (28) it is dissociated-from-[false-]view and likewise [unprompted, or (29) prompted].

                            \vismParagraph{XIV.91}{91}{}
                            (22) When a man is happy and content in placing wrong view foremost of the sort beginning “There is no danger in sense desires” (\textbf{\cite{M}I 307}), and either enjoys sense desires with consciousness that in its own individual essence is eager without being urged, or believes auspicious sights, etc., have a [real substantial] core, then the first kind of unprofitable consciousness arises (23); when it is with consciousness that is sluggish and urged on, then it is the second kind (24). But when a man is happy and content only, without placing wrong view foremost, and indulges in sexual intercourse, or covets others’ good fortune, or steals others’ goods, with consciousness that in its own individual essence is eager without being urged, then it is the third kind (25). When it is with consciousness that is sluggish and urged on, then it is the fourth kind (26)–(29). But when the consciousnesses are devoid of joy in these four instances through encountering no excellence in the sense desires, or through want of any such cause for joy, then the remaining four, which are accompanied by equanimity, arise.

                            So that rooted in greed should be understood as of eight kinds, being classed according to joy, equanimity, [false] view and prompting.

                            \vismParagraph{XIV.92}{92}{}
                            II. (b) That \emph{rooted in hate} is of two kinds: (30)–(31) being \emph{accompanied-by-grief }and \emph{associated-with-resentment}, it is either \emph{prompted} or \emph{unprompted}. It should be understood to occur at the times when [consciousness] is either keen [if unprompted] or sluggish [if prompted] in the killing of living things, and so on.

                            \vismParagraph{XIV.93}{93}{}
                            II. (c) That \emph{rooted in delusion} is of two kinds: (32)–(33) being \emph{accompanied-by-equanimity}, it is either associated-with uncertainty or associated-with-agitation. It should be understood to occur at the time of indecision or of distraction.

                            So unprofitable consciousness is of twelve kinds.
                        \par\noindent[\textsc{\textbf{III. Indeterminate}}]

                            \vismParagraph{XIV.94}{94}{}
                            III. The \emph{indeterminate} is of two kinds: (i) resultant and (ii) functional. Herein, III. i. \emph{resultant} is of four kinds according to plane; namely, (A) of the sense sphere, (B) of the fine-material sphere, (C) of the immaterial sphere, and (D) supramundane. Herein, III. i. A. that of the \emph{sense sphere} is of two kinds, namely, (a) profitable result and (b) unprofitable result. And III. i. A. (a) the \emph{profitable resultant }is of two kinds, namely, (1) without root-cause and (2) with root-cause.
                            \par\noindent[\emph{\textbf{III. i. Resultant}}]

                                \vismParagraph{XIV.95}{95}{}
                                III. i. A. (a) i. Herein, that \emph{without root-cause} is that devoid of non-greed, etc., as the cause of result. It is of eight kinds as (34) eye-consciousness (35)–(38), \marginnote{\textcolor{teal}{\footnotesize\{517|459\}}}{}ear-, nose-, tongue-, and body-consciousness (39), mind-element with the function of receiving (40)–(41), the two mind-consciousness-elements with the functions of investigating, and so on. \textcolor{brown}{\textit{[455]}}

                                \vismParagraph{XIV.96}{96}{}
                                Herein, (34) \emph{eye-consciousness} has the characteristic of being supported by the eye and cognizing visible data. Its function is to have only visible data as its object. It is manifested as occupation with visible data. Its proximate cause is the departure of (70) the functional mind-element that has visible data as its object.

                                (35)–(38) \emph{Ear-, nose-, tongue-}, and \emph{body-consciousness} [respectively] have the characteristic of being supported by the ear, etc., and of cognizing sounds, and so on. Their functions are to have only sounds, etc., as their [respective] objects. They are manifested as occupation with [respectively] sounds, and so on. Their proximate cause is the departure of (70) the functional mind-element that has [respectively] sounds, etc., as its object.

                                \vismParagraph{XIV.97}{97}{}
                                (39) [The resultant] \emph{mind-element} has the characteristic of cognizing [respectively] visible data, etc., immediately next to (34)–(38) eye-consciousness, and so on. Its function is to receive visible data, and so on. It is manifested as the state [of receiving] corresponding to that [last-mentioned function].\footnote{\vismAssertFootnoteCounter{37}\vismHypertarget{XIV.n37}{}The meaning of the expression \emph{tathābhāva-paccupaṭṭhāna }appears more clearly where it is used again at §108. In this definition \emph{(sādhana) }the function \emph{(kicca-rasa) }in fact describes the verb action \emph{(kicca) }while the manifestation \emph{(paccupaṭṭhāna) }describes the relevant nounal state \emph{(bhāva). }So \emph{“tathābhāva” }means that what has just been taken as a function (e.g. “receiving”) is to be taken also as a state (“reception”).} Its proximate cause is the departure of eye-consciousness, and so on.

                                (40)–(41) Also the twofold resultant \emph{mind-consciousness-element without root-cause} with the function of investigating, etc., has as its characteristic the cognizing of the six kinds of objects. Its function is that of investigating, and so on. It is manifested as the state [of investigating] corresponding to that [last-mentioned function]. Its proximate cause is the heart-basis.

                                \vismParagraph{XIV.98}{98}{}
                                But it is classed according to its association with joy or with equanimity, and according to its being divisible into that with two positions and that with five positions [in the cognitive series]. For of these, (40) one is associated-with-joy because of its presence when entirely desirable objects occur; and it has two positions [in the cognitive series] because it occurs as investigating at the five doors and as registration at the end of impulsion. (41) The other kind is associated-with-equanimity because of its presence when desirable-neutral objects occur, and it has five positions since it occurs as investigation, registration, rebirth-linking, life-continuum, and death.

                                \vismParagraph{XIV.99}{99}{}
                                And this eightfold resultant consciousness without root-cause is of two kinds as well because of having an invariable object and a variable object. It is of three kinds as classed according to [bodily] pleasure, [mental] joy, and equanimity. For (34)–(38) the five consciousnesses have each an invariable object since they occur respectively only with respect to visible data, and so on. The others (39)–(41) have a variable object. For here (39) the mind-element occurs with respect to the five beginning with visible data, and (40)–(41) the two mind-consciousness-elements \marginnote{\textcolor{teal}{\footnotesize\{518|460\}}}{}occur with respect to [all] six. Here, however, body-consciousness is associated with [bodily] pleasure. The mind-consciousness-element (40) with two positions is associated with [mental] joy; the other (41) is associated with equanimity.

                                So firstly, the profitable resultant without root-cause should be understood as of eight kinds.

                                \vismParagraph{XIV.100}{100}{}
                                III. i. A. (a) 2. But that with root-cause is (42)–(49) that associated with non-greed, etc., as the cause of the result. It is of eight kinds because it is classed according to joy, etc., like the profitable of the sense sphere (1)–(8). But it does not occur with respect to the six objects\footnote{\vismAssertFootnoteCounter{38}\vismHypertarget{XIV.n38}{}“To the six kinds of objects all classed as limited, etc., past, etc., internal, etc” (\textbf{\cite{Vism-mhṭ}474}).} through giving, etc., as the profitable does; for it occurs only with respect to the six objects that are included among limited states,\footnote{\vismAssertFootnoteCounter{39}\vismHypertarget{XIV.n39}{}Registration consciousness does not, it is stated, occur with an object of exalted consciousness—see \textbf{\cite{Vibh-a}154}.} as rebirth-linking, life-continuum, death, and registration. But the prompted and unprompted states should be understood here as due to the source it has come from, and so on.\footnote{\vismAssertFootnoteCounter{40}\vismHypertarget{XIV.n40}{}“‘\emph{The source it has come from, and so on’ }means the source it has come from and its condition. Here, in the opinion of certain teachers the result of the unprompted profitable is unprompted and the result of the prompted is prompted, like the movement of the face’s reflection in a looking-glass when the face moves; thus it is \emph{due to the source it has come from. }But in the opinion of other teachers the unprompted arises due to powerful kamma as condition and the prompted does so due to weak kamma; thus it is \emph{due to its condition” }(\textbf{\cite{Vism-mhṭ}474}).}\textcolor{brown}{\textit{[456]}} And while there is no difference in the associated states, the resultant should be understood as passive like the reflection of a face in a looking-glass while the profitable is active like the face.

                                \vismParagraph{XIV.101}{101}{}
                                III. i. A. (b) Unprofitable resultant, though, is without root-cause only. It is of seven kinds as (50) eye-consciousness, (51)–(54) ear-, nose-, tongue-, and body-consciousness, (55) mind-element with the function of receiving, and (56) mind-consciousness-element with the function of investigating, etc., and having five positions. It should be understood as to characteristic, etc., in the same way as the profitable resultant without root-cause (34)–(41).

                                \vismParagraph{XIV.102}{102}{}
                                Profitable resultant, though, has desirable or desirable-neutral objects only, while these have undesirable or undesirable-neutral objects only. The former are of three kinds, being classed according to equanimity, bodily pleasure, and mental joy, while these are of two kinds, being classed according to bodily pain and equanimity. For here it is only body-consciousness that is accompanied by bodily pain; the rest are accompanied by equanimity. And the equanimity in these is inferior, and not very sharp as the pain is; while in the former it is superior, and not very sharp as the pleasure is.

                                So with these seven kinds of unprofitable resultant and the previous sixteen kinds of profitable resultant, sense-sphere resultant consciousness is of twenty-three kinds.

                                \vismParagraph{XIV.103}{103}{}
                                \marginnote{\textcolor{teal}{\footnotesize\{519|461\}}}{}III. i. B. That of the \emph{fine-material sphere}, however, is of five kinds (57)–(61) like the profitable (9)–(13). But the profitable occurs in a cognitive series with the impulsions as an attainment [of jhāna], while this occurs in an existence [in the fine-material sphere] as rebirth-linking, life-continuum, and death.

                                \vismParagraph{XIV.104}{104}{}
                                III. i. C. And as that of the fine-material sphere [was like the profitable of that sphere] so that of the \emph{immaterial sphere} (62)–(65) is of four kinds like the profitable too (14)–(17). And its occurrence is classed in the same way as that of the fine-material sphere.

                                \vismParagraph{XIV.105}{105}{}
                                III. i. D. The \emph{supramundane resultant} is of four kinds (66)–(69) because it is [respectively] the fruitions of the consciousnesses associated with the four paths (18)–(21). It occurs in two ways, that is to say, as [fruition in] the cognitive series of the path and as fruition attainment (see \hyperlink{XXII}{Ch. XXII}{}).

                                So resultant consciousness in all the four planes is of thirty-six kinds.
                            \par\noindent[\emph{\textbf{III. ii. Functional}}]

                                \vismParagraph{XIV.106}{106}{}
                                III. ii. The functional, however, is of three kinds according to plane: (A) of the sense sphere, (B) of the fine-material sphere, (C) of the immaterial sphere. Herein, III. ii. A., that of the sense sphere, is of two kinds, namely, (1) without root-cause, and (2) with root-cause.

                                III. ii. A. 1. Herein, that without root-cause is that devoid of non-greed, etc., as the cause of result. That is of two kinds, being classed as (70) mind-element, and (71)–(72) mind-consciousness-element.

                                \vismParagraph{XIV.107}{107}{}
                                Herein, (70) the \emph{mind-element} has the characteristics of being the forerunner of eye-consciousness, etc., and of cognizing visible data and so on. Its function is to advert. It is manifested as confrontation of visible data, and so on. Its proximate cause is the interruption of [the continued occurrence of consciousness as] life-continuum. It is associated with equanimity only.

                                \vismParagraph{XIV.108}{108}{}
                                But the mind-consciousness-element is of two kinds, namely, shared by all and not shared by all. \textcolor{brown}{\textit{[457]}} Herein, (71) that shared by all is the functional [mind-consciousness-element] accompanied by equanimity without root-cause. It has the characteristic of cognizing the six kinds of objects. Its function is to determine at the five doors and to advert at the mind door. It is manifested as the states [of determining and adverting] corresponding to those [last-mentioned two functions]. Its proximate cause is the departure either of the resultant mind-consciousness-element without root-cause (40)–(41) [in the first case], or of one among the kinds of life-continuum [in the second]. (72) That \emph{not shared by all }is the functional [mind-consciousness-element] accompanied by joy without root-cause. It has the characteristic of cognizing the six kinds of objects. Its function is to cause smiling\footnote{\vismAssertFootnoteCounter{41}\vismHypertarget{XIV.n41}{}“With respect to such unsublime objects as the forms of skeletons or ghosts” (Vism-mhṭ 476). See e.g. \textbf{\cite{Vin}III 104}.} in Arahants about things that are not sublime. It is manifested as the state corresponding to that [last-mentioned function]. Its proximate cause is always the heart-basis.

                                So the sense-sphere functional without root-cause is of three kinds.

                                \vismParagraph{XIV.109}{109}{}
                                III. ii. A. 2. That, however, \emph{with root cause} is of eight kinds (73)–(80), like the profitable (1)–(8), being classed according to joy and so on. While the profitable \marginnote{\textcolor{teal}{\footnotesize\{520|462\}}}{}arises in trainers and ordinary men only, this arises in Arahants only. This is the difference here.

                                So firstly, that of the sense sphere is of eleven kinds.

                                III. ii. B., III. ii. C. That, however, of the \emph{fine-material sphere }(81)–(85), and that of the \emph{immaterial sphere} (86)–(89) are [respectively] of five kinds and of four kinds like the profitable. But they should be understood to differ from the profitable in that they arise only in Arahants.

                                So functional consciousness in the three planes is of twenty kinds in all.

                                \vismParagraph{XIV.110}{110}{}
                                So the 21 kinds of profitable, the 12 kinds of unprofitable, the 36 kinds of resultant, and the 20 kinds of functional, amount in all to 89 kinds of consciousness. And these occur in the fourteen modes of (a) rebirth-linking, (b) life-continuum, (c) adverting, (d) seeing, (e) hearing, (f) smelling, (g) tasting, (h) touching, (i) receiving, (j) investigating, (k) determining, (l) impulsion, (m) registration, and (n) death.
                    \subsubsection[\vismAlignedParas{§111–124}The 14 Modes of Occurrence of Consciousness]{The 14 Modes of Occurrence of Consciousness}

                        \vismParagraph{XIV.111}{111}{}
                        How so? (a) When, through the influence of the eight kinds of sense-sphere profitable [consciousness] (1)–(8), beings come to be reborn among deities and human beings, then the eight kinds of sense-sphere resultant with root-cause (42)–(49) occur, and also the resultant mind-consciousness-element without root-cause associated with equanimity (41), which is the weak profitable result with two root-causes in those who are entering upon the state of eunuchs, etc., among human beings—thus nine kinds of resultant consciousness in all occur as rebirth-linking; and they do so making their object whichever among the kamma, sign of kamma, or sign of destiny has appeared at the time of dying (see also \hyperlink{XVII.120}{XVII.120}{}).\footnote{\vismAssertFootnoteCounter{42}\vismHypertarget{XIV.n42}{}See also \textbf{\cite{M-a}IV 124f.} “Here \emph{‘kamma’ }is stored-up profitable kamma of the sense sphere that has got an opportunity to ripen; hence he said ‘\emph{that has appeared.’ ‘Sign of kamma’ }is the gift to be given that was a condition for the volition at the moment of accumulating the kamma. \emph{‘Sign of destiny’ }is the visible-data base located in the destiny in which he is about to be reborn” (\textbf{\cite{Vism-mhṭ}477}). See XVII. 136ff.}

                        \vismParagraph{XIV.112}{112}{}
                        When, through the influence of the profitable of the fine-material sphere (9)–(13) and the immaterial sphere (14)–(17), beings are reborn [respectively] in the fine-material and immaterial kinds of becoming, then the nine kinds of fine-material (57)–(61) and immaterial (62)–(65) resultant occur as \emph{rebirth-linking}; and they do so making their object only the sign of kamma that has appeared at the time of dying.\footnote{\vismAssertFootnoteCounter{43}\vismHypertarget{XIV.n43}{}“‘\emph{The sign of kamma” }here is only the kamma’s own object consisting of an earth kasiṇa, etc” (\textbf{\cite{Vism-mhṭ}478}).}

                        \vismParagraph{XIV.113}{113}{}
                        When, through the influence of the unprofitable (22)–(33), they are reborn in a state of loss, then the one kind of unprofitable resultant mind-consciousness-element without root-cause (56) occurs as rebirth-linking; and it does so making its object whichever among the kamma, sign of kamma, and sign of destiny has appeared at the time of dying. \textcolor{brown}{\textit{[458]}}

                        \marginnote{\textcolor{teal}{\footnotesize\{521|463\}}}{}This firstly is how the occurrence of nineteen kinds of resultant consciousness should be understood as rebirth-linking.

                        \vismParagraph{XIV.114}{114}{}
                        (b) When the rebirth-linking consciousness has ceased, then, following on whatever kind of rebirth-linking it may be, the same kinds, being the result of that same kamma whatever it may be, occur as life-continuum consciousness with that same object; and again those same kinds.\footnote{\vismAssertFootnoteCounter{44}\vismHypertarget{XIV.n44}{}“‘\emph{With that same object’: }if kamma is the life-continuum’s object, then it is that kamma; if the sign of the kamma, or the sign of the destiny, then it is one of those” (\textbf{\cite{Vism-mhṭ}478}).} And as long as there is no other kind of arising of consciousness to interrupt the continuity, they also go on occurring endlessly in periods of dreamless sleep, etc., like the current of a river.\footnote{\vismAssertFootnoteCounter{45}\vismHypertarget{XIV.n45}{}“‘\emph{Occurring endlessly’: }this is, in fact, thus called \emph{‘bhavaṅga’ }(life-continuum, lit. ‘limb’ (or ‘practice’—see II. 11) of becoming) because of its occurring as the state of an \emph{aṅga }(‘limb’ or ‘practice’) of the rebirth-process becoming \emph{(uppatti-bhava)” }(\textbf{\cite{Vism-mhṭ}478}).

                                For the commentarial description of dream consciousness and kamma effected during dreams, see Vibh-a (commentary to Ñāṇa-Vibhaṅga, Ekaka) and A-a, (commentary to AN 5:196) which largely but not entirely overlap. Vism-mhṭ says here: “The seeing of dreams is done with consciousness consisting only of the functional” (\textbf{\cite{Vism-mhṭ}478}).}

                        This is how the occurrence of those same [nineteen kinds of] consciousness should be understood as life-continuum.

                        \vismParagraph{XIV.115}{115}{}
                        (c) With the life-continuum continuity occurring thus, when living beings’ faculties have become capable of apprehending an object, then, when a visible datum has come into the eye’s focus, there is impinging upon the eye-sensitivity due to the visible datum. Thereupon, owing to the impact’s influence, there comes to be a disturbance in [the continuity of] the life-continuum.\footnote{\vismAssertFootnoteCounter{46}\vismHypertarget{XIV.n46}{}“‘\emph{A disturbance in the life-continuum’ }is a wavering of the life-continuum consciousness; the meaning is that there is the arrival at a state that is a reason for dissimilarity in its occurrence twice in that way. For it is called disturbance \emph{(calana) }because it is like a disturbance (movement) since there seems to be a cause for an occasion \emph{(avatthā) }in the mind’s continuity different from the previous occasion. Granted, firstly, that there is impact on the sensitivity owing to confrontation with an object, since the necessity for that is established by the existence of the objective field and the possessor of the objective field, but how does there come to be disturbance (movement) of the life-continuum that has a different support? Because it is connected with it. And here the example is this: when grains of sugar are put on the surface of a drum and one of the grains of sugar is tapped, a fly sitting on another grain of sugar moves” (\textbf{\cite{Vism-mhṭ}478}).} Then, when the life-continuum has ceased, the functional mind-element (70) arises making that same visible datum its object, as it were, cutting off the life-continuum and accomplishing the function of \emph{adverting}. So too in the case of the ear door and so on.

                        \vismParagraph{XIV.116}{116}{}
                        When an object of anyone of the six kinds has come into focus in the mind door, then next to the disturbance of the life-continuum the functional mind-consciousness-element without root-cause (71) arises accompanied by \marginnote{\textcolor{teal}{\footnotesize\{522|464\}}}{}equanimity, as it were, cutting off the life-continuum and accomplishing the function of adverting.

                        This is how the occurrence of two kinds of functional consciousness should be understood as \emph{adverting}.

                        \vismParagraph{XIV.117}{117}{}
                         (d)–(h) Next to adverting,\footnote{\vismAssertFootnoteCounter{47}\vismHypertarget{XIV.n47}{}“‘\emph{Next to adverting’ }means next to five-door adverting. For those who do not admit the cognitive series beginning with receiving, just as they do not admit the heart basis, the Pali has been handed down in various places in the way beginning, ‘For the eye-consciousness element as receiving \emph{(sampaṭicchanāya cakkhuviññāṇadhātuyā)’ }(see \hyperlink{IV.n13}{Ch. IV, n. 13}{}); for the Pali cannot be contradicted” (\textbf{\cite{Vism-mhṭ}479}). The quotation as it stands is not traced to the Piṭakas.}taking the eye door first, \emph{eye-consciousness} (d) arises accomplishing the function of \emph{seeing} in the eye door and having the eye-sensitivity as its physical basis. And [likewise] (e) \emph{ear-}, (f) \emph{nose-}, (g) \emph{tongue-}, and (h) \emph{body-consciousness} arise, accomplishing respectively the functions of \emph{hearing}, etc., in the ear door and so on.

                        These comprise the profitable resultant [consciousnesses] (34)–(38) with respect to desirable and desirable-neutral objective fields, and the unprofitable resultant (50)–(54) with respect to undesirable and undesirable-neutral objective fields.

                        This is how the occurrence of ten kinds of resultant consciousness should be understood as seeing, hearing, smelling, tasting, and touching.

                        \vismParagraph{XIV.118}{118}{}
                        (i) Because of the words, “Eye-consciousness having arisen and ceased, next to that there arises consciousness, mind, mentation … which is appropriate mind-element” (\textbf{\cite{Vibh}88}), etc., next to eye-consciousness, etc., and \emph{receiving} the same objective fields as they [deal with], mind-element arises as (39) profitable resultant next to profitable resultant [eye-consciousness, etc.,] and as (55) unprofitable resultant next to \textcolor{brown}{\textit{[459]}} unprofitable resultant [eye-consciousness, and so on].

                        This is how the occurrence of two kinds of resultant consciousness should be understood as receiving.

                        \vismParagraph{XIV.119}{119}{}
                        (j) Because of the words, “Mind-element having arisen and ceased, also, next to that there arises consciousness, mind, mentation … which is appropriate mind-element” (\textbf{\cite{Vibh}89}),\footnote{\vismAssertFootnoteCounter{48}\vismHypertarget{XIV.n48}{}See \hyperlink{IV.n13}{Ch. IV, note 13}{}.} then resultant mind-consciousness-element without root-cause arises \emph{investigating} the same objective field as that received by the mind-element. When next to (55) unprofitable-resultant mind-element it is (56) unprofitable-resultant, and when next to (39) profitable-resultant [mind-element] it is either (40) accompanied by joy in the case of a desirable object, or (41) accompanied by equanimity in the case of a desirable-neutral object.

                        This is how the occurrence of three kinds of resultant consciousness should be understood as investigating.

                        \vismParagraph{XIV.120}{120}{}
                        (k) Next to investigation, (71) functional mind-consciousness-element without root-cause arises accompanied by equanimity determining that same objective field.

                        \marginnote{\textcolor{teal}{\footnotesize\{523|465\}}}{}This is how the occurrence of one kind of resultant consciousness should be

                        \vismParagraph{XIV.121}{121}{}
                        understood as determining. (l) Next to determining, if the visible datum, etc., as object is vivid,\footnote{\vismAssertFootnoteCounter{49}\vismHypertarget{XIV.n49}{}“‘\emph{If … vivid (lit. large)’: }this is said because it is the occurrence of consciousness at the end of the impulsions that is being discussed. For an object is here intended as ‘vivid’ when its life is fourteen conscious moments; and that should be understood as coming into focus when it has arisen and is two or three moments past” (\textbf{\cite{Vism-mhṭ}479}).}then six or seven \emph{impulsions} impel with respect to the objective fields as determined. These are one among (1)–(8) the eight kinds of sense-sphere profitable, or (22)–(33) the twelve kinds of unprofitable, or (72)–(80) the nine remaining sense-sphere functional. This, firstly, is the way in the case of the five doors.

                        But in the case of the mind door those same [impulsions arise] next to (71) mind-door adverting.

                        Beyond [the stage of] change-of-lineage\footnote{\vismAssertFootnoteCounter{50}\vismHypertarget{XIV.n50}{}“This includes also the preliminary-work and the cleansing (see \hyperlink{XXII}{Ch. XXII}{}, note 7), not change-of-lineage only” (\textbf{\cite{Vism-mhṭ}479}). See also \hyperlink{IV.74}{IV.74}{} and XXI. 129.} any [of the following 26 kinds of impulsion] that obtains a condition\footnote{\vismAssertFootnoteCounter{51}\vismHypertarget{XIV.n51}{}“‘\emph{That obtains a condition’: }any impulsion that has obtained a condition for arising next to change-of-lineage, as that of the fine-material sphere, and so on” (\textbf{\cite{Vism-mhṭ}479}).} impels; that is, any kind among (9)–(13) the five profitable, and (81)–(85) the five functional, of the fine-material sphere, and (14)–(17) the four profitable, and (86)–(89) the four functional of the immaterial sphere, and also (18)–(21) the four path consciousnesses and (66)–(69) four fruition consciousnesses of the supramundane.

                        This is how the occurrence of fifty-five kinds of profitable, unprofitable,

                        \vismParagraph{XIV.122}{122}{}
                        functional, and resultant consciousness should be understood as impulsion. (m) At the end of the impulsions, if the object is a very vivid one\footnote{\vismAssertFootnoteCounter{52}\vismHypertarget{XIV.n52}{}“‘\emph{A very vivid one’ }is one with a life of sixteen conscious moments. For registration consciousness arises with respect to that, not with respect to any other. \emph{‘Clear’ }means very evident, and that is only in the sense sphere; for registration arises with respect to that” (\textbf{\cite{Vism-mhṭ}479}).} in the five doors, or is clear in the mind door, then in sense-sphere beings at the end of sense-sphere impulsions resultant consciousness occurs through any condition it may have obtained such as previous kamma, impulsion consciousness, etc., with desirable, etc., object.\footnote{\vismAssertFootnoteCounter{53}\vismHypertarget{XIV.n53}{}“‘\emph{Previous kamma’: }this is said in order to show the differences in kinds of registration; for kamma that generates rebirth-linking is not the only kind to generate registration; other kinds of kamma do so too. But the latter generates registration unlike that generatable by the kamma that generates rebirth-linking. \emph{‘Impulsion consciousness’: }this is said in order to show what defines the registration; for it is said, ‘Registration is definable by impulsion’ (?). The word ‘etc.’ includes rebirth-linking, however; for that is not a condition for registration that is more outstanding than itself. \emph{‘Any condition’: }any condition from among the desirable objects, etc., that has combined \emph{(samaveta) }to produce the arising of registration” (\textbf{\cite{Vism-mhṭ}479}).} [It occurs thus] as one among the eight sense-sphere resultant kinds with root cause (42)–(49) or the three resultant mind-consciousness elements without root-cause (40), (41), (56), and it [does so] twice \marginnote{\textcolor{teal}{\footnotesize\{524|466\}}}{}or \textcolor{brown}{\textit{[460]}} once, following after the impulsions that have impelled, and with respect to an object other than the life-continuum’s object, like some of the water that follows a little after a boat going upstream. Though ready to occur with the life-continuum’s object after the impulsions have ended, it nevertheless occurs making the impulsions’ object its object. Because of that it is called \emph{registration }(\emph{tadārammaṇa—}lit. “having-that-as-its-object”).

                        This is how the occurrence of eleven kinds of resultant consciousness should be understood as registration.

                        \vismParagraph{XIV.123}{123}{}
                        (n) At the end of registration the life-continuum resumes its occurrence. When the [resumed occurrence of the] life-continuum is again interrupted, adverting, etc., occur again, and when the conditions obtain, the conscious continuity repeats its occurrence as adverting, and next to adverting seeing, etc., according to the law of consciousness, again and again, until the life-continuum of one becoming is exhausted. For the last life-continuum consciousness of all in one becoming is called death (\emph{cuti}) because of falling (\emph{cavanatta}) from that [becoming]. So that is of nineteen kinds too [like rebirth-linking and life-continuum].

                        This is how the occurrence of nineteen kinds of resultant consciousness should be understood as death.

                        \vismParagraph{XIV.124}{124}{}
                        And after death there is rebirth-linking again; and after rebirth-linking, life-continuum. Thus the conscious continuity of beings who hasten through the kinds of becoming, destiny, station [of consciousness], and abode [of beings] occurs without break. But when a man attains Arahantship here, it ceases with the cessation of his death consciousness.

                        This is the section of the detailed explanation dealing with the consciousness aggregate.
                \subsection[\vismAlignedParas{§125–128}The Feeling Aggregate]{The Feeling Aggregate}

                    \vismParagraph{XIV.125}{125}{}
                    Now, it was said above, “Whatever has the characteristic of being felt should be understood, all taken together, as the feeling aggregate” (\hyperlink{XIV.81}{§81}{}). And here too, what is said to have the characteristic of being felt is feeling itself, according as it is said, “It is felt, friend, that is why it is called feeling” (\textbf{\cite{M}I 293}).

                    \vismParagraph{XIV.126}{126}{}
                    But though it is singlefold according to its individual essence as the characteristic of being felt, it is nevertheless threefold as to kind, that is to say, profitable, unprofitable, and indeterminate. Herein, it should be understood that when associated with the profitable consciousness described in the way beginning “(1)–(8) That of the sense sphere is eight-fold, being classified according to joy, equanimity, knowledge, and prompting” (\hyperlink{XIV.83}{§83}{}), it is profitable;\footnote{\vismAssertFootnoteCounter{54}\vismHypertarget{XIV.n54}{}“This should be regarded as a secondary characteristic \emph{(upalakkhaṇa) }of profitable feeling, that is to say, the fact that whatever profitable feeling there is, is all associated with profitable consciousness. That, however, is not for the purpose of establishing its profitableness. For the profitableness of profitable feeling is not due to its association with profitable consciousness, but rather to wise attention and so on. That is why he said \emph{‘as to kind.’ }So too in the case of the unprofitable and so on” (\textbf{\cite{Vism-mhṭ}481}).} \marginnote{\textcolor{teal}{\footnotesize\{525|467\}}}{}that associated with unprofitable consciousness is unprofitable; that associated with indeterminate consciousness is \emph{indeterminate}. \textcolor{brown}{\textit{[461]}}

                    \vismParagraph{XIV.127}{127}{}
                    It is fivefold according to the analysis of its individual essence into [bodily] pleasure, [bodily] pain, [mental] joy, [mental] grief, and equanimity.

                    Herein, \emph{pleasure} is associated with profitable resultant body-consciousness (38) and pain with unprofitable resultant body-consciousness (54). \emph{Joy} is associated with 62 kinds of consciousness, namely, as to sense sphere, with 4 kinds of profitable (1)–(4), with 4 resultant with root-cause (42)–(45), with 1 resultant without root-cause (40), with 4 functional with root-cause (73)–(76), with 1 functional without root-cause (72), and with 4 unprofitable (22)–(25); and as to the fine-material-sphere, with 4 kinds of profitable (9)–(12), 4 resultant (57)–(60), and 4 functional (81)–(84), leaving out that of the fifth jhāna in each case; but there is no supramundane without jhāna and consequently the [eight] kinds of supramundane (18)–(21) and (66)–(69) multiplied by the five jhāna make forty; but leaving out the eight associated with the fifth jhāna, it is associated with the remaining 32 kinds of profitable resultant. Grief is associated with two kinds of unprofitable (30)–(31). \emph{Equanimity} is associated with the remaining fifty-five kinds of consciousness.

                    \vismParagraph{XIV.128}{128}{}
                    Herein, \emph{pleasure} has the characteristic of experiencing a desirable tangible datum. Its function is to intensify associated states. It is manifested as bodily enjoyment. Its proximate cause is the body faculty.

                    \emph{Pain} has the characteristic of experiencing an undesirable tangible datum. Its function is to wither associated states. It is manifested as bodily affliction. Its proximate cause is the body faculty.

                    \emph{Joy} has the characteristic of experiencing a desirable object. Its function is to exploit\footnote{\vismAssertFootnoteCounter{55}\vismHypertarget{XIV.n55}{}\emph{Sambhoga—“exploiting”: }not in this sense in PED (see also \hyperlink{XVII.51}{XVII.51}{}).} in one way or another the desirable aspect. It is manifested as mental enjoyment. Its proximate cause is tranquillity.

                    \emph{Grief} has the characteristic of experiencing an undesirable object. Its function is to exploit in one way or another the undesirable aspect. It is manifested as mental affliction. Its proximate cause is invariably the heart-basis.

                    \emph{Equanimity} has the characteristic of being felt as neutral. Its function is not to intensify or wither associated states much. It is manifested as peacefulness. Its proximate cause is consciousness without happiness.\footnote{\vismAssertFootnoteCounter{56}\vismHypertarget{XIV.n56}{}“Pleasure and pain respectively gratify and afflict by acting in one way on the body and in another way on the mind, but not so equanimity, which is why the latter is described as of one class.

                            “Just as, when a man places a piece of cotton wool on an anvil and strikes it with an iron hammer, and his hammer goes right through the cotton and hits the anvil, the violence of the blow is great, so too because the violence of the impact’s blow is great, body-consciousness is accompanied by pleasure when the object is a desirable or desirable-neutral one, and by pain when the object is an undesirable or undesirable-neutral one. [It is the impact of primary matter (tangible object) on the primaries of the body.]

                            “Herein, though profitable-resultant and unprofitable-resultant consciousness discriminated according to the desirable and undesirable might logically be associated with pleasure and pain, nevertheless the eight kinds of consciousness that have the eye, etc., as their support ((34)–(37) and (50)–(53)) are invariably associated only with equanimity, because of the gentleness of the impact’s blow in the case of two instances of derived matter, like that of two pieces of cotton wool” (\textbf{\cite{Vism-mhṭ}482}). For a simile see \textbf{\cite{Dhs-a}263}.}

                    \marginnote{\textcolor{teal}{\footnotesize\{526|468\}}}{}This is the section of the detailed explanation dealing with the feeling aggregate.
                \subsection[\vismAlignedParas{§129–130}The Perception Aggregate]{The Perception Aggregate}

                    \vismParagraph{XIV.129}{129}{}
                    Now, it was said above, “Whatever has the characteristic of perceiving should be understood, all taken together, as the perception aggregate” (\hyperlink{XIV.81}{§81}{}). And here too, what is said to have the characteristic of perceiving is perception itself, according as it is said, “It perceives, friend, that is why it is called perception” (\textbf{\cite{M}I 293}).

                    But though it is singlefold according to its individual essence as the characteristic of perceiving, it is nevertheless threefold as to kind, that is to say, profitable, unprofitable, and indeterminate. Herein, \textcolor{brown}{\textit{[462]}} that associated with profitable consciousness is \emph{profitable}, that associated with unprofitable consciousness is \emph{unprofitable}, that associated with indeterminate consciousness is indeterminate. Since there is no consciousness dissociated from perception, perception therefore has the same number of divisions as consciousness [that is to say, eighty-nine].

                    \vismParagraph{XIV.130}{130}{}
                    But though classed in the same way as consciousness, nevertheless, as to characteristic, etc., it all has just the characteristic of perceiving. Its function is to make a sign as a condition for perceiving again that “this is the same,” as carpenters, etc., do in the case of timber, and so on. It is manifested as the action of interpreting by means of the sign as apprehended, like the blind who “see” an elephant (\textbf{\cite{Ud}68–69}). Its proximate cause is an objective field in whatever way that appears, like the perception that arises in fawns that see scarecrows as men.

                    This is the section of the detailed explanation dealing with the perception aggregate.
                \subsection[\vismAlignedParas{§131–184}The Formations Aggregate—see Tables II \& IV]{The Formations Aggregate—see Tables II \& IV}

                    \vismParagraph{XIV.131}{131}{}
                    Now, it was said above, “Whatever has the characteristic of forming should be understood, all taken together, as the formations aggregate” (\hyperlink{XIV.81}{§81}{}). And here too, what is said to have the characteristic of forming is that which has the characteristic of agglomerating.\footnote{\vismAssertFootnoteCounter{57}\vismHypertarget{XIV.n57}{}“‘\emph{The characteristic of agglomerating’ }means the characteristic of adding together \emph{(sampiṇḍana); }then they are said to have the function of accumulating, for the dhammas in the formations aggregate are so described because volition is their basis” (Vism-mhṭ 484).} What is that? It is formations themselves, according as it is said, “They form the formed, bhikkhus, that is why they are called formations” (\textbf{\cite{S}III 87}).

                    \vismParagraph{XIV.132}{132}{}
                    They have the characteristic of forming. Their function is to accumulate. They are manifested as intervening.\footnote{\vismAssertFootnoteCounter{58}\vismHypertarget{XIV.n58}{}\emph{Vipphāra—}“intervening” here is explained by \textbf{\cite{Vism-mhṭ}(p. 484)} as \emph{vyāpāra }(interest or work); not in this sense in PED. See \hyperlink{VI.n6}{Ch. VI, note 6}{}.} Their proximate cause is the remaining \marginnote{\textcolor{teal}{\footnotesize\{527|469\}}}{}three [immaterial] aggregates. So according to characteristic, etc., they are singlefold. And according to kind they are threefold, namely, (I) profitable, (II) unprofitable, and (III) indeterminate. As regards these, when associated with profitable consciousness they are profitable, when associated with unprofitable consciousness they are unprofitable, when associated with indeterminate consciousness they are indeterminate.
                    \subsubsection[\vismAlignedParas{§133–184}According to Association with Consciousness]{According to Association with Consciousness}

                        \vismParagraph{XIV.133}{133}{}
                        I. (1) Herein, firstly, those associated with the first sense-sphere profitable consciousness (1) amount to thirty-six, that is to say, the constant ones, which are the twenty-seven given in the texts as such, and the four “or-whatever-states,”\footnote{\vismAssertFootnoteCounter{59}\vismHypertarget{XIV.n59}{}\emph{Yevāpanaka (ye-vā-pana-ka) }is commentarial shorthand derived from the Dhammasaṇgaṇī phrase \emph{“ye-vā-pana tasmiṃ samaye aññe pi atthi paṭiccasamuppannā arūpino dhammā}”—“Or whatever other immaterial conditionally-arisen states (phenomena) there are too on that occasion” (Dhs 1). Cf. also \textbf{\cite{M}I 85}.} and also the five inconstant ones (cf. \textbf{\cite{Dhs}§1}).

                        Herein, the twenty-seven given as such are these:

                        
                            \begin{enumerate}[(i),nosep]
                                \item contact,
                                \item volition,
                                \item applied thought,
                                \item \textcolor{brown}{\textit{[463]}}sustained thought,
                                \item happiness (interest),
                                \item energy,
                                \item life,
                                \item concentration,
                                \item faith,
                                \item mindfulness,
                                \item conscience,
                                \item shame,
                                \item non-greed,
                                \item non-hate,
                                \item non-delusion,
                                \item tranquillity of the [mental] body,
                                \item tranquillity of consciousness,
                                \item lightness of the [mental] body,
                                \item lightness of consciousness,
                                \item malleability of the [mental] body,
                                \item malleability of consciousness,
                                \item wieldiness of the [mental] body,
                                \item wieldiness of consciousness,
                                \item proficiency of the [mental] body,
                                \item proficiency of consciousness,
                                \item rectitude of the [mental] body,
                                \item \marginnote{\textcolor{teal}{\footnotesize\{528|470\}}}{}rectitude of consciousness.
                            \end{enumerate}

                        The four ‘or-whatever-states’ are these:

                        
                            \begin{enumerate}[(i),nosep,start=28]
                                \item zeal (desire),
                                \item resolution,
                                \item attention (bringing to mind),
                                \item specific neutrality.
                            \end{enumerate}

                        And the five inconstant are these:

                        
                            \begin{enumerate}[(i),nosep,start=32]
                                \item compassion,
                                \item gladness,
                                \item abstinence from bodily misconduct,
                                \item abstinence from verbal misconduct,
                                \item abstinence from wrong livelihood.
                            \end{enumerate}

                        These last arise sometimes [but not always], and when they arise they do not do so together.

                        \vismParagraph{XIV.134}{134}{}
                        Herein, (i) it touches (\emph{phusati}), thus it is contact (\emph{phassa}). This has the characteristic of touching. Its function is the act of impingement. It is manifested as concurrence. Its proximate cause is an objective field that has come into focus.

                        [As to its characteristic], although this is an immaterial state, it occurs with respect to an object as the act of touching too.\footnote{\vismAssertFootnoteCounter{60}\vismHypertarget{XIV.n60}{}“‘\emph{As the act of touching too’: }by this he shows that this is its individual essence even though it is immaterial. And the characteristic of touching is obvious in its occurrence in such instances as, say, the watering of the mouth in one who sees another tasting vinegar or a ripe mango, the bodily shuddering in a sympathetic person who sees another being hurt, the trembling of the knees in a timid man standing on the ground when he sees a man precariously balanced on a high tree branch, the loss of power of the legs in one who sees something terrifying such as a \emph{pisāca }(goblin)” (\textbf{\cite{Vism-mhṭ}484–485}).} And [as to its function], although it is not adherent on anyone side\footnote{\vismAssertFootnoteCounter{61}\vismHypertarget{XIV.n61}{}For “non-adherent” see §46. “‘\emph{On any one side’ }means on any one side of itself, like a pair of planks and so on. \emph{‘Non-adherent’ }means not sticking \emph{(asaṃsilissamāna). }It is only the impact without adherence that contact shares with visible data and sound, not the objective field. Just as, though eye and ear are non-adherent respectively to visible data and sounds still they have the word ‘touched’ used of them, so too it can be said of contact’s touching and impinging on the object. Contact’s impinging is the actual concurrence (meeting) of consciousness and object” (\textbf{\cite{Vism-mhṭ}485}).} as eye-cum-visible-object and ear-cum-sound are, yet it is what makes consciousness and the object impinge. It is said to be manifested as concurrence because it has been described as its own action, namely, the concurrence of the three [(cf. \textbf{\cite{M}I 111}), that is, eye, visible object, and eye-consciousness]. And it is said to have as its proximate cause an objective field that has come into focus because it arises automatically through the appropriate [conscious] reaction and with a faculty when the objective field is presented. But it should be regarded as like a hideless cow (\textbf{\cite{S}II 99}) because it is the habitat\footnote{\vismAssertFootnoteCounter{62}\vismHypertarget{XIV.n62}{}\emph{Adhiṭṭhāna—}“habitat” (or site or location or foundation): this meaning not given in PED.} of feeling.

                        \vismParagraph{XIV.135}{135}{}
                        (ii) It wills (\emph{cetayati}), thus it is \emph{volition} (\emph{cetanā}); it collects, is the meaning. Its characteristic is the state of willing. Its function is to accumulate. It is \marginnote{\textcolor{teal}{\footnotesize\{529|471\}}}{}manifested as coordinating. It accomplishes its own and others’ functions, as a senior pupil, a head carpenter, etc., do. But it is evident when it occurs in the marshalling (driving) of associated states in connection with urgent work, remembering, and so on. \textcolor{brown}{\textit{[464]}}

                        \vismParagraph{XIV.136}{136}{}
                        (iii)–(v) What should be said about \emph{applied thought, sustained thought}, and \emph{happiness} has already been said in the commentary on the first jhāna in the Description of the Earth Kasiṇa (\hyperlink{IV.88}{IV.88}{}–\hyperlink{IV.98}{98}{}).

                        \vismParagraph{XIV.137}{137}{}
                        (vi) \emph{Energy} (\emph{viriya}) is the state of one who is vigorous (\emph{vīra}). Its characteristic is marshalling (driving). Its function is to consolidate conascent states. It is manifested as non-collapse. Because of the words: “Bestirred, he strives wisely” (\textbf{\cite{A}II 115}), its proximate cause is a sense of urgency; or its proximate cause is grounds for the initiation of energy. When rightly initiated, it should be regarded as the root of all attainments.

                        \vismParagraph{XIV.138}{138}{}
                        (vii) By its means they live, or it itself lives, or it is just mere living, thus it is life. But its characteristic, etc., should be understood in the way stated under material life (\hyperlink{XIV.59}{§59}{}); for that is life of material things and this is life of immaterial things. This is the only difference here.

                        \vismParagraph{XIV.139}{139}{}
                        (viii) It puts (\emph{ādhiyati}) consciousness evenly (\emph{samaṃ}) on the object, or it puts it rightly (\emph{sammā}) on it, or it is just the mere collecting (\emph{samādhāna}) of the mind, thus it is concentration (\emph{samādhi}). Its characteristic is non-wandering, or its characteristic is non-distraction. Its function is to conglomerate conascent states as water does bath powder. It is manifested as peace. Usually its proximate cause is bliss. It should be regarded as steadiness of the mind, like the steadiness of a lamp’s flame when there is no draught.

                        \vismParagraph{XIV.140}{140}{}
                        (ix) By its means they have faith (\emph{saddahanti}), or it itself is the having of faith, or it is just the act of having faith (\emph{saddahana}), thus it is faith (\emph{saddhā}). Its characteristic is having faith, or its characteristic is trusting. Its function is to clarify, like a water-clearing gem, or its function is to enter into, like the setting out across a flood (cf. \textbf{\cite{Sn}184}). It is manifested as non-fogginess, or it is manifested as resolution. Its proximate cause is something to have faith in, or its proximate cause is the things beginning with hearing the Good Dhamma (\emph{saddhamma}) that constitute the factors of stream-entry.\footnote{\vismAssertFootnoteCounter{63}\vismHypertarget{XIV.n63}{}The four factors of stream-entry (see \textbf{\cite{S}V 347}) are: waiting on good men, hearing the Good Dhamma, wise attention, and practice in accordance with the Dhamma. Again they are: absolute confidence in the Buddha, the Dhamma, and the Sangha, and possession of noble virtue (\textbf{\cite{S}V 343}).} It should be regarded as a hand [because it takes hold of profitable things], as wealth (\textbf{\cite{Sn}182}), and as seed (\textbf{\cite{Sn}77}).

                        \vismParagraph{XIV.141}{141}{}
                        (x) By its means they remember (\emph{saranti}), or it itself remembers, or it is just mere remembering (\emph{saraṇa}), thus it is \emph{mindfulness} (\emph{sati}). It has the characteristic of not wobbling.\footnote{\vismAssertFootnoteCounter{64}\vismHypertarget{XIV.n64}{}“\emph{Apilāpana (‘not wobbling’) }is the steadying of an object, the remembering and not forgetting it, keeping it as immovable as a stone instead of letting it go bobbing about like a pumpkin in water” (\textbf{\cite{Vism-mhṭ}487}).} Its function is not to forget. It is manifested as guarding, or it is manifested as the state of confronting an objective field. Its proximate cause is \marginnote{\textcolor{teal}{\footnotesize\{530|472\}}}{}strong perception, or its proximate cause is the foundations of mindfulness concerned with the body, and so on (see MN 10). It should be regarded, however, as like a pillar because it is firmly founded, or as like a door-keeper because it guards the eye-door, and so on.

                        \vismParagraph{XIV.142}{142}{}
                        (xi)–(xii) It has conscientious scruples (\emph{hiriyati}) about bodily misconduct, etc., thus it is conscience (\emph{hiri}). This is a term for modesty. It is ashamed (\emph{ottappati}) of those same things, thus it is shame (\emph{ottappa}). This is a term for anxiety about evil. Herein, \emph{conscience} has the characteristic of disgust at evil, while \emph{shame} has the characteristic of dread of it. \emph{Conscience} has the function of not doing evil and that in the mode of modesty, while \emph{shame} has the function of not doing it and that in the mode of dread. They are manifested as shrinking from evil in the way already stated. Their proximate causes are self-respect and respect of others [respectively]. \textcolor{brown}{\textit{[465]}} A man rejects evil through \emph{conscience} out of respect for himself, as the daughter of a good family does; he rejects evil through \emph{shame} out of respect for another, as a courtesan does. But these two states should be regarded as the guardians of the world (see \textbf{\cite{A}I 51}).

                        \vismParagraph{XIV.143}{143}{}
                        (xiii)–(xv) By its means they are not greedy (\emph{na lubbhanti}), or it itself is not greedy, or it is just the mere not being greedy (\emph{alubbhana}), thus it is \emph{non-greed }(\emph{alobha}). The same method applies to \emph{non-hate} (\emph{adosa}) and \emph{non-delusion} (\emph{amoha}) [\emph{na dussanti, adussana = adosa}, and \emph{na muyhanti, amuyhana = amoha} (see \hyperlink{XIV.171}{§§171}{},\hyperlink{XIV.161}{161}{})]. Of these, \emph{non-greed} has the characteristic of the mind’s lack of desire for an object, or it has the characteristic of non-adherence, like a water drop on a lotus leaf. Its function is not to lay hold, like a liberated bhikkhu. It is manifested as a state of not treating as a shelter, like that of a man who has fallen into filth. \emph{Non-hate} has the characteristic of lack of savagery, or the characteristic of non-opposing, like a gentle friend. Its function is to remove annoyance, or its function is to remove fever, as sandalwood does. It is manifested as agreeableness, like the full moon. \emph{Non-delusion} has the characteristic of penetrating [things] according to their individual essences, or it has the characteristic of sure penetration, like the penetration of an arrow shot by a skilful archer. Its function is to illuminate the objective field, like a lamp. It is manifested as non-bewilderment, like a guide in a forest. The three should be regarded as the roots of all that is profitable.

                        \vismParagraph{XIV.144}{144}{}
                        (xvi)–(xvii) The tranquillizing of the body is \emph{tranquillity of the body}. The tranquillizing of consciousness is \emph{tranquillity of consciousness}. And here body means the three [mental] aggregates, feeling, [perception and formations] (see \textbf{\cite{Dhs}40}). But both tranquillity of that body and of consciousness have, together, the characteristic of quieting disturbance of that body and of consciousness. Their function is to crush disturbance of the [mental] body and of consciousness. They are manifested as inactivity and coolness of the [mental] body and of consciousness. Their proximate cause is the [mental] body and consciousness. They should be regarded as opposed to the defilements of agitation, etc., which cause unpeacefulness in the [mental] body and in consciousness.

                        \vismParagraph{XIV.145}{145}{}
                        (xviii)–(xix) The light (quick) state of the [mental] body is lightness of the body. The light (quick) state of consciousness is lightness of consciousness. They have the characteristic of quieting heaviness in the [mental] body and in \marginnote{\textcolor{teal}{\footnotesize\{531|473\}}}{}consciousness. Their function is to crush heaviness in the [mental] body and in consciousness. They are manifested as non-sluggishness of the [mental] body and of consciousness. Their proximate cause is the [mental] body and consciousness. They should be regarded as opposed to the defilements of stiffness and torpor, which cause heaviness in the [mental] body and in consciousness.

                        \vismParagraph{XIV.146}{146}{}
                        (xx)–(xxi) The malleable state of the [mental] body is \emph{malleability of body}. The malleable state of consciousness is \emph{malleability of consciousness}. They have the characteristic of quieting rigidity in the [mental] body and in consciousness. Their function is to crush stiffening in the [mental] body and in consciousness. They are manifested as non-resistance. Their proximate cause is the [mental body and consciousness. They should be regarded as opposed to the defilements of views, conceit (pride), etc., which cause stiffening of the [mental body and of consciousness.

                        \vismParagraph{XIV.147}{147}{}
                        (xxii)–(xxiii) The wieldy state of the [mental] body is \emph{wieldiness of body}. The wieldy state of consciousness is \emph{wieldiness of consciousness}. They have the characteristic of quieting unwieldiness in the [mental] body and in consciousness. \textcolor{brown}{\textit{[466]}} Their function is to crush unwieldiness in the [mental] body and in consciousness. They are manifested as success in making [something] an object of the [mental] body and consciousness. Their proximate cause is the [mental] body and consciousness. As bringing trust in things that should be trusted in and as bringing susceptibility of application to beneficial acts, like the refining of gold, they should be regarded as opposed to the remaining hindrances, etc., that cause unwieldiness in the [mental] body and in consciousness.

                        \vismParagraph{XIV.148}{148}{}
                        (xxiv)–(xxv) The proficient state of the [mental] body is \emph{proficiency of body}. The proficient state of consciousness is \emph{proficiency of consciousness}. They have the characteristic of healthiness of the [mental] body and of consciousness. Their function is to crush unhealthiness of the [mental] body and of consciousness. They are manifested as absence of disability. Their proximate cause is the [mental] body and consciousness. They should be regarded as opposed to faithlessness, etc., which cause unhealthiness in the [mental] body and in consciousness.

                        \vismParagraph{XIV.149}{149}{}
                        (xxvi)–(xxvii) The straight state of the [mental] body is \emph{rectitude of body}. The straight state of consciousness is \emph{rectitude of consciousness}. They have the characteristic of uprightness of the [mental] body and of consciousness. Their function is to crush tortuousness in the [mental] body and in consciousness. They are manifested as non-crookedness. Their proximate cause is the [mental] body and consciousness. They should be regarded as opposed to deceit, fraud, etc., which cause tortuousness in the [mental] body and in consciousness.\footnote{\vismAssertFootnoteCounter{65}\vismHypertarget{XIV.n65}{}“And here by tranquilization, etc., of consciousness only consciousness is tranquillized and becomes light, malleable, wieldy, proficient and upright. But with tranquilization, etc., of the [mental] body also the material body is tranquillized, and so on. This is why the twofoldness of states is given by the Blessed One here, but not in all places” (\textbf{\cite{Vism-mhṭ}489}).}

                        \vismParagraph{XIV.150}{150}{}
                        \marginnote{\textcolor{teal}{\footnotesize\{532|474\}}}{}(xxviii) \emph{Zeal} (\emph{desire}) is a term for desire to act. So that zeal has the characteristic of desire to act. Its function is scanning for an object. It is manifested as need for an object. That same [object] is its proximate cause. It should be regarded as the extending of the mental hand in the apprehending of an object.

                        \vismParagraph{XIV.151}{151}{}
                        (xxix) The act of resolving\footnote{\vismAssertFootnoteCounter{66}\vismHypertarget{XIV.n66}{}“‘\emph{The act of resolving’ }should be understood as the act of being convinced \emph{(sanniṭṭhāna) }about an object, not as trusting \emph{(pasādana)” }(\textbf{\cite{Vism-mhṭ}489}). See §140.}is \emph{resolution}. It has the characteristic of conviction. Its function is not to grope. It is manifested as decisiveness. Its proximate cause is a thing to be convinced about. It should be regarded as like a boundary-post owing to its immovableness with respect to the object.

                        \vismParagraph{XIV.152}{152}{}
                        (xxx) It is the maker of what is to be made, it is the maker in the mind (\emph{manamhi kāro}), thus it is attention (\emph{bringing-to-mind—manasi-kāra}). It makes the mind different from the previous [life-continuum] mind, thus it is attention. It has three ways of doing this: as the controller of the object, as the controller of the cognitive series, and as the controller of impulsions. Herein, the controller of \emph{the object} is the maker in the mind, thus it is \emph{attention}. That has the characteristic of conducting (\emph{sāraṇa}). Its function is to yoke associated states to the object. It is manifested as confrontation with an object. Its proximate cause is an object. It should be regarded as the conductor (\emph{sārathi}) of associated states by controlling the object, itself being included in the formations aggregate. \emph{Controller of the cognitive series} is a term for five-door adverting (70). \emph{Controller of impulsions} is a term for mind-door adverting (71). These last two are not included here.

                        \vismParagraph{XIV.153}{153}{}
                        (xxxi) \emph{Specific neutrality} (\emph{tatra-majjhattatā—}lit. “neutrality in regard thereto”) is neutrality (\emph{majjhattatā}) in regard to those states [of consciousness and consciousness-concomitants arisen in association with it]. It has the characteristic of conveying consciousness and consciousness-concomitants evenly. Its function is to prevent deficiency and excess, \textcolor{brown}{\textit{[467]}} or its function is to inhibit partiality. It is manifested as neutrality. It should be regarded as like a conductor (driver) who looks with equanimity on thoroughbreds progressing evenly.

                        \vismParagraph{XIV.154}{154}{}
                        (xxxii)–(xxxiii) \emph{Compassion} and \emph{gladness} should be understood as given in the Description of the Divine Abodes (\hyperlink{IX.92}{IX.§92}{}, \hyperlink{IX.94}{94}{}, \hyperlink{IX.95}{95}{}), except that those are of the fine-material sphere and have attained to absorption, while these are of the sense sphere. This is the only difference. Some, however, want to include among the inconstant both loving-kindness and equanimity. That cannot be accepted for, as to meaning, non-hate itself is loving-kindness, and specific neutrality is equanimity.

                        \vismParagraph{XIV.155}{155}{}
                        (xxxiv)–(xxxvi) \emph{Abstinence from bodily misconduct}: the compound \emph{kāyaduccaritavirati} resolves as \emph{kāyaduccaritato virati}; so also with the other two. But as regards characteristic, etc., these three have the characteristic of non-transgression in the respective fields of bodily conduct, etc.; they have the characteristic of not treading there, is what is said. Their function is to draw back from the fields of bodily misconduct, and so on. They are manifested as the not doing of these things. Their proximate causes are the special qualities of \marginnote{\textcolor{teal}{\footnotesize\{533|475\}}}{}faith, conscience, shame, fewness of wishes, and so on. They should be regarded as the mind’s averseness from evil-doing.

                        \vismParagraph{XIV.156}{156}{}
                        So these are the thirty-six formations that should be understood to come into association with the first profitable consciousness of the sense sphere (1). And as with the first, so with the second (2), the only difference here being promptedness.

                        (3)–(4) Those associated with the third (3) should be understood as all the foregoing except non-delusion (xv). Likewise with the fourth (4), the only difference here being promptedness.

                        (5)–(6) All those stated in the first instance, except happiness (v), come into association with the fifth (5). Likewise with the sixth (6), the only difference here being promptedness.

                        (7)–(8) [Those associated] with the seventh (7) should be understood as [the last] except non-delusion (xv). Likewise with the eighth (8), the only difference here being promptedness.

                        \vismParagraph{XIV.157}{157}{}
                        (9)–(13) All those stated in the first instance, except the three abstinences (xxxiv-xxxvi), come into association with the first of the fine-material profitable [kinds of consciousness] (9). With the second (10) applied thought (iii) is also lacking. With the third (11) sustained thought (iv) is also lacking. With the fourth (12) happiness (v) is also lacking. With the fifth (13) compassion (xxxii) and gladness (xxxiii), among the inconstant, are also lacking.

                        (14)–(17) In the case of the four kinds of immaterial [profitable consciousness] these are the same as the last-mentioned, for it is only the immaterialness that is the difference here.

                        \vismParagraph{XIV.158}{158}{}
                        (18)–(21) As regards the supramundane, firstly, in the case of the path consciousness having the first jhāna they should be understood to be as stated in the case of the first fine-material-sphere consciousness (9). The paths classed as belonging to the second jhāna, etc., should be understood to be as stated in the cases [respectively] of the second fine-material-sphere jhāna, and so on (10)–(13). But the difference here is absence of compassion (xxxii) and gladness (xxxiii),\footnote{\vismAssertFootnoteCounter{67}\vismHypertarget{XIV.n67}{}“Because the path consciousnesses have Nibbāna as their object and because compassion, gladness, etc., have living beings as their object, there is no compassion, etc., in the path” (\textbf{\cite{Vism-mhṭ}491}).} constancy of the abstinences (xxxiv-xxxvi), and supramundaneness. \textcolor{brown}{\textit{[468]}}

                        \vismParagraph{XIV.159}{159}{}
                        II. (22) As regards the \emph{unprofitable}, there are firstly seventeen associated with the first unprofitable consciousness rooted in greed (22), that is to say, thirteen constant given in the texts as such (see \textbf{\cite{Dhs}§365}) and four or-whatever-states.

                        Herein, the thirteen given as such are these: 

                        \ifplastex
                        \begin{tabular}{rrl}
                            contact & (i),\\
                            volition & (ii),\\
                            applied thought & (iii),\\
                            \marginnote{\textcolor{teal}{\footnotesize\{534|476\}}}{}sustained thought & (iv),\\
                            happiness & (v),\\
                            energy & (vi),\\
                            life & (vii),\\
                            concentration & (viii),\\
                            & (xxxvii) & consciencelessness,\\
                            & (xxxviii) & shamelessness,\\
                            & (xxxix) & greed,\\
                            & (xl) & delusion,\\
                            & (xli) & wrong view.
                        \end{tabular}
                        \else
                        \begin{longtblr}[theme=vismNaked,presep=\smallskipamount,postsep=\smallskipamount]{colspec={X[1,r]Q[4em,r]X[1,l]},rowsep=0pt}
                            contact & (i),\\
                            volition & (ii),\\
                            applied thought & (iii),\\
                            \marginnote{\textcolor{teal}{\footnotesize\{534|476\}}}{}sustained thought & (iv),\\
                            happiness & (v),\\
                            energy & (vi),\\
                            life & (vii),\\
                            concentration & (viii),\\
                            & (xxxvii) & consciencelessness,\\
                            & (xxxviii) & shamelessness,\\
                            & (xxxix) & greed,\\
                            & (xl) & delusion,\\
                            & (xli) & wrong view.
                        \end{longtblr}\fi
                        \noindent
                        

                        The four or-whatever-states are these:

                        \ifplastex
                        \begin{tabular}{rrl}
                            zeal & (xxviii), &\\
                            resolution & (xxix), &\\
                            & (xlii) & agitation,\\
                            attention & (xxx). &
                        \end{tabular}
                        \else
                        \begin{longtblr}[theme=vismNaked,presep=\smallskipamount,postsep=\smallskipamount]{colspec={X[1,r]Q[4em,r]X[1,l]},rowsep=0pt}
                            zeal & (xxviii), &\\
                            resolution & (xxix), &\\
                            & (xlii) & agitation,\\
                            attention & (xxx). &
                        \end{longtblr}\fi
                        \noindent
                        

                        \vismParagraph{XIV.160}{160}{}
                        Herein, (xxxvii) it has no conscientious scruples, thus it is \emph{consciencelessness}. (xxxviii) It is unashamed, thus it is \emph{shamelessness}. Of these, \emph{consciencelessness} has the characteristic of absence of disgust at bodily misconduct, etc., or it has the characteristic of immodesty. \emph{Shamelessness} has the characteristic of absence of dread on their account, or it has the characteristic of absence of anxiety about them. This is in brief here. The detail, however, is the opposite of what was said above under conscience (xi) and shame (xii).

                        \vismParagraph{XIV.161}{161}{}
                        (xxxix) By its means they are greedy, or it itself is greedy, or it is just the mere being greedy, thus is it \emph{greed}. (xl) By its means they are deluded, or it itself is deluded, or it is just the mere being deluded, thus it is \emph{delusion}.

                        \vismParagraph{XIV.162}{162}{}
                        Of these, \emph{greed} has the characteristic of grasping an object, like birdlime (lit. “monkey lime”). Its function is sticking, like meat put in a hot pan. It is manifested as not giving up, like the dye of lamp-black. Its proximate cause is seeing enjoyment in things that lead to bondage. Swelling with the current of craving, it should be regarded as taking [beings] with it to states of loss, as a swift-flowing river does to the great ocean.

                        \vismParagraph{XIV.163}{163}{}
                        \emph{Delusion} has the characteristic of blindness, or it has the characteristic of unknowing. Its function is non-penetration, or its function is to conceal the individual essence of an object. It is manifested as the absence of right theory (see \hyperlink{XVII.52}{Ch. XVII, §52}{}), or it is manifested as darkness. Its proximate cause is unwise (unjustified) attention. It should be regarded as the root of all that is unprofitable.

                        \vismParagraph{XIV.164}{164}{}
                        (xli) By its means they see wrongly, or it itself sees wrongly, or it is just the mere seeing wrongly, thus it is \emph{wrong view}. Its characteristic \textcolor{brown}{\textit{[469]}} is unwise (unjustified) interpreting. Its function is to presume. It is manifested as wrong interpreting. Its proximate cause is unwillingness to see Noble Ones, and so on. It should be regarded as the most reprehensible of all.

                        \vismParagraph{XIV.165}{165}{}
                        (xlii) \emph{Agitation} is agitatedness. It has the characteristic of disquiet, like water whipped by the wind. Its function is unsteadiness, like a flag or banner whipped by the wind. It is manifested as turmoil, like ashes flung up by pelting \marginnote{\textcolor{teal}{\footnotesize\{535|477\}}}{}with stones. Its proximate cause is unwise attention to mental disquiet. It should be regarded as distraction of consciousness.

                        \vismParagraph{XIV.166}{166}{}
                        The remaining formations here should be understood as already stated under the profitable. For it is only the unprofitableness that differentiates them as bad.

                        So these are the seventeen formations that should be understood to come into association with the first unprofitable consciousness (22).

                        (23) And as with the first, so with the second (23), but here the difference is promptedness and inconstant [occurrence] of (xliii) stiffening and torpor.

                        \vismParagraph{XIV.167}{167}{}
                        Herein, (xliii) stiffening (\emph{thīnanatā}) is stiffness (\emph{thīna}); making torpid (\emph{middhanatā}) is torpor (\emph{middha}). The meaning is, paralysis due to lack of urgency, and loss of vigour. The compound \emph{thīnamiddha} (\emph{stiffness-and-torpor}) should be resolved into \emph{thīnañ ca middhañ ca}. Herein, \emph{stiffness} has the characteristic of lack of driving power. Its function is to remove energy. It is manifested as subsiding. \emph{Torpor} has the characteristic of unwieldiness. Its function is to smother. It is manifested as laziness, or it is manifested as nodding and sleep.\footnote{\vismAssertFootnoteCounter{68}\vismHypertarget{XIV.n68}{}“Because the paralysis \emph{(saṃhanana) }of consciousness comes about through stiffness, but that of matter through torpor like that of the three aggregates beginning with feeling, therefore torpor is manifested as nodding and sleep” (\textbf{\cite{Vism-mhṭ}493}).} The proximate cause of both is unwise attention to boredom, sloth, and so on.

                        \vismParagraph{XIV.168}{168}{}
                        (24) With the third [unprofitable consciousness] (24) there should be understood to be associated those given for the first (22), excepting wrong view (xli). But here the difference is that there is inconstant [occurrence] of (xliv) pride (conceit).

                        That [pride] has the characteristic of haughtiness. Its function is arrogance. It is manifested as vain gloriousness. Its proximate cause is greed dissociated from views. It should be regarded as like madness.

                        (25) With the fourth (25) should be understood to be associated those given for the second (23), excepting wrong view (xli). And here pride (xliv) is among the inconstant too.

                        \vismParagraph{XIV.169}{169}{}
                        (26) Those given for the first (22), excepting happiness (v), come into association with the fifth (26).

                        (27) And as with the fifth (26), so with the sixth too (27); but the difference here is promptedness and the inconstant [occurrence] of stiffness-and-torpor (xliii).

                        (28) With the seventh (28) should be understood to be associated those given for the fifth (26), except views (xli); but pride (xliv) is inconstant here. (29) With the eighth (29) should be understood to be associated those given for the sixth (27), except views (xli); and here too pride (xliv) is among the inconstant.

                        \vismParagraph{XIV.170}{170}{}
                        (30)–(31) As regards the two [kinds of unprofitable consciousness] rooted in hate, \textcolor{brown}{\textit{[470]}} there are, firstly, eighteen associated with the first (30), that is, eleven constant given in the texts as such (see \textbf{\cite{Dhs}§413}), four or-whatever-states, and three inconstant. Herein the eleven given as such are these:

                        \ifplastex
                        \begin{tabular}{rrl}
                            \marginnote{\textcolor{teal}{\footnotesize\{536|478\}}}{}contact & (i),\\
                            volition & (ii),\\
                            applied thought & (iii),\\
                            sustained thought & (iv),\\
                            energy & (vi),\\
                            life & (vii),\\
                            concentration & (viii),\\
                            consciencelessness & (xxxvii),\\
                            shamelessness & (xxxviii),\\
                            & (xiv) & hate,\\
                            delusion & (xl). &
                        \end{tabular}
                        \else
                        \begin{longtblr}[theme=vismNaked,presep=\smallskipamount,postsep=\smallskipamount]{colspec={X[1,r]Q[4em,r]X[1,l]},rowsep=0pt}
                            \marginnote{\textcolor{teal}{\footnotesize\{536|478\}}}{}contact & (i),\\
                            volition & (ii),\\
                            applied thought & (iii),\\
                            sustained thought & (iv),\\
                            energy & (vi),\\
                            life & (vii),\\
                            concentration & (viii),\\
                            consciencelessness & (xxxvii),\\
                            shamelessness & (xxxviii),\\
                            & (xiv) & hate,\\
                            delusion & (xl). &
                        \end{longtblr}\fi
                        \noindent
                        

                        The four or-whatever-states are these:

                            \ifplastex
                            \begin{tabular}{rrl}
                                zeal & (xxviii), &\\
                                resolution & (xxix),\\
                                agitation & (xlii),\\
                                attention & (xxx).
                            \end{tabular}
                            \else
                            \begin{longtblr}[theme=vismNaked,presep=\smallskipamount,postsep=\smallskipamount]{colspec={X[1,r]Q[4em,r]X[1,l]},rowsep=0pt}
                                zeal & (xxviii), &\\
                                resolution & (xxix),\\
                                agitation & (xlii),\\
                                attention & (xxx).
                            \end{longtblr}\fi
                            \noindent
                            

                        The three inconstant are these:

                            \ifplastex
                            \begin{tabular}{rrl}
                                & (xlvi) & envy,\\
                                & (xlvii) & avarice,\\
                                & (xlviii) & worry.
                            \end{tabular}
                            \else
                            \begin{longtblr}[theme=vismNaked,presep=\smallskipamount,postsep=\smallskipamount]{colspec={X[1,r]Q[4em,r]X[1,l]},rowsep=0pt}
                                & (xlvi) & envy,\\
                                & (xlvii) & avarice,\\
                                & (xlviii) & worry.
                            \end{longtblr}\fi
                            \noindent
                            

                        \vismParagraph{XIV.171}{171}{}
                        Herein, (xlv) by its means they hate, or it itself hates, or it is just mere hating, thus it is \emph{hate} (\emph{dosa}). It has the characteristic of savageness, like a provoked snake. Its function is to spread, like a drop of poison, or its function is to burn up its own support, like a forest fire. It is manifested as persecuting (\emph{dūsana}), like an enemy who has got his chance. Its proximate cause is the grounds for annoyance (see \textbf{\cite{A}V 150}). It should be regarded as like stale urine mixed with poison.

                        \vismParagraph{XIV.172}{172}{}
                        (xlvi) Envying is \emph{envy}. It has the characteristic of being jealous of other’s success. Its function is to be dissatisfied with that. It is manifested as averseness from that. Its proximate cause is another’s success. It should be regarded as a fetter.

                        \vismParagraph{XIV.173}{173}{}
                        (xlvii) Avariciousness is \emph{avarice}. Its characteristic is the hiding of one’s own success that has been or can be obtained. Its function is not to bear sharing these with others. It is manifested as shrinking, or it is manifested as meanness. Its proximate cause is one’s own success. It should be regarded as a mental disfigurement.

                        \vismParagraph{XIV.174}{174}{}
                        (xlviii) The vile (\emph{kucchita}) that is done (\emph{kata}) is villainy (\emph{kukata}).\footnote{\vismAssertFootnoteCounter{69}\vismHypertarget{XIV.n69}{}\emph{Kukata }is not in PED. It is impossible to render into English this “portmanteau” etymology, e.g. \emph{kucchita-kata—kukata, kukutatā … kukkucca, }which depends mostly on a fortuitous parallelism of meaning and verbal forms in the Pali. While useless to strict modern etymologists, it has a definite semantic and mnemonic use.} The state of that is \emph{worry} (\emph{kukkucca}). It has subsequent regret as its characteristic. Its function is to sorrow about what has and what has not been done. It is manifested as remorse. Its proximate cause is what has and what has not been done. It should be regarded as slavery.

                        \vismParagraph{XIV.175}{175}{}
                        \marginnote{\textcolor{teal}{\footnotesize\{537|479\}}}{}The rest are of the kind already described.

                        So these eighteen formations should be understood to come into association with the first [unprofitable consciousness] rooted in hate (30).

                        (31) And as with the first (30), so with the second (31), the only difference, however, being promptedness and the presence of stiffness and torpor (xliii) among the inconstant.

                        \vismParagraph{XIV.176}{176}{}
                        (32)–(33) As regards the two rooted in delusion, firstly: [associated] with [the consciousness that is] associated with uncertainty (32) \textcolor{brown}{\textit{[471]}} are the eleven given in the texts as such thus:

                        \ifplastex
                        \begin{tabular}{rrl}
                            contact & (i), &\\
                            volition & (ii), &\\
                            applied thought & (iii), &\\
                            sustained thought & (iv), &\\
                            energy & (vi), &\\
                            life & (vii), &\\
                            & (xlix) & steadiness of consciousness,\\
                            consciencelessness & (xxxvii), &\\
                            shamelessness & (xxxviii), &\\
                            delusion & (xl), &\\
                            & (l) & uncertainty.
                        \end{tabular}
                        \else
                        \begin{longtblr}[theme=vismNaked,presep=\smallskipamount,postsep=\smallskipamount]{colspec={X[1,r]Q[4em,r]X[1,l]},rowsep=0pt}
                            contact & (i), &\\
                            volition & (ii), &\\
                            applied thought & (iii), &\\
                            sustained thought & (iv), &\\
                            energy & (vi), &\\
                            life & (vii), &\\
                            & (xlix) & steadiness of consciousness,\\
                            consciencelessness & (xxxvii), &\\
                            shamelessness & (xxxviii), &\\
                            delusion & (xl), &\\
                            & (l) & uncertainty.
                        \end{longtblr}\fi
                        \noindent
                        

                        The or-whatever-states are these two:

                        \ifplastex
                        \begin{tabular}{rrl}
                            agitation & (xlii), &\\
                            attention & (xxx). &
                        \end{tabular}
                        \else
                        \begin{longtblr}[theme=vismNaked,presep=\smallskipamount,postsep=\smallskipamount]{colspec={X[1,r]Q[4em,r]X[1,l]},rowsep=0pt}
                            agitation & (xlii), &\\
                            attention & (xxx). &
                        \end{longtblr}\fi
                        \noindent
                        

                        And these together total thirteen.

                        \vismParagraph{XIV.177}{177}{}
                        Herein, (xlix) \emph{steadiness of consciousness} is weak concentration (viii) consisting in mere steadiness in occurrence.\footnote{\vismAssertFootnoteCounter{70}\vismHypertarget{XIV.n70}{}“‘\emph{Mere steadiness in occurrence’ }is mere presence for a moment. That it is only “mere steadiness in occurrence” owing to the mere condition for the steadiness of the mind \emph{(ceto) }is because of lack of real steadiness due to absence of decidedness \emph{(nicchaya), }and it is incapable of being a condition for such steadiness in continuity (see §188) as the steadiness of consciousness stated thus: ‘like the steadiness of a flame sheltered from a draught’ (\hyperlink{XIV.139}{XIV.139}{})” (\textbf{\cite{Vism-mhṭ}495}).}

                        (1) It is without wish to cure (\emph{vigatā cikicchā}), thus it is \emph{uncertainty} (\emph{vicikicchā}). It has the characteristic of doubt. Its function is to waver. It is manifested as indecisiveness, or it is manifested as taking various sides. Its proximate cause is unwise attention. It should be regarded as obstructive of theory (see \hyperlink{XVII.52}{XVII.52}{}).

                        The rest are as already described.

                        \vismParagraph{XIV.178}{178}{}
                        (33) [The consciousness] associated with agitation (33) has the same [formations as the consciousness] associated with uncertainty (32), except for uncertainty (1). But with the absence of uncertainty resolution (xxix) arises here. So with that they are likewise thirteen, and concentration (viii) is stronger because of the presence of resolution. Also agitation is given in the texts as such, while resolution (xxix) and attention (xxx) are among the or-whatever-states.

                        \marginnote{\textcolor{teal}{\footnotesize\{538|480\}}}{}Thus should the unprofitable formations be understood.

                        \vismParagraph{XIV.179}{179}{}
                        III. As regards the \emph{indeterminate}, firstly, the \emph{resultant indeterminate} (34)–(69) are twofold, classed as those without root-cause and those with root-cause. Those associated with resultant consciousness without root-cause (34)–(41), (50)–(56) are those without root-cause.

                        Herein, firstly, those associated with the profitable resultant (34) and unprofitable resultant (50) eye-consciousness are the four given in the texts as such, namely:

                            \ifplastex
                            \begin{tabular}{rrl}
                                contact & (i), &\\
                                volition & (ii), &\\
                                life & (vii), &\\
                                steadiness of consciousness & (xlix), &
                            \end{tabular}
                            \else
                            \begin{longtblr}[theme=vismNaked,presep=\smallskipamount,postsep=\smallskipamount]{colspec={X[1,r]Q[4em,r]X[1,l]},rowsep=0pt}
                                contact & (i), &\\
                                volition & (ii), &\\
                                life & (vii), &\\
                                steadiness of consciousness & (xlix), &
                            \end{longtblr}\fi
                            \noindent
                            which amount to five with

                            \ifplastex
                            \begin{tabular}{rrl}
                                attention & (xxx) &
                            \end{tabular}
                            \else
                            \begin{longtblr}[theme=vismNaked,presep=\smallskipamount,postsep=\smallskipamount]{colspec={X[1,r]Q[4em,r]X[1,l]},rowsep=0pt}
                                attention & (xxx) &
                            \end{longtblr}\fi
                            \noindent
                            as the only or-whatever-state.

                        These same kinds are associated with ear-, nose-, tongue-, and body-consciousness (35)–(38), (51)–(54).

                        \vismParagraph{XIV.180}{180}{}
                        Those associated with both kinds of resultant mind-element (39), (55) come to eight by adding applied thought (iii), sustained thought (iv) and resolution (xxix). Likewise those associated with the threefold mind-consciousness-element with root-cause (40), (41), (56). But here (40) that accompanied by joy should be understood to have happiness (v) also in addition to that.

                        \vismParagraph{XIV.181}{181}{}
                        The [formations] associated with resultant consciousness with root-cause (42)–(49) are those with root-cause. Of these, firstly, those associated with the sense-sphere resultant [consciousness] with root-cause are similar to the formations associated with the eight sense-sphere [consciousnesses] (1)–(8). But of the inconstant ones, compassion (xxxii) and gladness (xxxiii) are not among the resultant because they have living beings as their object. For the resultant ones of the sense-sphere have only limited objects. And not only compassion and gladness but also the three abstinences (xxxiv)–(xxxvi) are not among the resultant; \textcolor{brown}{\textit{[472]}} for it is said that “the five training precepts are profitable only” (\textbf{\cite{Vibh}291}).

                        \vismParagraph{XIV.182}{182}{}
                        (57)–(69) Those associated with the resultant consciousness of the fine-material sphere (57)–(61), the immaterial sphere (62)–(65), and the supramundane (66)–(69) are similar to the formations associated with the profitable consciousnesses of those kinds (9)–(21) too.

                        \vismParagraph{XIV.183}{183}{}
                        (70)–(89) \emph{Functional indeterminate} [formations] are also twofold classed as those without root-cause (70)–(72) and those with root-cause (73)–(80). Those without root-cause are associated with functional consciousness without root-cause; and they are the same as those associated [respectively] with profitable resultant mind-element (39) and the pair of mind-consciousness-elements without root-cause (40)–(41). But in the case of the two mind-consciousness-elements\marginnote{\textcolor{teal}{\footnotesize\{539|481\}}}{} (71)–(72), energy (vi) is additional, and because of the presence of energy, concentration (viii) is strong. This is the difference here.

                        \vismParagraph{XIV.184}{184}{}
                        Those associated with functional consciousness with root-cause (73)–(80) are those with root-cause. Of these, firstly, those associated with the eight sense-sphere functional consciousnesses (73)–(80) are similar to the formations associated with the eight sense-sphere profitable (1)–(8), except for the abstinences (xxxiv)–(xxxvi).

                        Those associated with the functional [consciousnesses] of the fine-material sphere (81)–(85) and the immaterial sphere (86)–(89) are in all aspects similar to those associated with profitable consciousness (9)–(17).

                        This is how formations should be understood as indeterminate.

                        This is the section of the detailed explanation dealing with the formations aggregate.
            \section[\vismAlignedParas{§185–210}C. Classification of the Aggregates]{C. Classification of the Aggregates}

                \vismParagraph{XIV.185}{185}{}
                The foregoing section, firstly, is that of the detailed explanation of the aggregates according to the Abhidhamma-Bhājaniya [of the Vibhaṅga]. But the aggregates have been given in detail by the Blessed One [in the Suttanta-Bhājaniya] in this way: “Any materiality whatever, whether past, future or present, internal or external, gross or subtle, inferior or superior, far or near: all that together in the mass and in the gross is called the materiality aggregate. Any feeling whatever … Any perception whatever … Any formations whatever … Any consciousness whatever, whether past, future or present … all that together in the mass and in the gross is called the consciousness aggregate” (Vibh 1–9; cf. \textbf{\cite{M}III 17}).
                \subsection[\vismAlignedParas{§186–196}Materiality]{Materiality}

                    \vismParagraph{XIV.186}{186}{}
                    Herein, the word \emph{whatever} includes without exception. \emph{Materiality} prevents over-generalization. Thus materiality is comprised without exception by the two expressions. Then he undertakes its exposition as \emph{past, future} and \emph{present}, etc.; for some of it is classed as past and some as future, and so on. So also in the case of feeling, and so on.

                    Herein, the materiality called (i) \emph{past} is fourfold, according to (a) extent, (b) continuity, (c) period, and (d) moment. Likewise (ii) the \emph{future} and (iii) the \emph{present}.\footnote{\vismAssertFootnoteCounter{71}\vismHypertarget{XIV.n71}{}“Here when the time is delimited by death and rebirth-linking the term ‘extent’ is applicable. It is made known through the Suttas in the way beginning ‘Was I in the past?’ (\textbf{\cite{M}I 8}); for the past state is likewise mentioned as ‘extent’ in the Bhaddekaratta Sutta too in the way beginning, ‘He does not follow what is past (the past extent)’ (\textbf{\cite{M}III 1} 88). But when it is delimited in the ultimate sense as in the Addhāniruttipatha Sutta thus, ‘Bhikkhus, there are three extents, the past extent, the future extent, and the present extent’ (It 53), then it is appropriate as delimited by moment. Herein, the existingness of the present is stated thus, ‘Bhikkhus, of matter that is born … manifested, it is said that: “It exists”’ (\textbf{\cite{S}IV 72}), and pastness and futureness are respectively called before and after that” (\textbf{\cite{Vism-mhṭ}496}).}

                    \vismParagraph{XIV.187}{187}{}
                    \marginnote{\textcolor{teal}{\footnotesize\{540|482\}}}{}Herein, (a) firstly, \emph{according to extent}: in the case of a single becoming of one [living being], previous to rebirth-linking is\emph{ past}, subsequent to death is \emph{future}, between these two is \emph{present}.

                    \vismParagraph{XIV.188}{188}{}
                    (b) \emph{According to continuity}: that [materiality] which has like or single origination\footnote{\vismAssertFootnoteCounter{72}\vismHypertarget{XIV.n72}{}“Cold temperature is like with cold, and hot with hot. But that temperature which falls on the body, whether hot or cold, and occurs as a continuity in one mode, being neither less nor more, is called ‘single temperature.’ The word ‘single’ is used because of the plurality of ‘like’ temperature. So too with nutriment. \emph{‘In one cognitive series, in one impulsion’ }refers respectively to five-door and mind-door consciousness. The explanations of \emph{continuity }and \emph{period }are given in the Commentaries for the purpose of helping the practice of insight” (\textbf{\cite{Vism-mhṭ}496}).} by temperature and single origination by nutriment, though it occurs successively, \textcolor{brown}{\textit{[473]}} is \emph{present}. That which, previous to that, was of unlike origination by temperature and nutriment is \emph{past}. That which is subsequent is \emph{future}. That which is born of consciousness and has its origination in one cognitive series, in one impulsion, in one attainment, is \emph{present}. Previous to that is \emph{past}. Subsequent to that is \emph{future}. There is no special classification into past continuity, etc., of that which has its origination in kamma, but its pastness, etc., should be understood according as it supports those which have their origination through temperature, nutriment, and consciousness.

                    \vismParagraph{XIV.189}{189}{}
                    (c) \emph{According to period}: any period among those such as one minute, morning, evening, day-and-night, etc., that occurs as a continuity, is called \emph{present}. Previous to that is \emph{past}. Subsequent is \emph{future}.

                    \vismParagraph{XIV.190}{190}{}
                    (d) \emph{According to moment}: what is included in the trio of moments, [that is to say, arising, presence, and dissolution] beginning with arising is called \emph{present}. At a time previous to that it is \emph{future}. At a time subsequent to that it is \emph{past}.\footnote{\vismAssertFootnoteCounter{73}\vismHypertarget{XIV.n73}{}In these two paragraphs “\emph{past” }and “\emph{future” }refer not to time, as in the other paragraphs, but to the materiality.}

                    \vismParagraph{XIV.191}{191}{}
                    Furthermore, that whose functions of cause and condition\footnote{\vismAssertFootnoteCounter{74}\vismHypertarget{XIV.n74}{}“‘\emph{Cause’ (hetu) }is what gives birth \emph{(janaka); ‘condition’ (paccaya) }is what consolidates \emph{(upatthambhaka). }Their respective functions are arousing and consolidating. Just as the seed’s function is to arouse the sprout and that of the earth, etc., is to consolidate it, and just as kamma’s function is to arouse result as matter that is due to kamma performed, and that of nutriment is to consolidate it, so the function of those [conditions] that give birth to each material group and each thought-arising and serve as kamma and proximity-conditions, etc., for them, and the function of those that consolidate them and serve as conascence, prenascence, and postnascence conditions for them may be construed accordingly as appropriate.

                            “Because there is similarity and dissimilarity in temperature, etc., in the way stated, the pastness, etc., of material instances originated by it are stated according to continuity. But there is no such similarity and dissimilarity in the kamma that gives birth to a single becoming, so instead of stating according to continuity the pastness, etc., of material instances originated by that, it is stated according to what consolidates. However, when there comes to be reversal of sex, then the male sex disappears owing to powerful unprofitable kamma, and the female sex appears owing to weak profitable kamma; and the female sex disappears owing to weak unprofitable kamma, while the male sex appears owing to powerful profitable kamma (see \textbf{\cite{Dhs-a}321}). So there is in fact dissimilarity in what is originated by kamma and consequent dissimilarity in what is past, etc., in accordance with the continuity of these as well. But it is not included because it does not happen always” (\textbf{\cite{Vism-mhṭ}497}).} have elapsed is \emph{past}. That whose function of cause is finished and whose function of condition \marginnote{\textcolor{teal}{\footnotesize\{541|483\}}}{}is unfinished is \emph{present}. That which has not attained to either function is \emph{future}. Or alternatively, the moment of the function is \emph{present}. At a time previous to that it is \emph{future}. At a time subsequent to that it is \emph{past}.

                    And here only the explanations beginning with the \emph{moment} are absolutely literal. The rest are in a figurative [or relative] sense.

                    \vismParagraph{XIV.192}{192}{}
                    (iv)–(v) The division into \emph{internal and external} is as already stated (\hyperlink{XIV.73}{§73}{}). Besides, it is internal in the sense of one’s own\footnote{\vismAssertFootnoteCounter{75}\vismHypertarget{XIV.n75}{}\emph{Niyakajjhatta—}“internally in the sense of one’s own”: four kinds of \emph{ajjhatta }(internal, lit. “belonging to oneself”) are mentioned in the commentaries and sub-commentaries (see \textbf{\cite{Dhs-a}46}): \emph{gocarajjhatta—}internally as range or resort (\textbf{\cite{M-a}IV 161}; II 90, 292), \emph{ajjhattajjhata—}internally as such (\textbf{\cite{Vism-mhṭ}152}), \emph{niyakajjhatta—}internally in the sense of one’s own (\hyperlink{IV.141}{IV.141}{}, \hyperlink{IX.114}{IX.114}{}, this ref.; \textbf{\cite{M-a}IV 161}), \emph{visayajjhatta—}internally as objective field (\textbf{\cite{M-a}IV 160}).} that should be understood here as internal and that of another person as external.

                    (vi)–(vii) \emph{Gross and subtle} are also as already stated (\hyperlink{XIV.73}{§73}{}).

                    \vismParagraph{XIV.193}{193}{}
                    (viii)–(ix) \emph{Inferior and superior} are twofold, namely, figuratively (relatively) and absolutely (literally). Herein, the materiality of the Sudassin deities is inferior to the materiality of the Akaniṭṭha (Highest) deities. That same materiality [of the Sudassin deities] is superior to the materiality of the Sudassa deities. Thus, firstly, should inferiority and superiority be understood figuratively (relatively) down as far as the denizens of hell. But absolutely (literally) it is inferior where it arises as unprofitable result, and it is superior where it arises as profitable result.\footnote{\vismAssertFootnoteCounter{76}\vismHypertarget{XIV.n76}{}Profitable result is superior because it produces a desirable object (see Vism-mhṭ 498). This question is treated at length at \textbf{\cite{Vibh-a}9f.}}

                    \vismParagraph{XIV.194}{194}{}
                    (x)–(xi) \emph{Far and near}: this is also as already described (\hyperlink{XIV.73}{§73}{}). Besides, relative farness and nearness should be understood here according to location.

                    \vismParagraph{XIV.195}{195}{}
                    \emph{All that together in the mass and in the gross}: by making all that materiality, separately described by the words “past,” etc., into a collection by understanding its oneness, in other words, its characteristic of being molested (\emph{ruppana}), it comes to be called the materiality (\emph{rūpa}) aggregate. This is the meaning here.

                    \vismParagraph{XIV.196}{196}{}
                    By this, too, it is shown that the materiality aggregate is all materiality, which all comes into the collection with the characteristic of being molested; for there is no materiality aggregate apart from materiality. \textcolor{brown}{\textit{[474]}}

                    And just as in the case of materiality, so also feeling, etc., [are respectively shown as the feeling aggregate, etc.,] since they come under the collections with the [respective] characteristics of being felt, etc.; for there is no feeling aggregate apart from feeling and so on.
                \subsection[\vismAlignedParas{§197–209}Feeling]{Feeling}

                    \vismParagraph{XIV.197}{197}{}
                    \marginnote{\textcolor{teal}{\footnotesize\{542|484\}}}{}In the classification (i)–(iii) into \emph{past}, etc., the past, future, and present state of feeling should be understood according to continuity and according to moment and so on.

                    Herein, \emph{according to continuity}, that included in a single cognitive series, a single impulsion, a single attainment, and that occurring in association with an objective field of one kind,\footnote{\vismAssertFootnoteCounter{77}\vismHypertarget{XIV.n77}{}“The feeling that accompanies the faith, etc., occurring in one who sees an image of the Buddha or who hears the Dhamma, even for a whole day, is ‘present’” (Vism-mhṭ 499).} is \emph{present}. Before that is \emph{past}. Subsequent is \emph{future}.

                    \emph{According to moment}, etc.: that feeling included in the trio of moments, which is in between the past time and the future time, and which is performing its own function, is \emph{present}. Before that is \emph{past}. Subsequent is \emph{future}.

                    \vismParagraph{XIV.198}{198}{}
                    (iv)–(v)The classification into\emph{ internal} and \emph{external} should be understood according to the internal in the sense of one’s own.

                    (vi)–(vii) The classification into \emph{gross }and \emph{subtle} should be understood (a) according to kind, (b) individual essence, (c) person, and (d) the mundane and supramundane, as stated in the Vibhaṅga in the way beginning “Unprofitable feeling is gross, profitable and indeterminate feeling is subtle, [profitable and unprofitable feeling is gross, indeterminate feeling is subtle]” (\textbf{\cite{Vibh}3}), and so on.

                    \vismParagraph{XIV.199}{199}{}
                    (a) \emph{According to kind}, firstly: unprofitable feeling is a state of disquiet, because it is the cause of reprehensible actions and because it produces burning of defilement, so it is \emph{gross} [compared] with profitable feeling. And because it is accompanied by interestedness and drive and result, and because of the burning of the defilements, and because it is reprehensible, it is gross compared with resultant indeterminate. Also because it is accompanied by result, because of the burning of the defilements, and because it is attended by affliction and is reprehensible, it is gross compared with functional indeterminate. But in the opposite sense profitable and indeterminate feeling are subtle compared with unprofitable feeling. Also the two, that is, profitable and unprofitable feeling, involve interestedness, drive and result, so they are respectively gross compared with the twofold indeterminate. And in the opposite sense the twofold indeterminate is subtle compared with them. This, firstly, is how grossness and subtlety should be understood according to kind.

                    \vismParagraph{XIV.200}{200}{}
                    (b) \emph{According to individual essence}: painful feeling is gross compared with the others because it is without enjoyment, it involves intervention, causes disturbance, creates anxiety, and is overpowering. The other two are subtle compared with the painful because they are satisfying, peaceful, and superior, and respectively agreeable and neutral. Both the pleasant and the painful are gross compared with the neither-painful-nor-pleasant because they involve intervention, cause disturbance and are obvious. The latter is subtle in the way \marginnote{\textcolor{teal}{\footnotesize\{543|485\}}}{}aforesaid compared with both the former. Thus should grossness and subtlety be understood according to individual essence.

                    \vismParagraph{XIV.201}{201}{}
                    (c) \emph{According to person}: feeling in one who has no attainment is gross compared with that in one who has one, because it is distracted by a multiple object. In the opposite sense the other is subtle. This is how grossness and subtlety should be understood according to person. \textcolor{brown}{\textit{[475]}}

                    \vismParagraph{XIV.202}{202}{}
                    (d) \emph{According to the mundane and supramundane}: feeling subject to cankers is mundane, and that is gross compared with that free from cankers, because it is the cause for the arising of cankers, is liable to the floods, liable to the bonds, liable to the ties, liable to the hindrances, liable to the clingings, defilable, and shared by ordinary men. The latter, in the opposite sense, is subtle compared with that subject to cankers. This is how grossness and subtlety should be understood according to the mundane and supramundane.

                    \vismParagraph{XIV.203}{203}{}
                    Herein, one should beware of mixing up [the classifications] according to kind and so on. For although feeling associated with unprofitable resultant body-consciousness is subtle according to kind because it is indeterminate, it is nevertheless gross according to individual essence, and so on. And this is said: “Indeterminate feeling is subtle, painful feeling is gross. The feeling in one with an attainment is subtle, that in one with no attainment is gross. Feeling free from cankers is subtle, feeling accompanied by cankers is gross” (\textbf{\cite{Vibh}3}). And like painful feeling, so also pleasant, etc., is gross according to kind and subtle according to individual essence.

                    \vismParagraph{XIV.204}{204}{}
                    Therefore feeling’s grossness and subtlety should be understood in such a way that there is no mixing up of the classifications according to kind and so on. For instance, [when it is said] “The indeterminate according to kind is subtle compared with the profitable and the unprofitable,” the individual-essence class, etc., must not be insisted upon like this: “Which kind of indeterminate? Is it the painful? Is it the pleasant? Is it that in one with an attainment? Is it that in one with no attainment? Is it that subject to cankers? Is it that free from cankers?” and so in each instance.

                    \vismParagraph{XIV.205}{205}{}
                    Furthermore, because of the words “Or feeling should be regarded as gross or subtle in comparison with this or that feeling” (\textbf{\cite{Vibh}4}), among the unprofitable, etc., feeling accompanied by hate, too, is gross compared with that accompanied by greed because it burns up its own support, like a fire; and that accompanied by greed is subtle. Also, that accompanied by hate is gross when the hate is constant, and subtle when it is inconstant. And the constant is gross when giving result that lasts for the aeon, while the other is subtle. And of those giving result lasting for the aeon the unprompted is gross, while the other is subtle. But that accompanied by greed is gross when associated with [false] view, while the other is subtle. That also when constant and giving result lasting for the aeon and unprompted is gross, while the others are subtle. And without distinction the unprofitable with much result is gross, while that with little result is subtle. But the profitable with little result is gross, while that with much result is subtle.

                    \vismParagraph{XIV.206}{206}{}
                    \marginnote{\textcolor{teal}{\footnotesize\{544|486\}}}{}Furthermore, the profitable of the sense sphere is gross; that of the fine-material sphere is subtle; next to which the immaterial, and next the supramundane [should be similarly compared]. That of the sense sphere is gross in giving, while it is subtle in virtue; next, that in development. Also, that in development is gross with two root-causes, while with three root-causes it is subtle. Also that with three root-causes is gross when prompted, while it is subtle when unprompted. That of the fine-material sphere is gross in the first jhāna, [while it is subtle in the second jhāna. That also of the second jhāna is gross] … of the fifth jhāna is subtle. And that of the immaterial sphere associated with the base consisting of boundless space is gross … \textcolor{brown}{\textit{[476]}} that associated with the base consisting of neither-perception-nor-non-perception is subtle only. And the supramundane associated with the stream-entry path is gross … that associated with the Arahant path is subtle only. The same method applies also to resultant and functional feeling in the various planes and to feeling stated according to pain, etc., according to one with no attainment, etc., and according to that subject to cankers, and so on.

                    \vismParagraph{XIV.207}{207}{}
                    Then according to location, painful feelings in hell are gross, while in the animal generation they are subtle … Those among the Paranimmitavasavatti Deities are subtle only. And the pleasant should be construed throughout like the painful where suitable.

                    \vismParagraph{XIV.208}{208}{}
                    And according to physical basis, any feeling that has an inferior physical basis is gross, while one with a superior physical basis is subtle.

                    (viii)–(ix) What is gross should be regarded as \emph{inferior} in the inferior-superior classification, and what is subtle \emph{superior}.

                    \vismParagraph{XIV.209}{209}{}
                    [(x)–(xi) The word \emph{far} is explained in the Vibhaṅga in the way beginning “The unprofitable is far from the profitable and indeterminate” (\textbf{\cite{Vibh}4}) and the word \emph{near} in the way beginning “Unprofitable feeling is near to unprofitable feeling” (\textbf{\cite{Vibh}4}). Therefore, unprofitable feeling is far from the profitable and the indeterminate because of dissimilarity, unconnectedness, and non-resemblance. The profitable and the indeterminate are likewise far from the unprofitable. And so in all instances. But unprofitable feeling is near to unprofitable feeling because of similarity and resemblance.

                    This is the section of the detailed explanation dealing with the past, etc., classifications of the feeling aggregate.
                \subsection[\vismAlignedParas{§210}Perception, Formations and Consciousness]{Perception, Formations and Consciousness}

                    \vismParagraph{XIV.210}{210}{}
                    This should also be understood of the perception, etc., associated with any kind of feeling.
            \section[\vismAlignedParas{§210–229}D. Classes of Knowledge of the Aggregates]{D. Classes of Knowledge of the Aggregates}

                Having understood this, again as regards these same aggregates:
                \begin{verse}
                    Knowledge of aggregates is classed\\{}
                    (1) As to order, and (2) distinction,\\{}
                    (3) As to neither less nor more,\\{}
                    (4) And likewise as to simile,\\{}
                    \marginnote{\textcolor{teal}{\footnotesize\{545|487\}}}{}(5) And twice as to how to be seen,\\{}
                    (6) And as to good for one seeing thus—\\{}
                    This is the way of exposition\\{}
                    That a wise man should rightly know.
                \end{verse}


                \vismParagraph{XIV.211}{211}{}
                \emph{1.} Herein, \emph{as to order}: order is of several kinds, namely, order of arising, order of abandoning, order of practice, order of plane, order of teaching.

                Herein, “First there comes to be the foetus in the first stage, then there comes to be the foetus in the second stage” (\textbf{\cite{S}I 206}), etc., is \emph{order of arising}. “Things to be abandoned by seeing, things to be abandoned by development” (\textbf{\cite{Dhs}1}), etc., is \emph{order of abandoning}. “Purification of virtue \textcolor{brown}{\textit{[477]}} … purification of consciousness” (\textbf{\cite{M}I 148}), etc., is \emph{order of practice}. “The sense sphere, the fine-material sphere” (\textbf{\cite{Paṭis}I 83}), etc., is \emph{order of plane}. “The four foundations of mindfulness, the four right efforts” (\textbf{\cite{D}II 120}), etc., or “Talk on giving, talk on virtue” (\textbf{\cite{M}I 379}), etc., is \emph{order of teaching}.

                \vismParagraph{XIV.212}{212}{}
                Of these, firstly, \emph{order of arising} is not applicable here because the aggregates do not arise in the order in which they are successively dealt with, as is the case with “the foetus in the first stage,” etc., nor is \emph{order of abandoning }applicable, because the profitable and indeterminate are not to be abandoned; nor is \emph{order of practice}, because what is unprofitable is not to be practiced; nor is \emph{order of plane}, because feeling, etc., are included in all four planes.

                \vismParagraph{XIV.213}{213}{}
                \emph{Order of teaching} is appropriate however; for there are those people who, while teachable, have fallen into assuming a self among the five aggregates owing to failure to analyze them; and the Blessed One is desirous of releasing them from the assumption by getting them to see how the [seeming] compactness of mass [in the five aggregates] is resolved; and being desirous of their welfare, he first, for the purpose of their easy apprehension, taught the materiality aggregate, which is gross, being the objective field of the eye, etc.; and after that, feeling, which feels matter as desirable and undesirable; then perception, which apprehends the aspects of feeling’s objective field, since “What one feels, that one perceives” (\textbf{\cite{M}I 293}); then formations, which form volitionally through the means of perception; and lastly, consciousness, which these things beginning with feeling have as their support, and which dominates them.\footnote{\vismAssertFootnoteCounter{78}\vismHypertarget{XIV.n78}{}“Consciousness dominates because of the words, ‘Dhammas have mind as their forerunner’ (Dhp l), ‘Dhammas (states) that have parallel turn-over with consciousness’ (\textbf{\cite{Dhs}§1522}), and ‘The king, lord of the six doors (?)’” (\textbf{\cite{Vism-mhṭ}503}).}

                This, in the first place, is how the exposition should be known as to order.

                \vismParagraph{XIV.214}{214}{}
                \emph{2. As to distinction}: as to the distinction between aggregates and aggregates-as-objects-of-clinging. But what is the distinction between them? Firstly, \emph{aggregates} is said without distinguishing.\emph{ Aggregates [as objects] of clinging} is said distinguishing those that are subject to cankers and are liable to the clingings, according as it is said: “Bhikkhus, I shall teach you the five aggregates and the five aggregates [as objects] of clinging. Listen … And what, bhikkhus, are the five aggregates? Any kind of materiality whatever, bhikkhus, whether past, future or present … far or \marginnote{\textcolor{teal}{\footnotesize\{546|488\}}}{}near: this is called the materiality aggregate. Any kind of feeling whatever … Any kind of perception whatever … Any kind of formations whatever … Any kind of consciousness whatever … far or near: this is called the consciousness aggregate. These, bhikkhus, are called the five aggregates. And what, bhikkhus, are the five aggregates [as objects] of clinging? Any kind of materiality whatever … far or near, that is subject to cankers and liable to the clingings: this is called the materiality aggregate [as object] of clinging. Any kind of feeling whatever … Any kind of perception whatever … Any kind of formations whatever … Any kind of consciousness whatever … far or near, that is subject to cankers and liable to the clingings: this is called the consciousness aggregate [as object] of clinging. These, bhikkhus, are called the five aggregates [as objects] of clinging” (\textbf{\cite{S}III 47}). \textcolor{brown}{\textit{[478]}}

                \vismParagraph{XIV.215}{215}{}
                Now, while there is feeling, etc., both free from cankers [and subject to them],\footnote{\vismAssertFootnoteCounter{79}\vismHypertarget{XIV.n79}{}\emph{Sammohavinodanī} (Be) (Khandha Vibhaṅga Commentary) in the identical passage, reads \emph{vedanādayo anāsavā pi sāsavā pi atthi. }Ee and Ae read \emph{vedanādayo anāsavā pi atthi.}} not so materiality. However, since materiality can be described as a [simple] aggregate in the sense of a total, it is therefore mentioned among the [simple] aggregates. And since it can be described as an aggregate [that is the object] of clinging in the sense of a total and in the sense of being subjected to cankers, that [same materiality] is therefore mentioned among the aggregates [as objects] of clinging too. But feeling, etc., are only mentioned among the [simple] aggregates when they are free from cankers. When they are subject to cankers, they are mentioned among the aggregates [as objects] of clinging. And here the meaning of the term “aggregates as objects of clinging” should be regarded as this: aggregates that are the resort of clinging are aggregates of clinging. But here all these taken together are intended as aggregates.

                \vismParagraph{XIV.216}{216}{}
                \emph{3. As to neither less nor more}: but why are five aggregates, neither less nor more, mentioned by the Blessed One? (a) Because all formed things that resemble each other fall into these groups, (b) because that is the widest limit as the basis for the assumption of self and what pertains to self, and (c) because of the inclusion\footnote{\vismAssertFootnoteCounter{80}\vismHypertarget{XIV.n80}{}\emph{Avarodha—}“inclusion”: not in PED. The term \emph{etaparama—}“the widest limit” is not mentioned in PED. See \textbf{\cite{M}I 80}, 339; \textbf{\cite{S}V 119}; \textbf{\cite{M-a}} III, 281. Cf. also \emph{etāvaparama, }\textbf{\cite{M}I 246}.} by them of the other sorts of aggregates.

                \vismParagraph{XIV.217}{217}{}
                (a) When the numerous categories of formed states are grouped together according to similarity,\footnote{\vismAssertFootnoteCounter{81}\vismHypertarget{XIV.n81}{}“When all formed dhammas are grouped together according to similarity, they naturally fall into five aggregates. Herein, it is the items that are the same owing to the sameness consisting respectively in ‘molesting,’ etc., that are to be understood as ‘similar.’ Among them, those that are strong in the volition whose nature is accumulating with the function of forming the formed, are called the formations aggregate. And the others, that is, contact, etc., which are devoid of the distinguishing characteristics of ‘being molested,’ etc., may also be so regarded under the generality of forming the formed. But the similarities consisting in touching are not describable separately by the word ‘aggregate,’ and so that is why no aggregates of contact, etc., have been stated by the Perfect One who knows the similarities of dhammas. ‘Bhikkhus, whatever ascetics or brahmans there are who are asserters of eternity and declare the self and the world to be eternal, all do so depending and relying on these same five aggregates or on one or other of them’ (cf. \textbf{\cite{S}IV 46}), and so on” (\textbf{\cite{Vism-mhṭ}503}).} materiality forms one aggregate through being grouped \marginnote{\textcolor{teal}{\footnotesize\{547|489\}}}{}together according to similarity consisting in materiality; feeling forms one aggregate through being grouped together according to similarity consisting in feeling; and so with perception and the other two. So they are stated as five because similar formed things fall into groups.

                \vismParagraph{XIV.218}{218}{}
                (b) And this is the extreme limit as the basis for the assumption of self and what pertains to self, that is to say, the five beginning with materiality. For this is said: “Bhikkhus, when matter exists, it is through clinging to matter, through insisting upon (interpreting) matter, that such a view as this arises: ‘This is mine, this is I, this is my self.’ When feeling exists … When perception exists … When formations exist … When consciousness exists, it is through clinging to consciousness, through insisting upon (interpreting) consciousness, that such a view as this arises: ‘This is mine, this is I, this is my self’”(\textbf{\cite{S}III 181–182}). So they are stated as five because this is the widest limit as a basis for the assumption of self and what pertains to self.

                \vismParagraph{XIV.219}{219}{}
                (c) And also, since those other [sorts of aggregates] stated as the five aggregates of things beginning with virtue\footnote{\vismAssertFootnoteCounter{82}\vismHypertarget{XIV.n82}{}The aggregates of virtue, concentration, understanding, liberation, and knowledge and vision of liberation (\textbf{\cite{S}I 99}), etc.} are comprised within the formations aggregate, they are included here too. Therefore they are stated as five because they include the other sorts.

                This is how the exposition should be known as to neither less nor more.

                \vismParagraph{XIV.220}{220}{}
                \emph{4. As to simile}: the materiality aggregate [as object] of clinging is like a sick-room because it is the dwelling-place, as physical basis, door, and object, of the sick man, namely, the consciousness aggregate as object of clinging. The feeling aggregate as object of clinging is like the sickness because it afflicts. The perception aggregate as object of clinging is like the provocation of the sickness because it gives rise to feeling associated with greed, etc., owing to perception of sense desires, and so on. The formations aggregate as object of clinging is like having recourse to what is unsuitable because it is the source of feeling, which is the sickness; \textcolor{brown}{\textit{[479]}} for it is said: “Feeling as feeling is the formed that they form” (\textbf{\cite{S}III 87}), and likewise: “Because of unprofitable kamma having been performed and stored up, resultant body-consciousness has arisen accompanied by pain” (\textbf{\cite{Dhs}§556}). The consciousness aggregate as object of clinging is like the sick man because it is never free from feeling, which is the sickness.

                \vismParagraph{XIV.221}{221}{}
                Also they are (respectively) like the prison, the punishment, the offence, the punisher, and the offender. And they are like the dish, the food, the curry sauce [poured over the food], the server, and the eater.\footnote{\vismAssertFootnoteCounter{83}\vismHypertarget{XIV.n83}{}“The matter of the body is like the \emph{prison }because it is the site of the punishment. \emph{Perception is like the offence }because owing to perception of beauty, etc., it is a cause of the \emph{punishment, }which is \emph{feeling. }The \emph{formations aggregate }is like the \emph{punisher }because it is a cause of feeling. \emph{Consciousness }is like the \emph{offender }because it is afflicted by feeling. Again, \emph{matter }is like the \emph{dish }because it bears the food. \emph{Perception }is like the \emph{curry sauce }because, owing to perception of beauty, etc., it hides the \emph{food, }which is \emph{feeling. }The \emph{formations aggregate }is like the \emph{server }because it is a cause of \emph{feeling; }and service is included since one who is taking a meal is usually served. \emph{Consciousness }is like the \emph{eater }because it is helped by feeling” (\textbf{\cite{Vism-mhṭ}504}). For \emph{cāraka }(prison) see \hyperlink{XVI.18}{XVI.18}{}.}

                \marginnote{\textcolor{teal}{\footnotesize\{548|490\}}}{}This is how the exposition should be known as to simile.

                \vismParagraph{XIV.222}{222}{}
                \emph{5. Twice as to how to be seen}: the exposition should be known twice as to how to be seen, namely, in brief and in detail.

                \vismParagraph{XIV.223}{223}{}
                In brief [that is, collectively] the five aggregates as objects of clinging should be seen as an enemy with drawn sword (\textbf{\cite{S}IV 174}) in the Snake Simile, as a burden (\textbf{\cite{S}III 25}) according to the Burden Sutta, as a devourer (\textbf{\cite{S}III 87}f) according to the To-be-devoured Discourse, and as impermanent, painful, not-self, formed, and murderous, according to the Yamaka Sutta (\textbf{\cite{S}III 112}f).

                \vismParagraph{XIV.224}{224}{}
                In detail [that is, individually,] matter should be regarded as a lump of froth because it will not stand squeezing, feeling as a bubble on water because it can only be enjoyed for an instant, perception as a mirage because it causes illusion, formations as a plantain trunk because they have no core, and consciousness as a conjuring trick because it deceives (\textbf{\cite{S}III 140–142}).

                In particular, even sublime internal materiality\footnote{\vismAssertFootnoteCounter{84}\vismHypertarget{XIV.n84}{}Ee and Ae both read \emph{visesato ca sūḷāram pi ajjhattikaṃ rūpaṃ. }But \emph{Sammohavinodanī }(Be) in identical passage reads \emph{visesato ca subhārammaṇam pi oḷārikam pi ajjhattika-rūpaṃ.}} should be regarded as foul (ugly); feeling should be regarded as painful because it is never free from the three kinds of suffering (see \hyperlink{XVI.34}{XVI.34}{}); perception and formations as not-self because they are unmanageable; and consciousness as impermanent because it has the nature of rise and fall.

                \vismParagraph{XIV.225}{225}{}
                \emph{6. As to good for one seeing thus}: good comes to be accomplished in one who sees in the two ways thus in brief and in detail. And the way of definition should be known according to that, that is to say, firstly, one who sees the five aggregates as objects of clinging in the form of an enemy with drawn sword, etc., is not worried by the aggregates, but one who sees materiality, etc., in detail as a lump of froth, etc., is not one who sees a core in the coreless.

                \vismParagraph{XIV.226}{226}{}
                And in particular, \textcolor{brown}{\textit{[480]}} one who sees internal materiality as foul (ugly) fully understands nutriment consisting of physical nutriment. He abandons the perversion [of perceiving] beauty in the foul (ugly), he crosses the flood of sense desire, he is loosed from the bond of sense desire, he becomes canker-free as regards the canker of sense desire, he breaks the bodily tie of covetousness. He does not cling with sense-desire clinging.

                \vismParagraph{XIV.227}{227}{}
                One who sees feeling as pain fully understands nutriment consisting of contact. He abandons the perversion of perceiving pleasure in the painful. He crosses the flood of becoming. He is loosed from the bond of becoming. He becomes canker-free as regards the canker of becoming. He breaks the bodily tie of ill will. He does not cling with rules-and-vows clinging.

                \vismParagraph{XIV.228}{228}{}
                One who sees perception and formations as not-self fully understands nutriment consisting of mental volition. He abandons the perversion of \marginnote{\textcolor{teal}{\footnotesize\{549|491\}}}{}perceiving self in the not-self. He crosses the flood of views. He is loosed from the bond of views. He breaks the bodily tie of interpretations (insistence) that “This is the truth.” He does not cling with self-theory clinging.

                \vismParagraph{XIV.229}{229}{}
                One who sees consciousness as impermanent fully understands nutriment consisting of consciousness. He abandons the perversion of perceiving permanence in the impermanent. He crosses the flood of ignorance. He is loosed from the bond of ignorance. He becomes canker-free as regards the canker of ignorance. He breaks the bodily tie of holding to rules and vows. He does not [cling with false-] view clinging.
                \begin{verse}
                    Such blessings there will be\\{}
                    From seeing them as murderers and otherwise,\\{}
                    Therefore the wise should see\\{}
                    The aggregates as murderers and otherwise.
                \end{verse}


                The fourteenth chapter called The Description of the Aggregates in the Treatise on the Development of Understanding in the\emph{ Path of Purification} composed for the purpose of gladdening good people.
        \chapter[The Bases and Elements]{The Bases and Elements\vismHypertarget{XV}\newline{\textnormal{\emph{Āyatana-dhātu-niddesa}}}}
            \label{XV}

            \section[\vismAlignedParas{§1–16}A. Description of the Bases]{A. Description of the Bases}

                \vismParagraph{XV.1}{1}{}
                \marginnote{\textcolor{teal}{\footnotesize\{550|492\}}}{}\textcolor{brown}{\textit{[481]}} The “bases” (\hyperlink{XIV.32}{XIV.32}{}) are the twelve bases, that is to say, the eye base, visible-data base, ear base, sound base, nose base, odour base, tongue base, flavour base, body base, tangible-data base, mind base, mental-data base.

                \vismParagraph{XV.2}{2}{}
                Herein:
                \begin{verse}
                    (1) Meaning, (2) character, (3) just so much,\\{}
                    (4) Order, and (5) in brief and detail,\\{}
                    (6) Likewise as to how to be seen—\\{}
                    Thus should be known the exposition.
                \end{verse}


                \vismParagraph{XV.3}{3}{}
                \emph{1. }Herein, [\emph{as to meaning}] firstly individually:

                It relishes (\emph{cakkhati}), thus it is an eye (\emph{cakkhu}); the meaning is that it enjoys a visible datum and turns it to account.

                It makes visible (\emph{rūpayati}), thus it is a visible datum (\emph{rūpa}); the meaning is that by undergoing an alteration in appearance (colour) it evidences what state is in the mind (lit. heart).

                It hears (\emph{suṇāti}), thus it is an ear (\emph{sota}).

                It is emitted (\emph{sappati}), thus it is sound (\emph{sadda}); the meaning is that it is uttered.

                It smells (\emph{ghāyati}), thus it is a nose (\emph{ghāna}).

                It is smelt (\emph{gandhayati}) thus it is odour (\emph{gandha}); the meaning is that it betrays its own physical basis.

                It evokes (\emph{avhayati}) life (\emph{jīvita}), thus it is a tongue (\emph{jivhā}).

                Living beings taste (\emph{rasanti}) it, thus it is flavour (\emph{rasa}); the meaning is that they enjoy it.

                It is the origin (\emph{āya}) of vile (\emph{kucchita}) states subject to cankers, thus it is a body (\emph{kāya}), origin being the place of arising.

                It is touched (\emph{phusiyati}), thus it is a tangible datum (\emph{phoṭṭhabba}).

                It measures (\emph{munāti}), thus it is a mind (\emph{mano}).

                They cause their own characteristic to be borne (\emph{dhārayanti}), thus they are mental data (\emph{dhammā}).\footnote{\vismAssertFootnoteCounter{1}\vismHypertarget{XV.n1}{}The following words in §3 are not in PED: \emph{cakkhati} (it relishes), \emph{rūpayati} (it makes visible—only referred to under \emph{rūpa}), \emph{sappati} (it is emitted; pass. of \emph{sapati}, to swear (\textbf{\cite{Ud}45})), \emph{udāhariyati} (it is uttered, lit. “is carried up to”), \emph{gandhayati} (it is smelt), \emph{sūcayati }(it betrays), \emph{rasati} (it tastes). Be ed. of \textbf{\cite{Vibh-a}} reads \emph{manayati} (not in PED) for \emph{muṇāti} in parallel passage.

                        \textbf{\cite{Vism-mhṭ}(p. 508)} explains \emph{cakkhati }(relishes) semantically by “tasting a flavour as in ‘relishing’ honey or sauce” and cites \textbf{\cite{M}I 503}. Linguistically it connects the word with \emph{ācikkhati} (to show).

                        “When a visible form (\emph{rūpa}) undergoes, like the visible aspect of a chameleon, an alteration in appearance (colour) at times when [the mind is] dyed with greed or corrupted with hate, etc., it makes visible what state [is prevalent] in the heart (i.e. the mind) and makes that evident as though it were an actual visible object; the meaning is that it demonstrates it by giving it, as it were, a graspable entity (\emph{saviggaha}). Or the word rūpa means demonstration, and that is the same as evidencing. Or the word \emph{rūpa} can be regarded as evidencing of elements too, since it has many meanings. \emph{Rūpayati} (it makes visible) is a derivative (\emph{nibbacana}) of the word \emph{rūpa} that expresses appearance (colour), while \emph{ruppati} (it is molested) is a derivative that expresses the materiality aggregate. [As to sound] only the sound of words (\emph{vacana-sadda}) would be covered by the meaning ‘\emph{is uttered} (\emph{udāhariyati}),’ and here sound is not only the sound of words, but rather all that can be cognized by the ear is what ‘\emph{is emitted} (\emph{sappati})’; the meaning is that by means of its own conditions it is emitted (\emph{sappiyati}), is made cognizable by the ear” (\textbf{\cite{Vism-mhṭ}508}) (cf. also \emph{sappari}, to crawl). “‘It \emph{evokes life} (\emph{jīvitaṃ avhayati})’ owing to appetite for tastes in food (\emph{āhāra}), which is the cause of life (\emph{jīvita}), since the act of swallowing is rooted in approval of tastes. This is the linguistic characteristic of the word \emph{jivhā} (tongue)” (\textbf{\cite{Vism-mhṭ}509}).}

                \vismParagraph{XV.4}{4}{}
                \marginnote{\textcolor{teal}{\footnotesize\{551|493\}}}{}[As to meaning] in general, however, base (\emph{āyatana}) should be understood as such (a) because of its actuating (\emph{āyatana}), (b) because of being the range (\emph{tanana}) of the origins (\emph{āya}), and (c) because of leading on (\emph{nayana}) what is actuated (\emph{āyata}).\footnote{\vismAssertFootnoteCounter{2}\vismHypertarget{XV.n2}{}The following words in §4 are not in PED: \emph{āyatana} (actuating: verbal n. fm. \emph{āyatati}, to actuate); \emph{tanana} (range: verbal n. fm. \emph{tanoti}, to provide a range for, to extend—q.v. PED—; mentioned under\emph{ āyatana}, base); \emph{nayana} (lead in on: verbal n. fm. \emph{neti}, to lead on; lit, meaning not in PED); \emph{āyatati} (to actuate—\emph{cakkhuviññāṇādīnaṃ uppādanaṃ āyatanaṃ}, \textbf{\cite{Vism-mhṭ}}). See also \emph{āyāpenti} \textbf{\cite{Paṭis}II 21}.}

                Now, the various states of consciousness and its concomitants belonging to such and such a door-cum-object among those consisting of the eye-cum-visible-datum, etc., (a) are actuated (\emph{āyatanti}), each by means of its individual function of experiencing, etc.; they are active, strive, and endeavour in these, is what is meant. And (b) these [doors-cum-objects] provide the range for (\emph{tananti}) those states that are origins (\emph{āya}); they give them scope, is what is meant. And (c) as long as this suffering of the round of rebirths, which has gone on occurring throughout the beginningless round of rebirths and so is enormously actuated (\emph{āyata}), does not recede, so long they lead on (\emph{nayanti}); they cause occurrence, is what is meant.

                So all these \textcolor{brown}{\textit{[482]}} things are called “bases” because they actuate, because they are the range of the origins, and because they lead on what is actuated.

                \vismParagraph{XV.5}{5}{}
                \marginnote{\textcolor{teal}{\footnotesize\{552|494\}}}{}Furthermore, “base, (\emph{āyatana}) should be understood in the sense of place of abode, store (mine),\footnote{\vismAssertFootnoteCounter{3}\vismHypertarget{XV.n3}{}\emph{Ākara} means either a mine or a store (PED apparently believes in mining for pearls—see \emph{ratanākara}).} meeting place, locality of birth, and cause. For accordingly in the world in such phrases as the lord’s sphere” (\emph{āyatana}) and “Vāsudeva’s sphere” (\emph{āyatana}), it is a place of abode that is called “base”; and in such phrases as “the sphere of gold” and “the sphere of silver” it is a store (mine) that is called “base.” But in the Dispensation, in such passages as:
                \begin{verse}
                    “And so in the delightful realm (\emph{āyatana})\\{}
                    Those flying in the air attend him” (\textbf{\cite{A}III 43}),
                \end{verse}


                it is a meeting place; and in such phrases as “The southern land is the realm (\emph{āyatana}) of cattle” (?) it is the locality of birth; and in such passages as “He acquires the ability to be a witness of it … whenever there is an occasion (\emph{āyatana}) for it’” (\textbf{\cite{M}I 494}; \textbf{\cite{A}I 258}), it is a cause.

                \vismParagraph{XV.6}{6}{}
                And these various states of consciousness and its concomitants dwell in the eye, etc., because they exist in dependence on them, so the eye, etc., are their \emph{place of abode}. And they frequent the eye, etc., because they have them [respectively] as their [material] support and as their object, so the eye, etc., are their \emph{store}. And the eye, etc., are their \emph{meeting place} because they meet together in one or other of them, [using them] as physical basis, door, and object. And the eye, etc., are the \emph{locality of their birth} because they arise just there, having them as their respective supports and objects. And the eye, etc., are their \emph{reason} because they are absent when the eye, etc., are absent.

                \vismParagraph{XV.7}{7}{}
                So for these reasons too these things are called “bases” in the sense of place of abode, store, meeting place, locality of birth, and reason.

                Consequently, in the sense already stated, it is an eye and that is a base, thus it is the eye base … They are mental data and those are a base, thus they are the mental-data base.

                This is how the exposition should be known here as to meaning.

                \vismParagraph{XV.8}{8}{}
                \emph{2. Character}: Here too the exposition should be known as to the characteristic of the eye and so on. But their characteristics should be understood in the way given above in the Description of the Aggregates (\hyperlink{XIV.37}{XIV.37ff.}{}).

                \vismParagraph{XV.9}{9}{}
                \emph{3. As to just so much}: as just so many.\footnote{\vismAssertFootnoteCounter{4}\vismHypertarget{XV.n4}{}“Because of the absence of anything whatever not included in the twelve bases, there is no arguing that they are more than twelve” (\textbf{\cite{Vism-mhṭ}510}).} What is meant is this: The eye, etc., are mental data too; that being so, why is “twelve bases” said instead of simply “mental-data base?” It is for the sake of defining door-cum-object for the arising of the six consciousness groups. And here they are stated as twelve since this is how they are classed when so defined. \textcolor{brown}{\textit{[483]}}

                \vismParagraph{XV.10}{10}{}
                For only the eye base is the door of arising, and only the visible-data base is the object, of the consciousness group comprised in a cognitive series containing eye-consciousness. Likewise the others for the others. \marginnote{\textcolor{teal}{\footnotesize\{553|495\}}}{}But only one part of the mind base, in other words, the life-continuum mind,\footnote{\vismAssertFootnoteCounter{5}\vismHypertarget{XV.n5}{}“‘\emph{In other words, the life-continuum mind}’: that which occurs twice in disturbance (see \hyperlink{XIV.n46}{Ch. XIV, note 46}{}). Only when there has been the occurrence of the life-continuum in a state of disturbance (in a state of dissimilar occurrence) is there the arising of adverting, not otherwise. Taking it thus as the reason for adverting, what is called ‘life-continuum mind’ is a door of arising. ‘\emph{Not common to all}’ means not common to eye-consciousness and the rest” (Vism-mhṭ 510). See \textbf{\cite{M}I 293}.} is the door of arising, and only the mental-data base not common to all is the object, of the sixth [consciousness group].

                So they are called “the twelve” because they define door-cum-object for the arising of the six consciousness groups. This is how the exposition should be known here as to just so much.

                \vismParagraph{XV.11}{11}{}
                \emph{4. As to order}: here too, from among “order of arising,” etc., mentioned above (\hyperlink{XIV.211}{XIV.211}{}), only “order of teaching” is appropriate. For the eye is taught first among the internal bases since it is obvious because it has as its objective field what is visible with resistance (see last triad, Dhs 2). After that the ear base, etc., which have as their objective fields what is invisible with resistance. Or alternatively, the eye base and ear base are taught first among the internal bases because of their great helpfulness as [respective] causes for the “incomparable of seeing” and the “incomparable of hearing” (see \textbf{\cite{D}III 250}). Next, the three beginning with the nose base. And the mind base is taught last because it has as its resort the objective fields of the [other] five (\textbf{\cite{M}I 295}). But among the external bases the visible-data base, etc., [are taught] each one next [to its corresponding internal base] because they are the respective resorts of the eye base, and so on.

                \vismParagraph{XV.12}{12}{}
                Furthermore, their order may be understood as that in which the reasons for consciousness’s arising are defined; and it is said: “Due to eye and to visible objects eye-consciousness arises, … due to mind and mental objects mind-consciousness arises” (\textbf{\cite{M}I 111}).

                This is how the exposition should be known here as to order.

                \vismParagraph{XV.13}{13}{}
                \emph{5. In brief and in detail}: in brief the twelve bases are simply mentality-materiality because the mind base and one part of the mental-data base are included in mentality, and the rest of the bases in materiality.

                \vismParagraph{XV.14}{14}{}
                But in detail, firstly as regards the internal bases, the eye base is, as to kind, simply eye sensitivity; but when it is classified according to condition, destiny, order [of beings], and person,\footnote{\vismAssertFootnoteCounter{6}\vismHypertarget{XV.n6}{}“‘\emph{Condition}’ is kamma, etc., ‘destiny’ is from hell upwards; ‘order [\emph{of beings}]’ refers to such species as elephants, horses, etc., or to the castes of the khattiyas (warrior nobles), and so on; ‘\emph{person}’ refers to any given living being’s continuity” (\textbf{\cite{Vism-mhṭ}511}).} it is of infinite variety. Likewise the four beginning with the ear base. And the mind base, when classified according to profitable, unprofitable, resultant, and functional consciousness, is of eighty-nine kinds or of one hundred and twenty-one kinds,\footnote{\vismAssertFootnoteCounter{7}\vismHypertarget{XV.n7}{}There are eighty-one mundane sorts of consciousness; and since there is no path or fruition without jhāna, when the four paths and four fruitions are multiplied by the five jhānas, there are forty kinds of supramundane consciousness: 81+40 = 121.} but it is of infinite variety when classified \marginnote{\textcolor{teal}{\footnotesize\{554|496\}}}{}according to physical basis, progress, and so on.\footnote{\vismAssertFootnoteCounter{8}\vismHypertarget{XV.n8}{}“‘\emph{Physical basis}’ is that consisting of the eye, etc.; according to that ‘\emph{Progress}’ is a painful progress, and the other three. ‘\emph{And so on}’ refers to jhāna, predominance, plane, object, and so on” (\textbf{\cite{Vism-mhṭ}512}).} The visible-data, sound, odour, and flavour bases are of infinite variety when classified according to dissimilarity, condition, and so on.\footnote{\vismAssertFootnoteCounter{9}\vismHypertarget{XV.n9}{}“Blue is similar to blue; it is dissimilar to any other colour. ‘\emph{Condition}’ is kamma, and so on” (\textbf{\cite{Vism-mhṭ}512}).} The tangible-data base is of three kinds as consisting of earth element, fire element, and air element; \textcolor{brown}{\textit{[484]}} but when classified according to condition, etc., it is of many kinds. The mental-data base is of many kinds when classified according to the several individual essences of feeling, perception, formations, subtle matter, and Nibbāna (see \textbf{\cite{Vibh}72}).

                This is how the exposition should be known in brief and in detail.

                \vismParagraph{XV.15}{15}{}
                \emph{6. As to how to be seen}: here all formed bases should be regarded as having no provenance and no destination. For they do not come from anywhere prior to their rise, nor do they go anywhere after their fall. On the contrary, before their rise they had no individual essence, and after their fall their individual essences are completely dissolved. And they occur without mastery [being exercisable over them] since they exist in dependence on conditions and in between the past and the future. Hence they should be regarded as having no provenance and no destination.

                Likewise they should be regarded as incurious and uninterested. For it does not occur to the eye and the visible datum, etc., “Ah, that consciousness might arise from our concurrence.” And as door, physical basis, and object, they have no curiosity about, or interest in, arousing consciousness. On the contrary, it is the absolute rule that eye-consciousness, etc., come into being with the union of eye with visible datum, and so on. So they should be regarded as incurious and uninterested.

                \vismParagraph{XV.16}{16}{}
                Furthermore, the internal bases should be regarded as an empty village because they are devoid of lastingness, pleasure, and self; and the external ones as village-raiding robbers (\textbf{\cite{S}IV 175}) because they raid the internal ones. And this is said: “Bhikkhus, the eye is harassed by agreeable and disagreeable visible objects” (\textbf{\cite{S}IV 175}). Furthermore, the internal ones should be regarded as like the six creatures (\textbf{\cite{S}IV 198–199}) and the external ones as like their resorts.

                This is how the exposition should be known here as to how to be seen.

                This, firstly, is the section of the detailed explanation dealing with the bases.
            \section[\vismAlignedParas{§17–43}B. Description of the Elements]{B. Description of the Elements}

                \vismParagraph{XV.17}{17}{}
                The “elements” next to that (\hyperlink{XIV.32}{XIV.32}{}) are the eighteen elements, that is to say, eye element, visible-data element, eye-consciousness element; ear element, sound element, ear-consciousness element; nose element, odour element, nose-consciousness element; tongue element, flavour element, tongue-consciousness element; body element, tangible-data element, body-consciousness element; mind element, mental-data element, mind-consciousness element.

                \vismParagraph{XV.18}{18}{}
                \marginnote{\textcolor{teal}{\footnotesize\{555|497\}}}{}Herein:
                \begin{verse}
                    (1) As to meaning, (2) characteristic, et cetera,\\{}
                    (3) Order, (4) just so much, and (5) reckoning,\\{}
                    (6) Then condition, and (7) how to be seen—\\{}
                    Thus should be known the exposition.
                \end{verse}


                \vismParagraph{XV.19}{19}{}
                \emph{1.} Herein, \emph{as to meaning}: first the exposition of “eye,” etc., should be known individually as to meaning in the way beginning: It relishes (\emph{cakkhati}), thus it is an eye (\emph{cakkhu}); it makes visible (\emph{rūpayati}), thus it is a visible datum; \textcolor{brown}{\textit{[485]}} and the consciousness of the eye is eye-consciousness (see \hyperlink{XV.3}{§3}{}).

                As to meaning in general: (a) it sorts out (\emph{vidahati}), (b) it assorts [well] (\emph{dhīyate}), (c) a sorting out (\emph{vidhāna}), (d) it is sorted out (\emph{vidhīyate}) by means of that, or (e) it causes to be sorted (\emph{dhīyati}) here, thus it is a sort (\emph{dhātu} = element).\footnote{\vismAssertFootnoteCounter{10}\vismHypertarget{XV.n10}{}The verb \emph{dahati}, the basis of all these derivatives, means literally “to put.” “There are five meanings stated, since the word \emph{dhātu} (element, sort, ‘putting’) has its form established (\emph{siddha}) here by (a) the transitive (\emph{kattu}), (b) the intransitive (\emph{kamma}), (c) the abstract noun (\emph{bhāva}), (d) the instrumental case (\emph{kāraṇa}), and (e) the causative voice (\emph{adhikaraṇa}). Supramundane elements do not sort out (\emph{vidahanti}) the suffering of the round of rebirths; on the contrary, they destroy (\emph{vidhaṃsenti}) it. That is why ‘\emph{mundane’ }is specified” (\textbf{\cite{Vism-mhṭ}513}).}

                \vismParagraph{XV.20}{20}{}
                (a) The mundane sorts (elements), when defined according to their instrumentality, \emph{sort out} (\emph{vidahanti}) the suffering of the round of rebirths, which is of many kinds, just as the “sorts” (ores—see \hyperlink{XI.20}{XI.20}{}) of gold and silver, etc., do gold and silver, and so on. (b) They assort [well] (\emph{dhīyante}) with living beings, as a burden does with burden bearers; they are borne (\emph{dhāriyanti}), is the meaning. (c) And they are only mere sortings out (\emph{vidhāna}) of suffering because no mastery is exercisable over them. (d) And by means of them as instruments the suffering of the round of rebirths is continually being sorted out (\emph{anuvidhīyati}) by living beings. (e) And that [suffering], being sorted out (\emph{vihita}) in this way, is caused to be sorted (\emph{dhīyati}) into those [sorts (elements)]; it is caused to be placed in them, is the meaning. So each thing (\emph{dhamma}) among those beginning with the eye is called a “sort” (\emph{dhātu—}element) in the meaning just stated beginning “It sorts out, it assorts well.”

                \vismParagraph{XV.21}{21}{}
                Furthermore, while the self of the sectarians does not exist with an individual essence, not so these. These, on the contrary, are elements (\emph{dhātu}) since they cause [a state’s] own individual essence to be borne (\emph{dhārenti}).\footnote{\vismAssertFootnoteCounter{11}\vismHypertarget{XV.n11}{}“‘\emph{Are elements since they cause [a state’s] own individual essence to be borne}’: here, while the establishment of the word’s form should be understood as “\emph{dadhātī ti dhātu }(it puts, sorts, thus it is an element),’ still taking the word \emph{dhā} to share the meanings [of both \emph{dadhāti} and \emph{dhāreti} (see \hyperlink{XI.104}{XI.104}{})], there is also the meaning of the active voice different from the first, because the meanings of \emph{vidhāna} (sorting out) and \emph{dhāraṇa }(causing to bear) are unconnected. The causing of the bearing of mere individual essences without any permanent living being, is a basic meaning of the word \emph{dhātu }(element), and so it is stated separately” (\textbf{\cite{Vism-mhṭ}513}).} And just as in the world the variously-coloured constituents of marble such as malachite, cinnabar, etc., are called “elements,” so too these [beginning with the \marginnote{\textcolor{teal}{\footnotesize\{556|498\}}}{}eye] are elements like those;\footnote{\vismAssertFootnoteCounter{12}\vismHypertarget{XV.n12}{}“‘\emph{Are elements like those elements}’: here, just as the word “lion” (\emph{sīha}), which is properly applicable to the bearer of a mane, [is used] of a man, so too the word ‘element,’ which is properly applicable to the constituents of marble, is used of the eye and so on” (\textbf{\cite{Vism-mhṭ}513}).} for they are the “variously-coloured” constituents of knowledge and the knowable. Or just as the general term “elements” is used for juices, blood, etc., which are constituents of the collection called the “carcass,” when they are distinguished from each other by dissimilarity of characteristic, so too the general term “elements” should be understood as used for the constituents of the selfhood (personality) called “the pentad of aggregates”; for these things beginning with the eye are distinguished from each other by dissimilarity of characteristic.

                \vismParagraph{XV.22}{22}{}
                Furthermore, “element” is a term for what is soulless; and for the purpose of abolishing the perception of soul the Blessed One accordingly taught the elements in such passages as “Bhikkhu, this man has six elements” (\textbf{\cite{M}III 239}). Therefore the exposition should be understood here firstly as to meaning thus: it is an eye and that is an element, thus it is the eye-element … It is mind-consciousness and that is an element, thus it is mind-consciousness element.

                \vismParagraph{XV.23}{23}{}
                \emph{2. As to characteristic, et cetera}: here too the exposition should be understood as to the characteristic, etc., of the eye, and so on. And that should be understood in the way given above in the Description of the Aggregates (\hyperlink{XIV.37}{XIV.37ff.}{}).

                \vismParagraph{XV.24}{24}{}
                \emph{3. As to order}: here too, from among “order of arising,” etc., mentioned above (\hyperlink{XIV.211}{XIV.211}{}), only “order of teaching” is appropriate. It is set forth according to successive definition of cause and fruit.\footnote{\vismAssertFootnoteCounter{13}\vismHypertarget{XV.n13}{}“‘\emph{Successive definition of cause and fruit}’ is just the state of cause and fruit” (\textbf{\cite{Vism-mhṭ}514}).} For the pair, eye element and visible-data element, are the cause and eye-consciousness element is the fruit. So in each case.

                \vismParagraph{XV.25}{25}{}
                \emph{4. As to just so much}: as just so many. What is meant is this: in various places in the Suttas and Abhidhamma the following as well as other \textcolor{brown}{\textit{[486]}} elements are met with—the illumination element, beauty element, base-consisting-of-boundless-space element, base-consisting-of-boundless-consciousness element, base-consisting-of-nothingness element, base-consisting-of-neither-perception-nor-non-perception element, cessation-of-perception-and-feeling element (\textbf{\cite{S}II 150}); sense-desire element, ill-will element, cruelty element, renunciation element, non-ill-will element, non-cruelty element (Vibh 86); bodily-pleasure element, bodily-pain element, joy element, grief element, equanimity element, ignorance element (Vibh 85); initiative element, launching element, persistence element (\textbf{\cite{S}V 66}); inferior element, medium element, superior element (\textbf{\cite{D}III 215}); earth element, water element, fire element, air element, space element, consciousness element (Vibh 82); formed element, unformed element (\textbf{\cite{M}III 63}); the world of many elements, of various elements (\textbf{\cite{M}I 70})—that being so, why is the classification only made according to these eighteen instead of making it \marginnote{\textcolor{teal}{\footnotesize\{557|499\}}}{}according to all of them? Because, as far as individual essence is concerned, all existing elements are included in that [classification].

                \vismParagraph{XV.26}{26}{}
                The visible data-element itself is the illumination element. The beauty element is bound up with visible-data and so on. Why? Because it is the sign of the beautiful. The sign of the beautiful is the beauty element and that does not exist apart from visible data and so on. Or since the visible data, etc., that are objects consisting of profitable kamma-result are themselves the beauty element, that is thus merely visible data and so on. As regards the base-consisting-of-boundless-space element, etc., the consciousness is mind-consciousness element only, while the remaining [states] are the mental-data element. But the cessation-of-perception-and-feeling element does not exist as an individual essence; for that is merely the cessation of two elements.\footnote{\vismAssertFootnoteCounter{14}\vismHypertarget{XV.n14}{}“It is the mere cessation of the mind-consciousness element and mental-data element because it is the ceasedness of thought-arisings in the fourth immaterial state” (\textbf{\cite{Vism-mhṭ}514}).}

                \vismParagraph{XV.27}{27}{}
                The sense-desire element is either merely the mental-data element, according as it is said, “Herein, what is the sense-desire element? It is the thought, applied thought, … wrong thinking, that is associated with sense desires” (\textbf{\cite{Vibh}86}), or it is the eighteen elements, according as it is said: “Making the Avīci hell the lower limit and making the Paranimmitavasavatti deities the upper limit, the aggregates, elements, bases, materiality, feeling, perception, formations, and consciousness that are in this interval, that belong here, are included here: these are called the sense desire element” (\textbf{\cite{Vibh}86}). \textcolor{brown}{\textit{[487]}}

                \vismParagraph{XV.28}{28}{}
                The renunciation element is the mental-data element; also, because of the passage, “Also all profitable states are the renunciation element” (\textbf{\cite{Vibh}86}), it is the mind-consciousness element too. The elements of ill-will, cruelty, non-ill-will, non-cruelty, bodily pleasure, bodily pain, joy, grief, equanimity, ignorance, initiative, launching, and persistence are the mental-data element too.

                \vismParagraph{XV.29}{29}{}
                The inferior, medium, and superior elements are the eighteen elements themselves; for inferior eyes, etc., are the inferior element, and medium and superior eyes, etc., are the medium and superior elements. But literally speaking, the unprofitable mental-data element and mind-consciousness element are the inferior element; both these elements, when mundane profitable or mundane indeterminate, and the eye element, etc., are the medium element; but the supramundane mental-data element and mind-consciousness element are the superior element.

                \vismParagraph{XV.30}{30}{}
                The earth, fire, and air elements are the tangible-data element; the water element and the space element are the mental-data element only; “consciousness element” is a term summarizing the seven consciousness elements beginning with eye-consciousness.

                \vismParagraph{XV.31}{31}{}
                Seventeen elements and one part of the mental-data element are the formed element; but the unformed element is one part of the mental-data element only. The “world of many elements, of various elements” is merely what is divided up into the eighteen elements.

                \marginnote{\textcolor{teal}{\footnotesize\{558|500\}}}{}So they are given as eighteen because, as to individual essence, all existing elements are included in that [classification].

                \vismParagraph{XV.32}{32}{}
                Furthermore, they are stated as eighteen for the purpose of eliminating the kind of perception to be found in those who perceive a soul in consciousness, the individual essence of which is cognizing; for there are beings who perceive a soul in consciousness, the individual essence of which is cognizing. And so the Blessed One, who was desirous of eliminating the long-inherent perception of a soul, has expounded the eighteen elements thus making evident to them not only consciousness’s multiplicity when classed as eye-, ear-, nose-, tongue-and body-consciousness elements, and mind, and mind-consciousness elements, but also its impermanence, which is due to its existing in dependence on eye-cum-visible-data, etc., as conditions.

                \vismParagraph{XV.33}{33}{}
                What is more, the inclinations of those who are teachable in this way [have to be considered]; and in order to suit the inclinations of beings who are teachable by a teaching that is neither too brief nor too long, eighteen are expounded. For:
                \begin{verse}
                    By methods terse and long as need may be\\{}
                    He taught the Dhamma, so that from beings’ hearts,\\{}
                    If they have wit to learn, the dark departs\\{}
                    Melting in the Good Dhamma’s brilliancy.
                \end{verse}


                This is how the exposition should be understood here as to just so much.

                \vismParagraph{XV.34}{34}{}
                \emph{5. As to reckoning}: the eye-element, firstly, is reckoned as one thing according to kind, \textcolor{brown}{\textit{[488]}} namely, eye sensitivity. Likewise, the ear, nose, tongue, body, visible-data, sound, odour, and flavour elements are reckoned as ear sensitivity, and so on (\hyperlink{XIV.37}{XIV.37ff.}{}). But the tangible-data element is reckoned as three things, namely, earth, fire and air. The eye-consciousness element is reckoned as two things, namely, profitable and unprofitable kamma-result; and likewise the consciousness elements of the ear, nose, tongue, and body. The mind element is reckoned as three things, namely, five-door adverting (70), and profitable (39) and unprofitable (55) resultant receiving. The mental-data element as twenty things, namely, three immaterial aggregates, sixteen kinds of subtle matter, and the unformed element (see \textbf{\cite{Vibh}88}).\footnote{\vismAssertFootnoteCounter{15}\vismHypertarget{XV.n15}{}In \hyperlink{XIV.35}{XIV.35}{}–\hyperlink{XIV.70}{70}{}, the material instances listed total 28, that is, 4 primary elements, 9 sense faculties (excluding the tangible-data faculty, which is the 3 elements except water), and 15 kinds of subtle materiality beginning with the femininity faculty (cf. treatment at Dhs §596). Other lists, however, sometimes give a total of 26 kinds, that is, 10 sense faculties (including the tangible-data faculty, which is the 3 primary elements) and 16 kinds of subtle materiality, that is, the above-mentioned 15 plus the water element, which is listed then after the space element (cf. treatment at Dhs §653 and list at \textbf{\cite{M-a}II 261}). See Table I.}Mind-consciousness element is reckoned as seventy-six things, namely, the remaining profitable, unprofitable, and indeterminate consciousnesses. This is how the exposition should be understood as to reckoning.

                \vismParagraph{XV.35}{35}{}
                \emph{6. Condition}: the eye element, firstly, is a condition, in six ways, namely, dissociation, pre-nascence, presence, non-disappearance, support, and faculty \marginnote{\textcolor{teal}{\footnotesize\{559|501\}}}{}for the eye-consciousness element. The visible-data element is a condition, in four ways, namely, prenascence, presence, non-disappearance, and object, for the eye-consciousness element. Similarly with the ear-element and the sound-element for the ear-consciousness element and so on.

                \vismParagraph{XV.36}{36}{}
                The adverting mind element (70) is a condition, as the five conditions, namely: proximity, contiguity, absence, disappearance, and proximity-decisive-support, for these five [beginning with the eye-consciousness element]. And these five are so too for the receiving mind element ((39), (55)). And so is the receiving mind element for the investigating mind-consciousness element ((40), (41), (56)). And so is that too for the determining mind-consciousness element (71). And so is the determining mind-consciousness element for impulsion mind-consciousness element. But the impulsion mind-consciousness element is a condition, as the six conditions, namely, as the five already stated and as repetition condition, for the immediately following impulsion mind-consciousness element.

                This, firstly, is the way in the case of the five doors.

                \vismParagraph{XV.37}{37}{}
                In the case of the mind door, however, the life-continuum mind-consciousness element is a condition, as the previously-stated five conditions, for the adverting mind-consciousness element (71). And the adverting mind-consciousness element is so for the impulsion mind-consciousness element.

                \vismParagraph{XV.38}{38}{}
                The mental-data element is a condition in many ways, as conascence, mutuality, support, association, presence, non-disappearance, etc.,\footnote{\vismAssertFootnoteCounter{16}\vismHypertarget{XV.n16}{}“Here the word ‘etc.’ stands for the mind-consciousness element’s states where suitable as root-cause, predominance, kamma, kamma-result, nutriment, faculty, jhāna, and path conditions” (\textbf{\cite{Vism-mhṭ}516}).} for the seven consciousness elements. The eye element, etc., and some of the mental-data element,\footnote{\vismAssertFootnoteCounter{17}\vismHypertarget{XV.n17}{}“I.e. subtle materiality and Nibbāna” (\textbf{\cite{Vism-mhṭ}516}).} are conditions, as object condition, etc., for some of the mind-consciousness element.

                \vismParagraph{XV.39}{39}{}
                And not only are the eye and visible data, etc., conditions for the eye-consciousness element, etc., [respectively], but also light, etc., are too. Hence the former teachers said: “Eye-consciousness arises due to eye, visible datum, light, and attention. \textcolor{brown}{\textit{[489]}} Ear-consciousness arises due to ear, sound, aperture, and attention. Nose-consciousness arises due to nose, odour, air, and attention. Tongue-consciousness arises due to tongue, flavour, water, and attention. Body-consciousness arises due to body, tangible datum, earth, and attention. Mind-consciousness arises due to life-continuum-mind,\footnote{\vismAssertFootnoteCounter{18}\vismHypertarget{XV.n18}{}“‘\emph{Life-continuum mind}’ is the life-continuum consciousness occurring twice in disturbance” (\textbf{\cite{Vism-mhṭ}516}).} mental datum, and attention.”

                This is in brief. But the kinds of conditions will be explained in detail in the Description of Dependent Origination (\hyperlink{XVII.66}{XVII.66ff.}{}).

                This is how the exposition should be understood here as to condition.

                \vismParagraph{XV.40}{40}{}
                \emph{7. How to be seen}: the meaning is that here too the exposition should be understood as to how they are to be regarded. For all formed elements are to be \marginnote{\textcolor{teal}{\footnotesize\{560|502\}}}{}regarded as secluded from the past and future,\footnote{\vismAssertFootnoteCounter{19}\vismHypertarget{XV.n19}{}“Formed elements are secluded in both instances (i.e. when past and future) because their individual essences are unapprehendable then” (\textbf{\cite{Vism-mhṭ}516}).} as void of any lastingness, beauty, pleasure, or self, and as existing in dependence on conditions.

                \vismParagraph{XV.41}{41}{}
                Individually, however, the eye element should be regarded as the surface of a drum, the visible-data element as the drumstick, and the eye-consciousness element as the sound. Likewise, the eye element should be regarded as the surface of a looking-glass, the visible-data element as the face, and the eye-consciousness element as the image of the face. Or else, the eye-element should be regarded as sugarcane or sesame, the visible-data element as the [sugarcane] mill or the [sesame] wheel rod, and the eye-consciousness element as the sugarcane juice or the sesame oil. Likewise, the eye-element should be regarded as the lower fire-stick, the visible-data element as the upper fire-stick,\footnote{\vismAssertFootnoteCounter{20}\vismHypertarget{XV.n20}{}\emph{Adharāraṇi }(\emph{adho-araṇi})—“lower fire-stick” and \emph{uttarāraṇi} (\emph{uttara-araṇi})—“upper fire-stick” are not in PED as such.} and the eye-consciousness element as the fire. So too in the case of the ear and so on.

                \vismParagraph{XV.42}{42}{}
                The mind element, however, should be regarded as the forerunner and follower of eye-consciousness, etc., as that arises.

                As to the mental-data element, the feeling aggregate should be regarded as a dart and as a stake, the perception and formations aggregates as a disease owing to their connection with the dart and stake of feeling. Or the ordinary man’s perception should be regarded as an empty fist because it produces pain through [disappointed] desire; or as a forest deer [with a scarecrow] because it apprehends the sign incorrectly. And the formations aggregate should be regarded as men who throw one into a pit of hot coals, because they throw one into rebirth-linking, or as thieves pursued by the king’s men because they are pursued by the pains of birth; or as the seeds of a poison-tree, because they are the root-cause of the aggregates’ continuity, which brings all kinds of harm. And materiality should be regarded as a razor-wheel (see \textbf{\cite{J-a}IV 3}), because it is the sign of various kinds of dangers.

                The unformed element, however, should be regarded as deathless, as peace, as safety. Why? Because it is the opposite of all ill.\textcolor{brown}{\textit{[490]}}

                \vismParagraph{XV.43}{43}{}
                The mind-consciousness element should be regarded as a forest monkey, because it does not stay still on its object; or as a wild horse, because it is difficult to tame; or as a stick flung into the air, because it falls anyhow; or as a stage dancer, because it adopts the guise of the various defilements such as greed and hate.

                The fifteenth chapter called “The Description of the Bases and Elements” in the Treatise on the Development of Understanding in the \emph{Path of Purification} composed for the purpose of gladdening good people.
        \chapter[The Faculties and Truths]{The Faculties and Truths\vismHypertarget{XVI}\newline{\textnormal{\emph{Indriya-sacca-niddesa}}}}
            \label{XVI}

            \section[\vismAlignedParas{§1–12}A. Description of the Faculties]{A. Description of the Faculties}

                \vismParagraph{XVI.1}{1}{}
                \marginnote{\textcolor{teal}{\footnotesize\{561|503\}}}{}\textcolor{brown}{\textit{[491]}} The “faculties” listed next to the elements (\hyperlink{XIV.32}{XIV.32}{}) are the twenty-two faculties, namely, eye faculty, ear faculty, nose faculty, tongue faculty, body faculty, mind faculty, femininity faculty, masculinity faculty, life faculty, [bodily] pleasure faculty, [bodily] pain faculty, [mental] joy faculty, [mental] grief faculty, equanimity faculty, faith faculty, energy faculty, mindfulness faculty, concentration faculty, understanding faculty, “I-shall-come-to-know-the-unknown” faculty, final-knowledge faculty, final-knower faculty.

                \vismParagraph{XVI.2}{2}{}
                Herein:
                \begin{verse}
                    (l) As to meaning, (2) character and so on,\\{}
                    (3) Order, (4) divided and undivided,\\{}
                    (5) Likewise function, and (6) also plane—\\{}
                    The exposition should be known.
                \end{verse}


                \vismParagraph{XVI.3}{3}{}
                \emph{1.} Herein, firstly, the \emph{meaning }of eye, etc., is explained in the way beginning: “It relishes (\emph{cakkhati}), thus it is an eye (\emph{cakkhu})” (\hyperlink{XV.3}{XV.3}{}). But as regards the last three, the first is called the “I-shall-come-to-know-the-unknown” faculty because it arises in the initial stage [of the stream-entry path moment] in one who has entered on the way thus “I shall come to know the deathless state, or the Dhamma of the Four (Noble) Truths, not known,”\footnote{\vismAssertFootnoteCounter{1}\vismHypertarget{XVI.n1}{}“In the noble path moment’s initial stage” (\textbf{\cite{Vism-mhṭ}519}).} and because it carries the meaning of faculty (rulership). The second of them is called the final-knowledge faculty because of knowing finally, and because it carries the meaning of faculty. The third is called the final-knower faculty because it arises in one who has destroyed cankers, who possesses final knowledge, and whose task of getting to know the four truths is finished, and because it carries the meaning of faculty.

                \vismParagraph{XVI.4}{4}{}
                But what is this meaning of faculty (rulership—\emph{indriyattha}) that they have? (a) The meaning of being the mark of a ruler (\emph{inda}) is the meaning of faculty (rulership). (b) The meaning of being taught by a ruler is the meaning of faculty, (c) The meaning of being seen by a ruler is the meaning of faculty, (d) The meaning of having been prepared by a ruler is the meaning of faculty, (e) The \marginnote{\textcolor{teal}{\footnotesize\{562|504\}}}{}meaning of having been fostered by a ruler is the meaning of faculty.\footnote{\vismAssertFootnoteCounter{2}\vismHypertarget{XVI.n2}{}The words \emph{siṭṭha} (prepared—\emph{sajjita, uppādita} \textbf{\cite{Vism-mhṭ}520}), and \emph{juṭṭha }(fostered—\emph{sevita}, \textbf{\cite{Vism-mhṭ}520}) are not in PED. The Pali is: \emph{indaliṅgaṭṭho indriyaṭṭho, indadesitaṭṭho indriyaṭṭho, indadiṭṭhaṭṭho indriyaṭṭho, indasiṭṭhaṭṭho indriyaṭṭho, indajuṭṭhaṭṭho indriyaṭṭho}; cf. \emph{Pāṇini} V 2,93: \emph{Indriyam indraliṅgam indradṛṣṭam indrasṛṣṭam indrajuṣṭam indradattam iti vā}.} And all that applies here in one instance or another.

                \vismParagraph{XVI.5}{5}{}
                The Blessed One, Fully Enlightened, is a ruler (\emph{inda}) because of supreme lordship. And so is kamma, profitable and unprofitable; for no one has lordship over the kinds of kamma. So here, the faculties (\emph{indriya}), \textcolor{brown}{\textit{[492]}} which are created by kamma, \emph{are the mark }of profitable and unprofitable kamma. And since they are \emph{prepared }by it, they are faculties in the sense of (a) \emph{being the mark of a ruler }and (d) in the sense of \emph{having been prepared by a ruler. }But since they have also been correctly made evident and disclosed by the Blessed One, they are all faculties (b) in the sense of \emph{being taught by a ruler }and (c) in the sense of \emph{being seen by a ruler}. And since some of them were cultivated by the Blessed One, Ruler of Sages, in his cultivation of domain and some in his cultivation of development, they are faculties (e) in the sense of \emph{being fostered by a ruler}.

                \vismParagraph{XVI.6}{6}{}
                Furthermore, they are faculties (rulership) in the sense of lordship called predominance. For predominance of the eye, etc., is implied in the occurrence of eye-consciousness, etc., because of the (consciousness’) keenness when that [faculty] is keen and slowness when it is slow.

                This, firstly, is the exposition as to meaning.

                \vismParagraph{XVI.7}{7}{}
                \emph{2. As to character and so on}: the meaning is that the exposition of the eye and so on should be known according to characteristic, function, manifestation, proximate cause, and so on. But these characteristics, etc., of theirs are given above in the Description of the Aggregates (\hyperlink{XIV.37}{XIV.37ff.}{}). For the four beginning with the understanding faculty are simply non-delusion as to meaning. The rest are each given there as such.

                \vismParagraph{XVI.8}{8}{}
                \emph{3. As to order: }this too is only order of teaching (see \hyperlink{XIV.211}{XIV.211}{}). Herein, the noble plane [which is the stage of stream-entry, etc.] is attained through the full-understanding of internal states, and so the eye faculty and the rest included in the selfhood are taught first. Then the femininity faculty and masculinity faculty, to show on what account that selfhood is called “woman” or “man.” Next, the life faculty, to make it known that although that selfhood is twofold, still its existence is bound up with the life faculty. Next the [bodily-] pleasure faculty, etc., to make it known that there is no remission of these feelings as long as that [selfhood] continues, and that all feeling is [ultimately] suffering. Next, the faith faculty, etc., to show the way, since these things are to be developed in order to make that [suffering] cease. Next, the “I-shall-come-to-know-the-unknown” faculty to show that the way is not sterile, since it is through this way that this state is first manifested in oneself. Next, the final-knowledge faculty, because it is the fruit of the last-mentioned faculty and so must be developed after it. Next, the final-knower faculty, the supreme reward, is taught last to make it known that it \marginnote{\textcolor{teal}{\footnotesize\{563|505\}}}{}is attained by development, and that when it is attained there is nothing more to be done. This is the order here.\textbf{ }\textcolor{brown}{\textit{[493]}}

                \vismParagraph{XVI.9}{9}{}
                \emph{4. As to divided and undivided}: here there is only division of the life faculty; for that is twofold as the material-life faculty and the immaterial-life faculty. There is no division of the others.

                This is how the exposition should be known here as to divided and undivided.

                \vismParagraph{XVI.10}{10}{}
                \emph{5. As to function: }what is the faculties’ function? Firstly, because of the words “The eye base is a condition, as faculty condition, for the eye-consciousness element and for the states associated therewith” (\textbf{\cite{Paṭṭh}} 1.5) the eye faculty’s function is to cause by its own keenness, slowness, etc., the occurrence of eye-consciousness and associated states, etc., in a mode parallel to its own,\footnote{\vismAssertFootnoteCounter{3}\vismHypertarget{XVI.n3}{}\emph{Anuvattāpana—}“causing occurrence parallel to”: not in PED; not in CPD.} which is called their keenness, slowness, etc., this function being accomplishable through the state of faculty condition. So too in the case of the ear, nose, tongue, and body. But the function of the mind faculty is to make conascent states subject to its own mastery. That of the life faculty is to maintain conascent states. That of the femininity faculty and the masculinity faculty is to allot the modes of the mark, sign, work and ways of women and men. That of the faculties of pleasure, pain, joy, and grief is to govern conascent states and impart their own particular mode of grossness to those states. That of the equanimity faculty is to impart to them the mode of quiet, superiority and neutrality. That of the faculties of faith, etc., is to overcome opposition and to impart to associated states the mode of confidence and so on. That of the “I-shall-come-to-know-the-unknown” faculty is both to abandon three fetters and to confront associated states with the abandonment of them. That of the final-knowledge faculty is both to attenuate and abandon respectively lust, ill will, etc., and to subject conascent states to its own mastery. That of the final-knower faculty is both to abandon endeavour in all functions and to condition associated states by confronting them with the Deathless.

                This is how the exposition should be known here as to function.

                \vismParagraph{XVI.11}{11}{}
                \emph{6. As to plane}: the faculties of eye, ear, nose, tongue, body, femininity, masculinity, pleasure, pain, and grief are of the sense sphere only. The mind faculty, life faculty, and equanimity faculty, and the faculties of faith, energy, mindfulness, concentration, and understanding are included in the four planes. The joy faculty is included in three planes, namely, sense sphere, fine-material sphere, and supramundane. The last three are supramundane only. This is how the exposition should be known here as to plane.
                \begin{verse}
                    The monk who knows the urgent need\\{}
                    To keep the faculties restrained\\{}
                    By fully understanding them\\{}
                    Will make an end of suffering.
                \end{verse}


                \vismParagraph{XVI.12}{12}{}
                This is the section of the detailed explanation dealing with the faculties.
            \section[\vismAlignedParas{§13–83}B. Description of the Truths]{B. Description of the Truths}

                \vismParagraph{XVI.13}{13}{}
                \marginnote{\textcolor{teal}{\footnotesize\{564|506\}}}{}\textcolor{brown}{\textit{[494]}} The “truths” next to that (\hyperlink{XIV.32}{XIV.32}{}) are the Four Noble Truths; that is to say, the noble truth of suffering, the noble truth of the origin of suffering, the noble truth of the cessation of suffering, the noble truth of the way leading to the cessation of suffering.

                \vismParagraph{XVI.14}{14}{}
                Herein:
                \begin{verse}
                    (1) As to class, and (2) derivation,\\{}
                    (3) Division by character, et cetera,\\{}
                    (4) As to meaning, (5) tracing out meaning,\\{}
                    And likewise (6) neither less nor more,\\{}
                    (7) As to order, (8) as to expounding\\{}
                    Birth and so on, (9) knowledge’s function,\\{}
                    (10) As to division of the content,\\{}
                    (11) As to a simile, and (12) tetrad,\\{}
                    (13) As to void, (14) singlefold and so on,\\{}
                    (15) Similar and dissimilar—\\{}
                    Thus should be known the exposition\\{}
                    By those who know the teaching’s order.
                \end{verse}


                \vismParagraph{XVI.15}{15}{}
                \emph{1.} Herein, \emph{as to class}: the meanings of [the truths of] suffering, etc., are analyzed as four in each case that are “real, not unreal, not otherwise” (\textbf{\cite{S}V 435}) and must be penetrated by those penetrating suffering, etc., according as it is said: “Suffering’s meaning of oppressing, meaning of being formed, meaning of burning, meaning of changing, these are suffering’s four meanings of suffering, which are real, not unreal, not otherwise. Origin’s meaning of accumulating, meaning of source, meaning of bondage, meaning of impeding … Cessation’s meaning of escape, meaning of seclusion, meaning of being unformed, meaning of deathlessness … The path’s meaning of outlet, meaning of cause, meaning of seeing, meaning of predominance, these are the path’s meanings of path, which are real, not unreal, not otherwise” (\textbf{\cite{Paṭis}II 104}; cf. \textbf{\cite{Paṭis}I 19}). Likewise, “Suffering’s meaning of oppressing, meaning of being formed, meaning of burning, meaning of change, are its meaning of penetration to” (cf. \textbf{\cite{Paṭis}I 118}), and so on. So suffering, etc., should be understood according to the four meanings analyzed in each case.

                \vismParagraph{XVI.16}{16}{}
                \emph{2. As to derivation, }3. \emph{division by character, et cetera: }here, however, firstly “as to derivation” [of the word \emph{dukkha }(suffering):] the word \emph{du }(“bad”) is met with in the sense of vile (\emph{kucchita}); for they call a vile child a \emph{du-putta }(“bad child”). The word \emph{kham }(“-ness”), however is met with in the sense of empty (\emph{tuccha}), for they call empty space “\emph{kham}.” And the first truth is vile because it is the haunt of many dangers, and it is empty because it is devoid of the lastingness, beauty, pleasure, and self conceived by rash people. So it is called \emph{dukkhaṃ }(“badness” = suffering, pain), because of vileness and emptiness. \textcolor{brown}{\textit{[495]}}

                \vismParagraph{XVI.17}{17}{}
                [\emph{Samudaya }(origin):] the word \emph{sam }(= prefix “con-”) denotes connection, as in the words \emph{samāgama }(concourse, coming together), \emph{sameta }(congregated, gone together), and so on. The word \emph{u }denotes rising up, as in the words \emph{uppanna }(arisen, uprisen), \emph{udita }(ascended, gone up), and so on. The word \emph{aya}\footnote{\vismAssertFootnoteCounter{4}\vismHypertarget{XVI.n4}{}\emph{Aya—}“reason”: not in PED in this sense.} denotes a \marginnote{\textcolor{teal}{\footnotesize\{565|507\}}}{}reason (\emph{kāraṇa}). And this second truth is the reason for the arising of suffering when combined with the remaining conditions. So it is called \emph{dukkha-samudaya }(the origin of suffering) because it is the reason in combination for the arising of suffering.

                \vismParagraph{XVI.18}{18}{}
                [\emph{Nirodha }(cessation):] the word \emph{ni }denotes absence, and the word \emph{rodha}, a prison.\footnote{\vismAssertFootnoteCounter{5}\vismHypertarget{XVI.n5}{}\emph{Cāraka—}“prison”: not in PED in this sense; see \hyperlink{XIV.221}{XIV.221}{}.} Now, the third truth is void of all destinies [by rebirth] and so there is no constraint (\emph{rodha}) of suffering here reckoned as the prison of the round of rebirths; or when that cessation has been arrived at, there is no more constraint of suffering reckoned as the prison of the round of rebirths. And being the opposite of that prison, it is called \emph{dukkha-nirodha }(cessation of suffering). Or alternatively, it is called “cessation of suffering” because it is a condition for the cessation of suffering consisting in non-arising.

                \vismParagraph{XVI.19}{19}{}
                [\emph{Nirodhagāminī paṭipadā }(way leading to cessation):] because the fourth truth goes (leads) to the cessation of suffering since it confronts that [cessation] as its object, and being the way to attain cessation of suffering, it is called \emph{dukkha-nirodha-gāminī paṭipadā}, the way leading to the cessation of suffering.

                \vismParagraph{XVI.20}{20}{}
                They are called Noble Truths because the Noble Ones, the Buddhas, etc., penetrate them, according as it is said: “Bhikkhus, there are these Four Noble Truths. What four? … These, bhikkhus are the Four Noble Truths” (\textbf{\cite{S}V 425}). The Noble Ones penetrate them, therefore they are called Noble Truths.

                \vismParagraph{XVI.21}{21}{}
                Besides, the Noble Truths are the Noble One’s Truths, according as it is said: “Bhikkhus, in the world with its deities, its Māras and its Brahmās, in this generation with its ascetics and brahmans, with its princes and men, the Perfect One is the Noble One. That is why they are called Noble Truths” (\textbf{\cite{S}V 435}). Or alternatively, they are called Noble Truths because of the nobleness implied by their discovery, according as it is said: “Bhikkhus, it is owing to the correct discovery of these Four Noble Truths that the Perfect One is called accomplished, fully enlightened” (\textbf{\cite{S}V 433}).

                \vismParagraph{XVI.22}{22}{}
                Besides, the Noble Truths are the Truths that are Noble. To be noble is to be not unreal; the meaning is, not deceptive, according as it is said: “Bhikkhus, these Four Noble Truths are real, not unreal, not otherwise that is why they are called Noble Truths” (\textbf{\cite{S}V 435}).

                This is how the exposition should be known here as to derivation.

                \vismParagraph{XVI.23}{23}{}
                \emph{3. How as to division by character, et cetera}? The truth of suffering has the characteristic of afflicting. \textcolor{brown}{\textit{[496]}} Its function is to burn. It is manifested as occurrence (as the course of an existence). The truth of origin has the characteristic of producing. Its function is to prevent interruption. It is manifested as impediment. The truth of cessation has the characteristic of peace. Its function is not to die. It is manifested as the signless.\footnote{\vismAssertFootnoteCounter{6}\vismHypertarget{XVI.n6}{}“‘\emph{Signless}’: being secluded from the sign of the five aggregates, it is taken as having no graspable entity (\emph{aviggaha})” (\textbf{\cite{Vism-mhṭ}525}).} The truth of the path has the \marginnote{\textcolor{teal}{\footnotesize\{566|508\}}}{}characteristic of an outlet. Its function is to abandon defilements. It is manifested as emergence. They have, moreover, the respective characteristics of occurrence, making occur, non-occurrence, and making not occur, and likewise the characteristics of the formed, craving, the unformed, and seeing. This is how the exposition should be understood here as to characteristic, et cetera.

                \vismParagraph{XVI.24}{24}{}
                \emph{4. As to meaning, }5. \emph{tracing out the meaning}: as to “meaning” firstly, what is the “meaning of truth” (\emph{saccattha})? It is that which, for those who examine it with the eye of understanding, is not misleading like an illusion, deceptive like a mirage, or undiscoverable like the self of the sectarians, but is rather the domain of noble knowledge as the real unmisleading actual state with its aspects of affliction, production, quiet, and outlet. It is this real unmisleading actualness that should be understood as the “meaning of truth” just as [heat is] the characteristic of fire, and just as [it is] in the nature of the world [that things are subject to birth, ageing and death], according as it is said, “Bhikkhus, this suffering is real, not unreal, not otherwise” (\textbf{\cite{S}V 430}), and so on, in detail.

                \vismParagraph{XVI.25}{25}{}
                Furthermore:
                \begin{verse}
                    There is no pain but is affliction.\\{}
                    And naught that is not pain afflicts:\\{}
                    This certainty that it afflicts\\{}
                    Is what is reckoned here as truth.
                \end{verse}

                \begin{verse}
                    No other source of pain than craving.\\{}
                    Nor aught that source provides but pain:\\{}
                    This certainty in causing pain\\{}
                    Is why it is considered truth.
                \end{verse}

                \begin{verse}
                    There is no peace except Nibbāna,\\{}
                    Nibbāna cannot but be peace:\\{}
                    This certainty that it is peace\\{}
                    Is what is reckoned here as truth.
                \end{verse}

                \begin{verse}
                    No outlet other than the path.\\{}
                    Nor fails the path to be the outlet:\\{}
                    Its status as the very outlet\\{}
                    Has made it recognized as truth.
                \end{verse}

                \begin{verse}
                    This real infallibility.\\{}
                    Which is their true essential core.\\{}
                    Is what the wise declare to be\\{}
                    Truth’s meaning common to all four.
                \end{verse}


                This is how the exposition should be understood as to meaning.

                \vismParagraph{XVI.26}{26}{}
                \emph{5. How as to tracing out the meaning}? This word “truth” (\emph{sacca}) is met with in various meanings. In such passages as “Let him speak truth and not be angry” (\textbf{\cite{Dhp}224}) it is verbal truth. In such passages as “Ascetics and brahmans base themselves on truth” (?) it is the truth of abstinence [from lying]. In such passages as \textcolor{brown}{\textit{[497]}} “Why do they declare diverse truths, the clever talkers that hold forth?” (\textbf{\cite{Sn}885}) it is truth as views. And in such passages as “Truth is one, there is no second” (\textbf{\cite{Sn}884}) it is, as truth in the ultimate sense, both Nibbāna and the path. \marginnote{\textcolor{teal}{\footnotesize\{567|509\}}}{}In such passages as “Of the four truths how many are profitable?” (Vibh 112; \textbf{\cite{Paṭis}II 108}) it is noble truth. And here too it is proper as noble truth.

                This is how the exposition should be understood as to tracing out the meaning.

                \vismParagraph{XVI.27}{27}{}
                \emph{6. As to neither less nor more: }but why are exactly four noble truths stated, neither less nor more? Because no other exists and because none can be eliminated. For there is none extra to them, nor can any one of them be eliminated, according as it is said: “Bhikkhus, that an ascetic or brahman here should come and say: ‘This is not the truth of suffering, the truth of suffering is another; I shall set aside this truth of suffering and make known another truth of suffering’—that is not possible” (?) and so on, and according as it is said: “Bhikkhus, that any ascetic or brahman should say thus: ‘This is not the first noble truth of suffering that is taught by the ascetic Gotama; rejecting this first noble truth of suffering, I shall make known another first noble truth of suffering’—that is not possible” (\textbf{\cite{S}V 428}) and so on.

                \vismParagraph{XVI.28}{28}{}
                Furthermore, when announcing occurrence, [that is, the process of existence,] the Blessed One announced it with a cause, and he announced non-occurrence as having a means thereto. So they are stated as four at the most as occurrence and non-occurrence and the cause of each. Likewise, they are stated as four since they have to be respectively fully understood, abandoned, realized, and developed; and also since they are the basis for craving, craving, the cessation of craving, and the means to the cessation of craving; and also since they are the reliance [depended upon], the delight in the reliance, removal of the reliance, and the means to the removal of the reliance.

                This is how the exposition should be understood here as to neither less nor more.

                \vismParagraph{XVI.29}{29}{}
                \emph{7. As to order, }this too is only order of teaching (see \hyperlink{XIV.211}{XIV.211}{}). The truth of suffering is given first since it is easy to understand because of its grossness and because it is common to all living beings. The truth of origin is given next to show its cause. Then the truth of cessation, to make it known that with the cessation of the cause there is the cessation of the fruit. The truth of the path comes last to show the means to achieve that. \textcolor{brown}{\textit{[498]}}

                \vismParagraph{XVI.30}{30}{}
                Or alternatively, he announced the truth of suffering first to instill a sense of urgency into living beings caught up in the enjoyment of the pleasure of becoming; and next to that, the truth of origin to make it known that that [suffering] neither comes about of itself as something not made nor is it due to creation by an Overlord, etc. (see \hyperlink{XVI.85}{§85}{}), but that on the contrary it is due to this [cause]; after that, cessation, to instill comfort by showing the escape to those who seek the escape from suffering with a sense of urgency because overwhelmed by suffering with its cause. And after that, the path that leads to cessation, to enable them to attain cessation. This is how the exposition should be understood here as to order.

                \vismParagraph{XVI.31}{31}{}
                \emph{8. As to expounding birth and so on: }the exposition should be understood here in accordance with the expositions of the things beginning with birth given by the Blessed One when describing the Four Noble Truths, that is to say, \marginnote{\textcolor{teal}{\footnotesize\{568|510\}}}{}(i) the twelve things in the description of suffering: “Birth is suffering, ageing is suffering,\footnote{\vismAssertFootnoteCounter{7}\vismHypertarget{XVI.n7}{}“Sickness is not included here (as at \textbf{\cite{D}II 305} for example) because no particular person is meant, and there are persons in whom sickness does not arise at all, like the venerable Bakkula (MN 124); otherwise it may be taken as already included by suffering itself; for in the ultimate sense sickness is bodily pain conditioned by disturbance of elements” (\textbf{\cite{Vism-mhṭ}527}).} death is suffering, sorrow, lamentation, pain, grief, and despair are suffering, association with the unloved is suffering, separation from the loved is suffering, not to get what one wants is suffering, in short, the five aggregates [as objects] of clinging are suffering” (\textbf{\cite{Vibh}99}); and (ii) the threefold craving in the description of origin: “That craving which produces further becoming, is accompanied by delight and greed, delighting in this and that, that is to say, craving for sense desires, craving for becoming, craving for non-becoming” (\textbf{\cite{Vibh}101}); and (iii) Nibbāna, which has one meaning only, in the description of cessation: “That which is the remainderless fading away and cessation of that same craving, giving it up, relinquishing it, letting it go, not relying on it” (\textbf{\cite{Vibh}103}); and (iv) the eight things in the description of the path: “What is the noble truth of the way leading to the cessation of suffering? It is this Noble Eightfold Path, that is to say, right view, right thinking, right speech, right action, right livelihood, right effort, right mindfulness, right concentration” (\textbf{\cite{Vibh}104}).
                \subsection[\vismAlignedParas{§32–60}The Truth of Suffering]{The Truth of Suffering}
                    \subsubsection[\vismAlignedParas{§32–43}(i) Birth]{(i) Birth}

                        \vismParagraph{XVI.32}{32}{}
                        Now, this word birth (\emph{jāti}) has many meanings. For in the passage “[He recollects … ] one birth (\emph{jāti}), two births” (\textbf{\cite{D}I 81}) it is becoming. In the passage, “Visākhā, there is a kind (\emph{jāti}) of ascetics called Nigaṇṭhas (Jains)” (\textbf{\cite{A}I 206}) it is a monastic order. In the passage, “Birth (\emph{jāti}) is included in two aggregates” (Dhātuk 15) it is the characteristic of whatever is formed. In the passage, “His birth is due to the first consciousness arisen, the first cognition manifested, in the mother’s womb” (\textbf{\cite{Vin}I 93}) it is rebirth-linking. \textcolor{brown}{\textit{[499]}} In the passage “As soon as he was born (\emph{sampatijāta}), Ānanda, the Bodhisatta …” (\textbf{\cite{M}III 123}) it is parturition. In the passage “One who is not rejected and despised on account of birth” (\textbf{\cite{A}III 152}) it is clan. In the passage “Sister, since I was born with the noble birth” (\textbf{\cite{M}II 103}) it is the Noble One’s virtue.

                        \vismParagraph{XVI.33}{33}{}
                        Here it should be regarded as the aggregates that occur from the time of rebirth-linking up to the exit from the mother’s womb in the case of the womb-born, and as only the aggregates of rebirth-linking in the case of the rest. But this is only an indirect treatment. In the direct sense, however, it is the first manifestation of any aggregates that are manifested in living beings when they are born anywhere that is called “birth.”

                        \vismParagraph{XVI.34}{34}{}
                        Its characteristic is the first genesis in any [sphere of] becoming. Its function is to consign [to a sphere of becoming]. It is manifested as an emerging here from a past becoming; or it is manifested as the variedness of suffering. \marginnote{\textcolor{teal}{\footnotesize\{569|511\}}}{}But why is it suffering? Because it is the basis for many kinds of suffering.\footnote{\vismAssertFootnoteCounter{8}\vismHypertarget{XVI.n8}{}“The question, “\emph{But why is it suffering}?” means this: granted firstly that birth in hell is painful, since hell is unalloyed pain, and that it is painful in the other unhappy destinies since it is originated by bad kamma; but how is it so in the happy destinies since it is there originated by kamma that leads to bliss? The answer, “\emph{Because it is the basis for many kinds of suffering}”, etc., shows that this birth is not called suffering because of having suffering as its individual essence—for there is no rebirth-linking associated with painful feeling—but rather because it is the foundation for suffering” (\textbf{\cite{Vism-mhṭ}528}).

                                Something must be said here about the words \emph{dukkha} and \emph{sukha,} the former being perhaps the hardest after \emph{dhamma} to render into English. \emph{Dukkha} is consistently rendered by either the vaguer general term “suffering” or by the more specific “[bodily] pain.” Different, but overlapping, ideas are expressed. The latter needs no explanation; but “suffering” must be stretched to include the general insecurity of the whole of experience, of the impermanent world. For this, “uneasiness” would certainly be preferable (“ill” is sometimes used), but multiplication of renderings is to be avoided as much as possible; local accuracy is only too often gained at the cost of general disorientation in a work of this sort, with these very general words capable of sharp focusing. Again, \emph{sukha} has been rendered as either “bliss” or “pleasure,” though the latter does not at all necessarily imply any hedonism construed with sensual pleasure (\emph{kāma}). Again, “ease” (in the sense of relief) is in many ways preferable for the first sense but has not been used for the reason already given.} For there are many kinds of suffering, that is to say, intrinsic suffering (\emph{dukkha-dukkha}), \footnote{\vismAssertFootnoteCounter{9}\vismHypertarget{XVI.n9}{}“Since also what does not have suffering as its individual essence is yet called suffering indirectly, consequently ‘intrinsic suffering’ (\emph{dukkha-dukkha}) is said particularizing what does have suffering as its individual essence, just as in the case of particularizing ‘concrete matter’” (\emph{rūpa-rūpa}) (see 14.77) (Vism-mhṭ 528). For these three kinds see \textbf{\cite{S}IV 259}.} suffering in change (\emph{vipariṇāma-dukkha}), and suffering due to formations (\emph{saṅkhāra-dukkha}); and then concealed suffering, exposed suffering, indirect suffering, and direct suffering.

                        \vismParagraph{XVI.35}{35}{}
                        Herein, bodily and mental, painful feeling are called \emph{intrinsic suffering }because of their individual essence, their name, and their painfulness. [Bodily and mental] pleasant feeling are called \emph{suffering in change }because they are a cause for the arising of pain when they change (\textbf{\cite{M}I 303}). Equanimous feeling and the remaining formations of the three planes are called \emph{suffering due to formations }because they are oppressed by rise and fall. Such bodily and mental affliction as earache, toothache, fever born of lust, fever born of hate, etc., is called \emph{concealed suffering }because it can only be known by questioning and because the infliction is not openly evident; it is also called “unevident suffering.” The affliction produced by the thirty-two tortures,\footnote{\vismAssertFootnoteCounter{10}\vismHypertarget{XVI.n10}{}See MN 13 and 129, though it is not clear where the figure “thirty-two” is taken from.} etc., is called \emph{exposed suffering }because it can be known without questioning and because the infliction is openly evident; it is also called “evident suffering.” Except intrinsic suffering, all given in the exposition of the truth of suffering [in the Vibhaṅga] (\textbf{\cite{Vibh}99}) beginning \marginnote{\textcolor{teal}{\footnotesize\{570|512\}}}{}with birth are also called \emph{indirect suffering }because they are the basis for one kind of suffering or another. But intrinsic suffering is called \emph{direct suffering}.

                        \vismParagraph{XVI.36}{36}{}
                        Herein, this birth is suffering because it is the basis for the suffering in the states of loss as made evident by the Blessed One by means of a simile in the Bālapaṇḍita Sutta (\textbf{\cite{M}III 165f.}), etc., and for the suffering that arises in the happy destinies in the human world and is classed as “rooted in the descent into the womb,” and so on. \textcolor{brown}{\textit{[500]}}

                        \vismParagraph{XVI.37}{37}{}
                        Here the suffering classed as “rooted in the descent into the womb,” and so on, is this: When this being is born in the mother’s womb, he is not born inside a blue or red or white lotus, etc., but on the contrary, like a worm in rotting fish, rotting dough, cesspools, etc., he is born in the belly in a position that is below the receptacle for undigested food (stomach), above the receptacle for digested food (rectum), between the belly-lining and the backbone, which is very cramped, quite dark, pervaded by very fetid draughts redolent of various smells of ordure, and exception-ally loathsome.\footnote{\vismAssertFootnoteCounter{11}\vismHypertarget{XVI.n11}{}\emph{Pavana—}“stench”: not in PED, in this sense. The \emph{Sammohavinodanī} (Be) reproducing this passage inserts the word \emph{asuci }(impurity), lacking in Ee and Ae eds. of Vism. \emph{Kuṇapa }is only given the meaning of “corpse or carcass” in PED; but \textbf{\cite{Vism-mhṭ}} says, “various ordures (\emph{kuṇapa}) such as bile, phlegm, pus, blood, excrement, gorge and so on” (\textbf{\cite{Vism-mhṭ}529}). “Whether the mother is [twenty], [thirty], or [forty] years old, it is ‘\emph{as exceptionally loathsome}’ as an excrement bucket that has not been washed for a like number of years” (\textbf{\cite{Vism-mhṭ}529}).} And on being reborn there, for ten months he undergoes excessive suffering, being cooked like a pudding in a bag by the heat produced in the mother’s womb, and steamed like a dumpling of dough, with no bending, stretching, and so on. So this, firstly, is the suffering rooted in the descent into the womb.

                        \vismParagraph{XVI.38}{38}{}
                        When the mother suddenly stumbles or moves or sits down or gets up or turns round, the extreme suffering he undergoes by being dragged back and forth and jolted up and down, like a kid fallen into the hands of a drunkard, or like a snake’s young fallen into the hands of a snake-charmer; and also the searing pain that he undergoes, as though he had reappeared in the cold hells, when his mother drinks cold water, and as though deluged by a rain of embers when she swallows hot rice gruel, rice, etc., and as though undergoing the torture of the “lye-pickling” (see \textbf{\cite{M}I 87}), when she swallows anything salty or acidic, etc.—this is the suffering rooted in gestation.

                        \vismParagraph{XVI.39}{39}{}
                        When the mother has an abortion, the pain that arises in him through the cutting and rending in the place where the pain arises that is not fit to be seen even by friends and intimates and companions—this is the suffering rooted in abortion.

                        \vismParagraph{XVI.40}{40}{}
                        The pain that arises in him when the mother gives birth, through his being turned upside-down by the kamma-produced winds [forces] and flung into that most fearful passage from the womb, like an infernal chasm, and lugged out through the extremely narrow mouth of the womb, like an elephant through a keyhole, like a denizen of hell being pounded to pulp by colliding rocks—this is the suffering rooted in parturition.

                        \vismParagraph{XVI.41}{41}{}
                        \marginnote{\textcolor{teal}{\footnotesize\{571|513\}}}{}The pain that arises in him after he is born, and his body, which is as delicate as a tender wound, is taken in the hands, bathed, washed, rubbed with cloths, etc., and which pain is like being pricked with needle points and gashed with razor blades, etc.—this is the suffering rooted in venturing outside the mother’s womb. \textcolor{brown}{\textit{[501]}}

                        \vismParagraph{XVI.42}{42}{}
                        The pain that arises afterwards during the course of existence in one who punishes himself, in one who devotes himself to the practice of mortification and austerity according to the vows of the naked ascetics, in one who starves through anger, and in one who hangs himself—this is the suffering rooted in self-violence.

                        \vismParagraph{XVI.43}{43}{}
                        And that arising in one who undergoes flogging, imprisonment, etc., at the hands of others is the suffering rooted in others’ violence.

                        So this birth is the basis for all this suffering. Hence this is said:
                        \begin{verse}
                            Now, were no being born in hell again\\{}
                            The pain unbearable of scorching fires\\{}
                            And all the rest would then no footing gain;\\{}
                            Therefore the Sage pronounced that birth is pain.
                        \end{verse}

                        \begin{verse}
                            Many the sorts of pain that beasts endure\\{}
                            When they are flogged with whips and sticks and goads,\\{}
                            Since birth among them does this pain procure,\\{}
                            Birth there is pain: the consequence is sure.
                        \end{verse}

                        \begin{verse}
                            While ghosts know pain in great variety\\{}
                            Through hunger, thirst, wind, sun and what not too,\\{}
                            None, unless born there, knows this misery;\\{}
                            So birth the Sage declares this pain to be.
                        \end{verse}

                        \begin{verse}
                            In the world-interspace, where demons dwell\\{}
                            In searing cold and inspissated gloom,\\{}
                            Is pain requiring birth there for its spell;\\{}
                            So with the birth the pain ensues as well.
                        \end{verse}

                        \begin{verse}
                            The horrible torment a being feels on coming out,\\{}
                            When he has spent long months shut up inside the mother’s womb—\\{}
                            A hellish tomb of excrement—would never come about\\{}
                            Without rebirth: that birth is pain there is no room for doubt.
                        \end{verse}

                        \begin{verse}
                            But why elaborate? At any time or anywhere\\{}
                            Can there exist a painful state if birth do not precede?\\{}
                            Indeed this Sage so great, when he expounded pain, took care\\{}
                            First to declare rebirth as pain, the condition needed there.
                        \end{verse}


                        This, firstly, is the exposition of birth. \textcolor{brown}{\textit{[502]}}
                    \subsubsection[\vismAlignedParas{§44–45}(ii) Ageing]{(ii) Ageing}

                        \vismParagraph{XVI.44}{44}{}
                        \marginnote{\textcolor{teal}{\footnotesize\{572|514\}}}{}\emph{Ageing is suffering}: ageing is twofold; as a characteristic of whatever is formed, and in the case of a continuity, as the oldness of aggregates included in a single becoming, which oldness is known as “brokenness” and so on (see \textbf{\cite{M}III 249}). The latter is intended here.

                        But this ageing has as its characteristic the maturing (ripening) of aggregates. Its function is to lead on to death. It is manifested as the vanishing of youth. It is suffering because of the suffering due to formations and because it is a basis for suffering.

                        \vismParagraph{XVI.45}{45}{}
                        Ageing is the basis for the bodily and mental suffering that arises owing to many conditions such as leadenness in all the limbs, decline and warping of the faculties, vanishing of youth, undermining of strength, loss of memory and intelligence, contempt on the part of others, and so on.

                        Hence this is said:
                        \begin{verse}
                            With leadenness in every limb,\\{}
                            With every faculty declining,\\{}
                            With vanishing of youthfulness,\\{}
                            With memory and wit grown dim,
                        \end{verse}

                        \begin{verse}
                            With strength now drained by undermining,\\{}
                            With growing unattractiveness\\{}
                            To wife and family and then\\{}
                            With dotage coming on, what pain
                        \end{verse}

                        \begin{verse}
                            Alike of body and of mind\\{}
                            A mortal must expect to find!\\{}
                            Since ageing all of this will bring,\\{}
                            Ageing is well named suffering.
                        \end{verse}


                        This is the exposition of ageing.
                    \subsubsection[\vismAlignedParas{§46–47}(iii) Death]{(iii) Death}

                        \vismParagraph{XVI.46}{46}{}
                        \emph{Death is suffering}: death too is twofold, as a characteristic of the formed, with reference to which it is said, “Ageing and death are included in the aggregates” (\textbf{\cite{Dhātuk}15}), and as the severing of the connection of the life faculty included in a single becoming, with reference to which it is said, “So mortals are in constant fear … that they will die” (\textbf{\cite{Sn}576}). The latter is intended here. Death with birth as its condition, death by violence, death by natural causes, death from exhaustion of the life span, death from exhaustion of merit, are names for it.

                        \vismParagraph{XVI.47}{47}{}
                        It has the characteristic of a fall. Its function is to disjoin. It is manifested as absence from the destiny [in which there was the rebirth]. It should be understood as suffering because it is a basis for suffering.

                        Hence this is said:
                        \begin{verse}
                            Without distinction as they die\\{}
                            Pain grips their minds impartially\\{}
                            When wicked men their foul deeds see
                        \end{verse}

                        \begin{verse}
                            \marginnote{\textcolor{teal}{\footnotesize\{573|515\}}}{}Or sign of new rebirth, may be.\\{}
                            Also when good men cannot bear\\{}
                            To part from all that they hold dear.\\{}
                            Then bodily pain severs sinews.\\{}
                            Joints and so on, and continues\textcolor{brown}{\textit{[503]}}\\{}
                            Torture unbearable, which racks\\{}
                            All those whose vitals death attacks\\{}
                            With grip that shall no more relax.\\{}
                            Death is the basis of such pain.\\{}
                            And this suffices to explain\\{}
                            Why death the name of pain should gain.
                        \end{verse}


                        This is the exposition of death.
                    \subsubsection[\vismAlignedParas{§48}(iv) Sorrow]{(iv) Sorrow}

                        \vismParagraph{XVI.48}{48}{}
                        As regards \emph{sorrow}, etc., sorrow is a burning in the mind in one affected by loss of relatives, and so on. Although in meaning it is the same as grief, nevertheless it has inner consuming as its characteristic, its function is completely to consume the mind. It is manifested as continual sorrowing. It is suffering because it is intrinsic suffering and because it is a basis for suffering. Hence this is said:
                        \begin{verse}
                            Sorrow is a poisoned dart\\{}
                            That penetrates a being’s heart;\\{}
                            Setting up a burning there\\{}
                            Like burning with a red-hot spear.
                        \end{verse}

                        \begin{verse}
                            This state of mind brings future pain (see \hyperlink{XVII.273}{XVII.273f.}{})\\{}
                            Such as disease, and then again\\{}
                            Ageing and death, so one may tell\\{}
                            Where for it is called pain as well.
                        \end{verse}


                        This is the exposition of sorrow.
                    \subsubsection[\vismAlignedParas{§49}(v) Lamentation]{(v) Lamentation}

                        \vismParagraph{XVI.49}{49}{}
                        \emph{Lamentation }is verbal clamour on the part of one affected by loss of relatives and so on. It has crying out as its characteristic. Its function is proclaiming virtues and vices. It is manifested as tumult. It is suffering because it is a state of suffering due to formations and because it is a basis for suffering. Hence this is said:
                        \begin{verse}
                            Now, when a man is struck by sorrows dart and he laments\\{}
                            The pain he is already undergoing he augments\\{}
                            With pain born of dry throat and lips and palate, hard to bear.\\{}
                            And so lamenting too is pain, the Buddha did declare.
                        \end{verse}


                        This is the exposition of lamentation.
                    \subsubsection[\vismAlignedParas{§50}(vi) Pain]{(vi) Pain}

                        \vismParagraph{XVI.50}{50}{}
                        \marginnote{\textcolor{teal}{\footnotesize\{574|516\}}}{}\emph{Pain }is bodily pain. Its characteristic is the oppression of the body. Its function is to cause grief in the foolish. It is manifested as bodily affliction. It is suffering because it is intrinsic suffering, and because it brings mental suffering. Hence this is said:
                        \begin{verse}
                            Pain distresses bodily.\\{}
                            Thereby distressing mentally again;\\{}
                            So acting fundamentally.\\{}
                            It therefore is especially called pain.
                        \end{verse}


                        This is the exposition of pain.\textcolor{brown}{\textit{[504]}}
                    \subsubsection[\vismAlignedParas{§51}(vii) Grief]{(vii) Grief}

                        \vismParagraph{XVI.51}{51}{}
                        \emph{Grief }is mental pain. Its characteristic is mental oppression. Its function is to distress the mind. It is manifested as mental affliction. It is suffering because it is intrinsic suffering, and because it brings bodily suffering. For those who are gripped by mental pain tear their hair, weep, thump their breasts, and twist and writhe; they throw themselves upside-down,\footnote{\vismAssertFootnoteCounter{12}\vismHypertarget{XVI.n12}{}Ee and Ae read \emph{uddhapādaṃ }(or \emph{uddhaṃ pādaṃ}) \emph{papatanti, }but \textbf{\cite{Vibh-a}} (Be) reads \emph{chinnapapātaṃ papatanti. }The former reading is favoured by \textbf{\cite{Vism-mhṭ}}.} use the knife, swallow poison, hang themselves with ropes, walk into fires, and undergo many kinds of suffering. Hence this is said:
                        \begin{verse}
                            Though grief itself distresses mind.\\{}
                            It makes distress of bodily kind occur.\\{}
                            And that is why this mental grief\\{}
                            Is pain, as those that have no grief aver.
                        \end{verse}


                        This is the exposition of grief.
                    \subsubsection[\vismAlignedParas{§52–53}(viii) Despair]{(viii) Despair}

                        \vismParagraph{XVI.52}{52}{}
                        \emph{Despair }is the same as the humour produced by excessive mental suffering in one affected by loss of relatives, and so on. Some say that it is one of the states included in the formations aggregate. Its characteristic is burning of the mind. Its function is to bemoan. It is manifested as dejection. It is suffering because it is suffering due to formations, because of the burning of the mind, and because of bodily dejection. Hence this is said:
                        \begin{verse}
                            So great the pain despair imparts\\{}
                            It burns the heart as with fever’s flame;\\{}
                            The body’s function it impairs\\{}
                            And so despair borrows from pain its name.
                        \end{verse}


                        This is the exposition of despair.

                        \vismParagraph{XVI.53}{53}{}
                        Sorrow is like the cooking [of oil, etc.]\footnote{\vismAssertFootnoteCounter{13}\vismHypertarget{XVI.n13}{}\textbf{\cite{Vibh-a}} (Be) adds \emph{telādīnaṃ; }not in Ee and Ae texts.} in a pot over a slow fire. Lamentation is like its boiling over from the pot when cooking over a quick fire. Despair is like \marginnote{\textcolor{teal}{\footnotesize\{575|517\}}}{}what remains in the pot after it has boiled over and is unable to do so any more, going on cooking in the pot till it dries up.
                    \subsubsection[\vismAlignedParas{§54}(ix) Association with the Unloved]{(ix) Association with the Unloved}

                        \vismParagraph{XVI.54}{54}{}
                        \emph{Association with the unloved }is meeting with disagreeable beings and formations (inanimate things). Its characteristic is association with the undesirable. Its function is to distress the mind. It is manifested as a harmful state. It is suffering because it is a basis for suffering. Hence this is said:
                        \begin{verse}
                            The mere sight of an unloved thing\\{}
                            Brings firstly mental suffering.\\{}
                            And suffering of body too\\{}
                            Through touching it can then ensue.
                        \end{verse}

                        \begin{verse}
                            And we therefore may recognize.\\{}
                            Since meeting the unloved gives rise\\{}
                            To either kind of pain, that\\{}
                            He decided pain its name should be.
                        \end{verse}


                        This is the exposition of association with the unloved. \textcolor{brown}{\textit{[505]}}
                    \subsubsection[\vismAlignedParas{§55}(x) Separation from the Loved]{(x) Separation from the Loved}

                        \vismParagraph{XVI.55}{55}{}
                        \emph{Separation from the loved }is to be parted from agreeable beings and formations (inanimate things). Its characteristic is dissociation from desirable objects. Its function is to arouse sorrow. It is manifested as loss. It is suffering because it is a basis for the suffering of sorrow. Hence this is said:
                        \begin{verse}
                            The dart of sorrow wounds the heart\\{}
                            Of fools who from their wealth must part or kin.\\{}
                            Which roughly should be grounds enough\\{}
                            For counting the loved lost as suffering.
                        \end{verse}


                        This is the exposition of separation from the loved.
                    \subsubsection[\vismAlignedParas{§56}(xi) Not to Get What One Wants]{(xi) Not to Get What One Wants}

                        \vismParagraph{XVI.56}{56}{}
                        \emph{Not to get what one wants}: the want itself of some unobtainable object [expressed] in such passages as “Oh, that we were not subject to birth!” (\textbf{\cite{Vibh}101}) is called suffering since one does not get what is wanted. Its characteristic is the wanting of an unobtainable object. Its function is to seek that. It is manifested as disappointment. It is suffering because it is a basis for suffering. Hence this is said:
                        \begin{verse}
                            When beings here expect to gain\\{}
                            Something they build their hopes upon\\{}
                            Which fails them, they are woebegone\\{}
                            With disappointment’s numbing pain.
                        \end{verse}

                        \begin{verse}
                            Thereof the cause is hope they wed\\{}
                            To something they cannot obtain:\\{}
                            “Not to get what one wants is pain”\\{}
                            The Conqueror has therefore said.
                        \end{verse}


                        This is the exposition of not to get what one wants.
                    \subsubsection[\vismAlignedParas{§57–60}(xii) The Five Aggregates]{(xii) The Five Aggregates}

                        \vismParagraph{XVI.57}{57}{}
                        \marginnote{\textcolor{teal}{\footnotesize\{576|518\}}}{}In short \emph{the five aggregates} [as objects] of clinging:
                        \begin{verse}
                            Now, birth and ageing and each thing\\{}
                            Told in describing suffering,\\{}
                            And those not mentioned, could not be\\{}
                            Were there no aggregates for clinging.
                        \end{verse}

                        \begin{verse}
                            Wherefore these aggregates for clinging\\{}
                            Are taken in totality\\{}
                            As pain by Him, the Dhamma’s King,\\{}
                            Who taught the end of suffering.
                        \end{verse}


                        \vismParagraph{XVI.58}{58}{}
                        For birth, etc., thus oppress the pentad of aggregates [as objects] of clinging as fire does fuel, as shooting does a target, as gadflies, flies, etc., do a cow’s body, as reapers do a field, as village raiders do a Village; and they are generated in the aggregates as weeds, creepers, etc., are on the ground, as flowers, fruits and sprouts are on trees.

                        \vismParagraph{XVI.59}{59}{}
                        And the aggregates [as objects] of clinging have \emph{birth }as their initial suffering, \emph{ageing }as their medial suffering, and \emph{death }as their final suffering. The suffering due to burning in one who is the victim of the pain that threatens death is \emph{sorrow}. The suffering consisting in crying out by one who is unable to bear that is \emph{lamentation. }Next, the suffering consisting in affliction of the body due to the contact of undesirable tangible data, in other words, disturbance of the elements, is \emph{pain}. \textcolor{brown}{\textit{[506]}} The suffering oppressing the mind through resistance to that in ordinary people oppressed by it, is \emph{grief. }The suffering consisting in brooding\footnote{\vismAssertFootnoteCounter{14}\vismHypertarget{XVI.n14}{}\emph{Anutthunana—}“brooding”: not in PED = \emph{anto nijjhāyana }(\textbf{\cite{Vism-mhṭ}532}).} in those dejected by the augmentation of sorrow, etc., is \emph{despair. }The suffering consisting in frustration of wants in those whose hopes are disappointed is \emph{not to get what one wants}. So when their various aspects are examined, the aggregates [as objects] of clinging are themselves suffering.

                        \vismParagraph{XVI.60}{60}{}
                        It is impossible to tell it [all] without remainder, showing each kind of suffering, even [by going on doing so] for many eons, so the Blessed One said, “In short the five aggregates [as objects] of clinging are suffering” in order to show in short how all that suffering is present in any of the five aggregates [as objects] of clinging in the same way that the taste of the water in the whole ocean is to be found in a single drop of its water.

                        This is the exposition of the aggregates [as objects] of clinging. This, firstly, is the method for the description of suffering.
                \subsection[\vismAlignedParas{§61}The Truth of the Origin of Suffering]{The Truth of the Origin of Suffering}

                    \vismParagraph{XVI.61}{61}{}
                    But in the description of the \emph{origin}, the expression \emph{yāyaṃ taṇhā }(that craving which) = \emph{yā ayaṃ taṇhā}. [As regards the expression] \emph{produces further becoming}: it is a making become again, thus it is “becoming again” (\emph{punabbhava}); becoming again is its habit, thus it “produces further becoming” (\emph{ponobbhavika}). The expression \emph{nandirāgasahagatā }(accompanied by concern and greed) = \emph{nandirāgena }\marginnote{\textcolor{teal}{\footnotesize\{577|519\}}}{}\emph{sahagatā}; what is meant is that it is identical in meaning with delight and greed. \emph{Concerned with this and that}: wherever personality is generated there is concern with that. The expression \emph{that is to say }(\emph{seyyathidaṃ}) is a particle; its meaning is “which is that.” \emph{Craving for sense desires, craving for becoming, craving for non-becoming }will be explained in the Description of Dependent Origination (\hyperlink{XVII.233}{XVII.233ff.}{}). Although this is threefold, it should nevertheless be understood as “the noble truth of the origin of suffering,” taking it as one in the sense of its generating the truth of suffering.
                \subsection[\vismAlignedParas{§62–83}The Truth of the Cessation of Suffering]{The Truth of the Cessation of Suffering}

                    \vismParagraph{XVI.62}{62}{}
                    In the description of the \emph{cessation }of suffering it is the cessation of the origin that is stated by the words \emph{that which is … of that same craving}, and so on. Why is that? Because the cessation of suffering comes about with the cessation of its origin. For it is with the cessation of its origin that suffering ceases, not otherwise. Hence it is said: \textcolor{brown}{\textit{[507]}}
                    \begin{verse}
                        “Just as a tree cut down grows up again\\{}
                        While yet its root remains unharmed and sound,\\{}
                        So with the tendency to crave intact\\{}
                        This suffering is ever reproduced” (\textbf{\cite{Dhp}338}).
                    \end{verse}


                    \vismParagraph{XVI.63}{63}{}
                    So it is because suffering ceases only through the cessation of its origin that, when teaching the cessation of suffering, the Blessed One therefore taught the cessation of the origin. For the Perfect Ones behave like lions.\footnote{\vismAssertFootnoteCounter{15}\vismHypertarget{XVI.n15}{}“Just as a lion directs his strength against the man who shot the arrow at him, not against the arrow, so the Buddhas deal with the cause, not with the fruit. But just as dogs, when struck with a clod, snarl and bite the clod and do not attack the striker, so the sectarians who want to make suffering cease devote themselves to mutilating the body, not to causing cessation of defilements” (\textbf{\cite{Vism-mhṭ}533}).} When they make suffering cease and when they teach the cessation of suffering, they deal with the cause, not the fruit. But the sectarians behave like dogs. When they make suffering cease and when they teach the cessation of suffering, by teaching devotion to self-mortification, etc., they deal with the fruit, not the cause. This, in the first place, is how the motive for teaching the cessation of suffering by means of the cessation of its origin should be understood.

                    \vismParagraph{XVI.64}{64}{}
                    This is the meaning\emph{. Of that same craving: }of that craving which, it was said, “produces further becoming,” and which was classed as “craving for sense desires” and so on. It is the path that is called \emph{fading away}; for “With the fading away [of greed] he is liberated” (\textbf{\cite{M}I 139}) is said. \emph{Fading away and cessation }is cessation through fading away. \emph{Remainderless fading away and cessation }is cessation through fading away that is remainderless because of eradication of inherent tendencies. Or alternatively, it is abandoning that is called \emph{fading away}; and so the construction here can be regarded as “remainderless fading away, remainderless cessation.”

                    \vismParagraph{XVI.65}{65}{}
                    But as to meaning, all of them are synonyms for Nibbāna. For in the ultimate sense it is Nibbāna that is called “the noble truth of the cessation of suffering.” \marginnote{\textcolor{teal}{\footnotesize\{578|520\}}}{}But because craving fades away and ceases on coming to that,\footnote{\vismAssertFootnoteCounter{16}\vismHypertarget{XVI.n16}{}“‘\emph{On coming to that }(\emph{taṃ āgamma})’: on reaching that Nibbāna by making it the object” (\textbf{\cite{Vism-mhṭ}533}). \emph{Āgamma }(ger. of \emph{āgacchati—}to come) is commonly used as an adverb in the sense of “owing to” (e.g. at \textbf{\cite{M}I 119}). Here, however, it is taken literally by the Commentaries and forms an essential part of the ontological proof of the positive existence of Nibbāna. The \emph{Sammohavinodanī} (commentary on the Āyatana-Vibhaṅga Abhidhamma-bhājaniya) refutes the suggestion of a disputant (\emph{vitaṇḍavādin}) who asserts that Nibbāna is “mere destruction” (\emph{khayamatta}). The arguments used are merely supplementary to those in §69 here, and so are not quoted. The conclusion of the argument is worth noting, however, because of the emphasis on the words “\emph{taṃ āgamma}.” It is this: “It is on coming to Nibbāna that greed, etc., are destroyed. It is the same Nibbāna that is called ‘destruction of greed, destruction of hate, destruction of delusion.’ These are just three terms for Nibbāna—When this was said, he asked: You say ‘On coming to’ (\emph{āgamma}); from where have you got this ‘on coming to’?—It is got from the Suttas—Quote the sutta—‘Thus ignorance and craving, on coming to that, are destroyed in that, are abolished in that, nor does anything anywhere … (\emph{evaṃ avijjā ca taṇhā ca taṃ āgamma tamhi khīṇaṃ tamhi bhaggaṃ na ca kiñci kadāci} … ).’ When this was said, the other was silent.” The quotation has not been traced.} it is therefore called “fading away” and “cessation.” And because there comes to be the giving up, etc., of that [craving] on coming to that [Nibbāna], and since there is not even one kind of reliance here [to be depended upon] from among the reliances consisting in the cords of sense desires, etc., it is therefore called \emph{giving it up, relinquishing it, letting it go, not relying on it.}

                    \vismParagraph{XVI.66}{66}{}
                    It has peace as its characteristic. Its function is not to die; or its function is to comfort. It is manifested as the signless; or it is manifested as non-diversification.\footnote{\vismAssertFootnoteCounter{17}\vismHypertarget{XVI.n17}{}\emph{Nippapañca }(non-diversification) is one of the synonyms for Nibbāna. The word \emph{papañca }is commonly used in the Commentaries in the sense (a) of an impediment or obstacle (\textbf{\cite{Dhp-a}I 18}), and (b) as a delay, or diffuseness (\hyperlink{XVII.73}{XVII.73}{}). The sense in which the word is used in the Suttas is that of diversifying and is best exemplified at \textbf{\cite{M}I 111}: “Friends, due to eye and to a visible object eye-consciousness arises. The coincidence of the three is contact. With contact as condition there is feeling. What a man feels that he perceives. What he perceives he thinks about. What he thinks about he diversifies (\emph{papañceti}). Owing to his having diversified, the evaluation of diversifying perceptions besets a man with respect to past, future, and present visible objects,” and so on. This kind of \emph{papañca} is explained by the Commentaries as “due to craving, pride and views” (\textbf{\cite{M-a}I 25}; II 10; II 75, etc.), and it may be taken as the diversifying action, the choosing and rejecting, the approval and disapproval (\textbf{\cite{M}I 65}), exercised by craving, etc., on the bare material supplied by perception and thought. Consequently, though it is bound up with craving, etc., a false emphasis is given in rendering \emph{papañca} in these contexts by “obsession” as is done in PED. \emph{Nippapañca} as a term for Nibbāna emphasizes the absence of that.}
                    \subsubsection[\vismAlignedParas{§67–74}Discussion on Nibbāna]{Discussion on Nibbāna}

                        \vismParagraph{XVI.67}{67}{}
                        [Question 1] Is Nibbāna non-existent because it is unapprehendable, like the hare’s horn? \marginnote{\textcolor{teal}{\footnotesize\{579|521\}}}{}[Answer] That is not so, because it is apprehendable by the [right] means. For it is apprehendable [by some, namely, the nobles ones] by the [right] means, in other words, by the way that is appropriate to it, [the way of virtue, concentration, and understanding]; it is like the supramundane consciousness of others, [which is apprehendable only by certain of the Noble Ones] by means of knowledge of penetration of others’ minds. Therefore it should not be said that it is non-existent because unapprehendable; for it should not be said that what the foolish ordinary man does not apprehend is unapprehendable.

                        \vismParagraph{XVI.68}{68}{}
                        Again, it should not be said that Nibbāna does not exist. Why not? Because it then follows that the way would be futile. \textcolor{brown}{\textit{[508]}} For if Nibbāna were non-existent, then it would follow that the right way, which includes the three aggregates beginning with virtue and is headed by right understanding, would be futile. And it is not futile because it does reach Nibbāna.

                        [Q. 2] But futility of the way does not follow because what is reached is absence, [that is, absence of the five aggregates consequent upon the cutting off of the defilements].

                        [A.] That is not so. Because, though there is absence of past and future [aggregates], there is nevertheless no reaching of Nibbāna [simply because of that].

                        [Q. 3] Then is the absence of present [aggregates] as well Nibbāna?

                        [A.] That is not so. Because their absence is an impossibility, since if they are absent their non-presence follows. [Besides, if Nibbāna were absence of present aggregates too,] that would entail the fault of excluding the arising of the Nibbāna element with result of past clinging left, at the path moment, which has present aggregates as its support.

                        [Q. 4] Then will there be no fault if it is non-presence of defilements [that is Nibbāna]?

                        [A.] That is not so. Because it would then follow that the noble path was meaningless. For if it were so, then, since defilements [can be] non-existent also before the moment of the noble path, it follows that the noble path would be meaningless. Consequently that is no reason; [it is unreasonable to say that Nibbāna is unapprehendable, that it is non-existence, and so on].

                        \vismParagraph{XVI.69}{69}{}
                        [Q. 5] But is not Nibbāna destruction, because of the passage beginning, “That, friend, which is the destruction of greed … [of hate … of delusion … is Nibbāna]?” (\textbf{\cite{S}IV 251}).

                        [A.] That is not so, because it would follow that Arahantship also was mere destruction. For that too is described in the [same] way beginning, “That, friend, which is the destruction of greed … of hate … of delusion … is Arahantship]” (\textbf{\cite{S}IV 252}).

                        And what is more, the fallacy then follows that Nibbāna would be temporary, etc.; for if it were so, it would follow that Nibbāna would be temporary, have the characteristic of being formed, and be obtainable regardless of right effort; and precisely because of its having formed characteristics it would be included in \marginnote{\textcolor{teal}{\footnotesize\{580|522\}}}{}the formed, and it would be burning with the fires of greed, etc., and because of its burning it would follow that it was suffering.

                        [Q. 6] Is there no fallacy if Nibbāna is that kind of destruction subsequent to which there is no more occurrence?

                        [A.] That is not so. Because there is no such kind of destruction. And even if there were, the aforesaid fallacies would not be avoided.

                        Also because it would follow that the noble path was Nibbāna. For the noble path causes the destruction of defects, and that is why it is called “destruction”; and subsequent to that there is no more occurrence of the defects.

                        \vismParagraph{XVI.70}{70}{}
                        But it is because the kind of destruction called “cessation consisting in non-arising,” [that is, Nibbāna,] serves figuratively speaking as decisive-support [for the path] that [Nibbāna] is called “destruction” as a metaphor for it.

                        [Q. 7] Why is it not stated in its own form?

                        [A.] Because of its extreme subtlety. And its extreme subtlety is established because it inclined the Blessed One to inaction, [that is, to not teaching the Dhamma (see \textbf{\cite{M}I 186})] and because a Noble One’s eye is needed to see it (see \textbf{\cite{M}I 510}).

                        \vismParagraph{XVI.71}{71}{}
                        It is not shared by all because it can only be reached by one who is possessed of the path. And it is uncreated because it has no first beginning.

                        [Q. 8] Since it is, when the path is, then it is not uncreated.

                        [A.] That is not so, because it is not arousable by the path; it is only reachable, not arousable, by the path; that is why it is uncreated. It is because it is uncreated that it is free from ageing and death. It is because of the absence of its creation and of its ageing and death that it is permanent. \textcolor{brown}{\textit{[509]}}

                        \vismParagraph{XVI.72}{72}{}
                        [Q. 9] Then it follows that Nibbāna, too, has the kind of permanence [claimed] of the atom and so on.

                        [A.] That is not so. Because of the absence of any cause [that brings about its arising].

                        [Q. 10] Because Nibbāna has permanence, then, these [that is, the atom, etc.] are permanent as well.

                        [A.] That is not so. Because [in that proposition] the characteristic of [logical] cause does not arise. [In other words, to say that Nibbāna is permanent is not to assert a reason why the atom, etc., should be permanent]

                        [Q. 11] Then they are permanent because of the absence of their arising, as Nibbāna is.

                        [A.] That is not so. Because the atom and so on have not been established as facts.

                        \vismParagraph{XVI.73}{73}{}
                        The aforesaid logical reasoning proves that only this [that is, Nibbāna] is permanent [precisely because it is uncreated]; and it is immaterial because it transcends the individual essence of matter.

                        The Buddhas’ goal is one and has no plurality. But this [single goal, Nibbāna,] is firstly called \emph{with result of past clinging left }since it is made known together with the [aggregates resulting from past] clinging still remaining [during the Arahant’s life], being thus made known in terms of the stilling of defilement \marginnote{\textcolor{teal}{\footnotesize\{581|523\}}}{}and the remaining [result of past] clinging that are present in one who has reached it by means of development. But [secondly, it is called \emph{without result of past clinging left}] since after the last consciousness of the Arahant, who has abandoned arousing [future aggregates] and so prevented kamma from giving result in a future [existence], there is no further arising of aggregates of existence, and those already arisen have disappeared. So the [result of past] clinging that remained is non-existent; and it is in terms of this non-existence, in the sense that “there is no [result of past] clinging here” that that [same goal is called] \emph{without result of past clinging left }(see \textbf{\cite{It}38}).

                        \vismParagraph{XVI.74}{74}{}
                        Because it can be arrived at by distinction of knowledge that succeeds through untiring perseverance, and because it is the word of the Omniscient One, Nibbāna is not non-existent as regards individual essence in the ultimate sense; for this is said: “Bhikkhus, there is an unborn, an unbecome, an unmade, an unformed” (\textbf{\cite{It}37}; \textbf{\cite{Ud}80}).\footnote{\vismAssertFootnoteCounter{18}\vismHypertarget{XVI.n18}{}This discussion falls under three headings: Questions one to four refute the assertion that Nibbāna is mythical and non-existent; questions five to seven refute the assertion that Nibbāna is “mere destruction;” (further argued in the \emph{Sammohavinodanī—}\textbf{\cite{Vibh-a}51f.}) the remaining questions deal with the proof that only Nibbāna (and not the atom, etc.,) is permanent because uncreated.

                                The \emph{Paramatthamañjūsā} (Vism-mhṭ) covers the subject at great length and reinforces the arguments given here with much syllogistic reasoning. However, only the following paragraph will be quoted here, which is reproduced in the commentaries to \textbf{\cite{Ud}80} and \textbf{\cite{It}37}. (The last sentence marked ** appears only in the Udāna Commentary. Readings vary considerably):

                                “Now, in the ultimate sense the existingness of the Nibbāna-element has been demonstrated by the Fully Enlightened One, compassionate for the whole world, by many sutta passages such as ‘Dhammas without condition,’ ‘Unformed dhammas’ (see \textbf{\cite{Dhs}2}), ‘Bhikkhus, there is that base (sphere) where neither earth … ’ (\textbf{\cite{Ud}80}), ‘This state is very hard to see, that is to say, the stilling of all formations, the relinquishing of all substance of becoming’ (\textbf{\cite{D}II 36}; \textbf{\cite{M}I 167}), ‘Bhikkhus, I shall teach you the unformed and the way leading to the unformed’ (\textbf{\cite{S}IV 362}), and so on, and in this sutta, ‘Bhikkhus, there is an unborn …” (\textbf{\cite{It}87}; \textbf{\cite{Ud}80}). So even if the wise trust completely in the Dispensation and have no doubts, though they may not yet have had direct perception of it, nevertheless there are persons who come to understand through another’s guidance (reading \emph{paraneyya-buddhino}); and the intention here is that this logical reasoning under the heading of deduction (\emph{niddhāraṇa}) should be for the purpose of removing their doubts.

                                “Just as it is owing to full-understanding (reading \emph{yathā pariññeyyatāya}) that from the sense desires and from materiality, etc (reading \emph{rūpādīnaṃ}), that have something beyond them, there is made known an escape [from them] that is their opposite and whose individual essence is devoid of them, so there must exist an escape that is the opposite of, and whose individual essence is devoid of, all formed dhammas, all of which have the aforesaid individual essence (reading \emph{evaṃ taṃ-sabhāvānaṃ}), and it is this escape that is the unformed element.

                                “Besides, insight knowledge, which has formed dhammas as its object, and also conformity knowledge, abandon the defilements with the abandoning consisting in substitution of opposites, being unable to abandon them with the abandoning consisting in cutting off. Likewise, the kind of knowledge that has conventional truth (\emph{sammuti-sacca}) [that is, concepts] as its object, in the first jhāna, etc., abandons the defilements only with the abandoning consisting in suppression, not by cutting them off. So, because the kind of knowledge that has formed dhammas as its object and that which has conventional truth as its object are both incapable of abandoning defilements by cutting them off, there must [consequently] exist an object for the noble-path-knowledge that effects their abandonment by cutting them off, [which object must be] of a kind opposite to both. And it is this that is the unformed element.

                                “Likewise, the words, ‘Bhikkhus, there is an unborn, an unbecome, an unmade, an unformed’ and so on, which demonstrate the existingness of Nibbāna in the ultimate sense, are not misleading because they are spoken by the Blessed One, like the words, ‘All formations are impermanent, all formations are painful, all \emph{dhammas }(states) are not self’ (Dhp 277–79; \textbf{\cite{A}I 286}, etc.).

                                “Likewise, in certain instances as regards scope, the word ‘Nibbāna’ has the correct ultimate meaning for its scope [precisely] because of the existence of its use as a mere metaphor—like the word ‘lion’ (see \hyperlink{XV.n12}{Ch. XV, note 12}{}, for the word lion). *Or alternatively, the unformed element exists in the ultimate sense also, because its individual essence is the opposite of, is free from, that of the other kind [of element such as] the earth element and feeling*” (\textbf{\cite{Vism-mhṭ}534–540}). The Pali of the last two paragraphs is taken to read thus:

                                “\emph{Tathā ‘atthi bhikkhave ajātaṃ abhūtaṃ akataṃ asaṅkhatan’ ti idaṃ nibbāna-padassa paramatthato atthibhāva-jotakaṃ vacanaṃ aviparītatthaṃ bhagavatā kathitattā; yaṃ hi bhagavatā bhāsitaṃ taṃ aviparitatthaṃ yathā taṃ ‘sabbe saṅkhārā aniccā sabbe saṅkhārā dukkhā sabbe dhammā anattā’ ti.}

                                \emph{“Tathā nibbāna-saddo katthaci (pi) visaye yathābhūta-paramatthavisayo upacāravuttimatta-sabhāvato (pi) seyyathā pi sīha-saddo. *Atha vā atth’eva paramatthato asaṅkhata-dhātu itaraṃ tabbiparītavimutta-sabhāvattā seyyathā pi pathavī-dhātu vedanā vā ti.}”*

                                The discussion is summarized and additional arguments are added in the \emph{Abhidhammāvatāra}. The later \emph{Abhidhammatthasaṅgaha} appears to have shelved the problem. It may be noted that in the whole of this discussion (particularly in the answer to Q. 4) no mention is made of the abandoning of the inherent tendencies (\emph{anusaya}) in the attainment of Nibbāna (see, e.g., MN 64; \textbf{\cite{S}II 66}). For derivations of the word “Nibbāna” see \hyperlink{VIII.247}{VIII.247}{} and note 72.}

                        \marginnote{\textcolor{teal}{\footnotesize\{582|524\}}}{}This is the section of the definition dealing with the description of the cessation of suffering.
                    \subsubsection[\vismAlignedParas{§75–83}The Truth of the Way]{The Truth of the Way}

                        \vismParagraph{XVI.75}{75}{}
                        In the description of the \emph{way leading to the cessation of suffering }eight things are given. Though they have, of course, already been explained as to meaning in the Description of the Aggregates, still we shall deal with them here in order to remain aware of the difference between them when they occur in a single moment [on the occasion of the path].

                        \vismParagraph{XVI.76}{76}{}
                        Briefly (see \hyperlink{XXII.31}{XXII.31}{} for details), when a meditator is progressing towards the penetration of the four truths, his eye of understanding with Nibbāna as its object eliminates the inherent tendency to ignorance, and that is \emph{right view}. It has right seeing as its characteristic. Its function is to reveal elements. It is manifested as the abolition of the darkness of ignorance.

                        \vismParagraph{XVI.77}{77}{}
                        When he possesses such view, his directing of the mind on to Nibbāna, which [directing] is associated with that [right view], abolishes wrong thinking, and that is \emph{right thinking. }Its characteristic is right directing of the mind on to [its object]. Its function is to bring about absorption [of the path consciousness in Nibbāna as object]. It is manifested as the abandoning of wrong thinking.

                        \vismParagraph{XVI.78}{78}{}
                        And when he sees and thinks thus, his abstinence from wrong speech, which abstinence is associated with that [right view], abolishes bad verbal \marginnote{\textcolor{teal}{\footnotesize\{583|525\}}}{}conduct, \textcolor{brown}{\textit{[510]}} and that is called \emph{right speech. }It has the characteristic of embracing.\footnote{\vismAssertFootnoteCounter{19}\vismHypertarget{XVI.n19}{}“Right speech has as its individual essence the embracing of associated states through affectionateness because it is the opposite of false speech and the other kinds, which, being rough owing to their respective functions of deceiving, etc., do not embrace” (\textbf{\cite{Vism-mhṭ}541}).} Its function is to abstain. It is manifested as the abandoning of wrong speech.

                        \vismParagraph{XVI.79}{79}{}
                        When he abstains thus, his abstinence from killing living things, which abstinence is associated with that [right view], cuts off wrong action, and that is called \emph{right action. }It has the characteristic of originating.\footnote{\vismAssertFootnoteCounter{20}\vismHypertarget{XVI.n20}{}“Bodily work (\emph{kāyika-kriyā}) originates (sets up) whatever has to be done. And that originating (setting up) is itself a combining, so the abstinence called right action is said to have originating as its individual essence. Or it is the picking up of associated states which is the causing of them to be originated, on the part of bodily work, like the picking up of a burden” (\textbf{\cite{Vism-mhṭ}541}).} Its function is to abstain. It is manifested as the abandoning of wrong action.

                        \vismParagraph{XVI.80}{80}{}
                        When his right speech and right action are purified, his abstinence from wrong livelihood, which abstinence is associated with that, [right view] cuts off scheming, etc., and that is called \emph{right livelihood. }It has the characteristic of cleansing.\footnote{\vismAssertFootnoteCounter{21}\vismHypertarget{XVI.n21}{}“The purification of a living being or of associated states is ‘\emph{cleansing’” }(Vism-mhṭ 541).} Its function is to bring about the occurrence of a proper livelihood. It is manifested as the abandoning of wrong livelihood.

                        \vismParagraph{XVI.81}{81}{}
                        When he is established on that plane of virtue called right speech, right action, and right livelihood, his energy, which is in conformity and associated with that [right view], cuts off idleness, and that is called \emph{right effort}. It has the characteristic of exerting. Its function is the non-arousing of unprofitable things, and so on. It is manifested as the abandoning of wrong effort.

                        \vismParagraph{XVI.82}{82}{}
                        When he exerts himself thus, the non-forgetfulness in his mind, which is associated with that [right view], shakes off wrong mindfulness, and that is called \emph{right mindfulness. }It has the characteristic of establishing.\footnote{\vismAssertFootnoteCounter{22}\vismHypertarget{XVI.n22}{}\emph{Viniddhunana—}“shaking off”: not in PED, (but see under \emph{dhunāti}); cf. \hyperlink{II.11}{II.11}{}.} Its function is not to forget. It is manifested as the abandoning of wrong mindfulness.

                        \vismParagraph{XVI.83}{83}{}
                        When his mind is thus guarded by supreme mindfulness, the unification of mind, which is associated with that [right view], abolishes wrong concentration, and that is called \emph{right concentration. }It has the characteristic of \marginnote{\textcolor{teal}{\footnotesize\{584|526\}}}{}non-distraction. Its function is to concentrate. It is manifested as the abandoning of wrong concentration.

                        This is the method in the description of the way leading to the cessation of suffering.

                        This is how the exposition should be understood here as to defining birth and so on.
            \section[\vismAlignedParas{§84–104}General]{General}

                \vismParagraph{XVI.84}{84}{}
                \emph{9. As to knowledge’s function }(see \hyperlink{XVI.14}{§14}{}): the exposition should be understood according to knowledge of the truths. For knowledge of the truths is twofold, namely, knowledge as idea and knowledge as penetration (cf. \textbf{\cite{S}V 431}f; also \hyperlink{XXII.92}{XXII.92ff.}{}). Herein, knowledge as idea is mundane and occurs through hearsay, etc., about cessation and the path. Knowledge consisting in penetration, which is supramundane, penetrates the four truths as its function by making cessation its object, according as it is said, “Bhikkhus, he who sees suffering sees also the origin of suffering, sees also the cessation of suffering, sees also the way leading to the cessation of suffering” (\textbf{\cite{S}V 437}), and it should be repeated thus of all [four truths]. But its function will be made clear in the purification by knowledge and vision (\hyperlink{XXII.92}{XXII.92f.}{}). \textcolor{brown}{\textit{[511]}}

                \vismParagraph{XVI.85}{85}{}
                When this knowledge is mundane, then, occurring as the overcoming of obsessions, the knowledge of suffering therein forestalls the [false] view of individuality; the knowledge of origin forestalls the annihilation view; the knowledge of cessation forestalls the eternity view; the knowledge of the path forestalls the moral-inefficacy-of-action view. Or alternatively, the knowledge of suffering forestalls wrong theories of fruit, in other words, [seeing] lastingness, beauty, pleasure, and self in the aggregates, which are devoid of lastingness, beauty, pleasure, and self; and knowledge of origin forestalls wrong theories of cause that occur as finding a reason where there is none, such as “The world occurs owing to an Overlord, a Basic Principle, Time, Nature (Individual Essence),” etc.;\footnote{\vismAssertFootnoteCounter{23}\vismHypertarget{XVI.n23}{}“Those who hold that there is an Overlord (Omnipotent Being) as reason say, ‘An Overlord (\emph{issara}) makes the world occur, prepares it, halts, it, disposes of it.’ Those who hold that there is a Basic Principle as reason say, ‘The world is manifested from out of a Basic Principle (\emph{padhāna}), and it is reabsorbed in that again.’ Those who hold the theory of Time say:

                        Time it is that creates beings, Disposes of this generation; Time watches over those who sleep; To outstrip Time is hard indeed.

                        Those who hold the theory of Nature (\emph{sabhāva}—individual essence) say, ‘The world appears and disappears (\emph{sambhoti vibhoti ca}) just because of its nature (individual essence), like the sharp nature (essence) of thorns, like the roundness of wood-apples (\emph{kabiṭṭha = Feronia elephantum}), like the variedness of wild beasts, birds, snakes, and so on.’ The word, ‘etc.’ refers to those who preach fatalism and say, ‘The occurrence of the world is due to atoms. All is due to causes effected in the past. The world is determined, like drilled gems threaded on an unbroken string. There is no doing by a man’; and to those who preach chance: It is by chance that they occur,

                        By chance as well that they do not; Pleasure and pain are due to chance, This generation [lives] by chance; and to those who preach liberation by chance.

                        \emph{“‘Taking final release to be in the immaterial world”} like that of Rāmudaka, Āḷāra (see MN 26), etc., or ‘\emph{in a World Apex} (\emph{World Shrine—lokathūpika})’ like that of the Nigaṇṭhas (Jains). And by the word, ‘etc.’ are included also the preachers of ‘Nibbāna here and now’ as the self’s establishment in its own self when it has become dissociated from the qualities (guṇa) owing to the non-occurrence of the Basic Principle (\emph{padhāna}, Skr. \emph{pradhāna}—see the Sāṃkhya system), and being in the same world as, in the presence of, or in union with, Brahmā” (\textbf{\cite{Vism-mhṭ}543}).} the knowledge of cessation forestalls such wrong theories of \marginnote{\textcolor{teal}{\footnotesize\{585|527\}}}{}cessation as taking final release to be in the immaterial world, in a World Apex (Shrine), etc.; and the path knowledge forestalls wrong theories of means that occur by taking to be the way of purification what is not the way of purification and consists in devotion to indulgence in the pleasures of sense desire and in self-mortification. Hence this is said:
                \begin{verse}
                    As long as a man is vague about the world.\\{}
                    About its origin, about its ceasing.\\{}
                    About the means that lead to its cessation.\\{}
                    So long he cannot recognize the truths.
                \end{verse}


                This is how the exposition should be understood here as to knowledge’s function.

                \vismParagraph{XVI.86}{86}{}
                \emph{10. As to division of content: }all states excepting craving and states free from cankers are included in the \emph{truth of suffering. }The thirty-six modes of behaviour of craving\footnote{\vismAssertFootnoteCounter{24}\vismHypertarget{XVI.n24}{}“The ‘\emph{thirty-six modes of behaviour of craving’ }are the three, craving for sense desires, for becoming, and for non-becoming, in the cases of each one of the twelve internal-external bases; or they are those given in the Khuddakavatthu-Vibhaṅga (\textbf{\cite{Vibh}391} and 396), leaving out the three periods of time, for with those they come to one hundred and eight” (\textbf{\cite{Vism-mhṭ}544}). “‘\emph{Thoughts of renunciation, etc.}’: in the mundane moment they are the three separately, that is, non-greed, loving kindness, and compassion; they are given as one at the path moment, owing to the cutting off of greed, ill will and cruelty” (\textbf{\cite{Vism-mhṭ}544}).

                        “‘Consciousness concentration (\emph{citta-samādhi})’ is the road to power consisting of [purity of] consciousness, they say” (\textbf{\cite{Vism-mhṭ}544}).} are included in the \emph{truth of origin. }The \emph{truth of cessation }is unmixed. As regards the \emph{truth of the path}: the heading of \emph{right view }includes the fourth road to power consisting in inquiry, the understanding faculty, the understanding power, and the investigation-of-states enlightenment factor. The term \emph{right thinking }includes the three kinds of applied thought beginning with that of renunciation (\textbf{\cite{D}III 215}). The term \emph{right speech }includes the four kinds of good verbal conduct (\textbf{\cite{A}II 131}). The term \emph{right action }includes the three kinds of good bodily conduct (cf. \textbf{\cite{M}I 287}). The heading \emph{right livelihood }includes fewness of wishes and contentment. Or all these [three] constitute the virtue loved by Noble Ones, and the virtue loved by Noble Ones has to be embraced by the hand of faith; \marginnote{\textcolor{teal}{\footnotesize\{586|528\}}}{}consequently the faith faculty, the faith power, and the road to power consisting in zeal are included because of the presence of these [three]. The term \emph{right effort }includes fourfold right endeavour, the energy faculty, energy power, and energy enlightenment factor. The term \emph{right mindfulness }includes the fourfold foundation of mindfulness, the mindfulness faculty, the mindfulness power, and the mindfulness enlightenment factor. The term \emph{right concentration }includes the three kinds of concentration beginning with that accompanied by applied and sustained thought (\textbf{\cite{D}III 219}), consciousness concentration, the concentration faculty, \textcolor{brown}{\textit{[512]}} the concentration power, and the enlightenment factors of happiness, tranquillity, concentration, and equanimity.

                This is how the exposition should be understood as to division of content.

                \vismParagraph{XVI.87}{87}{}
                \emph{11. As to simile}: The truth of suffering should be regarded as a burden, the truth of origin as the taking up of the burden, the truth of cessation as the putting down of the burden, the truth of the path as the means to putting down the burden (see \textbf{\cite{S}III 26}), The truth of suffering is like a disease, the truth of origin is like the cause of the disease, the truth of cessation is like the cure of the disease, and the truth of the path is like the medicine. Or the truth of suffering is like a famine, the truth of origin is like a drought, the truth of cessation is like plenty, and the truth of the path is like timely rain.

                Furthermore, these truths can be understood in this way by applying these similes: enmity, the cause of the enmity, the removal of the enmity, and the means to remove the enmity; a poison tree, the tree’s root, the cutting of the root, and the means to cut the root; fear, the cause of fear, freedom from fear, and the means to attain it; the hither shore, the great flood, the further shore, and the effort to reach it.

                This is how the exposition should be understood as to simile.

                \vismParagraph{XVI.88}{88}{}
                \emph{12. As to tetrad: }(a) there is suffering that is not noble truth, (b) there is noble truth that is not suffering, (c) there is what is both suffering and noble truth, and (d) there is what is neither suffering nor noble truth. So also with origin and the rest.

                \vismParagraph{XVI.89}{89}{}
                Herein, (a) though states associated with the path and the fruits of asceticism are suffering since they are suffering due to formations (see \hyperlink{XVI.35}{§35}{}) because of the words, “What is impermanent is painful” (\textbf{\cite{S}II 53}; III 22), still they are not the noble truth [of suffering], (b) Cessation is a noble truth but it is not suffering, (c) The other two noble truths can be suffering because they are impermanent, but they are not so in the real sense of that for the full-understanding of which (see \hyperlink{XVI.28}{§28}{}) the life of purity is lived under the Blessed One. The five aggregates [as objects] of clinging, except craving, are in all aspects both suffering and noble truth. \textcolor{brown}{\textit{[513]}} (d) The states associated with the path and the fruits of asceticism are neither suffering in the real sense of that for the full-understanding of which the life of purity is lived under the Blessed One, nor are they noble truth. Origin, etc., should also be construed in the corresponding way. This is how the exposition should be understood here as to tetrad.

                \vismParagraph{XVI.90}{90}{}
                \emph{13. As to void, singlefold, and so on: }firstly, \emph{as to void}: in the ultimate sense all the truths should be understood as void because of the absence of (i) any experiencer, (ii) any doer, (iii) anyone who is extinguished, and (iv) any goer. Hence this is said:
                \begin{verse}
                    \marginnote{\textcolor{teal}{\footnotesize\{587|529\}}}{}For there is suffering, but none who suffers;\\{}
                    Doing exists although there is no door.\\{}
                    Extinction is but no extinguished person;\\{}
                    Although there is a path, there is no goer.
                \end{verse}


                Or alternatively:
                \begin{verse}
                    So void of lastingness, and beauty, pleasure, self,\\{}
                    Is the first pair, and void of self the deathless state,\\{}
                    And void of lastingness, of pleasure and of self\\{}
                    Is the path too; for such is voidness in these four.
                \end{verse}


                \vismParagraph{XVI.91}{91}{}
                Or three are void of cessation, and cessation is void of the other three. Or the cause is void of the result, because of the absence of suffering in the origin, and of cessation in the path; the cause is not gravid with its fruit like the Primordial Essence of those who assert the existence of Primordial Essence. And the result is void of the cause owing to the absence of inherence of the origin in suffering and of the path in cessation; the fruit of a cause does not have its cause inherent in it, like the two atoms, etc., of those who assert inherence. Hence this is said:
                \begin{verse}
                    Here three are of cessation void;\\{}
                    Cessation void, too, of these three;\\{}
                    The cause of its effect is void,\\{}
                    Void also of its cause the effect must be.
                \end{verse}


                This, in the first place, is how the exposition should be understood as to void.\footnote{\vismAssertFootnoteCounter{25}\vismHypertarget{XVI.n25}{}It may be noted in passing that the word \emph{anattā }(not-self) is never applied directly to Nibbāna in the Suttas (and Abhidhamma), or in Bhante Buddhaghosa’s commentaries (Cf. \hyperlink{XXI.n4}{Ch. XXI, note 4}{}, where \textbf{\cite{Vism-mhṭ}} is quoted explaining the scope of applicability of the “three characteristics”). The argument introduced here that, since \emph{attā }(self) is a non-existent myth, therefore Nibbāna (the unformed dhamma, the truth of cessation) is \emph{void of self }(\emph{atta-suñña}) is taken up in the \emph{Saddhammappakāsinī} (Hewavitarne Ce, p. 464):

                        All dhammas whether grouped together In three ways, two ways, or one way, Are void: thus here in this dispensation Do those who know voidness make their comment.

                        “How so? Firstly, all mundane dhammas are void of lastingness, beauty, pleasure, and self because they are destitute of lastingness, beauty, pleasure, and self. Path and fruition dhammas are void of lastingness, pleasure, and self, because they are destitute of lastingness, pleasure, and self. Nibbāna dhammas (pl.) are void of self because of the non-existence (\emph{abhāva}) of self. [Secondly,] formed dhammas, both mundane and supramundane, are all void of a [permanent] living being (\emph{satta}) because of the non-existence of [such] a living being of any sort whatever. The unformed dhamma (sing.) is void of formations because of the non-existence (\emph{abhāva: }or absence) of those formations too. [Thirdly,] all dhammas formed and unformed are void of self because of the non-existence of any person (\emph{puggala}) called ‘self’ (\emph{attā}).”} \textcolor{brown}{\textit{[514]}}

                \vismParagraph{XVI.92}{92}{}
                \emph{14. As to singlefold and so on}: and here all \emph{suffering }is of one kind as the state of occurrence. It is of two kinds as mentality-materiality. It is of three kinds as \marginnote{\textcolor{teal}{\footnotesize\{588|530\}}}{}divided into rebirth-process becoming in the sense sphere, fine-material sphere, and immaterial sphere. It is of four kinds classed according to the four nutriments. It is of five kinds classed according to the five aggregates [as objects] of clinging.

                \vismParagraph{XVI.93}{93}{}
                Also \emph{origin }is of one kind as making occur. It is of two kinds as associated and not associated with [false] view. It is of three kinds as craving for sense desires, craving for becoming, and craving for non-becoming. It is of four kinds as abandonable by the four paths. It is of five kinds classed as delight in materiality, and so on. It is of six kinds classed as the six groups of craving.

                \vismParagraph{XVI.94}{94}{}
                Also \emph{cessation }is of one kind being the unformed element. But indirectly it is of two kinds as “with result of past clinging left” and as “without result of past clinging left”;\footnote{\vismAssertFootnoteCounter{26}\vismHypertarget{XVI.n26}{}“It is clung-to (\emph{upādiyati}) by the kinds of clinging (\emph{upādāna}), thus it is ‘result-of-past-clinging’ (\emph{upādi}): this is the pentad of aggregates [as objects] of clinging. Taking Nibbāna, which is the escape from that, as its stilling, its quieting, since there is remainder of it up till the last consciousness [of the Arahant], after which there is no remainder of it, the Nibbāna element is thus conventionally spoken of in two ways as ‘with result of past clinging left’ (\emph{sa-upādi-sesa}) and ‘without result of past clinging left’ (\emph{an-upādi-sesa})” (\textbf{\cite{Vism-mhṭ}547}).} and of three kinds as the stilling of the three kinds of becoming; and of four kinds as approachable by the four paths; and of five kinds as the subsiding of the five kinds of delight; and of six kinds classed according to the destruction of the six groups of craving.

                \vismParagraph{XVI.95}{95}{}
                Also the \emph{path }is of one kind as what should be developed. It is of two kinds classed according to serenity and insight, or classed according to seeing and developing. It is of three kinds classed according to the three aggregates; for the [path], being selective, is included by the three aggregates, which are comprehensive, as a city is by a kingdom, according as it is said: “The three aggregates are not included in the Noble Eightfold Path, friend Visākha, but the Noble Eightfold Path is included by the three aggregates. Any right speech, any right action, any right livelihood: these are included in the virtue aggregate. Any right effort, any right mindfulness, any right concentration: these are included in the concentration aggregate. Any right view, any right thinking: these are included in the understanding aggregate” (\textbf{\cite{M}I 301}).

                \vismParagraph{XVI.96}{96}{}
                For here the three beginning with right speech are virtue and so they are included in the virtue aggregate, being of the same kind. For although in the text the description is given in the locative case as “in the virtue aggregate,” still the meaning should be understood according to the instrumental case [that is, “by the virtue aggregate.”]

                As to the three beginning with right effort, concentration cannot of its own nature cause absorption through unification on the object; but with energy accomplishing its function of exerting and mindfulness accomplishing its function of preventing wobbling, it can do so.

                \vismParagraph{XVI.97}{97}{}
                Here is a simile: three friends, [thinking,] “We will celebrate the festival,” entered a park. Then one saw a champak tree in full blossom, but he could not reach the flowers by raising his hand. The second bent down for the first to climb \marginnote{\textcolor{teal}{\footnotesize\{589|531\}}}{}on his back. But although standing on the other’s back, he still could not pick them because of his unsteadiness. \textcolor{brown}{\textit{[515]}} Then the third offered his shoulder [as support]. So standing on the back of the one and supporting himself on the other’s shoulder, he picked as many flowers as he wanted and after adorning himself, he went and enjoyed the festival. And so it is with this.

                \vismParagraph{XVI.98}{98}{}
                For the three states beginning with right effort, which are born together, are like the three friends who enter the park together. The object is like the champak tree in full blossom. Concentration, which cannot of its own nature bring about absorption by unification on the object, is like the man who could not pick the flower by raising his arm. Effort is like the companion who bent down, giving his back to mount upon. Mindfulness is like the friend who stood by, giving his shoulder for support. Just as standing on the back of the one and supporting himself on the other’s shoulder he could pick as many flowers as he wanted, so too, when energy accomplishes its function of exerting and when mindfulness accomplishes its function of preventing wobbling, with the help so obtained concentration can bring about absorption by unification on the object. So here in the concentration aggregate it is only concentration that is included as of the same kind. But effort and mindfulness are included because of their action [in assisting].

                \vismParagraph{XVI.99}{99}{}
                Also as regards right view and right thinking, understanding cannot of its own nature define an object as impermanent, painful, not-self. But with applied thought giving [assistance] by repeatedly hitting [the object] it can.

                \vismParagraph{XVI.100}{100}{}
                How? Just as a money changer, having a coin placed in his hand and being desirous of looking at it on all sides equally, cannot turn it over with the power of his eye only, but by turning it over with his fingers he is able to look at it on all sides, similarly understanding cannot of its own nature define an object as impermanent and so on. But [assisted] by applied thought with its characteristic of directing the mind on to [the object] and its function of striking and threshing, as it were, hitting and turning over, it can take anything given and define it. So here in the understanding aggregate it is only right view that is included as of the same kind. But right thinking is included because of its action [in assisting].

                \vismParagraph{XVI.101}{101}{}
                So the path is included by the three aggregates. Hence it was said that it is of three kinds classed according to the three aggregates. And it is of four kinds as the path of stream-entry and so on.

                \vismParagraph{XVI.102}{102}{}
                In addition, all the truths are of one kind because they are not unreal, or because they must be directly known. They are of two kinds as (i and ii) mundane and (iii and iv) supramundane, or (i, ii, and iv) formed and (iii) unformed. They are of three kind as (ii) to be abandoned by seeing and development, (iii and iv) not to be abandoned, and (i) neither to be abandoned nor not to be abandoned. They are of four kinds classed according to what has to be fully understood, and so on (see \hyperlink{XVI.28}{§28}{}).

                This is how the exposition should be understood as to singlefold and so on.

                \textcolor{brown}{\textit{[516]}}

                \vismParagraph{XVI.103}{103}{}
                \marginnote{\textcolor{teal}{\footnotesize\{590|532\}}}{}\emph{15. As to similar and dissimilar, }all the truths are similar to each other because they are not unreal, are void of self, and are difficult to penetrate, according as it is said: “What do you think, Ānanda, which is more difficult to do, more difficult to perform, that a man should shoot an arrow through a small keyhole from a distance time after time without missing or that he should penetrate the tip of a hair split a hundred times with the tip [of a similar hair]?”—“This is more difficult to do, venerable sir, more difficult to perform, that a man should penetrate the tip of a hair split a hundred times with the tip [of a similar hair].”—“They penetrate something more difficult to penetrate than that, Ānanda, who penetrate correctly thus, ‘This is suffering’ … who penetrate correctly thus, ‘This is the way leading to the cessation of suffering’” (\textbf{\cite{S}V 454}). They are dissimilar when defined according to their individual characteristics.

                \vismParagraph{XVI.104}{104}{}
                And the first two are similar since they are profound because hard to grasp, since they are mundane, and since they are subject to cankers. They are dissimilar in being divided into fruit and cause, and being respectively to be fully understood and to be abandoned. And the last two are similar since they are hard to grasp because profound, since they are supramundane, and since they are free from cankers. They are dissimilar in being divided into object and what has an object, and in being respectively to be realized and to be developed. And the first and third are similar since they come under the heading of result. They are dissimilar in being formed and unformed. Also the second and fourth are similar since they come under the heading of cause. They are dissimilar in being respectively entirely unprofitable and entirely profitable. And the first and fourth are similar in being formed. They are dissimilar in being mundane and supramundane. Also the second and the third are similar since they are the state of neither-trainer-nor-non-trainer (see \textbf{\cite{Vibh}114}). They are dissimilar in being respectively with object and without object.
                \begin{verse}
                    A man of vision can apply\\{}
                    By suchlike means his talent so\\{}
                    That he among the truths may know\\{}
                    The similar and contrary.
                \end{verse}


                The sixteenth chapter called “The Description of the Faculties and Truths” in the Treatise on the Development of Understanding in the \emph{Path of Purification }composed for the purpose of gladdening good people.
        \chapter[The Soil of Understanding—Conclusion: Dependent Origination]{The Soil of Understanding—Conclusion: Dependent Origination\vismHypertarget{XVII}\newline{\textnormal{\emph{Paññā-bhūmi-niddesa}}}}
            \label{XVII}

            \section[\vismAlignedParas{§1–24}A. Definition of Dependent Origination]{A. Definition of Dependent Origination}

                \vismParagraph{XVII.1}{1}{}
                \marginnote{\textcolor{teal}{\footnotesize\{591|533\}}}{}\textcolor{brown}{\textit{[517]}} The turn has now come for the exposition of the dependent origination itself, and the dependently-originated states comprised by the word “etc.,” since these still remain out of the states called the “soil” (\emph{bhūmi}), of which it was said above, “The states classed as aggregates, bases, elements, faculties, truths, and dependent origination, etc., are the ‘soil’” (\hyperlink{XIV.32}{XIV.32}{}).

                \vismParagraph{XVII.2}{2}{}
                Herein, firstly, it is the states beginning with ignorance that should be understood as \emph{dependent origination}. For this is said by the Blessed One: “And what is the dependent origination, bhikkhus? With ignorance as condition there are [volitional] formations; with formations as condition, consciousness; with consciousness as condition, mentality-materiality; with mentality-materiality as condition, the sixfold base; with the sixfold base as condition, contact; with contact as condition, feeling; with feeling as condition, craving; with craving as condition, clinging; with clinging as condition, becoming; with becoming as condition, birth; with birth as condition there is ageing-and-death, and sorrow, lamentation, pain, grief, and despair; thus there is the arising of this whole mass of suffering. This is called the dependent origination, bhikkhus” (\textbf{\cite{S}II 1}).

                \vismParagraph{XVII.3}{3}{}
                Secondly, it is the states beginning with ageing-and-death that should be understood as \emph{dependently-originated states}. For this is said by the Blessed One: “And what are the dependently-originated states, bhikkhus? Ageing-and-death is impermanent, bhikkhus, formed, dependently originated, subject to destruction, subject to fall, subject to fading away, subject to cessation.\footnote{\vismAssertFootnoteCounter{1}\vismHypertarget{XVII.n1}{}“‘Subject to destruction’ (\emph{khaya-dhamma}) means that its individual essence is the state of being destroyed (\emph{khayana-sabhāva})” (\textbf{\cite{Vism-mhṭ}549}). The other expressions are explained in the same way.} Birth is impermanent, bhikkhus, … Becoming … Clinging … Craving … Feeling … Contact … The sixfold base … Mentality-materiality … Consciousness … Formations … Ignorance is impermanent, bhikkhus, formed, dependently originated, subject to destruction, subject to fall, subject to fading away, subject \marginnote{\textcolor{teal}{\footnotesize\{592|534\}}}{}to cessation. These are called the dependently-originated states, bhikkhus” (\textbf{\cite{S}II 26}). \textcolor{brown}{\textit{[518]}}

                \vismParagraph{XVII.4}{4}{}
                Here is a brief explanation. The states that are conditions should be understood as the \emph{dependent origination}. The states generated by such and such conditions are \emph{dependently-originated states}.

                \vismParagraph{XVII.5}{5}{}
                How is that to be known? By the Blessed One’s word. For it is precisely those states which are conditions, that with the synonyms beginning with “reality” have been called “dependent origination” by the Blessed One when teaching the dependent origination in the sutta on the Teaching of the Dependent Origination and Dependently-originated States thus:

                “And what is dependent origination, bhikkhus?

                “With birth as condition, bhikkhus, there is ageing and death. Whether Perfect Ones arise or do not arise, there yet remains that element, relatedness of states, regularity of states, specific conditionally. The Perfect One discovers it, penetrates to it. Having discovered it, penetrated to it, he announces it, teaches it, makes it known, establishes, exposes, expounds, and explains it: ‘See,’ he says, ‘With birth as condition there is ageing and death.’

                “With becoming as condition, bhikkhus, there is birth … With ignorance as condition, bhikkhus, there are formations. Whether Perfect Ones arise or do not arise, there yet remains that element, relatedness of states, regularity of states, specific conditionally. The Perfect One discovers it, penetrates to it. Having discovered it, penetrated to it, he announces it, teaches it, makes it known, establishes, exposes, expounds and explains it: ‘See,’ he says, ‘With ignorance as condition there are formations.’

                “So, bhikkhus, that herein which is reality, not unreality, not otherness, specific conditionality: that is called dependent origination” (\textbf{\cite{S}II 25f.}).

                Consequently, it should be understood that dependent origination has the characteristic of being the conditions for the states beginning with ageing-and-death. Its function is to continue [the process of] suffering. It is manifested as the wrong path.

                \vismParagraph{XVII.6}{6}{}
                Because particular states are produced by particular conditions, neither less nor more, it is called \emph{reality} (suchness). Because once the conditions have met in combination there is no non-producing, even for an instant, of the states they generate, it is called not \emph{unreality} (not unsuchness). Because there is no arising of one state with another state’s conditions, it is called \emph{not otherness}. Because there is a condition, or because there is a total of conditions, for these states beginning with ageing-and-death as already stated, it is called \emph{specific conditionality}.

                \vismParagraph{XVII.7}{7}{}
                Here is the word meaning: \emph{idappaccayā} (lit. that-conditions) = \emph{imesaṃ paccayā }(conditions for those); \emph{idappaccayā} (that-conditions) = \emph{idappaccayatā} (that-conditionality, conditionality for those, specific conditionality). Or alternatively, \emph{idappaccayatā} (that-conditionality) = \emph{idappaccayānaṃ samūho} (the total of that-conditions, total specific conditionality).

                \vismParagraph{XVII.8}{8}{}
                \marginnote{\textcolor{teal}{\footnotesize\{593|535\}}}{}The characteristic must be sought from grammar. Some, in fact, [say that the expression \emph{paṭicca samuppāda} (dependent origination) is characterized thus:] “having depended (\emph{paṭicca}), a right (\emph{sammā}) arising (\emph{uppāda}), [depending on causes rightly by] disregarding such causes conjectured by sectarians as the Primordial Essence (\emph{Prakṛti}), World Soul (\emph{Puruṣa}), and so on.” So what they call dependent origination (\emph{paṭicca samuppāda}) is a simple arising (\emph{uppāda}) [for they equate the prefix \emph{saṃ} only with \emph{sammā} (rightly) and ignore \emph{saṃ} (with, con-)]. That is untenable. \textcolor{brown}{\textit{[519]}} Why? (1) There is no such sutta; (2) it contradicts suttas; (3) it admits of no profound treatment; and (4) it is ungrammatical.

                \vismParagraph{XVII.9}{9}{}
                (1) No sutta describes the dependent origination as simple arising.

                (2) Anyone who asserts that dependent origination is of that kind involves himself in conflict with the Padesavihāra Sutta. How? The Newly Enlightened One’s abiding (\emph{vihāra}) is the bringing of the dependent origination to mind, because, of these words of the Blessed One’s: “Then in the first watch of the night the Blessed One brought to mind the dependent origination in direct and reverse order” [as origination and cessation] (\textbf{\cite{Vin}I 1}; \textbf{\cite{Ud}2}). Now, “\emph{padesavihāra}” is the abiding (\emph{vihāra}) in one part (\emph{desa}) of that, according as it is said, “Bhikkhus, I abode in a part of the abiding in which I abode when I was newly enlightened” (\textbf{\cite{S}V 12}; \textbf{\cite{Paṭis}I 107}). And there he abode in the vision of structure of conditions, not in the vision of simple arising, according as it is said, “So I understood feeling with wrong view as its condition, and feeling with right view as its condition, and feeling with wrong thinking as its condition …” (\textbf{\cite{S}V 12}), all of which should be quoted in full. So anyone who asserts that dependent origination is simple arising involves himself in conflict with the Padesavihāra Sutta.

                \vismParagraph{XVII.10}{10}{}
                There is likewise contradiction of the Kaccāna Sutta. For in the Kaccāna Sutta it is said, “When a man sees correctly with right understanding the origination of the world, Kaccāna, he does not say of the world that it is not” (\textbf{\cite{S}II 17}). And there it is the dependent origination in forward order, not simple arising, that, as the origination of the world from its conditions, is set forth in order to eliminate the annihilation view. For the annihilation view is not eliminated by seeing simple arising; but it is eliminated by seeing the chain of conditions as a chain of fruits following on a chain of conditions. So anyone who asserts that the dependent origination is simple arising involves himself in contradiction of the Kaccāna Sutta.

                \vismParagraph{XVII.11}{11}{}
                (3) \emph{It admits of no profound treatment}: this has been said by the Blessed One, “This dependent origination is profound, Ānanda, and profound it appears” (\textbf{\cite{D}II 55}; \textbf{\cite{S}II 92}). And the profundity is fourfold as we shall explain below (\hyperlink{XVII.304}{XVII.304f.}{}); but there is none of that in simple arising. And this dependent origination is explained [by the teachers] as adorned with the fourfold method (\hyperlink{XVII.309}{XVII.309}{}); but there is no [need of] any such tetrad of methods in simple arising. So dependent origination is not simple arising, since that admits of no profound treatment.

                \vismParagraph{XVII.12}{12}{}
                (4) \emph{It is ungrammatical}: \textcolor{brown}{\textit{[520]}} this word \emph{paṭicca} (lit. “having depended”; freely “due to,” “dependent”), [being a gerund of the verb \emph{paṭi + eti}, to go back to], \marginnote{\textcolor{teal}{\footnotesize\{594|536\}}}{}establishes a meaning [in a formula of establishment by verb] when it is construed as past with the same subject [as that of the principal verb], as in the sentence “Having depended on (\emph{paṭicca} = ‘due to’) the eye and visible objects, eye-consciousness arises (\emph{uppajjati})” (\textbf{\cite{S}II 72}). But if it is construed here with the word \emph{uppāda} (arising), [which is a noun], in a formula of establishment by noun, there is a breach of grammar, because there is no shared subject [as there is in above-quoted sentence], and so it does not establish any meaning al all. So the dependent origination is not simple arising because that is ungrammatical.

                \vismParagraph{XVII.13}{13}{}
                Here it might be [argued]: “We shall add the words ‘comes to be’ (\emph{hoti}) thus: ‘Having depended, arising comes to be’ (\emph{paṭicca, samuppādo hoti}).” That will not do. Why not? Because there is no instance in which it has been added, and because the fallacy of the arising of an arising follows. For in such passages as “\emph{Paṭiccasamuppādaṃ vo bhikkhave desessāmi. Katamo ca bhikkhave paṭiccasamuppādo … Ayaṃ vuccati bhikkhave paṭiccasamuppādo} (I shall teach you the dependent origination, bhikkhus. And what is the dependent origination? … This is called the dependent origination, bhikkhus)” (\textbf{\cite{S}II 1}), the words “comes to be” (\emph{hoti}) are not added in any single instance. And there is no [such expression as] “arising comes to be”: if there were, it would be tantamount to saying that arising itself had an arising too.

                \vismParagraph{XVII.14}{14}{}
                And those are wrong who imagine that specific conditionality (\emph{idappaccayatā}) is the specific conditions’ [abstract] essence—what is called “abstract essence” being a [particular] mode in ignorance, etc., that acts as cause in the manifestation of formations, etc.—and that the term “dependent origination” is used for an alteration in formations when there is that [particular mode in the way of occurrence of ignorance]. Why are they wrong? Because it is ignorance, etc., themselves that are called causes. For in the following passage it is ignorance, etc., themselves, not their alteration, that are called the causes [of these states]: “Therefore, Ānanda, just this is the cause, this is the source, this is the origin, this is the condition, for ageing-and-death, that is to say, birth … for formations, that is to say, (ignorance)” (\textbf{\cite{D}II 57–63}—the last clause is not in the Dīgha text). Therefore it is the actual states themselves as conditions that should be understood as “dependent origination.” So what was said above (\hyperlink{XVII.4}{§4}{}) can be understood as rightly said.

                \vismParagraph{XVII.15}{15}{}
                If any notion arises in the guise of a literal interpretation of the term “dependent origination” (\emph{paṭicca-samuppāda}) to the effect that it is only arising that is stated, it should be got rid of by apprehending the meaning of this expression in the following way. For:
                \begin{verse}
                    In double form this term relates to a totality of state\\{}
                    Produced from a conditionality;\\{}
                    Hence the conditions for that sum\\{}
                    Through metaphor’s device have come\\{}
                    To bear their fruits’ name figuratively\\{}
                    In the Blessed One’s exposition.
                \end{verse}


                \vismParagraph{XVII.16}{16}{}
                This term “dependent origination,” when applied to the total of states produced from the [total] conditionality, must be taken in two ways. \textcolor{brown}{\textit{[521]}} For \marginnote{\textcolor{teal}{\footnotesize\{595|537\}}}{}that [total] ought to be arrived at (\emph{paṭicco—}adj.),\footnote{\vismAssertFootnoteCounter{2}\vismHypertarget{XVII.n2}{}\emph{Paṭicco }as a declinable adjective is not in PED. \emph{Patīyamāna }(“when it is arrived at”): “When it is gone to by direct confrontation (\emph{paṭimukhaṃ upeyamāno}) by means of knowledge’s going; when it is penetrated to (\emph{abhisamiyamāna}), is the meaning” (Vism-mhṭ 555). The word \emph{paṭicca }(due to, depending on) and the word \emph{paccaya }(condition) are both gerunds of \emph{paṭi + eti }or \emph{ayati }(to go back to).} since when it is arrived at (\emph{paṭiyamāno}), it leads to [supramundane] welfare and bliss and so the wise [regard] it as worthy to be arrived at (\emph{paccetuṃ}); and then, when it arises (\emph{uppajjamāno}), it does so “together with” (\emph{saha}) and “rightly” (\emph{sammā}), not singly or causelessly, thus it is a co-arising (\emph{samuppādo}). Consequently: it is to be arrived at (\emph{paṭicco}) and it is a co-arising (\emph{samuppādo}), thus it is dependent origination (\emph{paṭicca-samuppāda}). Again: it arises as a togetherness (\emph{saha}), thus it is a co-arising (\emph{samuppāda}); but it does so having depended (\emph{paṭicca—}ger.) in combination with conditions, not regardless of them. Consequently: it, having depended (\emph{paṭicca}), is a co-arising (\emph{samuppāda}), thus in this way also it is dependent origination (\emph{paṭicca-samuppāda}). And the total of causes is a condition for that [total of states produced from the conditionality], so, because it is a condition for that, this [total of causes] is called, “dependent origination,” using for it the term ordinarily used for its fruit just as in the world molasses, which is a condition for phlegm, is spoken of thus, “Molasses is phlegm,” or just as in the Dispensation the arising of Buddhas, which is a condition for bliss, is spoken of thus, “The arising of Buddhas is bliss” (\textbf{\cite{Dhp}194}).

                \vismParagraph{XVII.17}{17}{}
                Or alternatively:
                \begin{verse}
                    The sum of causes too they call\\{}
                    “Facing its counterpart,” so all\\{}
                    Is in that sense “dependent,” as they tell;\\{}
                    This sum of causes too, as stated,\\{}
                    Gives fruits that rise associated,\\{}
                    So “co-arising” it is called as well.
                \end{verse}


                \vismParagraph{XVII.18}{18}{}
                This total of causes—indicated severally under the heading of each cause, beginning with ignorance—for the manifestation of formations, etc., is called “dependent” (\emph{paṭicco—}adj.), taking it as “facing, gone to, its counterpart” (\emph{paṭimukham ito}) owing to the mutual interdependence of the factors in the combination, in the sense both that they produce common fruit and that none can be dispensed with. And it is called a “co-arising” (\emph{samuppādo}) since it causes the states that occur in unresolved mutual interdependence to arise associatedly. Consequently: it is dependent (\emph{paṭicco}) and a co-arising (\emph{samuppādo}), thus in this way also it is dependent origination (\emph{paṭicca-samuppāda}).

                \vismParagraph{XVII.19}{19}{}
                Another method:
                \begin{verse}
                    This total conditionally, acting interdependently,\\{}
                    Arouses states together equally;\\{}
                    So this too is a reason here wherefore the Greatest Sage, the Seer,\\{}
                    Gave to this term its form thus succinctly.
                \end{verse}


                \vismParagraph{XVII.20}{20}{}
                \marginnote{\textcolor{teal}{\footnotesize\{596|538\}}}{}Among the conditions described under the headings of ignorance, etc., the respective conditions that make the [conditionally-arisen] states beginning with formations arise are incapable of making them arise when not mutually dependent and when deficient. Therefore this conditionality by depending (\emph{paṭicca—}ger.) makes states arise (\emph{uppādeti}) equally and together (\emph{samaṃ saha ca}), not piecemeal and successively—so it has been termed here thus by the Sage who is skilled in phraseology that conforms to its meaning: it has been accurately termed “dependent origination” (\emph{paṭicca samuppāda}), is the meaning.

                \vismParagraph{XVII.21}{21}{}
                And while so termed:
                \begin{verse}
                    The first component will deny the false view of eternity\\{}
                    And so on, and the second will prevent\\{}
                    The nihilistic type of view and others like it, while the two\\{}
                    Together show the true way that is meant.
                \end{verse}


                \vismParagraph{XVII.22}{22}{}
                \emph{The first}: the word “dependent” (\emph{paṭicca}) indicates the combination of the conditions, \textcolor{brown}{\textit{[522]}} since states in the process of occurring exist in dependence on the combining of their conditions; and it shows that they are not eternal, etc., thus denying the various doctrines of eternalism, no-cause, fictitious-cause, and power-wielder.\footnote{\vismAssertFootnoteCounter{3}\vismHypertarget{XVII.n3}{}“The doctrine of eternalism is that beginning ‘The world and self are eternal’” (\textbf{\cite{D}I 14}). That of no-cause is that beginning, ‘There is no cause, there is no condition, for the defilement of beings’ (\textbf{\cite{D}I 53}). That of fictitious-cause holds that the world’s occurrence is due to Primordial Essence (\emph{prakṛti}), atoms (\emph{aṇu}), time (\emph{kāla}), and so on. That of a power-wielder asserts the existence of an Overlord (\emph{issara}), or of a World-soul (\emph{Puruṣa}), or of Pajāpati (the Lord of the Race). Also the doctrines of Nature (\emph{sabhāva, }Skr. \emph{svabhāva }= individual essence), Fate (\emph{niyati}), and Chance (\emph{yadicchā}), should be regarded as included here under the doctrine of no-cause. Some, however, say that the doctrine of fictitious-cause is that beginning with ‘The eye is the cause of the eye,’ and that the doctrine of the power-wielder is that beginning, ‘Things occur owing to their own individual essence’ (see \hyperlink{XVI.n23}{Ch. XVI, n. 23}{})” (\textbf{\cite{Vism-mhṭ}557}).} What purpose indeed would the combining of conditions serve, if things were eternal, or if they occurred without cause, and so on?

                \vismParagraph{XVII.23}{23}{}
                \emph{The second}: the word “origination” (\emph{samuppāda}) indicates the arising of the states, since these occur when their conditions combine, and it shows how to prevent annihilationism, etc., thus preventing the various doctrines of annihilation [of a soul], nihilism, [“there is no use in giving,” etc.,] and moral-inefficacy-of-action, [“there is no other world,” etc.]; for when states [are seen to] arise again and again, each conditioned by its predecessor, how can the doctrines of annihilationism, nihilism, and moral-inefficacy-of-action be maintained?

                \vismParagraph{XVII.24}{24}{}
                \emph{The two together}: since any given states are produced without interrupting the [cause-fruit] continuity of any given combination of conditions, the whole expression “dependent origination” (\emph{paṭicca-samuppāda}) represents the middle way, which rejects the doctrines, “He who acts is he who reaps” and “One acts while another reaps” (\textbf{\cite{S}II 20}), and which is the proper way described thus, “Not insisting on local language and not overriding normal usage” (\textbf{\cite{M}III 234}).\footnote{\vismAssertFootnoteCounter{4}\vismHypertarget{XVII.n4}{}“Such terms as ‘woman,’ ‘man,’ etc., are \emph{local forms of speech }(\emph{janapada-nirutti}) because even wise men, instead of saying, ‘Fetch the five aggregates,’ or ‘Let the mentality-materiality come,’ use the current forms ‘woman’ and ‘man.’ This is how, in those who have not fully understood what a physical basis is, there comes to be the insistence (misinterpretation), ‘This is really a woman, this is really a man.’ But since this is a mere concept, which depends on states made to occur in such and such wise, one who sees and knows the dependent origination does not insist on (misinterpret) it as the ultimate meaning. ‘\emph{Current speech’ }is speech current in the world. ‘\emph{Not overriding}’ is not going beyond. For when ‘a being’ is said, instead of making an analysis like this, ‘What is the [lasting] being here? Is it materiality? Or feeling?’ and so on, one who does not override current usage should express a worldly meaning in ordinary language as those in the world do, employing the usage current in the world” (\textbf{\cite{Vism-mhṭ}557–558}). The explanation differs somewhat from MN 139.}

                \marginnote{\textcolor{teal}{\footnotesize\{597|539\}}}{}This, in the first place, is the meaning of the mere words “dependent origination” (\emph{paṭicca-samuppāda}).
            \section[\vismAlignedParas{§25–272}B. Exposition]{B. Exposition}
                \subsection[\vismAlignedParas{§25–26}I. Preamble]{I. Preamble}

                    \vismParagraph{XVII.25}{25}{}
                    Now, in teaching this dependent origination the Blessed One has set forth the text in the way beginning, “With ignorance as condition there are formations” (\textbf{\cite{S}II 20}). Its meaning should be commented on by one who keeps within the circle of the Vibhajjavādins,\footnote{\vismAssertFootnoteCounter{5}\vismHypertarget{XVII.n5}{}The term “analyzer” (\emph{vibhajjavādin}) appears at \textbf{\cite{A}V 190}, and at \textbf{\cite{M}II 197}, in this sense, used to describe the Buddha and his followers, who do not rashly give unqualified answers to questions that need analyzing before being answered.} who does not misrepresent the teachers, who does not advertise his own standpoint, who does not quarrel with the standpoint of others, who does not distort suttas, who is in agreement with the Vinaya, who looks to the principal authorities (\emph{mahāpadesa—}\textbf{\cite{D}II 123ff.}), who illustrates the law (\emph{dhamma}), who takes up the meaning (\emph{attha}), repeatedly reverting to that same meaning, describing it in various different ways.\footnote{\vismAssertFootnoteCounter{6}\vismHypertarget{XVII.n6}{}“The ‘law’ (\emph{dhamma}) is the text of the dependent origination. The “meaning” (\emph{attha}) is the meaning of that. Or they are the cause, and the fruit of the cause here, is what is meant. Or “law” (\emph{dhamma}) is regularity (\emph{dhammatā}). Now some, misinterpreting the meaning of the sutta passage, ‘Whether Perfect Ones arise or do not arise, there yet remains that element …’ (\textbf{\cite{S}II 25}), wrongly describe the regularity of the dependent origination as a ‘permanent dependent origination,’ instead of which it should be described as having the individual essence of a cause (\emph{kāraṇa}), defined according to its own fruit, in the way stated. And some misinterpret the meaning of the dependent origination thus, ‘Without cessation, without arising’ (\emph{anuppādaṃ anirodhaṃ}) instead of taking the unequivocal meaning in the way stated” (\textbf{\cite{Vism-mhṭ}561}). The last-mentioned quotation “Without cessation, without arising” (\emph{anuppādaṃ anirodhaṃ}), seems almost certainly to refer to a well-known stanza in Nāgārjuna’s \emph{Mūlamādhyamika Kārikā}:

                            \emph{Anirodhaṃ anutpādaṃ anucchedaṃ aśāsrataṃ Anekarthaṃ anānarthaṃ anāgamaṃ anirgamaṃ Yaḥ pratītyasamutpādaṃ prapañcopaśamaṃ śivaṃ Deśayamāsa sambuddhas taṃ vande vadatāṃ varaṃ.}} And it is inherently difficult to comment on the dependent origination, as the Ancients said:
                    \begin{verse}
                        \marginnote{\textcolor{teal}{\footnotesize\{598|540\}}}{}The truth, a being, rebirth-linking,\\{}
                        And the structure of conditions,\\{}
                        Are four things very hard to see\\{}
                        And likewise difficult to teach.
                    \end{verse}


                    Therefore, considering that to comment on the dependent origination is impossible except for those who are expert in the texts:
                    \begin{verse}
                        Whilst I would now begin the comment\\{}
                        On the structure of conditions\\{}
                        I find no footing for support\\{}
                        And seem to founder in a sea. \textcolor{brown}{\textit{[523]}}
                    \end{verse}

                    \begin{verse}
                        However, many modes of teaching\\{}
                        Grace the Dispensation here,\\{}
                        And still the former teachers’ way\\{}
                        Is handed down unbrokenly.
                    \end{verse}

                    \begin{verse}
                        Therefore on both of these relying\\{}
                        For my support, I now begin\\{}
                        Its meaning to elucidate:\\{}
                        Listen therefore attentively.
                    \end{verse}


                    \vismParagraph{XVII.26}{26}{}
                    For this has been said by the former teachers:
                    \begin{verse}
                        Whoever learns alertly this [discourse]\\{}
                        Will go from excellence to excellence,\\{}
                        And when perfected, he will then escape\\{}
                        Beyond the vision of the King of Death.
                    \end{verse}

                \subsection[\vismAlignedParas{§27–57}II. Brief Exposition]{II. Brief Exposition}

                    \vismParagraph{XVII.27}{27}{}
                    So as regards the passages that begin: “With ignorance as condition there are formations” (\textbf{\cite{S}II 20}), to start with:
                    \begin{verse}
                        (1) As different ways of teaching, (2) meaning,\\{}
                        (3) Character, (4) singlefold and so on,\\{}
                        (5) As to defining of the factors,\\{}
                        The exposition should be known.
                    \end{verse}


                    \vismParagraph{XVII.28}{28}{}
                    \emph{1.} Herein, as \emph{different ways of teaching}: the Blessed One’s teaching of the dependent origination is fourfold, namely, (i) from the beginning; or (ii) from the middle up to the end; and (iii) from the end; or (iv) from the middle down to the beginning. It is like four creeper-gatherers’ ways of seizing a creeper.

                    \vismParagraph{XVII.29}{29}{}
                    (i) For just as one of four men gathering creepers sees only the root of the creeper first, and after cutting it at the root, he pulls it all out and takes it away and uses it, so the Blessed One teaches the dependent origination from the beginning up to the end thus: “So, bhikkhus, with ignorance as condition there are formations; … with birth as condition ageing-and-death” (\textbf{\cite{M}I 261}).

                    \vismParagraph{XVII.30}{30}{}
                    (ii) Just as another of the four men sees the middle of the creeper first, and after cutting it in the middle, he pulls out only the upper part and takes it away and uses it, so the Blessed One teaches it from the middle up to the end thus: \marginnote{\textcolor{teal}{\footnotesize\{599|541\}}}{}“When he is delighted with, welcomes, remains committed to that feeling, then delight arises in him. Delight in feelings is clinging. With his clinging as condition there is becoming; with becoming as condition, birth” (\textbf{\cite{M}I 266}).

                    \vismParagraph{XVII.31}{31}{}
                    (iii) Just as another of the four men sees the tip of the creeper first, and seizing the tip, he follows it down to the root and takes all of it away and uses it, so the Blessed One teaches it from the end down to the beginning thus: “‘With birth as condition, ageing-and-death,’ so it was said. But is there ageing-and-death with birth as condition, or not, or how is it here?—There is ageing-and-death with birth as condition, so we think, venerable sir. \textcolor{brown}{\textit{[524]}} ‘With becoming as condition, birth,’ so it was said … ‘With ignorance as condition there are formations,’ so it was said. But are there formations with ignorance as condition, or not, or how is it here?—There are formations with ignorance as condition, so we think, venerable sir” (\textbf{\cite{M}I 261}).

                    \vismParagraph{XVII.32}{32}{}
                    (iv) Just as one of the four men sees only the middle of the creeper first, and after cutting it in the middle and tracing it down as far as the root, he takes it away and uses it, so the Blessed One teaches it from the middle down to the beginning thus: “And these four nutriments, bhikkhus: what is their source? What is their origin? From what are they born? By what are they produced? These four nutriments have craving as their source, craving as their origin, they are born from craving, produced by craving. Craving: what is its source? … Feeling: … Contact: … The sixfold base: … Mentality-materiality: … Consciousness: … Formations: what is their source? … By what are they produced? Formations have ignorance as their source … they are … produced by ignorance” (\textbf{\cite{S}II 11f.}).

                    \vismParagraph{XVII.33}{33}{}
                    Why does he teach it thus? Because the dependent origination is wholly beneficial and because he has himself acquired elegance in instructing. For the dependent origination is entirely beneficial: starting from any one of the four starting points, it leads only to the penetration of the proper way. And the Blessed One has acquired elegance in instructing: it is because he has done so through possession of the four kinds of perfect confidence and the four discriminations and by achieving the fourfold profundity (\hyperlink{XVII.304}{§304}{}) that he teaches the Dhamma by various methods.

                    \vismParagraph{XVII.34}{34}{}
                    But it should be recognized, in particular, that (i) when he sees that people susceptible of teaching are confused about the analysis of the causes of the process [of becoming], he employs his teaching of it forwards starting from the beginning in order to show that the process carries on according to its own peculiar laws and for the purpose of showing the order of arising. (iii) And it should be recognized that when he surveys the world as fallen upon trouble in the way stated thus, “This world has fallen upon trouble; it is born, ages, dies, passes away, and reappears” (\textbf{\cite{S}II 10}), he employs his teaching of it backwards starting from the end in order to show the [laws governing the] various kinds of suffering beginning with ageing and death, which he discovered himself in the early stage of his penetration. And (iv) it should be recognized that he employs his teaching of it backwards from the middle down to the beginning in order to show how the succession of cause and fruit extends back into the past [existence], \marginnote{\textcolor{teal}{\footnotesize\{600|542\}}}{}and again forwards from the past, in accordance with his definition of nutriment as the source [of ignorance] (see \textbf{\cite{M}I 47f.}). And (ii) it should be recognized that he employs his teaching of it forwards from the middle up to the end in order to show how the future [existence] follows on [through rebirth] from arousing in the present causes for [rebirth] in the future.

                    \vismParagraph{XVII.35}{35}{}
                    Of these methods of presentation, that cited here should be understood to be that stated in forward order starting from the beginning in order to show to people susceptible of teaching who are confused about the laws of the process [of becoming] that the process carries on according to its own peculiar laws, \textcolor{brown}{\textit{[525]}} and for the purpose of showing the order of arising.

                    \vismParagraph{XVII.36}{36}{}
                    But why is ignorance stated as the beginning here? How then, is ignorance the causeless root-cause of the world like the Primordial Essence of those who assert the existence of a Primordial Essence? It is not causeless. For a cause of ignorance is stated thus, “With the arising of cankers there is the arising of ignorance” (\textbf{\cite{M}I 54}). But there is a figurative way in which it can be treated as the root cause. What way is that? When it is made to serve as a starting point in an exposition of the round [of becoming].

                    \vismParagraph{XVII.37}{37}{}
                    For the Blessed One gives the exposition of the round with one of two things as the starting point: either ignorance, according as it is said, “No first beginning of ignorance is made known, bhikkhus, before which there was no ignorance, and after which there came to be ignorance. And while it is said thus, bhikkhus, nevertheless it is made known that ignorance has its specific condition” (\textbf{\cite{A}V 113}); or craving for becoming, according as it is said, “No first beginning of craving for becoming is made known, bhikkhus, before which there was no craving for becoming, and after which there came to be craving for becoming. And while it is said thus, bhikkhus, nevertheless it is made known that craving for becoming has its specific condition” (\textbf{\cite{A}V 116}).

                    \vismParagraph{XVII.38}{38}{}
                    But why does the Blessed One give the exposition of the round with those two things as starting points? Because they are the outstanding causes of kamma that leads to happy and unhappy destinies.

                    \vismParagraph{XVII.39}{39}{}
                    Ignorance is an outstanding cause of kamma that leads to unhappy destinies. Why? Because, just as when a cow to be slaughtered is in the grip of the torment of burning with fire and belabouring with cudgels, and being crazed with torment, she drinks the hot water although it gives no satisfaction and does her harm, so the ordinary man who is in the grip of ignorance performs kamma of the various kinds beginning with killing living things that leads to unhappy destinies, although it gives no satisfaction because of the burning of defilements and does him harm because it casts him into an unhappy destiny.

                    \vismParagraph{XVII.40}{40}{}
                    But craving for becoming is an outstanding cause of kamma that leads to happy destinies. Why? Because, just as that same cow, through her craving for cold water, starts drinking cold water, which gives satisfaction and allays her torment, so the ordinary man in the grip of craving for becoming performs kamma of the various kinds beginning with abstention from killing living things that leads to happy destinies and gives satisfaction because it is free from the \marginnote{\textcolor{teal}{\footnotesize\{601|543\}}}{}burning of defilements and, by bringing him to a happy destiny, allays the torment of suffering [experienced] in the unhappy destinies.

                    \vismParagraph{XVII.41}{41}{}
                    Now, as regards these two states that are starting points in expositions of the process [of becoming], in some instances the Blessed One teaches the Dhamma based on a single one of these states, for instance, \textcolor{brown}{\textit{[526]}} “Accordingly, bhikkhus, formations have ignorance as their cause, consciousness has formations as its cause” (\textbf{\cite{S}II 31}), etc.; likewise, “Bhikkhus, craving increases in one who dwells seeing enjoyment in things productive of clinging; with craving as condition there is clinging” (\textbf{\cite{S}II 84}), and so on. In some instances he does so based on both, for instance: “So, bhikkhus, for the fool who is hindered by ignorance and tethered by craving there arises this body. Now, this body [with its six internal bases] and externally [the six bases due to] mentality-materiality make a duality. Due to this duality there is contact, as well as the six [pairs of] bases, touched through which the fool feels pleasure and pain” (\textbf{\cite{S}II 23f.}), and so on.

                    \vismParagraph{XVII.42}{42}{}
                    Of these ways of presentation, that cited here in the form “With ignorance as condition there are formations” should be understood as one based on a single state. This, firstly, is how the exposition should be known “as to different ways of teaching.”

                    \vismParagraph{XVII.43}{43}{}
                    \emph{2. }As to meaning: as to the meaning of the words “ignorance” and so on. Bodily misconduct, etc., for example, “ought not to be found” (\emph{avindiya}), in the sense of being unfit to be carried out; the meaning is that it should not be permitted. It finds (\emph{vindati}) what ought not to be found (\emph{avindiya}), thus it is ignorance (\emph{avijjā}). Conversely, good bodily conduct, etc. “ought to be found” (\emph{vindiya}). It does not find (\emph{na vindati}) what ought to be found (\emph{vindiya}), thus it is ignorance (\emph{avijjā}). Also it prevents knowing (\emph{avidita}) the meaning of collection in the aggregates, the meaning of actuating in the bases, the meaning of voidness in the elements, the meaning of predominance in the faculties, the meaning of reality in the truths, thus it is ignorance (\emph{avijjā}). Also it prevents knowing the meaning of suffering, etc., described in four ways as “oppression,” etc. (\hyperlink{XVI.15}{XVI.15}{}), thus it is ignorance. Through all the kinds of generations, destinies, becoming, stations of consciousness, and abodes of beings in the endless round of rebirths it drives beings on (\emph{AntaVIrahite saṃsāre … satte JAvāpeti}), thus it is ignorance (\emph{avijjā}). Amongst women, men, etc., which are in the ultimate sense non-existent, it hurries on (\emph{paramatthato AVIJjamānesu itthi-purisādisu JAvati}), and amongst the aggregates, etc., which are existent, it does not hurry on (\emph{vijjamānesu pi khandhādisu na javati}), thus it is ignorance (\emph{avijjā}). Furthermore, it is ignorance because it conceals the physical bases and objects of eye-consciousness, etc., and the dependent origination and dependently-originated states.

                    \vismParagraph{XVII.44}{44}{}
                    That due to (\emph{paṭicca}) which fruit comes (\emph{eti}) is a condition (\emph{paccaya}). “Due to” (\emph{paṭicca}) = “not without that”; the meaning is, not dispensing with it. “Comes” (\emph{eti}) means both “arises” and “occurs.” Furthermore, the meaning of “condition” is the meaning of “help.” It is ignorance and that is a condition, thus it is “ignorance as condition,” whence the phrase “with ignorance as condition.” \marginnote{\textcolor{teal}{\footnotesize\{602|544\}}}{}“They form the formed” (\textbf{\cite{S}III 87}), thus they are formations. Furthermore, formations are twofold, namely, (a) formations with ignorance as condition, and (b) formations given in the texts with the word “formations” (\emph{saṅkhāra}). Herein, (a) the three, namely, formations of merit, of demerit, and of the imperturbable, and the three, namely, the bodily, the verbal, and the mental formations, which make six, are “formations with ignorance as condition.” And all these are simply mundane profitable and unprofitable volition.

                    \vismParagraph{XVII.45}{45}{}
                    But (b) these four, namely, (i) the formation consisting of the formed (\emph{saṅkhata-saṅkhāra}), \textcolor{brown}{\textit{[527]}} (ii) the formation consisting of the kamma-formed (\emph{abhisaṅkhata-saṅkhāra}), (iii) the formation consisting in the act of kamma-forming (forming by kamma—\emph{abhisaṅkharaṇa-saṅkhāra}), and (iv) the formation consisting in momentum (\emph{payogābhisaṅkhāra}), are the kinds of formations that have come in the texts with the word “formations.”

                    \vismParagraph{XVII.46}{46}{}
                    Herein, (i) all states with conditions, given in such passages as “Formations are impermanent” (\textbf{\cite{S}I 158}; \textbf{\cite{D}II 157}), are formations consisting of the formed. (ii) In the Commentaries material and immaterial states of the three planes generated by kamma are called formations consisting of the kamma-formed. These are also included in the passage, “Formations are impermanent.” But there is no instance in the texts where they are found separately. (iii) Profitable and unprofitable volition of the three planes is called the formation consisting in the act of kamma-forming. It is found in the texts in such passages as “Bhikkhus, this man in his ignorance forms the formation of merit” (\textbf{\cite{S}II 82}). (iv) But it is bodily and mental energy that is called the formation consisting in momentum. This is given in the texts in such passages as “The wheel, having gone as far as the impetus (\emph{abhisaṅkhāra}) carried it, stood as though it were fixed” (\textbf{\cite{A}I 112}).

                    \vismParagraph{XVII.47}{47}{}
                    And not only these, but many other kinds of formations are given in the texts with the word “formation” (\emph{saṅkhāra}), in the way beginning, “When a bhikkhu is attaining the cessation of perception and feeling, friend Visākha, first his verbal formation ceases, then his bodily formation, then his mental formation” (\textbf{\cite{M}I 302}). But there is no formation among them not included by (i) “formations consisting of the formed.”

                    \vismParagraph{XVII.48}{48}{}
                    What is said next after this in the [rest of the exposition] beginning, “With formations as condition, consciousness” should be understood in the way already stated. But as to those words not yet dealt with: It cognizes (\emph{vijānāti}), thus it is consciousness (\emph{viññāṇa—}see \textbf{\cite{M}I 292}). It bends [towards an object] (\emph{namati}), thus it is mentality (\emph{nāma}). It is molested (\emph{ruppati}), thus it is materiality (\emph{rūpa—}see \textbf{\cite{S}III 87}). It provides a range for the origins (\emph{āye tanoti}) and it leads on what is actuated (\emph{āyatañ ca nayati}), thus it is a base (\emph{āyatana—}see \hyperlink{XV.4}{XV.4}{}). It touches (\emph{phusati}), thus it is contact (\emph{phassa}). It is felt (\emph{vedayati}), thus it is feeling (\emph{vedanā—}see \textbf{\cite{M}I 293}). It frets (or it thirsts—\emph{paritassati}), thus it is craving (\emph{taṇhā}). It clings (\emph{upādiyati}), thus it is clinging (\emph{upādāna}). It becomes (\emph{bhavati}) and it makes become (\emph{bhāvayati}), thus it is becoming (\emph{bhava}). The act of being born is birth. The act of growing old is ageing. By means of it they die, thus it is death. The act of sorrowing is sorrow. The act of lamenting is lamentation. It makes [beings] suffer (\emph{dukkhayati}), thus it is pain (\emph{dukkha}); or it consumes in two ways (\emph{DVedhā }\marginnote{\textcolor{teal}{\footnotesize\{603|545\}}}{}\emph{KHAṇati—}see \hyperlink{IV.100}{IV.100}{}) by means of [the two moments (\emph{khaṇa})] arising and presence, thus it is pain (\emph{dukkha}). The state of a sad mind (\emph{dummana-bhāva}) is grief (\emph{domanassa}). Great misery (\emph{bhuso āyāso}) is despair (\emph{upāyāsa}). \emph{There is} means “is generated.”

                    \vismParagraph{XVII.49}{49}{}
                    And the words “There is” should be construed with all the terms, not only with those beginning with sorrow; for otherwise, when “With ignorance as condition, formations” was said, it would not be evident what they did, but by construing it with the words “There is” (or “there are”), since “ignorance as condition” stands for “it is ignorance and that is a condition,” consequently \textcolor{brown}{\textit{[528]}} the defining of the condition and the conditionally-arisen state is effected by the words “with ignorance as condition there are formations.” And so in each instance.

                    \vismParagraph{XVII.50}{50}{}
                    Thus signifies the process described. By that he shows that it is with ignorance, etc., as the causes and not with creation by an Overlord, and so on. Of that: of that aforesaid. Whole: unmixed, entire. \emph{Mass of suffering}: totality of suffering; not a living being, not pleasure, beauty, and so on. \emph{Arising}: generating. \emph{There is}: is brought about.

                    This is how the exposition should be known here “as to meaning.”

                    \vismParagraph{XVII.51}{51}{}
                    \emph{3.} As to \emph{character}, etc.: as to the characteristics of ignorance, etc., that is to say, \emph{ignorance} has the characteristic of unknowing. Its function is to confuse. It is manifested as concealing. Its proximate cause is cankers. \emph{Formations} have the characteristic of forming. Their function is to accumulate.\footnote{\vismAssertFootnoteCounter{7}\vismHypertarget{XVII.n7}{}“Formations ‘\emph{accumulate},’ work, for the purpose of rebirth. So that is their function. To accumulate is to heap up. Consciousness’s function is ‘\emph{to go before’ }since it precedes mentality-materiality at rebirth-linking. Mentality’s function is ‘\emph{to associate’ }since it joins with consciousness in a state of mutuality. ‘\emph{Inseparability of its components’ }is owing to their having no separate existence [mentality here being feeling, perception, and formations]. Materiality is dispersible since it has in itself nothing [beyond the water element] to hold it [absolutely] together, so ‘\emph{its function is to be dispersed’; }that is why, when rice grains, etc., are pounded, they get scattered and reduced to powder. It is called ‘\emph{indeterminate}’ to distinguish it from mentality, which is profitable, etc., at different times” (\textbf{\cite{Vism-mhṭ}571}).} They are manifested as volition. Their proximate cause is ignorance. \emph{Consciousness} has the characteristic of cognizing. Its function is to go before (see \textbf{\cite{Dhp}1}). It manifests itself as rebirth-linking. Its proximate cause is formations; or its proximate cause is the physical-basis-cum-object. \emph{Mentality }(\emph{nāma}) has the characteristic of bending (\emph{namana}). Its function is to associate. It is manifested as inseparability of its components, [that is, the three aggregates]. Its proximate cause is consciousness. \emph{Materiality} (\emph{rūpa}) has the characteristic of being molested (\emph{ruppana}). Its function is to be dispersed. It is manifested as [morally] indeterminate. Its proximate cause is consciousness. The sixfold base (\emph{saḷāyatana}) has the characteristic of actuating (\emph{āyatana}). Its function is to see, and so on. It is manifested as the state of physical basis and door. Its proximate cause is mentality-materiality. \emph{Contact} has the characteristic of touching. Its function is impingement. It manifests itself as coincidence [of internal and external base \marginnote{\textcolor{teal}{\footnotesize\{604|546\}}}{}and consciousness]. Its proximate cause is the sixfold base. \emph{Feeling} has the characteristic of experiencing. Its function is to exploit the stimulus of the objective field. It is manifested as pleasure and pain. Its proximate cause is contact. \emph{Craving }has the characteristic of being a cause [that is, of suffering]. Its function is to delight. It is manifested as insatiability. Its proximate cause is feeling. \emph{Clinging }has the characteristic of seizing. Its function is not to release. It is manifested as a strong form of craving and as [false] view. Its proximate cause is craving. \emph{Becoming} has the characteristic of being kamma and kamma-result. Its function is to make become and to become. It is manifested as profitable, unprofitable, and indeterminate. Its proximate cause is clinging. The characteristic of \emph{birth}, etc., should be understood as stated in the Description of the Truths (\hyperlink{XVI.32}{XVI.32f.}{}). This is how the exposition should be known here “as to character, etc.”

                    \vismParagraph{XVII.52}{52}{}
                    \emph{4. As to singlefold, and so on}: here \emph{ignorance} is singlefold as unknowing, unseeing, delusion, and so on. It is twofold as “no theory” and “wrong theory” (cf. \hyperlink{XVII.303}{§303}{});\footnote{\vismAssertFootnoteCounter{8}\vismHypertarget{XVII.n8}{}“‘\emph{No theory’ }is unknowing about suffering, etc., ‘\emph{wrong theory’ }is perverted perception of what is foul, etc., as beautiful, etc., or else ‘\emph{no theory’ }is unassociated with [false] view, and ‘\emph{wrong theory’ }is associated with it” (\textbf{\cite{Vism-mhṭ}751}). This use of the word \emph{paṭipatti }as “theory,” rare in Pali but found in Sanskrit, is not in PED. An alternative rendering for these two terms might be “agnosticism” and “superstition” (see also \hyperlink{XIV.163}{XIV.163}{}, \hyperlink{XIV.177}{177}{}).} likewise as prompted and unprompted. It is threefold as associated with the three kinds of feeling. It is fourfold as non-penetration of the four truths. It is fivefold as concealing the danger in the five kinds of destinies. \textcolor{brown}{\textit{[529]}} It should, however, be understood that all the immaterial factors [of the dependent origination] have a sixfold nature with respect to the [six] doors and objects.

                    \vismParagraph{XVII.53}{53}{}
                    \emph{Formations} are singlefold as states subject to cankers (\textbf{\cite{Dhs}3}), states with the nature of result (\textbf{\cite{Dhs}1}), and so on (cf. \textbf{\cite{Vibh}62}).\footnote{\vismAssertFootnoteCounter{9}\vismHypertarget{XVII.n9}{}“‘\emph{With the nature of result, and so on}’: the words ‘and so on’ here include ‘neither-trainer-nor-non-trainer,’ (\textbf{\cite{Dhs}2}) ‘conducive to fetters’ (\textbf{\cite{Dhs}3}), and so on. [§54] ‘\emph{Mundane resultant and so on}’: the words ‘and so on’ here include ‘indeterminate’ (\textbf{\cite{Dhs}2}), ‘formed’ (\textbf{\cite{Dhs}2}), and so on. ‘\emph{With root-cause and without root-cause, and so on}’: the words ‘and so on’ here include ‘prompted,’ ‘unprompted,’ and so on” (\textbf{\cite{Vism-mhṭ}}).} They are twofold as profitable and unprofitable; likewise as limited and exalted, inferior and medium, with certainty of wrongness and without certainty. They are threefold as the formation of merit and the rest. They are fourfold as leading to the four kinds of generation. They are fivefold as leading to the five kinds of destiny.

                    \vismParagraph{XVII.54}{54}{}
                    \emph{Consciousness} is singlefold as mundane (\textbf{\cite{Dhs}3}), resultant (\textbf{\cite{Dhs}1}), and so on. It is twofold as with root-cause and without root-cause and so on. It is threefold as included in the three kinds of becoming; as associated with the three kinds of feeling; and as having no root-cause, having two root-causes, and having three root-causes. It is fourfold and fivefold [respectively] according to generation and destiny.

                    \vismParagraph{XVII.55}{55}{}
                    \emph{Mentality-materiality} is singlefold as dependent on consciousness, and as having kamma as its condition. It is twofold as having an object [in the case of mentality], and having no object [in the case of materiality]. It is threefold as \marginnote{\textcolor{teal}{\footnotesize\{605|547\}}}{}past, and so on. It is fourfold and fivefold respectively according to generation and destiny.

                    \vismParagraph{XVII.56}{56}{}
                    \emph{The sixfold base} is singlefold as the place of origin and meeting. It is twofold as sensitivity of primary elements and as consciousness [of the sixth base], and so on. It is threefold as having for its domain [objective fields that are] contiguous, non-contiguous, and neither (see \hyperlink{XIV.46}{XIV.46}{}). It is fourfold and fivefold respectively as included in the kinds of generation and destiny.

                    The singlefoldness, etc., of \emph{contact}, etc., should be understood in this way too.

                    This is how the exposition should be known here “as to singlefold and so on.”

                    \vismParagraph{XVII.57}{57}{}
                    \emph{5. As to defining of the factors}: sorrow, etc., are stated here for the purpose of showing that the Wheel of Becoming never halts; for they are produced in the fool who is afflicted by ageing and death, according as it is said: “The untaught ordinary man, bhikkhus, on being touched by painful bodily feeling, sorrows, grieves and laments, beating his breast, he weeps and becomes distraught” (\textbf{\cite{M}III 285}; \textbf{\cite{S}IV 206}). And as long as these go on occurring so long does ignorance, and so the Wheel of Becoming renews [its revolution]: “With ignorance as condition there are formations” and so on. That is why the factors of the dependent origination should be understood as twelve by taking those [that is, sorrow, etc.,] along with ageing-and-death as one summarization. This is how the exposition should be known here “as to defining of the factors.”
                \subsection[\vismAlignedParas{§58–272}III. Detailed Exposition]{III. Detailed Exposition}

                    \vismParagraph{XVII.58}{58}{}
                    This, firstly, is the brief treatment. The following method, however, is in detail.
                    \subsubsection[\vismAlignedParas{§58–59}(1) Ignorance]{(1) Ignorance}

                        According to the Suttanta method \textcolor{brown}{\textit{[530]}} ignorance is unknowing about the four instances beginning with suffering. According to the Abhidhamma method it is unknowing about the eight instances [that is to say, the above-mentioned four] together with [the four] beginning with the past; for this is said: “Herein, what is ignorance? It is unknowing about suffering, [unknowing about the origin of suffering, unknowing about the cessation of suffering, unknowing about the way leading to the cessation of suffering], unknowing about the past, unknowing about the future, unknowing about the past and future, unknowing about specific conditionality and conditionally-arisen states” (cf. \textbf{\cite{Dhs}§1162}).

                        \vismParagraph{XVII.59}{59}{}
                        Herein, while ignorance about any instance that is not the two supra-mundane truths can also arise as object (see \hyperlink{XVII.102}{§102}{}), nevertheless here it is only intended [subjectively] as concealment. For when [thus] arisen it keeps the truth of \emph{suffering} concealed, preventing penetration of the true individual function and characteristic of that truth. Likewise, origin, cessation, and the path, bygone five aggregates called \emph{the past}, coming five aggregates called \emph{the future}, both of these together called \emph{the past and future}, and both specific conditionality and conditionally-arisen states together called \emph{specific conditionality and conditionally-arisen states—}all of which it keeps concealed, preventing their true individual \marginnote{\textcolor{teal}{\footnotesize\{606|548\}}}{}functions and characteristics being penetrated thus: “This is ignorance, these are formations.” That is why it is said, “It is unknowing about suffering … unknowing about specific conditionality and conditionally-arisen states.”
                    \subsubsection[\vismAlignedParas{§60–119}(2) Formations]{(2) Formations}

                        \vismParagraph{XVII.60}{60}{}
                        \emph{Formations} are the six mentioned in brief above thus, “the three, namely, formations of merit, etc., and the three, namely, the bodily formation, etc.” (\hyperlink{XVII.44}{§44}{}); but in detail here the [first] three formations are twenty-nine volitions, that is to say, the formation of merit consisting of thirteen volitions, counting the eight sense-sphere profitable volitions that occur in giving, in virtue, etc., and the five fine-material profitable volitions that occur in development [of meditation]; then the formation of demerit consisting of the twelve unprofitable volitions that occur in killing living things, etc.; then the formation of the imperturbable consisting in the four profitable volitions associated with the immaterial sphere, which occur in development [of those meditations].

                        \vismParagraph{XVII.61}{61}{}
                        As regards the other three, the bodily formation is bodily volition, the verbal formation is verbal volition, and the mental formation is mental volition. This triad is mentioned in order to show that at the moment of the accumulation of the kamma the formations of merit, etc., occur in these [three] kamma doors. For the eight sense-sphere profitable and twelve unprofitable volitions, making twenty, are the bodily formation when they occur in the body door and produce bodily intimation. Those same volitions \textcolor{brown}{\textit{[531]}} are called the verbal formation when they occur in the speech door and produce verbal intimation. But volition connected with direct-knowledge is not included here in these two cases because it is not a condition for [resultant rebirth-linking] consciousness later. And like direct-knowledge volition, so also volition connected with agitation is not included; therefore that too should not be included as a condition for [rebirth-linking] consciousness. However, all these have ignorance as their condition. And all the twenty-nine volitions are the mental formation when they arise in the mind door without originating either kind of intimation. So this triad comes within the first triad, and accordingly, as far as the meaning is concerned, ignorance can be understood as condition simply for formations of merit and so on.

                        \vismParagraph{XVII.62}{62}{}
                        Herein, it might be [asked]: How can it be known that these formations have ignorance as their condition?—By the fact that they exist when ignorance exists. For when unknowing—in other words, ignorance—of suffering, etc., is unabandoned in a man, owing firstly to his unknowing about suffering and about the past, etc., then he believes the suffering of the round of rebirths to be pleasant and he embarks upon the three kinds of formations which are the cause of that very suffering. Owing to his unknowing about suffering’s origin he embarks upon formations that, being subordinated to craving, are actually the cause of suffering, imagining them to be the cause of pleasure. And owing to his unknowing about cessation and the path, he misperceives the cessation of suffering to be in some particular destiny [such as the Brahmā-world] that is not in fact cessation; he misperceives the path to cessation, believing it to consist in sacrifices, mortification for immortality, etc., which are not in fact the path to cessation; and so while aspiring to the cessation of suffering, he embarks upon \marginnote{\textcolor{teal}{\footnotesize\{607|549\}}}{}the three kinds of formations in the form of sacrifices, mortification for immortality, and so on.

                        \vismParagraph{XVII.63}{63}{}
                        Furthermore, his non-abandonment of that ignorance about the four truths in particular prevents him from recognizing as suffering the kind of suffering called the fruit of merit, which is fraught with the many dangers beginning with birth, ageing, disease and death, and so he embarks upon the formation of merit classed as bodily, verbal, and mental formations, in order to attain that [kind of suffering], like one desiring celestial nymphs [who jumps over] a cliff. Also, not seeing how that fruit of merit reckoned as pleasure eventually breeds great distress owing to the suffering in its change and that it gives little satisfaction, he embarks upon the formation of merit of the kinds already stated, which is the condition for that very [suffering in change], like a moth that falls into a lamp’s flame, and like the man who wants the drop of honey and licks the honey-smeared knife-edge. Also, not seeing the danger in the indulgence of sense desires, etc., with its results, [wrongly] perceiving pleasure and overcome by defilements, he embarks upon the formation of demerit that occurs in the three doors [of kamma], like a child who plays with filth, and like a man who wants to die and eats poison. Also, unaware of the suffering due to formations and the suffering-in-change [inherent] in kamma-results in the immaterial sphere, owing to the perversion of [wrongly perceiving them as] eternal, etc., he embarks upon the formation of the imperturbable which is a mental formation, like one who has lost his way and takes the road to a goblin city.

                        \vismParagraph{XVII.64}{64}{}
                        So formations exist only when ignorance exists, \textcolor{brown}{\textit{[532]}} not when it does not; and that is how it can be known that these formations have ignorance as their condition.

                        This is said too: “Not knowing, bhikkhus, in ignorance, he forms the formation of merit, forms the formation of demerit, forms the formation of the imperturbable. As soon as a bhikkhu’s ignorance is abandoned and clear vision arisen, bhikkhus, with the fading away of ignorance and the arising of clear vision he does not form even formations of merit” (cf. \textbf{\cite{S}II 82}).

                        \vismParagraph{XVII.65}{65}{}
                        Here it might be said: “Let us then firstly agree that ignorance is a condition for formations. But it must now be stated for which formations, and in which way it is a condition.”

                        Here is the reply: “Twenty-four conditions have been stated by the Blessed One as follows.”
                        \par\noindent[\textsc{\textbf{The 24 conditions}}]

                            \vismParagraph{XVII.66}{66}{}
                            “(1) Root-cause condition, (2) object condition, (3) predominance condition, (4) proximity condition, (5) contiguity condition, (6) conascence condition, (7) mutuality condition, (8) support condition, (9) decisive-support condition, (10) prenascence condition, (11) postnascence condition, (12) repetition condition, (13) kamma condition, (14) kamma-result condition, (15) nutriment condition, (16) faculty condition, (17) jhāna condition, (18) path condition, (19) association condition, (20) dissociation condition, (21) presence condition, (22) absence \marginnote{\textcolor{teal}{\footnotesize\{608|550\}}}{}condition, (23) disappearance condition, (24) non-disappearance condition” (\textbf{\cite{Paṭṭh}I 1}).

                            \vismParagraph{XVII.67}{67}{}
                            (1) Herein, it is a root-cause and a condition, thus it is \emph{root-cause condition}. It is by its being a root-cause that it is a condition; what is meant is that it is a condition owing to its status as root-cause. The same method applies in the case of object condition and the rest.

                            Herein, “cause” (\emph{hetu}) is a term for a part of a syllogism, for a reason, and for a root. For with the words “proposition” (\emph{paṭiññā}), “cause” (\emph{hetu} = middle term), etc., in the world it is a member of a syllogism (\emph{vacanāvayava}) that is called a cause. But in the Dispensation, in such passages as “Those states that are produced from a cause” (\textbf{\cite{Vin}I 40}), it is a reason (\emph{kāraṇa}); and in such passages as “Three profitable [root-] causes, three unprofitable [root-]causes” (\textbf{\cite{Dhs}§1053}), it is a root (\emph{mūla}) that is called a cause. The last is intended here.

                            \vismParagraph{XVII.68}{68}{}
                            As to “condition” (\emph{paccaya}), the word-meaning here is this: It [the fruit] comes from that, depending thereon (\emph{paṭicca etasmā eti}), thus that is a condition; (\emph{paccaya}, see note 2) the meaning is, [a state] occurs by not dispensing with that. What is meant is: when a state is indispensable to another state’s presence or arising, the former is a condition for the latter. But as to characteristic, a condition has the characteristic of assisting; for any given state \textcolor{brown}{\textit{[533]}} that assists the presence or arising of a given state is called the latter’s condition. The words condition, cause, reason, source, originator, producer, etc., are one in meaning though different in the letter. So, since it is a cause in the sense of a root, and a condition in the sense of assistance, briefly a state that is assistantial in the sense of a root is a [root-]cause condition.

                            \vismParagraph{XVII.69}{69}{}
                            The intention of [some] teachers is that it establishes the profitable, etc., state in what is profitable, etc., as paddy seeds, etc., do for paddy, etc., and as the colour of gems, etc., do for the lustre of gems, and so on.\footnote{\vismAssertFootnoteCounter{10}\vismHypertarget{XVII.n10}{}“This refers to the teacher Revata” (\textbf{\cite{Vism-mhṭ}582}).} But if that is so, then [it follows that] the state of root-cause condition does not apply to the kinds of materiality originated by it, for it does not establish any profitableness, etc., in them. Nevertheless, it is a condition for them, for this is said: “Root-causes are a condition, as root-cause condition, for the states associated with a root-cause and for the kinds of materiality originated thereby” (\textbf{\cite{Paṭṭh}I 1}). Again, the indeterminateness of root-causeless consciousness is established without it. And the profitableness, etc., of those with root-cause is bound up with wise attention, etc., not with the associated root-causes. And if the profitableness, etc., resided in the associated root-causes as an individual essence, then either the non-greed bound up with the root-cause in the associated states would be only profitable or it would be only indeterminate; but since it can be both, profitableness, etc., in the root-causes must still be sought for, just as in the associated states [such as wise attention, and so on].

                            \vismParagraph{XVII.70}{70}{}
                            But when the root-causes’ sense of root is taken as establishing stableness, rather than as establishing profitableness, etc., there is no contradiction. For states that have obtained a root-cause condition are firm, like trees, and stable; \marginnote{\textcolor{teal}{\footnotesize\{609|551\}}}{}but those without root-cause are, like moss [with roots no bigger than] sesame seeds, etc., unstable. So an assistantial state may be understood as a root-cause condition, since it establishes stableness through being of assistance in the sense of a root.

                            \vismParagraph{XVII.71}{71}{}
                            (2) As to the others that follow, a state that assists by being an object is an \emph{object condition}. Now, there are no states that are not object conditions; for the passage beginning “The visible-data base [is a condition, as object condition,] for the eye-consciousness element” concludes thus: “When any states, as states of consciousness and consciousness-concomitants, arise contingent upon any states, these [latter] states are conditions, as object condition, for those [former] states” (\textbf{\cite{Paṭṭh}I 1}).\footnote{\vismAssertFootnoteCounter{11}\vismHypertarget{XVII.n11}{}“‘\emph{Which are contingent upon other such states}’: because it is said without distinction of all visible-data bases … and of all mental-data bases, there is consequently no dhamma (state) among the formed, unformed, and conceptual dhammas, classed as sixfold under visible data, etc., that does not become an object condition” (\textbf{\cite{Vism-mhṭ}584}).} For just as a weak man both gets up and stands by hanging on to (\emph{ālambitvā}) a stick or rope, so states of consciousness and consciousness-concomitants always arise and are present contingent upon visible data, etc., as their object (\emph{ārammaṇa = ālambana}). Therefore all states that are objects of consciousness and consciousness-concomitants should be understood as object condition. \textcolor{brown}{\textit{[534]}}

                            \vismParagraph{XVII.72}{72}{}
                            (3) A state that assists in the sense of being foremost is a \emph{predominance condition}. It is of two kinds as conascent and as object. Herein, because of the passage beginning “Predominance of zeal is a condition, as predominance condition, for states associated with zeal and for the kinds of materiality originated thereby” (\textbf{\cite{Paṭṭh}I 2}), it is the four states called zeal, [purity of] consciousness, energy, and inquiry, that should be understood as predominance condition; but not simultaneously, for when consciousness occurs with emphasis on zeal and putting zeal foremost, then it is zeal and not the others that is predominant. So with the rest. But the state, by giving importance to which, immaterial states occur, is their \emph{object-predominance}. Hence it is said: “When any states, as states of consciousness and consciousness-concomitants, arise by giving importance to any states, these [latter] states are a condition, as predominance condition, for those [former] states” (\textbf{\cite{Paṭṭh}I 2}).

                            \vismParagraph{XVII.73}{73}{}
                            (4), (5) A state that assists by being proximate is a \emph{proximity condition}. A state that assists by being contiguous is a \emph{contiguity condition}. The explanation of this pair of conditions is very diffuse, but substantially it is this:\footnote{\vismAssertFootnoteCounter{12}\vismHypertarget{XVII.n12}{}“Proximity and contiguity conditions are not stated in accordance with the distinction between making occur and giving opportunity, as the absence and disappearance conditions are: rather they are stated as the causes of the regular order of consciousness [in the cognitive series]” (\textbf{\cite{Vism-mhṭ}585}).} the regular order of consciousness begins thus, mind element is proximate (next) after eye-consciousness, mind-consciousness element is proximate (next) after mind element, and this is established only by each preceding consciousness, not otherwise; consequently, a state that is capable of arousing an appropriate kind \marginnote{\textcolor{teal}{\footnotesize\{610|552\}}}{}of consciousness proximate (next) to itself is a proximity condition. Hence it is said: “Proximity condition: eye-consciousness and the states associated therewith are a condition, as proximity condition, for mind element and for the states associated therewith” (\textbf{\cite{Paṭṭh}I 2}).

                            \vismParagraph{XVII.74}{74}{}
                            (5) Proximity condition is the same as contiguity condition. The difference here is only in the letter, there is none in the meaning; just as in the case of the words “growth” and “continuity” (\hyperlink{XIV.66}{XIV.66}{}), etc., and as in the case of the “terminology dyad,” “language dyad,” (\textbf{\cite{Dhs}§1306}) and so on.

                            \vismParagraph{XVII.75}{75}{}
                            The opinion of [certain] teachers\footnote{\vismAssertFootnoteCounter{13}\vismHypertarget{XVII.n13}{}“This refers to the Elder Revata too” (\textbf{\cite{Vism-mhṭ}586}).} is that proximity condition refers to proximity of aim (fruit) and contiguity condition refers to proximity of time. But that is contradicted by such statements as “The profitable [consciousness] belonging to the base consisting of neither perception nor non-perception in one who emerges from cessation is a condition, as contiguity condition, for fruition attainment [consciousness]” (\textbf{\cite{Paṭṭh}I 160}).

                            \vismParagraph{XVII.76}{76}{}
                            Now, they say in this context that “the ability of states to produce [their fruit] is not diminished, but the influence of meditative development prevents states from arising in proximity.” But that only establishes that there is no proximity of time; and we also say the same, namely, that there is no proximity of time there owing to the influence of development. \textcolor{brown}{\textit{[535]}} But since there is no proximity of time, the state of contiguity condition is therefore impossible [according to them] since their belief is that the contiguity condition depends on proximity of time (cf. \textbf{\cite{M-a}II 363}). Instead of adopting any such misinterpretation, the difference should be treated as residing in the letter only, not in the meaning. How? There is no interval (\emph{antara}) between them, thus they are proximate (\emph{anantara}); they are quite without interval because [even the distinction of] co-presence is lacking, thus they are contiguous (\emph{samanantara}).\footnote{\vismAssertFootnoteCounter{14}\vismHypertarget{XVII.n14}{}“The state \emph{of proximity condition }is the ability to cause arising proximately (without interval) because there is no interval between the cessation of the preceding and the arising of the subsequent. The state of \emph{contiguity condition }is the ability to cause arising by being quite proximate (without interval) through approaching, as it were, identity with itself owing to absence of any distinction that ‘This is below, above, or around that,’ which is because of lack of any such co-presence as in the case of the [components of the] material groups, and because of lack of any co-positionality of the condition and the conditionally arisen. And [in general], because of the uninterestedness of [all] states (\emph{dhamma}), when a given [state] has ceased, or is present, in a given mode, and [other] states (\emph{dhamma}) come to be possessed of that particular mode, it is that [state’s] mode that must be regarded as what is called ‘ability to cause arising’” (\textbf{\cite{Vism-mhṭ}586}).}

                            \vismParagraph{XVII.77}{77}{}
                            (6) A state that, while arising, assists [another state] by making it arise together with itself is a \emph{conascence condition}, as a lamp is for illumination. With the immaterial aggregates, etc., it is sixfold, according as it is said: “(i) The four immaterial aggregates are a condition, as conascence condition, for each other, (ii) the four great primaries are … for each other; (iii) at the moment of descent into the womb mentality and materiality are … for each other; (iv) states of consciousness and its concomitants are … for the kinds of materiality originated \marginnote{\textcolor{teal}{\footnotesize\{611|553\}}}{}by consciousness; (v) the great primaries are … for derived materiality; (vi) material states are sometimes [as at rebirth-linking] a condition, as conascence condition, and sometimes [as in the course of an existence] not a condition as conascence condition, for immaterial states” (\textbf{\cite{Paṭṭh}I 3}). This refers only to the heart-basis.

                            \vismParagraph{XVII.78}{78}{}
                            (7) A state that assists by means of mutual arousing and consolidating is a \emph{mutuality condition}, as the three sticks of a tripod give each other consolidating support. With the immaterial aggregates, etc., it is threefold, according as it is said: “The four immaterial aggregates are a condition, as mutuality condition, [for each other]; the four great primaries are a condition, as mutuality condition, [for each other]; at the moment of descent into the womb mentality and materiality are a condition, as mutuality condition, [for each other]” (\textbf{\cite{Paṭṭh}I 3}).

                            \vismParagraph{XVII.79}{79}{}
                            (8) A state that assists in the mode of foundation and in the mode of support is a \emph{support condition}, as the earth is for trees, as canvas is for paintings, and so on. It should be understood in the way stated for conascence thus: “The four immaterial aggregates are a condition, as support condition, for each other” (\textbf{\cite{Paṭṭh}I 3}), but the sixth instance has been set forth in this way here: “The eye base [is a condition, as support condition,] for the eye-consciousness element [and for the states associated therewith]; the ear base … the nose base … the tongue base … the body base is a condition, as support condition, for the body-consciousness element and for the states associated therewith; the materiality with which as their support the mind element and the mind-consciousness element occur is a condition, as support condition, for the mind element, for the mind-consciousness element, and for the states associated therewith” (\textbf{\cite{Paṭṭh}I 4}).

                            \vismParagraph{XVII.80}{80}{}
                            (9) \emph{Decisive-support condition}: firstly, here is the word-meaning: \textcolor{brown}{\textit{[536]}} it is treated as support, not dispensed with, by its own fruit because [its own fruit’s] existence is dependent on it, thus it is the support. But just as great misery is despair, so great support is decisive support. This is a term for a cogent reason. Consequently, a state that assists by being a cogent reason should be understood as a decisive-support condition.

                            It is threefold, namely, (a) object-decisive-support, (b) proximate-decisive-support, and (c) natural-decisive-support condition.

                            \vismParagraph{XVII.81}{81}{}
                            (a) Herein, firstly, \emph{object-decisive-support} condition is set forth without differentiating it from object-predominance in the way beginning: “Having given a gift, having undertaken the precepts of virtue, having done the duties of the Uposatha, a man gives that importance and reviews it; he gives importance to former things well done and reviews them. Having emerged from jhāna, he gives jhāna importance and reviews it. Trainers give importance to change-of-lineage and review it. They give importance to cleansing and review it.\footnote{\vismAssertFootnoteCounter{15}\vismHypertarget{XVII.n15}{}“Reviewing change-of-lineage” (the consciousness that precedes the path consciousness) applies to stream-enterers. “Reviewing cleansing” (the “cleansing” that consists in attaining a higher path than the first) applies to once-returners and non-returners (see \textbf{\cite{Vism-mhṭ}589}).} Trainers, having emerged from a path, give importance to the path and review it” (\textbf{\cite{Paṭṭh}I 165}). Herein, the object in giving importance to which consciousness \marginnote{\textcolor{teal}{\footnotesize\{612|554\}}}{}and consciousness concomitants arise, is necessarily a cogent one among these objects. So their difference may be understood in this way: \emph{object-predominance} is in the sense of what is to be given importance to, and \emph{object-decisive-support} is in the sense of a cogent reason.

                            \vismParagraph{XVII.82}{82}{}
                            (b) Also \emph{proximate-decisive-support condition} is set forth without differentiating it from the proximity condition in the way beginning, “Any preceding profitable aggregates are a condition, as decisive-support condition, for any succeeding aggregates” (\textbf{\cite{Paṭṭh}I 165}). But in the exposition there is a distinction, because in the exposition of the schedule (\emph{mātikā-nikkhepa}) they are given as proximity in the way beginning, “Eye-consciousness element and the states associated therewith are a condition, as proximity condition, for mind element and for the states associated therewith” (\textbf{\cite{Paṭṭh}I 2}) and as decisive-support in the way beginning, “Any preceding profitable states are a condition as decisive-support condition, for any succeeding profitable states” (\textbf{\cite{Paṭṭh}I 4}), though it comes to the same thing as regards the meaning. Nevertheless, \emph{proximity} may be understood as the ability to cause the occurrence of an appropriate conscious arising proximate (next) to itself, and \emph{decisive support} as the preceding consciousness’s cogency in the arousing of the succeeding consciousnesses.

                            \vismParagraph{XVII.83}{83}{}
                            For while in the cases of root-cause and other such conditions consciousness can arise actually without any of those conditions, there is no arising of consciousness without a proximate consciousness [to precede it], so this is a cogent condition. Their difference, then, may be understood in this way: \emph{proximity condition} arouses an appropriate consciousness proximate (next) to itself, while \emph{proximity-decisive-support condition} is a cogent reason.

                            \vismParagraph{XVII.84}{84}{}
                            (c) As to \emph{natural-decisive-support}: the decisive-support is natural, thus it is a natural-decisive-support. Faith, virtue, etc., produced in, or climate, food, etc., habitual to, one’s own continuity are called natural. Or else, it is a decisive-support by nature, \textcolor{brown}{\textit{[537]}} thus it is a natural-decisive-support. The meaning is that it is unmixed with object and proximity. It should be understood as variously divided up in the way beginning: “Natural-decisive-support: with faith as decisive-support a man gives a gift, undertakes the precepts of virtue, does the duties of the Uposatha, arouses jhāna, arouses insight, arouses the path, arouses direct-knowledge, arouses an attainment. With virtue … With learning … With generosity … With understanding as decisive-support a man gives a gift … arouses an attainment. Faith, virtue, learning, generosity, understanding, are conditions, as decisive-support condition, for [the repeated arising of] faith, virtue, learning, generosity, understanding” (\textbf{\cite{Paṭṭh}I 165}). So these things beginning with faith are natural-decisive-support since they are both natural and decisive-supports in the sense of a cogent reason.

                            \vismParagraph{XVII.85}{85}{}
                            (10) A state that assists by being present, having arisen previously, is a \emph{prenascence condition}. It is elevenfold as physical basis and object in the five doors, and as the heart-basis, according as it is said: “The eye base is a condition, as prenascence condition, for the eye-consciousness element and for the states associated therewith. The ear base … The nose base … The tongue base … The \marginnote{\textcolor{teal}{\footnotesize\{613|555\}}}{}body base … The visible-data base … The sound base … The odour base … The flavour base … The tangible-data base is a condition, as prenascence condition, for the body-consciousness element and for the states associated therewith. The visible-data base … The sound base … The odour base … The flavour base … the tangible data base [is a condition, as prenascence condition,] for the mind element. The materiality with which as their support the mind element and mind-consciousness element occur is a condition, as prenascence condition, for the mind-element and for the states associated therewith, and it is sometimes [as in the course of an existence] a condition, as prenascence condition, sometimes [as at rebirth-linking] not a condition, as prenascence condition, for the mind-consciousness element and for the states associated therewith” (\textbf{\cite{Paṭṭh}I 4–5}).

                            \vismParagraph{XVII.86}{86}{}
                            (11) An immaterial state that [while present] assists prenascent material states [also present] by consolidating them is a \emph{postnascence condition}, like the volition of appetite for food, which assists the bodies of vultures’ young. Hence it is said: “Postnascent \textcolor{brown}{\textit{[538]}} states of consciousness and its concomitants are a condition, as postnascence condition, for the prenascent [co-present] body” (\textbf{\cite{Paṭṭh}I 5}).

                            \vismParagraph{XVII.87}{87}{}
                            (12) A state that assists the efficiency and power of the proximate (next) in the sense of repetition is a \emph{repetition condition}, like repeated application to books and so on. It is threefold as profitable, unprofitable, and functional impulsion; for it is said: “Preceding profitable states are a condition, as repetition condition, for succeeding profitable states … Preceding unprofitable … Preceding functional indeterminate states are a condition, as repetition condition, for succeeding functional indeterminate states” (\textbf{\cite{Paṭṭh}I 5}).

                            \vismParagraph{XVII.88}{88}{}
                            (13) A state that assists by means of the action called intervening of consciousness is a \emph{kamma condition}. It is twofold as (a) profitable and unprofitable volition acting from a different time, and (b) as all conascent volition (see \textbf{\cite{Paṭṭh}I 172}), according as it is said: “Profitable and unprofitable kamma is a condition, as kamma condition, for resultant aggregates and for the kinds of materiality due to kamma performed. Conascent volition is a condition, as kamma condition, for associated states and for the kinds of materiality originated thereby” (\textbf{\cite{Paṭṭh}I 5}).

                            \vismParagraph{XVII.89}{89}{}
                            (14) A resultant state that, by effortless quiet, assists effortless quiet [in other states] is a \emph{kamma-result condition}. In the course of an existence it is a condition for states originated by it, and at rebirth-linking for the kinds of materiality due to kamma performed, and in both cases for the associated states, according as it is said: “One resultant indeterminate aggregate is a condition, as kamma-result condition, for three aggregates and for the kinds of materiality originated by consciousness … At the moment of rebirth-linking one resultant indeterminate aggregate [is a condition …] for three aggregates … Three aggregates [are a condition …] for one aggregate … Two aggregates are a condition, as kamma-result condition, for two aggregates and for the kinds of materiality due to kamma performed. Aggregates are a condition, as kamma-result condition, for the physical basis” (\textbf{\cite{Paṭṭh}I 173}).

                            \vismParagraph{XVII.90}{90}{}
                            (15) The four kinds of nutriment, which assist material and immaterial states by consolidating them, are \emph{nutriment conditions}, according as it is said: \marginnote{\textcolor{teal}{\footnotesize\{614|556\}}}{}“Physical nutriment is a condition, as nutriment condition, for this body. Immaterial nutriments are conditions, as nutriment condition, for associated states and for the kinds of materiality originated by them” (\textbf{\cite{Paṭṭh}I 5}). But in the Question Section it is said: “At the moment of rebirth-linking, resultant indeterminate nutriments are conditions, as nutriment condition, for aggregates associated therewith and for the kinds of materiality due to kamma performed” (\textbf{\cite{Paṭṭh}I 174}).

                            \vismParagraph{XVII.91}{91}{}
                            (16) Leaving out the femininity and masculinity faculties, the twenty remaining faculties (see \hyperlink{XIV.1}{XIV.1}{}), which assist in the sense of predominance, \textcolor{brown}{\textit{[539]}} are \emph{faculty conditions}. Herein, the five, namely, the eye faculty, etc., are conditions only for immaterial states, the rest are conditions for material and immaterial states, according as it is said: “The eye faculty [is a condition, as faculty condition,] for eye-consciousness element [and for the states associated therewith]. The ear faculty … The nose faculty … The tongue faculty … The body faculty is a condition, as faculty condition, for the body-consciousness element and for the states associated therewith. The material life faculty is a condition, as faculty condition, for the kinds of materiality due to kamma performed. The immaterial faculties are a condition, as faculty condition, for the states associated therewith and for the kinds of materiality originated thereby” (Paṭṭh 1, 5–6). But in the Question Section it is said: “At the moment of rebirth-linking resultant indeterminate faculties are a condition, as faculty condition, for associated aggregates and for the kinds of materiality due to kamma performed” (\textbf{\cite{Paṭṭh}I 175}).

                            \vismParagraph{XVII.92}{92}{}
                            (17) All the seven jhāna factors classed as profitable, etc.—leaving out the pair, pleasant and painful feeling, in the case of the two sets of five consciousnesses—which factors assist in the sense of constituting a state of jhāna, are \emph{jhāna conditions}, according as it is said: “The jhāna factors are a condition, as jhāna condition, for the states associated with jhāna and for the kinds of materiality originated thereby” (\textbf{\cite{Paṭṭh}I 6}). But in the Question Section it is said: “At the moment of rebirth-linking, resultant indeterminate jhāna factors are a condition, as jhāna condition, for associated aggregates and for the kinds of materiality due to kamma performed” (\textbf{\cite{Paṭṭh}I 175}).

                            \vismParagraph{XVII.93}{93}{}
                            (18) The twelve path factors classed as profitable, etc., which assist in the sense of an outlet from whatever it may be, are \emph{path conditions}, according as it is said: “The path factors are a condition, as path condition, for states associated therewith and for the kinds of materiality originated thereby” (\textbf{\cite{Paṭṭh}I 6}). But in the Question Section it is said: “At the moment of rebirth-linking, resultant indeterminate path factors are a condition, as path condition, for aggregates associated therewith and for the kinds of materiality due to kamma performed” (\textbf{\cite{Paṭṭh}I 176}).

                            But these two, namely, jhāna and path conditions, should be understood as inapplicable to the two sets of five consciousnesses and to the consciousnesses without root-cause ((34)–(41), (50)–(56), (70)–(72)).

                            \vismParagraph{XVII.94}{94}{}
                            (19) Immaterial states that assist by the kind of association consisting in having the same physical basis, same object, same arising, same cessation, are \marginnote{\textcolor{teal}{\footnotesize\{615|557\}}}{}\emph{association conditions}, according as it is said: “The four immaterial aggregates are a condition, as association condition, for each other” (\textbf{\cite{Paṭṭh}I 6}).

                            \vismParagraph{XVII.95}{95}{}
                            (20) Material states that assist immaterial states, and immaterial states that assist material states by not having sameness of physical basis, etc., are \emph{dissociation conditions}. This is threefold as conascent, postnascent, and prenascent, for this is said: “Conascent profitable aggregates are a condition, as dissociation condition, for the kinds of materiality originated by consciousness. Postnascent \textcolor{brown}{\textit{[540]}} profitable [mental] aggregates are a condition, as dissociation condition, for this prenascent body” (\textbf{\cite{Paṭṭh}I 176}). But in the analysis of the conascent in the indeterminate clause it is said: “At the moment of rebirth-linking, resultant indeterminate aggregates are a condition, as dissociation condition, for the kinds of materiality due to kamma performed. The aggregates are a condition, as dissociation condition, for the physical basis, and the physical basis for the aggregates” (\textbf{\cite{Paṭṭh}I 176}). But the prenascent should be understood as the eye faculty, etc., and the physical basis, according as it is said: “The prenascent eye base [is a condition, as dissociation condition,] for eye-consciousness … The body base is a condition, as dissociation condition, for body-consciousness. The physical basis [is a condition, as dissociation condition,] for resultant-indeterminate and functional-indeterminate aggregates … The physical basis [is a condition, as dissociation condition,] for profitable aggregates … The physical basis [is a condition, as dissociation condition,] for unprofitable aggregates” (\textbf{\cite{Paṭṭh}I 176–177}).

                            \vismParagraph{XVII.96}{96}{}
                            (21) A state that, by means of existingness characterized by presence, assists a like state by consolidating it, is a \emph{presence condition}. A sevenfold summary is laid down for it according to immaterial aggregates, great primaries, mentality-materiality, consciousness and consciousness-concomitants, great primaries, bases, and physical [heart] basis, according as it is said: “The four immaterial aggregates are a condition, as presence condition, for each other. The four great primaries … are … for each other. At the time of descent into the womb mentality and materiality [are a condition, as presence condition,] for each other. States of consciousness and consciousness-concomitants are … for the kinds of materiality originated by consciousness. The four great primaries are … for derived materiality. The eye base is … for the eye-consciousness element [and for the states associated therewith]. The [ear base … nose base … tongue base …] body base is … for the body-consciousness element … The visible-data base [is … for the eye-consciousness element … The sound base … odour base … flavour base …] tangible-data base is a condition, as presence condition, for the body-consciousness element and for the states associated therewith. The visible-data base … The [sound base … odour base … flavour base …] tangible-data base is a condition, as presence condition, for the mind element and for the states associated therewith. The materiality with which as their support the mind element and mind-consciousness element occur is a condition, as presence condition, for the mind element, for the mind-consciousness element, and for the states associated therewith” (\textbf{\cite{Paṭṭh}I 6}).

                            \vismParagraph{XVII.97}{97}{}
                            \marginnote{\textcolor{teal}{\footnotesize\{616|558\}}}{}But in the Question Section, after setting forth conascence, prenascence, postnascence, nutriment, and faculty, the description is given first under conascence in the way beginning, “One aggregate is a condition, as presence condition, for three aggregates and for materiality originated thereby” (\textbf{\cite{Paṭṭh}I 178}). Under prenascence the description is given according to the prenascent eye and so on. Under postnascence the description is given according to postnascent consciousness and consciousness-concomitants as conditions for this body. Under nutriments and faculties [respectively] the description is given thus: “Physical nutriment is a condition, as presence condition, for this body,” \textcolor{brown}{\textit{[541]}} and “The material life faculty is a condition, as presence condition, for materiality due to kamma performed” (\textbf{\cite{Paṭṭh}I 178}).\footnote{\vismAssertFootnoteCounter{16}\vismHypertarget{XVII.n16}{}“The presence (\emph{atthi}) condition is not applicable to Nibbāna. For a presence condition is that which is unhelpful by its absence of existingness (\emph{atthi-bhāvābhāva}) and becomes helpful by obtaining existingness. And Nibbāna does not, after being unhelpful by its own absence of existingness to those states that have Nibbāna as their object, become helpful to them by obtaining existingness. Or alternatively, the presence condition, which by its non-existingness is the opposite of helpfulness to those states that are associated with arising, etc., is helpful to them by its existingness. So Nibbāna is not a presence condition” (\textbf{\cite{Vism-mhṭ}597}).

                                    It may be noted that \emph{atthi }has more than one use, among which the following two may be mentioned: (1) \emph{atthi} (is) = \emph{upalabbhaniya} (is (a) “apprehendable,” and (b) not a self-contradictory impossibility)—“\emph{atthi, bhikkhave, ajātaṃ}—There is an unborn' (\textbf{\cite{Ud}80}), and the discussion on the existence of Nibbāna (\hyperlink{XVI.67}{XVI.67ff.}{}). (2) \emph{Atthi} (is) = \emph{uppanna }(arisen)—see \emph{“Yaṃ, bhikkhave, rūpaṃ jātaṃ pātubhūtaṃ atthī ti tassa saṅkhā}—Of the materiality that is born, manifested, it is said that ‘It is’” (\textbf{\cite{S}II 71f.}). The \emph{atthi-paccaya }(presence condition), being implicitly equated with the latter, cannot be applied to Nibbāna because Nibbāna is not subject to arising (\textbf{\cite{A}I 152}).}

                            \vismParagraph{XVII.98}{98}{}
                            (22) Immaterial states that, by their ceasing in contiguity [before], assist by giving opportunity to immaterial states that arise proximately (next) after them are \emph{absence conditions}, according as it is said: “States of consciousness and consciousness-concomitants that have ceased in contiguity are a condition, as absence condition, for present states of consciousness and consciousness-concomitants” (\textbf{\cite{Paṭṭh}I 7}).

                            \vismParagraph{XVII.99}{99}{}
                            (23) Those same states, because they assist by their disappearance, are a \emph{disappearance condition}, according as it is said: “States of consciousness and consciousness-concomitants that have disappeared in contiguity are a condition, as disappearance condition, for present states of consciousness and consciousness-concomitants” (\textbf{\cite{Paṭṭh}I 7}).

                            \vismParagraph{XVII.100}{100}{}
                            (24) The same states that are presence condition, because they assist by their non-disappearance, should be understood as a \emph{non-disappearance condition}. Or this dyad is stated as an embellishment of teaching to suit the needs of those who are teachable, just as [in the Mātikā of the Dhammasaṅgaṇī] the “dissociated-from-cause dyad” is given after the “causeless dyad.”
                        \par\noindent[\textsc{\textbf{How Ignorance is a Condition for Formations}}]

                            \vismParagraph{XVII.101}{101}{}
                            \marginnote{\textcolor{teal}{\footnotesize\{617|559\}}}{}Now, as regards these twenty-four conditions:
                            \begin{verse}
                                For those of merit ignorance\\{}
                                Is a condition in two ways\\{}
                                And for the next in many ways\\{}
                                But for the last kind only once.
                            \end{verse}


                            \vismParagraph{XVII.102}{102}{}
                            Herein, \emph{for those of merit ignorance is a condition in two ways}: it is a condition in two ways, namely, as object condition and as decisive-support condition. For ignorance is a condition, as object condition, for formations of merit of the sense sphere at the time of comprehending [by means of insight] ignorance as liable to destruction and fall; and it is likewise for those of the fine-material sphere at the time of knowing a confused mind by means of direct-knowledge consciousness [through penetrating others’ minds, and so on]. But it is a condition, as decisive-support condition, in two cases, that is to say, [for the sense-sphere formation] in one who, for the purpose of surmounting ignorance, fulfils the various instances of sense-sphere merit-making consisting in giving, etc., and [for the fine-material-sphere formation] in one who arouses the fine-material jhānas [for the same purpose]. Likewise in one who effects that merit while aspiring to the delight of sense-sphere becoming and fine-material becoming because he is confused by ignorance.

                            \vismParagraph{XVII.103}{103}{}
                            \emph{And for the next in many ways}: it is a condition for formations of demerit in many ways. How? As object condition at the time of the arising of greed, etc., contingent upon ignorance; as object-predominance and object-decisive-support respectively at the times of giving importance [to ignorance] and enjoying [it]; as decisive-support in one who, being confused by ignorance and unaware of danger, kills living things, etc.; as proximity, contiguity, proximity-decisive-support, repetition, absence, and disappearance, for the second impulsion and those that follow; as root-cause, conascence, mutuality, support, association, presence, and non-disappearance, in one doing anything unprofitable. It is thus a condition in many ways.

                            \vismParagraph{XVII.104}{104}{}
                            \emph{But for the last kind only once:} \textcolor{brown}{\textit{[542]}} it is reckoned as a condition in one way, namely, as decisive-support condition only, for formations of the imperturbable. But its relation as decisive-support condition should be understood as stated under formations of merit.
                        \par\noindent[\textsc{\textbf{No Single Fruit from Single Cause}}]

                            \vismParagraph{XVII.105}{105}{}
                            \emph{Here it may be asked}: But how is this? Is ignorance the only condition for formations, or are there other conditions? What is the position here? For firstly, if it is the only one, there follows the assertion of a single cause;\footnote{\vismAssertFootnoteCounter{17}\vismHypertarget{XVII.n17}{}“The assertion of a single cause (\emph{kāraṇa}) is undesirable because it follows that there would be production of everything all the time, and because it follows that there would be a single homogeneous state;” (\textbf{\cite{Vism-mhṭ}599}) cf. \hyperlink{XIX.3}{XIX.3}{}.} but then if there are others, the description of it as a single cause, namely, “With ignorance as \marginnote{\textcolor{teal}{\footnotesize\{618|560\}}}{}condition there are formations,” is incorrect—It is not incorrect. Why not? Here is the reason:
                            \begin{verse}
                                Nor from a single cause arise\\{}
                                One fruit or many, nor one fruit from many;\\{}
                                ’Tis helpful, though, to utilize\\{}
                                One cause and fruit as representative.
                            \end{verse}


                            \vismParagraph{XVII.106}{106}{}
                            Here there is no single or multiple fruit of any kind from a single cause, nor a single fruit from multiple causes, but only multiple fruit from multiple causes. So from multiple causes, in other words, from temperature, earth, seed, and moisture, is seen to arise a multiple fruit, in other words, the shoot, which has visible form, odour, taste, and so on. But one representative cause and fruit given in this way, “With ignorance as condition there are formations; with formations as condition, consciousness,” have a meaning and a use.

                            \vismParagraph{XVII.107}{107}{}
                            For the Blessed One employs one representative cause and fruit when it is suitable for the sake of elegance in instruction and to suit the idiosyncrasies of those susceptible of being taught. And he does so in some instances because it is a basic factor, and in some instances because it is the most obvious, and in some instances because it is not common to all.

                            In the passage “With contact as condition, feeling” (\textbf{\cite{M}I 261}) he mentions a single cause and fruit because they are basic factors. For contact is the basic cause of feeling since the kinds of feeling are defined according to the kinds of contact [as “eye-contact-born feeling” and so on], and feeling is contact’s basic fruit since contact is defined according to the kinds of feeling [that it produces]. He mentions a single cause in the passage “Disease due to phlegm” (\textbf{\cite{A}V 110}) because that is the most obvious. For here what is obvious is the phlegm, not the kamma, etc., [mentioned later in the same sutta]. He mentions a single cause in the passage “Bhikkhus, any states whatever that are unprofitable are all rooted in unwise attention” (cf. \textbf{\cite{S}V 91}) because it is not common to all. For unwise attention to unprofitable things is not common to all [states] in the way that, say, physical basis and object are common to all.

                            \vismParagraph{XVII.108}{108}{}
                            Consequently, although other causes of formations such as physical basis and object, conascent states, etc., are actually existent, still ignorance may be understood as the representative cause of formations [firstly] because it is the basic factor as the cause of other causes of formations such as craving, etc., as it is said: “Craving increases in one who dwells seeing enjoyment” (\textbf{\cite{S}II 84}), and “With the arising of ignorance there is the arising of cankers” (\textbf{\cite{M}I 55}); and again because it is the most obvious, “Not knowing, bhikkhus, in ignorance, he forms the formation of merit” (cf. \textbf{\cite{S}II 82}); and lastly because it is not common to all. \textcolor{brown}{\textit{[543]}} So the use of one representative cause and fruit should in each instance be understood according to this explanation of it.\footnote{\vismAssertFootnoteCounter{18}\vismHypertarget{XVII.n18}{}\emph{Parihāra-vacana—}“explanation”: not in PED in this sense.}

                            \vismParagraph{XVII.109}{109}{}
                            Here it may be said: “We admit that. But ignorance is reprehensible and has entirely undesirable fruit. How then can it rightly be a condition for formations of merit and of the imperturbable? Sugarcane does not grow from [bitter] nimba \marginnote{\textcolor{teal}{\footnotesize\{619|561\}}}{}seeds.” Why should it not be right? For in the world [that is, even among thinkers outside the Dispensation it is recognized that]
                            \begin{verse}
                                Both as opposed and unopposed\\{}
                                A state’s conditions may be found,\\{}
                                And both as like and unlike too:\\{}
                                That does not make it their \emph{result}.
                            \end{verse}


                            \vismParagraph{XVII.110}{110}{}
                            It is established in the world that when states have a condition, it may be opposed or unopposed to them as to presence, individual essence, function, and so on. For a preceding consciousness is a condition, opposed as to presence, for the succeeding consciousness; and the preceding training is a condition likewise for the plying of crafts, etc., which take place subsequently. Kamma is a condition, opposed as to individual essence, for materiality; and so are milk, etc., for curds, and so on. Light is a condition, opposed as to function, for eye-consciousness; and so are molasses, etc., for intoxicants, and so on. But eye-cum-visible-data, etc., are respectively a condition, unopposed as to presence, for eye-consciousness, and so on. And the first impulsion, and those that follow, are a condition, unopposed as to individual essence and function, for the impulsions that follow them. And just as conditions operate as opposed and unopposed, so also they operate as like and unlike. Materiality—for example, temperature and nutriment—is a condition for materiality: the like for the like. And so are paddy seeds, etc., for paddy crops, and so on. The material is a condition for the immaterial, and so is the immaterial for the material: the unlike for the like. And so are ox hair and ram’s hair, horns, curd, and sesame flour, etc., respectively for \emph{dabba} grass, reeds, \emph{bhūtanaka} grass, and so on.\footnote{\vismAssertFootnoteCounter{19}\vismHypertarget{XVII.n19}{}\emph{Avi—}“a goat or sheep”: not in PED. The Vism text reads “\emph{golomāvilomavisāṇa-dadhitilapiṭṭhādīni ca dubbāsarabhūtanakādīnaṃ}.” \textbf{\cite{Vism-mhṭ}} explains thus: “\emph{Golomāvilomādī ti ādisu golomāvilomāni dubbāya avī ti rattā eḷakā veditabbā visāṇaṃ sarassa dadhitilapiṭṭhagūlāni bhūtiṇakassa sevālaṃ taṇḍuleyyakassa kharavalavā assatarassā ti evam ādi ādisaddena saṅgahito},” which renders thus: “As to ‘\emph{Ox hair and ram’s hair, etc.},’ and the rest: ox hair and ram’s hair [are conditions for the unlike] \emph{dubbā} (\emph{dabba}) grass—a ram (\emph{avi}) should be understood as a red sheep (\emph{eḷakā}); horn is for reeds (\emph{sara}); curds, sesame flour and molasses are for \emph{bhūtiṇaka }grass; moss is for the \emph{taṇḍuleyyaka }plant; a she donkey is for a mule; and so on in this way as included by the word ‘etc.’” (\textbf{\cite{Vism-mhṭ}601}). Except for the last-mentioned, it seems problematical why these things, if rightly interpreted, should be conditions for the things mentioned.} And those states for which these are the opposed and unopposed, like and unlike, conditions are not the results of these states as well.

                            \vismParagraph{XVII.111}{111}{}
                            So although this ignorance has entirely undesirable fruit for its result and is reprehensible in its individual essence, yet it should be understood as a condition, opposed or unopposed and like or unlike as the case may be, as to presence, function, and individual essence, for all these formations of merit, etc. And its state as a condition has already been given in the way beginning, “For when unknowing—in other words, ignorance—of suffering, etc., is unabandoned in a man, owing firstly to his unknowing about suffering and about the past, etc., then be believes the suffering of the round of rebirths to be pleasant \marginnote{\textcolor{teal}{\footnotesize\{620|562\}}}{}and he embarks upon the three kinds of formations, which are the cause of that very suffering” (\hyperlink{XVII.62}{§62}{}).

                            \vismParagraph{XVII.112}{112}{}
                            Moreover, there is this way of explanation as well:
                            \begin{verse}
                                Now, when a man is ignorant\\{}
                                Of death and rebirth and the round,\\{}
                                The characteristics of the formed,\\{}
                                Dependently-arisen states, \textcolor{brown}{\textit{[544]}}
                            \end{verse}

                            \begin{verse}
                                And in his ignorance he forms\\{}
                                Formations of this triple kind,\\{}
                                Then ignorance itself will be\\{}
                                Condition for each of the three.
                            \end{verse}


                            \vismParagraph{XVII.113}{113}{}
                            But how does a man who is confused about these things perform these three kinds of formations? Firstly, when he is confused about death, instead of taking death thus, “Death in every case is break-up of aggregates,” he figures that it is a [lasting] being that dies, that it is a [lasting] being’s transmigration to another incarnation, and so on.

                            \vismParagraph{XVII.114}{114}{}
                            When he is confused about reappearance, instead of taking rebirth thus, “Birth in every case is manifestation of aggregates,” he figures that it is a lasting being’s manifestation in a new body.

                            \vismParagraph{XVII.115}{115}{}
                            When he is confused about the round of rebirths, instead of taking the round of rebirths as pictured thus:
                            \begin{verse}
                                The endless chain of aggregates,\\{}
                                Of elements, of bases too,\\{}
                                That carries on unbrokenly\\{}
                                Is what is called “the round of births,”
                            \end{verse}


                            he figures that it is a lasting being that goes from this world to another world, that comes from another world to this world.

                            \vismParagraph{XVII.116}{116}{}
                            When he is confused about the characteristics of formations, instead of apprehending their specific and general characteristics, he figures that formations are self, belong to a self, are lasting, pleasant, beautiful.

                            \vismParagraph{XVII.117}{117}{}
                            When he is confused about dependently-arisen states, instead of taking the occurrence of formations to be due to ignorance, etc., he figures that it is a self that knows or does not know, that acts and causes action, that appears in rebirth-linking, and he figures that atoms, an Overlord, etc., shape its body in the various states of the embryo and endow it with faculties, and that when it has been endowed with faculties it touches, feels, craves, clings, and endeavours, and that it becomes anew in the next becoming; or he figures thus, “All beings … [are] moulded by fate, coincidence and nature” (\textbf{\cite{D}I 53}).

                            \vismParagraph{XVII.118}{118}{}
                            Thus he figures, blinded by ignorance. He is like a blind man who wanders about the earth, encountering now right and now wrong paths, now heights and now hollows, now even and now uneven ground, and so he forms formations now of merit, now of demerit, now imperturbable.

                            \vismParagraph{XVII.119}{119}{}
                            Hence this is said:
                            \begin{verse}
                                \marginnote{\textcolor{teal}{\footnotesize\{621|563\}}}{}As one born blind, who gropes along\\{}
                                Without assistance from a guide,\\{}
                                Chooses a road that may be right\\{}
                                At one time, at another wrong,
                            \end{verse}

                            \begin{verse}
                                So while the foolish man pursues\\{}
                                The round of births without a guide,\\{}
                                Now to do merit he may choose\\{}
                                And now demerit in such plight.
                            \end{verse}

                            \begin{verse}
                                But when the Dhamma he comes to know\\{}
                                And penetrates the truths besides,\\{}
                                Then ignorance is put to flight\\{}
                                At last, and he in peace may go.
                            \end{verse}


                            This is the detailed explanation of the clause, “With ignorance as condition there are formations.” \textcolor{brown}{\textit{[545]}}
                    \subsubsection[\vismAlignedParas{§120–185}(3) Consciousness]{(3) Consciousness}
                        \par\noindent[\textsc{\textbf{1. Kinds of Mundane Resultant Consciousness}}]

                            \vismParagraph{XVII.120}{120}{}
                            In the clause, \emph{with formations as condition, consciousness}, consciousness is sixfold as eye-consciousness, and so on. Herein, eye-consciousness is twofold, namely, profitable [kamma-]resultant and unprofitable [kamma-]resultant (see Table II for bracketed numbers that follow). Likewise ear-, nose-, tongue-, and body-consciousness ((34)–(38) and (50)–(54)). But mind-consciousness is twenty-two-fold, namely, the two profitable and unprofitable resultant mind elements ((39) and (55)), the three root-causeless mind-consciousness elements ((40), (41) and (56)), the eight sense-sphere resultant consciousnesses with root-cause ((42)–(49)), the five of the fine-material sphere ((57)–(61)), and the four of the immaterial sphere ((62)–(65)). So all the thirty-two mundane resultant consciousnesses ((34)–(65)) are included by these six kinds of consciousness. But the supramundane kinds do not belong to the exposition of the round [of becoming], and so they are not included.

                            \vismParagraph{XVII.121}{121}{}
                            Here it may be asked: “But how is it to be known that this consciousness of the kind stated actually has formations as its condition?”—Because there is no kamma-result when there is no stored-up kamma. For this consciousness is kamma-result, and kamma-result does not arise in the absence of stored-up kamma. If it did, then all kinds of kamma-resultant consciousnesses would arise in all kinds of beings, and they do not do so. This is how it should be known that such consciousness has formations as its condition.

                            \vismParagraph{XVII.122}{122}{}
                            But which kind of consciousness has which kind of formations as its condition?

                            Firstly, the following sixteen kinds arise with the sense-sphere formation of merit as condition: the five profitable resultants beginning with eye-consciousness ((34)–(38)), and in the case of mind-consciousness one kind of mind element (39) and two kinds of mind-consciousness element ((40)–(41)), \marginnote{\textcolor{teal}{\footnotesize\{622|564\}}}{}and the eight kinds of sense-sphere resultant ((42)–(49)), according as it is said: “Owing to profitable kamma of the sense sphere having been performed, stored up, resultant eye-consciousness” (\textbf{\cite{Dhs}§431}), “ear-, nose-, tongue-, body-consciousness” (\textbf{\cite{Dhs}§443}), “resultant mind element arises” (\textbf{\cite{Dhs}§455}), “mind-consciousness element accompanied by joy arises” (\textbf{\cite{Dhs}§469}), “mind-consciousness element accompanied by equanimity arises” (\textbf{\cite{Dhs}§484}), “accompanied by joy and associated with knowledge … accompanied by joy, associated with knowledge and prompted … accompanied by joy and dissociated from knowledge … accompanied by joy, dissociated from knowledge and prompted … accompanied by equanimity and associated with knowledge … accompanied by equanimity, associated with knowledge and prompted … accompanied by equanimity and dissociated from knowledge … accompanied by equanimity, dissociated from knowledge and prompted” (\textbf{\cite{Dhs}§498}).

                            \vismParagraph{XVII.123}{123}{}
                            There are five kinds of resultant fine-material-sphere consciousness ((57)–(61)) with the fine-material-sphere formation of merit as condition, according as it is said: “Owing to that same profitable kamma of the fine-material sphere having been performed, stored up, [by the development of that same profitable jhāna,] \textcolor{brown}{\textit{[546]}} secluded from sense desires … he enters upon and dwells in the resultant first jhāna … fifth jhāna” (\textbf{\cite{Dhs}§499}).

                            \vismParagraph{XVII.124}{124}{}
                            There are seven kinds of consciousness with the formation of demerit as condition: the five unprofitable resultants beginning with eye-consciousness ((50)–(54)), one mind element (55), and one mind-consciousness element (56), according as it is said: “Because of unprofitable kamma having been performed and stored up, resultant eye-consciousness has arisen … ear-… nose-… tongue-… body-consciousness has arisen” (\textbf{\cite{Dhs}§556}), “resultant mind element” (\textbf{\cite{Dhs}§562}), “resultant mind-consciousness element has arisen” (\textbf{\cite{Dhs}§564}).

                            \vismParagraph{XVII.125}{125}{}
                            There are four kinds of immaterial resultant consciousness ((62)–(65)) with the formation of the imperturbable as condition, according as it is said: “Owing to that same profitable kamma of the immaterial sphere having been performed, stored up [by the development of that same profitable immaterial jhāna, with the abandoning of bodily pleasure and pain … he enters upon and dwells in the resultant fourth jhāna, which,] with the complete surmounting of perceptions of material form … is accompanied by the base consisting of boundless space” (\textbf{\cite{Dhs}§501}), “accompanied by the base consisting of boundless consciousness” (\textbf{\cite{Dhs}§502}), “accompanied by the base consisting of nothingness” (\textbf{\cite{Dhs}§503}), “accompanied by the base consisting of neither perception nor non-perception” (\textbf{\cite{Dhs}§504}).

                            \vismParagraph{XVII.126}{126}{}
                            After knowing what kind of consciousness has what formations as its condition, one should now understand how it occurs as follows.
                        \par\noindent[\textsc{\textbf{2. The Occurrence of Resultant Consciousness}}]

                            Now, this resultant consciousness all occurs in two ways, namely, (a) in the course of an individual existence (or continuity), and (b) at the rebirth-linking [moment].

                            Herein, there are the two fivefold consciousnesses ((34)–(38) and (50)–(54)), two mind elements ((39) and (55)), and root-causeless mind-consciousness \marginnote{\textcolor{teal}{\footnotesize\{623|565\}}}{}element accompanied by joy (40), that is, thirteen which occur only in the course of an existence in the five-constituent kind of becoming.\footnote{\vismAssertFootnoteCounter{20}\vismHypertarget{XVII.n20}{}For five-constituent becoming, etc., see §§253–54. “Unprofitable resultant eye-consciousness, etc. sometimes arise even in Brahmās when undesirable visible data, etc., come into focus” (\textbf{\cite{Vism-mhṭ}604}); cf. §180.} The remaining nineteen occur in the three kinds of becoming, either in the course of an existence or at rebirth-linking, as appropriate. How?
                            \par\noindent[\emph{\textbf{2a. In the Course of an Existence}}]

                                \vismParagraph{XVII.127}{127}{}
                                Firstly, in one who has been reborn by means of either profitable-result or unprofitable result: according as his faculties mature, so the five profitable-resultant eye-, etc., consciousnesses occur accomplishing the respective functions of \emph{seeing, hearing, smelling, tasting, and touching} ((d)–(h)), contingent respectively upon a desirable or desirable-neutral visible datum, etc., as object that has come into the focus of the eye, etc., and having the sensitivity of the eye, etc., as [material] support. And likewise the five unprofitable-resultant consciousnesses; the only difference being this, that the visible data, etc., as object for these are undesirable or undesirable-neutral. And these ten are invariable as to their door, object, physical basis, and position [in the cognitive series], and invariable as to their functions.

                                \vismParagraph{XVII.128}{128}{}
                                After that, next to the profitable-resultant eye-, etc., consciousness, the profitable-resultant mind element (39) occurs accomplishing the function of \emph{receiving} (i), contingent upon the same object as that of the former, and having the heart-basis as support. \textcolor{brown}{\textit{[547]}} And next after the unprofitable-resultant eye-, etc., consciousness, the unprofitable-resultant mind element (55) occurs likewise. But these two, while variable as to door and object, are invariable as to physical basis and position, and invariable as to function.

                                \vismParagraph{XVII.129}{129}{}
                                Then next to the profitable-resultant mind element, the root-causeless mind-consciousness element accompanied by joy (40) occurs accomplishing the function of \emph{investigation} (j), contingent upon the same object as that of the mind element, and having the heart-basis as support. And when the object is a vivid one in any of the six doors belonging to sense-sphere beings, usually at the end of impulsions associated with greed it holds up the [renewal of the] course of the life-continuum (b) by occurring either once or twice as \emph{registration }(m), having the same object as that apprehended by the impulsions—so it is said in the Majjhima Commentary.\footnote{\vismAssertFootnoteCounter{21}\vismHypertarget{XVII.n21}{}This refers to the old Sinhalese commentary no longer extant.} But in the Abhidhamma Commentary two turns of consciousness have been handed down with respect to registration. This consciousness has two names, “registration” (\emph{tad-ārammaṇa—}lit. \emph{having that object }that the preceding impulsions had) and “aftermath life-continuum” (\emph{piṭṭhi-bhavaṅga—}see \hyperlink{XIV.122}{XIV.122}{}). It is variable as to door and object, it is invariable as to physical basis, and it is variable as to position and function. \marginnote{\textcolor{teal}{\footnotesize\{624|566\}}}{}This, in the first place, it should be understood, is how thirteen kinds of consciousness occur only in the course of an individual existence in the five-constituent kind of becoming.

                                \vismParagraph{XVII.130}{130}{}
                                As to the remaining nineteen ((41)–(49) and (56)–(65)), there is none that does not occur as a rebirth-linking (a) appropriate to it (see \hyperlink{XVII.133}{§133}{}). But in the course of an individual existence, firstly, two, namely, profitable-resultant and unprofitable-resultant root-causeless mind-consciousness elements ((41) and (56)) occur accomplishing four functions, that is to say, the function of \emph{investigating }in the five doors (j) next after profitable-resultant and unprofitable-resultant mind element, the function of \emph{registration} (m) in the six doors in the way already stated, the function of \emph{life-continuum} (b) that continues after rebirth-linking given by themselves, as long as there is no thought-arising to interrupt the life-continuum, and lastly the function of \emph{death} (n) at the end [of the course of an existence]. And so these two are invariable as to [possession of heart-] basis, and variable as to door, object, position, and function.

                                \vismParagraph{XVII.131}{131}{}
                                The eight sense-sphere consciousnesses with root-cause ((42)–(49)) occur accomplishing three functions, namely, the function of \emph{registration} (m) in the six doors in the way already stated, the function of \emph{life-continuum} (b) after rebirth-linking given by themselves, as long as there is no thought-arising to interrupt the life-continuum, and lastly the function of death (n) at the end. And they are invariable as to [possession of heart-] basis, and variable as to door, position, and function.

                                \vismParagraph{XVII.132}{132}{}
                                The five fine-material consciousnesses ((57)–(61)) and the four immaterial consciousnesses ((62)–(65)) occur accomplishing two functions, namely, the function of \emph{life-continuum} (b) that continues after rebirth-linking given by themselves, as long as there is no thought-arising to interrupt the life-continuum, and the function of \emph{death} (n) at the end. As regards these, those of the fine-material sphere are invariable as to [possession of heart-]basis and as to their object, and they are variable as to position and function, while the others occur invariably without [heart-] basis, and they are invariable as to object, and variable as to position and function.

                                This, in the first place, is how the thirty-twofold resultant consciousness occurs in the course of an individual existence with formations as condition. And there [in the course of an existence] these several formations are conditions, as kamma condition and decisive-support condition, for this [thirty-twofold resultant consciousness]. \textcolor{brown}{\textit{[548]}}
                            \par\noindent[\emph{\textbf{2b. At Rebirth-Linking}}]

                                \vismParagraph{XVII.133}{133}{}
                                But what was said above, namely, “as to the remaining nineteen, there is none that does not occur as a rebirth-linking appropriate to it” (\hyperlink{XVII.130}{§130}{}), is hard to understand since it is too brief. Hence, in order to show the details it may be asked: (i) How many kinds of rebirth-linking are there? (ii) How many kinds of rebirth-linking consciousness? (iii) Where and by what means does rebirth-linking come about? (iv) What does rebirth-linking [consciousness] have as its object?

                                \vismParagraph{XVII.134}{134}{}
                                \marginnote{\textcolor{teal}{\footnotesize\{625|567\}}}{}(i) Including the rebirth-linking of non-percipient beings there are twenty kinds of rebirth-linking.

                                (ii) There are nineteen kinds of rebirth-linking consciousnesses, as already described.

                                (iii) Herein, rebirth-linking by means of the unprofitable-resultant root-causeless mind-consciousness element (56) comes about in the states of loss. Rebirth-linking by means of the profitable-resultant (41) comes about in the human world among those blind from birth, born deaf, born mad, born drivelling (see \textbf{\cite{M}I 20}; \textbf{\cite{M-a}I 118}), the sexless, and so on. Rebirth-linking by means of the eight principal resultant consciousnesses with root-cause ((42)–(49)) comes about among deities of the sense sphere and the meritorious among men. That by means of the five fine-material resultant kinds comes about in the fine-material Brahmā-world. That by means of the four immaterial-sphere resultant kinds comes about in the immaterial world. So rebirth-linking [consciousness] conforms to the means by which, and the place in which, it comes about.

                                (iv) Briefly, rebirth-linking [consciousness] has three kinds of objects, namely, past, present, and not-so-classifiable (see \hyperlink{III.n32}{Ch. III, n. 32}{}). Non-percipient rebirth-linking [consciousness] has no object.

                                \vismParagraph{XVII.135}{135}{}
                                Herein, in the base consisting of boundless consciousness and the base consisting of neither perception nor non-perception, the object of rebirth-linking is \emph{past}. That of the ten kinds of sense-sphere rebirth-linking is \emph{past} or \emph{present}. That of the rest is \emph{not-so-classifiable}. But while the rebirth-linking consciousness occurs thus with three kinds of objects, the death consciousness, next to which it occurs, has only a past or a not-so-classifiable object, there being no death consciousness with a present object. Consequently, it should be understood how it occurs in the happy destinies and the unhappy destinies as follows, that is to say, how rebirth-linking consciousness with one of three kinds of objects occurs next to death consciousness with one of two kinds of objects.
                                \par\noindent[\emph{From happy to unhappy destiny}]

                                    \vismParagraph{XVII.136}{136}{}
                                    [\emph{From happy to unhappy destiny}.] For example, firstly in the case of a person in the happy destinies of the sense-sphere who is an evildoer, when he is lying on his deathbed, his evil kamma according as it has been stored up, or its sign, comes into focus in the mind door. For it is said, “Then [the evil deeds that he did in the past] … cover him [and overspread him and envelop him]” (\textbf{\cite{M}III 164}), and so on. Then next to the cognitive series of impulsions ending in registration\footnote{\vismAssertFootnoteCounter{22}\vismHypertarget{XVII.n22}{}\textbf{\cite{Vibh-a}} (Be) adds “\emph{suddhāya va javanavīthiyā” }here, as in §140 below in all texts.} that arose contingent upon that [kamma or its sign], death consciousness arises making the life-continuum’s objective field its object. When it has ceased, rebirth-linking consciousness arises contingent upon that same kamma or kamma sign that had come into focus, and it does so located in the unhappy destiny, being driven there by the force of defilements that have not been cut off. \textcolor{brown}{\textit{[549]}} This is the kind of rebirth-linking that has a \emph{past} object and comes next to death consciousness with a \emph{past} object.

                                    \vismParagraph{XVII.137}{137}{}
                                    \marginnote{\textcolor{teal}{\footnotesize\{626|568\}}}{}In another’s case, owing to kamma of the kind already described, there comes into focus at the mind door at the time of death the sign of the unhappy destinies with the appearance of fire and flames, etc., in the hells, and so on.\footnote{\vismAssertFootnoteCounter{23}\vismHypertarget{XVII.n23}{}“‘\emph{With the appearance of fire and flames, etc., in the hells’ }is said owing to likeness to that; appearance of hell and fire does not itself come into focus for him then” (\textbf{\cite{Vism-mhṭ}607}).} So when the life-continuum has twice arisen and ceased, three sorts of cognitive-series consciousness arise contingent upon that object, namely, the one adverting, impulsions numbering five because of the slowing down due to the nearness of death, and two registrations. After that, one death consciousness arises making the life-continuum’s objective field its object. At this point eleven consciousnesses have elapsed. Then, having that same object, which has a life span of the remaining five conscious moments, his rebirth-linking consciousness arises. This is the kind of rebirth-linking that has a \emph{present} object and comes next to death with a \emph{past} object.

                                    \vismParagraph{XVII.138}{138}{}
                                    In another’s case, at the time of death there comes into focus in one of the five doors an inferior object that is a cause of greed, and so on. When a series of consciousnesses up to determining have arisen in due succession, there arise impulsions numbering five because of slowing down due to the nearness of death, and two registrations. After that, one death consciousness arises making the life-continuum’s objective field its object. At this point fifteen consciousnesses have elapsed, namely, two life-continuums, one each adverting, seeing, receiving, investigating and determining, five impulsions, two registrations, and one death consciousness. Then, having that same object, which has a life span of the remaining one conscious moment, his rebirth-linking consciousness arises. This also is the kind of rebirth-linking that has a \emph{present} object and comes next to a death consciousness with a \emph{past} object.

                                    This, firstly, is how rebirth-linking in an unhappy destiny with \emph{past} and \emph{present }objects occurs next to death consciousness in a happy destiny with a past object.
                                \par\noindent[\emph{From unhappy to happy destiny}]

                                    \vismParagraph{XVII.139}{139}{}
                                    [\emph{From unhappy to happy destiny}.] In the case of one in an unhappy destiny who has stored up blameless kamma all should be understood in the same way, substituting the bright for the dark side thus: His good kamma, [according as it has been stored up], or its sign, comes into focus in the mind door [and continuing] in the way already stated.

                                    This is how rebirth-linking occurs in a happy destiny with \emph{past} and \emph{present} objects next to death consciousness in an unhappy destiny with a \emph{past} object.
                                \par\noindent[\emph{From happy to happy destiny}]

                                    \vismParagraph{XVII.140}{140}{}
                                    [\emph{From happy to happy destiny}.] In the case of one in a happy destiny who has stored up blameless kamma, when he is lying on his death-bed, his blameless kamma according as it has been stored up, or its sign, comes into focus in the mind door. For it is said, “Then [the good deeds he did in the past] … cover him [and overspread him and envelop him]” (\textbf{\cite{M}III 171}), and so on. And that applies only in the case of one who has stored up blameless sense-sphere kamma. \textcolor{brown}{\textit{[550]}} But in the case of one who has stored up kamma of the exalted spheres only the sign of the kamma comes into focus. Then next to the cognitive series of \marginnote{\textcolor{teal}{\footnotesize\{627|569\}}}{}impulsions ending in registration, or of simple impulsions, that arose contingent upon that [kamma or its sign], death consciousness arises making the life-continuum’s objective field its object. When it has ceased, rebirth-linking consciousness arises contingent upon that same kamma or sign of kamma that had come into focus, and it does so located in the happy destiny, being driven there by the force of defilements that have not been cut off. This is the kind of rebirth-linking that has a \emph{past} or a \emph{not-so-classifiable} object and comes next to death consciousness with a past object.

                                    \vismParagraph{XVII.141}{141}{}
                                    In another’s case, owing to blameless sense-sphere kamma, there comes into focus in the mind door at the time of death the sign of a happy destiny, in other words, the appearance of the mother’s womb\footnote{\vismAssertFootnoteCounter{24}\vismHypertarget{XVII.n24}{}The \emph{Sammohavinodanī} adds more details here: “When hell appears it does so like a metal cauldron; when the human world appears, the mother’s womb appears like a woollen slipper (\emph{kambala-yāna—}for \emph{yāna }as footwear or sandals see \textbf{\cite{M-a}III 222}); when the heavenly world appears, wishing trees, divine palaces and couches, etc., appear.” Vism-mhṭ remarks here: “By the words ‘the appearance of the mother’s womb,’ etc., only visual appearance is given as the sign of destiny. Herein, in the first place it would be logical that sound has not been given in the Commentaries as a sign of destiny because it is included in the happy destinies as not-clung-to, but the reason for odour, etc., not having been given, will be inquired into” (\textbf{\cite{Vism-mhṭ}609}). This question is in fact dealt with at length at \textbf{\cite{Vism-mhṭ}611}, but the arguments are not reproduced here. See note 26 below.} in the case of the human world or the appearance of pleasure groves, divine palaces, wishing trees, etc., in the case of the divine world. His rebirth-linking consciousness arises next to the death consciousness in the order shown for the sign of an unhappy destiny. This is the kind of rebirth-linking that has a \emph{present} object and comes next to death consciousness with a \emph{past} object.

                                    \vismParagraph{XVII.142}{142}{}
                                    In another’s case, relatives present [objects to him] at the five sense doors, such as a visible datum as object, perhaps flowers, garlands, flags, banners, etc., saying, “This is being offered to the Blessed One for your sake, dear, set your mind at rest”; or a sound as object, perhaps, preaching of the Dhamma, offerings of music, etc.; or an odour as object, perhaps incense, scents, perfumes, etc.; or a taste as object, perhaps honey, molasses, etc., saying, “Taste this, dear, it is a gift to be given for your sake”; or a tangible datum as object, perhaps Chinese silk, silk of Somāra, saying, “Touch this, dear, it is a gift to be given for your sake.” Now, when that visible datum, or whatever it may be, as object has come into focus for him and the consciousnesses ending in determining have arisen in due succession, there arise in him impulsions numbering five because of slowing down due to the nearness of death, and two registrations; after that, one death consciousness, making the life-continuum’s objective field its object. At the end of that, having that same object, which lasts only a single conscious moment, rebirth-linking consciousness arises. This also is the kind of rebirth-linking with a \emph{present} object and comes next to a death consciousness with a \emph{past} object.

                                    \vismParagraph{XVII.143}{143}{}
                                    But in the case of another who is in a happy destiny and has obtained exalted [consciousness] through earth-kasiṇa jhāna, etc., at the time of his death \marginnote{\textcolor{teal}{\footnotesize\{628|570\}}}{}there comes into focus at the mind door the sense-sphere profitable kamma or the sign of the kamma or the sign of the destiny, or else the sign of the earth kasiṇa, etc., or else the exalted consciousness, \textcolor{brown}{\textit{[551]}} or else there comes into focus a superior object of the eye or ear that is a cause for profitable rebirth. When the consciousnesses ending in determining have arisen in due succession, there arise in him impulsions numbering five because of slowing down due to the nearness of death. But in those who belong to an exalted destiny there is no registration. So the one death consciousness arises next to the impulsion and making the life-continuum’s objective field its object. At the end of that, rebirth-linking consciousness arises located in one of the happy destinies of the sense sphere or exalted sphere, and having as its object whichever one among the aforesaid objects has appeared. This is the kind of rebirth-linking with a \emph{past, present}, or \emph{not-so-classifiable} object next to death consciousness in a happy destiny with a \emph{not-so-classifiable object}.

                                    \vismParagraph{XVII.144}{144}{}
                                    Rebirth-linking next to immaterial-sphere death should be understood in this way too. This is how rebirth-linking occurs with a \emph{past, present}, or \emph{not-so-classifiable} object next to death consciousness in a happy destiny with a past or \emph{not-so-classifiable} object.
                                \par\noindent[\emph{From unhappy to unhappy destiny}]

                                    \vismParagraph{XVII.145}{145}{}
                                    [\emph{From unhappy to unhappy destiny}.] In the case of one in an unhappy destiny who is an evil-doer, that kamma, or its sign, or the sign of the destiny, comes into focus in the mind door, or in the five doors, as the object that is the cause for the unprofitable rebirth. Then his rebirth-linking consciousness arises in due succession at the end of the death consciousness and located in the unhappy destiny and with one of those objects as its object. This is how rebirth-linking occurs with a past or present object next to death in an unhappy destiny with a past object.
                                \par\noindent[\emph{How Kamma is a Condition}]

                                    \vismParagraph{XVII.146}{146}{}
                                    Up to \emph{this} point there has been shown the occurrence of the nineteenfold consciousness as rebirth-linking. Also all this [is further classified; for]
                                    \begin{verse}
                                        While it occurs in linking thus,\\{}
                                        It has a double class beside\\{}
                                        Through kamma, and as mixed and not,\\{}
                                        And is still further classified.
                                    \end{verse}


                                    \vismParagraph{XVII.147}{147}{}
                                    When this nineteenfold kamma-resultant consciousness occurs thus in rebirth-linking, it does so by means of kamma in two ways; for according to the way in which the kamma that generates it occurs, the kamma can be its condition both as kamma condition acting from a different time and as decisive-support condition, since this is said: “Profitable … [and] unprofitable kamma is a condition, as decisive-support condition, for [its] result” (\textbf{\cite{Paṭṭh}I 167}, 169).

                                    \vismParagraph{XVII.148}{148}{}
                                    It should be understood that when it occurs thus, its \emph{double class}, etc., is \emph{mixed and not}, and it is \emph{still further classified}.

                                    For example: though this [type of consciousness] occurs in one way only as rebirth-linking, still it is twofold as divided into mixed and unmixed with \marginnote{\textcolor{teal}{\footnotesize\{629|571\}}}{}materiality; \textcolor{brown}{\textit{[552]}} it is threefold as divided according to sense-desire, fine-material, and immaterial becoming (\textbf{\cite{M}I 50}); it is fourfold as egg-born, womb-born, putrescence-(moisture-) born, and of apparitional generation (\textbf{\cite{M}I 73}); it is fivefold according to destiny (\textbf{\cite{M}I 73}); it is sevenfold according to the stations of consciousness (\textbf{\cite{D}III 253}), and it is eightfold according to the abodes of beings [excluding non-percipient beings] (see \textbf{\cite{D}III 263}).

                                    \vismParagraph{XVII.149}{149}{}
                                    Herein:
                                    \begin{verse}
                                        The mixed is double, sexed and not,\\{}
                                        And that with sex is double too;\\{}
                                        The least decads the first has got\\{}
                                        Respectively are three and two.
                                    \end{verse}


                                    \vismParagraph{XVII.150}{150}{}
                                    \emph{The mixed is double, sexed and not}: that rebirth-linking consciousness, which, leaving aside the immaterial becoming, arises here mixed with materiality, is twofold as “with sex” and “without sex,”\footnote{\vismAssertFootnoteCounter{25}\vismHypertarget{XVII.n25}{}\emph{Sa-bhāva }(with sex) and \emph{a-bhāva }(without sex) are not to be confused with \emph{sabhāva }(individual essence) and \emph{abhāva }(absence, non-existence).} because it arises in the fine-material sphere without the sex called femininity faculty and masculinity faculty, and because—leaving aside the rebirth-linking of one born as a eunuch—it arises in the sense-sphere becoming together with that [twofold] sex.

                                    And that with sex is double too: there also that with sex is twofold because it arises in association with either the female or the male sex.

                                    \vismParagraph{XVII.151}{151}{}
                                    \emph{The least decads the first has got respectively are three or two}: together with the rebirth-linking consciousness that is mixed with materiality and comes first in the pair “mixed and unmixed,” there arise, at the least, the two decads (see 18.5f.) of physical basis and body, or else the three decads of physical basis, body, and sex. There is no reducing the materiality below that.

                                    \vismParagraph{XVII.152}{152}{}
                                    But when that minimal amount arises in the two kinds of generation termed egg-born and womb-born, it amounts to no more than a drop of cream of ghee on a single fibre of new-born [kid’s] wool, and it is known as the “embryo in the first stage” (\textbf{\cite{S}I 206}).

                                    \vismParagraph{XVII.153}{153}{}
                                    Herein, how the different kinds of generation come about may be understood according to the kind of destiny. For as regards these:
                                    \begin{verse}
                                        No first three generations are\\{}
                                        In hell, or with the deities,\\{}
                                        Save those of earth; all four are found\\{}
                                        In the three other destinies.
                                    \end{verse}


                                    \vismParagraph{XVII.154}{154}{}
                                    Herein, by the words \emph{with deities} it should be understood that, as in hell and among deities—excepting earth deities—so also among the ghosts consumed with thirst, the first three kinds of generation are not found; for they are apparitional only. But in the remaining three kinds of destiny, in other words, among animals, ghosts and human beings, and among the earth deities excepted above, there are all four kinds of generation.

                                    \vismParagraph{XVII.155}{155}{}
                                    \marginnote{\textcolor{teal}{\footnotesize\{630|572\}}}{}Now:
                                    \begin{verse}
                                        The fine material gods have thirty-nine;\\{}
                                        The apparitional and moisture-born\\{}
                                        Have seventy material instances\\{}
                                        At most, and they have thirty at the least.
                                    \end{verse}


                                    \vismParagraph{XVII.156}{156}{}
                                    Firstly, among the fine-material Brahmās of apparitional generation there arise together with rebirth-linking consciousness thirty and also nine material instances \textcolor{brown}{\textit{[553]}} with the four groups, namely, the decads of the eye, ear, and physical basis, and the ennead of life. But leaving the fine-material Brahmās aside, among the others of apparitional generation and those of the moisture-born generation there are seventy instances of materiality at the most with the decads of the eye, ear, nose, tongue, body, physical basis and sex. And these are invariably to be found among deities [of the sense sphere]. Now, the group of material states comprising the ten material instances, namely, colour, odour, flavour, nutritive essence, and the four primary elements, with eye sensitivity and life, are called the “eye decad.” The remaining [groups of material states] should be understood in the same way.

                                    \vismParagraph{XVII.157}{157}{}
                                    At the least, thirty material instances arise with the decads of the tongue, body, and physical basis, in those who are blind from birth, deaf, noseless,\footnote{\vismAssertFootnoteCounter{26}\vismHypertarget{XVII.n26}{}\textbf{\cite{Vism-mhṭ}(p. 611)} has a long discussion here of the difficulty of speaking of the Brahmā-world (where there are only the senses of seeing and hearing) in terms of the decads, which contain the components of odour and flavour. (§156) It ends by defending the \emph{Visuddhimagga} standpoint.} and sexless. Between the most and the least, the allotment should be understood according as appropriate.

                                    \vismParagraph{XVII.158}{158}{}
                                    After knowing this, again:
                                    \begin{verse}
                                        One ought to consider the [pair] death and birth\\{}
                                        Under aggregates, object, cause, destiny, feeling,\\{}
                                        Happiness, and then thinking applied and sustained,\\{}
                                        Distinguishing them by unlikeness and likeness.
                                    \end{verse}


                                    \vismParagraph{XVII.159}{159}{}
                                    The meaning is this: there is rebirth-linking that is twofold as mixed and unmixed [with materiality], and there is the death consciousness next before it, and their unlikeness and likeness according to these aggregates, etc., must be known. How?

                                    \vismParagraph{XVII.160}{160}{}
                                    Sometimes, next to a four-aggregate immaterial death there is a four-aggregate rebirth-linking having a like object; sometimes there is an exalted rebirth-linking with an internal object next to an unexalted death with an external object. This, firstly, is the method in the case of the immaterial planes.

                                    Sometimes there is a five-aggregate sense-sphere rebirth-linking next to a four-aggregate immaterial death. Sometimes there is a four-aggregate immaterial rebirth-linking next to a five-aggregate sense-sphere death or fine-material-sphere death. Thus there is rebirth-linking with a \emph{present }object\footnote{\vismAssertFootnoteCounter{27}\vismHypertarget{XVII.n27}{}\emph{Sammohavinodanī} (Be) has “rebirth-linking with a past, not so-classifiable, and present object next to” and so on.} next to a death with a \emph{past }object, there is rebirth-linking in a certain unhappy destiny next to death in a \marginnote{\textcolor{teal}{\footnotesize\{631|573\}}}{}certain happy destiny, there is rebirth-linking with root-cause next to root-causeless death, there is triple-root-cause rebirth-linking next to double-root-cause death, there is rebirth-linking accompanied by joy next to death accompanied by equanimity, there is rebirth-linking with happiness next to death without happiness, there is rebirth-linking with applied thought next to death without applied thought, there is rebirth-linking with sustained thought next to death without sustained thought, there is rebirth-linking with applied and sustained thought next to death without applied and sustained thought.

                                    In this way they can be coupled together by opposites as appropriate.

                                    \vismParagraph{XVII.161}{161}{}
                                    
                                    \begin{verse}
                                        A mere state that has got its conditions\\{}
                                        Ushers in the ensuing existence;\\{}
                                        While it does not migrate from the past,\\{}
                                        With no cause in the past it is not.
                                    \end{verse}


                                    \vismParagraph{XVII.162}{162}{}
                                    So it is a mere material and immaterial state, arising when it has obtained its conditions, that is spoken of, saying that it comes into the next becoming; it is not a lasting being, \textcolor{brown}{\textit{[554]}} not a soul. And it has neither transmigrated from the past becoming nor yet is it manifested here without cause from that.

                                    \vismParagraph{XVII.163}{163}{}
                                    We shall explain this by the normal process of human death and rebirth-linking. When in the past becoming a man near to a natural or violent death is unable to bear the onset of the unbearable daggers of the [painful] feelings that end in death as they sever the ligatures of the joints in all the limbs, his body gradually withers like a green palm leaf lying in the glare of the sun, and when the faculties of the eye, etc., have ceased and the body faculty, mind faculty, and life faculty remain on in the heart-basis alone, then consciousness, which has as its support the heart-basis still remaining at that moment, either occurs contingent upon some kamma classed as “weighty,” “repeated,” performed “near” [to death] or previously,\footnote{\vismAssertFootnoteCounter{28}\vismHypertarget{XVII.n28}{}See the classification of kamma at \hyperlink{XIX.74}{XIX.74ff.}{} “Repeated” (\emph{samāsevita}) kamma is not mentioned there as such. Of “near” kamma Vism-mhṭ says, “It is that performed next to death, or which is conspicuous in the memory then, whenever it was performed” (\textbf{\cite{Vism-mhṭ}617}).} in other words, the formation that has obtained the remaining conditions, or contingent upon the objective field made to appear by that kamma, in other words, the sign of the kamma or sign of the destiny.\footnote{\vismAssertFootnoteCounter{29}\vismHypertarget{XVII.n29}{}“‘\emph{Sign of the kamma}’ is the event (\emph{vatthu}) by means of which a man accumulates kamma through making it the object at the time of accumulation. Even if the kamma was performed as much as a hundred thousand eons ago, nevertheless at the time of its ripening it appears as kamma or sign of kamma. The ‘\emph{sign of the destiny}’ is one of the visual scenes in the place where rebirth is due to take place. It consists in the visual appearance of flames of fire, etc., to one ready to be reborn in hell, and so on as already stated” (\textbf{\cite{Vism-mhṭ}617}).} And while it is occurring thus, because craving and ignorance have not been abandoned, craving pushes it and the conascent formations fling it forward\footnote{\vismAssertFootnoteCounter{30}\vismHypertarget{XVII.n30}{}“Owing to craving being unabandoned, and because the previously-arisen continuity is similarly deflected, consciousness occurs inclining, leaning and tending towards the place of rebirth-linking. The ‘\emph{conascent formations’ }are the volitions conascent with the impulsion consciousness next to death. Or they are all those that begin with contact. They fling consciousness on to that place of rebirth-linking, which is the object of the kamma and so on. The meaning is that they occur as the cause for the establishment of consciousness on the object by rebirth-linking as though flinging it there” (\textbf{\cite{Vism-mhṭ}617}).} \marginnote{\textcolor{teal}{\footnotesize\{632|574\}}}{}on to that objective field, the dangers in which are concealed by ignorance. And while, as a continuous process,\footnote{\vismAssertFootnoteCounter{31}\vismHypertarget{XVII.n31}{}“As a continuous process consisting of death, rebirth-linking, and the adjacent consciousnesses” (\textbf{\cite{Vism-mhṭ}617}).} it is being pushed by craving and flung forward by formations, it abandons its former support, like a man who crosses a river by hanging on to a rope tied to a tree on the near bank, and, whether or not it gets a further support originated by kamma, it occurs by means of the conditions consisting only in object condition, and so on.

                                    \vismParagraph{XVII.164}{164}{}
                                    The former of these [two states of consciousness] is called “death” (\emph{cuti}) because of falling (\emph{cavana}), and the latter is called “rebirth-linking” (\emph{paṭisandhi}) because of linking (\emph{paṭisandhāna}) across the gap separating the beginning of the next becoming. But it should be understood that it has neither come here from the previous becoming nor has it become manifest without the kamma, the formations, the pushing, the objective field, etc., as cause.

                                    \vismParagraph{XVII.165}{165}{}
                                    
                                    \begin{verse}
                                        An echo, or its like, supplies\\{}
                                        The figures here; connectedness\\{}
                                        By continuity denies\\{}
                                        Identity and otherness.
                                    \end{verse}


                                    \vismParagraph{XVII.166}{166}{}
                                    And here let the illustration of this consciousness be such things as an echo, a light, a seal impression, a looking-glass image, for the fact of its not coming here from the previous becoming and for the fact that it arises owing to causes that are included in past becomings. For just as an echo, a light, a seal impression, and a shadow have respectively sound, etc., as their cause and come into being without going elsewhere, so also this consciousness.

                                    \vismParagraph{XVII.167}{167}{}
                                    And with a stream of continuity there is neither identity nor otherness. For if there were absolute identity in a stream of continuity, there would be no forming of curd from milk. And yet if there were absolute otherness, the curd would not be derived from the milk. And so too with all causally arisen things. And if that were so there would be an end to all worldly usage, which is hardly desirable. So neither absolute identity nor absolute otherness should be assumed here. \textcolor{brown}{\textit{[555]}}

                                    \vismParagraph{XVII.168}{168}{}
                                    Here it might be asked: “If no transmigration is manifested, then after the cessation of the aggregates in this human person, that fruit could be another person’s or due to other [kamma], since the kamma that is the condition for the fruit does not pass on there [to where the fruit is]? And whose is the fruit since there is no experiencer? Therefore this formulation seems to be unsatisfactory.”

                                    \vismParagraph{XVII.169}{169}{}
                                    Here is the reply:
                                    \begin{verse}
                                        \marginnote{\textcolor{teal}{\footnotesize\{633|575\}}}{}In continuity the fruit\\{}
                                        Is neither of nor from another;\\{}
                                        Seed’s forming processes will suit\\{}
                                        To show the purport of this matter.
                                    \end{verse}


                                    \vismParagraph{XVII.170}{170}{}
                                    When a fruit arises in a single continuity, it is neither another’s nor from other [kamma] because absolute identity and absolute otherness are excluded\footnote{\vismAssertFootnoteCounter{32}\vismHypertarget{XVII.n32}{}\emph{Paṭisiddhattā—}“because … excluded”: \emph{paṭisiddha }is not in PED. \emph{Abhisaṅkhāra }here might mean “planting work,” not “formative processes.”} there. The formative processes of seeds establish the meaning of this. For once the formative processes of a mango seed, etc., have been set afoot, when the particular fruit arises in the continuity of the seed’s [growth], later on owing to the obtaining of conditions, it does so neither as the fruit of other seeds nor from other formative processes as condition; and those seeds or formative processes do not themselves pass on to the place where the fruit is. This is the analogy here. And the meaning can also be understood from the fact that the arts, crafts, medicine, etc., learnt in youth give their fruit later on in maturity.

                                    \vismParagraph{XVII.171}{171}{}
                                    Now, it was also asked, “Whose is the fruit, since there is no experiencer?” Herein:
                                    \begin{verse}
                                        “Experiencer” is a convention\\{}
                                        For mere arising of the fruit;\\{}
                                        They say “It fruits” as a convention,\\{}
                                        When on a tree appears its fruit.
                                    \end{verse}


                                    \vismParagraph{XVII.172}{172}{}
                                    Just as it is simply owing to the arising of tree fruits, which are one part of the phenomena called a tree, that it is said “The tree fruits” or “The tree has fruited,” so it is simply owing to the arising of the fruit consisting of the pleasure and pain called experience, which is one part of the aggregates called “deities” and “human beings,” that it is said “A deity or a human being experiences or feels pleasure or pain.” There is therefore no need at all here for a superfluous experiencer.

                                    \vismParagraph{XVII.173}{173}{}
                                    But it may be said: “That may be so; but then these formations must be the conditions for the fruit either when they are present or when they are not present, and if it is when they are present, their result must come about only at the moment of their occurrence; but if it is when they are not present, they must bear fruit constantly both before and after their occurrence.” It can be replied:
                                    \begin{verse}
                                        They are conditions when performed;\\{}
                                        They bear fruit once, but not again;\\{}
                                        The agent and such similes\\{}
                                        Will serve to make the meaning plain.
                                    \end{verse}


                                    \vismParagraph{XVII.174}{174}{}
                                    Formations are conditions for their own fruit because they have been performed, not because of presence or non-presence, according as it is said: \textcolor{brown}{\textit{[556]}} “Due to profitable kamma of the sense sphere having been performed, stored up [in the past], resultant eye-consciousness arises [in the present]” (\textbf{\cite{Dhs}§431}), and so on. Having become conditions for their own fruit according to \marginnote{\textcolor{teal}{\footnotesize\{634|576\}}}{}their capacity, they do not again bear fruit since the result has already ripened. And in explaining the meaning of this the analogy of the agent, etc., should be understood. For just as in the world when someone becomes an agent with the aim of completing some business or other, and he buys goods, say, or obtains a loan, it is simply the fact of his performing the transaction that is the condition for completing that business, not the transaction’s actual presence or non-presence; and after the completion of the business he has no further liability. Why not? Because the business has been completed. So it is because they have been performed that formations are conditions for their own fruit, and they do not bear fruit after they have already given fruit according to their capacity.

                                    Up to this point the occurrence, with formations as condition, of rebirth-linking consciousness that occurs in the two ways as mixed and unmixed [with materiality] has been illustrated.
                        \par\noindent[\textsc{\textbf{3. How Formations are a Condition for Consciousness}}]

                            \vismParagraph{XVII.175}{175}{}
                            Now, in order to eliminate confusion about all these thirty-two kinds of resultant consciousness:
                            \begin{verse}
                                One should of these formations see\\{}
                                For which and how they are conditions\\{}
                                In birth and life in all the three\\{}
                                Kinds of becoming and the rest.
                            \end{verse}


                            \vismParagraph{XVII.176}{176}{}
                            Herein, the three kinds of becoming, the four kinds of generation, the five kinds of destiny, the seven stations of consciousness, and the nine abodes of beings are what are called “The kinds of becoming and the rest.” The meaning is that it should be recognized for what kinds of resultant consciousness these [formations] are conditions in rebirth-linking and in the course of an individual existence, and in what way they are conditions in the various kinds of becoming and so on.

                            \vismParagraph{XVII.177}{177}{}
                            Herein, firstly as regard the \emph{formation of merit}: the formation of merit comprising the eight volitions of the sense sphere ((1)–(8)) is a condition in two ways, as kamma condition acting from a different time and as decisive-support condition, equally for all the nine kinds of resultant consciousness ((41)–(49)) in rebirth-linking in a happy destiny in the sense-sphere becoming. That formation comprising the five profitable volitions of the fine-material sphere ((9)–(13)) [is a condition] in like manner for the five kinds of rebirth-linking in the fine-material becoming ((57)–(61)).

                            \vismParagraph{XVII.178}{178}{}
                            That of the sense sphere divided up as aforesaid is a condition in two ways, as aforesaid, for seven kinds of limited [-sphere] resultant consciousness ((34)–(40))—excluding the root-causeless mind-consciousness element accompanied by equanimity (41)—in the course of an existence, but not in rebirth-linking, in the happy destinies in the sense-sphere becoming. And that same formation is a condition likewise for five kinds of resultant consciousness ((34), (35), (39)–(41)) in the course of an existence, not in rebirth-linking, in the fine-material becoming. It is a condition likewise for eight kinds of limited [-sphere] \marginnote{\textcolor{teal}{\footnotesize\{635|577\}}}{}resultant consciousness ((34)–(41)) in the course of existence, not in rebirth-linking, in the unhappy destinies in the sense-sphere becoming. \textcolor{brown}{\textit{[557]}} For then it is a condition [for such profitable-resultant consciousness occurring] in hell encountering a desirable object [on such occasions] as the Elder Mahā Moggallāna’s visits to hell, and so on. But among animals and powerful ghosts too a desirable object is obtained [through the same condition].

                            \vismParagraph{XVII.179}{179}{}
                            This eightfold formation of merit is also a condition likewise for sixteen kinds of profitable-resultant consciousness in the course of an existence ((34)–(41)) and in rebirth-linking ((42)–(49)) in the happy destinies in the sense-sphere becoming. It is also a condition equally for all ten kinds of resultant consciousness in the course of an existence ((34), (35), (39)–(41)) and in rebirth-linking ((57)–(61)) in the fine-material becoming.

                            \vismParagraph{XVII.180}{180}{}
                            The\emph{ formation of demerit}, comprising the twelve unprofitable volitions ((22)–(33)), is a condition likewise in the unhappy destinies in the sense-sphere becoming for one kind of consciousness in rebirth-linking (56), not in the course of an existence; also for six kinds in the course of an existence ((50)–(55)), not in rebirth-linking; and for all the seven kinds partly in the course of an existence and partly in rebirth-linking. And in the happy destinies in the sense-sphere becoming it is a condition likewise for those same seven kinds in the course of an existence, not in rebirth-linking. In the fine-material becoming it is a condition likewise for four kinds of resultant consciousness ((50)–(51), (55), (56)) in the course of an existence, not in rebirth-linking. Then it is a condition for [Brahmās’] seeing undesirable visible data and hearing undesirable sounds that are in the sense sphere: there are no undesirable visible data, etc., in the Brahmā-world itself; and likewise in the divine world of the sense sphere.\footnote{\vismAssertFootnoteCounter{33}\vismHypertarget{XVII.n33}{}\textbf{\cite{Vism-mhṭ}} points out that this is generally but not always so, since deities see such portents of their death as the fading of their flowers, etc., which are undesirable visible data (see note 43).}

                            \vismParagraph{XVII.181}{181}{}
                            The \emph{formation of the imperturbable} is a condition likewise for four kinds of resultant consciousness ((62)–(65)) in the course of an existence and in rebirth-linking in the immaterial becoming.

                            This, firstly, is how it should be understood what kinds of resultant consciousness these formations are conditions for in rebirth-linking and in the [three] kinds of becoming, and in what way they are conditions. And it should also be understood in the same way of the kinds of generation and so on.

                            \vismParagraph{XVII.182}{182}{}
                            Here is a statement of the bare headings starting from the beginning. Of these [three kinds of] formations, firstly the formation of merit, when giving rebirth-linking, produces the whole of its result in two of the kinds of becoming; likewise in the four kinds of generation beginning with the egg-born, in two of the kinds of destiny, in other words, the divine and the human; in four of the stations of consciousness, [the human, and the planes of the first, second and third jhānas,] described thus, “Different in body and different in perception … different in body and same in perception … same in body and different in perception … same in body and same in perception …;” (\textbf{\cite{D}III 253}) and in only \marginnote{\textcolor{teal}{\footnotesize\{636|578\}}}{}four of the abodes of beings, because in the abode of non-percipient beings it only forms materiality. Therefore it is a condition in the way already stated for twenty-one kinds of resultant consciousness in these two kinds of becoming, four kinds of generation, two kinds of destiny, four stations of consciousness, and four abodes of beings according as they are produced in rebirth-linking ((41)–(49), (57)–(61)) \textcolor{brown}{\textit{[558]}} and the course of an existence ((34)–(41)), as appropriate.

                            \vismParagraph{XVII.183}{183}{}
                            The formation of demerit as rebirth-linking ripens in the sense-sphere becoming only, in the four kinds of generation, in the remaining three destinies, in the one station of consciousness described thus “different in body and same in perception” (\textbf{\cite{D}III 253}), and in the one corresponding abode of beings. Therefore it is a condition in the way already stated for seven kinds of resultant consciousness in one kind of becoming, in four kinds of generation, in three kinds of destiny, in one station of consciousness, and in one abode of beings, both in rebirth-linking (56) and in the course of an existence ((50)–(56)).

                            \vismParagraph{XVII.184}{184}{}
                            The formation of the imperturbable as rebirth-linking ripens in the immaterial becoming, in the apparitional kind of generation only, in the divine destiny only, in the three stations of consciousness beginning with the base consisting of boundless space, and in the four abodes of beings beginning with the base consisting of boundless space ((62)–(65)). Therefore it is a condition in the way already stated for the four kinds of consciousness in one kind of becoming, in one kind of generation, in one kind of destiny, in three stations of consciousness, and in four abodes of beings, both in rebirth-linking and in the course of becoming.\footnote{\vismAssertFootnoteCounter{34}\vismHypertarget{XVII.n34}{}A Sinhalese text adds the following paragraph: “Also the bodily formation, when giving rebirth-linking, gives the whole of its results in the sense-sphere becoming alone in the four generations, in the five destinies, in the first two stations of consciousness, and in two abodes of beings. Therefore it is a condition in the way already stated for the twenty-three kinds of consciousness in one kind of becoming, four generations, five destinies, two stations of consciousness, and two abodes of beings, both in rebirth-linking and in the course of an existence. The same method applies to the verbal formation. But the mental formation does not fail to ripen anywhere except in one abode of beings. Therefore it is a condition in the way already stated for the thirty-two kinds of resultant consciousness, as appropriate, in the three kinds of becoming, four generations, five destinies, seven stations of consciousness, and eight abodes of beings, both in rebirth-linking and in the course of an existence. There is no consciousness with formations as condition in the non-percipient abode of beings. Furthermore, in the case of non-percipient beings, the formation of merit is a condition, as kamma condition acting from a different time, for the kinds of materiality due to kamma performed.”}

                            \vismParagraph{XVII.185}{185}{}
                            This is how:
                            \begin{verse}
                                One should of these formations see\\{}
                                For which and how they are conditions\\{}
                                \marginnote{\textcolor{teal}{\footnotesize\{637|579\}}}{}In birth and life and the three\\{}
                                Kinds of becoming and the rest.
                            \end{verse}


                            This is the detailed explanation of the clause, “With formations as condition, consciousness.”
                    \subsubsection[\vismAlignedParas{§186–202}(4) Mentality-Materiality]{(4) Mentality-Materiality}

                        \vismParagraph{XVII.186}{186}{}
                        For the clause, “With consciousness as condition, mentality-materiality”:
                        \begin{verse}
                            (1) By analysis of mind and matter,\\{}
                            (2) Occurrence in becoming, etc.,\\{}
                            (3) Inclusion, and (4) manner of condition,\\{}
                            The exposition should be known.
                        \end{verse}


                        \vismParagraph{XVII.187}{187}{}
                        \emph{1. By analysis of mind and matter}: here “mind” (\emph{nāma—}mentality) is the three aggregates, that is, feeling, perception, and formations, because of their bending (\emph{namana}) on to the object. “Matter” (\emph{rūpa—}materiality) is the four great primary elements and the materiality derived [by clinging] from the four great primaries. Their analysis is given in the Description of the Aggregates (\hyperlink{XIV.34}{XIV.34f.}{}, \hyperlink{XIV.125}{125f.}{}). This, in the first place, is how the exposition of mentality-materiality should be known “by analysis.”

                        \vismParagraph{XVII.188}{188}{}
                        \emph{2. By occurrence in becoming, et cetera}: excepting one abode of beings, [that is, the non-percipient,] mentality occurs in all the kinds of becoming, generation, destiny, and station of consciousness, and in the remaining abodes of beings. Materiality occurs in two kinds of becoming, four kinds of generation, five destinies, the first four stations of consciousness, and the first five abodes of beings.

                        \vismParagraph{XVII.189}{189}{}
                        Now, when this mentality-materiality occurs thus, \textcolor{brown}{\textit{[559]}} then in the case of sexless embryos and the egg-born, at the moment of their rebirth-linking there are manifested as materiality two organic continuities, that is, the two decads of physical basis and body, and also the three immaterial aggregates. So in their case there are in detail these twenty-three states, namely, twenty states as concrete matter and three immaterial aggregates, which should be understood as “mentality-materiality with consciousness as condition.” But omitting repetitions,\footnote{\vismAssertFootnoteCounter{35}\vismHypertarget{XVII.n35}{}Resolve compound \emph{agahitagahaṇena }as \emph{gahitassa a-gahaṇena, }not as \emph{a-gahitassa gahaṇena; }i.e. it is “by not taking what is taken,” not “by taking what has not been taken”; cf. \hyperlink{IV.75}{IV.75}{}.} and so cancelling nine material instances (see 11.88) from one of the organic continuities, fourteen states remain.

                        By adding the sex decad for those possessed of sex [before making the above cancellation] there are thirty-three. And omitting repetitions and so cancelling eighteen material instances [nine each] from two of the organic continuities, in this case fifteen states remain.

                        \vismParagraph{XVII.190}{190}{}
                        At the moment of rebirth-linking of those of Brahmā’s Retinue, among apparitionally born beings, four organic continuities are manifested as materiality, that is, the decads of eye, ear, and physical basis, and the ennead of the life faculty, and three immaterial aggregates. So in their case in detail these \marginnote{\textcolor{teal}{\footnotesize\{638|580\}}}{}forty-two states, namely, thirty-nine states as concrete materiality and three immaterial aggregates, should be understood as “mentality-materiality with consciousness as condition.” But omitting repetitions and so cancelling twenty-seven instances of materiality [nine each] from three of the organic continuities, fifteen states remain.

                        \vismParagraph{XVII.191}{191}{}
                        In the sense-sphere becoming, seven organic continuities are manifested as materiality, and also three immaterial aggregates at the moment of rebirth-linking of the remaining kinds of apparitionally born or of the moisture-born possessing sex and matured sense bases. So in their case in detail these seventy-three states, namely, seventy instances of concrete materiality and three immaterial aggregates, should be understood as “mentality-materiality with formations as condition.” But omitting repetitions and so cancelling fifty-four material instances [nine each] from six of the organic continuities, nineteen states remain.

                        This is the maximum. But at minimum the computation of “mentality-materiality with consciousness as condition” in the rebirth-linking of those who lack such and such an organic continuity can be understood in brief and detail by reducing it appropriately. [The blind, for instance, lack the eye decad.]

                        \vismParagraph{XVII.192}{192}{}
                        For mentality-materiality immaterial beings have only the three [mental] aggregates; while non-percipient beings have only the life-faculty ennead, and that represents materiality.

                        \vismParagraph{XVII.193}{193}{}
                        In the course of an existence, in all places where materiality occurs there is manifested the temperature-originated bare [material] octad, which is due [initially] to the temperature that occurred together with the rebirth-linking consciousness at the moment of its presence.\footnote{\vismAssertFootnoteCounter{36}\vismHypertarget{XVII.n36}{}“This means, due to the heat element in the materiality that arose together with the rebirth-linking consciousness. It is because the heart-basis is arisen only at that very moment, that there is weakness of the physical basis” (\textbf{\cite{Vism-mhṭ}622}).} Rebirth-linking consciousness does not originate materiality. For, just as a man who is falling into a chasm cannot support another, so it, too, is unable to originate materiality because of its weakness, which is due to the weakness of the physical basis. But from the first life-continuum after the rebirth-linking consciousness onwards, \textcolor{brown}{\textit{[560]}} the bare octad originated by consciousness appears. And at the time when sound becomes manifest there is the sound ennead due both to temperature occurring after the moment of rebirth-linking and to consciousness.

                        \vismParagraph{XVII.194}{194}{}
                        The bare octad originated by nutriment appears in beings in the womb who live on matter consisting of physical nutriment as soon as their body is suffused by nutriment swallowed by the mother; for it is said:
                        \begin{verse}
                            And so it is that when his mother\\{}
                            Eats, consuming food and drink,\\{}
                            One hidden in his mother’s womb\\{}
                            Thereby obtains his nourishment (\textbf{\cite{S}I 206}).
                        \end{verse}


                        And it appears in apparitionally born beings as soon as they first swallow the spittle that has come into their own mouths. \marginnote{\textcolor{teal}{\footnotesize\{639|581\}}}{}So, with the twenty-six [material instances] consisting of the bare octad originated by nutriment, and of the, at most, two [sound] enneads originated respectively by temperature and consciousness, and also with the already-mentioned seventy kamma-originated instances (\hyperlink{XVII.191}{§191}{}) that arise three times in each conscious moment [at the instants of arising, presence, and dissolution], there are thus ninety-six material instances; and with the three immaterial aggregates there is thus a total of ninety-nine states.

                        \vismParagraph{XVII.195}{195}{}
                        Or because sound is not regularly present since it is only sometimes manifested, subtracting it therefore as twofold [being temperature-originated and consciousness-originated], there are these ninety-seven states to be understood as “mentality-materiality with consciousness as condition” in all beings, according as it happens to be produced. For whether these beings are sleeping or idling or eating or drinking, these states keep on occurring in them day and night with consciousness as condition. And we shall explain later how they have consciousness as their condition (see \hyperlink{XVII.200}{§200ff.}{}).

                        \vismParagraph{XVII.196}{196}{}
                        Now, although this kamma-born materiality is the first to find a footing in the several kinds of becoming, generation, destiny, station of consciousness, and abode of beings, it is nevertheless unable to carry on without being consolidated by materiality of triple origination [by consciousness, temperature, and nutriment], nor can that of triple origination do so without being consolidated by the former. But when they thus give consolidating support to each other, they can stand up without falling, like sheaves of reeds propped together on all four sides, even though battered by the wind, and like [boats with] broken floats\footnote{\vismAssertFootnoteCounter{37}\vismHypertarget{XVII.n37}{}\emph{Vāhanika—}“having a float”: not in PED. The context suggests a catamaran, universal in Indian waters.} that have found a support, even though battered by waves somewhere in mid-ocean, and they can last one year, two years … a hundred years, until those beings’ life span or their merit is exhausted.

                        This is how the exposition should be understood here “by occurrence in becoming, etc.”

                        \vismParagraph{XVII.197}{197}{}
                        \emph{3. By inclusion}: now there is (a) the simple mentality with consciousness as condition in both the course of an existence and rebirth-linking in the immaterial sphere, and in the course of an existence in the five-constituent becoming, and (b) the simple materiality with consciousness as condition in both cases among the non-percipient, and in the course of an existence in the five-constituent becoming, and (c) the [combined] mentality-materiality \textcolor{brown}{\textit{[561]}} with consciousness as condition in both cases in the five-constituent becoming. All that mentality and materiality and mentality-materiality should be understood as “mentality-materiality with consciousness as condition,” including them under mentality-materiality according to the method that allows any one part to represent any remaining one of its kind.\footnote{\vismAssertFootnoteCounter{38}\vismHypertarget{XVII.n38}{}The expression “\emph{ekadesasarūpekasesa” }is grammatically explained at \textbf{\cite{Vism-mhṭ}623}; see allied expressions, “\emph{katekasesa}” (§204) and “\emph{ekasese kate” }(§223). Cf. \emph{Pāṇini} I 2, 64.}

                        \vismParagraph{XVII.198}{198}{}
                        \marginnote{\textcolor{teal}{\footnotesize\{640|582\}}}{}Is this correct in view of the absence of consciousness in non-percipient beings?—It is not incorrect. For:
                        \begin{verse}
                            This consciousness, as cause of mind\\{}
                            And matter, is twice reckoned:\\{}
                            Result, and also not-result.\\{}
                            Wherefore this is correctly said.
                        \end{verse}


                        \vismParagraph{XVII.199}{199}{}
                        The consciousness that is the cause of mentality-materiality is reckoned to be twofold classed as resultant and not resultant. And since in the case of non-percipient beings materiality is originated by kamma, it has as its condition kamma-formation consciousness that occurred in the five-constituent becoming. This applies also to the kamma-originated materiality arising in the course of an existence in the five-constituent becoming at the moment of profitable or any other consciousness. So this is correct.

                        This is how the exposition can also be known here “by inclusion.”

                        \vismParagraph{XVII.200}{200}{}
                        \emph{4. By manner of condition}: here:
                        \begin{verse}
                            Resultant-consciousness conditions\\{}
                            Mentality first in nine ways,\\{}
                            Then basis matter in nine ways,\\{}
                            And other matter in eight ways;
                        \end{verse}

                        \begin{verse}
                            Formation-consciousness conditions\\{}
                            This matter in a single way.\\{}
                            The rest of consciousness conditions\\{}
                            This matter as the case may be.
                        \end{verse}


                        \vismParagraph{XVII.201}{201}{}
                        Rebirth-linking or some other kind of resultant consciousness is a condition in nine ways, as conascence, mutuality, support, association, kamma-result, nutriment, faculty, presence, and non-disappearance conditions, either at rebirth-linking or in the course of an existence, for that mentality called resultant, whether mixed with materiality or not. At rebirth-linking it is a condition in nine ways, as conascence, mutuality, support, kamma-result, nutriment, faculty, dissociation, presence, and non-disappearance conditions, for the materiality of the physical [heart-] basis. It is a condition in eight ways, namely, as the above conditions omitting the mutuality condition, for materiality other than the materiality of the physical basis.

                        Kamma-formation consciousness is a condition in one way only, as decisive-support condition, for the materiality of non-percipient beings, or for the kamma-born materiality in the five-constituent becoming, according to the Suttanta method.

                        All the remaining kinds of consciousness from the time of the first life-continuum [consciousness following rebirth-linking] onwards should be understood as a condition for some kind of mentality-materiality as appropriate. But since the whole contents of the Paṭṭhāna must be cited in order to show how it acts in detail, we do not undertake that.

                        \vismParagraph{XVII.202}{202}{}
                        Here it may be asked: “But how is it to be known \textcolor{brown}{\textit{[562]}} that the mentality-materiality of rebirth-linking has consciousness as its condition?” From the \marginnote{\textcolor{teal}{\footnotesize\{641|583\}}}{}suttas and from logic. For in the suttas it is established in many places that feeling, etc., have consciousness as condition in the way beginning, “States with parallel occurrence through consciousness” (\textbf{\cite{Dhs}§1522}). But as to logic:
                        \begin{verse}
                            From matter seen here to be born\\{}
                            Of consciousness a man can tell\\{}
                            That consciousness is a condition\\{}
                            For matter when unseen as well.
                        \end{verse}


                        Whether consciousness likes it or not, [certain] material instances are seen to arise in conformity with it. And the unseen is inferred from the seen. So it can be known, by means of the consciousness-born materiality that is seen, that consciousness is also a condition for the unseen materiality of rebirth-linking. For it is said in the Paṭṭhāna that, like the consciousness-originated, also the kamma-originated has consciousness as its condition (see \textbf{\cite{Paṭṭh}I 172–173}).

                        This is how the exposition should be known “by manner of condition.”

                        This is the detailed explanation of the clause “With consciousness as condition, mentality-materiality.”
                    \subsubsection[\vismAlignedParas{§203–219}(5) The Sixfold Base]{(5) The Sixfold Base}

                        \vismParagraph{XVII.203}{203}{}
                        As to the clause “With mentality-materiality as condition, the sixfold base”:
                        \begin{verse}
                            Three aggregates are “mind”; the basis,\\{}
                            Primaries, and the rest are “matter”:\\{}
                            And while all that conditions this\\{}
                            A part can represent the rest.
                        \end{verse}


                        \vismParagraph{XVII.204}{204}{}
                        In the case of the mentality-materiality that is here a condition for the sixfold base, mentality is the three aggregates beginning with feeling, while materiality should be understood as that included in one’s own continuity stated thus “primaries and the rest are ‘matter’,” that is to say, the four primaries, six physical bases, and life faculty, [since they are conditioning factors] invariably. But this mentality and this materiality and this mentality-materiality, each one representing the rest as “mentality-materiality,” should be understood as a condition for the sixfold base consisting of the sixth base and the sixfold base each one representing the rest as the “sixfold base.” Why? Because in the immaterial becoming there is only mentality as a condition, and that is a condition only for the sixth base, [namely, the mind base,] not for any other. For it is said in the Vibhaṅga, “With mentality as condition, the sixth base” (\textbf{\cite{Vibh}179}).

                        \vismParagraph{XVII.205}{205}{}
                        Here it may be asked: “But how is it to be known that mentality-materiality is a condition for the sixfold base?” Because the latter exists when mentality-materiality exists. For a given base exists when a given kind of mentality and materiality exists, not otherwise. But the way in which the one comes to exist when the other does will be explained below in the section dealing with how it is a condition. \textcolor{brown}{\textit{[563]}} Therefore:
                        \begin{verse}
                            A wise man should contrive to tell\\{}
                            Which one conditions which, and how,\\{}
                            \marginnote{\textcolor{teal}{\footnotesize\{642|584\}}}{}At rebirth and in life as well;\\{}
                            [The explanation follows now.]
                        \end{verse}


                        \vismParagraph{XVII.206}{206}{}
                        Herein what follows illustrates the meaning.
                        \par\noindent[\textsc{\textbf{1. Mentality as Condition}}]
                            \begin{verse}
                                In immaterial rebirth\\{}
                                And life the mind alone will come\\{}
                                In seven ways and six to be\\{}
                                Condition at the minimum.
                            \end{verse}


                            \vismParagraph{XVII.207}{207}{}
                            How? In rebirth-linking, firstly, mentality is a condition in seven ways at the minimum, as conascence, mutuality, support, association, kamma-result, presence, and non-disappearance conditions, for the sixth base. Some mentality, however, is a condition, as root-cause condition [that is, greed, etc.,] and some as nutriment condition [that is, contact and mental volition]. So it is also a condition in other ways. It is by the [two latter] that the maximum and minimum should be understood. In the course of an existence, too, resultant mentality is a condition as already stated. But the other [non-resultant] kind is a condition in six ways at minimum, as the aforesaid conditions except for kamma-result condition. Some, however, are a condition, as root-cause condition, and some as nutriment condition. So it is also a condition in other ways. It is by these that the maximum and minimum should be understood.

                            \vismParagraph{XVII.208}{208}{}
                            
                            \begin{verse}
                                In five-constituent becoming\\{}
                                At rebirth, mind in the same ways\\{}
                                Acts as condition for the sixth,\\{}
                                And for the others in six ways.
                            \end{verse}


                            \vismParagraph{XVII.209}{209}{}
                            Besides the immaterial states, also in the five-constituent becoming that resultant mentality, in association with the heart-basis, is a condition in seven ways at the minimum for the sixth, the mind base, in the same way as was said with respect to the immaterial states. But in association with the four primary elements, it is a condition in six ways, as conascence, support, kamma-result, dissociation, presence, and non-disappearance conditions, for the other five beginning with the eye base. Some, however, are a condition as root-cause condition, and some as nutriment condition. It is by these that the maximum and minimum should be understood.

                            \vismParagraph{XVII.210}{210}{}
                            
                            \begin{verse}
                                Result is for result condition\\{}
                                During a life in the same ways;\\{}
                                While non-result the non-resultant\\{}
                                Sixth conditions in six ways.
                            \end{verse}


                            \vismParagraph{XVII.211}{211}{}
                            For, as in rebirth-linking, so also in the course of an existence in the five-constituent becoming, resultant mentality is a condition in the seven ways at minimum for the resultant sixth base. But non-resultant mentality is a condition in six ways at minimum for the non-resultant sixth base, leaving out kamma-result condition. The maximum and minimum should be understood in the way already stated.

                            \vismParagraph{XVII.212}{212}{}
                            
                            \begin{verse}
                                \marginnote{\textcolor{teal}{\footnotesize\{643|585\}}}{}And during life, result conditions\\{}
                                The other five in fourfold way;\\{}
                                The non-resultant kind can be\\{}
                                Explained in the aforesaid way. \textcolor{brown}{\textit{[564]}}
                            \end{verse}


                            \vismParagraph{XVII.213}{213}{}
                            Again, in the course of an existence, the other resultant mentality, which has as its physical basis the eye sensitivity, etc., is a condition in four ways, as postnascence, dissociation, presence, and non-disappearance conditions, for the rest of the five beginning with the eye base. And as the resultant, so also the non-resultant is explained; therefore [the mentality] classed as profitable, etc., should be understood as their condition in four ways.

                            This, firstly, is how it should be understood what bases mentality alone is a condition for in rebirth-linking and in the course of an existence, and how it is a condition.
                        \par\noindent[\textsc{\textbf{2. Materiality as Condition}}]
                            \begin{verse}
                                Not even for a single base\\{}
                                In immaterial becoming\\{}
                                Is matter a condition here.\\{}
                                But in five-aggregate becoming
                            \end{verse}

                            \begin{verse}
                                Basis as matter is condition\\{}
                                At rebirth in a sixfold way\\{}
                                For the sixth base; the primaries\\{}
                                Are for the five in fourfold way.
                            \end{verse}


                            \vismParagraph{XVII.215}{215}{}
                            As to matter, the materiality of the physical [heart-] basis is a condition in rebirth-linking in six ways, as conascence, mutuality, support, dissociation, presence, and non-disappearance conditions, for the sixth, the mind base. But the four primaries are in general, that is to say, in rebirth-linking and in the course of an existence, conditions in four ways, as conascence, support, presence, and non-disappearance conditions, for any of the five bases beginning with the eye, whenever they arise.

                            \vismParagraph{XVII.216}{216}{}
                            Life and in lifetime food as well. Conditions five in threefold way; These five, the sixth in sixfold way; Basis, the sixth in fivefold way.

                            \vismParagraph{XVII.217}{217}{}
                            But in rebirth-linking and in the course of an existence the material life [faculty] is a condition in three ways, as presence, non-disappearance, and faculty conditions, for these five beginning with the eye. Nutriment too is a condition in three ways, as presence, non-disappearance, and nutriment conditions, and that is so in the course of an existence, not in rebirth-linking, and applies when the bodies of beings subsisting on nutriment are suffused with the nutriment. In the course of an existence, not in rebirth-linking, those five bases beginning with the eye are conditions in six ways, as support, prenascence, faculty, dissociation, presence and non-disappearance conditions, for [that part of] the sixth, the mind base, comprising eye-, ear-, nose-, tongue-, and body-consciousness. But in the course of an existence, not at rebirth-linking, the materiality of the [heart-]\marginnote{\textcolor{teal}{\footnotesize\{644|586\}}}{} basis is a condition in five ways, as support, prenascence, dissociation, presence, and non-disappearance conditions, for the remaining mind base apart from the five consciousnesses. This is how it should be understood what bases materiality alone is a condition for in rebirth-linking and in the course of an existence, and how it is a condition. \textcolor{brown}{\textit{[565]}}
                        \par\noindent[\textsc{\textbf{3. Mentality-Materiality as Condition}}]

                            \vismParagraph{XVII.218}{218}{}
                            
                            \begin{verse}
                                Which mind-cum-matter combination\\{}
                                Is a condition for which kind\\{}
                                And how it is so in each case,\\{}
                                A wise man should now seek to find.
                            \end{verse}


                            \vismParagraph{XVII.219}{219}{}
                            For example, firstly, in rebirth-linking in the five-constituent becoming, the mentality-materiality, in other words, the trio of aggregates with the materiality of the [heart-] basis, is a condition, as conascence, mutuality, support, kamma-result, association, dissociation, presence, and non-disappearance conditions, etc., for the sixth, the mind base. This is merely the heading; but since it can all be construed in the way already stated, the detail is not given here.

                            This is the detailed explanation of the clause “With mentality-materiality as condition, the sixfold base.”
                    \subsubsection[\vismAlignedParas{§220–227}(6) Contact]{(6) Contact}

                        \vismParagraph{XVII.220}{220}{}
                        As to the clause “With the sixfold base as condition, contact”:
                        \begin{verse}
                            Contact is briefly of six kinds\\{}
                            With eye-contact and others too;\\{}
                            According to each consciousness\\{}
                            It is in detail thirty-two.
                        \end{verse}


                        \vismParagraph{XVII.221}{221}{}
                        Briefly, with the clause “With the sixfold base as condition, contact,” there are only the six kinds beginning with eye-contact, that is to say, eye-contact, ear-contact, nose-contact, tongue-contact, body-contact, and mind-contact. But in detail the five profitable resultant and the five unprofitable resultant beginning with eye-contact make ten; the rest, which are associated with the twenty-two kinds of mundane resultant consciousness, make twenty-two. So all these come to thirty-two ((34)–(65)), like the consciousness with formations as condition given above.

                        \vismParagraph{XVII.222}{222}{}
                        But as to the sixfold base that is a condition for this thirty-twofold contact. Herein:
                        \begin{verse}
                            Some wise men take the sixfold base\\{}
                            To be the five internal bases\\{}
                            With the sixth; but others count\\{}
                            These plus the six external bases.
                        \end{verse}


                        \vismParagraph{XVII.223}{223}{}
                        Herein, firstly, there are those who take this to be an exposition of the occurrence of what is clung to, [that is, kammically-acquired aggregates,] and they maintain that the conditioning [bases] and the conditionally-arisen [contact] are only what is included in one’s own continuity. They take any one part to \marginnote{\textcolor{teal}{\footnotesize\{645|587\}}}{}represent any remaining one of its kind, since the condition for contact in the immaterial states is the sixth base [only], according to the text “With the sixth base as condition, contact” (\textbf{\cite{Vibh}179}), and elsewhere it is the sixfold base inclusively. So they have it that “sixfold base” means the internal [five] beginning with the eye plus the sixth (mind) base. For that sixth base and that sixfold base are styled “sixfold base.” But there are those who maintain that it is only the conditionally-arisen [contact] that is contained in a single continuity, while the conditioning [bases] are contained in separate [that is, past] continuities as well. They maintain that all and any such bases are a condition for contact, and they include also the [six] external ones. So they have it that “sixfold base” means the same internal [five] plus the sixth plus the external ones beginning with visible data. For that sixth base and that [partial] sixfold base and the sixfold base along with these [external ones] each representing the rest \textcolor{brown}{\textit{[566]}} are styled sixfold base too.

                        \vismParagraph{XVII.224}{224}{}
                        Here it may be asked: “One kind of contact does not derive from all the bases, nor all the kinds of contact from one base. And yet ‘With the sixfold base as condition, contact’ is said in the singular. Why is that?”

                        \vismParagraph{XVII.225}{225}{}
                        Here is the answer: It is true that neither is one derived from all nor all from one. However, one is derived from many. For eye-contact is derived from the eye base, from the visible-data base, from the mind base reckoned as eye-consciousness, and from the mental-datum base consisting of the remaining associated states. And each case should be construed as appropriate in this way. Therefore:
                        \begin{verse}
                            Though stated in the singular,\\{}
                            He shows therewith in all such cases\\{}
                            That this contact, though only one,\\{}
                            Is yet derived from several bases.
                        \end{verse}


                        \emph{Though stated in the singular}: the meaning is, by this statement in the singular that “With the sixfold base as condition, contact,” it is pointed out by the Blessed One (\emph{Tādin}) that contact, which is of one kind, comes into being from many bases.
                        \par\noindent[\textsc{\textbf{How the Sixfold Base is a Condition for Contact}}]

                            \vismParagraph{XVII.226}{226}{}
                            But as regards these bases:
                            \begin{verse}
                                Five in six ways; and after that\\{}
                                One in nine ways; the external six\\{}
                                As contact’s conditionality\\{}
                                According to each case we fix.
                            \end{verse}


                            \vismParagraph{XVII.227}{227}{}
                            Here is the explanation: firstly, the five consisting of the eye base, etc., are conditions in six ways, as support, prenascence, faculty, dissociation, presence, and non-disappearance conditions, for contact classed in five ways as eye-contact, and so on. After that, the single resultant mind base is a condition in nine ways, as conascence, mutuality, support, result, nutriment, faculty, association, presence, and non-disappearance conditions, for the variously-classed\marginnote{\textcolor{teal}{\footnotesize\{646|588\}}}{} resultant mind contact. But in the case of the external bases, the visible-data base is a condition in four ways, as object, prenascence, presence, and non-disappearance conditions, for eye-contact. Likewise the sound base, etc., respectively for ear-contact, and so on. But these and mental data as object are conditions likewise, and as object condition too, for mind-contact, so “the external six as contact’s conditionality according to each case we fix.”

                            This is the detailed explanation of the clause “With the sixfold base as condition, contact.”
                    \subsubsection[\vismAlignedParas{§228–232}(7) Feeling]{(7) Feeling}

                        \vismParagraph{XVII.228}{228}{}
                        As to the clause “With contact as condition, feeling”:
                        \begin{verse}
                            Feelings, when named by way of door\\{}
                            “Eye-contact-born” and all the rest,\\{}
                            Are only six; but then they are\\{}
                            At nine and eighty sorts assessed.
                        \end{verse}


                        \vismParagraph{XVII.229}{229}{}
                        In the analysis of this clause [in the Vibhaṅga] only six kinds of feeling according to door are given thus, “Eye-contact-born feeling, ear-, nose-, tongue-, body-, mind-contact-born feeling” (\textbf{\cite{Vibh}136}). \textcolor{brown}{\textit{[567]}} Still, when classed according to association with the eighty-nine kinds of consciousness, they are “at nine and eighty sorts assessed.”

                        \vismParagraph{XVII.230}{230}{}
                        
                        \begin{verse}
                            But from the nine and eighty feelings\\{}
                            Thirty-two, no more, appear\\{}
                            Associated with result,\\{}
                            And only those are mentioned here.
                        \end{verse}

                        \begin{verse}
                            Herein, contact in the five doors\\{}
                            Conditions five in eightfold way,\\{}
                            And single way the rest; it acts\\{}
                            In the mind door in the same way.
                        \end{verse}


                        \vismParagraph{XVII.231}{231}{}
                        Herein, in the five doors contact beginning with eye-contact is a condition in eight ways, as conascence, mutuality, support, result, nutriment, association, presence, and non-disappearance conditions, for the five kinds of feeling that have respectively eye sensitivity, etc., as their physical basis. But that contact beginning with eye-contact is a condition in one way only, as decisive-support condition, for the rest of resultant feeling in the sense sphere occurring in each door as receiving, investigation and registration.

                        \vismParagraph{XVII.232}{232}{}
                        \emph{In the mind door in the same way: the contact} called conascent mind-contact is also a condition in the same eight ways for sense-sphere resultant feeling occurring as registration in the mind door, and so also for the kinds of resultant feeling in the three planes occurring with rebirth-linking, life-continuum and death. But the mind-contact associated with mind-door adverting is a condition in one way only, as decisive-support condition, for the kinds of feeling that occur in the mind door as registration in the sense sphere.

                        This is the detailed explanation of the clause “With contact as condition, feeling.”

                        
                    \subsubsection[\vismAlignedParas{§233–238}(8) Craving]{(8) Craving}

                        \vismParagraph{XVII.233}{233}{}
                        \marginnote{\textcolor{teal}{\footnotesize\{647|589\}}}{}As regards the clause “With feeling as condition, craving”:
                        \begin{verse}
                            Six cravings, for things visible\\{}
                            and all the rest, are treated here;\\{}
                            And each of these, when it occurs,\\{}
                            Can in one of three modes appear.
                        \end{verse}


                        \vismParagraph{XVII.234}{234}{}
                        Six kinds of craving are shown in the analysis of this clause [in the Vibhaṅga] as “visible-data craving, sound, odour, flavour, tangible-data, and mental-data craving” (\textbf{\cite{Vibh}136}), called after their objects, as a son is called after his father “banker’s son,” “brahman’s son.” Each of these six kinds of craving is reckoned threefold according to its mode of occurrence as craving for sense desires, craving for becoming, or craving for non-becoming.

                        \vismParagraph{XVII.235}{235}{}
                        When visible-data craving occurs enjoying with sense-desire enjoyment a visible datum as object that has come into the focus of the eye, it is called craving for sense desires. But when [that same visible-data craving] occurs along with the eternity view that assumes that same object to be lasting and eternal, \textcolor{brown}{\textit{[568]}} it is called craving for becoming; for it is the greed accompanying the eternity view that is called craving for becoming. When it occurs along with the annihilation view that assumes that same object to break up and be destroyed, it is called craving for non-becoming; for it is the greed accompanying the annihilation view that is called craving for non-becoming. So also in the case of craving for sounds, and so on.

                        These amount to eighteen kinds of craving. The eighteen with respect to one’s own visible data (one’s own appearance), etc., and eighteen with respect to external [visible data (another’s appearance), etc.,] together make thirty-six kinds. Thirty-six in the past, thirty-six in the future, and thirty-six in the present, make one-hundred-and-eight kinds of craving. When these are reduced again, they should be understood to amount to the six kinds only with visible data, etc., as object; and these, to three only, as craving for sense desires, and so on.

                        \vismParagraph{XVII.236}{236}{}
                        Out of selfish affection for feeling after taking pleasure in it when it arises through a visible datum as object, etc., these beings accord much honour to painters, musicians, perfumers, cooks, weavers, distillers of elixirs,\footnote{\vismAssertFootnoteCounter{39}\vismHypertarget{XVII.n39}{}\emph{Rasāyana—}“elixir”: not in PED; cf. \textbf{\cite{D-a}568}and \textbf{\cite{Ud-a}} (commentary to \textbf{\cite{Ud}8.5})} physicians, etc., who furnish respectively visible data as object, etc., just as out of affection for a child they reward the child’s nurse after taking pleasure in the child. That is why it should be understood that these three kinds of craving have feeling as their condition.

                        \vismParagraph{XVII.237}{237}{}
                        
                        \begin{verse}
                            What is intended here is but\\{}
                            Resultant pleasant feeling; hence\\{}
                            ’Tis a condition in one way\\{}
                            For all this craving’s occurrence.
                        \end{verse}


                        \emph{In one way}: it is a condition as decisive-support condition only.

                        \vismParagraph{XVII.238}{238}{}
                        Or alternatively:
                        \begin{verse}
                            \marginnote{\textcolor{teal}{\footnotesize\{648|590\}}}{}A man in pain for pleasure longs,\\{}
                            And finding pleasure, longs for more;\\{}
                            The peace of equanimity\\{}
                            Is counted pleasure too; therefore
                        \end{verse}

                        \begin{verse}
                            The Greatest Sage announced the law\\{}
                            “With feeling as condition, craving,”\\{}
                            Since all three feelings thus can be\\{}
                            Conditions for all kinds of craving.
                        \end{verse}

                        \begin{verse}
                            Though feeling is condition, still\\{}
                            Without inherent tendency\\{}
                            No craving can arise, and so\\{}
                            From this the perfect saint is free.\footnote{\vismAssertFootnoteCounter{40}\vismHypertarget{XVII.n40}{}“‘\emph{Though feeling is condition’ }is said in order to prevent a generalization from the preceding words ‘With feeling as condition’ to the effect that craving arises in the presence of every condition accompanied by feeling—But is it not impossible to prevent over-generalization in the absence of any such statements as ‘Feeling accompanied by inherent tendency is a condition for craving’?—No; for we are dealing with an exposition of the round of rebirths. Since there is no round of rebirths without inherent tendencies, so far as the meaning is concerned it may be taken for granted that the condition is accompanied by inherent tendency. Or alternatively, it may be recognized that this condition is accompanied by inherent tendency because it follows upon the words ‘With ignorance as condition.’ And with the words ‘With feeling as condition, craving’ the ruling needed is this: ‘There is craving only with feeling as condition,’ and not ‘With feeling as condition there is only craving’” (\textbf{\cite{Vism-mhṭ}}). For inherent tendencies see \hyperlink{XXII.45}{XXII.45}{}, \hyperlink{XXII.60}{60}{}; MN 64. The Arahant has none.}
                        \end{verse}


                        This is the detailed explanation of the clause “With feeling as condition, craving.”

                        
                    \subsubsection[\vismAlignedParas{§239–248}(9) Clinging]{(9) Clinging}

                        \vismParagraph{XVII.239}{239}{}
                        As regards the clause “With craving as condition, clinging”:
                        \begin{verse}
                            Four clingings need to be explained\\{}
                            (1) As to analysis of meaning,\\{}
                            (2) As to the brief and full account\\{}
                            Of states, (3) and also as to order. \textcolor{brown}{\textit{[569]}}
                        \end{verse}


                        \vismParagraph{XVII.240}{240}{}
                        Herein, this is the explanation: firstly, there are these four kinds of clinging here, namely, sense-desire clinging, [false-] view clinging, rules-and-vows clinging, and self-doctrine clinging.

                        \vismParagraph{XVII.241}{241}{}
                        \emph{1.} The \emph{analysis of meaning} is this: it clings to the kind of sense-desire called sense-desire’s physical object (see \hyperlink{IV.n24}{Ch. IV, n. 24}{}), thus it is sense-desire clinging. Also, it is sense-desire and it is clinging, thus it is sense-desire clinging. Clinging (\emph{upādāna}) is firm grasping; for here the prefix \emph{upa} has the sense of firmness, as in \emph{upāyāsa} (great misery—see \hyperlink{XVII.48}{§48}{}) and \emph{upakuṭṭha} (great pox),\footnote{\vismAssertFootnoteCounter{41}\vismHypertarget{XVII.n41}{}\emph{Upakuṭṭha—}“great pox” or “great leprosy”: not in PED; see \emph{kuṭṭha.}} and so on. Likewise, it is [false] view and it is clinging, thus it is [false-] view clinging; or, it clings to \marginnote{\textcolor{teal}{\footnotesize\{649|591\}}}{}[false] view, thus it is [false-] view clinging; for in [the case of the false view] “The world and self are eternal” (\textbf{\cite{D}I 14}), etc., it is the latter kind of view that clings to the former. Likewise, it clings to rite and ritual, thus it is rules-and-vows clinging; also, it is rite and ritual and it is clinging, thus it is rules-and-vows clinging; for ox asceticism, ox vows, etc. (see \textbf{\cite{M}I 387f.}), are themselves kinds of clinging, too, because of the misinterpretation (insistence) that purification comes about in this way. Likewise, they indoctrinate by means of that, thus that is doctrine; they cling by means of that, thus that is clinging. What do they indoctrinate with? What do they cling to? Self. The clinging to doctrines of self is self-doctrine clinging. Or by means of that they cling to a self that is a mere doctrine of self; thus that is self-doctrine clinging. This, firstly, is the “analysis of meaning.”

                        \vismParagraph{XVII.242}{242}{}
                        \emph{2. }But as regards \emph{the brief and full account of states}, firstly, in brief sense-desire clinging is called “firmness of craving” since it is said: “Herein, what is sense-desire clinging? That which in the case of sense desires is lust for sense desires, greed for sense desires, delight in sense desires, craving for sense desires, fever of sense desires, infatuation with sense desires, committal to sense desires: that is called sense-desire clinging” (\textbf{\cite{Dhs}§1214}). “Firmness of craving” is a name for the subsequent craving itself, which has become firm by the influence of previous craving, which acts as its decisive-support condition. But some have said: Craving is the aspiring to an object that one has not yet reached, like a thief’s stretching out his hand in the dark; clinging is the grasping of an object that one has reached, like the thief’s grasping his objective. These states oppose fewness of wishes and contentment and so they are the roots of the suffering due to seeking and guarding (see \textbf{\cite{D}II 58f.}). The remaining three kinds of clinging are in brief simply [false] view.

                        \vismParagraph{XVII.243}{243}{}
                        In detail, however, sense-desire clinging is the firm state of the craving described above as of one-hundred-and-eight kinds with respect to visible data and so on. [False-] view clinging is the ten-based wrong view, according as it is said: “Herein what is [false-] view clinging? There is no giving, no offering, … [no good and virtuous ascetics and brahmans who have themselves] realized by direct-knowledge and declare this world and the other world: such view as this … such perverse assumption is called [false-]view clinging” (\textbf{\cite{Vibh}375}; \textbf{\cite{Dhs}§1215}). Rules-and-vows clinging is the adherence [to the view that] purification comes through rules and vows, according as it is said: “Herein, what is rules-and-vows clinging? … That purification comes through a rite, that purification comes through a ritual, \textcolor{brown}{\textit{[570]}} that purification comes through a rite and ritual: such view as this … such perverse assumption is called rules-and-vows clinging” (\textbf{\cite{Dhs}§1216}). Self-doctrine clinging is the twenty-based [false] view of individuality, according as it is said: “Herein, what is self-doctrine clinging? Here the untaught ordinary man … untrained in good men’s Dhamma, sees materiality as self … such perverse assumption is called self-doctrine clinging” (\textbf{\cite{Dhs}§1217}). This is the “brief and full account of states.”

                        \vismParagraph{XVII.244}{244}{}
                        \emph{3. As to order}: here order is threefold (see \hyperlink{XIV.211}{XIV.211}{}), that is to say, order of arising, order of abandoning, and order of teaching. Herein, order of arising of \marginnote{\textcolor{teal}{\footnotesize\{650|592\}}}{}defilements is not meant literally because there is no first arising of defilements in the beginningless round of rebirths. But in a relative sense it is this: usually in a single becoming the misinterpretation of (insistence on) eternity and annihilation are preceded by the assumption of a self. After that, when a man assumes that this self is eternal, rules-and-vows clinging arises in him for the purpose of purifying the self. And when a man assumes that it breaks up, thus disregarding the next world, sense-desire clinging arises in him. So self-doctrine clinging arises first, and after that, [false-] view clinging, and rules-and-vows clinging or sense-desire clinging. This, then, is their order of arising in one becoming.

                        \vismParagraph{XVII.245}{245}{}
                        And here [false-] view clinging, etc., are abandoned first because they are eliminated by the path of stream-entry. Sense-desire clinging is abandoned later because it is eliminated by the path of Arahantship. This is the order of their abandoning.

                        \vismParagraph{XVII.246}{246}{}
                        Sense-desire clinging, however, is taught first among them because of the breadth of its objective field and because of its obviousness. For it has a broad objective field because it is associated with eight kinds of consciousness ((22)–(29)). The others have a narrow objective field because they are associated with four kinds of consciousness ((22), (23), (26) and (27)). And usually it is sense-desire clinging that is obvious because of this generation’s love of attachment (see \textbf{\cite{M}I 167}), not so the other kinds. One possessed of sense-desire clinging is much given to display and ceremony (see \textbf{\cite{M}I 265}) for the purpose of attaining sense desires. [False-] view clinging comes next to the [sense-desire clinging] since that [display and ceremony] is a [false-] view of his.\footnote{\vismAssertFootnoteCounter{42}\vismHypertarget{XVII.n42}{}Ee has “\emph{sassatan ti}”; Ae has “\emph{sā’ssa diṭṭhī ti”; }\textbf{\cite{Vibh-a}} (Be), “\emph{na sassatadiṭṭhī ti}.”} And that is then divided in two as rules-and-vows clinging and self-doctrine clinging. And of these two, rules-and-vows clinging is taught first, being gross, because it can be recognized on seeing [it in the forms of] ox practice and dog practice. And self-doctrine clinging is taught last because of its subtlety. This is the “order of teaching.”
                        \par\noindent[\textsc{\textbf{How Craving is a Condition for Clinging}}]

                            \vismParagraph{XVII.247}{247}{}
                            
                            \begin{verse}
                                For the first in a single way;\\{}
                                But for the three remaining kinds\\{}
                                In sevenfold or eightfold way.
                            \end{verse}


                            \vismParagraph{XVII.248}{248}{}
                            As regards the four kinds of clinging taught in this way, craving for sense desires is a condition in one way, as decisive-support, for the first kind, namely, sense-desire clinging, because it arises in relation to the objective field in which craving delights. But it is a condition in seven ways, as conascence, mutuality, support, association, presence, non-disappearance, and root-cause, or in eight ways, as [those and] decisive-support as well, for the remaining three kinds. And when it is a condition as decisive-support, then it is never conascent.

                            \marginnote{\textcolor{teal}{\footnotesize\{651|593\}}}{}This is the detailed explanation of the clause “With craving as condition, clinging.” \textcolor{brown}{\textit{[571]}}

                            
                    \subsubsection[\vismAlignedParas{§249–269}(10) Becoming (being)]{(10) Becoming (being)}

                        \vismParagraph{XVII.249}{249}{}
                        As to the clause “With clinging as condition, becoming”:
                        \begin{verse}
                            (1) As to meaning, (2) as to state,\\{}
                            (3) Purpose, (4) analysis, (5) synthesis,\\{}
                            (6) And which for which becomes condition,\\{}
                            The exposition should be known.
                        \end{verse}


                        \vismParagraph{XVII.250}{250}{}
                        \emph{1. As to meaning}: Herein, it becomes (\emph{bhavati}), thus it is becoming (\emph{bhava}). That is twofold as kamma-process becoming and rebirth-process becoming, according as it is said: “Becoming in two ways: there is kamma-process becoming and there is rebirth-process becoming” (\textbf{\cite{Vibh}137}). Herein, the kamma process itself as becoming is “kamma-process becoming”; likewise the rebirth process itself as becoming is “rebirth-process becoming.” And here, rebirth is becoming since it becomes; but just as “The arising of Buddhas is bliss” (\textbf{\cite{Dhp}194}) is said because it causes bliss, so too kamma should be understood as “becoming,” using for it the ordinary term for its fruit, since it causes becoming. This, firstly, is how the exposition should be known here “as to meaning.”

                        \vismParagraph{XVII.251}{251}{}
                        \emph{2. As to state}: firstly, kamma-process becoming in brief is both volition and the states of covetousness, etc., associated with the volition and reckoned as kamma too, according as it is said: “Herein, what is kamma-process becoming? The formation of merit, the formation of demerit, the formation of the imperturbable, either with a small (limited) plane or with a large (exalted) plane: that is called kamma-process becoming. Also all kamma that leads to becoming is called kamma-process becoming” (\textbf{\cite{Vibh}137}).

                        \vismParagraph{XVII.252}{252}{}
                        Here the formation of merit is, in terms of states, the thirteen kinds of volition ((1)–(13)), the formation of demerit is the twelve kinds ((22)–(33)), and the formation of the imperturbable is the four kinds ((14)–(17)). So with the words \emph{either with a small (limited) plane or with a large (exalted) plane} the insignificance or magnitude of these same volitions’ result is expressed here. But with the words \emph{also all kamma that leads to becoming} the covetousness, etc., associated with volition are expressed.

                        \vismParagraph{XVII.253}{253}{}
                        Rebirth-process becoming briefly is aggregates generated by kamma. It is of nine kinds, according as it is said: “Herein, what is rebirth-process becoming? Sense-desire becoming, fine-material becoming, immaterial becoming, percipient becoming, non-percipient becoming, neither-percipient-nor-non-percipient becoming, one-constituent becoming, \textcolor{brown}{\textit{[572]}} four-constituent becoming, five-constituent becoming: this is called rebirth-process becoming” (\textbf{\cite{Vibh}17}).

                        \vismParagraph{XVII.254}{254}{}
                        Herein, the kind of becoming called “having sense desires” is \emph{sense-desire becoming}. Similarly with the \emph{fine-material} and \emph{immaterial} kinds of becoming. It is the becoming of those possessed of perception, or there is perception here in becoming, thus it is \emph{percipient becoming}. The opposite kind is \emph{non-percipient becoming}. Owing to the absence of gross perception and to the presence of subtle \marginnote{\textcolor{teal}{\footnotesize\{652|594\}}}{}perception there is neither perception nor non-perception in that kind of becoming, thus it is \emph{neither-percipient-nor-non-percipient becoming}. It is becoming constituted out of the materiality aggregate only, thus it is \emph{one-constituent becoming}, or that kind of becoming has only one constituent, [the materiality aggregate, or dimension,] thus it is one-constituent becoming. And similarly the \emph{four-constituent }[has the four mental aggregates, or dimensions,] and the \emph{five-constituent} [has the material and the four mental aggregates, or dimensions].

                        \vismParagraph{XVII.255}{255}{}
                        Herein, sense-desire becoming is five aggregates acquired through kamma (clung to). Likewise the fine-material becoming. Immaterial becoming is four. Percipient becoming is four and five. Non-percipient becoming is one aggregate that is acquired through kamma (clung to). Neither-percipient-nor-non-percipient becoming is four. One-constituent becoming, etc., are respectively one, four, and five aggregates as aggregates that are acquired through kamma (clung to).

                        This is how the exposition should be known here “as to state.”

                        \vismParagraph{XVII.256}{256}{}
                        \emph{3. As to purpose}: although formations of merit, etc., are of course dealt with in the same way in the description of becoming and in the description of formations (see \textbf{\cite{Vibh}135}, 137), nevertheless the repetition has a purpose. For in the former case it was because it was a condition, as past kamma, for rebirth-linking here [in this becoming], while in the latter case it is because it is a condition, as present kamma, for rebirth-linking in the future [becoming]. Or alternatively, in the former instance, in the passage beginning, “Herein, what is the formation of merit? It is profitable volition of the sense sphere” (\textbf{\cite{Vibh}135}), it was only volitions that were called “formations”; but here, with the words “All kamma that leads to becoming” (\textbf{\cite{Vibh}137}), the states associated with the volition are also included. And in the former instance it was only such kamma as is a condition for consciousness that was called ‘formations’; but now also that which generates non-percipient becoming is included.

                        \vismParagraph{XVII.257}{257}{}
                        But why so many words? In the clause “With ignorance as condition there are formations,” only profitable and unprofitable states are expressed as the formation of merit, etc.; but in the clause “With clinging as condition, becoming,” profitable and unprofitable and also functional states are expressed because of the inclusion of rebirth-process becoming. So this repetition has a purpose in each case. This is how the exposition should be known “as to purpose here.”

                        \vismParagraph{XVII.258}{258}{}
                        \emph{4. As to analysis, synthesis} means as to both the analysis and the synthesis of becoming that has clinging as its condition. The kamma with sense-desire clinging as its condition that is performed and generates sense-desire becoming is “kamma-process becoming.” The aggregates generated by that are “rebirth-process becoming”; similarly in the case of fine-material and immaterial becoming. So \textcolor{brown}{\textit{[573]}} there are two kinds of sense-desire becoming with sense-desire clinging as condition, included in which are percipient becoming and five-constituent becoming. And there are two kinds of fine-material becoming, included in which are percipient, non-percipient, one-constituent, and five-constituent becoming. And there are two kinds of immaterial becoming, included in which are percipient becoming, neither-percipient-nor-non-percipient \marginnote{\textcolor{teal}{\footnotesize\{653|595\}}}{}becoming, and four-constituent becoming. So, together with what is included by them, there are six kinds of becoming with sense-desire clinging as condition. Similarly too with the [three] remaining kinds of clinging as condition. So, as to analysis, there are, together with what is included by them, twenty-four kinds of becoming with clinging as condition.

                        \vismParagraph{XVII.259}{259}{}
                        \emph{5. As to synthesis}, however, by uniting kamma-process becoming and rebirth-process becoming there is, together with what is included by it, one kind of sense-desire becoming with sense-desire clinging as its condition. Similarly with fine-material and immaterial becoming. So there are three kinds of becoming. And similarly with the remaining [three] kinds of clinging as condition. So by synthesis, there are, together with what is included by them, twelve kinds of becoming with clinging as condition.

                        \vismParagraph{XVII.260}{260}{}
                        Furthermore, without distinction the kamma with clinging as its condition that attains sense-desire becoming is kamma-process becoming. The aggregates generated by that are rebirth-process becoming. Similarly in the fine-material and immaterial becoming. So, together with what is included by them, there are two kinds of sense-desire becoming, two kinds of fine-material becoming, and two kinds of immaterial becoming. So, by synthesis, there are six kinds of becoming by this other method. Or again, without making the division into kamma-process becoming and rebirth-process becoming, there are, together with what is included by them, three kinds of becoming as sense-desire becoming, and so on. Or again, without making the division into sense-desire becoming, etc., there are, together with what is included by them, two kinds of becoming, as kamma-process becoming and rebirth-process becoming. And also without making the division into kamma process and rebirth process there is, according to the words “With clinging as condition, becoming,” only one kind of becoming.

                        This is how the exposition of becoming with clinging as condition should be known here “as to analysis and synthesis.”

                        

                        \vismParagraph{XVII.261}{261}{}
                        \emph{6. Which for which becomes condition} means that here the exposition should be known according to what kind of clinging is a condition for what [kind of becoming]. But what is condition for what here? Any kind is a condition for any kind. For the ordinary man is like a madman, and without considering “Is this right or not?” and aspiring by means of any of the kinds of clinging to any of the kinds of becoming, he performs any of the kinds of kamma. Therefore when some say that the fine-material and immaterial kinds of becoming do not come about through rules-and-vows clinging, that should not be accepted: what should be accepted is that all kinds come about through all kinds.

                        \vismParagraph{XVII.262}{262}{}
                        For example, someone thinks in accordance with hearsay or [false] view that sense desires come to be fulfilled in the human world among the great warrior (\emph{khattiya}) families, etc., and in the six divine worlds of the sense sphere. \textcolor{brown}{\textit{[574]}} Misled by listening to wrong doctrine, etc., and imagining that “by this kamma sense desires will come to be fulfilled,” he performs for the purpose of attaining them acts of bodily misconduct, etc., through sense-desire clinging. By fulfilling such misconduct he is reborn in the states of loss. Or he performs acts of bodily misconduct, etc., aspiring to sense desires visible here and now \marginnote{\textcolor{teal}{\footnotesize\{654|596\}}}{}and protecting those he has already acquired. By fulfilling such misconduct he is reborn in the states of loss. The kamma that is the cause of rebirth there is kamma-process becoming. The aggregates generated by the kamma are rebirth-process becoming. But percipient becoming and five-constituent becoming are included in that, too.

                        \vismParagraph{XVII.263}{263}{}
                        Another, however, whose knowledge has been intensified by listening to good Dhamma and so on, imagines that “by this kind of kamma sense desires will come to be fulfilled.” He performs acts of bodily good conduct, etc., through sense-desire clinging. By fulfilling such bodily good conduct he is reborn among deities or human beings. The kamma that is the cause of his rebirth there is kamma-process becoming. The aggregates generated by the kamma are rebirth-process becoming. But percipient becoming and five-constituent becoming are included in that, too.

                        So sense-desire clinging is a condition for sense-desire becoming with its analysis and its synthesis.

                        \vismParagraph{XVII.264}{264}{}
                        Another hears or conjectures that sense desires come to still greater perfection in the fine-material and immaterial kinds of becoming, and through sense-desire clinging he produces the fine-material and immaterial attainments, and in virtue of his attainments he is reborn in the fine-material or immaterial Brahmā-world. The kamma that is the cause of his rebirth there is kamma-process becoming. The aggregates generated by the kamma are rebirth-process becoming. But percipient, non-percipient, neither-percipient-nor-non-percipient, one-constituent, four-constituent, and five-constituent kinds of becoming are included in that, too. Thus sense-desire clinging is a condition for fine-material and immaterial becoming with its analysis and its synthesis.

                        \vismParagraph{XVII.265}{265}{}
                        Another clings to the annihilation view thus: “This self comes to be entirely cut off when it is cut off in the fortunate states of the sense sphere, or in the fine-material or immaterial kinds of becoming,” and he performs kamma to achieve that. His kamma is kamma-process becoming. The aggregates generated by the kamma are rebirth-process becoming. But the percipient, etc., kinds of becoming are included in that too. So [false-]view clinging is a condition for all three, namely, for the sense-desire, fine-material, and immaterial kinds of becoming with their analysis and their synthesis.

                        \vismParagraph{XVII.266}{266}{}
                        Another through self-theory clinging thinks, “This self comes to be blissful, or comes to be free from fever, in the becoming in the fortunate states in the sense sphere or in one or other of the fine-material and immaterial kinds of becoming,” and he performs kamma to achieve that. That kamma of his is kamma-process becoming. The aggregates generated by the kamma are \textcolor{brown}{\textit{[575]}} rebirth-process becoming. But the percipient, etc., kinds of becoming are included in that, too. Thus this self-theory clinging is a condition for all the three, namely, becoming with their analysis and their synthesis.

                        \vismParagraph{XVII.267}{267}{}
                        Another [thinks] through rules-and-vows clinging, “This rite and ritual leads him who perfects it to perfect bliss in becoming in the fortunate states of the sense sphere or in the fine-material or immaterial kinds of becoming,” and he performs kamma to achieve that. That kamma of his is kamma-process \marginnote{\textcolor{teal}{\footnotesize\{655|597\}}}{}becoming. The aggregates generated by the kamma are rebirth-process becoming. But the percipient, etc., kinds of becoming are included in that, too. So rules-and-vows clinging is a condition for all three, namely, the sense-desire, fine-material and immaterial kinds of becoming with their analysis and their synthesis.

                        This is how the exposition should be known here according to “which is condition for which.”
                        \par\noindent[\textsc{\textbf{How Clinging is a Condition for Becoming}}]

                            \vismParagraph{XVII.268}{268}{}
                            But which is condition for which kind of becoming in what way here?
                            \begin{verse}
                                Now, clinging as condition for becoming,\\{}
                                Both fine-material and immaterial,\\{}
                                Is decisive-support; and then conascence\\{}
                                And so on for the sense-desire kind.
                            \end{verse}


                            \vismParagraph{XVII.269}{269}{}
                            This clinging, though fourfold, is a condition in only one way as decisive-support condition \emph{for becoming both fine-material and immaterial}, [that is,] for the profitable kamma in the kamma-process becoming that takes place in sense-desire becoming and for the rebirth-process becoming. It is a condition, \emph{as conascence and so on}, that is, as conascence, mutuality, support, association, presence, non-disappearance, and root-cause conditions, for the unprofitable kamma-process becoming associated with [the fourfold clinging] itself in the sense-desire becoming. But it is a condition, as decisive-support only, for that which is dissociated.

                            This is the detailed explanation of the clause “With clinging as condition, becoming.”
                    \subsubsection[\vismAlignedParas{§270–272}(11)–(12) Birth, Etc.]{(11)–(12) Birth, Etc.}

                        \vismParagraph{XVII.270}{270}{}
                        As regards the clause “With becoming as condition, birth,” etc., the definition of birth should be understood in the way given in the Description of the Truths (\hyperlink{XIV.31}{XIV.31ff.}{})

                        Only kamma-process becoming is intended here as “becoming”; for it is that, not rebirth-process becoming, which is a condition for birth. But it is a condition in two ways, as kamma condition and as decisive-support condition.

                        \vismParagraph{XVII.271}{271}{}
                        Here it may be asked: “But how is it to be known that becoming is a condition for birth?” Because of the observable difference of inferiority and superiority. For in spite of equality of external circumstances, such as father, mother, seed, blood, nutriment, etc., the difference of inferiority and superiority of beings is observable even in the case of twins. And that fact is not causeless, since it is not present always and in all; \textcolor{brown}{\textit{[576]}} nor has it any cause other than kamma-process becoming since there is no other reason in the internal continuity of beings generated by it. Consequently, it has only kamma-process becoming for its cause. And because kamma is the cause of the difference of inferiority and superiority among beings the Blessed One said, “It is kamma that separates beings according to inferiority and superiority” (\textbf{\cite{M}III 203}). From that it can be known that becoming is a condition for birth.

                        \vismParagraph{XVII.272}{272}{}
                        \marginnote{\textcolor{teal}{\footnotesize\{656|598\}}}{}And when there is no birth, neither ageing and death nor the states beginning with sorrow come about; but when there is birth, then ageing and death come about, and also the states beginning with sorrow, which are either bound up with ageing and death in a fool who is affected by the painful states called ageing and death, or which are not so bound up in one who is affected by some painful state or other; therefore this birth is a condition for ageing and death and also for sorrow and so on. But it is a condition in one way, as decisive-support type.

                        This is the detailed explanation of the clause “With becoming as condition, birth.”

                        
            \section[\vismAlignedParas{§273–314}C. The Wheel of Becoming]{C. The Wheel of Becoming}
                \subsection[\vismAlignedParas{§273–283}(i) The Wheel]{(i) The Wheel}

                    \vismParagraph{XVII.273}{273}{}
                    Now, here at the end sorrow, etc., are stated. Consequently, the ignorance stated at the beginning of the Wheel of Becoming thus, “With ignorance as condition there are formations,” is established by the sorrow and so on. So it should accordingly be understood that:
                    \begin{verse}
                        Becoming’s Wheel reveals no known beginning;\\{}
                        No maker, no experiencer there;\\{}
                        Void with a twelvefold voidness, and nowhere\\{}
                        It ever halts; forever it is spinning.
                    \end{verse}


                    \vismParagraph{XVII.274}{274}{}
                    But (1) how is ignorance established by sorrow, etc.? (2) How has this Wheel of Becoming no known beginning? (3) How is there no maker or experiencer there? (4) How is it void with twelvefold voidness?

                    \vismParagraph{XVII.275}{275}{}
                    \emph{1. }Sorrow, grief and despair are inseparable from ignorance; and lamentation is found in one who is deluded. So, firstly, when these are established, ignorance is established. Furthermore, “With the arising of cankers there is the arising of ignorance” (\textbf{\cite{M}I 54}) is said, and with the arising of cankers these things beginning with sorrow come into being. How?

                    \vismParagraph{XVII.276}{276}{}
                    Firstly, sorrow about separation from sense desires as object has its arising in the canker of sense desire, according as it is said:
                    \begin{verse}
                        If, desiring and lusting, his desires elude him,\\{}
                        He suffers as though an arrow had pierced him (\textbf{\cite{Sn}767}),
                    \end{verse}


                    and according as it is said:
                    \begin{verse}
                        “Sorrow springs from sense desires” (\textbf{\cite{Dhp}215}).
                    \end{verse}


                    \vismParagraph{XVII.277}{277}{}
                    And all these come about with the arising of the canker of views, according as it is said: “In one who \textcolor{brown}{\textit{[577]}} possesses [the view] ‘I am materiality,’ ‘my materiality,’ with the change and transformation of materiality there arise sorrow and lamentation, pain, grief and despair” (\textbf{\cite{S}III 3}).

                    \vismParagraph{XVII.278}{278}{}
                    And as with the arising of the canker of views, so also with the arising of the canker of becoming, according as it is said: “Then whatever deities there are, long-lived, beautiful, blissful, long-resident in grand palaces, when they hear the Perfect One’s teaching of the Dhamma, they feel fear, anxiety and a sense of \marginnote{\textcolor{teal}{\footnotesize\{657|599\}}}{}urgency” (\textbf{\cite{S}III 85}), as in the case of deities harassed by the fear of death on seeing the five signs.\footnote{\vismAssertFootnoteCounter{43}\vismHypertarget{XVII.n43}{}Their flowers wither, their clothes get dirty, sweat comes from their armpits, their bodies become unsightly, and they get restless (see \textbf{\cite{M-a}IV 170}).}

                    \vismParagraph{XVII.279}{279}{}
                    And as with the arising of the canker of becoming, so also with the canker of ignorance, according as it is said: “The fool, bhikkhus, experiences pain and grief here and now in three ways” (\textbf{\cite{M}III 163}).

                    Now, these states come about with the arising of cankers, and so when they are established, they establish the cankers which are the cause of ignorance. And when the cankers are established, ignorance is also established because it is present when its condition is present. This, in the first place, is how ignorance, etc., should be understood to be established by sorrow and so on.

                    \vismParagraph{XVII.280}{280}{}
                    \emph{2. }But when ignorance is established since it is present when its condition is present, and when “with ignorance as condition there are formations; with formations as condition, consciousness,” there is no end to the succession of cause with fruit in this way. Consequently, the Wheel of Becoming with its twelve factors, revolving with the linking of cause and effect, is established as having “no known beginning.”

                    \vismParagraph{XVII.281}{281}{}
                    This being so, are not the words “With ignorance as condition there are formations,” as an exposition of a simple beginning, contradicted?—This is not an exposition of a simple beginning. It is an exposition of a basic state (see \hyperlink{XVII.107}{§107}{}). For ignorance is the basic state for the three rounds (see \hyperlink{XVII.298}{§298}{}). It is owing to his seizing ignorance that the fool gets caught in the round of the remaining defilements, in the rounds of kamma, etc., just as it is owing to seizing a snake’s head that the arm gets caught in [the coils of] the rest of the snake’s body. But when the cutting off of ignorance is effected, he is liberated from them just as the arm caught [in the coils] is liberated when the snake’s head is cut off, according as it is said, “With the remainderless fading away and cessation of ignorance” (\textbf{\cite{S}II 1}), and so on. So this is an exposition of the basic state whereby there is bondage for him who grasps it, and liberation for him who lets it go: it is not an exposition of a simple beginning.

                    This is how the Wheel of Becoming should be understood to have no known beginning. \textcolor{brown}{\textit{[578]}}

                    

                    \vismParagraph{XVII.282}{282}{}
                    \emph{3.} This Wheel of Becoming consists in the occurrence of formations, etc., with ignorance, etc., as the respective reasons. Therefore it is devoid of a maker supplementary to that, such as a Brahmā conjectured thus, “Brahmā the Great, the Highest, the Creator” (\textbf{\cite{D}I 18}), to perform the function of maker of the round of rebirths; and it is devoid of any self as an experiencer of pleasure and pain conceived thus, “This self of mine that speaks and feels” (cf. \textbf{\cite{M}I 8}). This is how it should be understood to be without any maker or experiencer.

                    \vismParagraph{XVII.283}{283}{}
                    \emph{4.} However, ignorance—and likewise the factors consisting of formations, etc.—is void of lastingness since its nature is to rise and fall, and it is void of beauty since it is defiled and causes defilement, and it is void of pleasure since \marginnote{\textcolor{teal}{\footnotesize\{658|600\}}}{}it is oppressed by rise and fall, and it is void of any selfhood susceptible to the wielding of power since it exists in dependence on conditions. Or ignorance—and likewise the factors consisting of formations, etc.—is neither self nor self’s nor in self nor possessed of self. That is why this Wheel of Becoming should be understood thus, “Void with a twelvefold voidness.”
                \subsection[\vismAlignedParas{§284–287}(ii) The Three Times]{(ii) The Three Times}

                    \vismParagraph{XVII.284}{284}{}
                    After knowing this, again:
                    \begin{verse}
                        Its roots are ignorance and craving;\\{}
                        Its times are three as past and so on,\\{}
                        To which there properly belong\\{}
                        Two, eight, and two, from its [twelve] factors.
                    \end{verse}


                    \vismParagraph{XVII.285}{285}{}
                    The two things, ignorance and craving, should be understood as the root of this Wheel of Becoming. Of the derivation from the past, ignorance is the root and feeling the end. And of the continuation into the future, craving is the root and ageing-and-death the end. It is twofold in this way.

                    \vismParagraph{XVII.286}{286}{}
                    Herein, the first applies to one whose temperament is [false] view, and the second to one whose temperament is craving. For in the round of rebirths ignorance leads those whose temperament favours [false] view, and craving those whose temperament favours craving. Or the first has the purpose of eliminating the annihilation view because, by the evidence of the fruit, it proves that there is no annihilation of the causes; and the second has the purpose of eliminating the eternity view because it proves the ageing and death of whatever has arisen. Or the first deals with the child in the womb because it illustrates successive occurrence [of the faculties], and the second deals with one apparitionally born because of [their] simultaneous appearance.

                    \vismParagraph{XVII.287}{287}{}
                    The past, the present and the future are its three times. Of these, it should be understood that, according to what is given as such in the texts, the two factors ignorance and formations belong to the past time, the eight beginning with consciousness belong to the present time, and the two, birth and ageing-and-death, belong to the future time. \textcolor{brown}{\textit{[579]}}
                \subsection[\vismAlignedParas{§288–298}(iii) Cause and Fruit]{(iii) Cause and Fruit}

                    \vismParagraph{XVII.288}{288}{}
                    Again, it should be understood thus:
                    \begin{verse}
                        (1) It has three links with cause, fruit, cause,\\{}
                        As first parts; and (2) four different sections;\\{}
                        (3) Its spokes are twenty qualities;\\{}
                        (4) With triple round it spins forever.
                    \end{verse}


                    \vismParagraph{XVII.289}{289}{}
                    \emph{1.} Herein, between formations and rebirth-linking consciousness there is one link consisting of cause-fruit. Between feeling and craving there is one link consisting of fruit-cause. And between becoming and birth there is one link consisting of cause-fruit. This is how it should be understood \emph{that it has three links with cause, fruit, cause, as first parts}.

                    \vismParagraph{XVII.290}{290}{}
                    \marginnote{\textcolor{teal}{\footnotesize\{659|601\}}}{}\emph{2.} But there are four sections, which are determined by the beginnings and ends of the links, that is to say, ignorance/ formations is one section; consciousness/mentality-materiality/ sixfold base/contact/feeling is the second; craving/clinging/ becoming is the third; and birth/ageing-and-death is the fourth. This is how it should be understood to have \emph{four different sections}.

                    \vismParagraph{XVII.291}{291}{}
                    \emph{3.} Then:
                    \begin{verse}
                        (a) There were five causes in the past,\\{}
                        (b) And now there is a fivefold fruit;\\{}
                        (c) There are five causes now as well,\\{}
                        (d) And in the future fivefold fruit.
                    \end{verse}


                    It is according to these twenty spokes called “qualities” that the words \emph{its spokes are twenty qualities} should be understood.

                    

                    \vismParagraph{XVII.292}{292}{}
                    (a) Herein, [as regards the words] \emph{There were five causes in the past}, firstly only these two, namely, ignorance and formations, are mentioned. But one who is ignorant hankers, and hankering, clings, and with his clinging as condition there is becoming; therefore craving, clinging and becoming are included as well. Hence it is said: “In the previous kamma-process becoming, there is delusion, which is \emph{ignorance}; there is accumulation, which is \emph{formations}; there is attachment, which is \emph{craving}; there is embracing, which is \emph{clinging}; there is volition, which is \emph{becoming}; thus these five things in the previous kamma-process becoming are conditions for rebirth-linking here [in the present becoming]” (\textbf{\cite{Paṭis}I 52}).

                    \vismParagraph{XVII.293}{293}{}
                    Herein, \emph{In the previous kamma-process becoming }means in kamma-process becoming done in the previous birth. \emph{There is delusion, which is ignorance} means that the delusion that there then was about suffering, etc., deluded whereby the man did the kamma, was ignorance. \emph{There is accumulation, which is formations} means the prior volitions arisen in one who prepares the things necessary for a gift during a month, perhaps, or a year after he has had the thought “I shall give a gift.” \textcolor{brown}{\textit{[580]}} But it is the volitions of one who is actually placing the offerings in the recipients’ hands that are called “becoming.” Or alternatively, it is the volition that is accumulation in six of the impulsions of a single adverting that is called “formations,” and the seventh volition is called “becoming.” Or any kind of volition is called “becoming” and the accumulations associated therewith are called “formations.” \emph{There is attachment, which is craving} means that in one performing kamma, whatever attachment and aspiration there is for its fruit as rebirth-process-becoming is called craving. \emph{There is embracing, which is clinging} means that the embracing, the grasping, the adherence, which is a condition for kamma-process becoming and occurs thus, “By doing this I shall preserve, or I shall cut off, sense desire in such and such a place,” is called clinging. \emph{There is volition, which is becoming} means the kind of volition stated already at the end of the [sentence dealing with] accumulation is becoming. This is how the meaning should be understood.

                    \vismParagraph{XVII.294}{294}{}
                    (b) \emph{And now there is a fivefold fruit} (\hyperlink{XVII.291}{§291}{}) means what is given in the text beginning with consciousness and ending with feeling, according as it is said: “Here [in the present becoming] there is rebirth-linking, which is \emph{consciousness}; there is descent [into the womb], which is \emph{mentality-materiality}; there is sensitivity, which is \emph{sense base}; there is what is touched, which is \emph{contact}; there is what is felt, \marginnote{\textcolor{teal}{\footnotesize\{660|602\}}}{}which is\emph{ feeling}; thus these five things here in the [present] rebirth-process becoming have their conditions\footnote{\vismAssertFootnoteCounter{44}\vismHypertarget{XVII.n44}{}As regards these four paragraphs from the Paṭisambhidā (see §§292, 294, 296, and 297), all four end with the word ‘\emph{paccayā’ }(nom. pl. and abl. s. of \emph{paccaya = }condition). In the first and third paragraphs (§§292 and 296) this is obviously nom. pl. and agrees with ‘\emph{ime pañca dhammā’ }(these five things). But in the second and fourth paragraphs the context suggests \emph{vipākā }(results) instead of conditions. However, there is no doubt that the accepted reading is \emph{paccayā }here too; for the passage is also quoted in \hyperlink{XIX.13}{XIX.13}{}, in the \emph{Sammohavinodanī }(Paccayākāra-Vibhaṅga commentary = present context), and at \textbf{\cite{M-a}I 53}. The \emph{Paramatthamañjūsā} and \emph{Mūla Ṭīkā} do not mention this point. The \emph{Saddhammappakāsinī }(Paṭisambhidā commentary) comments on the first paragraph: “\emph{Purimakammabhavasmin ti atītajātiyā kammabhave karīyamāne pavattā; idha paṭisandhiyā paccayā ti paccuppannā paṭisandhiyā paccayabhūtā,”} and on the second paragraph:\emph{ “Idh’upapattibhavasmiṃ pure katassa kammassa paccayā ti paccuppanne vipākabhave atītajātiyaṃ katassa kammassa paccayena pavattī ti attho}.” The \emph{Majjhima Nikāya Ṭīkā} (\textbf{\cite{M-a}I 53}) says of the second paragraph: “\emph{Ime paccayā ti ime viññāṇādayo pañca koṭṭhāsikā dhammā, purimabhave katassa kammassa, kammavaṭṭassa, paccayā, paccayabhāvato, taṃ paṭicca, idha, etarahi, upapattibhavasmiṃ upapattibhavabhāvena vā hontī ti attho.}” From these comments it is plain enough that “\emph{paccayā}” in the second and fourth paragraphs is taken as abl. sing. (e.g. \emph{avijjā-paccayā saṅkhārā). }There is a parallel ablative construction with genitive at \textbf{\cite{Paṭis}II 72}, 1.8: “\emph{Gatisampattiyā ñāṇasampayutte aṭṭhannaṃ hetūnaṃ paccayā uppatti hoti}.” Perhaps the literal rendering of the second and fourth paragraphs’ final sentence might be: “Thus there are these five things here in the [present] rebirth-process becoming with their condition [consisting] of kamma done in the past,” and so on. The point is unimportant.} in kamma done in the past” (\textbf{\cite{Paṭis}I 52}).

                    \vismParagraph{XVII.295}{295}{}
                    Herein, there is \emph{rebirth-linking, which is consciousness} means that it is what is called “rebirth-linking” because it arises linking the next becoming that is consciousness. \emph{There is descent [into the womb], which is mentality-materiality} means that it is what consists in the descent of the material and immaterial states into a womb, their arrival and entry as it were, that is mentality-materiality. \emph{There is sensitivity, which is sense base}: this is said of the five bases beginning with the eye. \emph{There is what is touched, which is contact} means that it is what is arisen when an object is touched or in the touching of it, that is contact. \emph{There is what is felt, which is feeling} means that it is what is felt as results [of kamma] that is arisen together with rebirth-linking consciousness, or with the contact that has the sixfold base as its condition, that is feeling. Thus should the meaning be understood.

                    \vismParagraph{XVII.296}{296}{}
                    (c) \emph{There are five causes now as well} (\hyperlink{XVII.291}{§291}{}) means craving, and so on. Craving, clinging and becoming are given in the text. But when becoming is included, the formations that precede it or that are associated with it are included too. And by including craving and clinging, the ignorance associated with them, deluded by which a man performs kamma, is included too. So they are five. Hence it is said: “Here [in the present becoming], with the maturing of the bases there is delusion, which is \emph{ignorance}; there is accumulation, which is \emph{formations}; there is attachment, which is \emph{craving}; there is embracing, which is \emph{clinging}; there is volition, which is \emph{becoming}; thus these five things here in the [present] kamma-process becoming are conditions for rebirth-linking in the future” (\textbf{\cite{Paṭis}I 52}). \textcolor{brown}{\textit{[581]}} \marginnote{\textcolor{teal}{\footnotesize\{661|603\}}}{}Herein, the words \emph{Here [in the present becoming], with the maturing of the bases }point out the delusion existing at the time of the performance of the kamma in one whose bases have matured. The rest is clear.

                    \vismParagraph{XVII.297}{297}{}
                    (d) \emph{And in the future fivefold fruit}: the five beginning with consciousness. These are expressed by the term “birth.” But “ageing-and-death” is the ageing and the death of these [five] themselves. Hence it is said: “In the future there is rebirth-linking, which is \emph{consciousness}; there is descent [into the womb], which is \emph{mentality-materiality}; there is sensitivity, which is \emph{sense base}; there is what is touched, which is \emph{contact}; there is what is felt, which is \emph{feeling}; thus these five things in the future rebirth-process becoming have their condition in kamma done here [in the present becoming]” (\textbf{\cite{Paṭis}I 52}).

                    So this [Wheel of Becoming] has twenty spokes with these qualities.

                    \vismParagraph{XVII.298}{298}{}
                    \emph{4. With triple round it spins forever }(\hyperlink{XVII.288}{§288}{}): here formations and becoming are the \emph{round of kamma}. Ignorance, craving and clinging are the \emph{round of defilements}. Consciousness, mentality-materiality, the sixfold base, contact and feeling are the \emph{round of result}. So this Wheel of Becoming, having a triple round with these three rounds, should be understood to spin, revolving again and again, forever, for the conditions are not cut off as long as the round of defilements is not cut off.
                \subsection[\vismAlignedParas{§299–314}(iv) Various]{(iv) Various}

                    \vismParagraph{XVII.299}{299}{}
                    As it spins thus:
                    \begin{verse}
                        (1) As to the source in the [four] truths,\\{}
                        (2) As to function, (3) prevention, (4) similes,\\{}
                        (5) Kinds of profundity, and (6) methods,\\{}
                        It should be known accordingly.
                    \end{verse}


                    \vismParagraph{XVII.300}{300}{}
                    \emph{1. }Herein, [\emph{as to source in the truths}:] profitable and unprofitable kamma are stated in the Saccavibhaṅga (\textbf{\cite{Vibh}106f.}) without distinction as the origin of suffering, and so formations due to ignorance [stated thus] “With ignorance as condition there are formations” are the second truth with the second truth as source. Consciousness due to formations is the first truth with the second truth as source. The states beginning with mentality-materiality and ending with resultant feeling, due respectively to consciousness, etc., are the first truth with the first truth as source. Craving due to feeling is the second truth with the first truth as source.

                    Clinging due to craving is the second truth with the second truth as source. Becoming due to clinging is the first and second truths with the second truth as source. Birth due to becoming is the first truth with the second truth as source. Ageing-and-death due to birth is the first truth with the first truth as source. This, in the first place, is how [the Wheel of Becoming] should be known “as to … source in the four truths” in whichever way is appropriate.

                    \vismParagraph{XVII.301}{301}{}
                    \emph{2. [As to function}:] ignorance confuses beings about physical objects [of sense desire] and is a condition for the manifestation of formations; likewise [kamma-] formations \textcolor{brown}{\textit{[582]}} form the formed and are a condition for consciousness; \marginnote{\textcolor{teal}{\footnotesize\{662|604\}}}{}consciousness recognizes an object and is a condition for mentality-materiality; mentality-materiality is mutually consolidating and is a condition for the sixfold base; the sixfold base occurs with respect to its own [separate] objective fields and is a condition for contact; contact touches an object and is a condition for feeling; feeling experiences the stimulus of the object and is a condition for craving; craving lusts after lust-arousing things and is a condition for clinging; clinging clings to clinging-arousing things and is a condition for becoming; becoming flings beings into the various kinds of destiny and is a condition for birth; birth gives birth to the aggregates owing to its occurring as their generation and is a condition for ageing-and-death; and ageing-and-death ensures the decay and dissolution of the aggregates and is a condition for the manifestation of the next becoming because it ensures sorrow, etc.\footnote{\vismAssertFootnoteCounter{45}\vismHypertarget{XVII.n45}{}“Sorrow, etc., have already been established as ignorance; but death consciousness itself is devoid of ignorance and formations and is not a condition for the next becoming; that is why ‘\emph{because it assures sorrow, etc.’ }is said” (\textbf{\cite{Vism-mhṭ}640}).} So this [Wheel of Becoming] should be known accordingly as occurring in two ways “as to function” in whichever way is appropriate to each of its parts.

                    \vismParagraph{XVII.302}{302}{}
                    \emph{3. [As to prevention}:] the clause “With ignorance as condition there are formations” prevents seeing a maker; the clause “With formations as condition, consciousness” prevents seeing the transmigration of a self; the clause “With consciousness as condition, mentality-materiality” prevents perception of compactness because it shows the analysis of the basis conjectured to be “self”; and the clauses beginning “With mentality-materiality as condition, the sixfold base” prevent seeing any self that sees, etc., cognizes, touches, feels, craves, clings, becomes, is born, ages and dies. So this Wheel of Becoming should be known “as to prevention” of wrong seeing appropriately in each instance.

                    \vismParagraph{XVII.303}{303}{}
                    \emph{4. [As to similes}:] ignorance is like a blind man because there is no seeing states according to their specific and general characteristics; formations with ignorance as condition are like the blind man’s stumbling; consciousness with formations as condition is like the stumbler’s falling; mentality-materiality with consciousness as condition is like the appearance of a tumour on the fallen man; the sixfold base with mentality-materiality as condition is like a gathering that makes the tumour burst; contact with the sixfold base as condition is like hitting the gathering in the tumour; feeling with contact as condition is like the pain due to the blow; craving with feeling as condition is like longing for a remedy; clinging with craving as condition is like seizing what is unsuitable through longing for a remedy; \textcolor{brown}{\textit{[583]}} becoming with clinging as condition is like applying the unsuitable remedy seized; birth with becoming as condition is like the appearance of a change [for the worse] in the tumour owing to the application of the unsuitable remedy; and ageing-and-death with birth as condition is like the bursting of the tumour after the change.

                    Or again, ignorance here as “no theory” and “wrong theory” (see \hyperlink{XVII.52}{§52}{}) befogs beings as a cataract does the eyes; the fool befogged by it involves himself in formations that produce further becoming, as a cocoon-spinning caterpillar does \marginnote{\textcolor{teal}{\footnotesize\{663|605\}}}{}with the strands of the cocoon; consciousness guided by formations establishes itself in the destinies, as a prince guided by a minister establishes himself on a throne; [death] consciousness conjecturing about the sign of rebirth generates mentality-materiality in its various aspects in rebirth-linking, as a magician does an illusion; the sixfold base planted in mentality-materiality reaches growth, increase and fulfilment, as a forest thicket does planted in good soil; contact is born from the impingement of the bases, as fire is born from the rubbing together of fire sticks; feeling is manifested in one touched by contact, as burning is in one touched by fire; craving increases in one who feels, as thirst does in one who drinks salt water; one who is parched [with craving] conceives longing for the kinds of becoming, as a thirsty man does for drinks; that is his clinging; by clinging he clings to becoming as a fish does to the hook through greed for the bait; when there is becoming there is birth, as when there is a seed there is a shoot; and death is certain for one who is born, as falling down is for a tree that has grown up.

                    So this Wheel of Becoming should be known thus “as to similes” too in whichever way is appropriate.

                    \vismParagraph{XVII.304}{304}{}
                    \emph{5. [Kinds of profundity}:] Now, the Blessed One’s words, “This dependent origination is profound, Ānanda, and profound it appears” (\textbf{\cite{D}II 55}), refer to profundity (a) of meaning, (b) of law, (c) of teaching, and (d) of penetration. So this Wheel of Becoming should be known “as to the kinds of profundity” in whichever way is appropriate.

                    \vismParagraph{XVII.305}{305}{}
                    (a) Herein, the meaning of ageing-and-death produced and originated with birth as condition is profound owing to difficulty in understanding its origin with birth as condition thus: Neither does ageing-and death not come about from birth, nor, failing birth, does it come about from something else; it arises [only] from birth with precisely that nature [of ageing-and-death]. And the meaning of birth with becoming as condition … and the meaning of formations produced and originated with ignorance as condition are treatable in like manner. That is why this Wheel of Becoming is profound in meaning. This, firstly, is the profundity of meaning here. \textcolor{brown}{\textit{[584]}} For it is the fruit of a cause that is called “meaning,” according as it is said, “Knowledge about the fruit of a cause is the discrimination of meaning” (\textbf{\cite{Vibh}293}).

                    \vismParagraph{XVII.306}{306}{}
                    (b) The meaning of ignorance as condition for formations is profound since it is difficult to understand in what mode and on what occasion\footnote{\vismAssertFootnoteCounter{46}\vismHypertarget{XVII.n46}{}\emph{Avatthā—}“occasion”: not in PED.} ignorance is a condition for the several formations … The meaning of birth as a condition for ageing-and-death is similarly profound. That is why this Wheel of Becoming is profound in law. This is the profundity of \emph{law} here. For “law” is a name for cause, according as it is said, “Knowledge about cause is discrimination of law” (\textbf{\cite{Vibh}293}).

                    \vismParagraph{XVII.307}{307}{}
                    (c) Then the teaching of this [dependent origination] is profound since it needs to be given in various ways for various reasons, and none but omniscient knowledge gets fully established in it; for in some places in the suttas it is taught in forward order, in some in backward order, in some in forward and \marginnote{\textcolor{teal}{\footnotesize\{664|606\}}}{}backward order, in some in forward or in backward order starting from the middle, in some in four sections and three links, in some in three sections and two links, and in some in two sections and one link. That is why this Wheel of Becoming is profound in teaching. This is the profundity of \emph{teaching}.

                    \vismParagraph{XVII.308}{308}{}
                    (d) Then the individual essences of ignorance, etc., owing to the penetration of which ignorance, etc., are rightly penetrated as to their specific characteristic, are profound since they are difficult to fathom. That is why this Wheel of Becoming is profound in penetration. For here the meaning of ignorance as unknowing and unseeing and non-penetration of the truth is profound; so is the meaning of formations as forming and accumulating with and without greed; so is the meaning of consciousness as void, uninterested, and manifestation of rebirth-linking without transmigration; so is the meaning of mentality-materiality as simultaneous arising, as resolved into components or not, and as bending [on to an object] (\emph{namana}) and being molested (\emph{ruppana}); so is the meaning of the sixfold base as predominance, world, door, field, and possession of objective field; so is the meaning of contact as touching, impingement, coincidence, and concurrence; so is the meaning of feeling as the experiencing of the stimulus of an object, as pleasure or pain or neutrality, as soulless, and as what is felt; so is the meaning of craving as a delighting in, as a committal to, as a current, as a bindweed, as a river, as the ocean of craving, and as impossible to fill; so is the meaning of clinging as grasping, seizing, misinterpreting, adhering, and hard to get by; so is the meaning of becoming as accumulating, forming, and flinging into the various kinds of generation, destiny, station, and abode; so is the meaning of birth as birth, coming to birth, descent [into the womb], rebirth, and manifestation; and so is the meaning of ageing-and-birth as destruction, fall, break-up and change. This is profundity of \emph{penetration}.

                    \vismParagraph{XVII.309}{309}{}
                    \emph{6.} [As to methods:] Then \textcolor{brown}{\textit{[585]}} there are four methods of treating the meaning here. They are (a) the method of identity, (b) the method of diversity, (c) the method of uninterest,\footnote{\vismAssertFootnoteCounter{47}\vismHypertarget{XVII.n47}{}\emph{Avyāpāra—}“uninterest”: here the equivalent of\emph{ anābhoga, }see \hyperlink{IV.171}{IV.171}{} and \hyperlink{IX.108}{IX.108}{}. The perhaps unorthodox form “uninterest” has been used to avoid the “unselfish” sense sometimes implied by “disinterestedness.” \emph{Vyāpāra }is clearly intended throughout this work as “motivated action” in contrast with “blind action of natural forces.” The word “interest” has therefore been chosen to bring out this effect.} and (d) the method of ineluctable regularity. So this Wheel of Becoming should also be known accordingly “as to the kinds of method.”\footnote{\vismAssertFootnoteCounter{48}\vismHypertarget{XVII.n48}{}The dependent origination, or structure of conditions, appears as a flexible formula with the intention of describing the ordinary human situation of a man in his world (or indeed any conscious event where ignorance and craving have not entirely ceased). That situation is always complex, since it is implicit that consciousness with no object, or being (\emph{bhava—}becoming, or however rendered) without consciousness (of it), is impossible except as an artificial abstraction. The dependent origination, being designed to portray the essentials of that situation in the limited dimensions of words and using only elements recognizable in experience, is not a logical proposition (Descartes’ \emph{cogito }is not a logical proposition). Nor is it a temporal cause-and-effect chain: each member has to be examined as to its nature in order to determine what its relations to the others are (e.g. whether successive in time or conascent, positive or negative, etc., etc.). A purely cause-and-effect chain would not represent the pattern of a situation that is always complex, always subjective-objective, static-dynamic, positive-negative, and so on. Again, there is no evidence of any historical development in the various forms given \emph{within the limit of the Sutta Piṭaka }(leaving aside the Paṭisambhidāmagga), and historical treatment within that particular limit is likely to mislead, if it is hypothesis with no foundation.

                            Parallels with European thought have been avoided in this translation. But perhaps an exception can be made here, with due caution, in the case of Descartes. The revolution in European thought started by his formula \emph{cogito ergo sum }(“I think, therefore I am”) is not yet ended. Now, it will perhaps not escape notice that the two elements, “I think” and “I am,” in what is not a logical proposition parallel to some extent the two members of the dependent origination, consciousness and being (becoming). In other words, consciousness activated by craving and clinging as the dynamic factory, guided and blinkered by ignorance (“I think” or “consciousness with the conceit ‘I am’”), conditions being (“therefore I am”) in a complex relationship with other factors relating subject and object (not accounted for by Descartes). The parallel should not be pushed too far. In fact it is only introduced because in Europe the dependent origination seems to be very largely misunderstood with many strange interpretations placed upon it, and because the \emph{cogito }does seem to offer some sort of reasonable approach. In this work, for convenience because of the special importance attached here to the aspect of the death-rebirth link, the dependent origination is considered from only one standpoint, namely, as applicable to a period embracing a minimum of three lives. But this is not the only application. With suitable modifications it is also used in the Vibhaṅga to describe the structure of the complex in each one of the 89 single type-consciousnesses laid down in the \emph{Dhammasaṅgaṇī}; and Bhadantācariya Buddhaghosa says: “This structure of conditions is present not only in (a continuity period consisting of) multiple consciousnesses but also in each single consciousness as well” (\textbf{\cite{Vibh-a}199–200}). Also the Paṭisambhidāmagga gives five expositions, four describing dependent origination in one life, the fifth being made to present a special inductive generalization to extend what is observable in this life (the fact that consciousness is always preceded by consciousness, cf. this Ch. §83f.—i.e. that it always has a past and is inconceivable without one) back beyond birth, and (since craving and ignorance ensure its expected continuance) on after death. There are, besides, various other, differing applications indicated by the variant forms given in the suttas themselves.}

                    \vismParagraph{XVII.310}{310}{}
                    \marginnote{\textcolor{teal}{\footnotesize\{665|607\}}}{}(a) Herein, the non-interruption of the continuity in this way, “With ignorance as condition there are formations; with formations as condition, consciousness,” just like a seed’s reaching the state of a tree through the state of the shoot, etc., is called the “method of identity.” One who sees this rightly abandons the annihilation view by understanding the unbrokenness of the continuity that occurs through the linking of cause and fruit. And one who sees it wrongly clings to the eternity view by apprehending identity in the non-interruption of the continuity that occurs through the linking of cause and fruit.

                    \vismParagraph{XVII.311}{311}{}
                    (b) The defining of the individual characteristic of ignorance, etc., is called the “method of diversity.” One who sees this rightly abandons the eternity view by seeing the arising of each new state. And one who sees it wrongly clings to \marginnote{\textcolor{teal}{\footnotesize\{666|608\}}}{}the annihilation view by apprehending individual diversity in the events in a single continuity as though it were a broken continuity.

                    \vismParagraph{XVII.312}{312}{}
                    (c) The absence of interestedness on the part of ignorance, such as “Formations must be made to occur by me,” or on the part of formations, such as “Consciousness must be made to occur by us,” and so on, is called the “method of uninterestedness.” One who sees this rightly abandons the self view by understanding the absence of a maker. One who sees it wrongly clings to the moral-inefficacy-of-action view, because he does not perceive that the causative function of ignorance, etc., is established as a law by their respective individual essences.

                    \vismParagraph{XVII.313}{313}{}
                    (d) The production of only formations, etc., respectively and no others with ignorance, etc., as the respective reasons, like that of curd, etc., with milk, etc., as the respective reasons, is called the “method of ineluctable regularity.” One who sees this rightly abandons the no-cause view and the moral-inefficacy-of-action view by understanding how the fruit accords with its condition. One who sees it wrongly by apprehending it as non-production of anything from anything, instead of apprehending the occurrence of the fruit in accordance with its conditions, clings to the no-cause view and to the doctrine of fatalism. So this Wheel of Becoming:
                    \begin{verse}
                        As to source in the [four] truths,\\{}
                        As to function, prevention, similes,\\{}
                        Kinds of profundity, and methods,\\{}
                        Should be known accordingly.
                    \end{verse}


                    \vismParagraph{XVII.314}{314}{}
                    There is no one, even in a dream, who has got out of the fearful round of rebirths, which is ever destroying like a thunderbolt, unless he has severed with the knife of knowledge well whetted on the stone of sublime concentration, this Wheel of Becoming, which offers no footing owing to its great profundity and is hard to get by owing to the maze of many methods. \textcolor{brown}{\textit{[586]}}

                    And this has been said by the Blessed One: “This dependent origination is profound, Ānanda, and profound it appears. And, Ānanda, it is through not knowing, through not penetrating it, that this generation has become a tangled skein, a knotted ball of thread, root-matted as a reed bed, and finds no way out of the round of rebirths, with its states of loss, unhappy destinies, … perdition” (\textbf{\cite{D}II 55}).

                    Therefore, practicing for his own and others’ benefit and welfare, and abandoning other duties:
                    \begin{verse}
                        Let a wise man with mindfulness\\{}
                        So practice that he may begin\\{}
                        To find a footing in the deeps\\{}
                        Of the dependent origin.
                    \end{verse}

        \chapter[Purification of View]{Purification of View\vismHypertarget{XVIII}\newline{\textnormal{\emph{Diṭṭhi-visuddhi-niddesa}}}}
            \label{XVIII}

            \section[\vismAlignedParas{§1}Introductory]{Introductory}

                \vismParagraph{XVIII.1}{1}{}
                \marginnote{\textcolor{teal}{\footnotesize\{667|609\}}}{}\textcolor{brown}{\textit{[587]}} Now, it was said earlier (\hyperlink{XIV.32}{XIV.32}{}) that he “should first fortify his knowledge by learning and questioning about those things that are the ‘soil’ after he has perfected the two purifications—purification of virtue and purification of consciousness—that are the ‘roots.’” Now, of those, purification of virtue is the quite purified fourfold virtue beginning with Pātimokkha restraint; and that has already been dealt with in detail in the Description of Virtue; (Chs. I and II) and the purification of consciousness, namely, the eight attainments together with access concentration, has also been dealt with in detail in all its aspects in the Description of Concentration, (Chs. III to XIII) stated under the heading of “consciousness” [in the introductory verse]. So those two purifications should be understood in detail as given there.
            \section[\vismAlignedParas{§2–14}Defining of Mentality-Materiality]{Defining of Mentality-Materiality}
                \subsection[\vismAlignedParas{§2–8}(1) Definition Based on the Four Primaries]{(1) Definition Based on the Four Primaries}

                    \vismParagraph{XVIII.2}{2}{}
                    But it was said above (\hyperlink{XIV.32}{XIV.32}{}) that “The five purifications, purification of view, purification by overcoming doubt, purification by knowledge and vision of what is the path and what is not the path, purification by knowledge and vision of the way, and purification by knowledge and vision, are the ‘trunk.’” Herein, “purification of view” is the correct seeing of mentality-materiality.\footnote{\vismAssertFootnoteCounter{1}\vismHypertarget{XVIII.n1}{}“Mentality should be taken here as the four aggregates beginning with feeling and belonging to the three planes, not omitting consciousness as in the case of ‘With consciousness as condition, mentality-materiality’ and not including the supramundane aggregates associated with Nibbāna” (\textbf{\cite{Vism-mhṭ}744} (Be)).}
                    \subsubsection[\vismAlignedParas{§3–4}(a) Starting with Mentality]{(a) Starting with Mentality}

                        \vismParagraph{XVIII.3}{3}{}
                        One who wants to accomplish this, if, firstly, his vehicle is serenity,\footnote{\vismAssertFootnoteCounter{2}\vismHypertarget{XVIII.n2}{}Serenity (\emph{samatha}) is a general term for concentration, as the complement of insight (\emph{vipassanā}), which is roughly the equivalent of understanding (\emph{paññā}).} should emerge from any fine-material or immaterial jhāna, except the base consisting of neither perception nor non-perception,\footnote{\vismAssertFootnoteCounter{3}\vismHypertarget{XVIII.n3}{}“One who is beginning this work has difficulty in discerning the highest form of becoming, that is, the base consisting of neither perception nor non-perception” (Vism-mhṭ 744). This is owing to the diminished perception (see \textbf{\cite{M}III 28}).} and he should discern, according to \marginnote{\textcolor{teal}{\footnotesize\{668|610\}}}{}characteristic, function, etc., the jhāna factors consisting of applied thought, etc., and the states associated with them, [that is, feeling, perception, and so on]. When he has done so, all that should be defined as “mentality” (\emph{nāma}) in the sense of bending (\emph{namana})\footnote{\vismAssertFootnoteCounter{4}\vismHypertarget{XVIII.n4}{}See \textbf{\cite{S}II 23f.} “Bending in the direction of the object means that there is no occurrence without an object; it is in the sense of that sort of bending, or it is in the sense of bestowing a name (\emph{nāma-karaṇa})” (\textbf{\cite{Vism-mhṭ}744}). “Name-and-form” has many advantages over “mentality-materiality” if only because it preserves the integrity of \emph{nāma }and excludes any metaphysical assumption of matter existing as a substance behind apparent forms.} because of its bending on to the object.

                        \vismParagraph{XVIII.4}{4}{}
                        Then, just as a man, by following a snake that he has seen in his house, finds its abode, so too this meditator scrutinizes that mentality, he seeks to find out what its occurrence is supported by and he sees that it is supported \textcolor{brown}{\textit{[588]}} by the matter of the heart. After that, he discerns as materiality the primary elements, which are the heart’s support, and the remaining, derived kinds of materiality that have the elements as their support. He defines all that as “materiality” (\emph{rūpa}) because it is “molested” (\emph{ruppana}) [by cold, etc.]. After that he defines in brief as “mentality-materiality” (\emph{nāma-rūpa}) the mentality that has the characteristic of “bending” and the materiality that has the characteristic of “being molested.”
                    \subsubsection[\vismAlignedParas{§5–8}(b) Starting with Materiality]{(b) Starting with Materiality}

                        \vismParagraph{XVIII.5}{5}{}
                        But one whose vehicle is pure insight, or that same aforesaid one whose vehicle is serenity, discerns the four elements in brief or in detail in one of the various ways given in the chapter on the definition of the four elements (\hyperlink{XI.27}{XI.27ff.}{}). Then, when the elements have become clear in their correct essential characteristics, firstly, in the case of head hair originated by kamma there become plain ten instances of materiality (\emph{rūpāni}) with the body decad thus: the four elements, colour, odour, flavour, nutritive essence, and life, and body sensitivity. And because the sex decad is present there too there are another ten [that is, the same nine with sex instead of body sensitivity]. And since the octad-with-nutritive-essence-as-eighth [that is, the four elements and colour, odour, flavour, and nutritive essence,] originated by nutriment, and that originated by temperature, and that originated by consciousness are present there too, there are another twenty-four. So there is a total of forty-four instances of materiality in the case of each of the twenty-four bodily parts of fourfold origination. But in the case of the four, namely, sweat, tears, spittle, and snot,\footnote{\vismAssertFootnoteCounter{5}\vismHypertarget{XVIII.n5}{}“Because sweat, etc., arise owing to heat, fatigue, etc., and owing to mental perturbation, they are called ‘originated by temperature and by consciousness’” (Vism-mhṭ 745). There are seven kinds of decads: those of the physical basis of mind (heart), sex, living, physical eye, ear, nose, tongue, and body. The first nine components of a decad are the same in all instances, and by themselves they are called the “life ennead.” The first eight components by themselves are called the “octad-with-nutritive-essence-as-eighth.” This octad plus sound is called the “sound ennead.” In general these are called “material groups” (\emph{rūpa-kalāpa}). But this kind of group (\emph{kalāpa}) has nothing to do with the “comprehension by groups” (\emph{kalāpa-sammasana}) of \hyperlink{XX}{Ch. XX}{}, which is simply generalization (from one’s own particular experience to each of the five aggregates as past, etc., i.e. as a “group”). The “material groups” are not in the Piṭakas.} which are originated \marginnote{\textcolor{teal}{\footnotesize\{669|611\}}}{}by temperature and by consciousness, there are sixteen instances of materiality with the two octads-with-nutritive-essence-as-eighth in each. In the case of the four, namely, gorge, dung, pus, and urine, which are originated by temperature, eight instances of materiality become plain in each with the octad-with-nutritive-essence-as-eighth in what is originated only by temperature. This, in the first place, is the method in the case of the thirty-two bodily aspects.

                        \vismParagraph{XVIII.6}{6}{}
                        But there are ten more aspects\footnote{\vismAssertFootnoteCounter{6}\vismHypertarget{XVIII.n6}{}The ten are four aspects of the fire element and six aspects of the air element; what heats, what consumes, what burns up, what digests; up-going winds (or forces), down-going winds, winds in the stomach, winds in the bowels, winds in the limbs, breath. See \hyperlink{XI.37}{XI.37}{}, \hyperlink{XI.82}{82}{}.} that become clear when those thirty-two aspects have become clear. And as regards these, firstly, nine instances of materiality, that is, the octad-with-nutritive-essence-as-eighth plus life, become plain in the case of the kamma-born part of heat (fire) that digests what is eaten, etc., and likewise nine [instances of materiality], that is, the octad-with-nutritive-essence-as-eighth plus sound, in the case of the consciousness-born part [of air consisting] of in-breaths and out-breaths; and thirty-three instances of materiality, that is, the [kamma-born] life-ennead and the three octads-with-nutritive-essence-as-eighth in the case of each of the remaining eight [parts] that are of fourfold origination.

                        \vismParagraph{XVIII.7}{7}{}
                        And when these instances of materiality derived [by clinging] from the primaries have thus become plain in detail in the case of these forty-two aspects, [that is, thirty-two parts of the body, four modes of fire, and six modes of air,] another sixty instances of materiality become plain with the physical [heart-] basis and the [five] sense doors, that is, with the heart-basis decad and the five decads beginning with the eye decad.

                        Taking all these together under the characteristic of “being molested,” he sees them as “materiality.”

                        

                        \vismParagraph{XVIII.8}{8}{}
                        When he has discerned materiality thus, the immaterial states become plain to him in accordance with the sense doors, that is to say, the eighty-one kinds\footnote{\vismAssertFootnoteCounter{7}\vismHypertarget{XVIII.n7}{}“The exalted consciousness of the fine-material and immaterial spheres is only quite plain to one who has attained the attainments” (\textbf{\cite{Vism-mhṭ}746}).} of mundane consciousness consisting of the two sets of five consciousnesses ((34)–(38) and (50)–(54)), the three kinds of mind element ((39), (55) and (70)) and the sixty-eight \textcolor{brown}{\textit{[589]}} kinds of mind-consciousness element; and then seven consciousness-concomitants, that is, (i) contact, feeling, perception, (ii) volition, (vii) life, (viii) steadiness of consciousness, and (xxx) attention, which are invariably conascent with all these consciousnesses. The supramundane kinds of consciousness, however, are not discernible either by one who is practicing pure insight or by one whose vehicle is serenity because they are out of their reach. Taking all these immaterial states together under the characteristic of “bending,” he sees them as “mentality.”

                        This is how one [meditator] defines mentality-materiality in detail through the method of defining the four elements.
                \subsection[\vismAlignedParas{§9–11}(2) Definition Based on the Eighteen Elements]{(2) Definition Based on the Eighteen Elements}

                    \vismParagraph{XVIII.9}{9}{}
                    \marginnote{\textcolor{teal}{\footnotesize\{670|612\}}}{}Another does it by means of the eighteen elements. How? Here a bhikkhu considers the elements thus: “There are in this person the eye element, … the mind-consciousness element.” Instead of taking the piece of flesh variegated with white and black circles, having length and breath, and fastened in the eye socket with a string of sinew, which the world terms “an eye,” he defines as “eye element” the eye sensitivity of the kind described among the kinds of derived materiality in the Description of the Aggregates (\hyperlink{XIV.47}{XIV.47}{}).

                    \vismParagraph{XVIII.10}{10}{}
                    But he does not define as “eye element” the remaining instances of materiality, which total fifty-three, that is, the nine conascent instances of materiality consisting of the four primary elements, which are its support, the four concomitant instances of materiality, namely, colour, odour, flavour, and nutritive essence, and the sustaining life faculty; and also the twenty kamma-born instances of materiality that are there too, consisting of the body decad and sex decad; and the twenty-four unclung-to instances of materiality consisting of the three octads-with-nutritive-essence-as-eighth, which are originated by nutriment and so on. The same method applies to the ear element and the rest. But in the case of the body element the remaining instances of materiality total forty-three, though some say forty-five by adding sound and making nine each for the temperature-born and consciousness-born [sound].

                    \vismParagraph{XVIII.11}{11}{}
                    So these five sensitivities, and their five respective objective fields, that is, visible data, sounds, odours, flavours, and tangible data, make ten instances of materiality, which are ten [of the eighteen] elements. The remaining instances of materiality are the mental-data element only.

                    The consciousness that occurs with the eye as its support and contingent upon a visible datum is called “eye-consciousness element” [and likewise with the ear and so on]. In this way the two sets of five consciousnesses are the five “consciousness elements.” The three kinds of consciousness consisting of mind element ((39), (55) and (70)) are the single “mind element.” The sixty-eight kinds of mind-consciousness element are the “mind-consciousness element.” So all the eighty-one kinds of mundane consciousness make up seven kinds of consciousness element; and the contact, etc., associated therewith are the mental-data element.

                    So ten-and-a-half elements are materiality and seven-and-a-half elements \textcolor{brown}{\textit{[590]}} are mentality. This is how one [meditator] defines mentality-materiality by means of the eighteen elements.
                \subsection[\vismAlignedParas{§12}(3) Definition Based on the Twelve Bases]{(3) Definition Based on the Twelve Bases}

                    \vismParagraph{XVIII.12}{12}{}
                    Another does it by means of the twelve bases. How? He defines as “eye base” the sensitivity only, leaving out the fifty-three remaining instances of materiality, in the way described for the eye element. And in the way described there [he also defines] the elements of the ear, nose, tongue, and body, as “ear base, nose base, tongue base, body base.” He defines five states that are their respective objective fields as “visible-data base, sound base, odour base, flavour base, tangible-data base.” He defines the seven mundane consciousness elements \marginnote{\textcolor{teal}{\footnotesize\{671|613\}}}{}as “mind base.” He defines the contact, etc., associated there with and also the remaining instances of materiality as “mental-data base.” So here ten-and-a-half bases are materiality and one-and-a-half bases are mentality. This is how one [meditator] defines mentality-materiality by means of the twelve bases.
                \subsection[\vismAlignedParas{§13}(4) Definition Based on the Five Aggregates]{(4) Definition Based on the Five Aggregates}

                    \vismParagraph{XVIII.13}{13}{}
                    Another defines it more briefly than that by means of the aggregates. How? Here a bhikkhu defines as “the materiality aggregate” all the following twenty-seven instances of materiality, that is, the seventeen instances of materiality consisting of the four primaries of fourfold origination in this body and dependent colour, odour, flavour, and nutritive essence, and the five sensitivities beginning with the eye sensitivity, and the materiality of the physical [heart-]basis, sex, life faculty, and sound of twofold origination, which seventeen instances of materiality are suitable for comprehension since they are produced and are instances of concrete materiality; and then the ten instances of materiality, that is, bodily intimation, verbal intimation, the space element, and the lightness, malleability, wieldiness, growth, continuity, aging, and impermanence of materiality, which ten instances of materiality are, however, not suitable for comprehension since they are merely the mode-alteration and the limitation-of-interval; they are not produced and are not concrete materiality, but they are reckoned as materiality because they are mode-alterations, and limitation-of-interval, of various instances of materiality. So he defines all these twenty-seven instances of materiality as the “the materiality aggregate.” He defines the feeling that arises together with the eighty-one kinds of mundane consciousness as the “feeling aggregate,” the perception associated therewith as the “perception aggregate,” the formations associated therewith as the “formations aggregate,” and the consciousness as the “consciousness aggregate.” So by defining the materiality aggregate as “materiality” and the four immaterial aggregates as “mentality,” he defines mentality-materiality by means of the five aggregates.
                \subsection[\vismAlignedParas{§14}(5) Brief Definition Based on the Four Primaries]{(5) Brief Definition Based on the Four Primaries}

                    \vismParagraph{XVIII.14}{14}{}
                    Another discerns “materiality” in his person briefly thus: “Any kind of materiality whatever all consists of the four primary elements and the materiality derived from the four primary elements” (\textbf{\cite{M}I 222}), and he likewise discerns the mind base and a part of the mental data base as “mentality.” Then he defines mentality-materiality in brief thus: “This mentality and this materiality are called ‘mentality-materiality.’” \footnote{\vismAssertFootnoteCounter{8}\vismHypertarget{XVIII.n8}{}“As well as by means of the elements, etc., materiality can also be discerned through the faculties, the truths, and the dependent origination. How?

                            “Firstly, through the faculties. These seven, namely, the five beginning with the eye plus femininity and masculinity are materiality; the eleven consisting of the mind faculty, the five feeling faculties, and the five beginning with faith, are mentality; the life faculty is both mentality and materiality. The last three, being supramundane, are not intended here. The truth of suffering is both mentality and materiality; the truth of origin is mentality; the other two are not intended here because they are supramundane. “In the structure of conditions, the first three members are mentality; the fourth and fifth are mentality and materiality; the sixth, seventh, eighth, and ninth are mentality; the tenth is both mentality and materiality; the last two are each mentality and materiality” (\textbf{\cite{Vism-mhṭ}747f.}).}
            \section[\vismAlignedParas{§15–17}If the Immaterial Fails to Become Evident]{If the Immaterial Fails to Become Evident}

                \vismParagraph{XVIII.15}{15}{}
                \marginnote{\textcolor{teal}{\footnotesize\{672|614\}}}{}\textcolor{brown}{\textit{[591]}} But if he has discerned materiality in one of these ways, and while he is trying to discern the immaterial it does not become evident to him owing to its subtlety, then he should not give up but should again and again comprehend, give attention to, discern, and define materiality only. For in proportion as materiality becomes quite definite, disentangled and quite clear to him, so the immaterial states that have that [materiality] as their object become plain of themselves too.

                \vismParagraph{XVIII.16}{16}{}
                Just as, when a man with eyes looks for the reflection of his face in a dirty looking-glass and sees no reflection, he does not throw the looking-glass away because the reflection does not appear; on the contrary, he polishes it again and again, and then the reflection becomes plain of itself when the looking-glass is clean—and just as, when a man needing oil puts sesame flour in a basin and wets it with water and no oil comes out with only one or two pressings, he does not throw the sesame flour away; but on the contrary, he wets it again and again with hot water and squeezes and presses it, and as he does so clear sesame oil comes out—or just as, when a man wanting to clarify water has taken a \emph{katuka }nut and put his hand inside the pot and rubbed it once or twice but the water does not come clear, he does not throw the \emph{katuka} nut away; on the contrary, he rubs it again and again, and as he does so the fine mud subsides and the water becomes transparent and clear—so too, the bhikkhu should not give up, but he should again and again comprehend, give attention to, discern and define materiality only.

                \vismParagraph{XVIII.17}{17}{}
                For in proportion as materiality becomes quite definite, disentangled and quite clear to him, so the defilements that are opposing him subside, his consciousness becomes clear like the water above the [precipitated] mud, and the immaterial states that have that [materiality] as their object become plain of themselves too. And this meaning can also be explained in this way by other analogies such as the [pressing of] sugarcane, [the beating of] criminals [to make them confess], [the taming of] an ox, the churning of curds [to produce butter], and [the cooking of] fish.
            \section[\vismAlignedParas{§18–23}How the Immaterial States Become Evident]{How the Immaterial States Become Evident}

                \vismParagraph{XVIII.18}{18}{}
                When he has quite cleared up his discerning of materiality, then the immaterial states become evident to him through one of three aspects, that is, through contact, through feeling, or through consciousness. How?

                \vismParagraph{XVIII.19}{19}{}
                \emph{1. (a) }When he discerns the [four primary] elements in the way beginning, “The earth element has the characteristic of hardness” (\hyperlink{XI.93}{XI.93}{}), contact becomes evident to him as the first conjunction. Then the feeling associated with that as the feeling aggregate, the associated perception as the perception aggregate, the associated volition together with the aforesaid contact as the formations aggregate, and the associated consciousness as the consciousness aggregate.

                \marginnote{\textcolor{teal}{\footnotesize\{673|615\}}}{}\emph{1. (b)} \textcolor{brown}{\textit{[592]}} Likewise [when he has discerned them in this way,] “In the head hair it is the earth element that has the characteristic of hardness … in the in-breaths and out-breaths it is the earth element that has the characteristic of hardness” (\hyperlink{XI.31}{XI.31}{}), contact becomes evident as the first conjunction. Then the feeling associated with it as the feeling aggregate, … the associated consciousness as the consciousness aggregate.

                This is how immaterial states become evident through contact.

                \vismParagraph{XVIII.20}{20}{}
                \emph{2. (a)} To another [who discerns the four primary elements in the way beginning] “The earth element has the characteristic of hardness,” the feeling that has that as its object and experiences its stimulus [as pleasant, etc.,] becomes evident as the feeling aggregate, the perception associated with that as the perception aggregate, the contact and the volition associated with that as the formations aggregate, and the consciousness associated with that as the consciousness aggregate.

                \emph{2. (b)} Likewise [to one who discerns them in this way] “In the head hair it is the earth element that has the characteristic of hardness … in the in-breaths and out-breaths it is the earth element that has the characteristic of hardness,” the feeling that has that as its object and experiences its stimulus becomes evident as the feeling aggregate … and the consciousness associated with that as the consciousness aggregate.

                This is how the immaterial states become evident through feeling.

                \vismParagraph{XVIII.21}{21}{}
                \emph{3. (a)} To another [who discerns the four primary elements in the way beginning] “The earth element has the characteristic of hardness,” the consciousness that cognizes the object becomes evident as the consciousness aggregate, the feeling associated with it as the feeling aggregate, the associated perception as the perception aggregate, and the associated contact and volition as the formations aggregate.

                \emph{3. (b)} Likewise [to one who discerns them in this way] “In the head hair it is the earth element that has the characteristic of hardness … in the in-breaths and out-breaths it is the earth element that has the characteristic of hardness,” the consciousness that cognizes the object becomes evident as the consciousness aggregate …\textbf{ }and the associated contact and volition as the formations aggregate.

                This is how the immaterial states become evident through consciousness.

                \vismParagraph{XVIII.22}{22}{}
                In the case of [the ways of discerning materiality as consisting of] the forty-two aspects of the elements beginning with the head hairs [that is, thirty-two aspects of the body, four aspects of the fire element and six aspects of the air element,] either by these same means given above or by means of the method beginning, “In the kamma-originated head hairs it is the earth element that has the characteristic of hardness—and also in the case of the methods of discerning materiality as consisting of the eye, etc.—by means of the four primary elements in each, the construing should be done by working out all the differences in each method.

                \vismParagraph{XVIII.23}{23}{}
                Now, it is only when he has become quite sure about discerning materiality in this way that immaterial states become quite evident to him in the three aspects. \marginnote{\textcolor{teal}{\footnotesize\{674|616\}}}{}Therefore he should only undertake the task of discerning the immaterial states after he has completed that, not otherwise. If he leaves off discerning materiality when, say, one or two material states have become evident in order to begin discerning the immaterial, then he falls from his meditation subject like the mountain cow already described under the Development of the Earth Kasiṇa (\hyperlink{IV.130}{IV.130}{}). \textcolor{brown}{\textit{[593]}} But if he undertakes the task of discerning the immaterial after he is already quite sure about discerning materiality thus, then his meditation subject comes to growth, increase and perfection.
            \section[\vismAlignedParas{§24–31}No Being Apart from Mentality-Materiality]{No Being Apart from Mentality-Materiality}

                \vismParagraph{XVIII.24}{24}{}
                He defines the four immaterial aggregates that have thus become evident through contact, etc., as “mentality.” And he defines their objects, namely, the four primaries and the materiality derived from the four primaries, as “materiality.” So, as one who opens a box with a knife, as one who splits a twin palmyra bulb in two, he defines all states of the three planes,\footnote{\vismAssertFootnoteCounter{9}\vismHypertarget{XVIII.n9}{}“‘\emph{All states of the three planes’ }is said all-inclusively owing to the necessity not to omit anything suitable for comprehension. For it must be fully understood without any exception, and greed must be made to fade away absolutely so that the mind may be liberated by the fading away of greed. That is why the Blessed One said: ‘Bhikkhus, without directly knowing, without fully understanding all, without causing the fading away of greed for it, without abandoning it, the mind is incapable of the destruction of suffering. Bhikkhus, it is by directly knowing, by fully understanding all, by causing the fading away of greed for it, by abandoning it, that the mind is capable of the destruction of suffering’ (\textbf{\cite{S}IV 17}). If all the states of the three planes are taken as mentality-materiality without exception, then how should one deal with what has been conceived by those outside the Dispensation as verbal meanings, such as the Primordial Essence (\emph{pakati}), etc. [e.g. of the Sāṃkhya], the substance (\emph{drabya}), etc. [e.g. of the Vaiśeṣika], the soul (\emph{jīva}), etc., and the body (\emph{kāya}), etc. [?] Since these are like the hallucination of lunatics and are taught by the not fully enlightened, what other way of dealing with them is there than to ignore them? Or alternatively, their existence or non-existence can be understood as established by their inclusion within mentality-materiality” (\textbf{\cite{Vism-mhṭ}751f.}). There follows a long paragraph showing how the concepts of these systems are to be assimilated into mentality-materiality whereby they lose their significance and are shown to be impermanent and formed. \textbf{\cite{Vism-mhṭ}} concludes by saying, “Wherever the verbal meaning of self is expressed by some such metaphor as world-soul (\emph{purisa}), self (\emph{attā, ātman}), soul (\emph{jīva}), etc., these being themselves conceived in their various ways on the basis of mere mentality-materiality, are mere mentality-materiality, too” (\textbf{\cite{Vism-mhṭ}754f.}).} the eighteen elements, twelve bases, five aggregates, in the double way as “mentality-materiality,” and he concludes that over and above mere mentality-materiality there is nothing else that is a being or a person or a deity or a Brahmā.

                \vismParagraph{XVIII.25}{25}{}
                After defining mentality-materiality thus according to its true nature, then in order to abandon this worldly designation of “a being” and “a person” more thoroughly, to surmount confusion about beings and to establish his mind on the plane of non-confusion, he makes sure that the meaning defined, namely, \marginnote{\textcolor{teal}{\footnotesize\{675|617\}}}{}“This is mere mentality-materiality, there is no being, no person” is confirmed by a number of suttas. For this has been said:
                \begin{verse}
                    As with the assembly of parts\\{}
                    The word “chariot” is countenanced,\\{}
                    So, when the aggregates are present,\\{}
                    “A being” is said in common usage (\textbf{\cite{S}I 135}).
                \end{verse}


                \vismParagraph{XVIII.26}{26}{}
                Again, this has been said: “Just as when a space is enclosed with timber and creepers and grass and clay, there comes to be the term ‘house,’ so too, when a space is enclosed with bones and sinews and flesh and skin, there comes to be the term ‘material form’ (\emph{rūpa})” (\textbf{\cite{M}I 190}).

                \vismParagraph{XVIII.27}{27}{}
                And again this has been said:
                \begin{verse}
                    It is ill alone that rises,\\{}
                    Ill that remains, ill that departs.\\{}
                    Nothing rises else than ill,\\{}
                    And nothing ceases else than ill (\textbf{\cite{S}I 135}).
                \end{verse}


                \vismParagraph{XVIII.28}{28}{}
                So in many hundred suttas it is only mentality-materiality that is illustrated, not a being, not a person. Therefore, just as when the component parts such as axles, wheels, frame poles, etc., are arranged in a certain way, there comes to be the mere term of common usage “chariot,” yet in the ultimate sense when each part is examined there is no chariot—and just as when the component parts of a house such as wattles, etc., are placed so that they enclose a space in a certain way, there comes to be the mere term of common usage “house,” yet in the ultimate sense there is no house—and just as when the fingers, thumb, etc., are placed in a certain way, there comes to be the mere term of common usage \textcolor{brown}{\textit{[594]}} “fist,”—with body and strings, “lute”; with elephants, horses, etc., “army”; with surrounding walls, houses, states, etc., “city”—just as when trunk, branches, foliage, etc., are placed in a certain way, there comes to be the mere term of common usage “tree,” yet in the ultimate sense, when each component is examined, there is no tree—so too, when there are the five aggregates [as objects] of clinging, there comes to be the mere term of common usage “a being,” “a person,” yet in the ultimate sense, when each component is examined, there is no being as a basis for the assumption “I am” or “I”; in the ultimate sense there is only mentality-materiality. The vision of one who sees in this way is called correct vision.

                \vismParagraph{XVIII.29}{29}{}
                But when a man rejects this correct vision and assumes that a [permanent] being exists, he has to conclude either that it comes to be annihilated or that it does not. If he concludes that it does not come to be annihilated, he falls into the eternity [view]. If he concludes that it does come to be annihilated, he falls into the annihilation [view]. Why? Because [the assumption] precludes any gradual change like that of milk into curd. So he either holds back, concluding that the assumed being is eternal, or he overreaches, concluding that it comes to be annihilated.

                \vismParagraph{XVIII.30}{30}{}
                Hence the Blessed One said: “There are two kinds of view, bhikkhus, and when deities and human beings are obsessed by them, some hold back and \marginnote{\textcolor{teal}{\footnotesize\{676|618\}}}{}some overreach; only those with eyes see. And how do some hold back? Deities and human beings love becoming, delight in becoming, rejoice in becoming. When Dhamma is taught to them for the ceasing of becoming, their minds do not enter into it, become settled, steady and resolute. Thus it is that some hold back. And how do some overreach? Some are ashamed, humiliated and disgusted by that same becoming, they are concerned with non-becoming in this way: ‘Sirs, when with the breakup of the body this self is cut off, annihilated, does not become any more after death, that is peaceful, that is sublime, that is true.’ Thus it is that some overreach. And how do those with eyes see? Here a bhikkhu sees what is become as become. Having seen what is become as become, he has entered upon the way to dispassion for it, to the fading away of greed for it, to its cessation. This is how one with eyes sees” (It 43; \textbf{\cite{Paṭis}I 159}).

                \vismParagraph{XVIII.31}{31}{}
                Therefore, just as a marionette is void, soulless and without curiosity, and while it walks and stands merely through the combination of strings and wood, \textcolor{brown}{\textit{[595]}} yet it seems as if it had curiosity and interestedness, so too, this mentality-materiality is void, soulless and without curiosity, and while it walks and stands merely through the combination of the two together, yet it seems as if it had curiosity and interestedness. This is how it should be regarded. Hence the Ancients said:
                \begin{verse}
                    The mental and material are really here,\\{}
                    But here there is no human being to be found,\\{}
                    For it is void and merely fashioned like a doll—\\{}
                    Just suffering piled up like grass and sticks.
                \end{verse}

            \section[\vismAlignedParas{§32–36}Interdependence of Mentality and Materiality]{Interdependence of Mentality and Materiality}

                \vismParagraph{XVIII.32}{32}{}
                And this should be explained not only by means of the simile of the marionette, but also by means of the analogies of the sheaves of reeds and so on. For just as when two sheaves of reeds are propped one against the other, each one gives the other consolidating support, and when one falls the other falls, so too, in the five-constituent becoming mentality-materiality occurs as an interdependent state, each of its components giving the other consolidating support, and when one falls owing to death, the other falls too. Hence the Ancients said:
                \begin{verse}
                    The mental and material\\{}
                    Are twins and each supports the other;\\{}
                    When one breaks up they both break up\\{}
                    Through interconditionality.
                \end{verse}


                \vismParagraph{XVIII.33}{33}{}
                And just as when sound occurs having as its support a drum that is beaten by the stick, then the drum is one and the sound another, the drum and the sound are not mixed up together, the drum is void of the sound and the sound is void of the drum, so too, when mentality occurs having as its support the materiality called the physical basis, the door and the object, then the materiality is one and the mentality is another, the mentality and materiality are not mixed up together, the mentality is void of the materiality and the materiality is void of \marginnote{\textcolor{teal}{\footnotesize\{677|619\}}}{}the mentality; yet the mentality occurs due to the materiality as the sound occurs due to the drum. Hence the Ancients said:
                \begin{verse}
                    The pentad based on contact comes not from the eye,\\{}
                    Or from things seen, or something that is in between;\\{}
                    Due to a cause it comes to be, and formed as well.\\{}
                    Just as the sound that issues from a beaten drum.
                \end{verse}

                \begin{verse}
                    The pentad based on contact comes not from the ear.\\{}
                    Or yet from sound, or something that is in between;\\{}
                    Due to a cause …\\{}
                    The pentad based on contact comes not from the nose
                \end{verse}

                \begin{verse}
                    Or yet from smells, or something that is in between;\\{}
                    Due to a cause …\\{}
                    The pentad based on contact comes not from the tongue,\\{}
                    Or yet from tastes, or something that is in between; \textcolor{brown}{\textit{[596]}}
                \end{verse}

                \begin{verse}
                    Due to a cause …\\{}
                    The pentad based on contact comes not from the body,\\{}
                    Or yet from touch, or something that is in between;\\{}
                    Due to a cause …
                \end{verse}

                \begin{verse}
                    Being formed, it does not come from the material basis.\\{}
                    Nor does it issue from the mental-datum base;\\{}
                    Due to a cause it comes to be, and formed as well.\\{}
                    Just as the sound that issues from a beaten drum.
                \end{verse}


                \vismParagraph{XVIII.34}{34}{}
                Furthermore, mentality has no efficient power, it cannot occur by its own efficient power. It does not eat, it does not drink, it does not speak, it does not adopt postures. And materiality is without efficient power; it cannot occur by its own efficient power. For it has no desire to eat, it has no desire to drink, it has no desire to speak, it has no desire to adopt postures. But rather it is when supported by materiality that mentality occurs; and it is when supported by mentality that materiality occurs. When mentality has the desire to eat, the desire to drink, the desire to speak, the desire to adopt a posture, it is materiality that eats, drinks, speaks, and adopts a posture.

                \vismParagraph{XVIII.35}{35}{}
                But for the purpose of explaining this meaning they gave this simile as an example: a man born blind and a stool-crawling cripple wanted to go somewhere. The blind man said to the cripple, “Look, I can do what should be done by legs, but I have no eyes with which to see what is rough and smooth.” The cripple said, “Look, I can do what should be done by eyes, but I have no legs with which to go and come.” The blind man was delighted, and he made the cripple climb up on his shoulder. Sitting on the blind man’s shoulder the cripple spoke thus, “Leave the left, take the right; leave the right, take the left.”

                Herein, the blind man has no efficient power; he is impotent; he cannot travel by his own efficient power, by his own strength. And the cripple has no efficient power; he is impotent; he cannot travel by his own efficient power, by his own strength. But there is nothing to prevent their going when they support each other. So too, mentality has no efficient power; it does not arise or occur in such \marginnote{\textcolor{teal}{\footnotesize\{678|620\}}}{}and such functions by its own efficient power. And materiality has no efficient power; it does not arise or occur in such and such functions by its own efficient power.

                But there is nothing to prevent their occurrence when they support each other.

                \vismParagraph{XVIII.36}{36}{}
                Hence this is said:
                \begin{verse}
                    They cannot come to be by their own strength,\\{}
                    Or yet maintain themselves by their own strength;\\{}
                    Relying for support on other states,\\{}
                    Weak in themselves, and formed, they come to be; \textcolor{brown}{\textit{[597]}}
                \end{verse}

                \begin{verse}
                    They come to be with others as condition.\\{}
                    They are aroused by others as their objects,\\{}
                    They are produced by object and condition,\\{}
                    And each by something other than itself.
                \end{verse}

                \begin{verse}
                    And just as men depend upon\\{}
                    A boat for traversing the sea.\\{}
                    So does the mental body need\\{}
                    The matter-body for occurrence.
                \end{verse}

                \begin{verse}
                    And as the boat depends upon\\{}
                    The men for traversing the sea.\\{}
                    So does the matter-body need\\{}
                    The mental body for occurrence.
                \end{verse}

                \begin{verse}
                    Depending each upon the other\\{}
                    The boat and men go on the sea.\\{}
                    And so do mind and matter both\\{}
                    Depend the one upon the other.
                \end{verse}

            \section[\vismAlignedParas{§37}Conclusion]{Conclusion}

                \vismParagraph{XVIII.37}{37}{}
                The correct vision of mentality and materiality, which, after defining mentality-materiality by these various methods, has been established on the plane of non-confusion by overcoming the perception of a being, is what should be understood as purification of view. Other terms for it are “defining of mentality-materiality” and “delimitation of formations.”

                The eighteenth chapter called “The Description of Purification of View” in the Treatise on the Development of Understanding in the \emph{Path of Purification }composed for the purpose of gladdening good people.
        \chapter[Purification by Overcoming Doubt]{Purification by Overcoming Doubt\vismHypertarget{XIX}\newline{\textnormal{\emph{Kaṅkhāvitaraṇa-visuddhi-niddesa}}}}
            \label{XIX}

            \section[\vismAlignedParas{§1}Introductory]{Introductory}

                \vismParagraph{XIX.1}{1}{}
                \marginnote{\textcolor{teal}{\footnotesize\{679|621\}}}{}\textcolor{brown}{\textit{[598]}} Knowledge established by overcoming doubt about the three divisions of time by means of discerning the conditions of that same mentality-materiality is called “purification by overcoming doubt.”
            \section[\vismAlignedParas{§2–20}Ways of Discerning Cause and Condition]{Ways of Discerning Cause and Condition}

                \vismParagraph{XIX.2}{2}{}
                The bhikkhu who wants to accomplish this sets about seeking the cause and condition for that mentality-materiality; just as when a skilled physician encounters a disease he seeks its origin, or just as when a compassionate man sees a tender little child lying on its back in the road he wonders who its parents are.
                \subsection[\vismAlignedParas{§3–4}1. Neither Created by a Creator nor Causeless]{1. Neither Created by a Creator nor Causeless}

                    \vismParagraph{XIX.3}{3}{}
                    To begin with, he considers thus: “Firstly this mentality-materiality is not causeless, because if that were so, it would follow that [having no causes to differentiate it,] it would be identical everywhere always and for all. It has no Overlord, etc., because of the non-existence of any Overlord, etc. (\hyperlink{XVI.85}{XVI.85}{}), over and above mentality-materiality. And because, if people then argue that mentality-materiality itself is its Overlord, etc., then it follows that their mentality-materiality, which they call the Overlord, etc., would itself be causeless. Consequently there must be a cause and a condition for it. What are they?”

                    \vismParagraph{XIX.4}{4}{}
                    Having thus directed his attention to mentality-materiality’s cause and condition, he first discerns the cause and condition for the material body in this way: “When this body is born it is not born inside a blue, red or white lotus or water-lily, etc., or inside a store of jewels or pearls, etc.; on the contrary, like a worm in rotting flesh, in a rotting corpse, in rotting dough, in a drain, in a cesspool, etc., it is born in between the receptacle for undigested food and the receptacle for digested food, behind the belly lining, in front of the backbone, surrounded by the bowel and the entrails, in a place that is stinking, disgusting, repulsive, and extremely cramped, being itself stinking, disgusting, and repulsive. When it is born thus, its causes (root-causes) are the four things, namely, ignorance, craving, clinging, and kamma, \textcolor{brown}{\textit{[599]}}\textbf{ }since it is they that bring about its birth; and nutriment is its condition, since it is that that consolidates it. So five things constitute its cause and condition. And of these, the three beginning with ignorance are the decisive-support for this body, as the \marginnote{\textcolor{teal}{\footnotesize\{680|622\}}}{}mother is for her infant, and kamma begets it, as the father does the child; and nutriment sustains it, as the wet-nurse does the infant.”
                \subsection[\vismAlignedParas{§5–6}2. Its Occurance is Always Due to Conditions]{2. Its Occurance is Always Due to Conditions}

                    \vismParagraph{XIX.5}{5}{}
                    After discerning the material body’s conditions in this way, he again discerns the mental body in the way beginning: “Due to eye and to visible object eye-consciousness arises” (\textbf{\cite{S}II 72}; \textbf{\cite{M}I 111}). When he has thus seen that the occurrence of mentality-materiality is due to conditions, then he sees that, as now, so in the past too its occurrence was due to conditions, and in the future too its occurrence will be due to conditions.

                    \vismParagraph{XIX.6}{6}{}
                    When he sees it in this way, all his uncertainty is abandoned, that is to say, the five kinds of uncertainty about the past stated thus: “Was I in the past? Was I not in the past? What was I in the past? How was I in the past? Having been what, what was I in the past?” (\textbf{\cite{M}I 8}), and also the five kinds of uncertainty about the future stated thus: “Shall I be in the future? Shall I not be in the future? What shall I be in the future? How shall I be in the future? Having been what, what shall I be in the future?” (\textbf{\cite{M}I 8}); and also the six kinds of uncertainty about the present stated thus: “Am I? Am I not? What am I? How am I? Whence will this being have come? Whither will it be bound?” (\textbf{\cite{M}I 8}).
                \subsection[\vismAlignedParas{§7–10}3. General and Particular Conditions]{3. General and Particular Conditions}

                    \vismParagraph{XIX.7}{7}{}
                    Another sees the conditions for mentality as two-fold, according to what is common to all and to what is not common to all, and that for materiality as fourfold, according to kamma, and so on.

                    \vismParagraph{XIX.8}{8}{}
                    The condition for mentality is twofold, as that which is common to all and that which is not common to all. Herein, the six doors beginning with the eye and the six objects beginning with visible data are a condition-common-to-all for mentality because the occurrence of all kinds [of mentality] classified as profitable, etc., is due to that [condition]. But attention, etc., are not common to all; for wise attention, hearing the Good Dhamma, etc., are a condition only for the profitable, \textcolor{brown}{\textit{[600]}} while the opposite kinds are a condition for the unprofitable. Kamma, etc., are a condition for the resultant mentality; and the life-continuum, etc., are a condition for the functional.

                    \vismParagraph{XIX.9}{9}{}
                    Kamma, consciousness, temperature, and nutriment constitute this fourfold condition for materiality beginning with kamma. Herein it is only when it is past that kamma is a condition for kamma-originated materiality; consciousness is a condition, when it is arising, for consciousness-originated materiality. Temperature and nutriment are conditions at the instant (moment) of their presence for temperature-originated and nutriment-originated materiality.\footnote{\vismAssertFootnoteCounter{1}\vismHypertarget{XIX.n1}{}“If the fruit were to arise from present kamma, the fruit would have arisen in the same moment in which the kamma was being accumulated; and that is not seen, nor is it desirable. For in the world (i.e. among non-Buddhists) kamma has never been shown to give fruit while it is actually being effected; nor is there any text to that effect—But is it not also the fact that no fruit has ever been shown to come from a vanished cause either? Or even a cock to crow because of that?—Certainly it has not been shown where the connectedness of material things is broken off. But the simile does not apply because there is connectedness of immaterial things here. For when the fruit arises from kamma that is actually past it does so because of kamma having been performed and because of storage. For this is said: ‘Because profitable sense-sphere kamma has been performed, stored up, there comes to be eye-consciousness’ (\textbf{\cite{Dhs}§431}).

                            “Since consciousness has efficient power only at the instant of its arising, with the acquisition of a proximity condition, etc., it therefore only gives rise to materiality while it is arising. But since materiality has efficient power at the instant of its presence, with the acquisition of a postnascence condition, etc., it is therefore said that ‘temperature and nutriment are conditions at the instant of their presence for temperature-originated and nutriment-originated materiality.’ Temperature and nutriment give rise to materiality at the instant of their own presence by acquiring outside temperature and nutriment as their condition, is the meaning” (\textbf{\cite{Vism-mhṭ}768}).}

                    \marginnote{\textcolor{teal}{\footnotesize\{681|623\}}}{}This is how one man discerns the conditions for mentality-materiality.

                    \vismParagraph{XIX.10}{10}{}
                    When he has seen that the occurrence of mentality-materiality is due to conditions in this way, he sees also that, as now, so too in the past its occurrence was due to conditions, and in the future its occurrence will be due to conditions. When he sees it in this way, his uncertainty about the three periods of time is abandoned in the way already stated.
                \subsection[\vismAlignedParas{§11}4. Dependent Origination in Reverse Order]{4. Dependent Origination in Reverse Order}

                    \vismParagraph{XIX.11}{11}{}
                    Another, when he has seen that the formations called mentality-materiality arrive at aging and that those that have aged dissolve, discerns mentality-materiality’s conditions by means of dependent origination in reverse order in this way: “This is called aging-and-death of formations; it comes to be when there is birth, and birth when there is becoming, and becoming when there is clinging, and clinging when there is craving, and craving when there is feeling, and feeling when there is contact, and contact when there is the sixfold base, and the sixfold base when there is mentality-materiality, and mentality-materiality when there is consciousness, and consciousness when there are formations, and formations when there is ignorance.” Then his uncertainty is abandoned in the way already stated.
                \subsection[\vismAlignedParas{§12}5. Dependent Origination in Direct Order]{5. Dependent Origination in Direct Order}

                    \vismParagraph{XIX.12}{12}{}
                    Another discerns mentality-materiality’s conditions by means of dependent origination in direct order as already shown (\hyperlink{XVII.29}{XVII.29}{}) in detail, doing so in this way: “So, with ignorance as condition there are formations” (\textbf{\cite{M}I 261}). Then his uncertainty is abandoned in the way already stated.
                \subsection[\vismAlignedParas{§13–19}6. Kamma and Kamma-Result]{6. Kamma and Kamma-Result}

                    \vismParagraph{XIX.13}{13}{}
                    Another discerns mentality-materiality’s conditions by means of the round of kamma and the round of kamma-result in this way:

                    “In the previous kamma-process becoming there is delusion, which is ignorance; there is accumulation, which is formations; there is attachment, which is craving; there is embracing, which is clinging; there is volition, which is becoming; thus \marginnote{\textcolor{teal}{\footnotesize\{682|624\}}}{}these five things in the previous kamma-process becoming are conditions for rebirth-linking here [in the present becoming].

                    “Here [in the present becoming] there is rebirth-linking, which is consciousness; there is descent [into the womb], which is mentality-materiality; there is sensitivity, which is sense base; there is what is touched, which is contact; there is what is felt, which is feeling; thus these five things here in the [present] rebirth-process becoming have their conditions in kamma done in the past.

                    “Here [in the present becoming] with the maturing of the bases there is delusion, which is ignorance; there is accumulation, which is formations; there is attachment, which is craving; there is embracing, which is clinging; there is volition, which is becoming; thus these five things here in the [present] kamma-process becoming are conditions for rebirth-linking in the future.

                    “In the future there is rebirth-linking, which is consciousness; there is descent [into the womb], which is mentality-materiality; there is sensitivity, which is sense base; there is what is touched, which is contact; there is what is felt, which is feeling; thus these five things in the future rebirth-process becoming have their conditions in kamma done here [in the present becoming]” (\textbf{\cite{Paṭis}I 52}). \textcolor{brown}{\textit{[601]}}

                    \vismParagraph{XIX.14}{14}{}
                    Herein, kamma is fourfold: to be experienced here and now, to be experienced on rebirth, to be experienced in some subsequent becoming, and lapsed kamma.\footnote{\vismAssertFootnoteCounter{2}\vismHypertarget{XIX.n2}{}“To be experienced here and now” means kamma whose fruit is to be experienced in this present selfhood. “To be experienced on rebirth” means kamma whose fruit is to be experienced [in the becoming] next to the present becoming. “To be experienced in some subsequent existence” means kamma whose fruit is to be experienced in some successive selfhood other than either that here and now or next to that here and now. “Lapsed kamma” is kamma of which it has to be said, “There has been kamma, but there has not been, is not, and will not be kamma-result.” “The volition of the first impulsion, which has efficient power by not being prevented by opposition and by having acquired the distinction of a condition, and which has definitely occurred as a prior kamma-formation of the appropriate kind, giving its fruit in this same selfhood, is called ‘\emph{to be experienced here and now.}’ For while that first-impulsion volition, being effective in the way stated, is helpful to what is associated with its special qualities in the impulsion continuity, yet because it wields little power over aspects and because it has little result owing to lack of repetition, it is not, like the other two kinds, kamma that looks beyond the occurring continuity and looks to obtain an opportunity; it gives its fruit here only as mere result during the course of becoming, like a mere flower. ‘\emph{But if it cannot do so}’: kamma’s giving of result comes about only through the due concurrence of conditions consisting of (suitable) essentials of becoming, means, etc., failing which it is unable to give its result in that selfhood. ‘\emph{That accomplishes its purpose}’: that fulfils its purpose consisting in giving, etc., and in killing, etc. For the seventh impulsion to which this refers is the final impulsion in the series, and when it has acquired distinction in the way already stated and has acquired the service of repetition by the previous impulsions, it gives its result in the next selfhood and is called ‘to be experienced on rebirth‘” (\textbf{\cite{Vism-mhṭ}769}).}

                    Of these, (i) the volition, either profitable or unprofitable, of the first of the seven impulsion consciousnesses in a single cognitive series of impulsions is called kamma to be experienced here and now: it gives its result in this same \marginnote{\textcolor{teal}{\footnotesize\{683|625\}}}{}selfhood. But if it cannot do so, it is called (iv) lapsed kamma (\emph{ahosi-kamma}), according to the triad described thus, “There has been (\emph{ahosi}) kamma, there has been no kamma-result, there will be no kamma-result” (see \textbf{\cite{Paṭis}II 78}). (ii) The volition of the seventh impulsion that accomplishes its purpose is called kamma to be experienced on rebirth: it gives its result in the next selfhood. If it cannot do so, it is called (iv) \emph{lapsed kamma} in the way already described. (iii) The volition of the five impulsions between these two is called \emph{kamma to be experienced in some subsequent becoming}: it gives its result in the future when it gets the opportunity, and however long the round of rebirths continues it never becomes lapsed kamma.

                    \vismParagraph{XIX.15}{15}{}
                    Another fourfold classification of kamma is this: weighty, habitual, death-threshold, and kamma [stored up] by being performed.\footnote{\vismAssertFootnoteCounter{3}\vismHypertarget{XIX.n3}{}“‘Weighty’ kamma is very reprehensible unprofitable kamma and very powerful profitable kamma. ‘Habitual’ kamma is what is habitually, continually done and repeated. ‘Death-threshold’ kamma is what is remembered with great vividness at the time next before death; what is meant is that there is no question about what is done at the time of death. ‘That has been often repeated’: he draws a distinction between this kind of kamma as stated and the ‘habitual’ kind and he likewise excludes kamma to be experienced here and now from it because the bringing on of rebirth-linking is admitted; for the tetrad beginning with the ‘weighty’ is stated as productive of rebirth-linking. “Herein, the weighty ripens first of all and that is why it is so called. When weighty kamma is lacking, what has been much done ripens. When that is lacking, death-threshold kamma ripens. When that too is lacking, then kamma done in previous births, which is called ‘kamma [stored up] by being performed,’ ripens. And the last three when produced can be strong or weak” (\textbf{\cite{Vism-mhṭ}769ff.}). \textbf{\cite{Vism-mhṭ}} then cites various Birth Stories and MN 136 in order to show how, for various reasons, the result of one kind of kamma may be delayed or displaced by the result of another. Vism-mhṭ concludes: “This is the province of the Tathāgata’s Knowledge of the Great Exposition of Kamma, in other words, the mastery of the order of ripening of such and such kamma for such and such reasons.”}

                    Herein, (v) when there is weighty and unweighty kamma, the \emph{weightier}, whether profitable or unprofitable, whether kamma consisting in matricide or kamma of the exalted spheres, takes precedence in ripening. (vi) Likewise, when there is habitual and unhabitual kamma, the \emph{more habitual}, whether consisting in good or bad conduct, takes precedence in ripening. (vii) \emph{Death-threshold }kamma is that remembered at the time of death; for when a man near death can remember [kamma], he is born according to that. (viii) Kamma not included in the foregoing three kinds that has been often repeated is called \emph{kamma [stored up] by being performed}. This brings about rebirth-linking if other kinds fail.

                    \vismParagraph{XIX.16}{16}{}
                    Another fourfold classification of kamma is this: productive, consolidating, frustrating, and supplanting.\footnote{\vismAssertFootnoteCounter{4}\vismHypertarget{XIX.n4}{}“‘Productive’ kamma is what produces resultant continuity by providing rebirth-linking and so on. ‘Consolidating’ kamma prolongs the occurrence of the continuity of pleasure or pain, or the endurance of materiality. ‘Frustrating’ kamma slowly diminishes the endurance of pleasure or pain when they occur. It cuts off the result of other kamma without giving any result of its own. ‘Supplanting’ kamma, however, cuts off weak kamma and makes its own result arise. This is their difference” (\textbf{\cite{Vism-mhṭ}771}).} \marginnote{\textcolor{teal}{\footnotesize\{684|626\}}}{}Herein, (ix) what is called \emph{productive} is both profitable and unprofitable. It produces the material and immaterial aggregates both at rebirth-linking and during the course of an existence. (x) \emph{Consolidating} kamma cannot produce result, but when result has already been produced in the provision of rebirth-linking by other kamma, it consolidates the pleasure or pain that arises and makes it last. (xi) And when result has already been produced in the provision of rebirth-linking by other kamma, \emph{frustrating} kamma frustrates and obstructs the pleasure or pain that arises and does not allow it to last. (xii) Supplanting kamma is itself both profitable and unprofitable; \textcolor{brown}{\textit{[602]}} and it supplants other, weaker kamma, prevents its resulting and usurps that kamma’s opportunity in order to cause its own result. But when the opportunity has thus been furnished by the [other] kamma, it is that [supplanting kamma’s] result that is called arisen.\footnote{\vismAssertFootnoteCounter{5}\vismHypertarget{XIX.n5}{}See the various meanings of “arisen” given in \hyperlink{XXII.81}{XXII.81f.}{}

                            “Another method is this: when some kamma has been done and there is, either in rebirth-linking or in the course of an existence, the arising of material instances due to the result of kamma performed, that kamma is ‘\emph{productive}.’ When some kamma has been performed and the desirable or undesirable fruit generated by other kamma has its production facilitated and its endurance aided and lengthened by the suppression of conditions that would interfere with it and by the arousing of conditions that would strengthen it, that kamma is ‘\emph{supporting}.’ When some kamma has been performed and profitable fruit or unprofitable fruit generated by productive kamma is obstructed by it respectively in the form of sickness or of disquieting of elements, that is ‘\emph{frustrating}’ kamma. But when some kamma has been done by which the fruit of other kamma is ruined and cut off by being supplanted by what cuts it off although it was fit for longer endurance because of the efficacy of the kamma that was producing it, that kamma is ‘\emph{supplanting}’” (\textbf{\cite{Vism-mhṭ}772}).}

                    \vismParagraph{XIX.17}{17}{}
                    The succession of kamma and its result in the twelve classes of kamma is clear in its true nature only to the Buddhas’ “knowledge of kamma and its result,” which knowledge is not shared by disciples.\footnote{\vismAssertFootnoteCounter{6}\vismHypertarget{XIX.n6}{}“Because it is a speciality of the Buddha and because it is the province of the knowledge that is not shared by disciples (see \textbf{\cite{Paṭis}I 121f.}), it is called ‘not shared by disciples.’ That is why only a part can be known; it cannot all be known because it is not the province of such knowledge” (\textbf{\cite{Vism-mhṭ}772}).} But the succession of kamma and its result can be known in part by one practicing insight. That is why this explanation of difference in kamma shows only the mere headings.

                    This is how one man discerns mentality-materiality by means of the round of kamma and the round of kamma-result, applying this twelve-fold kamma classification to the round of kamma.

                    \vismParagraph{XIX.18}{18}{}
                    When he has thus seen by means of the round of kamma and the round of kamma-result how mentality-materiality’s occurrence is due to a condition, he sees that as now, so in the past, its occurrence was due to a condition by means of the round of kamma and the round of kamma-result, and that in the future its occurrence will be due to a condition by means of the round of kamma and the round of kamma-result. This is kamma and kamma-result, the round of kamma and the round of kamma-result, the occurrence of kamma and the occurrence of \marginnote{\textcolor{teal}{\footnotesize\{685|627\}}}{}kamma-result, the continuity of kamma and the continuity of kamma-result, action and the fruit of action:
                    \begin{verse}
                        Kamma-result proceeds from kamma,\\{}
                        Result has kamma for its source,\\{}
                        Future becoming springs from kamma,\\{}
                        And this is how the world goes round.
                    \end{verse}


                    \vismParagraph{XIX.19}{19}{}
                    When he sees thus, he abandons all his uncertainty, that is to say, the sixteen kinds described in the way beginning, “Was I in the past?” [see \hyperlink{XIX.6}{§6}{}].
                \subsection[\vismAlignedParas{§19–20}7. No Doer Apart from Kamma and Result]{7. No Doer Apart from Kamma and Result}

                    In all kinds of becoming, generation, destiny, station, and abode there appears only mentality-materiality, which occurs by means of linking of cause with fruit. He sees no doer over and above the doing, no experiencer of the result over and above the occurrence of the result. But he sees clearly with right understanding that the wise say “doer” when there is doing and “experiencer” when there is experiencing simply as a mode of common usage.

                    \vismParagraph{XIX.20}{20}{}
                    Hence the Ancients said:
                    \begin{verse}
                        There is no doer of a deed\\{}
                        Or one who reaps the deed’s result;\\{}
                        Phenomena alone flow on—\\{}
                        No other view than this is right.
                    \end{verse}

                    \begin{verse}
                        And so, while kamma and result\\{}
                        Thus causally maintain their round,\\{}
                        As seed and tree succeed in turn,\\{}
                        No first beginning can be shown.
                    \end{verse}

                    \begin{verse}
                        Nor in the future round of births\\{}
                        Can they be shown not to occur:\\{}
                        Sectarians, not knowing this,\\{}
                        Have failed to gain self-mastery. \textcolor{brown}{\textit{[603]}}
                    \end{verse}

                    \begin{verse}
                        They assume a being, see it as\\{}
                        Eternal or annihilated.\\{}
                        Adopt the sixty-two wrong views,\\{}
                        Each contradicting one another.
                    \end{verse}

                    \begin{verse}
                        The stream of craving bears them on\\{}
                        Caught in the meshes of their views:\\{}
                        And as the stream thus bears them on\\{}
                        They are not freed from suffering.
                    \end{verse}

                    \begin{verse}
                        A monk, disciple of the Buddha,\\{}
                        With direct knowledge of this fact\\{}
                        Can penetrate this deep and subtle\\{}
                        Void conditionality.
                    \end{verse}

                    \begin{verse}
                        \marginnote{\textcolor{teal}{\footnotesize\{686|628\}}}{}There is no kamma in result,\\{}
                        Nor does result exist in kamma;\\{}
                        Though they are void of one another,\\{}
                        There is no fruit without the kamma.
                    \end{verse}

                    \begin{verse}
                        As fire does not exist inside\\{}
                        The sun, a gem, cow-dung, nor yet\\{}
                        Outside them, but is brought to be\\{}
                        By means of its component parts,
                    \end{verse}

                    \begin{verse}
                        So neither can result be found\\{}
                        Within the kamma, nor without;\\{}
                        Nor does the kamma still persist\\{}
                        [In the result it has produced].
                    \end{verse}

                    \begin{verse}
                        The kamma of its fruit is void;\\{}
                        No fruit exists yet in the kamma;\\{}
                        And still the fruit is born from it,\\{}
                        Wholly depending on the kamma.
                    \end{verse}

                    \begin{verse}
                        For here there is no Brahmā God,\\{}
                        Creator of the round of births,\\{}
                        Phenomena alone flow on—\\{}
                        Cause and component their condition.
                    \end{verse}

            \section[\vismAlignedParas{§21–27}Full-Understanding of the Known]{Full-Understanding of the Known}

                \vismParagraph{XIX.21}{21}{}
                When he has discerned the conditions of mentality-materiality in this way by means of the round of kamma and the round of kamma-result, and has abandoned uncertainty about the three periods of time, then all past, future and present states are understood by him in accordance with death and rebirth-linking. This is his full-understanding of the known (see \hyperlink{XX.3}{XX.3}{}).

                \vismParagraph{XIX.22}{22}{}
                He understands thus: “Aggregates produced in the past with kamma as condition ceased there too. But other aggregates are produced in this becoming with past kamma as their condition, although there is no single thing that has come over from the past becoming to this becoming. And aggregates produced in this becoming with kamma as their condition will cease. And in the future becoming other aggregates will be produced, although no single thing will go over from this becoming to the future becoming.

                “Furthermore, just as, while the recitation from the teacher’s mouth does not enter into the pupil’s mouth, yet recitation does not because of that fail to take place in the pupil’s mouth—and while the potion drunk by the proxy does not enter the sick man’s stomach, yet the sickness does not because of that fail to be cured—and while the arrangement of the ornaments on the face does not pass over to the reflection of the face in the looking glass, yet the arrangement of the ornaments does not because of that fail to appear—and while the flame of a lamp does not move over from one wick to another, yet the flame does not because of that fail to be produced—so too, while nothing whatever moves over from the past becoming to this becoming, or from this to the future becoming, \textcolor{brown}{\textit{[604]}} yet \marginnote{\textcolor{teal}{\footnotesize\{687|629\}}}{}aggregates, bases, and elements do not fail to be produced here with aggregates, bases, and elements in the past becoming as their condition, or in the future becoming with aggregates, bases, and elements here as their condition.”

                \vismParagraph{XIX.23}{23}{}
                
                \begin{verse}
                    Just as eye-consciousness comes next\\{}
                    Following on mind element,\\{}
                    Which, though it does not come from that,\\{}
                    Yet fails not next to be produced,
                \end{verse}

                \begin{verse}
                    So too, in rebirth-linking, conscious\\{}
                    Continuity takes place:\\{}
                    The prior consciousness breaks up,\\{}
                    The subsequent is born from that.
                \end{verse}

                \begin{verse}
                    They have no interval between,\\{}
                    Nor gap [that separates the two];\\{}
                    While naught whatever passes over,\\{}
                    Still rebirth-linking comes about.
                \end{verse}


                \vismParagraph{XIX.24}{24}{}
                When all states are understood by him thus in accordance with death and rebirth-linking, his knowledge of discerning the conditions of mentality-materiality is sound in all its aspects and the sixteen kinds of doubt are more effectively abandoned. And not only that, but the eight kinds of doubt that occur in the way beginning thus, “He is doubtful about the Master” (\textbf{\cite{A}III 248}; \textbf{\cite{Dhs}§1004}) are abandoned too, and the sixty-two kinds of views are suppressed (See DN 1 and MN 102).

                \vismParagraph{XIX.25}{25}{}
                The knowledge that has been established by the overcoming of doubt about the three periods of time by discerning the conditions of mentality-materiality according to the various methods should be understood as “purification by overcoming doubt.” Other terms for it are “knowledge of the relations of states” and “correct knowledge” and “right vision.”

                \vismParagraph{XIX.26}{26}{}
                For this is said: “Understanding of discernment of conditions thus, ‘Ignorance is a condition, formations are conditionally arisen, and both these states are conditionally arisen,’ is knowledge of the causal relationship of states” (\textbf{\cite{Paṭis}I 50}). And:

                “When he brings to mind as impermanent, what states does he correctly know and see? How is there right seeing? How, by inference from that, are all formations clearly seen as impermanent? Wherein is doubt abandoned? When he brings to mind as painful … When he brings to mind as not-self, what states does he correctly know and see? … Wherein is doubt abandoned?

                “When he brings to mind as impermanent, he correctly knows and sees the sign. Hence ‘right seeing’ is said. Thus, by inference from that, all formations are clearly seen as impermanent. Herein doubt is abandoned. When he brings to mind as painful, he correctly knows and sees occurrence. Hence … When he brings to mind as not-self, he correctly knows and sees the sign and occurrence. Hence ‘right seeing’ is said. Thus, by inference from that, all states are clearly seen as not-self. Herein doubt is abandoned. \marginnote{\textcolor{teal}{\footnotesize\{688|630\}}}{}“Correct knowledge and right seeing and overcoming of doubt \textcolor{brown}{\textit{[605]}}—are these things different in meaning and different in the letter or are they one in meaning and only the letter is different? Correct knowledge and right seeing and overcoming of doubt—these things are one in meaning and only the letter is different” (\textbf{\cite{Paṭis}II 62f.}).

                \vismParagraph{XIX.27}{27}{}
                When a man practicing insight has become possessed of this knowledge, he has found comfort in the Buddhas’ Dispensation, he has found a foothold, he is certain of his destiny, he is called a “lesser stream-enterer.”
                \begin{verse}
                    So would a bhikkhu overcome\\{}
                    His doubts, then ever mindfully\\{}
                    Let him discern conditions both\\{}
                    Of mind and matter thoroughly.
                \end{verse}


                The nineteenth chapter called “The Description of Purification by Overcoming Doubt” in the Treatise on the Development of Understanding in the \emph{Path of Purification }composed for the purpose of gladdening good people.
        \chapter[Purification by Knowledge and Vision of What is the Path and What is Not the Path]{Purification by Knowledge and Vision of What is the Path and What is Not the Path\vismHypertarget{XX}\newline{\textnormal{\emph{Maggāmagga-ñāṇadassana-visuddhi-niddesa}}}}
            \label{XX}

            \section[\vismAlignedParas{§1–5}Introductory]{Introductory}

                \vismParagraph{XX.1}{1}{}
                \marginnote{\textcolor{teal}{\footnotesize\{689|631\}}}{}\textcolor{brown}{\textit{[606]}} The knowledge established by getting to know the path and the not-path thus, “This is the path, this is not the path,” is called “purification by knowledge and vision of what is the path and what is not the path.”
                \subsection[\vismAlignedParas{§2}The Fifth Purification]{The Fifth Purification}

                    \vismParagraph{XX.2}{2}{}
                    One who desires to accomplish this should first of all apply himself to the inductive insight called “comprehension by groups.\footnote{\vismAssertFootnoteCounter{1}\vismHypertarget{XX.n1}{}“Comprehension by placing together in groups (totals) the states that are differentiated into past, future and present is ‘\emph{comprehension by groups}.’ This, it seems, is the term used by the inhabitants of Jambudīpa (India). However, insight into states by means of the method beginning, ‘Any materiality whatever’ (\textbf{\cite{M}III 16}) is ‘\emph{inductive insight}.’ This, it seems, is the term used by the inhabitants of Tambapaṇṇidīpa (Sri Lanka). That is why he said “\emph{to inductive insight called comprehension by groups}’” (\textbf{\cite{Vism-mhṭ}778}).}” Why? Because knowledge of what is the path and what is not the path appears in connection with the appearance of illumination, etc. (\hyperlink{XX.105}{XX.105f.}{}) in one who has begun insight. For it is after illumination, etc., have appeared in one who has already begun insight that there comes to be knowledge of what is the path and what is not the path. And comprehension by groups is the beginning of insight. That is why it is set forth next to the overcoming of doubt. Besides, knowledge of what is the path and what is not the path arises when “full-understanding as investigation” is occurring, and full-understanding as investigation comes next to full-understanding as the known (see \hyperlink{XIX.21}{XIX.21}{}). So this is also a reason why one who desires to accomplish this purification by knowledge and vision of what is the path and what is not the path should first of all apply himself to comprehension by groups.
                \subsection[\vismAlignedParas{§3–5}The Three Kinds of Full-Understanding]{The Three Kinds of Full-Understanding}

                    \vismParagraph{XX.3}{3}{}
                    Here is the exposition: there are three kinds of mundane full-understanding, that is, full-understanding as the known, full-understanding as investigation, and full-understanding as abandoning, with reference to which it was said: “Understanding that is direct-knowledge is knowledge in the sense of being known. Understanding that is full-understanding is knowledge in the sense of \marginnote{\textcolor{teal}{\footnotesize\{690|632\}}}{}investigating. Understanding that is abandoning is knowledge in the sense of giving up” (\textbf{\cite{Paṭis}I 87}).

                    Herein, the understanding that occurs by observing the specific characteristics of such and such states thus, “Materiality (\emph{rūpa}) has the characteristic of being molested (\emph{ruppana}); feeling has the characteristic of being felt,” is called \emph{full-understanding as the known. }The understanding consisting in insight with the general characteristics as its object that occurs in attributing a general characteristic to those same states in the way beginning, “Materiality is impermanent, \textcolor{brown}{\textit{[607]}} feeling is impermanent” is called \emph{full-understanding as investigation.}\footnote{\vismAssertFootnoteCounter{2}\vismHypertarget{XX.n2}{}\emph{Tīraṇa} could also be rendered by “judging.” On specific and general characteristics Vism-mhṭ says: “Hardness, touching, etc., as the respective characteristics of earth, contact, etc., which are observable at all three instants [of arising, presence and dissolution], are apprehended by their being established as the respective individual essences of definite materialness. But it is not so with the characteristics of impermanence, and so on. These are apprehended as though they were attributive material instances because they have to be apprehended under the respective headings of dissolution and rise and fall, of oppression, and of insusceptibility to the exercise of mastery” (\textbf{\cite{Vism-mhṭ}779}). See \hyperlink{XXI.n3}{Ch. XXI, note 3}{}.

                            The “planes” given here in §4 are not quite the same as described in \hyperlink{XXII.107}{XXII.107}{}.} The understanding consisting in insight with the characteristics as its object that occurs as the abandoning of the perception of permanence, etc., in those same states is called\emph{ full-understanding as abandoning}.

                    \vismParagraph{XX.4}{4}{}
                    Herein, the plane of \emph{full-understanding as the known }extends from the delimitation of formations (\hyperlink{XVIII}{Ch. XVIII}{}) up to the discernment of conditions (Ch. XIX); for in this interval the penetration of the specific characteristics of states predominates. The plane of \emph{full-understanding as investigation }extends from comprehension by groups up to contemplation of rise and fall (\hyperlink{XXI.3}{XXI.3f.}{}); for in this interval the penetration of the general characteristics predominates. The plane of \emph{full-understanding as abandoning }extends from contemplation of dissolution onwards (\hyperlink{XXI.10}{XXI.10}{}); for from there onwards the seven contemplations that effect the abandoning of the perception of permanence, etc., predominate thus:

                    (1) Contemplating [formations] as impermanent, a man abandons the perception of permanence.

                    (2) Contemplating [them] as painful, he abandons the perception of pleasure.

                    (3) Contemplating [them] as not-self, he abandons the perception of self.

                    (4) Becoming dispassionate, he abandons delighting.

                    (5) Causing fading away, he abandons greed.

                    (6) Causing cessation, he abandons originating.

                    (7) Relinquishing, he abandons grasping (\textbf{\cite{Paṭis}I 58}).\footnote{\vismAssertFootnoteCounter{3}\vismHypertarget{XX.n3}{}“‘\emph{Contemplating as impermanent}’ is contemplating, comprehending, formations in the aspect of impermanence. ‘\emph{The perception of permanence}’ is the wrong perception that they are permanent, eternal; the kinds of consciousness associated with wrong view should be regarded as included under the heading of ‘perception.’ So too with what follows. ‘\emph{Becoming dispassionate}’ is seeing formations with dispassion by means of the contemplation of dispassion induced by the contemplations of impermanence, and so on. ‘\emph{Delighting}’ is craving accompanied by happiness. ‘Causing fading away’ is contemplating in such a way that greed (\emph{rāga}) for formations does not arise owing to the causing of greed to fade (\emph{virajjana}) by the contemplation of fading away (\emph{virāgānupassanā}); for one who acts thus is said to abandon greed. ‘\emph{Causing cessation}’ is contemplating in such a way that, by the contemplation of cessation, formations cease only, they do not arise in the future through a new becoming; since one who acts thus is said to abandon the arousing (originating) of formations because of producing the nature of non-arising. ‘\emph{Relinquishing}’ is relinquishing in such a way that, by the contemplation of relinquishment, formations are not grasped anymore; hence he said, ‘\emph{He abandons grasping}’; or the meaning is that he relinquishes apprehending [them] as permanent, and so on” (\textbf{\cite{Vism-mhṭ}780}).}

                    \vismParagraph{XX.5}{5}{}
                    \marginnote{\textcolor{teal}{\footnotesize\{691|633\}}}{}So, of these three kinds of full-understanding, only full-understanding as the known has been attained by this meditator as yet, which is because the delimitation of formations and the discernment of conditions have already been accomplished; the other two still remain to be attained. Hence it was said above: “Besides, knowledge of what is the path and what is not the path arises when full-understanding as investigation is occurring, and full-understanding as investigation comes next to full-understanding as the known. So this is also a reason why one who desires to accomplish this purification by knowledge and vision of what is the path and what is not the path should first of all apply himself to comprehension by groups” (\hyperlink{XX.2}{§2}{}).
            \section[\vismAlignedParas{§6–130}Insight]{Insight}
                \subsection[\vismAlignedParas{§6–12}Comprehension by Groups]{Comprehension by Groups}

                    \vismParagraph{XX.6}{6}{}
                    Here is the text:

                    “How is it that understanding of defining past, future and present states by summarization is knowledge of comprehension?

                    “Any materiality whatever, whether past, future or present, internal or external, gross or subtle, inferior or superior, far or near—he defines all materiality as impermanent: this is one kind of comprehension. He defines it as painful: this is one kind of comprehension. He defines it as not-self: this is one kind of comprehension. Any feeling whatever … Any perception whatever … Any formations whatever … Any consciousness whatever … He defines all consciousness as impermanent: … He defines it as not-self: this is one kind of comprehension. The eye … (etc.) … ageing-and-death, whether past, future or present, he defines it\textbf{ }as impermanent: this is one kind of comprehension. He defines it as painful: this is one kind of comprehension. He defines it as not-self: this is one kind of comprehension.

                    \vismParagraph{XX.7}{7}{}
                    “Understanding of defining by summarization thus, ‘Materiality, whether past, future, or present, is impermanent in the sense of destruction, painful in the sense of terror, not-self in the sense of having no core,’ is knowledge of comprehension. Understanding of defining by generalization thus, ‘Feeling … \marginnote{\textcolor{teal}{\footnotesize\{692|634\}}}{}\textcolor{brown}{\textit{[608]}} (etc.) … Consciousness … Eye … (etc.) … Ageing-and-death, whether past …’ is knowledge of comprehension.

                    “Understanding of defining by summarization thus, ‘Materiality, whether past, future, or present, is impermanent, formed, dependently arisen, subject to destruction, subject to fall, subject to fading away, subject to cessation,’ is knowledge of comprehension. Understanding of defining by generalization thus, ‘Feeling … (etc.) … Consciousness … Eye … (etc.) … Ageing-and-death, whether past, future, or present, is impermanent, formed, dependently arisen, subject to destruction, subject to fall, subject to fading away, subject to cessation’ is knowledge of comprehension.

                    \vismParagraph{XX.8}{8}{}
                    “Understanding of defining by summarization thus, ‘With birth as condition there is ageing-and-death; without birth as condition there is no ageing-and-death,’ is knowledge of comprehension. Understanding of defining by generalization thus, ‘In the past and in the future with birth as condition there is ageing-and-death; without birth as condition there is no ageing-and-death,’ is knowledge of comprehension. Understanding of defining by generalization thus, ‘With becoming as condition there is birth … With ignorance as condition there are formations; without ignorance as condition there are no formations,’ is knowledge of comprehension. Understanding of defining by generalization thus, ‘In the past and in the future with ignorance as condition there are formations; without ignorance as condition there are no formations’ is knowledge of comprehension.

                    “Knowledge is in the sense of that being known and understanding is in the sense of the act of understanding that. Hence it was said: ‘Understanding of defining past, future, and present states by summarization is knowledge of comprehension’” (\textbf{\cite{Paṭis}I 53f.}).

                    \vismParagraph{XX.9}{9}{}
                    Herein, the abbreviation, “The eye … (etc.) … Ageing-and-death,” should be understood to represent the following sets of things elided:

                    
                        \begin{enumerate}[1.,nosep]
                            \item The states that occur in the doors [of consciousness] together with the doors and the objects.
                            \item The five aggregates.
                            \item The six doors.
                            \item The six objects.
                            \item The six kinds of consciousness.
                            \item The six kinds of contact.
                            \item The six kinds of feeling.
                            \item The six kinds of perception.
                            \item The six kinds of volition.
                            \item The six kinds of craving.
                            \item The six kinds of applied thought.
                            \item The six kinds of sustained thought.
                            \item The six elements.
                            \item The ten kasiṇas.
                            \item The thirty-two bodily aspects.
                            \item The twelve bases.
                            \item \marginnote{\textcolor{teal}{\footnotesize\{693|635\}}}{}The eighteen elements.
                            \item The twenty-two faculties.
                            \item The three elements.
                            \item The nine kinds of becoming.
                            \item The four jhānas.
                            \item The four measureless states.
                            \item The four [immaterial] attainments.
                            \item The twelve members of the dependent origination.
                        \end{enumerate}

                    \vismParagraph{XX.10}{10}{}
                    For this is said in the Paṭisambhidā in the description of what is to be directly known: “Bhikkhus, all is to be directly known. And what is all that is to be directly known? \textcolor{brown}{\textit{[609]}} (1) Eye is to be directly known; visible objects are to be directly known; eye-consciousness … eye-contact … feeling, pleasant or painful or neither-painful-nor-pleasant, that arises due to eye-contact is also to be directly known. Ear … Mind … feeling, pleasant or painful or neither-painful-nor-pleasant, that arises due to mind-contact is also to be directly known.

                    \vismParagraph{XX.11}{11}{}
                    “(2) Materiality is to be directly known … consciousness is to be directly known. (3) Eye … mind … (4) Visible objects … mental objects … (5) Eye-consciousness … mind-consciousness … (6) Eye-contact … mind-contact … (7) Eye-contact-born feeling … mind-contact-born feeling … (8) Perception of visible objects … perception of mental objects … (9) Volition regarding visible objects … volition regarding mental objects … (10) Craving for visible objects … craving for mental objects … (11) Applied thought about visible objects … applied thought about mental objects … (12) Sustained thought about visible objects … sustained thought about mental objects … (13) The earth element … the consciousness element … (14) The earth kasiṇa … the consciousness kasiṇa … (15) Head hairs … brain … (16) The eye base … the mental object base … (17) The eye element … the mind-consciousness element … (18) The eye faculty … the final-knower faculty … (19) The sense-desire element … the fine-material element … the immaterial element … (20) Sense-desire becoming … fine-material becoming … immaterial becoming … percipient becoming … non-percipient becoming … neither percipient nor non-percipient becoming … one-constituent becoming … four-constituent becoming … five-constituent becoming … (21) The first jhāna … the fourth jhāna … (22) The mind-deliverance of loving-kindness … the mind-deliverance of equanimity … (23) The attainment of the base consisting of boundless space … the attainment of the base consisting of neither perception nor non-perception … (24) Ignorance is to be directly known … ageing-and-death is to be directly known” (\textbf{\cite{Paṭis}I 5f.}).

                    \vismParagraph{XX.12}{12}{}
                    Since all this detail is given there it has been abbreviated here. But what is thus abbreviated includes the supramundane states. These should not be dealt with at this stage because they are not amenable to comprehension. And as regards those that are amenable to comprehension a beginning should be made by comprehending those among them that are obvious to and easily discernible by the individual [meditator].
                \subsection[\vismAlignedParas{§13–17}Comprehension by Groups—Application of Text]{Comprehension by Groups—Application of Text}

                    \vismParagraph{XX.13}{13}{}
                    Here is the application of the directions dealing with the aggregates: “Any materiality whatever, (i–iii) whether past, future, or present, (iv–v) internal or \marginnote{\textcolor{teal}{\footnotesize\{694|636\}}}{}external, (vi–vii) gross or subtle, (viii–ix) inferior or superior, (x-xi) far or near—he defines all materiality as impermanent: this is one kind of comprehension. He defines it as painful: \textcolor{brown}{\textit{[610]}} this is one kind of comprehension. He defines it as not-self: this is one kind of comprehension” (see \hyperlink{XX.6}{§6}{}). At this point this bhikkhu [takes] all materiality, which is described without specifying as “\emph{any materiality whatever},” and having delimited it in the eleven instances, namely, with the past triad and with the four dyads beginning with the internal dyad, he “\emph{defines all materiality as impermanent},” he comprehends that it is impermanent. How? In the way stated next. For this is said: “\emph{Materiality, whether past, future or present, is impermanent in the sense of destruction}.”

                    \vismParagraph{XX.14}{14}{}
                    Accordingly, he comprehends the materiality that is past as “\emph{impermanent in the sense of destruction” }because it was destroyed in the past and did not reach this becoming; and he comprehends the materiality that is future as “\emph{impermanent in the sense of destruction” }since it will be produced in the next becoming, will be destroyed there too, and will not pass on to a further becoming; and he comprehends the materiality that is present as “\emph{impermanent in the sense of destruction” }since it is destroyed here and does not pass beyond. And he comprehends the materiality that is internal as “\emph{impermanent in the sense of destruction}” since it is destroyed as internal and does not pass on to the external state. And he comprehends the materiality that is external … gross … subtle … inferior … superior … far … And he comprehends the materiality that is near as “\emph{impermanent in the sense of destruction}” since it is destroyed there and does not pass on to the far state. And all this is impermanent in the sense of destruction. Accordingly, there is “\emph{one kind of comprehension” }in this way; but it is effected in eleven ways.

                    \vismParagraph{XX.15}{15}{}
                    And all that [materiality] is “\emph{painful in the sense of terror}.” In the sense of terror because of its terrifyingness; for what is impermanent brings terror, as it does to the deities in the Sīhopama Sutta (\textbf{\cite{S}III 84}). So this is also painful in the sense of terror. Accordingly, there is one kind of comprehension in this way too; but it is effected in eleven ways.

                    \vismParagraph{XX.16}{16}{}
                    And just as it is painful, so too all that [materiality] is “\emph{not-self in the sense of having no core}.” In the sense of having no core because of the absence of any core of self conceived as a self, an abider, a doer, an experiencer, one who is his own master; for what is impermanent is painful (\textbf{\cite{S}III 82}), and it is impossible to escape the impermanence, or the rise and fall and oppression, of self, so how could it have the state of a doer, and so on? Hence it is said, “Bhikkhus, were materiality self, it would not lead to affliction” (\textbf{\cite{S}III 66}), and so on. So this is also not-self in the sense of having no core. Accordingly, there is one kind of comprehension in this way too, but it is effected in eleven ways. \textcolor{brown}{\textit{[611]}} The same method applies to feeling, and so on.

                    \vismParagraph{XX.17}{17}{}
                    But what is impermanent is necessarily classed as formed, etc., and so in order to show the synonyms for that [impermanence], or in order to show how the attention given to it occurs in different ways, it is restated in the text thus: “Materiality, whether past, future, or present, is impermanent, formed, dependently arisen, subject to destruction, subject to fall, subject to fading away, subject to cessation” (\hyperlink{XX.7}{§7}{}). The same method applies to feeling, and so on.
                \subsection[\vismAlignedParas{§18–20}Strengthening of Comprehension in Forty Ways]{Strengthening of Comprehension in Forty Ways}

                    \vismParagraph{XX.18}{18}{}
                    \marginnote{\textcolor{teal}{\footnotesize\{695|637\}}}{}Now, when the Blessed One was expounding conformity knowledge, he [asked the question]: “By means of what forty aspects does he acquire liking that is in conformity? By means of what forty aspects does he enter into the certainty of rightness?” (P‘8).\footnote{\vismAssertFootnoteCounter{4}\vismHypertarget{XX.n4}{}“‘\emph{Liking that is in conformity}’ is a liking for knowledge that is in conformity with the attainment of the path. Actually the knowledge itself is the ‘liking’ (\emph{khanti}) since it likes (\emph{khamati}), it endures, defining by going into the individual essence of its objective field. The ‘\emph{certainty of rightness}’ is the noble path; for that is called the rightness beginning with right view and also the certainty of an irreversible trend” (\textbf{\cite{Vism-mhṭ}784}).} In the answer to it comprehension of impermanence, etc., is set forth by him analytically in the way beginning: “[Seeing] the five aggregates as impermanent, as painful, as a disease, a boil, a dart, a calamity, an affliction, as alien, as disintegrating, as a plague, a disaster, a terror, a menace, as fickle, perishable, unenduring, as no protection, no shelter, no refuge, as empty, vain, void, not-self, as a danger, as subject to change, as having no core, as the root of calamity, as murderous, as due to be annihilated, as subject to cankers, as formed, as Māra’s bait, as subject to birth, subject to ageing, subject to illness, subject to death, subject to sorrow, subject to lamentation, subject to despair, subject to defilement. Seeing the five aggregates as impermanent, he acquires liking that is in conformity. And seeing that the cessation of the five aggregates is the permanent Nibbāna, he enters into the certainty of rightness” (\textbf{\cite{Paṭis}II 238}). So in order to strengthen that same comprehension of impermanence, pain, and not-self in the five aggregates, this [meditator] also comprehends these five aggregates by means of that [kind of comprehension].

                    \vismParagraph{XX.19}{19}{}
                    How does he do it? He does it by means of comprehension as impermanent, etc., stated specifically as follows: He comprehends each aggregate as \emph{impermanent }because of non-endlessness, and because of possession of a beginning and an end; as \emph{painful }because of oppression by rise and fall, and because of being the basis for pain; as \emph{a disease }because of having to be maintained by conditions, and because of being the root of disease; as \emph{a boil }because of being consequent upon impalement by suffering, because of oozing with the filth of defilements, and because of being swollen by arising, ripened by ageing, and burst by dissolution; as \emph{a dart }because of producing oppression, because of penetrating inside, and because of being hard to extract; as \emph{a calamity }because of having to be condemned, because of bringing loss, and \textcolor{brown}{\textit{[612]}} because of being the basis for calamity; as \emph{an affliction }because of restricting freedom, and because of being the foundation for affliction; as \emph{alien }because of inability to have mastery exercised over them, and because of intractability; as \emph{disintegrating }because of crumbling through sickness, ageing and death; as \emph{a plague }because of bringing various kinds of ruin; as \emph{a disaster }because of bringing unforeseen and plentiful adversity, and because of being the basis for all kinds of terror, and because of being the opposite of the supreme comfort called the stilling of all suffering; as a \emph{menace }because of being bound up with many kinds of adversity, because of being menaced\footnote{\vismAssertFootnoteCounter{5}\vismHypertarget{XX.n5}{}\emph{Upasaṭṭhatā—}“being menaced;” abstr. noun fr. pp. of \emph{upa + saj}; not as such in PED.} by ills, and because of unfitness, as a menace, to be entertained; as \emph{fickle }because of fickle \marginnote{\textcolor{teal}{\footnotesize\{696|638\}}}{}insecurity due to sickness, ageing and death, and to the worldly states of gain, etc.;\footnote{\vismAssertFootnoteCounter{6}\vismHypertarget{XX.n6}{}The eight worldly states are: gain and non-gain, fame and non-fame, blame and praise, and pleasure and pain (\textbf{\cite{D}III 160}).} as \emph{perishable }because of having the nature of perishing both by violence and naturally; as \emph{unenduring }because of collapsing on every occasion\footnote{\vismAssertFootnoteCounter{7}\vismHypertarget{XX.n7}{}\emph{Avatthā—}“occasion”: not in PED.} and because of lack of solidity; \emph{as no protection }because of not protecting, and because of affording no safety; \emph{as no shelter }because of unfitness to give shelter,\footnote{\vismAssertFootnoteCounter{8}\vismHypertarget{XX.n8}{}\emph{Allīyituṃ—}“to give shelter”: not in PED, but see \emph{leṇa}.} and because of not performing the function of a shelter for the unsheltered;\footnote{\vismAssertFootnoteCounter{9}\vismHypertarget{XX.n9}{}\emph{Allīnānaṃ—}“for the unsheltered”: \emph{allīna} = pp. of \emph{ā + līyati} (see note 8 above), the “un-sheltered.” Not in PED. Not to be confused with \emph{allīna} = adherent (pp. of \emph{ā + līyati}, to stick, to be contiguous); see e.g. \hyperlink{XIV.46}{XIV.46}{}.} as \emph{no refuge }because of failure to disperse fear\footnote{\vismAssertFootnoteCounter{10}\vismHypertarget{XX.n10}{}\textbf{\cite{Vism-mhṭ}} has “\emph{Jāti-ādi-bhayānaṃ hiṃsanaṃ vidhamanaṃ bhayasāraṇattaṃ},” which suggests the rendering “because of not being a refuge from fear.”} in those who depend on them; as \emph{empty }because of their emptiness of the lastingness, beauty, pleasure and self that are conceived about them; as \emph{vain }because of their emptiness, or because of their triviality; for what is trivial is called “vain” in the world; as \emph{void }because devoid of the state of being an owner, abider, doer, experiencer, director; as \emph{not-self }because of itself having no owner, etc.; as \emph{danger }because of the suffering in the process of becoming, and because of the danger in suffering or, alternatively, as \emph{danger }(\emph{ādīnava}) because of resemblance to misery (\emph{ādīna})\footnote{\vismAssertFootnoteCounter{11}\vismHypertarget{XX.n11}{}\emph{Ādīna—}“misery” or “miserable”: not in PED. \emph{Ādīna—}“misery” or “miserable”: not in PED.}\emph{ }since “danger” (\emph{ādīnava}) means that it is towards misery (\emph{ādīna}) that it moves (\emph{vāti}), goes, advances, this being a term for a wretched man, and the aggregates are wretched too; as \emph{subject to change }because of having the nature of change in two ways, that is, through ageing and through death; as \emph{having no core }because of feebleness, and because of decaying soon like sapwood; as \emph{the root of calamity }because of being the cause of calamity; as \emph{murderous }because of breaking faith like an enemy posing as a friend; as \emph{due to be annihilated }because their becoming disappears, and because their non-becoming comes about; as \emph{subject to cankers }because of being the proximate cause for cankers; as \emph{formed }because of being formed by causes and conditions; as \emph{Māra’s bait }because of being the bait [laid] by the Māra of death and the Māra of defilement; as \emph{subject to birth, to ageing, to illness, }and \emph{to death }because of having birth, ageing, illness and death as their nature; \emph{as subject to sorrow, to lamentation }and \emph{to despair }because of being the cause of sorrow, lamentation and despair; as \emph{subject to defilement }because of being the objective field of the defilements of craving, views and misconduct.

                    \vismParagraph{XX.20}{20}{}
                    Now, there are \textcolor{brown}{\textit{[613]}}\textbf{ }fifty kinds of contemplation of impermanence here by taking the following ten in the case of each aggregate: as impermanent, as disintegrating, as fickle, as perishable, as unenduring, as subject to change, as having no core, as due to be annihilated, as formed, as subject to death. There are twenty-five kinds of contemplation of not-self by taking the following five in the case of each aggregate: as alien, as empty, as vain, as void, as not-self. There are \marginnote{\textcolor{teal}{\footnotesize\{697|639\}}}{}one hundred and twenty-five kinds of contemplation of pain by taking the rest beginning with “as painful, as a disease” in the case of each aggregate.

                    So when a man comprehends the five aggregates by means of this comprehending as impermanent, etc., in its two hundred aspects, his comprehending as impermanent, painful and not-self, which is called “inductive insight,” is strengthened. These in the first place are the directions for undertaking comprehension here in accordance with the method given in the texts.
                \subsection[\vismAlignedParas{§21}Nine Ways of Sharpening the Faculties, Etc.]{Nine Ways of Sharpening the Faculties, Etc.}

                    \vismParagraph{XX.21}{21}{}
                    While thus engaged in inductive insight, however, if it does not succeed, he should sharpen his faculties [of faith, etc.,] in the nine ways stated thus: “The faculties become sharp in nine ways: (1) he sees only the destruction of arisen formations; (2) and in that [occupation] he makes sure of working carefully, (3) he makes sure of working perseveringly, (4) he makes sure of working suitably, and (5) by apprehending the sign of concentration and (6) by balancing the enlightenment factors (7) he establishes disregard of body and life, (8) wherein he overcomes [pain] by renunciation and (9) by not stopping halfway.\footnote{\vismAssertFootnoteCounter{12}\vismHypertarget{XX.n12}{}\emph{Abyosāna—}“not stopping halfway” (another less good reading is \emph{accosāna}): not in PED; but it is a negative form of \emph{vosāna} (q.v.), which is used of Devadatta in the Vinaya Cūḷavagga ( = It 85) and occurs in this sense at \textbf{\cite{M}I 193}. Not in CPD.} He should avoid the seven unsuitable things in the way stated in the Description of the Earth Kasiṇa (\hyperlink{IV.34}{IV.34}{}) and cultivate the seven suitable things, and he should comprehend the material at one time and the immaterial at another.
                \subsection[\vismAlignedParas{§22–42}Comprehension of the Material]{Comprehension of the Material}

                    \vismParagraph{XX.22}{22}{}
                    While comprehending materiality he should see how materiality is generated,\footnote{\vismAssertFootnoteCounter{13}\vismHypertarget{XX.n13}{}“First it has to be seen by inference according to the texts. Afterwards it gradually comes to be seen by personal experience when the knowledge of development gets stronger” (\textbf{\cite{Vism-mhṭ}790}).} that is to say, how this materiality is generated by the four causes beginning with kamma. Herein, when materiality is being generated in any being, it is first generated from kamma. For at the actual moment of rebirth-linking of a child in the womb, first thirty instances of materiality are generated in the triple continuity, in other words, the decads of physical [heart-]basis, body, and sex. And those are generated at the actual instant of the rebirth-linking consciousness’s arising. And as at the instant of its arising, so too at the instant of its presence and at the instant of its dissolution.\footnote{\vismAssertFootnoteCounter{14}\vismHypertarget{XX.n14}{}“It is first generated from kamma because the temperature-born kinds, etc., are rooted in that” (\textbf{\cite{Vism-mhṭ}790}).}

                    \vismParagraph{XX.23}{23}{}
                    Herein, the cessation of materiality is slow and its transformation ponderous, while the cessation of consciousness is swift and its transformation quick (light); hence it is said, “Bhikkhus, I see no other one thing that is so quickly transformed as \textcolor{brown}{\textit{[614]}} the mind” (\textbf{\cite{A}I 10}).

                    \vismParagraph{XX.24}{24}{}
                    For the life-continuum consciousness arises and ceases sixteen times while one material instant endures. With consciousness the instant of arising, instant \marginnote{\textcolor{teal}{\footnotesize\{698|640\}}}{}of presence, and instant of dissolution are equal; but with materiality only the instants of arising and dissolution are quick like those [of consciousness], while the instant of its presence is long and lasts while sixteen consciousnesses arise and cease.

                    \vismParagraph{XX.25}{25}{}
                    The second life-continuum arises with the prenascent physical [heart-]basis as its support, which has already reached presence and arose at the rebirth-linking consciousness’s instant of arising. The third life-continuum arises with the prenascent physical basis as its support, which has already reached presence and arose together with that [second life-continuum consciousness]. The occurrence of consciousness can be understood to happen in this way throughout life. But in one who is facing death sixteen consciousnesses arise with a single prenascent physical [heart-]basis as their support, which has already reached presence.

                    \vismParagraph{XX.26}{26}{}
                    The materiality that arose at the instant of arising of the rebirth-linking consciousness ceases along with the sixteenth consciousness after the rebirth-linking consciousness. That arisen at the instant of presence of the rebirth-linking consciousness ceases together with the instant of arising of the seventeenth. That arisen at the instant of its dissolution ceases on arriving at the instant of presence of the seventeenth.\footnote{\vismAssertFootnoteCounter{15}\vismHypertarget{XX.n15}{}The relationship of the duration of moments of matter and moments of consciousness is dealt with in greater detail in the \emph{Sammohavinodanī} (\textbf{\cite{Vibh-a}25f.}). See also Introduction, note 18.} It goes on occurring thus for as long as the recurrence [of births] continues.

                    Also seventy instances of materiality occur in the same way with the sevenfold continuity [beginning with the eye decad] of those apparitionally born.
                    \subsubsection[\vismAlignedParas{§27–29}(a) Kamma-Born Materiality]{(a) Kamma-Born Materiality}

                        \vismParagraph{XX.27}{27}{}
                        Herein, [as regards kamma-born materiality] the analysis should be understood thus: (1) kamma, (2) what is originated by kamma, (3) what has kamma as its condition, (4) what is originated by consciousness that has kamma as its condition, (5) what is originated by nutriment that has kamma as its condition, (6) what is originated by temperature that has kamma as its condition (\hyperlink{XI.111}{XI.111}{}–\hyperlink{XI.14}{14}{}).

                        \vismParagraph{XX.28}{28}{}
                        Herein, (1)\emph{ kamma} is profitable and unprofitable volition. (2)\emph{ What is originated by kamma }is the kamma-resultant aggregates and the seventy instances of materiality beginning with the eye decad. (3)\emph{ What has kamma as its condition }is the same [as the last] since kamma is the condition that upholds what is originated by kamma.

                        \vismParagraph{XX.29}{29}{}
                        (4) \emph{What is originated by consciousness that has kamma as its condition} is materiality originated by kamma-resultant consciousness. (5)\emph{ What is originated by nutriment that has kamma as its condition} is so called since the nutritive essence that has reached presence in the instances of materiality originated by kamma originates a further octad-with-nutritive-essence-as-eighth, and the nutritive essence there that has reached presence also originates a further one, and so it \marginnote{\textcolor{teal}{\footnotesize\{699|641\}}}{}links up four or five occurrences of octads. (6) \emph{What is originated by temperature that has kamma as its condition} is so called since the kamma-born fire element that has reached presence originates an octad-with-nutritive-essence-as-eighth, which is temperature-originated, and the temperature in that originates a further octad-with-nutritive-essence-as eighth, and so it links up four or five occurrences of octads.

                        This is how the generation of kamma-born materiality in the first place should be seen. \textcolor{brown}{\textit{[615]}}

                        
                    \subsubsection[\vismAlignedParas{§30–34}(b) Consciousness-Born Materiality]{(b) Consciousness-Born Materiality}

                        \vismParagraph{XX.30}{30}{}
                        Also as regards the consciousness-born kinds, the analysis should be understood thus: (1) consciousness, (2) what is originated by consciousness, (3) what has consciousness as its condition, (4) what is originated by nutriment that has consciousness as its condition, (5) what is originated by temperature that has consciousness as its condition.

                        \vismParagraph{XX.31}{31}{}
                        Herein, (1) \emph{consciousness }is the eighty-nine kinds of consciousness. Among these:
                        \begin{verse}
                            Consciousnesses thirty-two,\\{}
                            And twenty-six, and nineteen too,\\{}
                            Are reckoned to give birth to matter,\\{}
                            Postures, also intimation;\\{}
                            Sixteen kinds of consciousness\\{}
                            Are reckoned to give birth to none.
                        \end{verse}


                        As regards the sense sphere, thirty-two consciousnesses, namely, the eight profitable consciousnesses ((1)–(8)), the twelve unprofitable ((22)–(33)), the ten functional excluding the mind element ((71)–(80)), and the two direct-knowledge consciousnesses as profitable and functional, give rise to materiality, to postures, and to intimation. The twenty-six consciousnesses, namely, the ten of the fine-material sphere ((9)–(13), (81)–(85)) and the eight of the immaterial sphere ((14)–(17), (86)–(89)) excluding the resultant [in both cases], and the eight supramundane ((18)–(21), (66)–(69)), give rise to materiality, to postures but not to intimation. The nineteen consciousnesses, namely, the ten life-continuum consciousnesses in the sense sphere ((41)–(49), (56)), the five in the fine-material sphere ((57)–(61)), the three mind elements ((39), (55), (70)), and the one resultant mind-consciousness element without root-cause and accompanied by joy (40), give rise to materiality only, not to postures or to intimation. The sixteen consciousnesses, namely, the two sets of five consciousnesses ((34)–(38), (50)–(54)), the rebirth-linking consciousness of all beings, the death consciousness of those whose cankers are destroyed, and the four immaterial resultant consciousnesses ((62)–(65)) do not give rise to materiality or to postures or to intimation. And those herein that do give rise to materiality do not do so at the instant of their presence or at the instant of their dissolution, for consciousness is weak then. But it is strong at the instant of arising. Consequently it originates materiality then with the prenascent physical basis as its support.

                        \vismParagraph{XX.32}{32}{}
                        (2)\emph{ What is originated by consciousness }is the three other immaterial aggregates and the seventeenfold materiality, namely, the sound ennead, bodily intimation, \marginnote{\textcolor{teal}{\footnotesize\{700|642\}}}{}verbal intimation, the space element, lightness, malleability, wieldiness, growth, and continuity.

                        (3) \emph{What has consciousness as its condition }is the materiality of fourfold origination stated thus: “Postnascent states of consciousness and consciousness-concomitants are a condition, as postnascence condition, for this prenascent body” (\textbf{\cite{Paṭṭh}I 5}).

                        \vismParagraph{XX.33}{33}{}
                        (4) \emph{What is originated by nutriment that has consciousness as its condition: }the nutritive essence that has reached presence in consciousness-originated material instances originates a further octad-with-nutritive-essence-as-eighth, and thus links up two or three occurrences of octads.

                        \vismParagraph{XX.34}{34}{}
                        (5) \emph{What is originated by temperature that has consciousness as its condition: }the consciousness-originated temperature that has \textcolor{brown}{\textit{[616]}} reached presence originates a further octad-with-nutritive-essence-as-eighth, and thus links up two or three occurrences.

                        This is how the generation of consciousness-born materiality should be seen.
                    \subsubsection[\vismAlignedParas{§35–38}(c) Nutriment-Born Materiality]{(c) Nutriment-Born Materiality}

                        \vismParagraph{XX.35}{35}{}
                        Also as regards the nutriment-born kinds, the analysis should be understood thus: (1) nutriment, (2) what is originated by nutriment, (3) what has nutriment as its condition, (4) what is originated by nutriment that has nutriment as its condition, (5) what is originated by temperature that has nutriment as its condition.

                        \vismParagraph{XX.36}{36}{}
                        Herein, (1)\emph{ nutriment }is physical nutriment. (2)\emph{ What is originated by nutriment }is the fourteenfold materiality, namely, (i–viii) that of the octad-with-nutritive-essence-as-eighth originated by nutritive essence that has reached presence by obtaining as its condition kamma-born materiality that is clung to (kammically acquired) and basing itself on that,\footnote{\vismAssertFootnoteCounter{16}\vismHypertarget{XX.n16}{}“‘\emph{By obtaining as its condition kamma-born materiality that is clung-to}’: by this he points out that external un-clung-to nutritive essence does not perform the function of nourishing materiality. He said ‘\emph{and basing itself on that}’ meaning that its obtaining of a condition is owing to its being supported by what is kamma-born. And ‘\emph{clung-to}’ is specifically mentioned in order to rule out any question of there being a ‘kamma-born’ method for ‘materiality originated by consciousness that has kamma as its condition’ just because it happens to be rooted in kamma [There is no such method]” (\textbf{\cite{Vism-mhṭ}793f.}).} and (ix) space element, (x–xiv) lightness, malleability, wieldiness, growth, and continuity.

                        (3) \emph{What has nutriment as its condition }is the materiality of fourfold origination stated thus: “Physical nutriment is a condition, as nutriment condition, for this body” (\textbf{\cite{Paṭṭh}I 5}).

                        \vismParagraph{XX.37}{37}{}
                        (4)\emph{ What is originated by nutriment that has nutriment as its condition: }the nutritive essence that has reached presence in nutriment-originated material instances originates a further octad-with-nutritive-essence-as-eighth and the nutritive essence in that octad originates a further octad, and thus links up the occurrence of octads ten or twelve times. Nutriment taken on one day sustains \marginnote{\textcolor{teal}{\footnotesize\{701|643\}}}{}for as long as seven days; but divine nutritive essence sustains for as long as one or two months. The nutriment taken by a mother originates materiality by pervading the body of the child [in gestation]. Also nutriment smeared on the body originates materiality. Kamma-born nutriment is a name for nutriment that is clung to. That also originates materiality when it has reached presence. And the nutritive essence in it originates a further octad. Thus it links up four or five occurrences.

                        \vismParagraph{XX.38}{38}{}
                        (5)\emph{ What is originated by temperature that has nutriment as its condition}: nutriment-originated fire element that has reached presence originates an octad-with-nutritive-essence-as-eighth that is thus temperature-originated. Here this nutriment is a condition for nutriment-originated material instances as their progenitor. It is a condition for the rest as support, nutriment, presence, and non-disappearance.

                        This is how the generation of nutriment-born materiality should be seen.
                    \subsubsection[\vismAlignedParas{§39–42}(d) Temperature-Born Materiality]{(d) Temperature-Born Materiality}

                        \vismParagraph{XX.39}{39}{}
                        Also as regards the temperature-born kinds, the analysis should be understood thus: (1) temperature, (2) what is originated by temperature, (3) what has temperature as its condition, (4) what is originated by temperature that has temperature as its condition, (5) what is originated by nutriment that has temperature as its condition.

                        \vismParagraph{XX.40}{40}{}
                        Herein, (1)\emph{ temperature} is the fire element of fourfold origination; but it is twofold as hot temperature and cold temperature. (2)\emph{ What is originated by temperature: }the temperature of fourfold origination that has reached presence by obtaining a clung-to condition originates materiality in the body. \textcolor{brown}{\textit{[617]}} That materiality is fifteenfold, namely, sound ennead, space element, lightness, malleability, wieldiness, growth, continuity. (3)\emph{ What has temperature as its condition }is so called since temperature is a condition for the occurrence and for the destruction of materiality of fourfold origination.

                        \vismParagraph{XX.41}{41}{}
                        (4) \emph{What is originated by temperature that has temperature as its condition: }the temperature-originated fire element that has reached presence originates a further octad-with-nutritive-essence-as-eighth, and the temperature in that octad originates a further octad. Thus temperature-originated materiality both goes on occurring for a long period and also maintains itself as well in what is not clung to.\footnote{\vismAssertFootnoteCounter{17}\vismHypertarget{XX.n17}{}“What is intended is head hair, body hair, nails, teeth, skin, callosities, warts, etc., which are separate from the flesh in a living body; otherwise a corpse, and so on” (\textbf{\cite{Vism-mhṭ}795}).}

                        \vismParagraph{XX.42}{42}{}
                        (5)\emph{ What is originated by nutriment that has temperature as its condition: }the temperature-originated nutritive essence that has reached presence originates a further octad-with-nutritive-essence-as-eighth, and the nutritive essence in that originates a further one, thus it links up ten or twelve occurrences of octads.

                        Herein, this temperature is a condition for temperature-originated material instances as their progenitor. It is a condition for the rest as support, presence, and non-disappearance.

                        \marginnote{\textcolor{teal}{\footnotesize\{702|644\}}}{}This is how the generation of temperature-born materiality should be seen.

                        One who sees the generation of materiality thus is said to “comprehend the material at one time” (\hyperlink{XX.21}{§21}{}).\footnote{\vismAssertFootnoteCounter{18}\vismHypertarget{XX.n18}{}“When the generation of materiality is seen its dissolution also is seen, and so he said, ‘\emph{One who sees the generation of materiality thus is said to comprehend the material at one time}’ because of the brevity of states’ occurrence; for it is not the seeing of mere generation that is called comprehension but there must be seeing of rise and fall besides. So too the apprehending of generation in the other instances” (\textbf{\cite{Vism-mhṭ}795}).}

                        
                \subsection[\vismAlignedParas{§43–45}Comprehension of the Immaterial]{Comprehension of the Immaterial}

                    \vismParagraph{XX.43}{43}{}
                    And just as one who is comprehending the material should see the generation of the material, so too one who is comprehending the immaterial should see the generation of the immaterial. And that is through the eighty-one mundane arisings of consciousness, that is to say, it is by kamma accumulated in a previous becoming that this immaterial [mentality] is generated. And in the first place it is generated as [one of] the nineteen kinds of arisings of consciousness as rebirth-linking (\hyperlink{XVII.130}{XVII.130}{}). But the modes in which it is generated should be understood according to the method given in the Description of the Dependent Origination (\hyperlink{XVII.134}{XVII.134f.}{}). That same [nineteenfold arising of consciousness is generated] as life-continuum as well, starting from the consciousness next to rebirth-linking consciousness, and as death consciousness at the termination of the life span. And when it is of the sense sphere, and the object in the six doors is a vivid one, it is also generated as registration.

                    \vismParagraph{XX.44}{44}{}
                    In the course of an existence, eye-consciousness, together with its associated states, supported by light and caused by attention is generated because the eye is intact and because visible data have come into focus. For it is actually when a visible datum has reached presence that it impinges on the eye at the instant of the eye-sensitivity’s presence. When it has done so, the life-continuum arises and ceases twice. Next to arise is the functional mind element with that same object, accomplishing the function of adverting. Next to that, eye-consciousness, which is the result of profitable or of unprofitable [kamma] and sees that same visible datum. \textcolor{brown}{\textit{[618]}} Next, the resultant mind element, which receives that same visible datum. Next, the resultant root-causeless mind-consciousness element, which investigates that same visible datum. Next, the functional mind-consciousness element without root-cause and accompanied by equanimity, which determines that same visible datum. Next, [it is generated either] as one from among the profitable ((l)–(8)), unprofitable ((22)–(33)), or functional ((71) and (73)–(80)) kinds of consciousness belonging to the sense sphere, either as consciousness accompanied by equanimity and without root-cause (71),\footnote{\vismAssertFootnoteCounter{19}\vismHypertarget{XX.n19}{}“This refers to determining” (\textbf{\cite{Vism-mhṭ}795}).} or as five or seven impulsions. Next, in the case of sense-sphere beings, [it is generated] as any of the eleven kinds of registration consciousness conforming [as to object] with the impulsions. The same applies to the remaining doors. But in the case of the mind door-exalted consciousnesses also arise.

                    This is how the generation of the immaterial should be seen in the case of the six doors.

                    \marginnote{\textcolor{teal}{\footnotesize\{703|645\}}}{}One who sees the generation of the immaterial thus is said to “comprehend the immaterial at another time” (\hyperlink{XX.21}{§21}{}).

                    \vismParagraph{XX.45}{45}{}
                    This is how one [meditator] accomplishes the development of understanding, progressing gradually by comprehending at one time the material and at another time the immaterial, by attributing the three characteristics to them.
                \subsection[\vismAlignedParas{§45–75}The Material Septad]{The Material Septad}

                    Another comprehends formations by attributing the three characteristics to them through the medium of the material septad and the immaterial septad.

                    \vismParagraph{XX.46}{46}{}
                    Herein, one who comprehends [them] by attributing [the characteristics] in the following seven ways is said to comprehend by attributing through the medium of the material septad, that is to say, (1) as taking up and putting down, (2) as disappearance of what grows old in each stage, (3) as arising from nutriment, (4) as arising from temperature, (5) as kamma-born, (6) as consciousness-originated, and (7) as natural materiality. Hence the Ancients said:
                    \begin{verse}
                        “(1) As taking up and putting down,\\{}
                        (2) As growth and decline in every stage,\\{}
                        (3) As nutriment, (4) as temperature,\\{}
                        (5) As kamma, and (6) as consciousness,\\{}
                        (7) As natural materiality—\\{}
                        He sees with seven detailed insights.”
                    \end{verse}


                    \vismParagraph{XX.47}{47}{}
                    \emph{1.} Herein, \emph{taking up }is rebirth-linking. \emph{Putting down }is death. So the meditator allots one hundred years for this “taking up” and “putting down” and he attributes the three characteristics to formations. How? All formations between these limits are impermanent. Why? Because of the occurrence of rise and fall, because of change, because of temporariness, and because of preclusion of permanence. But since arisen formations have arrived at presence, and when present are afflicted by ageing, and on arriving at ageing are bound to dissolve, they are therefore painful because of continual oppression, because of being hard to bear, because of being the basis of suffering, and because of precluding pleasure. And since no one has any power over arisen formations in the three instances, “Let them not reach presence”, “Let those that have reached presence not age,” and “Let those that have reached ageing not dissolve,” and they are void of the possibility of any power being exercised over them, they are therefore not-self because void, because ownerless, because unsusceptible to the wielding of power, and because of precluding a self.\footnote{\vismAssertFootnoteCounter{20}\vismHypertarget{XX.n20}{}“No one, not even the Blessed One, has such mastery; for it is impossible for anyone to alter the three characteristics. The province of supernormal power is simply the alteration of a state” (\textbf{\cite{Vism-mhṭ}797}).

                            “‘\emph{Because of precluding a self}’ means because of precluding the self conceived by those outside the Dispensation; for the non-existence in dhammas of any self as conceived by outsiders is stated by the words, ‘\emph{because void}’; but by this expression [it is stated] that there is no self because there is no such individual essence” (\textbf{\cite{Vism-mhṭ}797}).} \textcolor{brown}{\textit{[619]}}

                    \vismParagraph{XX.48}{48}{}
                    \marginnote{\textcolor{teal}{\footnotesize\{704|646\}}}{}\emph{2. (a) }Having attributed the three characteristics to materiality allotted one hundred years for the “taking up” and “putting down” thus, he next attributes them according to \emph{disappearance of what grows old in each stage. }Herein, “disappearance of what grows old in each stage” is a name for the disappearance of the materiality that has grown old during a stage [of life]. The meaning is that he attributes the three characteristics by means of that.

                    \vismParagraph{XX.49}{49}{}
                    How? He divides that same hundred years up into three stages, that is, the first stage, the middle stage, and the last stage. Herein, the first thirty-three years are called the first stage, the next thirty-four years are called the middle stage, and the next thirty-three years are called the last stage. So after dividing it up into these three stages, [he attributes the three characteristics thus:] The materiality occurring in the first stage ceased there without reaching the middle stage: therefore it is impermanent; what is impermanent is painful; what is painful is not-self. Also the materiality occurring in the middle stage ceased there without reaching the last stage: therefore it is impermanent too and painful and not-self. Also there is no materiality occurring in the thirty-three years of the last stage that is capable of out-lasting death: therefore that is impermanent too and painful and not-self. This is how he attributes the three characteristics.

                    \vismParagraph{XX.50}{50}{}
                    \emph{2. (b)} Having attributed the three characteristics according to “disappearance of what grows old in each stage” thus by means of the first stage, etc., he again attributes the three characteristics according to “disappearance of what grows old in each stage” by means of the following ten decades: the tender decade, the sport decade, the beauty decade, the strength decade, the understanding decade, the decline decade, the stooping decade, the bent decade, the dotage decade, and the prone decade.

                    \vismParagraph{XX.51}{51}{}
                    Herein, as to these decades: in the first place, the first ten years of a person with a hundred years’ life are called the \emph{tender decade}; for then he is a tender unsteady child. The next ten years are called the \emph{sport decade}; for he is very fond of sport then. The next ten years are called the \emph{beauty decade}; for his beauty reaches its full extent then. The next ten years are called the \emph{strength decade}; for his strength and power reach their full extent then. The next ten years are called the \emph{understanding decade}; for his understanding is well established by then. Even in one naturally weak in understanding some understanding, it seems, arises at that time. The next ten years are called the \emph{decline decade}; for his fondness for sport and his beauty, strength, and understanding decline then. The next ten years are called the \emph{stooping decade}; for his figure \textcolor{brown}{\textit{[620]}} stoops forward then. The next ten years are called the \emph{bent decade}; for his figure becomes bent like the end of a plough then. The next ten years are called the \emph{dotage decade}; for he is doting then and forgets what he does. The next ten years are called the \emph{prone decade}; for a centenarian mostly lies prone.

                    \vismParagraph{XX.52}{52}{}
                    Herein, in order to attribute the three characteristics according to “disappearance of what grows old in each stage” by means of these decades, the meditator considers thus: The materiality occurring in the first decade ceases there without reaching the second decade: therefore it is impermanent, painful, not-self. The materiality occurring in the second decade … the materiality \marginnote{\textcolor{teal}{\footnotesize\{705|647\}}}{}occurring in the ninth decade ceases there without reaching the tenth decade; the materiality occurring in the tenth decade ceases there without reaching the next becoming: therefore it is impermanent, painful, not-self. That is how he attributes the three characteristics.

                    \vismParagraph{XX.53}{53}{}
                    \emph{2. (c) }Having attributed the three characteristics according to “disappearance of what grows old in each stage” thus by means of the decades, he again attributes the three characteristics according to “disappearance of what grows old in each stage” by taking that same hundred years in twenty parts of five years each.

                    \vismParagraph{XX.54}{54}{}
                    How? He considers thus: The materiality occurring in the first five years ceases there without reaching the second five years: therefore it is impermanent, painful, not-self. The materiality occurring in the second five years … in the third … in the nineteenth five years ceases there without reaching the twentieth five years. There is no materiality occurring in the twentieth five years that is capable of outlasting death; therefore that is impermanent too, painful, not-self.

                    \vismParagraph{XX.55}{55}{}
                    \emph{2. (d) }Having attributed the three characteristics according to “disappearance of what grows old in each stage” thus by means of the twenty parts, he again attributes the three characteristics according to “disappearance of what grows old in each stage” by taking twenty-five parts of four years each. (e) Next, by taking thirty-three parts of three years each, (f) by taking fifty parts of two years each, (g) by taking a hundred parts of one year each.

                    \emph{2. (h)} Next he attributes the three characteristics according to “disappearance of what grows old in each stage” by means of each of the three seasons, taking each year in three parts.

                    \vismParagraph{XX.56}{56}{}
                    How? The materiality occurring in the four months of the rains (\emph{vassāna}) ceases there without reaching the winter (\emph{hemanta}). The materiality occurring in the winter ceases there without reaching the summer (\emph{gimha}). The materiality occurring in the summer ceases there without reaching the rains again: therefore it is impermanent, \textcolor{brown}{\textit{[621]}} painful, not-self.

                    \vismParagraph{XX.57}{57}{}
                    \emph{2. (i)} Having attributed them thus, he again takes one year in six parts and attributes the three characteristics to this materiality according to “disappearance of what grows old in each stage” thus: The materiality occurring in the two months of the rains (\emph{vassāna}) ceases there without reaching the autumn (\emph{sarada}). The materiality occurring in the autumn … in the winter (\emph{hemanta}) … in the cool (\emph{sisira}) … in the spring (\emph{vasanta}) … the materiality occurring in the summer (\emph{gimha}) ceases there without reaching the rains again: therefore it is impermanent too, painful, not-self.

                    \vismParagraph{XX.58}{58}{}
                    \emph{2. (j) }Having attributed them thus, he next attributes the characteristics by means of the dark and bright halves of the moon thus: The materiality occurring in the dark half of the moon ceases there without reaching the bright half; the materiality occurring in the bright half ceases there without reaching the dark half: therefore it is impermanent, painful, not-self.

                    \vismParagraph{XX.59}{59}{}
                    \emph{2. (k)} Next he attributes the three characteristics by means of night and day thus: The materiality occurring in the night ceases there without reaching the \marginnote{\textcolor{teal}{\footnotesize\{706|648\}}}{}day; the materiality occurring in the day ceases there without reaching the night: therefore it is impermanent, painful, not-self.

                    \vismParagraph{XX.60}{60}{}
                    \emph{2. (l)} Next he attributes the three characteristics by taking that same day in six parts beginning with the morning thus: The materiality occurring in the morning ceased there without reaching the noon; the materiality occurring in the noon … without reaching the evening; the materiality occurring in the evening … the first watch; the materiality occurring in the first watch … the middle watch; the materiality occurring in the middle watch ceased there without reaching the last watch; the materiality occurring in the last watch ceased there without reaching the morning again: therefore it is impermanent, painful, not-self.

                    \vismParagraph{XX.61}{61}{}
                    \emph{2. (m) }Having attributed them thus, he again attributes the three characteristics to that same materiality by means of moving forward and moving backward, looking toward and looking away, bending and stretching, thus: The materiality occurring in the moving forward ceases there without reaching the moving backward; the materiality occurring in the moving backward … the looking toward; the materiality occurring in the looking toward … the looking away; the materiality occurring in the looking away … the bending; the materiality occurring in the bending ceases there without reaching the stretching: therefore it is impermanent, painful, not-self (cf. \textbf{\cite{M-a}I 260}).

                    \vismParagraph{XX.62}{62}{}
                    \emph{2. (n)} Next he divides a single footstep into six parts as “lifting up,” “shifting forward,” “shifting sideways,” “lowering down,” “placing down,” and “fixing down\footnote{\vismAssertFootnoteCounter{21}\vismHypertarget{XX.n21}{}\emph{Vītiharaṇa—}“shifting sideways,” \emph{sannikkhepana—}“placing down,” and \emph{sannirujjhana—}“fixing down,” are not in PED; cf. \textbf{\cite{M-a}I 260}.}.”

                    \vismParagraph{XX.63}{63}{}
                    Herein, \emph{lifting up }is raising the foot from the ground. \emph{Shifting forward }is shifting it to the front. \emph{Shifting sideways }is moving the foot to one side or the other in seeing a thorn, stump, snake, and so on. \emph{Lowering down} is letting the foot down. \textcolor{brown}{\textit{[622]}} \emph{Placing down }is putting the foot on the ground. \emph{Fixing down }is pressing the foot on the ground while the other foot is being lifted up.

                    \vismParagraph{XX.64}{64}{}
                    Herein, in the \emph{lifting up }two elements, the earth element and the water element, are subordinate\footnote{\vismAssertFootnoteCounter{22}\vismHypertarget{XX.n22}{}\emph{Omatta—}“subordinate”: not in PED.} and sluggish while the other two are predominant and strong. Likewise in the \emph{shifting forward }and \emph{shifting sideways. }In the \emph{lowering down }two elements, the fire element and the air element, are subordinate and sluggish while the other two are predominant and strong. Likewise in the \emph{placing down }and \emph{fixing down.}

                    He attributes the three characteristics to materiality according to “disappearance of what grows old in each stage” by means of these six parts into which he has thus divided it.

                    \vismParagraph{XX.65}{65}{}
                    How? He considers thus: The elements and the kinds of derived materiality occurring in the lifting up all ceased there without reaching the shifting forward: therefore they are impermanent, painful, not-self. Likewise those occurring in \marginnote{\textcolor{teal}{\footnotesize\{707|649\}}}{}the shifting forward … the shifting sideways; those occurring in the shifting sideways … the lowering down; those occurring in the lowering down … the placing down; those occurring in the placing down cease there without reaching the fixing down; thus formations keep breaking up, like crackling sesame seeds put into a hot pan; wherever they arise, there they cease stage by stage, section by section, term by term, each without reaching the next part: therefore they are impermanent, painful, not-self.

                    \vismParagraph{XX.66}{66}{}
                    When he sees formations stage by stage with insight thus, his comprehension of materiality has become subtle. Here is a simile for its subtlety. A border dweller, it seems, who was familiar with torches of wood and grass, etc., but had never seen a lamp before, came to a city. Seeing a lamp burning in the market, he asked a man, “I say, what is that lovely thing called?”—“What is lovely about that? It is called a lamp. Where it goes to when its oil and wick are used up no one knows.” Another told him, “That is crudely put; for the flame in each third portion of the wick as it gradually burns up ceases there without reaching the other parts.” Other told him, “That is crudely put too; for the flame in each inch, in each half-inch, in each thread, in each strand, will cease without reaching the other strands; but the flame cannot appear without a strand.”

                    \vismParagraph{XX.67}{67}{}
                    \textcolor{brown}{\textit{[623]}} Herein, the meditator’s attribution of the three characteristics to materiality delimited by the hundred years as “taking up” and “putting down” is like the man’s knowledge stated thus, “Where it goes when its oil and wick are used up no one knows.” The meditator’s attribution of the three characteristics according to “disappearance of what grows old in each stage” to the materiality delimited by the third part of the hundred years is like the man’s knowledge stated thus, “The flame in each third portion of the wick ceases without reaching the other parts.” The meditator’s attribution of the three characteristics to materiality delimited by the periods of ten, five, four, three, two years, one year, is like the man’s knowledge stated thus, “The flame in each inch will cease without reaching the others.” The meditator’s attribution of the three characteristics to materiality delimited by the four-month and two-month periods by classing the year as threefold and sixfold respectively according to the seasons is like the man’s knowledge stated thus, “The flame in each half-inch will cease without reaching the others.” The meditator’s attribution of the three characteristics to materiality delimited by means of the dark and bright halves of the moon, by means of night and day, and by means of morning, etc., taking one night and day in six parts, is like the man’s knowledge stated thus, “The flame in each thread will cease without reaching the others.” The meditator’s attribution of the three characteristics to materiality delimited by means of each part, namely, “moving forward,” etc., and “lifting up,” etc., is like the man’s knowledge stated thus, “The flame in each strand will cease without reaching the others.”

                    \vismParagraph{XX.68}{68}{}
                    \emph{3–6. }Having in various ways thus attributed the three characteristics to materiality according to “disappearance of what grows old in each stage,” he analyzes that same materiality and divides it into four portions as “arising from nutriment,” etc., and he again attributes the three characteristics to each portion. \marginnote{\textcolor{teal}{\footnotesize\{708|650\}}}{}\emph{3.} Herein, materiality \emph{arising from nutriment }becomes evident to him through hunger and its satisfaction. For materiality that is originated when one is hungry is parched and stale, and it is as ugly and disfigured as a parched stump, as a crow perching in a charcoal pit. That originated when hunger is satisfied is plump, fresh, tender, smooth and soft to touch. Discerning that, he attributes the three characteristics to it thus: The materiality occurring when hunger is satisfied ceases there without reaching the time when one is hungry; therefore it is impermanent, painful, not-self.

                    \vismParagraph{XX.69}{69}{}
                    \emph{4.} That \emph{arising from temperature }becomes evident through cool and heat. For materiality that is originated when it is hot is parched, stale and ugly. \textcolor{brown}{\textit{[624]}} Materiality originated by cool temperature is plump, fresh, tender, smooth, and soft to touch. Discerning that, he attributes the three characteristics to it thus: The materiality occurring when it is hot ceases there without reaching the time when it is cool. The materiality occurring when it is cool ceases there without reaching the time when it is hot: therefore it is impermanent, painful, not-self.

                    \vismParagraph{XX.70}{70}{}
                    \emph{5. The kamma-born }becomes evident through the sense doors, that is, the base [of consciousness]. For in the case of the eye door there are thirty material instances with decads of the eye, the body, and sex; but with the twenty-four instances originated by temperature, consciousness, and nutriment, [that is to say, three bare octads,] which are their support, there are fifty-four. Likewise in the case of the doors of the ear, nose, and tongue. In the case of the body door there are forty-four with the decads of body and sex and the instances originated by temperature, and so on. In the case of the mind door there are fifty-four, too, with the decads of the heart-basis, the body, and sex, and those instances originated by the temperature, and so on. Discerning all that materiality, he attributes the three characteristics to it thus: The materiality occurring in the eye door ceases there without reaching the ear door; the materiality occurring in the ear door … the nose door; the materiality occurring in the nose door … the tongue door; the materiality occurring in the tongue door … the body door; the materiality occurring in the body door ceases there without reaching the mind door: therefore it is impermanent, painful, not-self.

                    \vismParagraph{XX.71}{71}{}
                    \emph{6.} The consciousness-originated becomes evident through [the behaviour of] one who is joyful or grieved. For the materiality arisen at the time when he is joyful is smooth, tender, fresh and soft to touch. That arisen at the time when he is grieved is parched, stale and ugly. Discerning that, he attributes the three characteristics to it thus: The materiality occurring at the time when one is joyful ceases there without reaching the time when one is grieved; the materiality occurring at the time when one is grieved ceases there without reaching the time when one is joyful: therefore it is impermanent, painful, not-self.

                    \vismParagraph{XX.72}{72}{}
                    When he discerns consciousness-originated materiality and attributes the three characteristics to it in this way, this meaning becomes evident to him:
                    \begin{verse}
                        Life, person, pleasure, pain just these alone\\{}
                        Join in one conscious moment that flicks by.\\{}
                        Gods, though they live for four-and-eighty thousand\\{}
                        Eons, are not the same for two such moments. \textcolor{brown}{\textit{[625]}}
                    \end{verse}

                    \begin{verse}
                        \marginnote{\textcolor{teal}{\footnotesize\{709|651\}}}{}Ceased aggregates of those dead or alive\\{}
                        Are all alike, gone never to return;\\{}
                        And those that break up meanwhile, and in future,\\{}
                        Have traits no different from those ceased before.
                    \end{verse}

                    \begin{verse}
                        No [world is] born if [consciousness is] not\\{}
                        Produced; when that is present, then it lives;\\{}
                        When consciousness dissolves, the world is dead:\\{}
                        The highest sense this concept will allow.
                    \end{verse}

                    \begin{verse}
                        No store of broken states, no future stock;\\{}
                        Those born balance like seeds on needle points.\\{}
                        Breakup of states is foredoomed at their birth;\\{}
                        Those present decay, unmingled with those past.\\{}
                        They come from nowhere, break up, nowhere go;\\{}
                        Flash in and out, as lightning in the sky\footnote{\vismAssertFootnoteCounter{23}\vismHypertarget{XX.n23}{}This verse is quoted twice in the Mahāniddesa (\textbf{\cite{Nidd}I 42} \& 118). For \textbf{\cite{Vism-mhṭ}}’s comment see \hyperlink{VIII}{Ch. VIII}{}, note 11. \textbf{\cite{Vism-mhṭ}} and the Sinhalese translation have been taken as guides in rendering this rather difficult verse. There is another stanza in the Niddesa not quoted here:

                                “… this concept will allow. States happen as their tendencies dictate; And they are modelled by desire; their stream Uninterruptedly flows ever on Conditioned by the sixfold base of contact. No store of broken states …”}(\textbf{\cite{Nidd}I 42}).
                    \end{verse}


                    \vismParagraph{XX.73}{73}{}
                    \emph{7. }Having attributed the three characteristics to that arising from nutriment, etc., he again attributes the three characteristics to natural materiality. \emph{Natural materiality }is a name for external materiality that is not bound up with faculties and arises along with the eon of world expansion, for example, iron, copper, tin, lead, gold, silver, pearl, gem, beryl, conch shell, marble, coral, ruby, opal, soil, stone, rock, grass, tree, creeper, and so on (see \textbf{\cite{Vibh}83}). That becomes evident to him by means of an asoka-tree shoot.

                    \vismParagraph{XX.74}{74}{}
                    For that to begin with is pale pink; then in two or three days it becomes dense red, again in two or three days it becomes dull red, next [brown,] the colour of a tender [mango] shoot; next, the colour of a growing shoot; next, the colour of pale leaves; next, the colour of dark green leaves. After it has become the colour of dark green leaves, as it follows out the successive stages of such material continuity, it eventually becomes withered foliage, and at the end of the year it breaks loose from its stem and falls off.

                    \vismParagraph{XX.75}{75}{}
                    Discerning that, he attributes the three characteristics to it thus: The materiality occurring when it is pale pink ceases there without reaching the time when it is dense red; the materiality occurring when it is dense red … dull red; the materiality occurring when it is dull red … the colour of a tender [mango] shoot; the materiality occurring when it is the colour of a tender [mango] shoot … the colour of a growing shoot; the materiality occurring when it is the colour of a growing shoot … the colour of pale green leaves; the materiality occurring \marginnote{\textcolor{teal}{\footnotesize\{710|652\}}}{}when it is the colour of pale green leaves … the colour of dark green leaves; the materiality occurring when it is the colour of dark green leaves … the time when it is withered foliage; the materiality occurring when it is withered foliage ceases there without \textcolor{brown}{\textit{[626]}} reaching the time when it breaks loose from its stem and falls off: therefore it is impermanent, painful, not-self.

                    He comprehends all natural materiality in this way.

                    This is how, firstly, he comprehends formations by attributing the three characteristics to them by means of the material septad.
                \subsection[\vismAlignedParas{§76–88}The Immaterial Septad]{The Immaterial Septad}

                    \vismParagraph{XX.76}{76}{}
                    The headings of what was called above “the immaterial septad” are these: (1) by groups, (2) by pairs, (3) by moments, (4) by series, (5) by removal of [false] view, (6) by abolition of conceit, (7) by ending of attachment.

                    \vismParagraph{XX.77}{77}{}
                    \emph{1. }Herein, \emph{by groups }means the states belonging to the contact pentad.\footnote{\vismAssertFootnoteCounter{24}\vismHypertarget{XX.n24}{}The “contact pentad” (\emph{phassa-pañcamaka}) is a term used for the first five things listed in \textbf{\cite{Dhs}§1}, that is, contact, feeling, perception, volition, and consciousness, which are invariably present whenever there is consciousness.} How? Here, “he comprehends by groups” [means that] a bhikkhu considers thus: The states belonging to the contact pentad arising in the comprehending of head hairs as “impermanent, painful, not-self”; the states belonging to the contact pentad arising in the comprehending of body hairs as … in the contemplation of brain as “impermanent, painful, not-self”—all these states disintegrate section by section, term by term, like crackling sesame seeds put into a hot pan, each without reaching the next: therefore they are impermanent, painful, not-self. This, firstly, is the method according to the Discourse on Purification.\footnote{\vismAssertFootnoteCounter{25}\vismHypertarget{XX.n25}{}The “Discourse on Purification” (\emph{visuddhi-kathā}) and the “Discourse on the Noble Ones’ Heritages” (\emph{ariyavaṃsa-kathā}) are presumably names of chapters in the old Sinhalese commentaries no longer extant.}

                    \vismParagraph{XX.78}{78}{}
                    According to the Discourse on the Noble Ones’ Heritages, however, he is said to “comprehend by groups” when by means of a subsequent consciousness he comprehends as “impermanent, painful, not-self” that consciousness which occurred [comprehending] materiality as “impermanent, painful, not-self” in the seven instances of the material septad given above. As this method is more suitable we shall therefore confine ourselves to it in explaining the rest.

                    \vismParagraph{XX.79}{79}{}
                    \emph{2. By pairs: }after the bhikkhu has comprehended as “impermanent, painful, not-self” the materiality of the “taking up and putting down” (\hyperlink{XX.46}{§46f.}{}), he comprehends that consciousness [with which he was comprehending the materiality] too as “impermanent, painful, not-self” by means of a subsequent consciousness. After he has comprehended as “impermanent, painful, not-self” the materiality of the “disappearance of what grows old in each stage” and that “arising from nutriment,” “arising from temperature,” “kamma-born,” “consciousness-originated” and “natural,” he comprehends that consciousness too as “impermanent, painful, not-self” by means of a subsequent consciousness. In this way he is said to comprehend by pairs.

                    \vismParagraph{XX.80}{80}{}
                    \marginnote{\textcolor{teal}{\footnotesize\{711|653\}}}{}\emph{3. By moments}: after the bhikkhu has comprehended as “impermanent, painful, not-self” the materiality of the “taking up and putting down,” he comprehends that first consciousness [with which he was comprehending the materiality] as “impermanent, painful, not-self” by means of a second consciousness, and that second consciousness by means of a third, and the third by means of a fourth, and the fourth by means of a fifth, and that too he comprehends as “impermanent, painful, not-self.” After he has comprehended as “impermanent, painful, not-self” the materiality of “disappearance of what grows old in each stage” and that “arising from nutriment,” “arising from temperature,” \textcolor{brown}{\textit{[627]}} “kamma-born,” “consciousness-originated” and “natural,” he comprehends that first consciousness as “impermanent, painful, not-self” by means of a second consciousness, and that second consciousness by means of a third, and the third by means of a fourth, and the fourth by means of a fifth, and that too he comprehends as “impermanent, painful, not-self.” Comprehending thus four [consciousnesses] from each discerning of materiality he is said to comprehend by moments.

                    \vismParagraph{XX.81}{81}{}
                    \emph{4. By series}: after he has comprehended as “impermanent, painful, not-self” the materiality of the “taking up and putting down,” he comprehends that first consciousness as “impermanent, painful, not-self” by means of a second consciousness, and the second by means of a third, and the third by means of a fourth … and the tenth by means of an eleventh, and that too he comprehends as “impermanent, painful, not-self.” After he has comprehended as “impermanent, painful, not-self” the materiality of the “disappearance of what grows old in each stage” and that “arising from nutriment,” “arising from temperature,” “kamma-born,” “consciousness-originated” and “natural,” he comprehends that consciousness as “impermanent, painful, not-self” by means of a second consciousness, and the second by means of a third, … and the tenth by means of an eleventh, and that too he comprehends as “impermanent, painful, not-self.” It would be possible to go on comprehending it in this way with serial insight even for a whole day. But both the material meditation subject and the immaterial meditation subject become familiar when the comprehending is taken as far as the tenth consciousness. That is why it is said\footnote{\vismAssertFootnoteCounter{26}\vismHypertarget{XX.n26}{}“Said in the Discourse on the Noble Ones’ Heritages” (\textbf{\cite{Vism-mhṭ}804}).} that it can be stopped at the tenth. It is when he comprehends in this way that he is said to comprehend by series.

                    \vismParagraph{XX.82}{82}{}
                    \emph{5. By removal of [false] view, }6\emph{. by abolition of conceit, }7.\emph{ by ending of attachment}: there is no individual method for any of these three. But when he has discerned this materiality as described above and this immateriality as described here, then he sees that there is no living being over and above the material and the immaterial. As soon as he no longer sees a being, the perception of a being is removed. When he discerns formations with consciousness from which perception of a being has been removed, then [false] view does not arise in him. When [false] view does not arise in him, then [false] view is said to be removed.

                    When he discerns formations with consciousness from which [false] view has been removed, then conceit does not arise in him. When conceit does not \marginnote{\textcolor{teal}{\footnotesize\{712|654\}}}{}arise, conceit is said to be abolished. When he discerns formations with consciousness from which conceit has been abolished, then craving does not arise in him. When craving does not arise in him, attachment is said to be ended. This firstly is what is said in the Discourse on Purification.

                    \vismParagraph{XX.83}{83}{}
                    But in the Discourse on the Noble Ones’ Heritages, after setting forth the headings thus: “As removal of [false] view, as abolition of conceit, as ending of attachment,” the following method is set forth: “There is no removal of [false] view in one who takes it thus, ‘I see with insight, my insight’; \textcolor{brown}{\textit{[628]}} there is removal of [false] view in one who takes it thus, ‘Only formations see formations with insight, comprehend, define, discern, and delimit them.’ There is no abolition of conceit in one who takes it thus, ‘I see thoroughly with insight, I see well with insight’; there is abolition of conceit in one who takes it thus, ‘Only formations see formations with insight, comprehend, define, discern, and delimit them.’ There is no ending of attachment in one who is pleased with insight thus, ‘I am able to see with insight’; there is ending of attachment in one who takes it thus, ‘Only formations see formations with insight, comprehend, define, discern, and delimit them.’

                    \vismParagraph{XX.84}{84}{}
                    “There is removal of [false] view in one who sees thus: ‘If formations were self, it would be right to take them as self; but being not-self they are taken as self. Therefore they are not-self in the sense of no power being exercisable over them; they are impermanent in the sense of non-existence after having come to be; they are painful in the sense of oppression by rise and fall.’

                    \vismParagraph{XX.85}{85}{}
                    “There is abolition of conceit in one who sees thus: ‘If formations were permanent, it would be right to take them as permanent; but being impermanent they are taken as permanent. Therefore they are impermanent in the sense of non-existence after having come to be; they are painful in the sense of oppression by rise and fall; they are not-self in the sense of no power being exercisable over them.’

                    \vismParagraph{XX.86}{86}{}
                    “There is ending of attachment in one who sees thus: ‘If formations were pleasant, it would be right to take them as pleasant; but being painful they are taken as pleasant. Therefore they are painful in the sense of oppression by rise and fall; they are impermanent in the sense of non-existence after having come to be; they are not-self in the sense of no power being exercisable over them.’

                    \vismParagraph{XX.87}{87}{}
                    “Thus there comes to be the removal of [false] view in one who sees formations as not-self; there comes to be the abolishing of conceit in one who sees them as impermanent; there comes to be the ending of attachment in one who sees them as painful. So this insight is valid in each instance.”

                    \vismParagraph{XX.88}{88}{}
                    This is how he comprehends formations by attributing the three characteristics to them by means of the immaterial septad.

                    At this stage both the material meditation subject and the immaterial meditation subject have become familiar to him.
                \subsection[\vismAlignedParas{§89–92}The Eighteen Principal Insights]{The Eighteen Principal Insights}

                    \vismParagraph{XX.89}{89}{}
                    Having thus become familiar with the material and immaterial meditation subjects, and so having penetrated here already a part of those eighteen principal \marginnote{\textcolor{teal}{\footnotesize\{713|655\}}}{}insights\footnote{\vismAssertFootnoteCounter{27}\vismHypertarget{XX.n27}{}The first seven of the eighteen principal insights are known as the “seven contemplations”; see 20.4. Further descriptions are given in \hyperlink{XXII.113}{XXII.113f.}{}} which are later on to be attained in all their aspects by means of full-understanding as abandoning starting with contemplation of dissolution, he consequently abandons things opposed [to what he has already penetrated].

                    \vismParagraph{XX.90}{90}{}
                    \emph{Eighteen principal insights }is a term for understanding that consists in the kinds of insight beginning with contemplation of impermanence. Now, as regards these: (1) One who develops the contemplation of impermanence abandons the perception of permanence, (2) one who develops the contemplation of pain \textcolor{brown}{\textit{[629]}} abandons the perception of pleasure, (3) one who develops the contemplation of not-self abandons the perception of self, (4) one who develops the contemplation of dispassion abandons delighting, (5) one who develops the contemplation of fading away abandons greed, (6) one who develops the contemplation of cessation abandons origination, (7) one who develops the contemplation of relinquishment abandons grasping, (8) one who develops the contemplation of destruction abandons the perception of compactness, (9) one who develops the contemplation of fall [of formations] abandons accumulation [of kamma], (10) one who develops the contemplation of change abandons the perception of lastingness, (11) one who develops the contemplation of the signless abandons the sign, (12) one who develops the contemplation of the desireless abandons desire, (13) one who develops the contemplation of voidness abandons misinterpreting (insistence), (14) one who develops the insight into states that is higher understanding abandons misinterpreting (insistence) due to grasping at a core, (15) one who develops correct knowledge and vision abandons misinterpreting (insistence) due to confusion, (16) one who develops the contemplation of danger abandons misinterpreting (insistence) due to reliance, (17) one who develops the contemplation of reflection abandons non-reflection, (18) one who develops the contemplation of turning away abandons misinterpreting (insistence) due to bondage (see \textbf{\cite{Paṭis}I 32f.}).\footnote{\vismAssertFootnoteCounter{28}\vismHypertarget{XX.n28}{}For \textbf{\cite{Vism-mhṭ}}’s comments on the first seven see note 3 to this chapter.

                            ‘\emph{Contemplation of destruction}’ is the contemplation of the momentary dissolution of formations. ‘\emph{Perception of compactness}’ is the assumption of unity in a continuity or mass or function or object. ‘\emph{Contemplation of destruction}’ is contemplation of non-existence after having been, they say. Contemplation of destruction is the understanding by means of which he resolves the compact into its elements and sees that it is impermanent in the sense of destruction. Its completion starts with contemplation of dissolution, and so there is abandoning of perception of compactness then, but before that there is not, because it has not been completed. (9) The seeing of the dissolution of formations both by actual experience and by inference and the directing of attention to their cessation, in other words, their dissolution, is \emph{contemplation of fall}; through it accumulation [of kamma] is abandoned; his consciousness does not incline with craving to the occurrence of that [aggregate-process of existence] for the purpose of which one accumulates [kamma]. (10) Seeing change in the two ways through aging and through death in what is born, or seeing another essence subsequent to the delimitation of such and such [an essence supervening] in what was discerned by means of the material septad, and so on, is ‘\emph{contemplation of change}’; by its means he abandons the ‘\emph{perception of lastingness},’ the assumption of stability. (11)–(13) The three beginning with ‘\emph{contemplation of the signless}’ are the same as the three beginning with contemplation of impermanence. (11) ‘\emph{The sign}’ is the mere appearance of formations as if graspable entities, which is due to the individualization of particular functions and which, owing to perception of unity in continuity and in mass, is assumed to be temporarily enduring or permanent. (12) ‘\emph{Desire}’ is longing for pleasure, or it is desire consisting in greed, and so on; it means inclinationto formations owing to craving. (13) ‘\emph{Misinterpreting}’ is misinterpreting as self. It is owing to their opposing the ‘\emph{sign},’ etc., that the contemplations of impermanence, etc., are called by the names of ‘signless,’ etc.; so they should be regarded as opposed to the apprehension of a sign, etc., just as they are to the perception of permanence, and so on. (14) Insight that occurs by knowing an object consisting of a visible datum, etc., and by seeing the dissolution of the consciousness that had that visible datum, etc., as its object, and by apprehending voidness through the dissolution thus, ‘Only formations dissolve, there is nothing beyond the death of formations,’ is the higher understanding, and that is insight into states, thus it is ‘\emph{insight into states that is higher understanding}’; by its means he abandons the view accompanied by craving that is the misinterpretation occurring as grasping at a permanent core, and so on. (15) ‘\emph{Correct knowledge and vision}’ is a term for the seeing of mentality-materiality with its conditions; by its means he abandons the ‘\emph{misinterpreting due to confusion}’ that begins thus, ‘Was I in the past?’ (\textbf{\cite{M}I 8}) and that begins thus, ‘Thus the world is created by an Overlord’ (?). (16) The knowledge consisting in the seeing of danger in all kinds of becomings, etc., which has arisen owing to the appearance of terror is ‘\emph{contemplation of danger}’; by its means he abandons the craving occurring as ‘\emph{misinterpreting due to reliance}’ because he does not see any reliance or support. (17) The knowledge of reflection that is the means to deliverance from formations is ‘\emph{contemplation of reflection}’; by its means he abandons the ignorance that is ‘\emph{non-reflection}’ on impermanence, etc., and is opposed to reflection on them. (18) Equanimity about formations and conformity knowledge are ‘\emph{contemplation of turning away}’; for owing to it the mind retreats and recoils from all formations, like a water drop on a lotus leaf, so by its means he abandons the ‘\emph{misinterpretation due to bondage},’ which is the occurrence of the defilements consisting of the fetters of sense desire, and so on. (Vism-mh 806f.)}

                    \vismParagraph{XX.91}{91}{}
                    \marginnote{\textcolor{teal}{\footnotesize\{714|656\}}}{}Now the meditator has seen formations by means of the three characteristics beginning with impermanence, and so he has therefore already penetrated among these eighteen insights the contemplations of impermanence, pain, and not-self. And then (1) the contemplation of impermanence and (11) the contemplation of the signless are one in meaning and different only in the letter, and so are (2) the contemplation of pain and (12) the contemplation of the desireless, and so are (3) the contemplation of not-self and (13) the contemplation of voidness (see \textbf{\cite{Paṭis}II 63}). Consequently these have been penetrated by him as well. But (14) insight into states that is higher understanding is all kinds of insight, and (15) correct knowledge and vision is included in purification by overcoming doubt (\hyperlink{XIX}{Ch. XIX}{}). Consequently, these two have been penetrated by him as well. As to the remaining kinds of insight, some have been penetrated and some not. We shall deal with them below.\footnote{\vismAssertFootnoteCounter{29}\vismHypertarget{XX.n29}{}See \hyperlink{XXII.113}{XXII.113f.}{} “When (1) the contemplation of impermanence is established, then the contemplations of (6) cessation, (8) destruction, (9) fall, and (10) change are partly established. When (2) the contemplation of pain is established, then the contemplations of (4) dispassion and (16) danger are partly established. And when (3) the contemplation of not-self is established, then the rest are partly established” (\textbf{\cite{Vism-mhṭ}807}).}

                    \vismParagraph{XX.92}{92}{}
                    \marginnote{\textcolor{teal}{\footnotesize\{715|657\}}}{}For it was with reference only to what has already been penetrated that it was said above: “having thus become familiar with the material and immaterial meditation subjects, and so having penetrated here already a part of those eighteen principal insights, which are later on to be attained in all their aspects by means of full understanding as abandoning starting with contemplation of dissolution, he consequently abandons things opposed [to what he has already penetrated]” (\hyperlink{XX.89}{§89}{}).
                \subsection[\vismAlignedParas{§93–130}Knowledge of Rise and Fall—I]{Knowledge of Rise and Fall—I}

                    \vismParagraph{XX.93}{93}{}
                    Having purified his knowledge in this way by abandoning the perceptions of permanence, etc., which oppose the contemplations of impermanence, etc., he passes on from comprehension knowledge and begins the task of attaining that of contemplation of rise and fall, which is expressed thus: “Understanding \textcolor{brown}{\textit{[630]}}\textbf{ }of contemplating present states’ change is knowledge of contemplation of rise and fall” (\textbf{\cite{Paṭis}I 1}), and which comes next after comprehension knowledge.

                    \vismParagraph{XX.94}{94}{}
                    When he does so, he does it first in brief. Here is the text: “How is it that understanding of contemplating present states’ change is knowledge of contemplation of rise and fall? Present materiality is born [materiality]; the characteristic of its generation is rise, the characteristic of its change is fall, the contemplation is knowledge. Present feeling … perception … formations … consciousness … eye … (etc.) … Present becoming is born [becoming]; the characteristic of its generation is rise, the characteristic of its change is fall, the contemplation is knowledge” (\textbf{\cite{Paṭis}I 54}).\footnote{\vismAssertFootnoteCounter{30}\vismHypertarget{XX.n30}{}“The interpreting of rise and fall must be done on a state that is present according to continuity or present according to instant but not on one that is past or future, which is why ‘of present states’ is said” (\textbf{\cite{Vism-mhṭ}808}). “Present materiality is called born materiality; it is included in the trio of instants [of arising, presence and dissolution], is what is meant. But that is hard to discern at the start, so the interpreting by insight should be done by means of presence according to continuity” (\textbf{\cite{Vism-mhṭ}808}). For the elision represented by “… (etc.) …” see \hyperlink{XX.9}{XX.9}{}. In this case, however, the last two members of the dependent origination are left out. “Although states possessed of aging-and-death are mentioned under the heading of birth and of aging-and-death in comprehension by groups, etc., nevertheless here in the description of knowledge of rise and fall, if it were said ‘present birth is born; the characteristic of its generation is rise, the characteristic of its change is fall,’ etc., it would be tantamount to an affirmation and approval of the proposition that birth and aging-and-death were possessed of birth and of aging-and-death. So the text ends with ‘becoming’ in order to avoid that” (\textbf{\cite{Vism-mhṭ}808}).}

                    \vismParagraph{XX.95}{95}{}
                    In accordance with the method of this text he sees the characteristic of generation, the birth, the arising, the aspect of renewal, of born materiality, as “rise,” and he sees its characteristic of change, its destruction, its dissolution, as “fall.”

                    \vismParagraph{XX.96}{96}{}
                    \marginnote{\textcolor{teal}{\footnotesize\{716|658\}}}{}He understands thus: “There is no heap or store of unarisen mentality-materiality [existing] prior to its arising. When it arises, it does not come from any heap or store; and when it ceases, it does not go in any direction. There is nowhere any depository in the way of a heap or store or hoard of what has ceased. But just as there is no store, prior to its arising, of the sound that arises when a lute is played, nor does it come from any store when it arises, nor does it go in any direction when it ceases, nor does it persist as a store when it has ceased (cf. \textbf{\cite{S}IV 197}), but on the contrary, not having been, it is brought into being owing to the lute, the lute’s neck, and the man’s appropriate effort, and having been, it vanishes—so too all material and immaterial states, not having been, are brought into being, and having been, they vanish.”

                    \vismParagraph{XX.97}{97}{}
                    Having given attention to rise and fall in brief thus, he again [does so in detail according to condition and instant by seeing those characteristics] as given in the exposition of that same knowledge of rise and fall thus: “(1) He sees the rise of the materiality aggregate in the sense of conditioned arising thus: With the arising of ignorance there is the arising of materiality; (2) … with the arising of craving … (3) … with the arising of kamma … (4) he sees the rise of the materiality aggregate in the sense of conditioned arising thus: With the arising of nutriment there is the arising of materiality; (5) one who sees the characteristic of generation sees the rise of the materiality aggregate. One who sees the rise of the materiality aggregate sees these five characteristics.

                    “(1) He sees the fall of the materiality aggregate in the sense of conditioned cessation thus: With the cessation of ignorance there is the cessation of materiality; (2) … with the cessation of craving … (3) … with the cessation of kamma … (4) he sees the fall of the materiality aggregate in the sense of conditioned cessation thus: With the cessation of nutriment there is the cessation of materiality; \textcolor{brown}{\textit{[631]}} (5) one who sees the characteristic of change sees the fall of the materiality aggregate. One who sees the fall of the materiality aggregate sees these five characteristics” (\textbf{\cite{Paṭis}I 55f.}).

                    Likewise: “(1) He sees the rise of the feeling aggregate in the sense of conditioned arising thus: With the arising of ignorance there is the arising of feeling; (2) … with the arising of craving … (3) … with the arising of kamma … (4) he sees the rise of the feeling aggregate in the sense of conditioned arising thus: With the arising of contact there is the arising of feeling; (5) one who sees the characteristic of generation sees the rise of the feeling aggregate. One who sees the rise of the feeling aggregate sees those five characteristics.

                    “(1) He sees the fall of the feeling aggregate in the sense of conditioned cessation thus: With the cessation of ignorance there is the cessation of feeling; (2) … with the cessation of craving … (3) … with the cessation of kamma … (4) he sees the fall of the feeling aggregate in the sense of conditioned cessation thus: With the cessation of contact there is the cessation of feeling; (5) one who sees the characteristic of change sees the fall of the feeling aggregate. One who sees the fall of the feeling aggregate sees these five characteristics” (\textbf{\cite{Paṭis}I 55f.}).

                    And as in the case of the feeling aggregate, [that is, substituting “contact” for the “nutriment” in the case of materiality,] so for the perception and formations \marginnote{\textcolor{teal}{\footnotesize\{717|659\}}}{}aggregates. So also for the consciousness aggregate with this difference, that for the phrases containing “contact” there are substituted “with the arising of mentality-materiality” and “with the cessation of mentality-materiality.”

                    So there are fifty characteristics stated with the ten in the case of each aggregate by seeing rise and fall, by means of which he gives attention in detail \emph{according to condition }and \emph{according to instant }(\emph{moment}) in this way: “The rise of materiality is thus; its fall is thus; so it rises, so it falls.”

                    \vismParagraph{XX.98}{98}{}
                    As he does so his knowledge becomes clearer thus: “So, it seems, these states, not having been, are brought into being; having been, they vanish.”

                    When he thus sees rise and fall in the two ways, according to condition and according to instant, the several truths, aspects of the dependent origination, methods, and characteristics become evident to him.

                    \vismParagraph{XX.99}{99}{}
                    When he sees the arising of aggregates with the arising of ignorance and the cessation of aggregates with the cessation of ignorance, this is his seeing of rise and fall \emph{according to condition. }When he sees the rise and fall of aggregates by seeing the characteristic of generation and the characteristic of change, this is his seeing of rise and fall \emph{according to instant. }For it is only at the instant of arising that there is the characteristic of generation, and only at the instant of dissolution that there is the characteristic of change.

                    \vismParagraph{XX.100}{100}{}
                    So when he sees rise and fall in the two ways, according to condition and according to instant thus, the \emph{truth }of origination becomes evident to him through seeing rise according to condition owing to his discovery of the progenitor. \textcolor{brown}{\textit{[632]}} The truth of suffering becomes evident to him through seeing rise according to instant owing to his discovery of the suffering due to birth. The truth of cessation becomes evident to him through seeing fall according to condition owing to his discovery of the non-arising of things produced by conditions when their conditions do not arise. The truth of suffering becomes evident to him too through seeing fall according to instant owing to his discovery of the suffering due to death. And his seeing of rise and fall becomes evident to him as the truth of the path thus: “This is the mundane path” owing to abolition of confusion about it.

                    \vismParagraph{XX.101}{101}{}
                    The \emph{dependent origination }in forward order becomes evident to him through seeing rise according to condition owing to his discovery that “When this exists, that comes to be” (\textbf{\cite{M}I 262}). The dependent origination in reverse order becomes evident to him through seeing fall according to condition owing to his discovery that “When this does not exist, that does not come to be” (\textbf{\cite{M}I 264}). Dependently-arisen states become evident to him through seeing rise and fall according to instant owing to his discovery of the characteristic of the formed; for the things possessed of rise and fall are formed and conditionally arisen.

                    \vismParagraph{XX.102}{102}{}
                    The \emph{method }of identity becomes evident to him through seeing rise according to condition owing to his discovery of unbroken continuity in the connection of cause with fruit. Then he more thoroughly abandons the annihilation view. The method of diversity becomes evident to him through seeing rise according to instant owing to his discovery that each [state] is new [as it arises]. Then he more thoroughly abandons the eternity view. The method of uninterestedness becomes evident to \marginnote{\textcolor{teal}{\footnotesize\{718|660\}}}{}him through seeing rise and fall according to condition owing to his discovery of the inability of states to have mastery exercised over them. Then he more thoroughly abandons the self view. The method of ineluctable regularity becomes evident to him through seeing rise according to condition owing to his discovery of the arising of the fruit when the suitable conditions are there. Then he more thoroughly abandons the moral-inefficacy-of-action view.

                    \vismParagraph{XX.103}{103}{}
                    The \emph{characteristic }of not-self becomes evident to him through seeing rise according to condition owing to his discovery that states have no curiosity and that their existence depends upon conditions. The characteristic of impermanence becomes evident to him through seeing rise and fall according to instant owing to his discovery of non-existence after having been and owing to his discovery that they are secluded from past and future. The characteristic of pain becomes evident to him [through that] too owing to his discovery of oppression by rise and fall. And the characteristic of individual essence becomes evident to him [through that] too owing to his discovery of delimitation [of states] by rise and fall.\footnote{\vismAssertFootnoteCounter{31}\vismHypertarget{XX.n31}{}“With the seeing of rise and fall not only the characteristics of impermanence and pain become evident, but also the characteristics, in other words, the individual essences, of earth, contact, etc., termed hardness, touching, etc., respectively, become clearly evident and discrete (\emph{avacchinna}) in their individual essences” (\textbf{\cite{Vism-mhṭ}814}).} And in the characteristic of individual essence the temporariness of the characteristic of what is formed becomes evident to him [through that] too owing to his discovery of the non-existence of fall at the instant of rise and the non-existence of rise at the instant of fall.\footnote{\vismAssertFootnoteCounter{32}\vismHypertarget{XX.n32}{}“The inclusion of only rise and fall here is because this kind of knowledge occurs as seeing only rise and fall, not because of non-existence of the instant of presence” (\textbf{\cite{Vism-mhṭ}814}). See Introduction, note 18.}

                    \vismParagraph{XX.104}{104}{}
                    When the several truths, aspects of the dependent origination, methods, and characteristics have become evident to him thus, then formations appear to him as perpetually renewed: “So these states, it seems, being previously unarisen, critic, and being arisen, they cease.” \textcolor{brown}{\textit{[633]}}\textbf{ }And they are not only perpetually renewed, but they are also short-lived like dew-drops at sunrise (\textbf{\cite{A}IV 137}), like a bubble on water (\textbf{\cite{S}III 14} I), like a line drawn on water (\textbf{\cite{A}IV 137}), like a mustard seed on an awl’s point (\textbf{\cite{Nidd}I 42}), like a lightning flash (\textbf{\cite{Nidd}I 43}). And they appear without core, like a conjuring trick (\textbf{\cite{S}III 141}), like a mirage (Dhp 46), like a dream (Sn 807), like the circle of a whirling firebrand (source untraced), like a goblin city (source untraced), like froth (Dhp 46), like a plantain trunk (\textbf{\cite{S}III 142}), and so on.

                    At this point he has attained tender insight-knowledge called contemplation of rise and fall, which has become established by penetrating the fifty characteristics in this manner: “Only what is subject to fall arises; and to be arisen necessitates fall.” With the attainment of this he is known as a “beginner of insight.”
                    \subsubsection[\vismAlignedParas{§105–130}The Ten Imperfections of Insight]{The Ten Imperfections of Insight}

                        \vismParagraph{XX.105}{105}{}
                        Now, when he is a beginner of insight with this tender insight, ten imperfections of insight arise in him. For imperfections of insight do not arise either in a noble disciple who has reached penetration [of the truths] or in persons \marginnote{\textcolor{teal}{\footnotesize\{719|661\}}}{}erring in virtue, neglectful of their meditation subject and idlers. They arise only in a clansman who keeps to the right course, devotes himself continuously [to his meditation subject] and is a beginner of insight. But what are these ten imperfections? They are: (1) illumination, (2) knowledge, (3) rapturous happiness, (4) tranquillity, (5) bliss (pleasure), (6) resolution, (7) exertion, (8) assurance, (9) equanimity, and (10) attachment.

                        \vismParagraph{XX.106}{106}{}
                        For this is said: “How does the mind come to be seized by agitation about higher states? When a man is bringing [formations] to mind as impermanent, illumination arises in him. He adverts to the illumination thus, ‘Illumination is a [Noble One’s] state.’\footnote{\vismAssertFootnoteCounter{33}\vismHypertarget{XX.n33}{}“He adverts to it as Nibbāna or as the path or as fruition” (\textbf{\cite{Vism-mhṭ}816}). “The agitation, the distraction, that occurs about whether or not the illumination, etc., are noble states is ‘\emph{agitation about higher states’}” (Vism-mhṭ 815). In this connection Vism-mhṭ quotes the following text: “Friends, any bhikkhu or bhikkhunī who declares the attainment of Arahantship in my presence has always arrived there by four paths or by one of them. What four? Here, friends, a bhikkhu develops insight preceded by serenity. While he is developing insight preceded by serenity the path is born in him. He cultivates, develops, repeats that path. As he does so his fetters are abandoned and his inherent tendencies are brought to an end. Again, friends, a bhikkhu develops serenity preceded by insight … He develops serenity and insight yoked equally. Again, friends, a bhikkhu’s mind is seized by agitation about highest states. When that consciousness settles down internally, becomes steady, unified and concentrated, then the path is born in him … his inherent tendencies are brought to an end” (\textbf{\cite{A}II 157}).} The distraction due to that is agitation. When his mind is seized by that agitation, he does not understand correctly [their] appearance as impermanent, he does not understand correctly [their] appearance as painful, he does not understand correctly [their] appearance as not-self.

                        \vismParagraph{XX.107}{107}{}
                        “Likewise, when he is bringing [formations] to mind as impermanent, knowledge arises in him … happiness … tranquillity … bliss … resolution … exertion … establishment … equanimity … attachment arises in him. He adverts to the attachment thus, ‘Attachment is a [Noble One’s] state.’ The distraction due to that is agitation. When his mind is seized by that agitation, he does not correctly understand [their] appearance as impermanent, \textcolor{brown}{\textit{[634]}} he does not correctly understand [their] appearance as painful, he does not correctly understand [their] appearance as not-self” (\textbf{\cite{Paṭis}II 100}). \emph{1. }Herein, illumination is illumination due to insight.\footnote{\vismAssertFootnoteCounter{34}\vismHypertarget{XX.n34}{}“‘\emph{Illumination due to insight}’ is the luminous materiality originated by insight consciousness, and that originated by temperature belonging to his own continuity. Of these, that originated by insight consciousness is bright and is found only in the meditator’s body. The other kind is independent of his body and spreads all round over what is capable of being experienced by knowledge. It becomes manifest to him too, and he sees anything material in the place touched by it” (\textbf{\cite{Vism-mhṭ}816}).} When it arises, the meditator thinks, “Such illumination never arose in me before. I have surely reached the path, reached fruition;” thus he takes what is not the path to be the path and what is not fruition to be fruition. When he takes what is not the path to be the path \marginnote{\textcolor{teal}{\footnotesize\{720|662\}}}{}and what is not fruition to be fruition, the course of his insight is interrupted. He drops his own basic meditation subject and sits just enjoying the illumination.

                        \vismParagraph{XX.108}{108}{}
                        But this illumination arises in one bhikkhu illuminating only as much as the seat he is sitting on; in another, the interior of his room; in another, the exterior of his room; in another the whole monastery … a quarter league … a half league … a league … two leagues … three leagues; in another bhikkhu it arises making a single light from the earth’s surface up to the Brahmā-world. But in the Blessed One it arose illuminating the ten-thousandfold world-element.

                        \vismParagraph{XX.109}{109}{}
                        This story illustrates how it varies. Two elders, it seems, were sitting inside a room with a double wall at Cittalapabbata. It was the Uposatha of the dark of the moon that day. All directions were covered by a blanket of cloud, and at night the four-factored gloom\footnote{\vismAssertFootnoteCounter{35}\vismHypertarget{XX.n35}{}\emph{Caturaṅga-samannāgataṃ tamaṃ—}“four-factored gloom” is mentioned also at \textbf{\cite{S-a}I 170}, \textbf{\cite{M-a}V 16} (c. \emph{andhakāra}), and \textbf{\cite{Ud-a}66, 304}.} prevailed. Then one elder said, “Venerable sir, the flowers of the five colours on the lion table on the shrine terrace are visible to me now.” The other said, “What you say is nothing wonderful, friend. Actually the fishes and turtles in the ocean a league away are visible to me now.”

                        \vismParagraph{XX.110}{110}{}
                        This imperfection of insight usually arises in one who has acquired serenity and insight. Because the defilements suppressed by the attainments do not manifest themselves, he thinks, “I am an Arahant,” like the Elder Mahā-Nāga who lived at Uccavālika, like the Elder Mahā-Datta who lived at Haṅkana, like the Elder Cūḷa-Sumana who lived in the Nikapenna meditation house at Cittalapabbata.

                        \vismParagraph{XX.111}{111}{}
                        Here is one story as an illustration. The Elder Dhammadinna, it seems, who lived at Talaṅgara—one of the great ones with cankers destroyed who possessed the categories of discrimination—was the instructor of a large community of bhikkhus. One day, as he was sitting in his own daytime quarters, he wondered, “Has our teacher, the Elder Mahā-Nāga who lives at Uccavālika, \textcolor{brown}{\textit{[635]}} brought his work of asceticism to its conclusion, or not?” He saw that he was still an ordinary man, and he knew that if he did not go to him, he would die an ordinary man. He rose up into the air with supernormal power and alighted near the elder, who was sitting in his daytime quarters. He paid homage to him, doing his duty, and sat down at one side. To the question, “Why have you come unexpectedly, friend Dhammadinna?” he replied, “I have come to ask a question, venerable sir.” He was told, “Ask, friend. If we know, we shall say.” He asked a thousand questions.

                        \vismParagraph{XX.112}{112}{}
                        The elder replied without hesitation to each question. To the remark, “Your knowledge is very keen, venerable sir; when was this state attained by you?” he replied, “Sixty years ago, friend.” “Do you practice concentration, venerable sir?”—“That is not difficult, friend.”—“Then make an elephant, venerable sir.” The elder made an elephant all white. “Now, venerable sir, make that elephant come straight at you with his ears outstretched, his tail extended, putting his trunk in his mouth and making a horrible trumpeting.” The elder did so. Seeing the frightful aspect of the rapidly approaching elephant, he sprang up and made to run away. Then the elder with cankers destroyed put out his hand, and catching him by the hem of his robe, he said, “Venerable sir, is there any timidity in one whose cankers are destroyed?”

                        \vismParagraph{XX.113}{113}{}
                        \marginnote{\textcolor{teal}{\footnotesize\{721|663\}}}{}Then he recognized that he was still an ordinary man. He knelt at Dhammadinna’s feet and said, “Help me, friend Dhammadinna.”—“Venerable sir, I will help you; that is why I came. Do not worry.” Then he expounded a meditation subject to him. The elder took the meditation subject and went up on to the walk, and with the third footstep he reached Arahantship. The elder was a bhikkhu of hating temperament, it seems. Such bhikkhus waver on account of illumination.

                        \vismParagraph{XX.114}{114}{}
                        \emph{2. Knowledge }is knowledge due to insight. As he is estimating and judging material and immaterial states perhaps knowledge that is unerring, keen, incisive, and very sharp arises in him, like a lightning flash.

                        \vismParagraph{XX.115}{115}{}
                        \emph{3. Rapturous happiness }is happiness due to insight. Perhaps at that time the five kinds of happiness, namely, minor happiness, momentary happiness, showering happiness, uplifting happiness, and pervading (rapturous) happiness arise in him filling his whole body.

                        \vismParagraph{XX.116}{116}{}
                        \emph{4. Tranquillity }is tranquillity due to insight. As he is sitting at that time in his night or day quarters perhaps \textcolor{brown}{\textit{[636]}} there is no fatigue or heaviness or rigidity or unwieldiness or sickness or crookedness in his body and his mind, but rather his body and mind are tranquillized, light, malleable, wieldy, quite sharp, and straight. With his body and mind aided by this tranquillity, etc., he experiences at that time the superhuman delight with reference to which it is said:
                        \begin{verse}
                            A bhikkhu when his mind is quiet\\{}
                            Retires to an empty place,\\{}
                            And his right insight in the Dhamma\\{}
                            Gives him superhuman delight.\\{}
                            It is because he comprehends\\{}
                            The rise and fall of aggregates\\{}
                            That he finds happiness and joy\\{}
                            And knows it to be deathless (\textbf{\cite{Dhp}373f.}).
                        \end{verse}


                        This is how tranquillity, associated with lightness, etc., arises in him, bringing about this superhuman delight.

                        \vismParagraph{XX.117}{117}{}
                        \emph{5. Bliss }(\emph{pleasure}) is bliss due to insight. At that time perhaps there arises in him exceedingly refined bliss (pleasure) flooding his whole body.

                        \vismParagraph{XX.118}{118}{}
                        \emph{6. Resolution }is faith. For strong faith arises in him in association with insight in the form of extreme confidence of consciousness and its concomitants.

                        \vismParagraph{XX.119}{119}{}
                        \emph{7. Exertion }is energy. For well-exerted energy, neither too lax nor too strained, arises in him in association with insight.

                        \vismParagraph{XX.120}{120}{}
                        \emph{8. Assurance }(lit. establishment) is mindfulness. For well-established (well-assured), well-founded mindfulness, which is dug in and as immovable as the king of mountains, arises in him in association with insight. Whatever subject he adverts to, consciously reacts to, gives attention to, reviews, appears to him (he is assured of) owing to mindfulness, which descends into it,\footnote{\vismAssertFootnoteCounter{36}\vismHypertarget{XX.n36}{}\emph{Okkhandati—}“to descend into”: not in PED; see \hyperlink{XXII.34}{XXII.34}{} and \textbf{\cite{M-a}I 238}.} enters into it, just as the other world does to one who has the divine eye.

                        \vismParagraph{XX.121}{121}{}
                        \marginnote{\textcolor{teal}{\footnotesize\{722|664\}}}{}\emph{9. Equanimity }is both equanimity about insight and equanimity in adverting.\footnote{\vismAssertFootnoteCounter{37}\vismHypertarget{XX.n37}{}“‘\emph{Equanimity about insight}’ is neutrality in the investigation of formations owing to the objective field having been already investigated. But in meaning, when it occurs thus, it is only neutrality. The volition associated with mind-door adverting is called ‘\emph{equanimity }(\emph{upekkhā}) \emph{in adverting}’ because it occurs in adverting as onlooking (\emph{ajjhupekkhana})” (Vism-mhṭ 819).} For equanimity about insight, which is neutrality about formations, arises strongly in him at that time. It is also equanimity in adverting in the mind door. For whatever the subject he adverts to, his adverting works as incisively and sharply as a lightning flash, like a red-hot spear plunged into a basket of leaves.

                        \vismParagraph{XX.122}{122}{}
                        10. \emph{Attachment }is attachment due to insight. For when his insight is adorned with illumination, etc., attachment arises in him, which is subtle and peaceful in aspect, and it relies on (clings to) that insight; and he is not able to discern that attachment as a defilement. \textcolor{brown}{\textit{[637]}}

                        \vismParagraph{XX.123}{123}{}
                        And as in the case of illumination, so too in the case of the other imperfections that may arise, the meditator thinks thus: “Such knowledge … such rapturous happiness … tranquillity … bliss … resolution … exertion … assurance … equanimity … attachment never arose in me before. I have surely reached the path, reached fruition.” Thus he takes what is not the path to be the path and what is not fruition to be fruition. When he takes what is not the path to be the path and what is not fruition to be fruition, the course of his insight is interrupted. He drops his basic meditation subject and sits just enjoying the attachment.

                        \vismParagraph{XX.124}{124}{}
                        And here illumination, etc., are called imperfections because they are the basis for imperfection, not because they are [kammically] unprofitable. But attachment is both an imperfection and the basis for imperfection.

                        As basis only they amount to ten; but with the different ways of taking them they come to thirty.

                        \vismParagraph{XX.125}{125}{}
                        How? When a man takes it thus, “illumination has arisen in me,” his way of taking is due to [false] view. When he takes it thus, “How agreeable this illumination that has arisen is,” his way of taking is due to pride (conceit). When he relishes the illumination, his way of taking is due to craving. So there are three ways of taking it in the case of illumination, that is to say, due to [false] view, to pride (conceit), and to craving. Likewise with the rest. So they come to thirty with the three ways of taking them. Owing to their influence an unskilful, unwary meditator wavers and gets distracted about illumination, etc., and he sees each one of them-illumination and the rest-as “This is mine, this is I, this is my self” (\textbf{\cite{M}I 135}). Hence the Ancients said:
                        \begin{verse}
                            He wavers about illumination,\\{}
                            And knowledge, rapturous happiness,\\{}
                            About the tranquilness, the bliss,\\{}
                            Whereby his mind becomes confused;\\{}
                            He wavers about resolution,\\{}
                            Exertion, and assurance, too,\\{}
                            The adverting-equanimity,\\{}
                            And equanimity and attachment (\textbf{\cite{Paṭis}II 102}).
                        \end{verse}


                        \vismParagraph{XX.126}{126}{}
                        \marginnote{\textcolor{teal}{\footnotesize\{723|665\}}}{}But when illumination, etc., arise, a skilful, wary meditator who is endowed with discretion either defines and examines it with understanding thus: “This illumination has arisen.\footnote{\vismAssertFootnoteCounter{38}\vismHypertarget{XX.n38}{}Be \textbf{\cite{Vism-mhṭ}} reads “\emph{ayaṃ kho so}” instead of the “\emph{ayaṃ kho me}” in the Ee and Ae editions.} But it is impermanent, formed, conditionally arisen, subject to destruction, subject to fall, subject to fading away, subject to cessation.” Or he thinks: “If illumination were self, it would be right to take it as self; but being not-self, it is taken as self. Therefore it is not-self in the sense of no power being exercisable over it; it is impermanent in the sense of non-existence after having come to be; it is painful in the sense of oppression by rise and fall,” all of which should be treated in detail according to the method given under the immaterial septad (\hyperlink{XX.83}{§83}{}). And as in the case of illumination, so too with the rest.

                        \vismParagraph{XX.127}{127}{}
                        Having investigated it thus, he sees the illumination as “This is not mine, this is not I, this is not my self.” \textcolor{brown}{\textit{[638]}}\textbf{ }He sees knowledge … (etc.) … attachment as “This is not mine, this is not I, this is not my self.” Seeing thus, he does not waver or vacillate about the illumination, and so on. Hence the Ancients said:
                        \begin{verse}
                            So when a man of understanding has\\{}
                            Examined these ten things and is now skilled\\{}
                            In agitation about higher states\\{}
                            He no more falls a prey to wavering (\textbf{\cite{Paṭis}II 102}).
                        \end{verse}


                        \vismParagraph{XX.128}{128}{}
                        So he unravels this thirtyfold skein of imperfections without falling a prey to wavering. He defines what is the path and what is not the path thus: “The states consisting in illumination, etc., are not the path; but it is insight knowledge that is free from imperfections and keeps to its course that is the path.”

                        \vismParagraph{XX.129}{129}{}
                        The knowledge that is established in him by his coming to know the path and the not-path thus, “This is the path, this is not the path,” should he understood as the purification by knowledge and vision of what is the path and what is not the path.

                        \vismParagraph{XX.130}{130}{}
                        So at this point the defining of three truths has been effected by him. How? The defining of the truth of suffering has been effected with the defining of mentality-materiality in the purification of view. The defining of the truth of origination has been effected with the discerning of conditions in the purification by overcoming doubt. The defining of the truth of the path has been effected with the emphasizing of the right path in this purification by knowledge and vision of what is the path and what is not the path. So the defining of three truths has been effected firstly by means of mundane knowledge only.

                        The twentieth chapter called “The Description of Purification by Knowledge and Vision of What Is the Path and What is Not the Path” in the Treatise on the Development of Understanding in the \emph{Path of Purification }composed for the purpose of gladdening good people.
        \chapter[Purification by Knowledge and Vision of the Way]{Purification by Knowledge and Vision of the Way\vismHypertarget{XXI}\newline{\textnormal{\emph{Paṭipadā-ñāṇadassana-visuddhi-niddesa}}}}
            \label{XXI}

            \section[\vismAlignedParas{§1–2}Introductory]{Introductory}

                \vismParagraph{XXI.1}{1}{}
                \marginnote{\textcolor{teal}{\footnotesize\{724|666\}}}{}\textcolor{brown}{\textit{[639]}} Now, insight reaches its culmination with the eight knowledges, and knowledge in conformity with truth\footnote{\vismAssertFootnoteCounter{1}\vismHypertarget{XXI.n1}{}“He calls conformity knowledge ‘\emph{knowledge in conformity with truth}’ because it is suitable for penetrating the truths owing to the disappearance of the grosser darkness of delusion that conceals the truths” (\textbf{\cite{Vism-mhṭ}822}). The term \emph{saccānulomikañāṇa—}“knowledge in conformity with truth,” occurs at \textbf{\cite{Vibh}315}. The term \emph{anulomañāṇa—}“conformity knowledge,” occurs in the Paṭṭhāna (\textbf{\cite{Paṭṭh}I 159}), but not elsewhere in the Piṭakas apparently.} is ninth; these are what is called purification by knowledge and vision of the way.

                The eight should be understood as follows: (1) knowledge of contemplation of rise and fall, which is insight free from imperfections and steady on its course, (2) knowledge of contemplation of dissolution, (3) knowledge of appearance as terror, (4) knowledge of contemplation of danger, (5) knowledge of contemplation of dispassion, (6) knowledge of desire for deliverance, (7) knowledge of contemplation of reflection, and (8) knowledge of equanimity about formations.\footnote{\vismAssertFootnoteCounter{2}\vismHypertarget{XXI.n2}{}“\emph{Knowledge of rise and fall} that has become familiar should be understood as belonging to full-understanding as abandoning. The contemplation of only the dissolution of formations is contemplation of dissolution; that same contemplation as knowledge is \emph{knowledge of contemplation of dissolution}. One who, owing to it, sees things as they are is terrified, thus it is terror. The knowledge that seizes the terrifying aspect of states of the three planes when they appear as terrifying is \emph{knowledge of appearance as terror}. One desires to be delivered, thus it is one desiring deliverance: that is, either as a consciousness or as a person. His (its) state is desire for deliverance. That itself as \emph{knowledge is knowledge of desire for deliverance}. Knowledge that occurs in the mode of reflecting again is \emph{knowledge of contemplation of reflection}. Knowledge that occurs as looking on (\emph{upekkhanā}) at formations with indifference (\emph{nirapekkhatā}) is knowledge of equanimity (\emph{upekkhā}) about formations” (\textbf{\cite{Vism-mhṭ}822–823}).}

                “Knowledge in conformity with truth as ninth” is a term for conformity.

                So one who wants to perfect this should make these kinds of knowledge his task, starting with knowledge of rise and fall free from imperfections.

                \vismParagraph{XXI.2}{2}{}
                But why does he again pursue knowledge of rise and fall? To observe the [three] characteristics. The knowledge of rise and fall already dealt with, being \marginnote{\textcolor{teal}{\footnotesize\{725|667\}}}{}disabled by the ten imperfections, was not capable of observing the three characteristics in their true nature; but once freed from imperfections, it is able to do so. So he should pursue it again here in order to observe the characteristics. \textcolor{brown}{\textit{[640]}}
            \section[\vismAlignedParas{§3–136}Insight: The Eight Knowledges]{Insight: The Eight Knowledges}
                \subsection[\vismAlignedParas{§3–9}1. Knowledge of Rise and Fall—II]{1. Knowledge of Rise and Fall—II}

                    \vismParagraph{XXI.3}{3}{}
                    Now, the characteristics fail to become apparent when something is not given attention and so something conceals them. What is that? Firstly, the characteristic of impermanence does not become apparent because when rise and fall are not given attention, it is concealed by continuity. The characteristic of pain does not become apparent because, when continuous oppression is not given attention, it is concealed by the postures. The characteristic of not-self does not become apparent because when resolution into the various elements is not given attention, it is concealed by compactness.

                    \vismParagraph{XXI.4}{4}{}
                    However, when continuity is disrupted by discerning rise and fall, the characteristic of impermanence becomes apparent in its true nature.

                    When the postures are exposed by attention to continuous oppression, the characteristic of pain becomes apparent in its true nature. When the resolution of the compact is effected by resolution into elements, the characteristic of not-self becomes apparent in its true nature.\footnote{\vismAssertFootnoteCounter{3}\vismHypertarget{XXI.n3}{}Cf. \textbf{\cite{Peṭ}128}. In the commentary to the Āyatana-Vibhaṅga we find: “Impermanence is obvious, as when a saucer (say) falls and breaks; … pain is obvious, as when a boil (say) appears in the body; … the characteristic of not-self is not obvious; … Whether Perfect Ones arise or do not arise the characteristics of impermanence and pain are made known, but unless there is the arising of a Buddha the characteristic of not-self is not made known” (\textbf{\cite{Vibh-a}49–50}, abridged for clarity).

                            Again, in the commentary to Majjhima Nikāya Sutta 22: “Having been, it is not, therefore it is impermanent; it is impermanent for four reasons, that is, in the sense of the state of rise and fall, of change, of temporariness, and of denying permanence. It is painful on account of the mode of oppression; it is painful for four reasons, that is, in the sense of burning, of being hard to bear, of being the basis for pain, and of opposing pleasure … It is not-self on account of the mode of insusceptibility to the exercise of power; it is not-self for four reasons, that is, in the sense of voidness, of having no owner-master, of having no Overlord, and of opposing self (\textbf{\cite{M-a}II 113}, abridged for clarity).

                            Commenting on this Vism paragraph, \textbf{\cite{Vism-mhṭ}} says: “‘\emph{When continuity is disrupted}’ means when continuity is exposed by observing the perpetual otherness of states as they go on occurring in succession. For it is not through the connectedness of states that the characteristic of impermanence becomes apparent to one who rightly observes rise and fall, but rather the characteristic becomes more thoroughly evident through their disconnectedness, as if they were iron darts. ‘\emph{When the postures are exposed}’ means when the concealment of the pain that is actually inherent in the postures is exposed. For when pain arises in a posture, the next posture adopted removes the pain, as it were, concealing it. But once it is correctly known how the pain in any posture is shifted by substituting another posture for that one, then the concealment of the pain that is in them is exposed because it has become evident that formations are being incessantly overwhelmed by pain. ‘\emph{Resolution of the compact}’ is effected by resolving [what appears compact] in this way, ‘The earth element is one, the water element is another’ etc., distinguishing each one; and in this way, ‘Contact is one, feeling is another’ etc., distinguishing each one. ‘\emph{When the resolution of the compact is effected}’ means that what is compact as a mass and what is compact as a function or as an object has been analyzed. For when material and immaterial states have arisen mutually steadying each other, [mentality and materiality, for example,] then, owing to misinterpreting that as a unity, compactness of mass is assumed through failure to subject formations to pressure. And likewise compactness of function is assumed when, although definite differences exist in such and such states’ functions, they are taken as one. And likewise compactness of object is assumed when, although differences exist in the ways in which states that take objects make them their objects, those objects are taken as one. But when they are seen after resolving them by means of knowledge into these elements, they disintegrate like froth subjected to compression by the hand. They are mere states (\emph{dhamma}) occurring due to conditions and void. In this way the characteristic of not-self becomes more evident” (\textbf{\cite{Vism-mhṭ}824}).}

                    \vismParagraph{XXI.5}{5}{}
                    \marginnote{\textcolor{teal}{\footnotesize\{726|668\}}}{}And here the following differences should be understood: the impermanent, and the characteristic of impermanence; the painful, and the characteristic of pain; the not-self, and the characteristic of not-self.

                    \vismParagraph{XXI.6}{6}{}
                    Herein, the five aggregates are impermanent. Why? Because they rise and fall and change, or because of their non-existence after having been. Rise and fall and change are the characteristic of impermanence; or mode alteration, in other words, non-existence after having been [is the characteristic of impermanence].\footnote{\vismAssertFootnoteCounter{4}\vismHypertarget{XXI.n4}{}“These modes, [that is, the three characteristics,] are not included in the aggregates because they are states without individual essence (\emph{asabhāva-dhammā}); and they are not separate from the aggregates because they are unapprehendable without the aggregates. But they should be understood as appropriate conceptual differences (\emph{paññatti-visesā}) that are reason for differentiation in the explaining of dangers in the five aggregates, and which are allowable by common usage in respect of the five aggregates” (\textbf{\cite{Vism-mhṭ}825}).}

                    \vismParagraph{XXI.7}{7}{}
                    Those same five aggregates are painful because of the words, “What is impermanent is painful” (\textbf{\cite{S}III 22}). Why? Because of continuous oppression. The mode of being continuously oppressed is the characteristic of pain.

                    \vismParagraph{XXI.8}{8}{}
                    Those same five aggregates are not-self because of the words, “What is painful is not-self” (\textbf{\cite{S}III 22}). Why? Because there is no exercising of power over them. The mode of insusceptibility to the exercise of power is the characteristic of not-self.

                    \vismParagraph{XXI.9}{9}{}
                    The meditator observes all this in its true nature with the knowledge of the contemplation of rise and fall, in other words, with insight free from imperfections and steady on its course.
                \subsection[\vismAlignedParas{§10–28}2. Knowledge of Dissolution]{2. Knowledge of Dissolution}

                    \vismParagraph{XXI.10}{10}{}
                    When he repeatedly observes in this way, and examines and investigates material and immaterial states, [to see] that they are impermanent, painful, and \marginnote{\textcolor{teal}{\footnotesize\{727|669\}}}{}not-self, then if his knowledge works keenly, formations quickly become apparent.\footnote{\vismAssertFootnoteCounter{5}\vismHypertarget{XXI.n5}{}“The keenness of knowledge comes about owing to familiarity with development. And when it is familiar, development occurs as though it were absorbed in the object owing to the absence of distraction” (\textbf{\cite{Vism-mhṭ}825}).} Once his knowledge works keenly and formations quickly become apparent, he no longer extends his mindfulness to their arising or presence or occurrence or sign, but brings it to bear only on their cessation as destruction, fall and breakup.\footnote{\vismAssertFootnoteCounter{6}\vismHypertarget{XXI.n6}{}“‘\emph{Arising}’ is the alteration consisting in generation. ‘\emph{Presence}’ is the arrival at presence: ageing is what is meant. ‘\emph{Occurrence}’ is the occurrence of what is clung to. ‘\emph{The sign}’ is the sign of formations; the appearance of formations like graspable entities, which is due to compactness of mass, etc., and to individualization of function, is the sign of formations” (\textbf{\cite{Vism-mhṭ}826}). See also n.12.

                            “It is momentary cessation that is in other words ‘\emph{cessation as destruction, fall and}

                            \emph{breakup}’” (\textbf{\cite{Vism-mhṭ}826}).} \textcolor{brown}{\textit{[641]}}

                    \vismParagraph{XXI.11}{11}{}
                    When insight knowledge has arisen in him in this way so that he sees how the field of formations, having arisen thus, ceases thus, it is called contemplation of dissolution at that stage,\footnote{\vismAssertFootnoteCounter{7}\vismHypertarget{XXI.n7}{}\emph{Etasmiṃ khaṇe} (or \emph{etasmiṃ ṭhāne}) seems a better reading here than \emph{ekasmiṃ khaṇe}’; cf. parallel phrases at the end of §29, 30, 31.} with reference to which it is said:

                    “Understanding of contemplation of dissolution, after reflecting on an object—how is this knowledge of insight?

                    “Consciousness with materiality as its object arises and dissolves. Having reflected on that object, he contemplates the dissolution of that consciousness.

                    “‘He contemplates’: how does he contemplate? He contemplates as impermanent, not as permanent; he contemplates as painful, not as pleasant; he contemplates as not-self, not as self; he becomes dispassionate, he does not delight; he causes fading away of greed, he does not inflame it; he causes cessation, not origination; he relinquishes, he does not grasp. Contemplating as impermanent, he abandons the perception of permanence. Contemplating as painful, he abandons the perception of pleasure. Contemplating as not-self, he abandons the perception of self. Becoming dispassionate, he abandons delight. Causing fading away, he abandons greed. Causing cessation, he abandons originating. Relinquishing, he abandons grasping.

                    “Consciousness with feeling as its object … Consciousness with perception as its object … with formations as its object … with consciousness as its object … with eye as its object … (etc.—see \hyperlink{XX.9}{XX.9}{}) … with ageing-and-death as its object … Relinquishing, he abandons grasping.
                    \begin{verse}
                        “The substitution of the object,\\{}
                        The transference of understanding,\\{}
                        The power of adverting—these\\{}
                        Are insight following reflection.
                    \end{verse}

                    \begin{verse}
                        “Defining both to be alike\\{}
                        By inference from that same object,\\{}
                        \marginnote{\textcolor{teal}{\footnotesize\{728|670\}}}{}Intentness on cessation—these\\{}
                        Are insight in the mark of fall.
                    \end{verse}

                    \begin{verse}
                        “Having reflected on the object\\{}
                        Dissolution he contemplates,\\{}
                        Appearance then as empty—this\\{}
                        Is insight of higher understanding.
                    \end{verse}

                    \begin{verse}
                        “Skilled in the three contemplations,\\{}
                        And in the fourfold insight too,\\{}
                        Skilled in the three appearances,\\{}
                        The various views will shake him not.
                    \end{verse}


                    “Knowledge is in the sense of that being known and understanding in the sense of the act of understanding that. Hence it was said: ‘Understanding of contemplating dissolution, after reflecting on an object, is knowledge of insight’” (\textbf{\cite{Paṭis}I 57}f).

                    \vismParagraph{XXI.12}{12}{}
                    Herein, \emph{after reflecting on an object} is having reflected on, having known, any object; the meaning is, having seen it as liable to destruction and fall. \emph{Understanding of the contemplation of dissolution}: any understanding of the contemplation of the dissolution of the knowledge arisen after reflecting on the object as liable to destruction and fall is called \emph{knowledge of insight}. \textcolor{brown}{\textit{[642]}} \emph{How }has the meaning of a question showing desire to expound.

                    \vismParagraph{XXI.13}{13}{}
                    Next, in order to show how that comes about, consciousness with materiality as its object, etc., is said. Herein, consciousness with materiality as its object arises and dissolves: \emph{rūpārammaṇaṃ cittaṃ uppajjitvā bhijjati }[is the equivalent of] \emph{rūpārammaṇaṃ cittaṃ uppajjitvā bhijjati}; or the meaning is\emph{ rūpārammaṇabhāve cittaṃ uppajjitvā bhijjati} [alternative grammatical substitution]. \emph{Having reflected on that object}: having reflected on, having known, that object consisting of materiality; the meaning is, having seen it as liable to destruction and fall. \emph{He contemplates the dissolution of that consciousness}: by means of a subsequent consciousness he contemplates the dissolution of that consciousness with which that object consisting of materiality was seen as liable to destruction and fall. Hence the Ancients said: “He sees with insight both the known and the knowledge.”

                    \vismParagraph{XXI.14}{14}{}
                    \emph{He contemplates} (\emph{anupassati}): he sees always accordingly (\emph{anu anu passati}); the meaning is, he sees again and again in various modes. Hence it is said: “He contemplates”: \emph{how does he contemplate? He contemplates as impermanent}, and so on.

                    \vismParagraph{XXI.15}{15}{}
                    Herein, dissolution is the culminating point of impermanence, and so the meditator contemplating dissolution contemplates the whole field of formations as \emph{impermanent, not as permanent}.\footnote{\vismAssertFootnoteCounter{8}\vismHypertarget{XXI.n8}{}“‘He contemplates as impermanent’ here not by inferential knowledge thus, “Impermanent in the sense of dissolution”, like one who is comprehending formations by groups (\hyperlink{XX.13}{XX.13}{}–\hyperlink{XX.14}{14}{}), nor by seeing fall preceded by apprehension of rise, like a beginner of insight (\hyperlink{XX.93}{XX.93ff.}{}); but rather it is after rise and fall have become apparent as actual experience through the influence of knowledge of rise and fall that he then leaves rise aside in the way stated and contemplates formations as impermanent by seeing only their dissolution. But when he sees them thus, there is no trace in him of any apprehension of them as permanent” (\textbf{\cite{Vism-mhṭ}827}).} Then, because of the painfulness of what is \marginnote{\textcolor{teal}{\footnotesize\{729|671\}}}{}impermanent and because of the non-existence of self in what is painful, he contemplates that same whole field of formations as \emph{painful, not as pleasant, he contemplates it as not-self, not as self}.

                    \vismParagraph{XXI.16}{16}{}
                    But what is impermanent, painful, not-self, is not something to delight in; and what is not something to delight in is not something to arouse greed for; consequently, when that field of formations is seen as impermanent, painful, not-self, in accordance with the contemplation of dissolution, then \emph{he becomes dispassionate, he does not delight; he causes fading away of greed, he does not inflame it}. When he does not inflame greed thus, \emph{he causes cessation} of greed, not its \emph{origination}, which happens firstly by means of mundane knowledge;\footnote{\vismAssertFootnoteCounter{9}\vismHypertarget{XXI.n9}{}“‘\emph{Causes cessation}’: he causes greed to reach the cessation of suppression; he suppresses it, is the meaning. That is why he said ‘\emph{by means of mundane knowledge}.’ And since there is suppression, how can there be arousing? Therefore he said ‘\emph{not its origination}’” (\textbf{\cite{Vism-mhṭ}828}).} the meaning is, he does not cause origination.

                    \vismParagraph{XXI.17}{17}{}
                    Or alternatively, having thus caused the fading away of greed, and caused the cessation of the seen field of formations, he causes the cessation of the unseen too by means of inferential knowledge, he does not originate it. He gives attention only to its cessation, he sees only its cessation, not its origin, is the meaning.

                    \vismParagraph{XXI.18}{18}{}
                    Progressing in this way, \emph{he relinquishes, he does not grasp}. What is meant? [What is meant is that] this contemplation of impermanence, etc., is also called both “relinquishment as giving up” and “relinquishment as entering into” (see \textbf{\cite{Paṭis}I 194}) because, by substitution of opposite qualities, it gives up defilements along with aggregate producing kamma-formations, and because, by seeing the unsatisfactoriness of what is formed, \textcolor{brown}{\textit{[643]}} it also enters into, by inclining towards, Nibbāna, which is the opposite of the formed. Therefore the bhikkhu who possesses that [contemplation] gives up defilements and enters into Nibbāna in the way stated, he does not grasp (cling to) defilements by causing rebirth, nor does he grasp (cling to) a formed object through failing to see its unsatisfactoriness. Hence it was said: \emph{he relinquishes, he does not grasp}.

                    \vismParagraph{XXI.19}{19}{}
                    Now, in order to show which states are abandoned by these three kinds of knowledge, \emph{contemplating as impermanent, he abandons the perception of permanence}, etc., is said. Herein, delight is craving accompanied by happiness. The rest is as already stated.

                    \vismParagraph{XXI.20}{20}{}
                    As to the stanzas: \emph{the substitution of the object} [means that] after seeing the dissolution of materiality, there is the substitution of another object for that first object by seeing the dissolution of the consciousness by which the dissolution [of materiality] was seen. \emph{Transference of understanding} is the abandoning of rise and the specializing in fall. \emph{The power of adverting} is the ability, after seeing the dissolution of materiality, to advert immediately for the purpose of seeing the dissolution of the consciousness that had that dissolution as its object. \emph{Are insight following reflection}: this is called contemplation of dissolution after reflecting on an object.

                    \vismParagraph{XXI.21}{21}{}
                    \emph{Defining both to be alike by inference from that same object}: the meaning is that by inference, by induction, from the object seen by actual experience he defines \marginnote{\textcolor{teal}{\footnotesize\{730|672\}}}{}both [the seen and the unseen] to have a single individual essence thus, “The field of formations dissolved in the past, and will break up in the future, just as it does [in the present].” And this is said by the Ancients:
                    \begin{verse}
                        “With vision of those present purified\\{}
                        He infers those past and future to be alike;\\{}
                        He infers that all formations disappear,\\{}
                        Like dew-drops when the morning sun comes up.”
                    \end{verse}


                    \vismParagraph{XXI.22}{22}{}
                    \emph{Intentness on cessation}: after thus giving to both a single definition based on their dissolution, he thus becomes intent on cessation, in other words, on that same dissolution. The meaning is that he attaches importance to it, inclines, tends, leans towards it. \emph{Are insight in the mark of fall}: what is meant is that this is called insight into the characteristic of fall.

                    \vismParagraph{XXI.23}{23}{}
                    \emph{Having reflected on the object}: having first known the object consisting of materiality, and so on. \emph{Dissolution he contemplates}: having seen the dissolution of that object, he contemplates the dissolution of the consciousness that had that as its object. \textcolor{brown}{\textit{[644]}}

                    \vismParagraph{XXI.24}{24}{}
                    \emph{Appearance then as empty}: while he is contemplating dissolution in this way, he succeeds in making [formations] appear as void thus, “Only formations breakup; their breakup is death; there is nothing else at all\footnote{\vismAssertFootnoteCounter{10}\vismHypertarget{XXI.n10}{}“Here in this world there is no self that is something other than and apart from the aggregates” (Vism-mhṭ 830). Cf. also: “When any ascetics or brahmans whatever see self in its various forms, they all of them see the five aggregates, or one of them” (\textbf{\cite{S}IV 46}).}.” Hence the Ancients said:
                    \begin{verse}
                        “Aggregates cease and nothing else exists;\\{}
                        Breakup of aggregates is known as death.\\{}
                        He watches their destruction steadfastly,\\{}
                        As one who with a diamond drills a gem.” \footnote{\vismAssertFootnoteCounter{11}\vismHypertarget{XXI.n11}{}“As a skilled man drilling a gem with a tool watches and keeps in mind only the hole he is drilling, not the gem’s colour, etc., so too the meditator wisely keeps in mind only the ceaseless dissolution of formations, not the formations” (\textbf{\cite{Vism-mhṭ}830}).}
                    \end{verse}


                    \vismParagraph{XXI.25}{25}{}
                    \emph{Is insight of higher understanding}: what is meant is that the reflection on the object, the contemplation of dissolution, and the appearance as void are called insight of higher understanding.

                    \vismParagraph{XXI.26}{26}{}
                    \emph{Skilled in the three contemplations}: a bhikkhu who is competent in the three beginning with contemplation of impermanence. \emph{And in the fourfold insight too}: in the four kinds of insight beginning with dispassion. \emph{Skilled in the three appearances}: and owing to skill in this threefold appearance, namely, as liable to destruction and fall, as terror, and as void.\footnote{\vismAssertFootnoteCounter{12}\vismHypertarget{XXI.n12}{}The Harvard text reads “\emph{khayato vayato suññato ti—}as destruction, as fall, as void.” But \textbf{\cite{Vism-mhṭ}} says: “‘\emph{The three appearances’}: in the threefold appearance as impermanent and so on. For appearance as destruction and fall is appearance as impermanent, appearance as terror is appearance as pain, and appearance as void is appearance as not-self (\textbf{\cite{Vism-mhṭ}830}).} \emph{The various views will shake him not}: he does not vacillate on account of the various kinds of views such as the eternity view.

                    \vismParagraph{XXI.27}{27}{}
                    \marginnote{\textcolor{teal}{\footnotesize\{731|673\}}}{}When he no longer vacillates and so constantly bears in mind that the unceased will also cease, the undissolved will also dissolve, then he disregards the arising, presence, occurrence and sign of all formations, which keep on breaking up, like fragile pottery being smashed, like fine dust being dispersed, like sesame seeds being roasted, and he sees only their breakup. Just as a man with eyes standing on the bank of a pond or on the bank of a river during heavy rain would see large bubbles appearing on the surface of the water and breaking up as soon as they appeared, so too he sees how formations break up all the time. The Blessed One said of such a meditator:
                    \begin{verse}
                        “And he who looks upon the world\\{}
                        As one who looks upon a bubble,\\{}
                        As one who looks upon a mirage,\\{}
                        Is out of sight of Death the King” (\textbf{\cite{Dhp}170}).
                    \end{verse}


                    \vismParagraph{XXI.28}{28}{}
                    When he constantly sees that all formations thus break up all the time, then contemplation of dissolution grows strong in him, bringing eight advantages, which are these: abandoning of [false] view of becoming, giving up attachment to life, constant application, a purified livelihood, no more anxiety, absence of fear, acquisition of patience and gentleness, and conquest of aversion (boredom) and sensual delight. \textcolor{brown}{\textit{[645]}} Hence the Ancients said:
                    \begin{verse}
                        “On seeing these eight perfect qualities\\{}
                        He comprehends formations constantly,\\{}
                        Seeing breakup in order to attain\\{}
                        The Deathless, like the sage with burning turban.”\\{}
                        (see \textbf{\cite{S}V 440})
                    \end{verse}


                    Knowledge of contemplation of dissolution is ended.
                \subsection[\vismAlignedParas{§29–34}3. Knowledge of Appearance as Terror]{3. Knowledge of Appearance as Terror}

                    \vismParagraph{XXI.29}{29}{}
                    As he repeats, develops and cultivates in this way the contemplation of dissolution, the object of which is cessation consisting in the destruction, fall and breakup of all formations, then formations classed according to all kinds of becoming, generation, destiny, station, or abode of beings, appear to him in the form of a great terror, as lions, tigers, leopards, bears, hyenas, spirits, ogres, fierce bulls, savage dogs, rut-maddened wild elephants, hideous venomous serpents, thunderbolts, charnel grounds, battlefields, flaming coal pits, etc., appear to a timid man who wants to live in peace. When he sees how past formations have ceased, present ones are ceasing, and those to be generated in the future will cease in just the same way, then what is called knowledge of appearance as terror arises in him at that stage.

                    \vismParagraph{XXI.30}{30}{}
                    \emph{Here is a simile}: a woman’s three sons had offended against the king, it seems. The king ordered their heads to be cut off. She went with her sons to the place of their execution. When they had cut off the eldest one’s head, they set about cutting off the middle one’s head. Seeing the eldest one’s head already cut off and the middle one’s head being cut off, she gave up hope for the youngest, thinking, “He too will fare like them.” Now, the meditator’s seeing the cessation of past formations is like the woman’s seeing the eldest son’s head cut off. His \marginnote{\textcolor{teal}{\footnotesize\{732|674\}}}{}seeing the cessation of those present is like her seeing the middle one’s head being cut off. His seeing the cessation of those in the future, thinking, “Formations to be generated in the future will cease too,” is like her giving up hope for the youngest son, thinking, “He too will fare like them.” When he sees in this way, knowledge of appearance as terror arises in him at that stage.

                    \vismParagraph{XXI.31}{31}{}
                    Also another simile: a woman with an infected womb had, it seems, given birth to ten children. \textcolor{brown}{\textit{[646]}} Of these, nine had already died and one was dying in her hands. There was another in her womb. Seeing that nine were dead and the tenth was dying, she gave up hope about the one in her womb, thinking, “It too will fare just like them.” Herein, the meditator’s seeing the cessation of past formations is like the woman’s remembering the death of the nine children. The meditator’s seeing the cessation of those present is like her seeing the moribund state of the one in her hands. His seeing the cessation of those in the future is like her giving up hope about the one in her womb. When he sees in this way, knowledge of appearance as terror arises in him at that stage.

                    \vismParagraph{XXI.32}{32}{}
                    But does the knowledge of appearance as terror [itself] fear or does it not fear? It does not fear. For it is simply the mere judgment that past formations have ceased, present ones are ceasing, and future ones will cease. Just as a man with eyes looking at three charcoal pits at a city gate is not himself afraid, since he only forms the mere judgment that all who fall into them will suffer no little pain;—or just as when a man with eyes looks at three spikes set in a row, an acacia spike, an iron spike, and a gold spike, he is not himself afraid, since he only forms the mere judgment that all who fall on these spikes will suffer no little pain;—so too the knowledge of appearance as terror does not itself fear; it only forms the mere judgment that in the three kinds of becoming, which resemble the three charcoal pits and the three spikes, past formations have ceased, present ones are ceasing, and future ones will cease.

                    \vismParagraph{XXI.33}{33}{}
                    But it is called “appearance as terror” only because formations in all kinds of becoming, generation, destiny, station, or abode are fearful in being bound for destruction and so they appear only as a terror.

                    Here is the text about its appearance to him as terror: “When he brings to mind as impermanent, what appears to him as terror? When he brings to mind as painful, what appears to him as terror? When he brings to mind as not-self, what appears to him as terror? When he brings to mind as impermanent, the sign appears to him as terror. When he brings to mind as painful, occurrence appears to him as terror. When he brings to mind as not-self, the sign and occurrence appear to him as terror” (\textbf{\cite{Paṭis}II 63}).

                    \vismParagraph{XXI.34}{34}{}
                    Herein, \emph{the sign }is the sign of formations. This is a term for past, future and present formations themselves. \textcolor{brown}{\textit{[647]}} He sees only the death of formations when he brings them to mind as impermanent and so the sign appears to him as a terror. \emph{Occurrence} is occurrence in material and immaterial becoming. He sees occurrence—though ordinarily reckoned as pleasure—only as a state of being continuously oppressed when he brings them to mind as painful, and so occurrence appears to him as a terror. \marginnote{\textcolor{teal}{\footnotesize\{733|675\}}}{}He sees both the sign and the occurrence as empty, vain, void, without power or guide, like an empty village, a mirage, a goblin city, etc., when he brings [them] to mind as not-self, and so the sign and occurrence appear to him as a terror.

                    Knowledge of appearance as terror is ended.
                \subsection[\vismAlignedParas{§35–42}4. Knowledge of Danger]{4. Knowledge of Danger}

                    \vismParagraph{XXI.35}{35}{}
                    As he repeats, develops and cultivates the knowledge of appearance as terror he finds no asylum, no shelter, no place to go to, no refuge in any kind of becoming, generation, destiny, station, or abode. In all the kinds of becoming, generation, destiny, station, and abode there is not a single formation that he can place his hopes in or hold on to. The three kinds of becoming appear like charcoal pits full of glowing coals, the four primary elements like hideous venomous snakes (\textbf{\cite{S}IV 174}), the five aggregates like murderers with raised weapons (\textbf{\cite{S}IV 174}), the six internal bases like an empty village, the six external bases like village-raiding robbers (\textbf{\cite{S}IV 174–175}), the seven stations of consciousness and the nine abodes of beings as though burning, blazing and glowing with the eleven fires (see \textbf{\cite{S}IV 19}), and all formations appear as a huge mass of dangers destitute of satisfaction or substance, like a tumour, a disease, a dart, a calamity, an affliction (see \textbf{\cite{M}I 436}). How?

                    \vismParagraph{XXI.36}{36}{}
                    They appear as a forest thicket of seemingly pleasant aspect but infested with wild beasts, a cave full of tigers, water haunted by monsters and ogres, an enemy with raised sword, poisoned food, a road beset by robbers, a burning coal, a battlefield between contending armies appear to a timid man who wants to live in peace. And just as that man is frightened and horrified and his hair stands up when he comes upon a thicket infested by wild beasts, etc., and he sees it as nothing but danger, so too when all formations have appeared as a terror by contemplation of dissolution, this meditator sees them as utterly destitute of any core or any satisfaction and as nothing but danger.

                    \vismParagraph{XXI.37}{37}{}
                    “How is it that understanding of appearance as terror is knowledge of danger? \textcolor{brown}{\textit{[648]}}

                    “(1.a.) Understanding of appearance as terror thus, ‘Arising is terror,’ is knowledge of danger. Understanding of appearance as terror thus, ‘Occurrence is terror’ … ‘The sign is terror’ … ‘Accumulation is terror’ … ‘Rebirth-linking is terror’ … ‘Destiny is terror’ … ‘Generation is terror’ … ‘Re-arising is terror’ … ‘Birth is terror’ … ‘Ageing is terror’ … ‘Sickness is terror’ … ‘Death is terror’ … ‘Sorrow is terror’ … Understanding of appearance as terror thus, ‘Lamentation is terror,’ is knowledge of danger. Understanding of appearance as terror thus, ‘Despair is terror,’ is knowledge of danger.

                    “(1.b.) Knowledge of the state of peace is this: ‘Non-arising is safety.’ Knowledge of the state of peace is this: ‘Non-occurrence is safety’ … (etc.) … Knowledge of the state of peace is this: ‘Non-despair is safety.’

                    “(1.c.) Knowledge of the state of peace is this: ‘Arising is terror; non-arising is safety.’ Knowledge of the state of peace is this: ‘Occurrence is terror; non-occurrence\marginnote{\textcolor{teal}{\footnotesize\{734|676\}}}{} is safety’ … (etc.) … Knowledge of the state of peace is this: ‘Despair is terror; non-despair is safety.’

                    “(2.a.) Understanding of appearance as terror thus, ‘Arising is suffering,’ is knowledge of danger. Understanding of appearance as terror thus, ‘Occurrence is suffering’ … (etc.) … ‘Despair is suffering’ is knowledge of danger.

                    “(2.b.) Knowledge of the state of peace is this: ‘Non-occurrence is bliss’ … (etc.) … Knowledge of the state of peace is this: ‘Non-despair is bliss.’

                    “(2.c.) Knowledge of the state of peace is this: ‘Arising is suffering; non-arising is bliss.’ Knowledge of the state of peace is this: ‘Occurrence is suffering; non-occurrence is bliss’ … (etc.) … Knowledge of the state of peace is this: ‘Despair is suffering; non-despair is bliss.’

                    “(3.a.) Understanding of appearance as terror thus, ‘Arising is worldly,’ is knowledge of danger. Understanding of appearance as thus, ‘Occurrence is worldly’ … (etc.) … ‘Despair is worldly’ is knowledge of danger.

                    “(3.b.) Knowledge of the state of peace is this: ‘Non-arising is unworldly.’ Knowledge of the state of peace is this: ‘Non-occurrence is unworldly’ … (etc.) … Knowledge of the state of peace is this: ‘Non-despair is unworldly.’

                    “(3.c.) Knowledge of the state of peace is this: ‘Arising is worldly; non-arising is unworldly.’ Knowledge of the state of peace is this: ‘Occurrence is worldly; non-occurrence is unworldly’ … (etc.) … Knowledge of the state of peace is this: ‘Despair is worldly; non-despair is unworldly.’

                    “(4.a.) Understanding of appearance as terror thus, ’Arising is formations,’ is knowledge of danger. Understanding of appearance as terror thus, Occurrence is formations’ … (etc.) … ‘Despair is formations’ is knowledge of danger.

                    “(4.b.) Knowledge of the state of peace is this: ‘Non-arising is Nibbāna.” Knowledge of the state of peace is this: ‘Non-occurrence is Nibbāna’ … (etc.) … Knowledge of the state of peace is this Non-despair is Nibbāna.’

                    “(4.c.) Knowledge of the state of peace is this: ‘Arising is formations; non-arising is Nibbāna.’ Knowledge of the state of peace is this: ‘Occurrence is formations; non-occurrence is Nibbāna’ … (etc.) … Knowledge of the state of peace is this: ‘Despair is formations; non-despair is Nibbāna.’ \textcolor{brown}{\textit{[649]}}
                    \begin{verse}
                        “He contemplates as suffering\\{}
                        Arising, occurrence, and the sign,\\{}
                        Accumulation, rebirth-linking—\\{}
                        And this his knowledge is of danger.
                    \end{verse}

                    \begin{verse}
                        “He contemplates as bliss no arising,\\{}
                        And no occurrence, and no sign,\\{}
                        No accumulation, no rebirth-linking—\\{}
                        And this his knowledge is of peace.
                    \end{verse}

                    \begin{verse}
                        “This knowledge about danger has\\{}
                        Five sources for its origin;\\{}
                        Knowledge of peace has also five—\\{}
                        Ten knowledges he understands.
                    \end{verse}

                    \begin{verse}
                        \marginnote{\textcolor{teal}{\footnotesize\{735|677\}}}{}“When skilled in these two kinds of knowledge\\{}
                        The various views will shake him not.
                    \end{verse}


                    “Knowledge is in the sense of that being known and understanding is in the sense of the act of understanding that. Hence it was said: ‘Understanding of appearance as terror is knowledge of danger’” (\textbf{\cite{Paṭis}I 59}f).

                    \vismParagraph{XXI.38}{38}{}
                    Herein, arising is appearance here [in this becoming] with previous kamma as condition. \emph{Occurrence} is the occurrence of what has arisen in this way. \emph{The sign }is the sign of all formations. \emph{Accumulation} is the kamma that is the cause of future rebirth-linking. Rebirth-linking is future appearance. \emph{Destiny} is the destiny in which the rebirth-linking takes place. \emph{Generation} is the generating of aggregates. \emph{Re-arising} is the occurrence of kamma-result stated thus, “In one who has attained [to it] or in one who has been reborn [in it]” (\textbf{\cite{Dhs}§1282}). \emph{Birth} is birth with becoming as its condition, itself a condition for ageing and so on. Ageing, sickness, death, etc., are obvious.

                    \vismParagraph{XXI.39}{39}{}
                    And here only the five beginning with arising are mentioned as actual objects of knowledge of danger; the rest are synonyms for them. For the pair, \emph{generation} and \emph{birth}, are synonyms both for \emph{arising} and for \emph{rebirth-linking}. The pair, \emph{destiny} and \emph{re-arising}, are synonyms for \emph{occurrence. Ageing}, etc., are synonyms for the sign. Hence it was said:
                    \begin{verse}
                        “He contemplates as suffering\\{}
                        Arising, occurrence, and the sign,\\{}
                        Accumulation, rebirth-linking—\\{}
                        And this his knowledge is of danger.”
                    \end{verse}


                    And:
                    \begin{verse}
                        “This knowledge about danger has\\{}
                        Five sources for its origin” (\hyperlink{XXI.37}{§37}{}).
                    \end{verse}


                    \vismParagraph{XXI.40}{40}{}
                    \emph{Knowledge of the state of peace is this: “Non-arising is safety},” etc.: this, however, should be understood as said for the purpose of showing the opposite kind of knowledge to knowledge of danger. Or when it is stated in this way, that there is safety without terror and free from danger, it is for the purpose of comforting those who are upset in their hearts by seeing danger through appearance as terror. Or else, when arising, etc., have clearly appeared to a man as terror, his mind inclines towards their opposites, and so this is said \textcolor{brown}{\textit{[650]}} for the purpose of showing the advantages in the knowledge of danger established by the appearance as terror.

                    \vismParagraph{XXI.41}{41}{}
                    And here (1.a.) what is terror is certainly (2.a) \emph{suffering}, and what is suffering is purely (3.a.) worldly since it is not free from the worldliness of the rounds [of becoming], of the world, and of defilements,\footnote{\vismAssertFootnoteCounter{13}\vismHypertarget{XXI.n13}{}\textbf{\cite{Vism-mhṭ}} defines the three kinds of worldliness (\emph{āmisa}) as follows: \emph{Worldliness of the round} (\emph{vaṭṭāmisa}) is that of the threefold round of past, future and present becoming; worldliness of the world (\emph{lokāmisa}) is the five cords of sense desire (i.e. objects of sense desire including food, etc.) because they are accessible to defilements; \emph{worldliness of defilement} (\emph{kilesāmisa}) is the defilements themselves (see \textbf{\cite{Vism-mhṭ}836}).} and what is worldly consists solely \marginnote{\textcolor{teal}{\footnotesize\{736|678\}}}{}of (4.a) \emph{formations}. Therefore it is said that (2.a) \emph{understanding of appearance as terror thus}, “Arising is suffering,” is knowledge of danger, and so on. And while this is so, still there is a difference to be understood here in the way these things [beginning with “arising”] occur, since there is a difference in their mode with the mode of terror, the mode of suffering, and the mode of worldliness.

                    \vismParagraph{XXI.42}{42}{}
                    \emph{Ten knowledges he understands}: one who understands knowledge of danger understands, penetrates, realizes, ten kinds of knowledge, that is, the five based on arising, etc., and the five on non-arising and so on. \emph{When skilled in these two kinds of knowledge}: with skill in the two, that is, knowledge of danger and knowledge of the state of peace. \emph{The various views will shake him not}: he does not vacillate about views that occur such as “The ultimate Nibbāna is here and now.” The rest is clear.

                    Knowledge of contemplation of danger is ended.
                \subsection[\vismAlignedParas{§43–44}5. Knowledge of Dispassion]{5. Knowledge of Dispassion}

                    \vismParagraph{XXI.43}{43}{}
                    When he sees all formations in this way as danger, he becomes dispassionate towards, is dissatisfied with, takes no delight in the manifold field of formations belonging to any kind of becoming, destiny, station of consciousness, or abode of beings. Just as a golden swan that loves the foothills of Citta Peak finds delight, not in a filthy puddle at the gate of a village of outcastes, but only in the seven great lakes (see \hyperlink{XIII.38}{XIII.38}{}), so too this meditator swan finds delight, not in the manifold formations seen clearly as danger, but only in the seven contemplations, because he delights in development. And just as the lion, king of beasts, finds delight, not when put into a gold cage, but only in Himalaya with its three thousand leagues’ extent, so too the meditator lion finds delight, not in the triple becoming of the happy destiny,\footnote{\vismAssertFootnoteCounter{14}\vismHypertarget{XXI.n14}{}The reference is to the happy destinies of the sense-desire world (human beings and deities), the fine-material Brahmā-world, and the immaterial Brahmā-world.} but only in the three contemplations. And just as Chaddanta, king of elephants, all white with sevenfold stance, possessed of supernormal power, who travels through the air,\footnote{\vismAssertFootnoteCounter{15}\vismHypertarget{XXI.n15}{}For “ten kinds of elephants” of which the Chaddanta (Six-toothed) is the “best” see \textbf{\cite{M-a}II 25}. Cf. also the description of the elephant called “Uposatha,” one of the seven treasures of the Wheel-turning Monarch (\textbf{\cite{M}II 173}). On the expression “with sevenfold stance” (\emph{sattappatiṭṭha}) \textbf{\cite{Vism-mhṭ}} says “\emph{Hatthapāda-vālavatthikosehi bhūmiphusanehi sattahi patiṭṭhito ti sattapatiṭṭho}” (\textbf{\cite{Vism-mhṭ}838}).} finds pleasure, not in the midst of a town, but only in the Chaddanta Lake and Wood in the Himalaya, \textcolor{brown}{\textit{[651]}} so too this meditator elephant finds delight, not in any formation, but only in the state of peace seen in the way beginning “Non-arising is safety,” and his mind tends, inclines, and leans towards that.

                    Knowledge of contemplation of dispassion is ended.

                    \vismParagraph{XXI.44}{44}{}
                    [Knowledge of contemplation of danger] is the same as the last two kinds of knowledge in meaning. Hence the Ancients said: “Knowledge of appearance as terror while one only has three names: It saw all formations as terror, thus the name ‘appearance as terror’ arose; it aroused the [appearance of] danger in those same formations, thus the name ‘contemplation of danger’ arose; it arose, \marginnote{\textcolor{teal}{\footnotesize\{737|679\}}}{}becoming dispassionate towards those same formations, thus the name ‘contemplation of dispassion’ arose.” Also it is said in the text: “Understanding of appearance as terror, knowledge of danger, and dispassion: these things are one in meaning, only the letter is different” (\textbf{\cite{Paṭis}II 63}).
                \subsection[\vismAlignedParas{§45–46}6. Knowledge of Desire for Deliverance]{6. Knowledge of Desire for Deliverance}

                    \vismParagraph{XXI.45}{45}{}
                    When, owing to this knowledge of dispassion, this clansman becomes dispassionate towards, is dissatisfied with, takes no delight in any single one of all the manifold formations in any kind of becoming, generation, destiny, station of consciousness, or abode of beings, his mind no longer sticks fast, cleaves, fastens on to them, and he becomes desirous of being delivered from the whole field of formations and escaping from it. Like what?

                    \vismParagraph{XXI.46}{46}{}
                    Just as a fish in a net, a frog in a snake’s jaws, a jungle fowl shut into a cage, a deer fallen into the clutches of a strong snare, a snake in the hands of a snake charmer, an elephant stuck fast in a great bog, a royal nāga in the mouth of a supaṇṇa, the moon inside Rāhu’s mouth,\footnote{\vismAssertFootnoteCounter{16}\vismHypertarget{XXI.n16}{}Rāhu is the name for the eclipse of the sun or moon, personalized as a demon who takes them in his mouth (see \textbf{\cite{S}I 50–51} and \textbf{\cite{M}I 87}).} a man encircled by enemies, etc.—just as these are desirous of being delivered, of finding an escape from these things, so too this meditator’s mind is desirous of being delivered from the whole field of formations and escaping from it. Then, when he thus no longer relies on any formations and is desirous of being delivered from the whole field of formations, knowledge of desire for deliverance arises in him.

                    Knowledge of desire for deliverance is ended.
                \subsection[\vismAlignedParas{§47–60}7. Knowledge of Reflection]{7. Knowledge of Reflection}

                    \vismParagraph{XXI.47}{47}{}
                    Being thus desirous of deliverance from all the manifold formations in any kind of becoming, generation, destiny, station, or abode, in order to be delivered from the whole field of formations \textcolor{brown}{\textit{[652]}} he again discerns those same formations, attributing to them the three characteristics by knowledge of contemplation of reflection.

                    \vismParagraph{XXI.48}{48}{}
                    He sees all formations as impermanent for the following reasons: because they are non-continuous, temporary, limited by rise and fall, disintegrating, fickle, perishable, unenduring, subject to change, coreless, due to be annihilated, formed, subject to death, and so on.

                    He sees them as painful for the following reasons: because they are continuously oppressed, hard to bear, the basis of pain, a disease, a tumour, a dart, a calamity, an affliction, a plague, a disaster, a terror, a menace, no protection, no shelter, no refuge, a danger, the root of calamity, murderous, subject to cankers, Māra’s bait, subject to birth, subject to ageing, subject to illness, subject to sorrow, subject to lamentation, subject to despair, subject to defilement, and so on.

                    He sees all formations as foul (ugly)—the ancillary characteristic to that of pain—for the following reasons: because they are objectionable, stinking, disgusting, repulsive, unaffected by disguise, hideous, loathsome, and so on. \marginnote{\textcolor{teal}{\footnotesize\{738|680\}}}{}He sees all formations as not-self for the following reasons: because they are alien, empty, vain, void, ownerless, with no Overlord, with none to wield power over them, and so on.

                    It is when he sees formations in this way that he is said to discern them by attributing to them the three characteristics.

                    \vismParagraph{XXI.49}{49}{}
                    But why does he discern them in this way? In order to contrive the means to deliverance. Here is a simile: a man thought to catch a fish, it seems, so he took a fishing net and cast it in the water. He put his hand into the mouth of the net under the water and seized a snake by the neck. He was glad, thinking, “I have caught a fish.” In the belief that he had caught a big fish, he lifted it up to see. When he saw three marks, he perceived that it was a snake and he was terrified. He saw danger, felt dispassion (revulsion) for what he had seized, and desired to be delivered from it. Contriving a means to deliverance, he unwrapped [the coils from] his hand, starting from the tip of its tail. Then he raised his arm, and when he had weakened the snake by swinging it two or three times round his head, he flung it away, crying “Go, foul snake.” Then quickly scrambling up on to dry land, he stood looking back whence he had come, thinking, “Goodness, I have been delivered from the jaws of a huge snake!”

                    \vismParagraph{XXI.50}{50}{}
                    Herein, the time when the meditator was glad at the outset to have acquired a person is like the time when the man was glad to have seized the snake by the neck. This meditator’s seeing the three characteristics in formations after effecting resolution of the compact [into elements] is like the man’s seeing the three marks on pulling the snake’s head out of the mouth of the net. \textcolor{brown}{\textit{[653]}} The meditator’s knowledge of appearance as terror is like the time when the man was frightened. Knowledge of contemplation of danger is like the man’s thereupon seeing the danger. Knowledge of contemplation of dispassion is like the man’s dispassion (revulsion) for what he had seized. Knowledge of desire for deliverance is like the man’s deliverance from the snake. The attribution of the three characteristics to formations by knowledge of contemplation of reflection is like the man’s contriving a means to deliverance. For just as the man weakened the snake by swinging it, keeping it away and rendering it incapable of biting, and was thus quite delivered, so too this meditator weakens formations by swinging them with the attribution of the three characteristics, rendering them incapable of appearing again in the modes of permanence, pleasure, beauty, and self, and is thus quite delivered. That is why it was said above that he discerns them in this way “in order to contrive the means to deliverance.”

                    \vismParagraph{XXI.51}{51}{}
                    At this point knowledge of reflection has arisen in him, with reference to which it is said: “When he brings to mind as impermanent, there arises in him knowledge after reflecting on what? When he brings to mind as painful, … as not-self, there arises in him knowledge after reflecting on what? When he brings to mind as impermanent, there arises in him knowledge after reflecting on the sign. When he brings to mind as painful, there arises in him knowledge after reflecting on occurrence. When he brings to mind as not-self, there arises in him knowledge after reflecting on the sign and occurrence” (\textbf{\cite{Paṭis}II 63}).

                    \vismParagraph{XXI.52}{52}{}
                    \marginnote{\textcolor{teal}{\footnotesize\{739|681\}}}{}As here \emph{after reflecting on the sign} [means] having known the sign of formations by means of the characteristic of impermanence as unlasting and temporary. Of course, it is not\footnote{\vismAssertFootnoteCounter{17}\vismHypertarget{XXI.n17}{}The sense seems to require a reading, “\emph{Kāmañ ca na paṭhamaṃ}”…} that, first having known, subsequently knowledge arises; but it is expressed in this way according to common usage, as in the passage beginning, “Due to (lit. having depended upon) mind and mental object, mind-consciousness arises” (\textbf{\cite{M}I 112}), and so on. Or alternatively, it can be understood as expressed thus according to the method of identity by identifying the preceding with the subsequent. The meaning of the remaining two expressions [that is, “occurrence” and “the sign and occurrence”] should be understood in the same way.

                    Knowledge of contemplation of reflection is ended.
                    \subsubsection[\vismAlignedParas{§53–60}Discerning Formations as Void]{Discerning Formations as Void}

                        \vismParagraph{XXI.53}{53}{}
                        Having thus discerned by knowledge of contemplation of reflection that “All formations are void” (see \textbf{\cite{S}III 167}), he again discerns voidness in the double logical relation\footnote{\vismAssertFootnoteCounter{18}\vismHypertarget{XXI.n18}{}\emph{Dvikoṭika} (“double logical relation”) and \emph{catukoṭika} (“quadruple logical relation”): Skr. \emph{catuḥkoṭi} (cf. \textbf{\cite{Th}}. Stcherbatsky, \emph{Buddhist Logi}c, pp. 60–61, note 5).} thus: “This is void of self or of what belongs to self” (\textbf{\cite{M}II 263}; \textbf{\cite{Paṭis}II 36}).

                        When he has thus seen that there is neither a self nor any other [thing or being] occupying the position of a self s property, he again discerns voidness in the quadruple logical relation as set forth in this \textcolor{brown}{\textit{[654]}} passage: “I am not anywhere anyone’s owning, nor is there anywhere my owning in anyone (\emph{nāhaṃ kvacani kassaci kiñcanat’ asmiṃ na ca mama kvacani kismiñci kiñcanat’ atthi})” (\textbf{\cite{M}II 263}).\footnote{\vismAssertFootnoteCounter{19}\vismHypertarget{XXI.n19}{}There are a number of variant readings to this sutta passage (which is met with elsewhere as follows: \textbf{\cite{A}I 206}; II 177; cf. III 170). There are also variant readings of the commentary, reproduced at \textbf{\cite{M-a}IV 63–65} and in the commentary to \textbf{\cite{A}II 177}. The readings adopted are those which a study of the various contexts has indicated. The passage is a difficult one.

                                The sutta passage seems from its various settings to have been a phrase current among non-Buddhists, as a sort of slogan for naked ascetics (\textbf{\cite{A}I 206}); and it is used to describe the base consisting of nothingness (\textbf{\cite{M}II 263}), in which latter sense it is incorporated in the Buddha’s teaching as a description that can be made the basis for right view or wrong view according as it is treated.

                                The commentarial interpretation given here is summed up by \textbf{\cite{Vism-mhṭ}} as follows: “‘\emph{Nāhaṃ kvacini}’: he sees the non-existence of a self of his own. ‘\emph{Na kassaci kiñcanat’asmiṃ}’: he sees of his own self too that it is not the property of another’s self. ‘\emph{Na ca mama}’: these words should be construed as indicated. ‘\emph{Atthi}’ applies to each clause. He sees the nonexistence of another’s self thus, ‘There is no other’s self anywhere.’ He sees of another that that other is not the property of his own self thus, ‘My owning of that other’s self does not exist.’ So this mere conglomeration of formations is seen, by discerning it with the voidness of the quadruple logical relation, as voidness of self or property of a self in both internal and external aggregates’” (\textbf{\cite{Vism-mhṭ}840–841} = \emph{ṭīkā }to MN 106).} How?

                        \vismParagraph{XXI.54}{54}{}
                        \marginnote{\textcolor{teal}{\footnotesize\{740|682\}}}{}(i) This [meditator, thinking] I … not anywhere (\emph{nāhaṃ kvacani}), does not see a self anywhere; (ii) [Thinking] am … anyone’s owning (\emph{kassaci kiñcanat’ asmiṃ}), he does not see a self of his own to be inferred in another’s owning; the meaning is that he does not see [a self of his own] deducible by conceiving a brother [to own it] in the case of a brother,\footnote{\vismAssertFootnoteCounter{20}\vismHypertarget{XXI.n20}{}\emph{Bhātiṭṭhāne—}“in the case of a brother”: the form \emph{bhāti} is not given in PED.} a friend [to own it] in the case of a friend, or a chattel [to own it] in the case of a chattel; (iii) [As regards the phrase] nor … anywhere my (\emph{na ca mama kvacani}), leaving aside the word my (\emph{mama}) here for the moment, [the words] nor anywhere (\emph{na ca kvacani}) [means that] he does not\footnote{\vismAssertFootnoteCounter{21}\vismHypertarget{XXI.n21}{}Reading “… \emph{ṭhapetvā na ca kvacini} (:) \emph{parassa ca attānaṃ kvaci na passatī ti ayaṃ attho; idāni} …” with Ce of \textbf{\cite{M-a}} and \textbf{\cite{A-a}}.} see another’s self anywhere; (iv) Now, bringing in the word my (\emph{mama}), [we have] is there … my owning in anyone (\emph{mama kismiñci kiñcanat’ atthi}): he does not see thus, “Another’s self exists owing to some state of my owning\footnote{\vismAssertFootnoteCounter{22}\vismHypertarget{XXI.n22}{}\textbf{\cite{M-a}} Sinhalese (Aluvihāra) ed. has \emph{kiñcanabhāvena} here instead of \emph{kiñcana-bhāve}.} [of it]”; the meaning is that he does not see in any instance another’s self deducible owing to this fact of his owning a brother in the case of a brother, a friend in the case of a friend, chattel in the case of a chattel. So (i) he sees no self anywhere [of his own]; (ii) nor does he see it as deducible in the fact of another’s owning; (iii) nor does he see another’s self; (iv) nor does he see that as deducible in the fact of his own owning.\footnote{\vismAssertFootnoteCounter{23}\vismHypertarget{XXI.n23}{}Sinhalese eds. of \textbf{\cite{M-a}} and \textbf{\cite{A-a}} both read here: “… \emph{upanetabbaṃ passati, na parassa attānaṃ passati, na parassa attano kiñcanabhāve upanetabbaṃ passati},” which the sense demands.} This is how he discerns voidness in the quadruple logical relation.

                        \vismParagraph{XXI.55}{55}{}
                        Having discerned voidness in the quadruple logical relation in this way, he discerns voidness again in six modes. How? Eye (i) is void of self, (ii) or of the property of a self, (iii) or of permanence, (iv) or of lastingness, (v) or of eternalness, (vi) or of non-subjectness to change; … mind … visible data … mental data … eye-consciousness … mind-consciousness … mind-contact … (\textbf{\cite{Nidd}II 187} (Se); \textbf{\cite{Nidd}II 279} (Ee); cf. \textbf{\cite{S}IV 54}) and this should be continued as far as ageing-and-death (see \hyperlink{XX.9}{XX.9}{}).

                        \vismParagraph{XXI.56}{56}{}
                        Having discerned voidness in the six modes in this way, he discerns it again in eight modes, that is to say: “Materiality has no core, is coreless, without core, as far as concerns (i) any core of permanence, or (ii) core of lastingness, or (iii) core of pleasure, or (iv) core of self, or as far as concerns (v) what is permanent, or (vi) what is lasting, or (vii) what is eternal, or (viii) what is not subject to change. Feeling … perception … formations … consciousness … eye … (etc., see \hyperlink{XX.9}{XX.9}{}) … ageing-and-death has no core, is coreless, without a core, as far as concerns any core of permanence, or core of lastingness, or core of pleasure, or core of self, or as far as concerns what is permanent, or what is lasting, or what is eternal, or what is not subject to change. Just as a reed has no core, is coreless, without core; just as a castor-oil plant, an \emph{udumbara} (fig) tree, a \emph{setavaccha} tree, a \emph{pāḷibhaddaka} tree, a lump of froth, a bubble on water, a mirage, a plantain trunk, \textcolor{brown}{\textit{[655]}} a conjuring trick, has no core, is coreless, without core, so too materiality … \marginnote{\textcolor{teal}{\footnotesize\{741|683\}}}{}(etc) … ageing-and-death has no core … or what is subject to change” (\textbf{\cite{Nidd}II 184–185} (Se); \textbf{\cite{Nidd}II 278–289} (Ee)).

                        \vismParagraph{XXI.57}{57}{}
                        Having discerned voidness in eight modes in this way, he discerns it again in ten modes. How? “He sees materiality as empty, as vain, as not-self, as having no Overlord, as incapable of being made into what one wants, as incapable of being had [as one wishes], as insusceptible to the exercise of mastery, as alien, as secluded [from past and future]. He sees feeling … (etc.) … consciousness as empty, … as secluded”\footnote{\vismAssertFootnoteCounter{24}\vismHypertarget{XXI.n24}{}The cause and the fruit being secluded from each other (see \textbf{\cite{Vism-mhṭ}842}).} (\textbf{\cite{Nidd}II 279} (Ee)).

                        \vismParagraph{XXI.58}{58}{}
                        Having discerned voidness in ten modes in this way, he discerns it again in twelve modes, that is to say: “Materiality is no living being,\footnote{\vismAssertFootnoteCounter{25}\vismHypertarget{XXI.n25}{}“A meaning such as ‘what in common usage in the world is called a being is not materiality’ is not intended here because it is not implied by what is said; for the common usage of the world does not speak of mere materiality as a being. What is intended as a being is the self that is conjectured by outsiders” (\textbf{\cite{Vism-mhṭ}842}).} no soul, no human being, no man, no female, no male, no self, no property of a self, not I, not mine, not another’s, not anyone’s. Feeling … (etc.) … consciousness … not anyone’s (\textbf{\cite{Nidd}II 186} (Se); \textbf{\cite{Nidd}II 280} (Ee)).

                        \vismParagraph{XXI.59}{59}{}
                        Having discerned voidness in twelve modes in this way, he discerns it again in forty-two modes through full-understanding as investigating. He sees materiality as impermanent, as painful, as a disease, a tumour, a dart, a calamity, an affliction, as alien, as disintegrating, a plague, a disaster, a terror, a menace, as fickle, perishable, unenduring, as no protection, no shelter, no refuge, as unfit to be a refuge, as empty, vain, void, not-self, as without satisfaction,\footnote{\vismAssertFootnoteCounter{26}\vismHypertarget{XXI.n26}{}“This is not in the text. If it were there would be forty-three ways” (\textbf{\cite{Vism-mhṭ}842}).} as a danger, as subject to change, as having no core, as the root of calamity, as murderous, as due to be annihilated, as subject to cankers, as formed, as Māra’s bait, as subject to birth, subject to ageing, subject to illness, subject to death, subject to sorrow, lamentation, pain, grief and despair; as arising, as departing; as danger,\footnote{\vismAssertFootnoteCounter{27}\vismHypertarget{XXI.n27}{}“Although it has already been described as a danger in order to show it as such, the word is used again in order to show that it is opposed to enjoyment (satisfaction)” (\textbf{\cite{Vism-mhṭ}843}).} as (having an) escape. He sees feeling … (etc.) … consciousness … as (having an) escape (cf. \textbf{\cite{Paṭis}II 238}).

                        \vismParagraph{XXI.60}{60}{}
                        And this is said too:\footnote{\vismAssertFootnoteCounter{28}\vismHypertarget{XXI.n28}{}Vism-mhṭ (p. 843) seems to suggest that this is quoted from the Niddesa, but it is not in Nidd II in this form. Cf. \textbf{\cite{Nidd}II 162} (Be): \emph{Atha vā, vedanaṃ aniccato … dukkhato rogato gaṇḍato sallato aghato ābādhato … pe … nissaraṇato passanto vedanaṃ nābhinandati …}} “When he sees materiality as impermanent … as (having an) escape, he looks upon the world as void. When he sees feeling … (etc.) … consciousness as impermanent … as (having an) escape, he looks upon the world as void.” \textcolor{brown}{\textit{[656]}}
                        \begin{verse}
                            “Let him look on the world as void:\\{}
                            Thus, Mogharāja, always mindful,\\{}
                            He may escape the clutch of death
                        \end{verse}

                        \begin{verse}
                            \marginnote{\textcolor{teal}{\footnotesize\{742|684\}}}{}By giving up belief in self.\\{}
                            For King Death cannot see the man\\{}
                            That looks in this way on the world”\footnote{\vismAssertFootnoteCounter{29}\vismHypertarget{XXI.n29}{}Sn 1119: \textbf{\cite{Nidd}II 190} (Se); \textbf{\cite{Nidd}II 278} (Ee)}
                        \end{verse}

                \subsection[\vismAlignedParas{§61–127}8. Knowledge of Equanimity about Formations]{8. Knowledge of Equanimity about Formations}

                    \vismParagraph{XXI.61}{61}{}
                    When he has discerned formations by attributing the three characteristics to them and seeing them as void in this way, he abandons both terror and delight, he becomes indifferent to them and neutral, he neither takes them as “I” nor as mine,” he is like a man who has divorced his wife.

                    \vismParagraph{XXI.62}{62}{}
                    Suppose a man were married to a lovely, desirable, charming wife and so deeply in love with her as to be unable to bear separation from her for a moment. He would be disturbed and displeased to see her standing or sitting or talking or laughing with another man, and would be very unhappy; but later, when he had found out the woman’s faults, and wanting to get free, had divorced her, he would no more take her as “mine”; and thereafter, even though he saw her doing whatever it might be with whomsoever it might be, he would not be disturbed or displeased, but would on the contrary be indifferent and neutral. So too this [meditator], wanting to get free from all formations, discerns formations by the contemplation of reflection; then, seeing nothing to be taken as “I” or “mine,” he abandons both terror and delight and becomes indifferent and neutral towards all formations.

                    \vismParagraph{XXI.63}{63}{}
                    When he knows and sees thus, his heart retreats, retracts and recoils from the three kinds of becoming, the four kinds of generation, the five kinds of destiny, the seven stations of consciousness, and the nine abodes of beings; his heart no longer goes out to them. Either equanimity or repulsiveness is established. Just as water drops retreat, retract and recoil on a lotus leaf that slopes a little and do not spread out, so too his heart … And just as a fowl’s feather or a shred of sinew thrown on a fire retreats, retracts and recoils, and does not spread out, so too his heart retreats, retracts and recoils from the three kinds of becoming … Either equanimity or repulsiveness is established.

                    In this way there arises in him what is called knowledge of equanimity about formations.

                    \vismParagraph{XXI.64}{64}{}
                    But if this [knowledge] sees Nibbāna, the state of peace, as peaceful, it rejects the occurrence of all formations and enters only into Nibbāna. If it does not see Nibbāna as peaceful, \textcolor{brown}{\textit{[657]}} it occurs again and again with formations as its object, like the sailors’ crow.

                    \vismParagraph{XXI.65}{65}{}
                    When traders board a ship, it seems, they take with them what is called a land-finding crow. When the ship gets blown off its course by gales and goes adrift with no land in sight, then they release the land-finding crow. It takes off from the mast-head,\footnote{\vismAssertFootnoteCounter{30}\vismHypertarget{XXI.n30}{}\emph{Kūpaka-yaṭṭhi—}“mast-head” (?): the word \emph{kūpaka} appears in PED, only as an equivalent for \emph{kūpa} = a hole. Cf. \textbf{\cite{D}I 222} for this simile.} and after exploring all the quarters, if it sees land, it flies straight in the direction of it; if not, it returns and alights on the mast-head. So \marginnote{\textcolor{teal}{\footnotesize\{743|685\}}}{}too, if knowledge of equanimity about formations sees Nibbāna, the state of peace, as peaceful, it rejects the occurrence of all formations and enters only into Nibbāna. If it does not see it, it occurs again and again with formations as its object.

                    \vismParagraph{XXI.66}{66}{}
                    Now, after discerning formations in the various modes, as though sifting flour on the edge of a tray, as though carding cotton from which the seeds have been picked out,\footnote{\vismAssertFootnoteCounter{31}\vismHypertarget{XXI.n31}{}\emph{Vaṭṭayamāna—}“sifting”: not in PED; \textbf{\cite{Vism-mhṭ}} glosses with \emph{niccoriyamāna}, also not in PED. \emph{Nibbaṭṭita—}“picked out”: not in PED. \textbf{\cite{Vism-mhṭ}} glosses \emph{nibbaṭṭita-kappāsaṃ }with \emph{nibaṭṭita-bīja-kappāsaṃ.” Vihaṭamāna—}“carding”: not in PED; glossed by Vism-mhṭ with \emph{dhūnakena} (not in PED) \emph{vihaññamānaṃ viya} (\textbf{\cite{Vism-mhṭ}844}).} and after abandoning terror and delight, and after becoming neutral in the investigation of formations, he still persists in the triple contemplation. And in so doing, this [insight knowledge] enters upon the state of the triple gateway to liberation, and it becomes a condition for the classification of noble persons into seven kinds.
                    \subsubsection[\vismAlignedParas{§66–73}The Triple Gateway to Liberation]{The Triple Gateway to Liberation}

                        It enters upon the state of the triple gateway to liberation now with the predominance of [one of] three faculties according as the contemplation occurs in [one of] the three ways.\footnote{\vismAssertFootnoteCounter{32}\vismHypertarget{XXI.n32}{}When insight reaches its culmination, it settles down in one of the three contemplations [impermanence, pain, or not-self] and at this stage of the development the “seven contemplations” and the “eighteen contemplations” (or “principal insights”) are all included by the three (see \textbf{\cite{Vism-mhṭ}844}).}

                        \vismParagraph{XXI.67}{67}{}
                        For it is the three contemplations that are called the three gateways to liberation, according as it is said: “But these three gateways to liberation lead to the outlet from the world, [that is to say,] (i) to the seeing of all formations as limited and circumscribed and to the entering of consciousness into the signless element, (ii) to the stirring up of the mind with respect to all formations and to the entering of consciousness into the desireless element, (iii) to the seeing of all things (\emph{dhamma}) as alien and to the entering of consciousness into the voidness element. These three gateways to liberation lead to the outlet from the world” (\textbf{\cite{Paṭis}II 48}).\footnote{\vismAssertFootnoteCounter{33}\vismHypertarget{XXI.n33}{}“Contemplation of impermanence sees formations as limited by rise in the beginning and by fall in the end, and it sees that it is because they have a beginning and an end that they are impermanent. ‘\emph{Into the signless element}’: into the unformed element, which is given the name ‘signless’ because it is the opposite of the sign of formations. ‘\emph{To the entering of consciousness}’: to the higher consciousness’s completely going into by means of the state of conformity knowledge, after delimiting. ‘\emph{Into the desireless}’: into the unformed element, which is given the name ‘desireless’ owing to the non-existence of desire due to greed and so on. ‘\emph{Into the void}’: into the unformed element, which is given the name ‘void’ because of voidness of self” (\textbf{\cite{Vism-mhṭ}845}).}

                        \vismParagraph{XXI.68}{68}{}
                        \marginnote{\textcolor{teal}{\footnotesize\{744|686\}}}{}Herein, \emph{as limited and circumscribed} [means] both as limited by rise and fall and as circumscribed by them; for contemplation of impermanence limits them thus, “Formations do not exist previous to their rise,” and in seeking their destiny, sees them as circumscribed thus, “They do not go beyond fall, they vanish there.” \emph{To the stirring up of the mind}: by giving consciousness a sense of urgency; for with the contemplation of pain consciousness acquires a sense of urgency with respect to formations. \textcolor{brown}{\textit{[658]}} \emph{To the seeing … as alien}: to contemplating them as not-self thus: “Not I,” “Not mine.”

                        \vismParagraph{XXI.69}{69}{}
                        So these three clauses should be understood to express the contemplations of impermanence, and so on. Hence in the answer to the next question [asked in the Paṭisambhidā] it is said: “When he brings [them] to mind as impermanent, formations appear as liable to destruction. When he brings them to mind as painful, formations appear as a terror. When he brings them to mind as not-self, formations appear as void” (\textbf{\cite{Paṭis}II 48}).

                        \vismParagraph{XXI.70}{70}{}
                        What are the liberations to which these contemplations are the gateways? They are these three, namely, the signless, the desireless, and the void. For this is said: “When one who has great resolution brings [formations] to mind as impermanent, he acquires the signless liberation. When one who has great tranquillity brings [them] to mind as painful, he acquires the desireless liberation. When one who has great wisdom brings [them] to mind as not-self, he acquires the void liberation” (\textbf{\cite{Paṭis}II 58}).

                        \vismParagraph{XXI.71}{71}{}
                        And here the \emph{signless liberation} should be understood as the noble path that has occurred by making Nibbāna its object through the signless aspect. For that path is signless owing to the signless element having arisen, and it is a liberation owing to deliverance from defilements.\footnote{\vismAssertFootnoteCounter{34}\vismHypertarget{XXI.n34}{}“One who is pursuing insight by discerning formations according to their sign by means of the contemplation of impermanence and resolves according to the signless aspect thus, ‘Where this sign of formations is entirely nonexistent, that is, the signless Nibbāna’ joins insight leading to emergence with the path. Then the path realizes Nibbāna for him as signless. The signless aspect of Nibbāna is not created by the path or by insight; on the contrary, it is the establishment of the individual essence of

                                Nibbāna, and the path is called signless because it has that as its object. One who resolves upon the desireless by keeping desire away by means of the contemplation of pain, and one who resolves upon the void by keeping the belief in self away by means of the contemplation of not-self, should both be construed in the same way” (\textbf{\cite{Vism-mhṭ}846}).} In the same way the path that has occurred by making Nibbāna its object through the desireless aspect is \emph{desireless}. And the path that has occurred by making Nibbāna its object through the void aspect is void.

                        \vismParagraph{XXI.72}{72}{}
                        But it is said in the Abhidhamma: “On the occasion when he develops the supramundane jhāna that is an outlet and leads to dispersal, having abandoned the field of [false] views with the reaching of the first grade, secluded from sense desires he enters upon and dwells in the first jhāna, which is desireless … is void,” (\textbf{\cite{Dhs}§510}) thus mentioning only two liberations. This refers to the way in which insight arrives [at the path] and is expressed literally.

                        \vismParagraph{XXI.73}{73}{}
                        However, in the Paṭisambhidā insight knowledge is expressed as follows: (i) It is expressed firstly as the void liberation by its liberating from misinterpreting [formations]: “Knowledge of contemplation of impermanence is the void liberation since it liberates from interpreting [them] as permanent; knowledge of contemplation of pain is the void liberation since it liberates from interpreting \marginnote{\textcolor{teal}{\footnotesize\{745|687\}}}{}[them] as pleasant; knowledge of contemplation of not-self is the void liberation since it liberates from interpreting [them] as self” (\textbf{\cite{Paṭis}II 67}). (ii) Then it is expressed as the signless liberation by liberating from signs: “Knowledge of contemplation of impermanence is the signless liberation since it liberates from the sign [of formations] as permanent; knowledge of contemplation of pain is the signless liberation since it liberates from the sign [of formations] as pleasant; knowledge of contemplation of not-self is the signless liberation since it liberates from the sign [of formations] as self” (\textbf{\cite{Paṭis}II 68}). \textcolor{brown}{\textit{[659]}} (iii) Lastly it is expressed as the desireless liberation by its liberating from desire: “Knowledge of contemplation of impermanence is the desireless liberation since it liberates from desire [for formations] as permanent; knowledge of contemplation of pain is the desireless liberation since it liberates from the desire [for them] as pleasant; knowledge of contemplation of not-self is the desireless liberation since it liberates from the desire [for them] as self” (\textbf{\cite{Paṭis}II 68}). But although stated in this way, insight knowledge is not literally signless because there is no abandoning of the sign of formations [as formed, here, as distinct from their sign as impermanent and so on]. It is however literally void and desireless. And it is at the moment of the noble path that the liberation is distinguished, and that is done according to insight knowledge’s way of arrival at the path.\footnote{\vismAssertFootnoteCounter{35}\vismHypertarget{XXI.n35}{}“Why is signless insight unable to give its own name to the path when it has come to the point of arrival at the path? Of course, signless insight is mentioned in the suttas thus, ‘Develop the signless and get rid of the inherent tendency to conceit’ (\textbf{\cite{Sn}342}). Nevertheless, though it eliminates the signs of permanence, of lastingness, and of self, it still possesses a sign itself and is occupied with states that possess a sign. Again, the Abhidhamma is the teaching in the ultimate sense, and in the ultimate sense the cause of a signless path is wanting. For the signless liberation is stated in accordance with the contemplation of impermanence, and in that the faith faculty predominates. But the faith faculty is not represented by any one of the factors of the path. And so it cannot give its name to the path since it forms no part of it. In the case of the other two, the desireless liberation is due to the contemplation of pain, and the void liberation is due to the contemplation of not-self. Now the concentration faculty predominates in the desireless liberation and the understanding faculty in the void liberation. So since these are factors of the path as well, they can give their own names to the path; but there is no signless path because the factor is wanting. So some say. But there are others who say that there is a signless path, and that although it does not get its name from the way insight arrives at it, still it gets its name from a special quality of its own and from its object. In their opinion the desireless and void paths should also get their names from special qualities of their own and from their objects too. That is wrong. Why? Because the path gets its names for two reasons, that is, because of its own nature and because of what it opposes—the meaning is, because of its individual essence and because of what it is contrary to. For the desireless path is free from desire due to greed, etc., and the void path is free from greed too, so they both get their names from their individual essence. Similarly, the desireless path is the contrary of desire and the void path is the contrary of misinterpretation as self, so they get their names from what they oppose. On the other hand, the signless path gets its name only from its own nature owing to the non-existence in it of the signs of greed, etc., or of the signs of permanence, etc., but not owing to what it opposes. For it does not oppose the contemplation of impermanence, which has as its object the sign of formations [as formed], but remains in agreement with it. So a signless path is altogether inadmissible by the Abhidhamma method. This is why it is said, ‘This refers to the way in which insight arrives at the path and is expressed in the literal sense’ (§72).

                                “However, by the Suttanta method a signless path is admissible. For according to that, in whatever way insight leading to emergence (see §83) effects its comprehending it still leads on to emergence of the path, and when it is at the point of arrival it gives its own name to the path accordingly—when emerging owing to comprehension as impermanent the path is signless, when emerging owing to comprehension as painful it is desireless, and when emerging owing to comprehension as not-self it is void. Taking this as a sutta commentary, therefore, three liberations are differentiated here. But in the Paṭisambhidā the deliverance from misinterpreting, from the sign and from desire, are taken respectively as the arrival of the three kinds of comprehension at that deliverance, and what is described is a corresponding state of void liberation, etc., respectively in the paths that follow upon that deliverance. There is no question of treating that literally, which is why he said, ‘However, in the Paṭisambhidā insight knowledge’ and so on” (\textbf{\cite{Vism-mhṭ}846–848}).} That, it \marginnote{\textcolor{teal}{\footnotesize\{746|688\}}}{}should be understood, is why only two liberations are stated [in the Abhidhamma], namely, the desireless and the void.

                        This, firstly, is the treatise on the liberations here.
                    \subsubsection[\vismAlignedParas{§74–78}The Seven Kinds of Noble Persons]{The Seven Kinds of Noble Persons}

                        \vismParagraph{XXI.74}{74}{}
                        It was said above, “It becomes a condition for the classification of noble persons into seven kinds.” (\hyperlink{XXI.66}{§66}{}) Herein, there are firstly these seven kinds of noble person: (1) the faith devotee, (2) one liberated by faith, (3) the body witness, (4) the both-ways liberated, (5) the Dhamma devotee, (6) one attained to vision, and (7) one liberated by understanding. This knowledge of equanimity about formations is a condition for their being placed as these seven classes.

                        \vismParagraph{XXI.75}{75}{}
                        When a man brings [formations] to mind as impermanent and, having great resolution, acquires the faith faculty, (1) he becomes a faith devotee at the moment of the stream-entry path; and in the other seven instances [that is, in the three higher paths and the four fruitions] he becomes (2) one liberated by faith. When a man brings [them] to mind as painful and, having great tranquillity, acquires the faculty of concentration, (3) he is called a body witness in all eight instances. (4) He is called both-ways liberated when he has reached the highest fruition after also reaching the immaterial jhānas. When a man brings [them] to mind as not-self and, having great wisdom, acquires the faculty of understanding, he becomes (5) a Dhamma devotee at the moment of the stream-entry path; (6) in the next six instances he becomes one attained to vision; and (7) in the case of the highest fruition he becomes one liberated by understanding.

                        \vismParagraph{XXI.76}{76}{}
                        (1) This is said: “When he brings [formations] to mind as impermanent, the faith faculty is in excess in him. With the faith faculty in excess he acquires the stream-entry path. Hence he is called a ’faith devotee’” (\textbf{\cite{Paṭis}II 53}). \textcolor{brown}{\textit{[660]}} Likewise, (2) “When he brings [formations] to mind as impermanent, the faith faculty is in excess in him. With the faith faculty in excess the fruition of stream-entry is realized. Hence he is called ‘one liberated by faith’” (\textbf{\cite{Paṭis}II 53}).

                        \vismParagraph{XXI.77}{77}{}
                        \marginnote{\textcolor{teal}{\footnotesize\{747|689\}}}{}It is said further: “[At the moment of the first path:] (2) he has been liberated by having faith (\emph{saddahanto vimutto)}, thus he is one liberated by faith; (3) he has realized [Nibbāna] by experiencing, thus he is a body witness; (6) he has attained [Nibbāna] by vision, thus he is one attained to vision. [At the moments of the three remaining paths:] (2) he is liberated by faith (\emph{saddahanto vimuccati}), thus he is one liberated by faith; (3) he first experiences the experience of jhāna and afterwards realizes cessation, Nibbāna, thus he is a body witness; (6) it is known, seen, recognized, realized, and experienced with understanding, that formations are painful and cessation is bliss, thus he is one attained to vision” (\textbf{\cite{Paṭis}II 52}).

                        \vismParagraph{XXI.78}{78}{}
                        As to the remaining four, however, the word meaning should be understood thus: (1) he follows (\emph{anusarati}) faith, thus he is a faith devotee (\emph{saddhānusāri}); or he follows, he goes, by means of faith, thus he is a faith devotee. (5) Likewise, he follows the Dhamma called understanding, or he follows by means of the Dhamma, thus he is a Dhamma devotee. (4) He is liberated in both ways, by immaterial jhāna and the noble path, thus he is both-ways liberated. (7) Understanding, he is liberated, thus he is one liberated by understanding.
                    \subsubsection[\vismAlignedParas{§79–82}The Last Three Knowledges are One]{The Last Three Knowledges are One}

                        \vismParagraph{XXI.79}{79}{}
                        This [knowledge of equanimity about formations] is the same in meaning as the two kinds that precede it. Hence the Ancients said: “This knowledge of equanimity about formations is one only and has three names. At the outset it has the name of knowledge of desire for deliverance. In the middle it has the name knowledge of reflection. At the end, when it has reached its culmination, it is called knowledge of equanimity about formations.”

                        \vismParagraph{XXI.80}{80}{}
                        “How is it that understanding of desire for deliverance, of reflection, and of composure is knowledge of the kinds of equanimity about formations? Understanding of desire for deliverance, of reflection, and composure [occupied with] arising is knowledge of equanimity about formations. Understanding of desire for deliverance, of reflection, and of composure [occupied with] occurrence … the sign … (etc., see \hyperlink{XXI.37}{§37}{}) … with despair is knowledge of equanimity about formations” (\textbf{\cite{Paṭis}I 60f.}).

                        \vismParagraph{XXI.81}{81}{}
                        Herein, the compound \emph{muñcitukamyatā-paṭisaṅkhā-santiṭṭhanā} (“consisting in desire for deliverance, in reflection, and in composure”) should be resolved into\emph{ muñcitukamyatā ca sā paṭisaṅkhā ca santiṭṭhanā ca}. So \textcolor{brown}{\textit{[661]}} in the first stage it is desire to give up, the desire to be delivered from, arising, etc., in one who has become dispassionate by knowledge of dispassion that is \emph{desire for deliverance}. It is reflection in the middle stage for the purpose of finding a means to deliverance that is \emph{reflection}. It is equanimous onlooking in the end stage on being delivered that is \emph{composure}. It is said with reference to this: “Arising is formations; he looks with equanimity on those formations; thus it is equanimity about formations” (\textbf{\cite{Paṭis}I 61}), and so on. So this is only one kind of knowledge.

                        \vismParagraph{XXI.82}{82}{}
                        Furthermore, it may be understood that this is so from the following text; for this is said: “Desire for deliverance, and contemplation of reflection, and equanimity about formations: these things are one in meaning and only the letter is different” (\textbf{\cite{Paṭis}II 64}).
                    \subsubsection[\vismAlignedParas{§83–89}Insight Leading to Emergence]{Insight Leading to Emergence}

                        \vismParagraph{XXI.83}{83}{}
                        \marginnote{\textcolor{teal}{\footnotesize\{748|690\}}}{}Now, when this clansman has reached equanimity about formations thus, his insight has reached its culmination and leads to emergence. “Insight that has reached culmination” or “insight leading to emergence” are names for the three kinds of knowledge beginning with equanimity about formations, [that is, equanimity about formations, conformity, and change-of-lineage]. It has “reached its culmination” because it has reached the culminating final stage. It is called “leading to emergence” because it goes towards emergence. The path is called “emergence” because it emerges externally from the objective basis interpreted as a sign and also internally from occurrence [of defilement].\footnote{\vismAssertFootnoteCounter{36}\vismHypertarget{XXI.n36}{}“‘\emph{From the object interpreted as the sign}’: from the pentad of aggregates as the object of insight; for that pentad of aggregates is called the ‘object interpreted’ on account of the interpreting, in other words, on account of being made the domain of insight. And although it is included in one’s own continuity, it is nevertheless called ‘external’ because it is seen as alien to it; it is that too which in other contexts is spoken of as ‘externally from all signs’ (\textbf{\cite{Paṭis}I 71}). ‘Internally from occurrence’: from the occurrence of wrong view, etc., in one’s own continuity, and from the defilements and from the aggregates that occur consequent upon them. For it is stated in this way because there is occurrence of defilement in one’s own continuity and because there is occurrence of clung-to aggregates produced by that [defilement] when there is no path development. And emergence consists both in making these the object and in producing their non-liability to future arising” (\textbf{\cite{Vism-mhṭ}853}).} It goes to that, thus it leads to emergence; the meaning is that it joins with the path.

                        \vismParagraph{XXI.84}{84}{}
                        Herein, for the purpose of clarification there is this list of the kinds of emergence classed according to the manner of interpreting: (1) after interpreting the internal\footnote{\vismAssertFootnoteCounter{37}\vismHypertarget{XXI.n37}{}“‘\emph{Emerges from the internal}’ is said figuratively owing to the fact that in this case the insight leading to emergence has an internal state as its object. In the literal sense, however, the path emerges from both” (\textbf{\cite{Vism-mhṭ}853}).} it emerges from the internal, (2) after interpreting the internal it emerges from the external, (3) after interpreting the external it emerges from the external, (4) after interpreting the external it emerges from the internal; (5) after interpreting the material it emerges from the material, (6) after interpreting the material it emerges from the immaterial, (7) after interpreting the immaterial it emerges from the immaterial, (8) after interpreting the immaterial it emerges from the material; (9) it emerges at one stroke from the five aggregates; (10) after interpreting as impermanent it emerges from the impermanent, (11) after interpreting as impermanent it emerges from the painful, (12) after interpreting as impermanent it emerges from the not-self; (13) after interpreting as painful it emerges from the painful, (14) after interpreting as painful it emerges from the impermanent, (15) after interpreting as painful it emerges from the not-self, (16) after interpreting as not-self it emerges from the not-self, (17) after interpreting as not-self it emerges from the impermanent, (18) after interpreting as not-self it emerges from the painful. How?

                        \vismParagraph{XXI.85}{85}{}
                        \marginnote{\textcolor{teal}{\footnotesize\{749|691\}}}{}Here (1) someone does his interpreting at the start with his own internal formations. After interpreting them he sees them. But emergence of the path does not come about through seeing the bare internal only since the external must be seen too, so he sees that another’s aggregates, as well as unclung-to formations [inanimate things], are impermanent, painful, not-self. At one time \textcolor{brown}{\textit{[662]}} he comprehends the internal and at another time the external. As he does so, insight joins with the path while he is comprehending the internal. It is said of him that “after interpreting the internal it emerges from the internal.” (2) If his insight joins with the path at the time when he is comprehending the external, it is said of him that “after interpreting the internal it emerges from the external.” (3) Similarly in the case of “after interpreting the external it emerges from the external,” and (4) “from the internal.”

                        \vismParagraph{XXI.86}{86}{}
                        (5) Another does his interpreting at the start with materiality. When he has done that, he sees the materiality of the primaries and the materiality derived from them all together. But emergence does not come about through the seeing of bare materiality only since the immaterial must be seen too, so he sees as the immaterial [mentality] the feeling, perception, formations and consciousness that have arisen by making that materiality their object. At one time he comprehends the material and at another the immaterial. As he does so, insight joins with the path while he is comprehending materiality. It is said of him that “after interpreting the material it emerges from the material.” (6) But if his insight joins with the path at the time when he is comprehending the immaterial, it is said of him that “after interpreting the material it emerges from the immaterial.” (7) Similarly in the case of “after interpreting the immaterial it emerges from the immaterial,” and (8) “from the material.”

                        \vismParagraph{XXI.87}{87}{}
                        (9) When he has done his interpreting in this way, “All that is subject to arising is subject to cessation” (\textbf{\cite{M}I 380}), and so too at the time of emergence, it is said that “it emerges at one stroke from the five aggregates.”

                        \vismParagraph{XXI.88}{88}{}
                        (10) One man comprehends formations as impermanent at the start. But emergence does not come about through mere comprehending as impermanent since there must be comprehension of them as painful and not-self too, so he comprehends them as painful and not-self. As he does so, emergence comes about while he is comprehending them as impermanent. It is said of him that “after interpreting as impermanent it emerges from the impermanent,” (11)–(12) But if emergence comes about in him while he is comprehending them as painful … as not-self, then it is said that “after interpreting as impermanent it emerges from the painful … from the not-self.” Similarly in the cases of emergence after interpreting (13)–(15) as painful and (16)–(18) as not-self.

                        \vismParagraph{XXI.89}{89}{}
                        And whether they have interpreted [at the start] as impermanent or as painful or as not-self, when the time of emergence comes, if the emergence takes place [while contemplating] as impermanent, then all three persons acquire the faculty of faith since they have great resolution; they are liberated by the signless liberation, and so they become faith devotees at the moment of the first path; and in the remaining seven stages they are liberated by faith. \textcolor{brown}{\textit{[663]}} If the emergence is from the painful, then the three persons acquire the faculty of concentration \marginnote{\textcolor{teal}{\footnotesize\{750|692\}}}{}since they have great tranquillity; they are liberated by the desireless liberation, and in all eight states they are body witnesses. Of them, the one who has an immaterial jhāna as the basis for his insight is, in the case of the highest fruition, both-ways liberated. And then if the emergence takes place [while contemplating] as not-self, the three persons acquire the faculty of understanding since they have great wisdom; they are liberated by the void liberation. They become Dhamma devotees at the moment of the first path. In the next six instances they become attained to vision. In the case of the highest fruit they are liberated by understanding.
                    \subsubsection[\vismAlignedParas{§90–110}The Twelve Similes]{The Twelve Similes}

                        \vismParagraph{XXI.90}{90}{}
                        Now, twelve similes should be understood in order to explain this insight leading to emergence and the kinds of knowledge that precede and follow it. Here is the list:
                        \begin{verse}
                            (1) The fruit bat, (2) the black snake, and (3) the house,\\{}
                            (4) The oxen, and(5) the ghoul, (6) the child,\\{}
                            (7) Hunger, and (8) thirst, and (9) cold, and (10) heat,\\{}
                            And (11) darkness, and (12) by poison, too.
                        \end{verse}


                        A pause can be made to bring in these similes anywhere among the kinds of knowledge from appearance as terror onwards. But if they are brought in here, then all becomes clear from appearance as terror up to fruition knowledge, which is why it was said that they should be brought in here.\footnote{\vismAssertFootnoteCounter{38}\vismHypertarget{XXI.n38}{}“Said in the Discourse on Purification (\emph{visuddhi-kathā})” (\textbf{\cite{Vism-mhṭ}855}). See \hyperlink{XX.77}{XX.77}{}.}

                        \vismParagraph{XXI.91}{91}{}
                        \emph{1. The Fruit Bat.} There was a fruit bat, it seems. She had alighted on a honey tree (\emph{madhuka}) with five branches, thinking, “I shall find flowers or fruits here.” She investigated one branch but saw no flowers or fruits there worth taking. And as with the first so too she tried the second, the third, the fourth, and the fifth, but saw nothing. She thought, “This tree is barren; there is nothing worth taking here,” so she lost interest in the tree. She climbed up on a straight branch, and poking her head through a gap in the foliage, she looked upwards, flew up into the air and alighted on another tree.

                        \vismParagraph{XXI.92}{92}{}
                        Herein, the meditator should be regarded as like the fruit bat. The five aggregates as objects of clinging are like the honey tree with the five branches. The meditator’s interpreting of the five aggregates is like the fruit bat’s alighting on the tree. His comprehending the materiality aggregate and, seeing nothing there worth taking, comprehending the remaining aggregates is like her trying each branch and, seeing nothing there worth taking, trying the rest. His triple knowledge beginning with desire for deliverance, after he has become dispassionate towards the five aggregates \textcolor{brown}{\textit{[664]}} through seeing their characteristic of impermanence, etc., is like her thinking “This tree is barren; there is nothing worth taking here” and losing interest. His conformity knowledge is like her climbing up the straight branch. His change-of-lineage knowledge is like her poking her head out and looking upwards. His path knowledge is like her flying up into the air. His fruition knowledge is like her alighting on a different tree.

                        \vismParagraph{XXI.93}{93}{}
                        \marginnote{\textcolor{teal}{\footnotesize\{751|693\}}}{}\emph{2. The Black Snake.} This simile has already been given (\hyperlink{XXI.49}{§49}{}). But the application of the simile here is this. Change-of-lineage knowledge is like throwing the snake away. Path knowledge is like the man’s standing and looking back whence he had come after getting free from it. Fruition knowledge is like his standing in a place free from fear after he had got away. This is the difference.

                        \vismParagraph{XXI.94}{94}{}
                        \emph{3. The House.} The owner of a house, it seems, ate his meal in the evening, climbed into his bed and fell asleep. The house caught fire. When he woke up and saw the fire, he was frightened. He thought, “It would be good if I could get out without getting burnt.” Looking round, he saw a way. Getting out, he quickly went away to a safe place and stayed there.

                        \vismParagraph{XXI.95}{95}{}
                        Herein, the foolish ordinary man’s taking the five aggregates as “I” and “mine” is like the house-owner’s falling asleep after he had eaten and climbed into bed. Knowledge of appearance as terror after entering upon the right way and seeing the three characteristics is like the time when the man was frightened on waking up and seeing the fire. Knowledge of desire for deliverance is like the man’s looking for a way out. Conformity knowledge is like the man’s seeing the way. Change-of-lineage is like the man’s going away quickly. Fruition knowledge is like his staying in a safe place.

                        \vismParagraph{XXI.96}{96}{}
                        \emph{4. The Oxen.} One night, it seems, while a farmer was sleeping his oxen broke out of their stable and escaped. When he went there at dawn and looked in, he found that they had escaped. Going to find them, he saw the king’s oxen. He thought that they were his and drove them back. When it got light, he recognized that they were not his but the king’s oxen. He was frightened. Thinking, “I shall escape before the king’s men seize me for a thief and bring me to ruin and destruction,” he abandoned the oxen. Escaping quickly, he stopped in a place free from fear.

                        \vismParagraph{XXI.97}{97}{}
                        Herein, the foolish ordinary man’s taking the five aggregates as “I” and “mine” is like the man’s taking the king’s oxen. The meditator’s recognizing the five aggregates as impermanent, painful, and not-self by means of the three characteristics is like the man’s recognizing the oxen as the king’s when it got light. Knowledge of appearance as terror is like the time when the man was frightened. Desire for deliverance is like the man’s desire to leave them and go away. Change-of-lineage is like the man’s actual leaving. The path is like his escaping. Fruition is like the man’s staying in a place without fear after escaping. \textcolor{brown}{\textit{[665]}}

                        \vismParagraph{XXI.98}{98}{}
                        \emph{5. The Ghoul.} A man went to live with a ghoul, it seems. At night, thinking he was asleep, she went to the place where the dead were exposed and ate human flesh. He wondered where she was going and followed her. When he saw her eating human flesh, he knew that she was a non-human being. He was frightened, and he thought, “I shall escape before she eats me.” Quickly escaping, he went to a safe place and stayed there.

                        \vismParagraph{XXI.99}{99}{}
                        Herein, taking the aggregates as “I” and “mine” is like the man’s living with the ghoul. Recognizing the aggregates as impermanent, etc., by seeing the three characteristics is like the man’s recognizing that she was a ghoul on seeing her eating human flesh in the place for the dead. Appearance as terror is \marginnote{\textcolor{teal}{\footnotesize\{752|694\}}}{}like the time when the man was frightened. Desire for deliverance is like his desire to escape. Change-of-lineage is like his leaving the place for the dead. The path is like his escaping quickly. Fruition is like his standing in the place without fear.

                        \vismParagraph{XXI.100}{100}{}
                        \emph{6. The Child.} A woman was very fond of her son, it seems. While sitting on an upper floor she heard the sound of a child in the street. Wondering, “Is someone hurting my child?,” she hurried down. Mistaking the child for her own son, she picked up someone else’s son. Then she recognized that it was someone else’s son, and she was ashamed and looked about her. She thought, “Let no one say I am a baby thief” and she put the child down there and then, and she quickly returned to the upper floor and sat down.

                        \vismParagraph{XXI.101}{101}{}
                        Herein, taking the five aggregates as “I” and “mine” is like the woman’s mistaking someone else’s child for her own. The recognition that “This is not I, not mine” by means of the three characteristics is like her recognizing it as someone else’s child. Knowledge of desire for deliverance is like her looking about her. Conformity knowledge is like her putting the child down there and then. Change-of-lineage is like the time when she stood in the street after putting the child down. The path is like her return to the upper floor. Fruition is like her sitting down after returning.

                        \vismParagraph{XXI.102}{102}{}
                        \emph{7–12. Hunger, Thirst, Cold, Heat, Darkness, and By Poison}. These six similes, however, are given for the purpose of showing that one with insight that leads to emergence tends, inclines and leans in the direction of the supramundane states.

                        \vismParagraph{XXI.103}{103}{}
                        \emph{7.} Just as a man faint with hunger and famished longs for delicious food, so too the meditator famished with the hunger of the round of rebirths longs for the food consisting of mindfulness occupied with the body, which tastes of the deathless.

                        \vismParagraph{XXI.104}{104}{}
                        \emph{8.} Just as a thirsty man whose throat and mouth are parched longs for a drink with many ingredients, so too this meditator \textcolor{brown}{\textit{[666]}} who is parched with the thirst of the round of rebirths longs for the noble drink of the Eightfold Path.

                        \vismParagraph{XXI.105}{105}{}
                        \emph{9.} Just as a man frozen by cold longs for heat, so too this meditator frozen by the cold of craving and [selfish] affection in the round of rebirths longs for the fire of the path that burns up the defilements.

                        \vismParagraph{XXI.106}{106}{}
                        \emph{10.} Just as a man faint with heat longs for cold, so too this meditator scorched by the burning of the eleven fires (see \textbf{\cite{S}IV 19}) in the round of rebirths longs for Nibbāna.

                        \vismParagraph{XXI.107}{107}{}
                        \emph{11.} Just as a man smothered in darkness longs for light, so too this meditator wrapped and enveloped in the darkness of ignorance longs for the light of knowledge consisting in path development.

                        \vismParagraph{XXI.108}{108}{}
                        \emph{12.} Just as a man sick with poison longs for an antidote, so too this meditator sick with the poison of defilement longs for Nibbāna, the deathless medicine that destroys the poison of defilement.

                        \vismParagraph{XXI.109}{109}{}
                        That is why it was said above: “When he knows and sees thus, his heart retreats, retracts and recoils from the three kinds of becoming, the four kinds of generation, the five kinds of destiny, the seven stations of consciousness, and the \marginnote{\textcolor{teal}{\footnotesize\{753|695\}}}{}nine abodes of beings; his heart no longer goes out to them. Just as water drops retreat, retract and recoil on a lotus leaf that slopes a little …” (\hyperlink{XXI.63}{§63}{}), all of which should be given in the way already stated.

                        \vismParagraph{XXI.110}{110}{}
                        But at this point he is called “one who walks aloof,” with reference to whom it is said:
                        \begin{verse}
                            “When a bhikkhu keeps apart\\{}
                            And cultivates seclusion of the mind,\\{}
                            It will befit him, as they say,\\{}
                            To show himself no more in this becoming” (\textbf{\cite{Sn}810}).
                        \end{verse}

                    \subsubsection[\vismAlignedParas{§111–127}The Difference in the Noble Path’s Factors, Etc.]{The Difference in the Noble Path’s Factors, Etc.}

                        \vismParagraph{XXI.111}{111}{}
                        This knowledge of equanimity about formations governs the fact that the meditator keeps apart. It furthermore governs the difference in the [number of the] noble path’s enlightenment factors, path factors, and jhāna factors, the mode of progress, and the kind of liberation. For while some elders say that it is the jhāna used as the basis for insight [leading to emergence] that governs the difference in the [number of] enlightenment factors, path factors, and jhāna factors, and some say that it is the aggregates made the object of insight that govern it, and some say that it is the personal bent that governs it,\footnote{\vismAssertFootnoteCounter{39}\vismHypertarget{XXI.n39}{}“The first ‘\emph{some}’ refers to the Elder Tipiṭaka Cūḷa-Nāga. The second ‘\emph{some}’ refers to the Elder Mahā Datta, dweller at Moravāpi. The third ‘some’ refers to the Elder Tipiṭaka Cūḷa Abhaya” (\textbf{\cite{Vism-mhṭ}856}).} yet it is only this preliminary insight and insight leading to emergence that should be understood to govern it in their doctrine.

                        \vismParagraph{XXI.112}{112}{}
                        To deal with these [three theories] in order: According to governance by insight, the path arisen in a bare-insight (dry-insight) worker, and the path arisen in one who possesses a jhāna attainment but who has not made the jhāna the basis for insight, and the path made to arise by comprehending unrelated formations after using the first jhāna as the basis for insight, are \textcolor{brown}{\textit{[667]}} paths of the first jhāna only. In each case there are seven enlightenment factors, eight path factors, and five jhāna factors. For while their preliminary insight can be accompanied by joy and it can be accompanied by equanimity, when their insight reaches the state of equanimity about formations at the time of emergence it is accompanied by joy.

                        \vismParagraph{XXI.113}{113}{}
                        When paths are made to arise by using the second, third, and fourth jhānas in the fivefold reckoning as the basis for insight, then the jhāna in those paths has respectively four, three, and two factors. In each case, however, the path factors number seven, and in the fourth case there are six enlightenment factors. This difference is due both to governance by the basic jhāna and to governance by insight. For again, while their preliminary insight can be accompanied by joy and it can be accompanied by equanimity, their insight leading to emergence is accompanied by joy only.

                        \vismParagraph{XXI.114}{114}{}
                        However, when the path is produced by making the fifth jhāna the basis for insight, then the jhāna factors number two, that is, equanimity and unification \marginnote{\textcolor{teal}{\footnotesize\{754|696\}}}{}of the mind, and there are six enlightenment factors and seven path factors. This difference too is due to both kinds of governance. For in this case the preliminary insight is either accompanied by joy or accompanied by equanimity, but that leading to emergence is accompanied by equanimity only. The same method applies in the case of the path made to arise by making the immaterial jhānas the basis for insight.

                        Also when, after emerging from jhāna made the basis for insight, the path has been produced by comprehending no matter what formations [unrelated to that jhāna], then it is the attainment emerged from at the point nearest to the path that makes it like itself, as the colour of the soil does an monitor lizard’s colour.

                        \vismParagraph{XXI.115}{115}{}
                        But in the case of the second elder’s theory the path is like the attainment, whatever it may be, which was instrumental in producing the path through the comprehension of any of its states after emergence from it. And here governance by insight should be understood in the same way as before.

                        \vismParagraph{XXI.116}{116}{}
                        In the case of the third elder’s theory the path is like that jhāna, whichever it may be, that suits the personal bent, which jhāna was instrumental in producing the path through the comprehension of any of its states in using it as the basis for insight. But this is not accomplished by mere bent alone unless the jhāna has been made the basis for insight or unless the jhāna has been comprehended; and this meaning should be illustrated by the Nandakovāda Sutta (see \textbf{\cite{M}III 277}, and Commentary). And here too, governance by insight should be understood in the same way as before.

                        This, firstly, is how it should be understood that equanimity about formations governs the [numbers of] enlightenment factors, path factors, and jhāna factors.

                        \vismParagraph{XXI.117}{117}{}
                        [\emph{Progress.}] But if [insight] has from the start only been able to suppress defilements with difficulty, with effort and with prompting, then it is called “of difficult progress.” \textcolor{brown}{\textit{[668]}} The opposite kind is called “of easy progress.” And when the manifestation of the path, the goal of insight, is slowly effected after defilements have been suppressed, then it is called “of sluggish direct-knowledge.” The opposite kind is called “of swift direct-knowledge.” So this equanimity about formations stands at the arrival point and gives its own name to the path in each case, and so the path has four names [according to the kind of progress] (see \textbf{\cite{D}III 228}).

                        \vismParagraph{XXI.118}{118}{}
                        For one bhikkhu this progress is different in the four paths, while for another it is the same. For Buddhas, however, the four paths are of easy progress and swift direct-knowledge. Likewise in the case of the General of the Dhamma [the Elder Sāriputta]. But in the Elder Mahā Moggallāna’s case the first path was of easy progress and swift direct-knowledge, but the others were of difficult progress and sluggish direct-knowledge.

                        \vismParagraph{XXI.119}{119}{}
                        [\emph{Predominance.}] And as with the kinds of progress, so also with the kinds of predominance,\footnote{\vismAssertFootnoteCounter{40}\vismHypertarget{XXI.n40}{}The four predominances are those of zeal (desire), energy, consciousness, and inquiry. Cf. four roads to power (\textbf{\cite{Dhs}§73–74}; \textbf{\cite{Vibh}216} and Comy.).} which are different in the four paths for one bhikkhu and the \marginnote{\textcolor{teal}{\footnotesize\{755|697\}}}{}same for another. So it is equanimity about formations that governs the difference in the progress.

                        [\emph{Liberation.}] But it has already been told how it governs the difference in the liberation [\hyperlink{XXI.66}{§66f.}{}].

                        \vismParagraph{XXI.120}{120}{}
                        Furthermore, the path gets its names for five reasons, that is to say, (1) owing to its own nature, or (2) owing to what it opposes, or (3) owing to its own special quality, or (4) owing to its object, or (5) owing to the way of arrival.

                        \vismParagraph{XXI.121}{121}{}
                        \emph{1. }If equanimity about formations induces emergence by comprehending formations as impermanent, liberation takes place with the signless liberation. If it induces emergence by comprehending them as painful, liberation takes place with the desireless liberation. If it induces emergence by comprehending them as not-self, liberation takes place with the void liberation. This is its name according to \emph{its own nature}.

                        \vismParagraph{XXI.122}{122}{}
                        \emph{2.} When this path is arrived at with the abandoning of the signs of permanence, lastingness, and eternalness, by effecting the resolution of the compact in formations by means of the contemplation of impermanence, it is then called signless. When it is arrived at with the drying up of desire and longing, by abandoning perception of pleasure by means of the contemplation of pain, it is then called desireless. When formations are seen as void by abandoning perception of self, of a living being, of a person, by means of the contemplation of not-self, it is then called void. This is its name according to \emph{what it opposes}.

                        \vismParagraph{XXI.123}{123}{}
                        \emph{3. }It is void because void of greed, and so on. It is signless owing either to absence of the sign of materiality, etc., or to absence only of the sign of greed, and so on. It is desireless because of absence of desire as greed, and so on. This is its name according to its \emph{own special quality}.

                        \vismParagraph{XXI.124}{124}{}
                        \emph{4. }It is called void, signless, and desireless, too, because it makes the void, signless, desireless Nibbāna its object. This is its name according to \emph{its object}. \textcolor{brown}{\textit{[669]}}

                        \vismParagraph{XXI.125}{125}{}
                        \emph{5. }The way of arrival is twofold, namely, insight’s way of arrival applies to the path, and the path’s way of arrival applies to fruition.

                        Now, contemplation of not-self is called void and the path [arrived at] by void insight is [called] void.

                        Again, contemplation of impermanence is called signless and the path

                        \vismParagraph{XXI.126}{126}{}
                        [arrived at] by signless insight is [called] signless. But while this name is inadmissible by the Abhidhamma method,\footnote{\vismAssertFootnoteCounter{41}\vismHypertarget{XXI.n41}{}“If this is so, then is the path that follows on the contemplation of impermanence not included in the Abhidhamma?—That is not so; for it is included in the method of ‘simple progress’ (\emph{suddhika paṭipadā—}see \textbf{\cite{Dhs}§§339–340})” (\textbf{\cite{Vism-mhṭ}861}).} it is admissible by the Suttanta method; for, they say, by that method change-of-lineage takes the name “signless” by making the signless Nibbāna its object, and while itself remaining at the arrival point, it gives its name to the path. \marginnote{\textcolor{teal}{\footnotesize\{756|698\}}}{}Hence the path is called signless. And its fruition can be called signless too according to the path’s way of arrival.

                        \vismParagraph{XXI.127}{127}{}
                        Lastly, contemplation of pain is called desireless because it arrives [at the path] by drying up desire for formations. The path [arrived at] by desireless insight is [called] desireless. The fruition of the desireless path is [called] desireless.

                        In this way insight gives its own name to the path, and the path hands it on to its fruition. This is its name according to \emph{the way of arrival.}

                        This is how equanimity about formations governs the difference in the liberations.

                        Equanimity about formations is ended.
                \subsection[\vismAlignedParas{§128–136}9. Conformity Knowledge]{9. Conformity Knowledge}

                    \vismParagraph{XXI.128}{128}{}
                    As he repeats, develops and cultivates that equanimity about formations, his faith becomes more resolute, his energy better exerted, his mindfulness better established, his mind better concentrated, while his equanimity about formations grows more refined.

                    \vismParagraph{XXI.129}{129}{}
                    He thinks, “Now the path will arise.” Equanimity about formations, after comprehending formations as impermanent, or as painful, or as not-self, sinks into the life-continuum. Next to the life-continuum, mind-door adverting arises making formations its object as impermanent or as painful or as not-self according to the way taken by equanimity about formations. Then next to the functional [adverting] consciousness that arose displacing the life-continuum, the first impulsion consciousness arises making formations its object in the same way, maintaining the continuity of consciousness.\footnote{\vismAssertFootnoteCounter{42}\vismHypertarget{XXI.n42}{}“‘\emph{Maintaining the continuity of consciousness}’ by absence of interruption, in other words, of occurrence of dissimilar consciousness. For when the life-continuum [which is mind-consciousness element] is displaced by the functional mind element [of five-door adverting (70)], the occurrence of the functional consciousness makes an interruption, an interval, between the occurrence of the resultant consciousness [i.e. the life-continuum and the consciousness that follows]. But this is not so with mind-door adverting (71) [which is mind-consciousness element]” (\textbf{\cite{Vism-mhṭ}862}). See Table V, Cognitive Series.} This is called the “preliminary work.” Next to that a second impulsion consciousness arises making formations its object in the same way. This is called the “access.” Next to that \textcolor{brown}{\textit{[670]}} a third impulsion consciousness also arises making formations its object in the same way. This is called “conformity.”

                    \vismParagraph{XXI.130}{130}{}
                    These are their individual names. But it is admissible to call all three impulsions “repetition” or “preliminary-work” or “access” or “conformity” indiscriminately.

                    Conformity to what? To what precedes and to what follows. For it conforms to the functions of truth both in the eight preceding kinds of insight knowledge and in the thirty-seven states partaking of enlightenment that follow.

                    \vismParagraph{XXI.131}{131}{}
                    \marginnote{\textcolor{teal}{\footnotesize\{757|699\}}}{}Since its occurrence is contingent upon formations through [comprehending] the characteristics of impermanence, etc., it, so to speak, says, “Knowledge of rise and fall indeed saw the rise and fall of precisely those states that possess rise and fall” and “Contemplation of dissolution indeed saw the dissolution of precisely those states that possess dissolution” and “It was indeed precisely what was terrible that appeared as terror to [knowledge of] appearance as terror” and “Contemplation of danger indeed saw danger in precisely what was dangerous” and “Knowledge of dispassion indeed became dispassionate towards precisely that which should be regarded with dispassion” and “Knowledge of desire for deliverance indeed produced desire for deliverance from precisely what there should be deliverance from” and “What was reflected upon by knowledge of reflection was indeed precisely what should be reflected upon” and “What was looked on at with equanimity by equanimity about formations was indeed precisely what should be looked on at with equanimity.” So it conforms to the functions of truth both in these eight kinds of knowledge and in the thirty-seven states partaking of enlightenment which follow, because they are to be reached by entering upon it.

                    \vismParagraph{XXI.132}{132}{}
                    Just as a righteous king, who sits in the place of judgement hearing the pronouncements of the judges while excluding bias and remaining impartial, conforms both to their pronouncements and to the ancient royal custom by saying, “So be it,” so it is here too.

                    \vismParagraph{XXI.133}{133}{}
                    Conformity is like the king. The eight kinds of knowledge are like eight judges. The thirty-seven states partaking of enlightenment are like the ancient royal custom. Herein, just as the king conforms by saying “So be it” both to the judges’ pronouncements and to the royal custom, so this conformity, which arises contingent upon formations through [comprehending] impermanence, etc., conforms to the function of truth both in the eight kinds of knowledge and in the thirty-seven states partaking of enlightenment that follow. Hence it is called “knowledge in conformity with truth.” \textcolor{brown}{\textit{[671]}}

                    Knowledge of conformity is ended.

                    \vismParagraph{XXI.134}{134}{}
                    Though this conformity knowledge is the end of the insight leading to emergence that has formations as its object, still change-of-lineage knowledge is the last of all the kinds of insight leading to emergence.
                    \subsubsection[\vismAlignedParas{§135–136}Sutta References]{Sutta References}

                        \vismParagraph{XXI.135}{135}{}
                        Now, the following sutta references should be understood in order not to be confused about insight leading to emergence. For this insight leading to emergence is called “aloofness” (\emph{atammayatā})\footnote{\vismAssertFootnoteCounter{43}\vismHypertarget{XXI.n43}{}“Aloofness”—\emph{atammayatā}: not in PED. See also \textbf{\cite{M}III 43}. The word is made up of \emph{a + taṃ + maya + tā} = “not-made-of-that-ness.” Its meaning is non-attachment to any form of being.} in the Saḷāyatana-vibhaṅga Sutta thus, “Bhikkhus, by depending and relying on aloofness abandon, surmount, equanimity that is unified, based on unity” (\textbf{\cite{M}III 220}). In the Alagadda Sutta it \marginnote{\textcolor{teal}{\footnotesize\{758|700\}}}{}is called “dispassion” (\emph{nibbidā}) thus, “Being dispassionate his greed fades away. With the fading away of greed he is liberated” (\textbf{\cite{M}I 139}).

                        In the Susīma Sutta it is called “knowledge of the relationship of states” (\emph{dhammaṭṭhiti-ñāṇa}) thus, “Previously, Susīma, there is knowledge of relationship of states; subsequently there is knowledge of Nibbāna” (\textbf{\cite{S}II 124}). In the Poṭṭhapāda Sutta it is called the “culmination of perception” (\emph{saññagga}) thus, “First, Poṭṭhapāda, the culmination of perception arises, and afterwards knowledge” (\textbf{\cite{D}I 185}). In the Dasuttara Sutta it is called the “principal factor of purity” (\emph{parisuddhi-padhāniyaṅga}) thus, “Purification by knowledge and vision of the way is the principal factor of purity” (\textbf{\cite{D}III 288}).

                        In the Paṭisambhidāmagga it is called by the three names thus, “Desire for deliverance, and contemplation of reflection, and equanimity about formations: these things are one in meaning and only the letter is different” (\textbf{\cite{Paṭis}II 64}). In the Paṭṭhāna it is called by two names thus, “conformity to change-of-lineage” and “conformity to cleansing”\footnote{\vismAssertFootnoteCounter{44}\vismHypertarget{XXI.n44}{}The word \emph{vodāna} (“cleansing”) is used, in its loose sense of “purifying” in general, in \hyperlink{I.143}{I.143}{}. For its technical Abhidhamma sense here see \hyperlink{XXII.n7}{Ch. XXII note 7}{}.} (\textbf{\cite{Paṭṭh}} 1, 159).

                        In the Rathavinīta Sutta it is called “purification by knowledge and vision of the way” thus, “But how, friend, is it for the purpose of the purification by knowledge and vision of the way that the life of purity is lived under the Blessed One?” (\textbf{\cite{M}I 147}).

                        \vismParagraph{XXI.136}{136}{}
                        The Greatest Sage did thus proclaim
                        \begin{verse}
                            This insight stilled and purified,\\{}
                            That to emergence leads beside,\\{}
                            With many a neatly chosen name.\\{}
                            The round of rebirth’s slough of pain\\{}
                            Is vast and terrible; a man\\{}
                            Wisely should strive as best he can,\\{}
                            If he would this emergence gain.
                        \end{verse}


                        The twenty-first chapter called “The Description of Purification by Knowledge and Vision of the Way” in the Treatise on the Development of Understanding in the \emph{Path of Purification} composed for the purpose of gladdening good people.
        \chapter[Purification by Knowledge and Vision]{Purification by Knowledge and Vision\vismHypertarget{XXII}\newline{\textnormal{\emph{Ñāṇadassana-visuddhi-niddesa}}}}
            \label{XXII}

            \section[\vismAlignedParas{§1–31}I. Change-of-Lineage, Paths, and Fruits]{I. Change-of-Lineage, Paths, and Fruits}

                \vismParagraph{XXII.1}{1}{}
                \marginnote{\textcolor{teal}{\footnotesize\{759|701\}}}{}\textcolor{brown}{\textit{[672]}} Change-of-lineage knowledge comes next. Its position is to advert to the path, and so it belongs neither to purification by knowledge and vision of the way nor to purification by knowledge and vision, but being intermediate, it is unassignable. Still it is reckoned as insight because it falls in line with insight.

                \vismParagraph{XXII.2}{2}{}
                Purification by knowledge and vision properly consists in knowledge of the four paths, that is to say, the path of stream-entry, the path of once-return, the path of non-return, and the path of Arahantship.
                \subsection[\vismAlignedParas{§3–14}The First Path—First Noble Person]{The First Path—First Noble Person}

                    \vismParagraph{XXII.3}{3}{}
                    Herein, nothing further needs to be done by one who wants to achieve, firstly, the knowledge of the first path. For what he needs to do has already been done by arousing the insight that ends in conformity knowledge.

                    \vismParagraph{XXII.4}{4}{}
                    As soon as conformity knowledge has arisen in him in this way, and the thick murk that hides the truths has been dispelled by the respective force peculiar to each of the three kinds of conformity (see \hyperlink{XXI.129}{XXI.129f.}{}), then his consciousness no longer enters into or settles down on or resolves upon any field of formations at all, or clings, cleaves or clutches on to it, but retreats, retracts and recoils as water does from a lotus leaf, and every sign as object, every occurrence as object, appears as an impediment.

                    \vismParagraph{XXII.5}{5}{}
                    Then, while every sign and occurrence appears to him as an impediment, when conformity knowledge’s repetition has ended, change-of-lineage knowledge arises in him, which takes as its object the signless, non-occurrence, non-formation, cessation, Nibbāna,—which knowledge passes out of the lineage, the category, the plane, of the ordinary man and enters the lineage, the category, the plane, of the Noble Ones,—which, being the first adverting, the first concern, the first reaction, to Nibbāna as object, fulfils the state of a condition for the path in six ways, as proximity, \textcolor{brown}{\textit{[673]}} contiguity, repetition, decisive-support, absence, and disappearance conditions,—which is the culminating peak of insight,—which is irrevocable,—of which it is said:

                    \marginnote{\textcolor{teal}{\footnotesize\{760|702\}}}{}“How is it that understanding of emergence and turning away from the external\footnote{\vismAssertFootnoteCounter{1}\vismHypertarget{XXII.n1}{}“‘\emph{Of emerging and turning away from the external}’: it is the understanding of turning away that is being effected, which turning away is emergence from the field of formations; it is termed external because the unformed element’s existence is external” (\textbf{\cite{Vism-mhṭ}866}). The unformed element (=Nibbāna) is classed as “external” under the internal (\emph{ajjhattika}) triad of the \emph{Abhidhamma Mātikā} (see \textbf{\cite{Dhs}2} and p. 241).} is change-of-lineage knowledge?

                    “It overcomes arising, thus it is change-of-lineage. It overcomes occurrence … [the sign … accumulation … rebirth-linking … destiny … generation … re-arising … birth … ageing … sickness … death … sorrow … lamentation … ]. It overcomes despair, thus it is change-of-lineage. It overcomes the sign of formations externally, thus it is change-of-lineage.

                    “It enters into\footnote{\vismAssertFootnoteCounter{2}\vismHypertarget{XXII.n2}{}\emph{Pakkhandati}—“enters into is glossed there by \emph{anupavisati} (enters in \textbf{\cite{Vism-mhṭ}(p. 566)}, which is the sense required and may be taken as based on the idiom in the Suttas, “\emph{Cittaṃ pakkhandati pasīdati santiṭṭhati adhimuccati}—the mind enters into [that], becomes settled, steady and resolute” (\textbf{\cite{M}I 186}), which is obviously inappropriate here.} non-arising, thus it is change-of-lineage. It enters into non-occurrence, thus it is change-of-lineage … (etc.) … It enters into non-despair, thus it is change-of-lineage. It enters into cessation, Nibbāna, thus it is change-of-lineage.

                    “Having overcome arising, it enters into non-arising, thus it is change-of-lineage …” (\textbf{\cite{Paṭis}I 56}) and so on, all of which should be quoted.

                    \vismParagraph{XXII.6}{6}{}
                    Here is a simile that illustrates how conformity and change-of-lineage occur with different objects though occurring in a single cognitive series with a single adverting. Suppose a man wanted to leap across a broad stream and establish himself on the opposite bank, he would run fast, and seizing a rope fastened to the branch of a tree on the stream’s near bank and hanging down, or a pole, he would leap with his body tending, inclining and leaning towards the opposite bank, and when he had arrived above the opposite bank, he would let go, fall on to the opposite bank, staggering first and then steady himself there; so too this meditator, who wants to establish himself on Nibbāna, the bank opposite to the kinds of becoming, generation, destiny, station, and abode, runs fast by means of the contemplation of rise and fall, etc., and seizing with conformity’s adverting to impermanence, pain or not-self the rope of materiality fastened to the branch of his selfhood and hanging down, or one among the poles beginning with feeling, he leaps with the first conformity consciousness without letting go and with the second he tends, inclines and leans towards Nibbāna, like the body that was tending, inclining and leaning towards the opposite bank; then, being with the third next to Nibbāna, which is now attainable, like the other’s arriving above the opposite bank, he lets go that information as object with the ceasing of that consciousness, and with the change-of-lineage consciousness he falls on to the unformed Nibbāna, the bank opposite; but staggering, as the man did, for lack of [previous] repetition, he is not yet properly steady on the single object. After that he is steadied by path knowledge.

                    \vismParagraph{XXII.7}{7}{}
                    \marginnote{\textcolor{teal}{\footnotesize\{761|703\}}}{}Herein, conformity is able to dispel the murk of defilements that conceals the truths, but is unable to make Nibbāna its object. Change-of-lineage is only able to make Nibbāna its object, but it is unable to dispel the murk that conceals the truths.

                    \vismParagraph{XXII.8}{8}{}
                    Here is a simile: \textcolor{brown}{\textit{[674]}} A man with eyes went out at night, it seems, to find out the conjunction of the stars, and he looked up to see the moon. It was invisible because it was concealed by clouds. Then a wind sprang up and blew away the thick clouds; another blew away the medium clouds; and another blew away the fine clouds as well. Then the man saw the moon in the sky free from clouds, and he found out the conjunction of the stars.

                    \vismParagraph{XXII.9}{9}{}
                    Herein, the thick, medium and fine kinds of darkness that conceal the truths are like the three kinds of cloud. The three kinds of conformity consciousness are like the three winds. Change-of-lineage knowledge is like the man with eyes. Nibbāna is like the moon. The dispelling of the murk that conceals the truths by each kind of conformity consciousness is like the successive blowing away of the clouds by each wind. Change-of-lineage knowledge’s seeing the clear Nibbāna when the murk that concealed the truths has disappeared is like the man’s seeing the clear moon in the sky free from cloud.

                    \vismParagraph{XXII.10}{10}{}
                    Just as the three winds are able only to blow away the clouds that conceal the moon but cannot see the moon, so the three kinds of conformity are able only to dispel the murk that conceals the truths but cannot see Nibbāna. Just as the man can only see the moon but cannot blow away the clouds, so change-of-lineage knowledge can only see Nibbāna but cannot dispel the defilements. Hence it is called “adverting to the path.”

                    \vismParagraph{XXII.11}{11}{}
                    For although it is not adverting, it occupies the position of adverting; and then, after, as it were, giving a sign to the path to come into being, it ceases. And without pausing after the sign given by that the change-of-lineage knowledge, the path follows upon it in uninterrupted continuity, and as it comes into being it pierces and explodes the mass of greed, the mass of hate, and the mass of delusion never pierced and exploded before (cf. \textbf{\cite{Paṭis}II 20}).

                    \vismParagraph{XXII.12}{12}{}
                    Here is a simile for this. An archer, it seems, had a target\footnote{\vismAssertFootnoteCounter{3}\vismHypertarget{XXII.n3}{}\emph{Phalakasataṃ—}“target”: not in PED. \textbf{\cite{Vism-mhṭ}} says “\emph{Phalakasatan ti asana-sāra-mayaṃ phalakasataṃ—}a “\emph{phalakasata}” is one made of the heart (pith) of the \emph{asana} tree.” The “wheel contrivance” resembles a potter’s wheel according to \textbf{\cite{Vism-mhṭ}(p. 867)}.} set up at a distance of eight \emph{usabhas} (about 100 yards), and wrapping his face in a cloth and arming himself with an arrow, he stood on a wheel contrivance (a revolving platform). Another man turned the wheel contrivance, and when the target was opposite the archer, he gave him a sign with a stick. Without pausing after the sign the archer shot the arrow and hit the target.

                    \vismParagraph{XXII.13}{13}{}
                    Herein, change-of-lineage knowledge is like the sign with the stick. Path knowledge is like the archer. Path knowledge’s \textcolor{brown}{\textit{[675]}} making Nibbāna its object without pausing after the sign given by change-of-lineage, and its piercing and exploding the mass of greed, hate and delusion never pierced and exploded before, is like the archer’s hitting the target without pausing after the sign.

                    \vismParagraph{XXII.14}{14}{}
                    \marginnote{\textcolor{teal}{\footnotesize\{762|704\}}}{}And not only does it cause the piercing of this mass of greed, etc., but it also dries up the ocean of suffering of the round in the beginningless round of rebirths. It closes all doors to the states of loss. It provides actual experience of the seven noble treasures.\footnote{\vismAssertFootnoteCounter{4}\vismHypertarget{XXII.n4}{}The seven (noble) treasures are: faith, virtue, conscience, shame, learning, generosity, and understanding (\textbf{\cite{D}III 251}).} It abandons the eightfold wrong path. It allays all enmity and fear.\footnote{\vismAssertFootnoteCounter{5}\vismHypertarget{XXII.n5}{}See the five kinds of enmity and fear at \textbf{\cite{S}II 68f.} \textbf{\cite{Vism-mhṭ}}, however, says: “The five kinds of enmity beginning with killing living things and the twenty-five great terrors (\emph{mahā-bhayāni}) are what constitute ‘\emph{all enmity and fear}’” (\textbf{\cite{Vism-mhṭ}867}).} It leads to the state of the Fully Enlightened One’s breast-born son (see \textbf{\cite{S}II 221}). And it leads to the acquisition of many hundred other blessings. So it is the knowledge associated with the path of stream-entry, the provider of many hundred blessings, that is called knowledge of the path of stream-entry.

                    The first kind of knowledge is ended.
                \subsection[\vismAlignedParas{§15–21}The First Fruition—Second Noble Person]{The First Fruition—Second Noble Person}

                    \vismParagraph{XXII.15}{15}{}
                    Immediately next to that knowledge, however, there arise either two or three fruition consciousnesses, which are its result. For it is owing to this very fact that supramundane profitable [consciousness] results immediately that it is said, “And which he called the concentration with immediate result” (Sn 226), and “Sluggishly he reaches what has immediate result for the destruction of the cankers” (\textbf{\cite{A}II 149}), and so on.

                    \vismParagraph{XXII.16}{16}{}
                    Some, however, say that there are one, two, three, four, or five fruition consciousnesses. That is inadmissible. For change-of-lineage knowledge arises at the end of conformity’s repetition, so at the minimum there must be two conformity consciousnesses, since one alone does not act as repetition condition. And a single series of impulsions has a maximum of seven [impulsion] consciousnesses. Consequently, that series which has two conformities and change-of-lineage as a third and path consciousness as fourth has three fruition consciousnesses. That which has three conformities and change-of-lineage as fourth and path consciousness as fifth has two fruition consciousnesses. That is why it was said above, “There arise either two or three fruition consciousnesses.”

                    \vismParagraph{XXII.17}{17}{}
                    Then some say that which has four conformities and change-of-lineage as fifth and path consciousness as sixth has one fruition consciousness. But that is refuted because it is the fourth or fifth [impulsion] that reaches [the path], not those after that, owing to their nearness to the life-continuum (see \hyperlink{IV.75}{IV.75}{}). So that cannot be accepted as correct. \textcolor{brown}{\textit{[676]}}

                    \vismParagraph{XXII.18}{18}{}
                    And at this point this stream-enterer is called the second noble person. However negligent he may be, he is bound to make an end of suffering when he has travelled and traversed the round of rebirths among deities and human beings for the seventh time.

                    \vismParagraph{XXII.19}{19}{}
                    At the end of the fruition his consciousness enters the life-continuum. After that, it arises as mind-door adverting interrupting the life-continuum for the purpose of reviewing the path. When that has ceased, seven impulsions of path reviewing \marginnote{\textcolor{teal}{\footnotesize\{763|705\}}}{}arise. After re-entry into the life-continuum, adverting, etc., arise again in the same way for the purpose of reviewing fruition, and so on. With the arising of these he reviews the path, he reviews the fruition, he reviews the defilements abandoned, he reviews the defilements still remaining, and he reviews Nibbāna.

                    \vismParagraph{XXII.20}{20}{}
                    He reviews the path in this way, “So this is the path I have come by.” Next he reviews the fruition after that in this way, “This is the blessing I have obtained.” Next he reviews the defilements that have been abandoned, “These are the defilements abandoned in me.” Next he reviews the defilements still to be eliminated by the three higher paths, “These are the defilements still remaining in me.” And lastly he reviews the deathless Nibbāna in this way, “This is the state (\emph{dhamma}) that has been penetrated by me as object.” So the noble disciple who is a stream-enterer has five kinds of reviewing.

                    \vismParagraph{XXII.21}{21}{}
                    And as in the case of the stream-enterer, so also in the cases of the once-returner and non-returner. But the Arahant has no reviewing of remaining defilements. So all the kinds of reviewing total nineteen. This is the maximum number. Trainers may or may not have the reviewing of the defilements abandoned and those still remaining. In fact it was owing to the absence of such reviewing that Mahānāma asked the Blessed One, “What state is there still unabandoned by me internally owing to which at times states of greed invade my mind and remain?” (\textbf{\cite{M}I 91}) all of which should be quoted.
                \subsection[\vismAlignedParas{§22–23}The Second Path—Third Noble Person]{The Second Path—Third Noble Person}

                    \vismParagraph{XXII.22}{22}{}
                    However, after reviewing in this way, either while sitting in the same session or on another occasion, the noble disciple who is a stream-enterer makes it his task to reach the second plane by attenuating both greed for sense desires and ill-will. He brings to bear the faculties,\footnote{\vismAssertFootnoteCounter{6}\vismHypertarget{XXII.n6}{}For the use of the expression “brings to bear”—\emph{samodhāneti} in this sense see \textbf{\cite{Paṭis}I 181}.} the powers, and the enlightenment factors, and he works over and turns up that same field of formations, classed as materiality, feeling, perception, formations, and consciousness, with the knowledge that they are impermanent, painful, not-self, and he embarks upon the progressive series of insights.

                    \vismParagraph{XXII.23}{23}{}
                    When he has \textcolor{brown}{\textit{[677]}} done so, and when, at the end of equanimity about formations, conformity and change-of-lineage\footnote{\vismAssertFootnoteCounter{7}\vismHypertarget{XXII.n7}{}“Here ‘\emph{change-of-lineage}’ means ‘like change-of-lineage’; for the knowledge that ushers in the [first] path is called that in the literal sense because it overcomes the ordinary man’s lineage and develops the Noble One’s lineage. But this is called ‘change-of-lineage’ figuratively because of its similarity to the other. It is also called ‘cleansing’ (\emph{vodāna}) because it purifies from certain defilements and because it makes absolute purification its object. Hence it is said in the Paṭṭhāna, ‘Conformity is a condition, as proximity condition, for cleansing’ (\textbf{\cite{Paṭṭh}I 59}). But ‘\emph{next to change-of-lineage}’ is said here because it is said in the Paṭisambhidāmagga that for the purpose of ‘overcoming arising,’ etc., ‘eight states of change-of-lineage arise through concentration’ and ‘ten states of change-of-lineage arise through concentration’ and ‘ten states of change-of-lineage arise through insight’ (\textbf{\cite{Paṭis}I 68–69}), and it is given in the same way in this page” (\textbf{\cite{Vism-mhṭ}869}).} knowledge have arisen in a single \marginnote{\textcolor{teal}{\footnotesize\{764|706\}}}{}adverting in the way already described, then the path of once-return arises next to change-of-lineage. The knowledge associated with that is knowledge of the path of once-return.

                    The second kind of knowledge is ended.
                \subsection[\vismAlignedParas{§24}The Second Fruition—Fourth Noble Person]{The Second Fruition—Fourth Noble Person}

                    \vismParagraph{XXII.24}{24}{}
                    The fruition consciousness should be understood to follow immediately upon this knowledge in the same way as before. And at this point this once-returner is called the fourth noble person. He is bound to make an end of suffering after returning once to this world. Next there comes reviewing in the way already described.
                \subsection[\vismAlignedParas{§25–26}The Third Path—Fifth Noble Person]{The Third Path—Fifth Noble Person}

                    \vismParagraph{XXII.25}{25}{}
                    Now, after reviewing in this way, either while sitting in the same session or on another occasion, this noble disciple who is a once-returner makes it his task to reach the third plane by abandoning, without remainder, both greed for the sense desires and ill-will. He brings to bear the faculties, the powers, and the enlightenment factors, and he works over and turns up that same field of formations with the knowledge that they are impermanent, painful, not-self, and he embarks upon the progressive series of insights.

                    \vismParagraph{XXII.26}{26}{}
                    When he has done so, and when, at the end of equanimity about formations, conformity and change-of-lineage have arisen in a single adverting in the way already described, then the path of non-return arises next to change-of-lineage. The knowledge associated with that is knowledge of the path of non-return.

                    The third kind of knowledge is ended.
                \subsection[\vismAlignedParas{§27}The Third Fruition—Sixth Noble Person]{The Third Fruition—Sixth Noble Person}

                    \vismParagraph{XXII.27}{27}{}
                    The fruition consciousnesses should be understood to follow immediately upon this knowledge in the same way as before. And at this point this non-returner is called the sixth noble person. [After death] he reappears apparitionally [elsewhere] and attains complete extinction there without ever returning, without ever coming to this world again through rebirth-linking. Next there comes reviewing in the way already described.
                \subsection[\vismAlignedParas{§28–29}The Fourth Path—Seventh Noble Person]{The Fourth Path—Seventh Noble Person}

                    \vismParagraph{XXII.28}{28}{}
                    Now, after reviewing in this way, either while sitting in the same session or on another occasion, this noble disciple who is a non-returner makes it his task to reach the fourth plane by abandoning, without remainder, greed for the fine-material and immaterial, conceit (pride), agitation, and ignorance. He brings to bear the faculties, the powers, and the enlightenment factors, and he works over \textcolor{brown}{\textit{[678]}} and turns up that same field of formations with the knowledge that they are impermanent, painful, not-self, and he embarks upon the progressive series of insights.

                    \vismParagraph{XXII.29}{29}{}
                    When he has done so, and when, at the end of equanimity about formations, conformity and change-of-lineage have arisen in a single adverting, then the \marginnote{\textcolor{teal}{\footnotesize\{765|707\}}}{}path of Arahantship arises next to change-of-lineage. The knowledge associated with that is knowledge of the path of Arahantship.

                    The fourth kind of knowledge is ended.
                \subsection[\vismAlignedParas{§30–31}The Fourth Fruition—Eighth Noble Person]{The Fourth Fruition—Eighth Noble Person}

                    \vismParagraph{XXII.30}{30}{}
                    The fruition consciousness should be understood to follow immediately upon this knowledge in the same way as before. And at this point this Arahant is called the eighth noble person. He is one of the Great Ones with cankers destroyed, he bears this last body, he has laid down the burden, reached his goal and destroyed the fetter of becoming, he is rightly liberated with [final] knowledge and worthy of the highest offering of the world with its deities.

                    \vismParagraph{XXII.31}{31}{}
                    So when it was said above, “However, purification by knowledge and vision properly consists in knowledge of the four paths, that is to say, the path of stream-entry, the path of once-return, the path of non-return, and the path of Arahantship” (\hyperlink{XXII.2}{§2}{}), that referred to these four kinds of knowledge to be reached in this order.
            \section[\vismAlignedParas{§32–129}II. The States Associated with the Path, Etc.]{II. The States Associated with the Path, Etc.}

                \vismParagraph{XXII.32}{32}{}
                Now, in order to appreciate the value of this same purification by knowledge and vision with its four kinds of knowledge:

                (1) fulfilment of states sharing in enlightenment,

                (2) Emergence, and (3) the coupling of the powers,

                (4) The kinds of states that ought to be abandoned,

                (5) Also the act of their abandoning,

                (6) Functions of full-understanding, and the rest

                As stated when truths are penetrated to,

                (7) Each one of which ought to be recognized According to its individual essence.
                \subsection[\vismAlignedParas{§33–43}The 37 states partaking of enlightenment]{The 37 states partaking of enlightenment}

                    \vismParagraph{XXII.33}{33}{}
                    \emph{1.} Herein, the \emph{fulfilment of states sharing in enlightenment} is the fulfilledness of those states partaking in enlightenment. For they are the following thirty-seven states: the four foundations of mindfulness (MN 10), the four right endeavours (\textbf{\cite{M}II 11}), the four roads to power (\textbf{\cite{M}I 103}), the five faculties (\textbf{\cite{M}II 12}), the five powers (\textbf{\cite{M}II 12}), the seven enlightenment factors (\textbf{\cite{M}I 11}), and the Noble Eightfold Path (\textbf{\cite{D}II 311f.}). And they are called “partaking of enlightenment” because they take the part of the Noble Eightfold Path, which is called “enlightenment” in the sense of enlightening, and they “take the part” of that because they are helpful.\footnote{\vismAssertFootnoteCounter{8}\vismHypertarget{XXII.n8}{}The four foundations of mindfulness are fully commented on in the commentary to MN 10 (= commentary to DN 22). The right endeavours are fully commented on in the commentary to the Sammappadhāna Vibhaṅga (cf. \textbf{\cite{M-a}II 243}; also A-a commenting on AN 1:II 1). The four roads to power are briefly commented on at \textbf{\cite{M-a}II 69} and fully in the commentary to the \textbf{\cite{M-a}I 82f.} and more fully in the commentary to the Bojjhaṅga Vibhaṅga. The Noble Eightfold Path is commented on at \textbf{\cite{M-a}I 105} and from a different angle in the commentary to the Magga Vibhaṅga. The five faculties and the five powers are not apparently dealt with in the Nikāya and the Abhidhamma Commentaries by adding anything further to what is said here (§37).}

                    \vismParagraph{XXII.34}{34}{}
                    \marginnote{\textcolor{teal}{\footnotesize\{766|708\}}}{}“Foundation” (\emph{paṭṭhāna}) is because of establishment (\emph{upaṭṭhāna}) by going down into, by descending upon, such and such objects.\footnote{\vismAssertFootnoteCounter{9}\vismHypertarget{XXII.n9}{}The Paṭisambhidā (\textbf{\cite{Paṭis}I 177}) derives \emph{satipaṭṭhāna} from \emph{sati} (mindfulness) and \emph{paṭṭhāna} (foundation, establishment). The commentaries prefer to derive it from \emph{sati }and \emph{upaṭṭhāna} (establishment, appearance, and also waiting upon: see \textbf{\cite{M-a}I 238}). The readings of the Ee and Ae eds. disagree here and that of the former has been followed though the result is much the same.} Mindfulness itself as foundation (establishment) is “foundation of mindfulness.” It is of four kinds because it occurs with respect to the body, feeling, consciousness, and mental objects (\emph{dhamma}), taking them as foul, painful, impermanent, and non-self, and because it accomplishes the function of abandoning perception of beauty, pleasure, permanence, and self. \textcolor{brown}{\textit{[679]}} That is why “four foundations of mindfulness” is said.

                    \vismParagraph{XXII.35}{35}{}
                    By it they endeavour (\emph{padahanti}), thus it is endeavour (\emph{padhāna}); a good endeavour is a right (\emph{sammā}) endeavour. Or alternatively: by its means people endeavour rightly (\emph{sammā padahanti}), thus it is right endeavour (\emph{sammappadhāna}). Or alternatively: it is good because of abandoning the unseemliness of defilement, and it is endeavour because of bringing about improvement and giving precedence (\emph{padhāna-bhāva-kāraṇa}) in the sense of producing well-being and bliss, thus it is right endeavour. It is a name for energy. It accomplishes the functions of abandoning arisen unprofitable things, preventing the arising of those not yet arisen, arousing unarisen profitable things, and maintaining those already arisen; thus it is fourfold. That is why “four right endeavours” is said.

                    \vismParagraph{XXII.36}{36}{}
                    Power (\emph{iddhi}) is in the sense of success (\emph{ijjhana}) as already described (\hyperlink{XII.44}{XII.44}{}). It is the road (basis—\emph{pāda}) to that power (for that success—\emph{iddhi}) in the sense of being the precursor of that success which is associated with it and in the sense of being the prior cause of that success which is its fruit, thus it is a road to power (basis for success). It is fourfold as zeal (desire), and so on. That is why “four roads to power” are spoken of, according as it is said: “Four roads to power: the road to power consisting in zeal, the road to power consisting in energy, the road to power consisting in [natural purity of] consciousness, the road to power consisting in inquiry” (\textbf{\cite{Vibh}223}). These are supramundane only. But because of the words “If a bhikkhu obtains concentration, obtains mental unification by making zeal predominant, this is called concentration through zeal” (\textbf{\cite{Vibh}216}), etc., they are also mundane as states acquired by predominance of zeal, etc., respectively.

                    \vismParagraph{XXII.37}{37}{}
                    “Faculty” is in the sense of predominance, in other words, of overcoming, because [these states, as faculties] respectively overcome faithlessness, idleness, negligence, distraction, and confusion.

                    “Power” is in the sense of unwaveringness because [these states, as powers] are incapable of being overcome respectively by faithlessness, and so on. Both are fivefold as consisting in faith, [energy, mindfulness, concentration, and understanding]. That is why “five faculties” and “five powers” is said.

                    \vismParagraph{XXII.38}{38}{}
                    \marginnote{\textcolor{teal}{\footnotesize\{767|709\}}}{}Mindfulness, [investigation-of-states, energy, happiness, tranquillity concentration, and equanimity,] as factors in a being who is becoming enlightened, are the “seven enlightenment factors.” And right view, [right thinking, right speech, right action, right livelihood, right effort, right mindfulness, and right concentration,] are the eight “path factors” in the sense of being an outlet. Hence, “seven enlightenment factors” and “the Noble Eightfold Path” is said.

                    \vismParagraph{XXII.39}{39}{}
                    So there are these thirty-seven states partaking of enlightenment.

                    Now, in the prior stage when mundane insight is occurring, they are found in a plurality of consciousnesses as follows: the foundation of mindfulness consisting in contemplation of the body [is found] in one discerning the body in the fourteen ways;\footnote{\vismAssertFootnoteCounter{10}\vismHypertarget{XXII.n10}{}These figures refer to the numbers of different contemplations described in the tenth sutta of the Majjhima Nikāya (= DN 22).} the foundation of mindfulness consisting in contemplation of feeling, in one discerning feeling in the nine ways; the foundation of mindfulness consisting in the contemplation of mind, in one discerning the [manner of] consciousness in sixteen ways; \textcolor{brown}{\textit{[680]}} the foundation of mindfulness consisting in contemplation of mental objects, in one discerning mental objects in the five ways. And at the time when, on seeing an unprofitable state arisen in someone else, which has not yet arisen in his own person, he strives for its non-arising thus, “I shall not behave as he has done in whom this is now arisen, and so this will not arise in me,” then he has the first right endeavour; when, seeing something unprofitable in his own behaviour, he strives to abandon it, then he has the second; when he strives to arouse jhāna or insight so far unarisen in this person, he has the third; and when he arouses again and again what has already arisen so that it shall not diminish, he has the fourth. And at the time of arousing a profitable state with zeal as the motive force, there is the road to power consisting in zeal, [and so on with the remaining three roads to power]. And at the time of abstaining from wrong speech there is right speech, [and so on with abstaining from wrong action and wrong livelihood].\footnote{\vismAssertFootnoteCounter{11}\vismHypertarget{XXII.n11}{}These three abstinences are the “prior state” of the Eightfold Path (see \textbf{\cite{M}III 289}). “Only the road to power consisting in zeal, and right speech, are actually included here; but when these are mentioned, the remaining roads to power and remaining two abstinences are implied in the meaning too. The meaning of this sentence should be understood according to the ‘category of characteristics’ (\emph{lakkhaṇa-hāra}—see Nettipakaraṇa)” (\textbf{\cite{Vism-mhṭ}872}). This \textbf{\cite{Netti}} rule says:

                            “When one thing has been stated, then those things That are in characteristic one with it Are stated too–this is the formulation Of the category of characteristics” (\textbf{\cite{Netti}3}).}

                    At the time of arising of [any one of] the four kinds of [path] knowledge, [all these states] are found in a single consciousness. In the moment of fruition the thirty-three excepting the four right endeavours are found.

                    \vismParagraph{XXII.40}{40}{}
                    \marginnote{\textcolor{teal}{\footnotesize\{768|710\}}}{}When these are found in a single consciousness in this way, it is the one kind of mindfulness whose object is Nibbāna that is called “the four foundations of mindfulness” because it accomplishes the function of abandoning the [four] perceptions of beauty, etc., in the [four things] beginning with the body. And also the one kind of energy is called “four right endeavours” because it accomplishes the [four] functions beginning with preventing the arising of the unarisen [unprofitable]. But there is no decrease or increase with the rest.

                    \vismParagraph{XXII.41}{41}{}
                    Furthermore it is said of them:
                    \begin{verse}
                        Nine in one way, one in two ways,\\{}
                        Then in four ways, and in five ways,\\{}
                        In eight ways, and in nine ways, too—\\{}
                        So in six ways they come to be.
                    \end{verse}


                    \vismParagraph{XXII.42}{42}{}
                    (i) \emph{Nine in one way}: these nine are zeal, consciousness, happiness, tranquillity, equanimity, thinking, speech, action, and livelihood, and they are found “in one way” as road to power consisting in zeal, etc., since they do not belong to any other group. (ii) \emph{One in two ways}: faith is found “in two ways,” as a faculty and as a power. (iii) \emph{Then in four ways, and} (iv) \emph{in five ways}: the meaning is that another one is found in four ways and another in five. Herein, concentration is the “one in four ways” since it is a faculty, a power, an enlightenment factor, and a path factor; understanding is the “one in five ways” since it is these four and also a road to power. (v) \emph{In eight ways, and }(vi)\emph{ in nine ways, too}: the meaning is that another one is found in eight ways and another in nine ways. Mindfulness is one “in eight ways” since it is the four foundations of mindfulness, a faculty, a power, an enlightenment factor, and a path factor; energy is the one “in nine ways” since it is four right endeavours, a road to power, a faculty, a power, an enlightenment factor, and a path factor. \textcolor{brown}{\textit{[681]}} So:

                    \vismParagraph{XXII.43}{43}{}
                    
                    \begin{verse}
                        States sharing in enlightenment\\{}
                        Are fourteen, undistributed;\\{}
                        They total thirty-seven states\\{}
                        Among the groups distributed.
                    \end{verse}

                    \begin{verse}
                        While each performs the proper task\\{}
                        That to its special lot falls due,\\{}
                        They all come into being when\\{}
                        The Noble Eightfold Path comes true.
                    \end{verse}


                    This is how, firstly, the “fulfilment of states partaking in enlightenment” should be understood here.
                \subsection[\vismAlignedParas{§44–46}Emergence and coupling of the powers]{Emergence and coupling of the powers}

                    \vismParagraph{XXII.44}{44}{}
                    \emph{2. Emergence }and \emph{3. coupling of the powers}: the resolution of the compound \emph{vuṭṭhānabalasamāyoga} is \emph{vuṭṭhānañ c’eva bala-samāyogo ca}.

                    [\emph{2. Emergence}:] mundane insight induces no emergence either from occurrence [of defilement internally], because it does not cut off originating, which is the act of causing occurrence,\footnote{\vismAssertFootnoteCounter{12}\vismHypertarget{XXII.n12}{}“Emergence from the sign consists in relinquishing the sign of formations and making Nibbāna the object. Emergence from occurrence consists in entering upon the state of non-liability to the occurrence of kamma-result in the future by causing the cessation of cause” (\textbf{\cite{Vism-mhṭ}874}).} or from the sign [of formations externally], because it has the sign as object. \marginnote{\textcolor{teal}{\footnotesize\{769|711\}}}{}Change-of-lineage knowledge does not induce emergence from occurrence [internally] because it does not cut off originating, but it does induce emergence from the sign [externally] because it has Nibbāna as its object; so there is emergence from one. Hence it is said, “Understanding of emergence and turning away from the external is knowledge of change-of-lineage” (\textbf{\cite{Paṭis}I 66}). Likewise the whole passage, “Having turned away from arising, it enters into non-arising, thus it is change-of-lineage. Having turned away from occurrence … (etc.—for elision see Ch. \hyperlink{XXI.37}{XXI.37}{}) … [Having turned away from the sign of formations externally, it enters into cessation, Nibbāna, thus it is change-of-lineage]” (\textbf{\cite{Paṭis}I 67}), should be understood here.

                    These four kinds of [path] knowledge emerge from the sign because they have the signless as their object, and also from occurrence because they cut off origination. So they emerge from both. Hence it is said:

                    \vismParagraph{XXII.45}{45}{}
                    “How is it that understanding of emergence and turning away from both is knowledge of the path?

                    “At the moment of the stream-entry path, right view in the sense of seeing (a) emerges from wrong view, and it emerges from defilements and from the aggregates that occur consequent upon that [wrong view],\footnote{\vismAssertFootnoteCounter{13}\vismHypertarget{XXII.n13}{}“It emerges from the defilements of uncertainty, etc., that occur consequent upon that view, which is wrong since it leads to states of loss” (\textbf{\cite{Vism-mhṭ}874}).} and (b) externally it emerges from all signs; hence it was said: Understanding of emergence and turning away from both is knowledge of the path. Right thinking in the sense of directing emerges from wrong thinking … Right speech in the sense of embracing emerges from wrong speech … Right action in the sense of originating emerges from wrong action … Right livelihood in the sense of cleansing emerges from wrong livelihood … Right effort in the sense of exerting emerges from wrong effort … Right mindfulness in the sense of establishment emerges from wrong mindfulness … Right concentration in the sense of non-distraction emerges from wrong concentration and it emerges from defilements and from the aggregates that occur consequent upon that [wrong concentration], and externally it emerges from all signs; hence it was said: Understanding of emergence and turning away from both is knowledge of the path.

                    “At the moment of the once-return path, right view in the sense of seeing … Right concentration in the sense of non-distraction (a) emerges from the gross fetter of greed for sense desires, from the gross fetter of resentment, from the gross inherent tendency to greed for sense desires, and from the gross inherent tendency to resentment, [and it emerges from defilements and from the aggregates consequent upon that, and (b) externally it emerges from all signs; hence it was said: Understanding of emergence and turning away from both is knowledge of the path].

                    “At the moment of the non-return path, right view in the sense of seeing … Right concentration in the sense of non-distraction (a) emerges \textcolor{brown}{\textit{[682]}} from the \marginnote{\textcolor{teal}{\footnotesize\{770|712\}}}{}residual fetter of greed for sense desires, from the residual fetter of resentment, from the residual inherent tendency to greed for sense desires, from the residual inherent tendency to resentment, [and it emerges from defilements and from the aggregates that occur consequent upon that, and (b) externally it emerges from all signs; hence it was said: Understanding of emergence and turning away from both is knowledge of the path].

                    “At the moment of the Arahant path, right view in the sense of seeing … Right concentration in the sense of non-distraction (a) emerges from greed for the fine-material [existence], from greed for immaterial [existence], from conceit (pride), from agitation, from ignorance, from the inherent tendency to conceit (pride), from the inherent tendency to greed for becoming, from the inherent tendency to ignorance, and it emerges from defilements and from the aggregates that occur consequent upon that, and (b) externally it emerges from all signs; hence it was said: Understanding of emergence and turning away from both is knowledge of the path” (\textbf{\cite{Paṭis}I 69f.}).

                    \vismParagraph{XXII.46}{46}{}
                    [\emph{3. Coupling of the powers}:] At the time of developing the eight mundane attainments the serenity power is in excess, while at the time of developing the contemplations of impermanence, etc., the insight power is in excess. But at the noble path moment they occur coupled together in the sense that neither one exceeds the other. So there is coupling of the powers in the case of each one of these four kinds of knowledge, according as it is said: “When he emerges from the defilements associated with agitation, and from the aggregates, his mental unification, non-distraction, concentration, has cessation as its domain. When he emerges from the defilements associated with ignorance and from the aggregates, his insight in the sense of contemplation has cessation as its domain. So serenity and insight have a single nature in the sense of emergence, they are coupled together, and neither exceeds the other. Hence it was said: He develops serenity and insight coupled together in the sense of emergence” (\textbf{\cite{Paṭis}II 98}).

                    “Emergence” and “coupling of the powers” should be understood here in this way.
                \subsection[\vismAlignedParas{§47–91}States to be abandoned]{States to be abandoned}

                    \vismParagraph{XXII.47}{47}{}
                    \emph{4. The kinds of states that ought to be abandoned}, 5. \emph{also the act of their abandoning}: now which states are to be abandoned by which kind of knowledge among these four should be understood, and also the act of abandoning them. For they each and severally bring about the act of abandoning of the states called fetters, defilements, wrongnesses, worldly states, kinds of avarice, perversions, ties, bad ways, cankers, floods, bonds, hindrances, adherences, clingings, inherent tendencies, stains, unprofitable courses of action, and unprofitable thought-arisings.

                    \vismParagraph{XXII.48}{48}{}
                    Herein, the \emph{fetters} are the ten states beginning with greed for the fine material, so called because they fetter aggregates [in this life] to aggregates [of the next], or kamma to its fruit, or beings to suffering. For as long as those exist there is no cessation of the others. And of these fetters, greed for the fine material, greed for the immaterial, conceit (pride), agitation, and ignorance are called the five higher fetters because they fetter beings to aggregates, etc., produced in higher [forms of becoming], \textcolor{brown}{\textit{[683]}} while false view of individuality, uncertainty, adherence to \marginnote{\textcolor{teal}{\footnotesize\{771|713\}}}{}rules and vows, greed for sense desires, and resentment are called the five lower fetters because they fetter beings to aggregates, etc., produced in the lower [forms of becoming].

                    \vismParagraph{XXII.49}{49}{}
                    The \emph{defilements} are the ten states, namely, greed, hate, delusion, conceit (pride), [false] view, uncertainty, stiffness [of mind], agitation, consciencelessness, shamelessness. They are so called because they are themselves defiled and because they defile their associated states.

                    \vismParagraph{XXII.50}{50}{}
                    The \emph{wrongnesses} are the eight states, namely, wrong view, wrong thinking, wrong speech, wrong action, wrong livelihood, wrong effort, wrong mindfulness, wrong concentration, which with wrong knowledge and wrong deliverance,\footnote{\vismAssertFootnoteCounter{14}\vismHypertarget{XXII.n14}{}“‘\emph{Wrong knowledge},’ which is wrong because it does not occur rightly [i.e. in conformity with the truth], and is wrong and mistaken owing to misinterpretations, etc., is just delusion. ‘\emph{Wrong deliverance}’ is the wrong notion of liberation that assumes liberation to take place in a ‘World Apex’ (\emph{lokathūpika–}see \hyperlink{XVI.85}{XVI.85}{}), and so on” (\textbf{\cite{Vism-mhṭ}886}).} come to ten. They are so called because they occur wrongly.

                    \vismParagraph{XXII.51}{51}{}
                    The \emph{worldly states} are the eight, namely, gain, loss, fame, disgrace, pleasure, pain, blame, and praise. They are so called because they continually succeed each other as long as the world persists. But when the worldly states are included, then by the metaphorical use of the cause’s name [for its fruit], the approval that has the gain, etc., as its object and the resentment that has the loss, etc., as its object should also be understood as included.

                    \vismParagraph{XXII.52}{52}{}
                    The \emph{kinds of avarice} are the five, namely, avarice about dwellings, families, gain, Dhamma, and praise, which occur as inability to bear sharing with others any of these things beginning with dwellings.

                    \vismParagraph{XXII.53}{53}{}
                    The \emph{perversions} are the three, namely, perversions of perception, of consciousness, and of view, which occur apprehending objects that are impermanent, painful, not-self, and foul (ugly), as permanent, pleasant, self, and beautiful.

                    \vismParagraph{XXII.54}{54}{}
                    The \emph{ties} are the four beginning with covetousness, so called because they tie the mental body and the material body. They are described as “the bodily tie of covetousness, the bodily tie of ill will, the bodily tie of adherence to rules and vows, and the bodily tie of insisting (misinterpreting) that ‘This [only] is the truth’” (\textbf{\cite{Vibh}374}).

                    \vismParagraph{XXII.55}{55}{}
                    \emph{Bad ways} is a term for doing what ought not to be done and not doing what ought to be done, out of zeal (desire), hate, delusion, and fear. They are called “bad ways” because they are ways not to be travelled by Noble Ones.

                    \vismParagraph{XXII.56}{56}{}
                    \emph{Cankers }(\emph{āsava}): as far as (\emph{ā}) change-of-lineage [in the case of states of consciousness] and as far as (\emph{ā}) the acme of becoming [in the case of the kinds of becoming, that is to say, the fourth immaterial state,] there are exudations (\emph{savana}) owing to the [formed nature of the] object. This is a term for greed for sense desires, greed for becoming, wrong view, and ignorance, because of the exuding (\emph{savana}) [of these defilements] from unguarded sense-doors like water from cracks \marginnote{\textcolor{teal}{\footnotesize\{772|714\}}}{}in a pot in the sense of constant trickling, or because of their producing (\emph{savana}) the suffering of the round of rebirths.\footnote{\vismAssertFootnoteCounter{15}\vismHypertarget{XXII.n15}{}The meaning of this paragraph is made clearer by reference to the \emph{Atthasālinī }(\textbf{\cite{Dhs-a}48}) and \emph{Mūla Ṭīkā} (\textbf{\cite{Dhs-ṭ}51}), where the use of \emph{ā} as an adverb in the sense of “as far as” indirectly with the ablative (\emph{gotrabhuto}, etc.) is explained; the abl. properly belongs to \emph{savana} (i.e. exudations from). \textbf{\cite{Vism-mhṭ}} only says: “‘\emph{Exudations}’ (\emph{savana}) because of occurring [due to], \emph{savanato} (“\emph{because of exuding}”) is because of flowing out as filth of defilement. \emph{Savanato} (“\emph{because of producing}”) the second time is because of giving out (\emph{pasavana})” (Vism-mhṭ 876. Cf. also \textbf{\cite{M-a}I 61}).} \textcolor{brown}{\textit{[684]}}

                    The \emph{floods} are so called in the sense of sweeping away into the ocean of becoming, and in the sense of being hard to cross.

                    The \emph{bonds} are so called because they do not allow disengagement from an object and disengagement from suffering. Both “floods” and “bonds” are terms for the cankers already mentioned.

                    \vismParagraph{XXII.57}{57}{}
                    The \emph{hindrances} are the five, namely, lust, [ill will, stiffness and torpor, agitation and worry, and uncertainty,] in the sense of obstructing and hindering and concealing [reality] from consciousness (\hyperlink{IV.86}{IV.86}{}).

                    \vismParagraph{XXII.58}{58}{}
                    \emph{Adherence }(\emph{misapprehension—parāmāsa}) is a term for wrong view, because it occurs in the aspect of missing the individual essence of a given state (\emph{dhamma}) and apprehending (\emph{āmasana}) elsewhere (\emph{parato}) an unactual individual essence.

                    \vismParagraph{XXII.59}{59}{}
                    The \emph{clingings} are the four beginning with sense-desire clinging described in all their aspects in the Description of the Dependent Origination (Ch. \hyperlink{XVII.240}{XVII.240f.}{}).

                    \vismParagraph{XXII.60}{60}{}
                    The \emph{inherent tendencies} are the seven, namely, greed for sense desires, etc., in the sense of the inveterateness, stated thus: the inherent tendency to greed for sense desires, the inherent tendency to resentment, conceit (pride), [false] view, uncertainty, greed for becoming, and ignorance. For it is owing to their inveteracy that they are called inherent tendencies (\emph{anusaya}) since they inhere (\emph{anusenti}) as cause for the arising of greed for sense desires, etc., again and again.

                    \vismParagraph{XXII.61}{61}{}
                    The \emph{stains} are the three, namely, greed, hate, and delusion. They are so called because they are themselves dirty like oil, black, and mud, and because they dirty other things.

                    \vismParagraph{XXII.62}{62}{}
                    The \emph{unprofitable courses of action} are the ten, namely, killing living things, taking what is not given, sexual misconduct; false speech, malicious speech, harsh speech, gossip; covetousness, ill will, and wrong view. They are so called since they are both unprofitable action (\emph{kamma}) and courses that lead to unhappy destinies.

                    \vismParagraph{XXII.63}{63}{}
                    The \emph{unprofitable thought-arisings} are the twelve consisting of the eight rooted in greed, the two rooted in hate, and the two rooted in delusion (\hyperlink{XIV.89}{XIV.89f.}{}).

                    \vismParagraph{XXII.64}{64}{}
                    So these [four kinds of knowledge] each and severally abandon these states beginning with the fetters. How?

                    The five states eliminated by the first knowledge in the case of the \emph{fetters}, firstly, are: false view of personality, doubt, adherence to rules and vows, and \marginnote{\textcolor{teal}{\footnotesize\{773|715\}}}{}then greed for sense desires and resentment that are [strong enough] to lead to states of loss. The remaining gross greed for sense desires and resentment are eliminated by the second knowledge. Subtle greed for sense desires and resentment are eliminated by the third knowledge. The five beginning with greed for the fine material are only [actually] eliminated by the fourth knowledge.

                    In what follows, we shall not in every instance specify the fact with the expression “only [actually]”; nevertheless, whatever we shall say is eliminated by one of the [three] higher knowledges should be understood as only the [residual] state eliminated by the higher knowledge; for that state will have already been rendered not conducive to states of loss by the preceding knowledge.

                    \vismParagraph{XXII.65}{65}{}
                    In the case of the \emph{defilements}, [false] view and uncertainty are eliminated by the first knowledge. Hate is eliminated by the third knowledge. Greed, delusion, conceit (pride), mental stiffness, agitation, consciencelessness, and shameless-ness are eliminated by the fourth knowledge.

                    \vismParagraph{XXII.66}{66}{}
                    In the case of the \emph{wrongnesses}, wrong view, false speech, wrong action, and wrong \textcolor{brown}{\textit{[685]}} livelihood are eliminated by the first knowledge. Wrong thinking, malicious speech, and harsh speech are eliminated by the third knowledge. And here only volition is to be understood as speech. Gossip, wrong effort, wrong mindfulness, wrong concentration, wrong deliverance, and wrong knowledge are eliminated by the fourth knowledge.

                    \vismParagraph{XXII.67}{67}{}
                    In the case of the \emph{worldly states}, resentment is eliminated by the third knowledge, and approval is eliminated by the fourth knowledge. Some say that approval of fame and praise is eliminated by the fourth knowledge.

                    The \emph{kinds of avarice} are eliminated by the first knowledge only.

                    \vismParagraph{XXII.68}{68}{}
                    In the case of the \emph{perversions}, the perversions of perception, consciousness, and view, which find permanence in the impermanent and self in the not-self, and the perversion of view finding pleasure in pain and beauty in the foul, are eliminated by the first knowledge. The perversions of perception and consciousness finding beauty in the foul are eliminated by the third path. The perversions of perception and consciousness finding pleasure in the painful are eliminated by the fourth knowledge.

                    \vismParagraph{XXII.69}{69}{}
                    In the case of \emph{ties}, the bodily ties of adherence to rules and vows and of the insistence (misinterpretation) that “This is the truth” are eliminated by the first knowledge. The bodily tie of ill will is eliminated by the third knowledge. The remaining one is eliminated by the fourth path.

                    The \emph{bad ways} are eliminated by the first knowledge only.

                    \vismParagraph{XXII.70}{70}{}
                    In the case of the \emph{cankers}, the canker of view is eliminated by the first knowledge. The canker of sense desire is eliminated by the third knowledge. The other two are eliminated by the fourth knowledge.

                    The same thing applies in the case of the \emph{floods} and the \emph{bonds}.

                    \vismParagraph{XXII.71}{71}{}
                    In the case of the \emph{hindrances}, the hindrance of uncertainty is eliminated by the first knowledge. The three, namely, lust, ill will, and worry, are eliminated by \marginnote{\textcolor{teal}{\footnotesize\{774|716\}}}{}the third knowledge. Stiffness and torpor and agitation are eliminated by the fourth knowledge.

                    \emph{Adherence} is eliminated by the first knowledge only.

                    \vismParagraph{XXII.72}{72}{}
                    In the case of the \emph{clingings}, since according to what is given in the texts all worldly states are sense desires, that is, sense desires as object (see \textbf{\cite{Nidd}I 1–2}), and so greed both for the fine material and the immaterial falls under sense-desire clinging, consequently that sense-desire clinging is eliminated by the fourth knowledge. The rest are eliminated by the first knowledge.

                    \vismParagraph{XXII.73}{73}{}
                    In the case of the \emph{inherent tendencies}, the inherent tendencies to [false] view and to uncertainty are eliminated by the first knowledge. The inherent tendencies to greed for sense desires and to resentment are eliminated by the third knowledge. The inherent tendencies to conceit (pride), to greed for becoming, and to ignorance are eliminated by the fourth knowledge.

                    \vismParagraph{XXII.74}{74}{}
                    In the case of the \emph{stains}, the stain of hate is eliminated by the third knowledge, the others are eliminated by the fourth knowledge.

                    \vismParagraph{XXII.75}{75}{}
                    In the case of the \emph{unprofitable courses of action}, killing living things, taking what is not given, sexual misconduct, false speech, and wrong view are eliminated by the first knowledge. The three, namely, malicious speech, harsh speech, and ill will, are eliminated by the third knowledge. Gossip and covetousness are eliminated by the fourth knowledge.

                    \vismParagraph{XXII.76}{76}{}
                    In the case of the \emph{unprofitable thought-arisings}, the four associated with [false] view, and that associated with uncertainty, making five, are eliminated by the first knowledge. The two associated with resentment are eliminated by the third knowledge. The rest are eliminated by the fourth knowledge.

                    \vismParagraph{XXII.77}{77}{}
                    And what is eliminated by any one of them is abandoned by it. That is why it was said above, “So these [four kinds of knowledge] each and severally abandon these states beginning with the fetters.”

                    \vismParagraph{XXII.78}{78}{}
                    \emph{5. The act of the abandoning}: but how then? Do these [knowledges] abandon these states when they are past, or when they are future, or when \textcolor{brown}{\textit{[686]}} they are present? What is the position here? For, firstly, if [they are said to abandon them] when past or future, it follows that the effort is fruitless. Why? Because what has to be abandoned is non-existent. Then if it is when they are present, it is likewise fruitless because the things to be abandoned exist simultaneously with the effort, and it follows that there is development of a path that has defilement, or it follows that defilements are dissociated [from consciousness] though there is no such thing as a present defilement dissociated from consciousness.\footnote{\vismAssertFootnoteCounter{16}\vismHypertarget{XXII.n16}{}“The intention is: or it follows that there is dissociation of defilements from consciousness, like that of formations according to those who assert that formations exist dissociated form consciousness. He said, ‘there is no such thing as a present defilement dissociated from consciousness’ in order to show that that is merely the opinion of those who make the assertion. For it is when immaterial states are actually occurring by their having a single basis and being included in the three instants that they are present; so how could that be dissociated from consciousness? Consequently there is no dissociation from consciousness here” (\textbf{\cite{Vism-mhṭ}878}).}

                    \vismParagraph{XXII.79}{79}{}
                    \marginnote{\textcolor{teal}{\footnotesize\{775|717\}}}{}That is not an original argument; for in the text first the question is put: “When a man abandons defilements, does he abandon past defilements? Does he abandon future defilements? Does he abandon present defilements?” Then the objection is put in this way: “If he abandons past defilements, he destroys what has already been destroyed, causes to cease what has already ceased, causes to vanish what has already vanished, causes to subside what has already subsided. What is past, which is non-existent, that he abandons.” But this is denied in this way: “He does not abandon past defilements.” Then the objection is put in this way: “If he abandons future defilements, he abandons what has not been born, he abandons what has not been generated, he abandons what has not arisen, he abandons what has not become manifest. What is future, which is non-existent, that he abandons.” But this is denied in this way: “He does not abandon future defilements.” Then the objection is put in this way: “If he abandons present defilements, then though inflamed with greed he abandons greed, though corrupted with hate he abandons hate, though deluded he abandons delusion, though shackled\footnote{\vismAssertFootnoteCounter{17}\vismHypertarget{XXII.n17}{}“‘\emph{Shackled}’: one whose consciousness is shackled by conceit (pride)” (\textbf{\cite{Vism-mhṭ}878}).} he abandons conceit (pride), though misconceiving he abandons [false] view, though distracted he abandons agitation, though not having made up his mind he abandons uncertainty, though not having inveterate habits he abandons inherent tendency, dark and bright states occur coupled together, and there is development of a path that has defilement.” But this is all denied in this way: “He does not abandon past defilements, he does not abandon future defilements, he does not abandon present defilements.” Finally it is asked: “Then there is no path development, there is no realization of fruition, there is no abandoning of defilements, there is no penetration to the Dhamma (convergence of states)?” Then it is claimed: “There is path development … there is penetration to the Dhamma (convergence of states).”

                    And when it is asked: “In what way?” this is said: “Suppose there were a young tree with unborn fruit, and a man cut its root, then the unborn fruits of the tree would remain unborn and not come to be born, remain ungenerated and not come to be generated, remain unarisen and not come to be arisen, remain unmanifested and not come to be manifested. So too, arising is a cause, arising is a condition, for the generation of defilements. Seeing danger in defilements, consciousness enters into non-arising. With consciousness’s entering into non-arising the defilements that would be generated with arising as their condition remain unborn and do not come to be born … remain unmanifest and do not come to be manifested. So with the cessation of the cause there is the cessation of suffering. \textcolor{brown}{\textit{[687]}} Occurrence is a cause … The sign is a cause … Accumulation is a cause, accumulation is a condition, for the generation of defilements. Seeing danger in accumulation, consciousness enters into non-accumulation. With consciousness’s entering into non-accumulation the defilements that would be generated with accumulation as their condition remain unborn and do not come to be born … remain unmanifest and do not come to be manifested. So with the \marginnote{\textcolor{teal}{\footnotesize\{776|718\}}}{}cessation of the cause there is cessation of suffering. So there is path development, there is realization of fruition, there is abandoning of defilements, and there is penetrating to the Dhamma” (\textbf{\cite{Paṭis}II 217–219}).

                    \vismParagraph{XXII.80}{80}{}
                    What does that show? It shows abandoning of defilements that have soil [to grow in]. But are defilements that have soil [to grow in] past, future or present? They are simply those described as “arisen by having soil [to grow in].”

                    \vismParagraph{XXII.81}{81}{}
                    Now, there are various meanings of “arisen,” that is to say, (i) arisen as “actually occurring,” (ii) arisen as “been and gone,” (iii) arisen “by opportunity,” and (iv) arisen “by having [soil to grow in].”

                    Herein, (i) all that is reckoned to possess [the three moments of] arising, ageing, [that is, presence] and dissolution, is called \emph{arisen as actually occurring}.

                    (ii) Profitable and unprofitable [kamma-result] experienced as the stimulus of an object and ceased-reckoned as “experienced and gone” (\emph{anubhūtāpagata)—}, and also anything formed, when it has reached the three instants beginning with arising and has ceased-reckoned as ‘been and gone’ \emph{(hutvāpagata)—, }are called \emph{arisen as been and gone }(\emph{bhūtāpagata}).

                    (iii) Kamma described in the way beginning, “Deeds that he did in the past” (\textbf{\cite{M}III 164}), even when actually past, is called \emph{arisen by opportunity made} because it reaches presence by inhibiting other [ripening] kamma and making that the opportunity for its own result (see \hyperlink{XIX.16}{XIX.16}{}.) And kamma-result that has its opportunity made in this way, even when as yet unarisen, is called “arisen by opportunity made,” too, because it is sure to arise when an opportunity for it has been made in this way.

                    (iv) While unprofitable [kamma] is still unabolished in any given soil (plane)\footnote{\vismAssertFootnoteCounter{18}\vismHypertarget{XXII.n18}{}“‘\emph{In any given plane}’ means aggregates as objects of clinging, reckoned as a human or divine person” (\textbf{\cite{Vism-mhṭ}879}).} it is called \emph{arisen by having soil [to grow in}].

                    \vismParagraph{XXII.82}{82}{}
                    And here the difference between the soil and what has soil should be understood. For “soil” (plane) means the five aggregates in the three planes of becoming, which are the object of insight.\footnote{\vismAssertFootnoteCounter{19}\vismHypertarget{XXII.n19}{}“By the words ‘which are the object of insight’ he points out the non-fully-understood state of the aggregates, not merely the fact that they are the object of insight, which is proved by his taking only the three planes. For it is not-fully-understood aggregates among the aggregates constituting the [subjective] basis that are intended as the ‘soil of defilements’” (\textbf{\cite{Vism-mhṭ}880}).} “What has soil” is an expression for defilements capable of arising with respect to those aggregates. Those defilements have that soil (plane). That is why “by having soil [to grow in]” is said.

                    \vismParagraph{XXII.83}{83}{}
                    And that is not meant objectively. For defilements occupied with an object arise with respect to any aggregates including past or future ones as well [as present], and also with respect to the [subjectively] fully-understood aggregates in someone [else] whose cankers are destroyed, like those that arose in the rich man Soreyya with respect to the aggregates in Mahā Kaccāna (\textbf{\cite{Dhp-a}I 325}) and in the brahman student Nanda with respect to Uppalavaṇṇā (\textbf{\cite{Dhp-a}II 49}), and \marginnote{\textcolor{teal}{\footnotesize\{777|719\}}}{}so on. And if that were what is called “arisen by having soil [to grow in]” no one could abandon the root of becoming because it would be unabandonable. But “arisen by having soil [to grow in]” should be understood [subjectively] with respect to the basis [for them in oneself].\footnote{\vismAssertFootnoteCounter{20}\vismHypertarget{XXII.n20}{}“No one would be able to abandon the root of becoming if it were in another’s continuity. ‘\emph{With respect to the basis [for them in oneself}]’ means as the place of their arising; in that particular becoming or continuity” (\textbf{\cite{Vism-mhṭ}880}).} For the defilements that are the root of the round are inherent in [one’s own] aggregates not fully understood by insight from the instant those aggregates arise. And that is what should be understood as “arisen by having the soil [to grow in],” in the sense of its being unabandoned. \textcolor{brown}{\textit{[688]}}

                    \vismParagraph{XXII.84}{84}{}
                    Now, when defilements are inherent, in the sense of being unabandoned, in someone’s aggregates, it is only those aggregates of his that are the basis for those defilements, not aggregates belonging to another. And only past aggregates, not others, are the basis for defilements that inhere unabandoned in past aggregates. Likewise in the case of future aggregates, and so on. Similarly too only sense-sphere aggregates, not others, are the basis for defilements that inhere unabandoned in sense-sphere aggregates. Likewise in the case of the fine material and immaterial.

                    \vismParagraph{XXII.85}{85}{}
                    But in the case of the stream-enterer, etc., when a given defilement, which is a root of the round, has been abandoned by means of a given path in a given noble person’s aggregates, then his aggregates are no longer called “soil” for such defilement since they are no longer a basis for it. But in an ordinary man the defilements that are the root of the round are not abandoned at all, and so whatever kamma he performs is always either profitable or unprofitable. So for him the round goes on revolving with kamma and defilements as its condition.

                    \vismParagraph{XXII.86}{86}{}
                    But while it is thus the root of the round it cannot be said that it is only in his materiality aggregate, and not in his other aggregates beginning with feeling … that it is only in his consciousness aggregate, and not in his other aggregates beginning with materiality. Why? Because it is inherent in all five aggregates indiscriminately. How? Like the juice of humus, etc., in a tree.

                    \vismParagraph{XXII.87}{87}{}
                    For when a great tree is growing on the earth’s surface supported by the essences of humus and water and, with that as condition, increases its roots, trunks, branches, twigs, shoots, foliage, flowers, and fruit, till it fills the sky, and continues the tree’s lineage through the succession of the seed up till the end of the eon, it cannot be said that the essence of humus, etc., are found only in its root and not in the trunk, etc., … that they are only in the fruit and not in the root, etc., Why? Because they spread indiscriminately through the whole of it from the root onwards.

                    \vismParagraph{XXII.88}{88}{}
                    But some man who felt revulsion for that same tree’s flowers, fruits, etc., might puncture it on four sides with the poison thorn called “\emph{maṇḍūka} thorn,” and then the tree, being poisoned, would be no more able to prolong its continuity since it would have become barren with the contamination of the essences of humus and water. \marginnote{\textcolor{teal}{\footnotesize\{778|720\}}}{}So too the clansman who feels revulsion (dispassion) for the occurrence of the aggregates, undertakes to develop the four paths in his own continuity which is like the man’s application of poison to the tree on all four sides. Then the continuity of his aggregates is rendered incapable of prolonging the continuity to a subsequent becoming. It is now unproductive of future becoming since all kinds of kamma beginning with bodily kamma are now merely functional: for the effect of the four paths’ poison has entirely exterminated the defilements that are the root of the round. \textcolor{brown}{\textit{[689]}} Being without clinging, he inevitably attains with the cessation of the last consciousness the complete extinction [of Nibbāna], like a fire with no more fuel. This is how the difference between the soil and what has soil should be understood.

                    \vismParagraph{XXII.89}{89}{}
                    Besides these there are four other ways of classing “arisen,” namely, (v) arisen as happening, (vi) arisen with apprehension of an object, (vii) arisen through non-suppression, (viii) arisen through non-abolition.

                    Herein, (v) \emph{arisen as happening} is the same as (i) “arisen as actually occurring.”

                    (vi) When an object has at some previous time come into focus in the eye, etc., and defilement did not arise then but arose in full force later on simply because the object had been apprehended, then that defilement is called \emph{arisen with apprehension of an object}. Like the defilement that arose in the Elder Mahā-Tissa after seeing the form of a person of the opposite sex while wandering for alms in the village of Kalyāna (cf. \textbf{\cite{M-a}I 66} and A-a to \textbf{\cite{A}I 4}).

                    (vii) As long as a defilement is not suppressed by either serenity or insight, though it may not have actually entered the conscious continuity, it is nevertheless called \emph{arisen through non-suppression} because there is no cause to prevent its arising [if suitable conditions combine]. (viii) But even when they are suppressed by serenity or insight they are still called \emph{arisen through non-abolition} because the necessity for their arising has not been transcended unless they have been cut off by the path. Like the elder who had obtained the eight attainments, and the defilements that arose in him while he was going through the air on his hearing the sound of a woman singing with a sweet voice as she was gathering flowers in a grove of blossoming trees.

                    \vismParagraph{XXII.90}{90}{}
                    And the three kinds, namely, (vi) arisen with apprehension of an object, (vii) arisen through non-suppression, and (vii) arisen through non-abolition, should be understood as included by (iv) arisen by having soil [to grow in].

                    \vismParagraph{XXII.91}{91}{}
                    So as regard the kinds of “arisen” stated, the four kinds, namely, (i) as actually occurring, (ii) as been and gone, (iii) by opportunity made, and (v) as happening, cannot be abandoned by any [of these four kinds of] knowledge because they cannot be eliminated by the paths. But the four kinds of “arisen,” namely, (iv) by having soil [to grow in], (vi) with apprehension of an object, (vii) through non-suppression, and (viii) through non-abolition, can all be abandoned because a given mundane or supramundane knowledge, when it arises, nullifies a given one of these modes of being arisen.

                    So here “the kinds of states that ought to be abandoned, also the act of their abandoning” (\hyperlink{XXII.32}{§32}{}) should be known in this way.
                \subsection[\vismAlignedParas{§92–128}The Four Functions]{The Four Functions}

                    \vismParagraph{XXII.92}{92}{}
                    \marginnote{\textcolor{teal}{\footnotesize\{779|721\}}}{}
                    \begin{verse}
                        (6) Functions of full-understanding and the rest\\{}
                        As stated when truths are penetrated to,\\{}
                        (7) Each one of which ought to be recognized\\{}
                        According to its individual essence. (\hyperlink{XXII.32}{§32}{})
                    \end{verse}

                    \subsubsection[\vismAlignedParas{§92–103}The Four Functions in a Single Moment]{The Four Functions in a Single Moment}

                        \emph{6.} Now, at the times of penetrating to the truths each one of the four [path] knowledges is said to exercise four functions in a single moment. These are full-understanding, abandoning, realizing, and developing; and each one of them ought to be recognized according to its individual essence. \textcolor{brown}{\textit{[690]}} For this is said by the Ancients: “Just as a lamp performs the four functions simultaneously in a single moment—it burns the wick, dispels darkness, makes light appear, and uses up the oil—, so too, path knowledge penetrates to the four truths simultaneously in a single moment—it penetrates to suffering by penetrating to it with full-understanding, penetrates to origination by penetrating to it with abandoning, penetrates to the path by penetrating to it with developing, and penetrates cessation by penetrating to it with realizing” (see \textbf{\cite{Peṭ}134}). What is meant? By making cessation its object it reaches, sees and pierces the four truths.”

                        \vismParagraph{XXII.93}{93}{}
                        For this is said: “Bhikkhus, he who sees suffering sees also the origin of suffering, sees also the cessation of suffering, sees also the way leading to the cessation of suffering” (\textbf{\cite{S}V 437}), etc., and so it should be understood [for all the other three truths]. And further it is said: “The knowledge of one who possesses the path is knowledge of suffering and it is knowledge of the origin of suffering and it is knowledge of the cessation of suffering and it is knowledge of the way leading to the cessation of suffering” (\textbf{\cite{Paṭis}I 119}).

                        \vismParagraph{XXII.94}{94}{}
                        As the lamp burns the wick, so his path knowledge fully understands suffering; as the lamp dispels the darkness, so the knowledge abandons origin; as the lamp makes the light appear, so the knowledge [as right view] develops the path, in other words, the states consisting in right thinking, etc., [by acting] as conascence, etc., for them; and as the lamp uses up the oil, so the knowledge realizes cessation, which brings defilements to an end. This is how the application of the simile should be understood.

                        \vismParagraph{XXII.95}{95}{}
                        Another method: as the sun, when it rises, performs four functions simultaneously with its appearance—it illuminates visible objects, dispels darkness, causes light to be seen, and allays cold—, so too, path knowledge … penetrates to cessation by penetrating to it with realizing. And here also, as the sun illuminates visible objects, so path knowledge fully understands suffering; as the sun dispels darkness, so path knowledge abandons origin; as the sun causes light to be seen, so path knowledge [as right view] develops the [other] path [factors] by acting as [their] conascence condition, etc.; as the sun allays cold, so path knowledge realizes the cessation, which is the tranquilizing of defilements. This is how the application of the simile should be understood.

                        \vismParagraph{XXII.96}{96}{}
                        Another method: as a boat performs four functions simultaneously in a single moment—it leaves the hither shore, it cleaves the stream, it carries its \marginnote{\textcolor{teal}{\footnotesize\{780|722\}}}{}cargo, \textcolor{brown}{\textit{[691]}} and it approaches the further shore—, so too, path knowledge … penetrates to cessation by penetrating to it with realizing. And here, as the boat leaves the hither shore, so path knowledge fully understands suffering; as the boat cleaves the stream, so path knowledge abandons origin; as the boat carries its cargo, so path knowledge develops the [other] path [factors] by acting as [their] conascence condition, etc.; as the boat approaches the further shore, so path knowledge realizes cessation, which is the further shore. This is how the application of the simile should be understood.

                        \vismParagraph{XXII.97}{97}{}
                        So when his knowledge occurs with the four functions in a single moment at the time of penetrating the four truths, then the four truths have a single penetration in the sense of trueness (reality) in sixteen ways, as it is said: “How is there single penetration of the four truths in the sense of trueness? There is single penetration of the four truths in the sense of trueness in sixteen aspects: suffering has the meaning of oppressing, meaning of being formed, meaning of burning (torment), meaning of change, as its meaning of trueness; origin has the meaning of accumulation, meaning of source, meaning of bondage, meaning of impediment, as its meaning of trueness; cessation has the meaning of escape, meaning of seclusion, meaning of being not formed, meaning of deathlessness, as its meaning of trueness; the path has the meaning of outlet, meaning of cause, meaning of seeing, meaning of dominance, as its meaning of trueness. The four truths in these sixteen ways are included as one. What is included as one is unity. Unity is penetrated by a single knowledge. Thus the four truths have a single penetration” (\textbf{\cite{Paṭis}II 107}).

                        \vismParagraph{XXII.98}{98}{}
                        Here it may be asked: “Since there are other meanings of suffering, etc., too, such as ‘a disease, a tumour’ (\textbf{\cite{Paṭis}II 238}; \textbf{\cite{M}I 435}), etc., why then are only four mentioned for each?” We answer that in this context it is better because of what is evident through seeing the other [three truths in each case].

                        Firstly, in the passage beginning, “Herein, what is knowledge of suffering? It is the understanding, the act of understanding … that arises contingent upon suffering” (\textbf{\cite{Paṭis}I 119}), knowledge of the truths is presented as having a single truth as its object [individually]. But in the passage beginning, “Bhikkhus, he who sees suffering also sees its origin” (\textbf{\cite{S}V 437}), it is presented as accomplishing its function with respect to the other three truths simultaneously with its making one of them its object.

                        \vismParagraph{XXII.99}{99}{}
                        As regards these [two contexts], when, firstly, knowledge makes each truth its object singly, then [when suffering is made the object], suffering has the characteristic of \emph{oppressing} as its individual essence, but its \emph{sense of being formed }becomes evident through seeing origin because that suffering is accumulated, formed, agglomerated, by the origin, which has the characteristic of accumulating. Then the cooling path removes the burning of the defilements, \textcolor{brown}{\textit{[692]}} and so suffering’s \emph{sense of burning }becomes evident through seeing the path, as the beauty’s (Sundarī’s) ugliness did to the venerable Nanda through seeing the celestial nymphs (see \textbf{\cite{Ud}23}). But its \emph{sense of changing} becomes evident through seeing cessation as not subject to change, which needs no explaining.

                        \vismParagraph{XXII.100}{100}{}
                        \marginnote{\textcolor{teal}{\footnotesize\{781|723\}}}{}Likewise, [when origin is made the object,] origin has the characteristic of \emph{accumulating} as its individual essence; but its \emph{sense of source} becomes evident through seeing suffering, just as the fact that unsuitable food is the source of a sickness, becomes evident through seeing how a sickness arises owing to such food. Its \emph{sense of bondage} becomes evident through seeing cessation, which has no bonds. And its \emph{sense of impediment} becomes evident through seeing the path, which is the outlet.

                        \vismParagraph{XXII.101}{101}{}
                        Likewise, [when cessation is made the object,] cessation has the characteristic of an \emph{escape}. But its \emph{sense of seclusion} becomes evident through seeing origin as unsecluded. Its \emph{sense of being not formed} becomes evident through seeing the path; for the path has never been seen by him before in the beginningless round of rebirths, and yet even that is formed since it has conditions, and so the unformedness of the conditionless becomes quite clear. But its \emph{sense of being deathless} becomes evident through seeing suffering; for suffering is poison and Nibbāna is deathless.

                        \vismParagraph{XXII.102}{102}{}
                        Likewise, [when the path is made the object,] the path has the characteristic of the \emph{outlet}. But its \emph{sense of cause} becomes evident through seeing origin thus, “That is not the cause, [but on the contrary] this is the cause, for the attaining of Nibbāna.” Its \emph{sense of seeing} becomes evident through seeing cessation, as the eye’s clearness becomes evident to one who sees very subtle visible objects and thinks, “How clear my eye is!” Its \emph{sense of dominance} becomes evident through seeing suffering, just as the superiority of lordly people becomes evident through seeing wretched people afflicted with many diseases.

                        \vismParagraph{XXII.103}{103}{}
                        So in that [first] context four senses are stated for each truth because in the case of each truth [individually] one sense becomes evident as the specific characteristic, while the other three become evident through seeing the remaining three truths.

                        At the path moment, however, all these senses are penetrated simultaneously by a single knowledge that has four functions with respect to suffering and the rest. But about those who would have it that [the different truths] are penetrated to separately, more is said in the Abhidhamma in the Kathāvatthu (\textbf{\cite{Kv}212–220}).
                    \subsubsection[\vismAlignedParas{§104–128}The Four Functions Described Separately]{The Four Functions Described Separately}

                        \vismParagraph{XXII.104}{104}{}
                        \emph{7.} Now, as to those four functions beginning with full-understanding, which were mentioned above (\hyperlink{XXII.92}{§92}{}):
                        \begin{verse}
                            (a) Full-understanding is threefold;\\{}
                            So too (b) abandoning, and (c) realizing,\\{}
                            And (d) two developings are reckoned—\\{}
                            Thus should be known the exposition.
                        \end{verse}


                        \vismParagraph{XXII.105}{105}{}
                        (a) \emph{Full-understanding is threefold}, that is, (i) full understanding as the known, (ii) full-understanding as investigating (judging), and (iii) full-understanding as abandoning (see \hyperlink{XX.3}{XX.3}{}).

                        \vismParagraph{XXII.106}{106}{}
                        (i) Herein, \emph{full-understanding as the known} \textcolor{brown}{\textit{[693]}} is summarized thus: “Understanding that is direct-knowledge is knowledge in the sense of the \marginnote{\textcolor{teal}{\footnotesize\{782|724\}}}{}known” (\textbf{\cite{Paṭis}I 87}). It is briefly stated thus: “Whatever states are directly known are known” (\textbf{\cite{Paṭis}I 87}). It is given in detail in the way beginning: “Bhikkhus, all is to be directly known. And what is all that is to be directly known? Eye is to be directly known …” (\textbf{\cite{Paṭis}I 5}). Its particular plane is the direct knowing of mentality-materiality with its conditions.

                        \vismParagraph{XXII.107}{107}{}
                        (ii) \emph{Full-understanding as investigating (judging)} is summarized thus: “Understanding that is full-understanding is knowledge in the sense of investigation (judging)” (\textbf{\cite{Paṭis}I 87}). It is briefly stated thus: “Whatever states are fully understood are investigated (judged)” (\textbf{\cite{Paṭis}I 87}). It is given in detail in the way beginning: “Bhikkhus, all is to be fully understood. And what is all that is to be fully understood? The eye is to be fully understood …” (\textbf{\cite{Paṭis}I 22}) Its particular plane starts with comprehension by groups, and occurring as investigation of impermanence, suffering, and not-self, it extends as far as conformity (cf. \hyperlink{XX.4}{XX.4}{}).

                        \vismParagraph{XXII.108}{108}{}
                        (iii) \emph{Full-understanding as abandoning} is summarized thus: “Understanding that is abandoning is knowledge in the sense of giving up” (\textbf{\cite{Paṭis}I 87}). It is stated in detail thus: Whatever states are abandoned are given up” (\textbf{\cite{Paṭis}I 87}). It occurs in the way beginning: “Through the contemplation of impermanence he abandons the perception of permanence …” (cf. \textbf{\cite{Paṭis}I 58}). Its plane extends from the contemplation of dissolution up to path knowledge. This is what is intended here.

                        \vismParagraph{XXII.109}{109}{}
                        Or alternatively, full-understanding as the known and full-understanding as investigating have that [third kind] as their aim, too, and whatever states a man abandons are certainly known and investigated, and so all three kinds of full-understanding can be understood in this way as the function of path knowledge.

                        \vismParagraph{XXII.110}{110}{}
                        (b) \emph{So too abandoning}: abandoning is threefold too, like full-understanding, that is, (i) abandoning by suppressing, (ii) abandoning by substitution of opposites, and (iii) abandoning by cutting off.

                        \vismParagraph{XXII.111}{111}{}
                        (i) Herein, when any of the mundane kinds of concentration suppresses opposing states such as the hindrances, that act of suppressing, which is like the pressing down of water-weed by placing a porous pot on weed-filled water, is called \emph{abandoning by suppressing}. But the suppression of only the hindrances is given in the text thus: “And there is abandoning of the hindrances by suppression in one who develops the first jhāna” (\textbf{\cite{Paṭis}I 27}). However, that should be understood as so stated because of the obviousness [of the suppression then]. For even before and after the jhāna as well hindrances do not invade consciousness suddenly; but applied thought, etc., [are suppressed] only at the moment of actual absorption [in the second jhāna, etc.,] and so the suppression of the hindrances then is obvious.

                        \vismParagraph{XXII.112}{112}{}
                        (ii) But what is called \emph{abandoning by substitution of opposites} is the abandoning of any given state that ought to be abandoned through the means of a particular factor of knowledge, which as a constituent of insight is opposed to it, like the abandoning of darkness at night through the means of a light. \textcolor{brown}{\textit{[694]}} It is in fact the abandoning firstly of the [false] view of individuality through \marginnote{\textcolor{teal}{\footnotesize\{783|725\}}}{}the means of delimitation of mentality-materiality; the abandoning of both the no-cause view and the fictitious-cause view and also of the stain of doubt through the means of discerning conditions; the abandoning of apprehension of a conglomeration as “I” and “mine” through the means of comprehension by groups; the abandoning of perception of the path in what is not the path through the means of the definition of what is the path and what is not the path; the abandoning of the annihilation view through the means of seeing rise; the abandoning of the eternity view through the means of seeing fall; the abandoning of the perception of non-terror in what is terror through the means of appearance as terror; the abandoning of the perception of enjoyment through the means of seeing danger; the abandoning of the perception of delight through the means of the contemplation of dispassion (revulsion); the abandoning of lack of desire for deliverance through the means of desire for deliverance; the abandoning of non-reflection through the means of reflection; the abandoning of not looking on equably through the means of equanimity; the abandoning of apprehension contrary to truth through the means of conformity.

                        \vismParagraph{XXII.113}{113}{}
                        And also in the case of the eighteen principal insights the abandoning by substitution of opposites is: (1) the abandoning of the perception of the perception of permanence, through the means of the contemplation of impermanence; (2) of the perception of pleasure, through the means of the contemplation of pain; (3) of the perception of self, through the means of the contemplation of not-self; (4) of delight, through the means of the contemplation of dispassion (revulsion); (5) of greed, through the means of the contemplation of fading away; (6) of originating, through the means of the contemplation of cessation; (7) of grasping, through the means of the contemplation of relinquishment; (8) of the perception of compactness, through the means of the contemplation of destruction; (9) of accumulation, through the means of the contemplation of fall; (10) of the perception of lastingness, through the means of the contemplation of change; (11) of the sign, through the means of the contemplation of the signless; (12) of desire, through the means of the contemplation of the desireless; (13) of misinterpreting (insisting), through the means of the contemplation of voidness; (14) of misinterpreting (insisting) due to grasping at a core, through the means of insight into states that is higher understanding; (15) of misinterpreting (insisting) due to confusion, through the means of correct knowledge and vision; (16) of misinterpreting (insisting) due to reliance [on formations], through the means of the contemplation of danger [in them]; (17) of non-reflection, through the means of the contemplation of reflection; (18) of misinterpreting (insisting) due to bondage, through means of contemplation of turning away (cf. \textbf{\cite{Paṭis}I 47}).

                        \vismParagraph{XXII.114}{114}{}
                        Herein, (1)–(7) the way in which the abandoning of the perception of permanence, etc., takes place through the means of the seven contemplations beginning with that of impermanence has already been explained under the contemplation of dissolution (Ch. \hyperlink{XXI.15}{XXI.15f.}{}).

                        (8) \emph{Contemplation of destruction}, however, is the knowledge in one who effects the resolution of the compact and so sees destruction as “impermanent in the \marginnote{\textcolor{teal}{\footnotesize\{784|726\}}}{}sense of destruction.” Through the means of that knowledge there comes to be the abandoning of the perception of compactness.

                        \vismParagraph{XXII.115}{115}{}
                        (9) \emph{Contemplation of fall} is stated thus:
                        \begin{verse}
                            “Defining both to be alike\\{}
                            By inference from that same object.\\{}
                            Intentness on cessation—these\\{}
                            Are insight in the mark of fall” (\textbf{\cite{Paṭis}I 58}).
                        \end{verse}


                        It is intentness on cessation, in other words, on that same dissolution, after seeing dissolution of [both seen and unseen] formations by personal experience and by inference [respectively]. Through the means of that contemplation there comes to be the abandoning of accumulation. When a man sees with insight that “The things for the sake of which I might accumulate [kamma] are thus \textcolor{brown}{\textit{[695]}} subject to fall,” his consciousness no longer inclines to accumulation.

                        \vismParagraph{XXII.116}{116}{}
                        (10) \emph{Contemplation of change} is the act of seeing, according to the material septad, etc., how [momentary] occurrences [in continuity] take place differently by [gradually] diverging from any definition; or it is the act of seeing change in the two aspects of the ageing and the death of what is arisen. Through the means of that contemplation the perception of lastingness is abandoned.

                        \vismParagraph{XXII.117}{117}{}
                        (11) \emph{Contemplation of the signless} is the same as the contemplation of impermanence. Through its means the sign of permanence is abandoned.

                        (12) \emph{Contemplation of the desireless} is the same as the contemplation of pain.

                        Through its means desire for pleasure and hope for pleasure are abandoned.

                        (13) \emph{Contemplation of voidness} is the same as the contemplation of not-self. Through its means the misinterpreting (insisting) that “a self exists” (see \textbf{\cite{S}IV 400}) is abandoned.

                        \vismParagraph{XXII.118}{118}{}
                        (14) Insight into states that is higher understanding is stated thus:
                        \begin{verse}
                            “Having reflected on the object,\\{}
                            Dissolution he contemplates,\\{}
                            Appearance then as empty—this\\{}
                            Is insight of higher understanding” (\textbf{\cite{Paṭis}I 58}).
                        \end{verse}


                        Insight so described occurs after knowing materiality, etc., as object, by seeing the dissolution both of that object and of the consciousness whose object it was, and by apprehending voidness through the dissolution in this way: “Only formations break up. It is the death of formations. There is nothing else.” Taking that insight as higher understanding and as insight with respect to states, it is called “insight into states that is higher understanding.” Through its means misinterpreting (insisting) due to grasping at a core is abandoned, because it has been clearly seen that there is no core of permanence and no core of self.

                        \vismParagraph{XXII.119}{119}{}
                        (15)\emph{ Correct knowledge and vision} is the discernment of mentality-materiality with its conditions. Through its means misinterpreting (insisting) due to confusion that occurs in this way, “Was I in the past?” (\textbf{\cite{M}I 8}), and in this way, “The world was created by an Overlord,” are abandoned.

                        \vismParagraph{XXII.120}{120}{}
                        \marginnote{\textcolor{teal}{\footnotesize\{785|727\}}}{}(16) \emph{Contemplation of danger} is knowledge seeing danger in all kinds of becoming, etc., which as arisen owing to appearance as terror. Through its means misinterpreting (insisting) due to reliance is abandoned, since he does not see any [formation] to be relied on for shelter.

                        (17) \emph{Contemplation of reflection }is the reflection that effects the means to liberation. Through its means non-reflection is abandoned.

                        \vismParagraph{XXII.121}{121}{}
                        (18) \emph{Contemplation of turning away} is equanimity about formations and conformity. For at that point his mind is said to retreat, retract and recoil from the whole field of formations, as a water drop does on a lotus leaf that slopes a little. That is why through its means misinterpreting (insisting) due to bondage is abandoned. \textcolor{brown}{\textit{[696]}} The meaning is: abandoning of the occurrence of defilement that consists in misinterpreting defiled by the bondage of sense desires, and so on.

                        Abandoning by substitution of the opposites should be understood in detail in this way. But in the texts it is stated in brief thus: “Abandoning of views by substitution of opposites comes about in one who develops concentration partaking of penetration” (\textbf{\cite{Paṭis}I 27}).

                        \vismParagraph{XXII.122}{122}{}
                        (iii) The abandoning of the states beginning with the fetters by the noble path knowledge in such a way that they never occur again, like a tree struck by a thunderbolt, is called \emph{abandoning by cutting off}. With reference to this it is said: “Abandoning by cutting off comes about in one who develops the supramundane path that leads to the destruction [of defilements]” (\textbf{\cite{Paṭis}I 27}).

                        \vismParagraph{XXII.123}{123}{}
                        So of these three kinds of abandoning, it is only abandoning by cutting off that is intended here. But since that meditator’s previous abandoning by suppression and by substitution by opposites have that [third kind] as their aim, too, all three kinds of abandoning can therefore be understood in this way as the function of path knowledge. For when a man has gained an empire by killing off the opposing kings, what was done by him previous to that is also called “done by the king.”

                        \vismParagraph{XXII.124}{124}{}
                        (c) \emph{Realizing} is divided into two as (i) mundane realizing, and (ii) supramundane realizing. And it is threefold too with the subdivision of the supramundane into two as seeing and developing.

                        \vismParagraph{XXII.125}{125}{}
                        (i) Herein, the touch (\emph{phassanā}) of the first jhāna, etc., as given in the way beginning, “I am an obtainer, a master, of the first jhāna; the first jhāna has been realized by me” (\textbf{\cite{Vin}III 93–94}), is called \emph{mundane realizing}. “Touch” (\emph{phassanā}) is the touching (\emph{phusanā}) with the contact (\emph{phassa}) of knowledge by personal experience on arriving, thus, “This has been arrived at by me”.\footnote{\vismAssertFootnoteCounter{21}\vismHypertarget{XXII.n21}{}“‘\emph{With the contact of knowledge by personal experience}’ means by personal experience of it as object, which is what the ‘contact of knowledge’ is called. The words, ‘By personal experience’ exclude taking it as an object by inference. For what is intended here as the ‘contact of knowledge’ is knowing by personal experience through reviewing thus, ‘This is like this’” (\textbf{\cite{Vism-mhṭ}888}).} With reference to this meaning realization is summarized thus, “Understanding that is \marginnote{\textcolor{teal}{\footnotesize\{786|728\}}}{}realization is knowledge in the sense of touch” (\textbf{\cite{Paṭis}I 87}), after which it is described thus, “Whatever states are realized are touched” (\textbf{\cite{Paṭis}I 87}).

                        \vismParagraph{XXII.126}{126}{}
                        Also, those states which are not aroused in one’s own continuity and are known through knowledge that depends on another are realized; for it is said, referring to that, “Bhikkhus, all should be realized. And what is all that should be realized? The eye should be realized” (\textbf{\cite{Paṭis}I 35}), and so on. And it is further said: “One who sees materiality realizes it. One who sees \textcolor{brown}{\textit{[697]}} feeling … perception … formations … consciousness realizes it. One who sees the eye … (etc., see \hyperlink{XX.9}{XX.9}{}) … ageing and death realizes it. [One who sees suffering] … (etc.)\footnote{\vismAssertFootnoteCounter{22}\vismHypertarget{XXII.n22}{}The first elision here—“The eye … ageing-and-death”—is explained in \hyperlink{XX.9}{XX.9}{}. The second elision—“One who sees suffering … One who sees Nibbāna, which merges in the deathless in the sense of end …”—covers all things listed from \textbf{\cite{Paṭis}I 8}, line 18 (N.B. the new para in the \textbf{\cite{Paṭis}} text should begin with the words “\emph{dukkhaṃ abhiññeyyaṃ}” up to p. 22, line 11, \emph{amatogadhaṃ nibbānaṃ pariyosānatthaṃ abhiññeyyaṃ}). In this case, however (\textbf{\cite{Paṭis}I 35}), \emph{sacchikātabba} (“to be realized”), etc., is substituted for \emph{abhiññeyya }(“to be directly known”).} … One who sees Nibbāna, which merges in the deathless [in the sense of the end] realizes it. Whatever states are realized are touched” (\textbf{\cite{Paṭis}I 35}).

                        \vismParagraph{XXII.127}{127}{}
                        (ii) The seeing of Nibbāna at the moment of the first path is \emph{realizing as seeing}. At the other path moments it is \emph{realizing as developing}. And it is intended as twofold here. So realizing of Nibbāna as seeing and as developing should be understood as a function of this knowledge.

                        \vismParagraph{XXII.128}{128}{}
                        (d) \emph{And two developings are reckoned}: but developing is also reckoned as twofold, namely as (i) mundane developing, and (ii) as supramundane developing.

                        (i) Herein, the arousing of mundane virtue, concentration and understanding, and the influencing of the continuity by their means, is \emph{mundane developing}. And (ii) the arousing of supramundane virtue, concentration and understanding, and the influencing of the continuity by them, is \emph{supramundane developing}. Of these, it is the supramundane that is intended here. For this fourfold knowledge arouses supramundane virtue, etc., since it is their conascence condition, and it influences the continuity by their means. So it is only supramundane developing that is a function of it. Therefore these are the:
                        \begin{verse}
                            Functions of full-understanding, and the rest\\{}
                            As stated when truths are penetrated to,\\{}
                            Each one of which ought to be recognized\\{}
                            According to its individual essence.
                        \end{verse}

                \subsection[\vismAlignedParas{§129}Conclusion]{Conclusion}

                    \vismParagraph{XXII.129}{129}{}
                    Now, with reference to the stanza:
                    \begin{verse}
                        “When a wise man, established well in virtue,\\{}
                        Develops consciousness and understanding” (\hyperlink{I.1}{I.1}{}),
                    \end{verse}


                    it was said above “After he has perfected the two purifications that are the ‘roots,’ then he can develop the five purifications that are the ‘trunk’”(\hyperlink{XIV.32}{XIV.32}{}). And at this point the detailed exposition of the system for developing \marginnote{\textcolor{teal}{\footnotesize\{787|729\}}}{}understanding in the proper way as it has been handed down is completed. So the question, “How should it be developed?” (\hyperlink{XIV.32}{XIV.32}{}) is now answered.

                    The twenty-second chapter called “The Description of Purification by Knowledge and Vision” in the Treatise on the Development of Understanding in the \emph{Path of Purification }composed for the purpose of gladdening good people.
        \chapter[The Benefits In Developing Understanding]{The Benefits In Developing Understanding\vismHypertarget{XXIII}\newline{\textnormal{\emph{Paññābhāvanānisaṃsa-niddesa}}}}
            \label{XXIII}

            \section[\vismAlignedParas{§1–60}(vi) What are the benefits in developing understanding?]{(vi) What are the benefits in developing understanding?}

                \vismParagraph{XXIII.1}{1}{}
                \marginnote{\textcolor{teal}{\footnotesize\{788|730\}}}{}(vi) \textsc{What are the benefits in developing understanding?} (See \hyperlink{XIV.1}{XIV.1}{})\textbf{ }\textcolor{brown}{\textit{[698]}}

                To that question, which was asked above, we reply that this development of understanding has many hundred benefits. But it would be impossible to explain its benefits in detail, however long a time were taken over it. Briefly, though, its benefits should be understood as these: (A) removal of the various defilements, (B) experience of the taste of the noble fruit, (C) ability to attain the attainment of cessation, and (D) achievement of worthiness to receive gifts and so on.
                \subsection[\vismAlignedParas{§2}A. Removal of the Defilements]{A. Removal of the Defilements}

                    \vismParagraph{XXIII.2}{2}{}
                    Herein, it should be understood that one of the benefits of the mundane development of understanding is the removal of the various defilements beginning with [mistaken] view of individuality. This starts with the delimitation of mentality-materiality. Then one of the benefits of the supramundane development of understanding is the removal, at the path moment, of the various defilements beginning with the fetters.
                    \begin{verse}
                        With dreadful thump the thunderbolt\\{}
                        Annihilates the rock.\\{}
                        The fire whipped by the driving wind\\{}
                        Annihilates the wood.
                    \end{verse}

                    \begin{verse}
                        The radiant orb of solar flame\\{}
                        Annihilates the dark.\\{}
                        Developed understanding, too,\\{}
                        Annihilates inveterate
                    \end{verse}

                    \begin{verse}
                        Defilements’ netted overgrowth,\\{}
                        The source of every woe.\\{}
                        This blessing in this very life\\{}
                        A man himself may know.
                    \end{verse}

                \subsection[\vismAlignedParas{§3–15}B. The Taste of the Noble Fruit]{B. The Taste of the Noble Fruit}

                    \vismParagraph{XXIII.3}{3}{}
                    Not only the removal of the various defilements but also the experience of the taste of the noble fruit is a benefit of the development of understanding. \textcolor{brown}{\textit{[699]}} For it is the fruitions of stream-entry, etc.—the fruits of asceticism—that are called \marginnote{\textcolor{teal}{\footnotesize\{789|731\}}}{}the “noble fruit.” Its taste is experienced in two ways, that is to say, in its occurrence in the cognitive series of the path, and in its occurrence in the attainment of fruition. Of these, only its occurrence in the cognitive series of the path has been shown (\hyperlink{XXII.3}{XXII.3f.}{}).

                    \vismParagraph{XXIII.4}{4}{}
                     Furthermore, when people say that the fruit is the mere abandoning of fetters\footnote{\vismAssertFootnoteCounter{1}\vismHypertarget{XXIII.n1}{}“It is the Andhakas, etc., who maintain this; for they take the sutta wrongly which says, ‘“Arahantship” is said, friend Sāriputta; what is Arahantship?—The destruction of greed, the destruction of hate, the destruction of delusion: that is what is called Arahantship” (\textbf{\cite{S}IV 252}), taking it literally and asserting that nothing exists called Arahantship and that it is only the abandoning of defilements that is so called by common usage. And they deny that there are any other fruitions” (\textbf{\cite{Vism-mhṭ}891}).} and nothing more than that, the following sutta can be cited in order to convince them that they are wrong: “How is it that understanding of the tranquilizing of effort is knowledge of fruit? At the moment of the stream-entry path right view in the sense of seeing emerges from wrong view, and it emerges from the defilements and from the aggregates that occur consequent upon that [wrong view], and externally it emerges from all signs. Right view arises because of the tranquilizing of that effort. This is the fruit of the path” (\textbf{\cite{Paṭis}I 71}), and this should be given in detail. Also such passages as, “The four paths and the four fruits—these states have a measureless object” (\textbf{\cite{Dhs}§1408}), and, “An exalted state is a condition, as proximity condition, for a measureless state” (\textbf{\cite{Paṭṭh}II 227} (Be)), establish the meaning here.

                    \vismParagraph{XXIII.5}{5}{}
                    However, in order to show how it occurs in the attainment of fruition there is the following set of questions:

                    (i) What is fruition attainment? (ii) Who attains it? (iii) Who do not attain it? (iv) Why do they attain it? (v) How does its attainment come about? (vi) How is it made to last? (vii) How does the emergence from it come about? (viii) What is next to fruition? (ix) What is fruition next to?

                    \vismParagraph{XXIII.6}{6}{}
                    Herein, (i) \emph{What is fruition attainment}? It is absorption in the cessation in which the noble fruition consists.

                    (ii) \emph{Who attains it?} (iii) \emph{Who do not attain it}? No ordinary men attain it. Why? Because it is beyond their reach. But all Noble Ones attain it. Why? Because it is within their reach. But those who have reached a higher path do not attain a lower fruition because the state of each successive person is more tranquilized than the one below. And those who have only reached a lower path do not attain a higher fruition because it is beyond their reach. But each one attains his appropriate fruition. This is what has been agreed here.

                    \vismParagraph{XXIII.7}{7}{}
                    But there are some who say that the stream-enterer and once-returner do not attain it, and that only the two above them attain it. The reason they give is that \marginnote{\textcolor{teal}{\footnotesize\{790|732\}}}{}only these two show achievement in concentration. But that is no reason, since even the ordinary man attains such mundane concentration as is within his reach. But why argue here over what is and what is not a reason? Is it not said in the texts as follows?

                    “Which ten states of change-of-lineage arise \textcolor{brown}{\textit{[700]}} through insight?

                    “For the purpose of obtaining the stream-entry path it overcomes arising, occurrence … (etc., see \hyperlink{XXII.5}{XXII.5}{}) … despair, and externally the sign of formations, thus it is change-of-lineage.

                    “For the purpose of attaining the stream-entry fruition …

                    “For the purpose of attaining the once-return path …

                    “For the purpose of attaining the once-return fruition …

                    “For the purpose of attaining the non-return path …

                    “For the purpose of attaining the non-return fruition …

                    “For the purpose of attaining the Arahant path …

                    “For the purpose of attaining the Arahant fruition …

                    “For the purpose of attaining the void abiding …

                    “For the purpose of attaining the signless abiding it overcomes arising, occurrence … (etc.) … despair, and externally the sign of formations, thus it is change-of-lineage” (\textbf{\cite{Paṭis}I 68}).\footnote{\vismAssertFootnoteCounter{2}\vismHypertarget{XXIII.n2}{}The quotation in the Vism texts does not quite agree with the \textbf{\cite{Paṭis}} text (Ee) where (as the sense demands) the words “\emph{bahiddhā saṅkhāranimittaṃ}” do not follow the four fruitions and the two abidings but only the four paths.} From that it must be concluded that all Noble Ones attain each their own fruit.

                    \vismParagraph{XXIII.8}{8}{}
                    (iv) \emph{Why do they attain it}? For the purpose of abiding in bliss here and now. For just as a king experiences royal bliss and a deity experiences divine bliss, so too the Noble Ones think, “We shall experience the noble supramundane bliss,” and after deciding on the duration, they attain the attainment of fruition whenever they choose.\footnote{\vismAssertFootnoteCounter{3}\vismHypertarget{XXIII.n3}{}“Although they are resultant states, nevertheless the states of fruition attainment occur in the noble person only when he chooses since they do not arise without the preliminary work and do so only when they are given predominance” (\textbf{\cite{Vism-mhṭ}895}).}

                    \vismParagraph{XXIII.9}{9}{}
                    (v) \emph{How does its attainment come about}? (vi) \emph{How is it made to last}? (vii) \emph{How does the emergence from it come about}?

                    (v) In the first place its attainment comes about for two reasons: with not bringing to mind any object other than Nibbāna, and with bringing Nibbāna to mind, according as it is said: “Friend, there are two conditions for the attainment of the signless mind-deliverance; they are the non-bringing to mind of all signs, and the bringing to mind of the signless element” (\textbf{\cite{M}I 296}).

                    \vismParagraph{XXIII.10}{10}{}
                    Now, the process of attaining it is as follows. A noble disciple who seeks the attainment of fruition should go into solitary retreat. He should see formations with insight according to rise and fall and so on. When that insight has progressed [as far as conformity], then comes change-of-lineage knowledge \marginnote{\textcolor{teal}{\footnotesize\{791|733\}}}{}with formations as its object.\footnote{\vismAssertFootnoteCounter{4}\vismHypertarget{XXIII.n4}{}“Why does change-of-lineage not have Nibbāna as its object here as it does when it precedes the path? Because states belonging to fruition are not associated with an outlet [as in the case of the path]. For this is said: ‘What states are an outlet? The four unincluded paths’ (\textbf{\cite{Dhs}§1592})” (\textbf{\cite{Vism-mhṭ}895}).} And immediately next to it consciousness becomes absorbed in cessation with the attainment of fruition. And here it is only fruition, not path, that arises even in a trainer, because his tendency is to fruition attainment.

                    \vismParagraph{XXIII.11}{11}{}
                    But there are those\footnote{\vismAssertFootnoteCounter{5}\vismHypertarget{XXIII.n5}{}“Those of the Abhayagiri Monastery in Anurādhapura” (\textbf{\cite{Vism-mhṭ}895}).} who say that when a stream-enterer embarks on insight, thinking, “I shall attain fruition attainment,” he becomes a once-returner, and a once-returner, a non-returner. They should be told: “In that case a non-returner becomes an Arahant and an Arahant, a Paccekabuddha and a Paccekabuddha, a Buddha. For that reason, and because it is contradicted as well by the text quoted above, none of that should be accepted. Only this should be accepted: fruition itself, not path, arises also in the trainer. And if the path he has arrived at had the first jhāna, his fruition will have the first jhāna too when it arises. If the path has the second, so will the fruition. And so with the other jhānas.”

                    This, firstly, is how attaining comes about. \textcolor{brown}{\textit{[701]}}

                    \vismParagraph{XXIII.12}{12}{}
                    (vi) It is made to last in three ways, because of the words: “Friend, there are three conditions for the persistence of the signless mind-deliverance: they are the non-bringing to mind of all signs, the bringing to mind of the signless element, and the \emph{prior volition}” (\textbf{\cite{M}I 296–297}). Herein, the prior volition is the predetermining of the time before attaining;\footnote{\vismAssertFootnoteCounter{6}\vismHypertarget{XXIII.n6}{}“The ‘\emph{volition}’ is attaining after deciding the time limit in this way, ‘When the moon, or the sun, has gone so far, I shall emerge,’ which is an act of volition” (\textbf{\cite{Vism-mhṭ}897}).} for it is by determining it thus, “I shall emerge at such a time,” that it lasts until that time comes. This is how it is made to last.

                    \vismParagraph{XXIII.13}{13}{}
                    (vii) Emergence from it comes about in two ways, because of the words: “Friend, there are two conditions for the emergence from the signless mind-deliverance: they are the bringing to mind of all signs, and the non-bringing to mind of the signless element” (\textbf{\cite{M}I 297}). Herein, of all signs means the sign of materiality, sign of feeling, perception, formations, and consciousness. Of course, a man does not bring all those to mind at once, but this is said in order to include all. So the emergence from the attainment of fruition comes about in him when he brings to mind whatever is the object of the life-continuum.\footnote{\vismAssertFootnoteCounter{7}\vismHypertarget{XXIII.n7}{}“It is because he is called ‘emerged from attainment’ as soon as the life-continuum consciousness has arisen that ‘\emph{he brings to mind that which is the object of the life-continuum}’ is said. Kamma, etc., are called the object of the life-continuum (see \hyperlink{XVII.133}{Ch. XVII, §133ff.}{})” (\textbf{\cite{Vism-mhṭ}897}).}

                    \vismParagraph{XXIII.14}{14}{}
                    (viii) \emph{What is next to fruition}? (ix) \emph{What is fruition next to}? In the first case (viii) either fruition itself is next to fruition or the life-continuum is next to it. But (ix) there is fruition that is (a) next to the path, (b) there is that next to fruition, (c) there is that next to change-of-lineage, and (d) there is that next to the base consisting of neither perception nor non-perception. \marginnote{\textcolor{teal}{\footnotesize\{792|734\}}}{}Herein, (a) it is \emph{next to the path} in the cognitive series of the path. (b) Each one that is subsequent to a previous one is \emph{next to fruition}. (c) Each first one in the attainments of fruition is \emph{next to change-of-lineage}. And conformity should be understood here as “change-of-lineage”; for this is said in the Paṭṭhāna: “In the Arahant, conformity is a condition, as proximity condition, for fruition attainment. In trainers, conformity is a condition, as proximity condition, for fruition attainment” (\textbf{\cite{Paṭṭh}I 159}). (d) The fruition by means of which there is emergence from the attainment of cessation is \emph{next to the base consisting of neither perception non-perception}.

                    \vismParagraph{XXIII.15}{15}{}
                    Herein, all except the fruition that arises in the cognitive series of the path occur as fruition attainment. So whether it arises in the cognitive series of the path or in fruition attainment:
                    \begin{verse}
                        Asceticism’s fruit sublime,\\{}
                        Which tranquilizes all distress,\\{}
                        Its beauty from the Deathless draws,\\{}
                        Its calm from lack of worldliness. \textcolor{brown}{\textit{[702]}}
                    \end{verse}

                    \begin{verse}
                        Of a sweet purifying bliss\\{}
                        It is the fountainhead besides,\\{}
                        Whose honey-sweet ambrosia\\{}
                        A deathless sustenance provides.
                    \end{verse}

                    \begin{verse}
                        Now, if a wise man cultivates\\{}
                        His understanding, he shall know\\{}
                        This peerless bliss, which is the taste\\{}
                        The noble fruit provides; and so
                    \end{verse}

                    \begin{verse}
                        This is the reason why they call\\{}
                        Experience here and now aright\\{}
                        Of flavour of the noble fruit\\{}
                        A blessing of fulfilled insight.
                    \end{verse}

                \subsection[\vismAlignedParas{§16–52}C. The Attainment of Cessation]{C. The Attainment of Cessation}

                    \vismParagraph{XXIII.16}{16}{}
                    And not only the experience of the taste of the noble fruit but also the ability to attain the attainment of cessation should be understood as a benefit of the development of understanding.

                    \vismParagraph{XXIII.17}{17}{}
                    Now, in order to explain the attainment of cessation there is this set of questions:

                    
                        \begin{enumerate}[(i),nosep]
                            \item What is the attainment of cessation?
                            \item Who attains it?
                            \item Who do not attain it?
                            \item Where do they attain it?
                            \item Why do they attain it?
                            \item How does its attainment come about?
                            \item How is it made to last?
                            \item How does the emergence from it come about?
                            \item Towards what does the mind of one who has emerged tend?
                            \item \marginnote{\textcolor{teal}{\footnotesize\{793|735\}}}{}What is the difference between one who has attained it and one who is dead?
                            \item Is the attainment of cessation formed or unformed, mundane or supramundane, produced or unproduced?
                        \end{enumerate}

                    \vismParagraph{XXIII.18}{18}{}
                    Herein, (i) \emph{What is the attainment of cessation}? It is the non-occurrence of consciousness and its concomitants owing to their progressive cessation.

                    (ii) \emph{Who attains it}? (iii) \emph{Who do not attain it}? No ordinary men, no stream-enterers or once-returners, and no non-returners and Arahants who are bare-insight workers attain it. But both non-returners and those with cankers destroyed (Arahants) who are obtainers of the eight attainments attain it. For it is said: “Understanding that is mastery, owing to possession of two powers, to the tranquilization of three formations, to sixteen kinds of exercise of knowledge, and to nine kinds of exercise of concentration, is knowledge of the attainment of cessation” (\textbf{\cite{Paṭis}I 97}). And these qualifications are not to be found together in any persons other than non-returners and those whose cankers are destroyed, who are obtainers of the eight attainments. That is why only they and no others attain it.

                    \vismParagraph{XXIII.19}{19}{}
                    But which are the \emph{two powers}? And the [\emph{three formations}] … and mastery? Here there is no need for us to say anything; for it has all been said in the description of the summary [quoted above], according as it is said:

                    \vismParagraph{XXIII.20}{20}{}
                    “\emph{Of the two powers}: of the two powers, the serenity power and the insight power. \textcolor{brown}{\textit{[703]}}

                    “What is \emph{serenity} as a power? The unification of the mind and non-distraction due to renunciation are serenity as a power. The unification of the mind and non-distraction due to non-ill will are serenity as a power. The unification of the mind and non-distraction due to perception of light … [to non-distraction … to defining of states (\emph{dhamma}) … to knowledge … to gladness … to the eight attainments, the ten kasiṇas, the ten recollections, the nine charnel-ground contemplations, and the thirty-two modes of mindfulness of breathing]\footnote{\vismAssertFootnoteCounter{8}\vismHypertarget{XXIII.n8}{}The list in brackets represents in summarized form the things listed at \textbf{\cite{Paṭis}I 94–95}, repeated in this context in the Paṭisambhidā but left out in the Vism quotation.} … the unification of the mind and non-distraction due to breathing out in one who is contemplating relinquishment\footnote{\vismAssertFootnoteCounter{9}\vismHypertarget{XXIII.n9}{}The serenity shown here is access concentration (see \textbf{\cite{Vism-mhṭ}899}).} is serenity as a power.

                    \vismParagraph{XXIII.21}{21}{}
                    “In what sense is serenity a \emph{power}? Owing to the first jhāna it does not waver on account of the hindrances, thus serenity is a power. Owing to the second jhāna it does not waver on account of applied and sustained thought, thus serenity is a power … (etc.) … Owing to the base consisting of neither perception nor non-perception it does not waver on account of the perception of the base consisting of nothingness, thus serenity is a power. It does not waver and vacillate and hesitate on account of agitation and on account of the defilements and the aggregates that accompany agitation, thus serenity is a power. This is the serenity power.

                    \vismParagraph{XXIII.22}{22}{}
                    \marginnote{\textcolor{teal}{\footnotesize\{794|736\}}}{}“What is \emph{insight} as a power? Contemplation of impermanence is insight as a power. Contemplation of pain … Contemplation of not-self … Contemplation of dispassion … Contemplation of fading away … Contemplation of cessation … Contemplation of relinquishment is insight as a power. Contemplation of impermanence in materiality … (etc.) … Contemplation of relinquishment in materiality is insight as a power. Contemplation of impermanence in feeling … in perception … in formations … in consciousness is insight as a power … Contemplation of relinquishment in consciousness is insight as a power. Contemplation of impermanence in the eye … (etc., see \hyperlink{XX.9}{XX.9}{}) … Contemplation of impermanence in ageing-and-death … (etc.) … Contemplation of relinquishment in ageing-and-death is insight as a power.

                    \vismParagraph{XXIII.23}{23}{}
                    “In what sense is insight a \emph{power}? Owing to the contemplation of impermanence it does not waver on account of perception of permanence, thus insight is a power. Owing to the contemplation of pain it does not waver on account of perception of pleasure … Owing to the contemplation of not-self it does not waver on account of the perception of self … Owing to the contemplation of dispassion it does not waver on account of delight … Owing to the contemplation of fading away it does not waver on account of greed … Owing to the contemplation of cessation it does not waver on account of arising … Owing to the contemplation of relinquishment it does not waver on account of grasping, thus insight is a power. It does not waver and vacillate and hesitate on account of ignorance and on account of the defilements and the aggregates that accompany ignorance, thus insight is a power.

                    \vismParagraph{XXIII.24}{24}{}
                    “\emph{Owing to the tranquilization of three formations}: owing to the tranquilization of what three formations? In one who has attained the second jhāna the verbal formations consisting in applied and sustained thought are quite tranquilized. In one who has attained the fourth jhāna the bodily formations consisting in in-breaths and out-breaths are quite tranquilized. In one who has attained cessation of perception and feeling the mental formations consisting in feeling and perception are quite tranquilized. It is owing to the tranquilization of these three formations.

                    \vismParagraph{XXIII.25}{25}{}
                    “\emph{Owing to sixteen kinds of exercise of knowledge}: owing to what sixteen kinds of exercise of knowledge? Contemplation of impermanence is a kind of exercise of knowledge. Contemplation of pain … Contemplation of not-self … Contemplation of dispassion … Contemplation of fading away … Contemplation of cessation … Contemplation of relinquishment … Contemplation of turning away is a kind of exercise of knowledge. \textcolor{brown}{\textit{[704]}} The stream-entry path is a kind of exercise of knowledge. The attainment of the fruition of stream-entry … The once-return path … The attainment of the fruition of once-return … The non-return path … The attainment of the fruition of non-return … The Arahant path … The attainment of the fruition of Arahantship is a kind of exercise of knowledge. It is owing to these sixteen kinds of exercise of knowledge.

                    \vismParagraph{XXIII.26}{26}{}
                    “\emph{Owing to nine kinds of exercise of concentration}: owing to what nine kinds of exercise of concentration? The first jhāna is a kind of exercise of concentration. The second jhāna … [The third jhāna … The fourth jhāna … \textbf{\cite{Th}} \marginnote{\textcolor{teal}{\footnotesize\{795|737\}}}{}e attainment of the base consisting of boundless space … The attainment of the base consisting of boundless consciousness … The attainment of the base consisting of nothingness … ]. The attainment of the base consisting of neither perception nor non-perception is a kind of exercise of concentration. And the applied thought and sustained thought and happiness and bliss and unification of mind that have the purpose of attaining the first jhāna … (etc.) … And the applied thought and sustained thought and happiness and bliss and unification of mind that have the purpose of attaining the attainment of the base consisting of neither perception nor non-perception. It is owing to these nine kinds of exercise of concentration.\footnote{\vismAssertFootnoteCounter{10}\vismHypertarget{XXIII.n10}{}The nine are the four fine-material jhānas, the four immaterial jhānas, and the access concentration preceding each of the eight attainments, described in the last sentence and counted as one.}

                    \vismParagraph{XXIII.27}{27}{}
                    “Mastery: there are five kinds of mastery. There is mastery in adverting, in attaining, in resolving, in emerging, in reviewing. He adverts to the first jhāna where, when, and for as long as he wishes, he has no difficulty in adverting, thus it is mastery in adverting. He attains the first jhāna where, when, and for as long as he wishes, he has no difficulty in attaining, thus it is mastery in attaining. He resolves upon [the duration of] the first jhāna where, … thus it is mastery in resolving. He emerges from the first jhāna, … thus it is mastery in emerging. He reviews the first jhāna where, when, and for as long as he wishes, he has no difficulty in reviewing, thus it is mastery in reviewing. He adverts to the second jhāna … (etc.) … He reviews the attainment of the base consisting of neither perception nor non-perception where, when, and for as long as he wishes, he has no difficulty in reviewing, thus it is mastery in reviewing. These are the five kinds of mastery” (\textbf{\cite{Paṭis}I 97–100}).

                    \vismParagraph{XXIII.28}{28}{}
                    And here the words: “Owing to sixteen kinds of exercise of knowledge” state the maximum. But in a non-returner the mastery is owing to fourteen kinds of exercise of knowledge. If that is so, then does it not come about also in the once-returner owing to twelve? And in the stream-enterer owing to ten?—It does not. Because the greed based on the cords of sense desire, which is an obstacle to concentration, is unabandoned in them. It is because that is not abandoned in them that the serenity power is not perfected. Since it is not perfected they are not, owing to want of power, able to attain the attainment of cessation, which has to be attained by the two powers. But it is abandoned in the non-returner and so his power is perfected. Since his power is perfected he is able to attain it.

                    Hence the Blessed One said: “Profitable [consciousness] of the base consisting of neither perception nor non-perception in one emerging from cessation is a condition, as proximity condition, for the attainment of fruition” (\textbf{\cite{Paṭṭh}I 159}). For this is said in the Great Book of the Paṭṭhāna\footnote{\vismAssertFootnoteCounter{11}\vismHypertarget{XXIII.n11}{}“The word ‘\emph{profitable}’ used in this Paṭṭhāna passage shows that it app1ies only to non-returners, otherwise ‘functional’ would have been said” (\textbf{\cite{Vism-mhṭ}902}).} with reference only to non-returners’ emerging from cessation. \textcolor{brown}{\textit{[705]}}

                    \vismParagraph{XXIII.29}{29}{}
                    \marginnote{\textcolor{teal}{\footnotesize\{796|738\}}}{}(iv) \emph{Where do they attain it}? In the five-constituent becoming. Why? Because of the necessity for the succession of [all] the attainments (cf. \textbf{\cite{S}IV 217}). But in the four-constituent becoming there is no arising of the first jhāna, etc., and so it is not possible to attain it there. But some say that is because of the lack of a physical basis [for the mind there].\footnote{\vismAssertFootnoteCounter{12}\vismHypertarget{XXIII.n12}{}“They say so because of absence of heart-basis; but the meaning is because of absence of basis called physical body. For if anyone were to attain cessation in the immaterial worlds he would become indefinable (\emph{appaññattika}) owing to the non-existence of any consciousness or consciousness concomitant at all, and he would be as though attained to final Nibbāna without remainder of results of past clinging; for what remainder of results of past clinging could be predicated of him when he had entered into cessation? So it is because of the lack of the necessary factors that there is no attaining of the attainment of cessation in the immaterial worlds” (\textbf{\cite{Vism-mhṭ}902}).}

                    \vismParagraph{XXIII.30}{30}{}
                    (v) \emph{Why do they attain it}? Being wearied by the occurrence and dissolution of formations, they attain it thinking, “Let us dwell in bliss by being without consciousness here and now and reaching the cessation that is Nibbāna.”\footnote{\vismAssertFootnoteCounter{13}\vismHypertarget{XXIII.n13}{}“‘\emph{Reaching the cessation that is Nibbāna}’: as though reaching Nibbāna without remainder of result of past clinging. ‘In bliss’ means without suffering” (\textbf{\cite{Vism-mhṭ}902}).}

                    \vismParagraph{XXIII.31}{31}{}
                    (vi) \emph{How does its attainment come about}? It comes about in one who performs the preparatory tasks by striving with serenity and insight and causes the cessation of [consciousness belonging to] the base consisting of neither perception nor non-perception. One who strives with serenity alone reaches the base consisting of neither perception nor non-perception and remains there, while one who strives with insight alone reaches the attainment of fruition and remains there. But it is one who strives with both, and after performing the preparatory tasks, causes the cessation of [consciousness belonging to] the base consisting of neither perception nor non-perception, who attains it. This is in brief.

                    \vismParagraph{XXIII.32}{32}{}
                    But the detail is this. When a bhikkhu who desires to attain cessation has finished all that has to do with his meal and has washed his hands and feet well, he sits down on a well-prepared seat in a secluded place. Having folded his legs crosswise, set his body erect, established mindfulness in front of him, he attains the first jhāna, and on emerging he sees the formations in it with insight as impermanent, painful, not-self.

                    \vismParagraph{XXIII.33}{33}{}
                    This insight is threefold as insight that discerns formations, insight for the attainment of fruition, and insight for the attainment of cessation. Herein, insight that discerns formations, whether sluggish or keen, is the proximate cause only for a path. Insight for the attainment of fruition, which is only valid when keen, is similar to that for the development of a path. Insight for the attainment of cessation is only valid when it is not over-sluggish and not over-keen. Therefore he sees those formations with insight that is not over-sluggish and not over-keen.

                    \vismParagraph{XXIII.34}{34}{}
                    After that, he attains the second jhāna, and on emerging he sees formations with insight in like manner. After that, he attains the third jhāna … (etc.) … After that, he attains the base consisting of boundless consciousness, and on emerging he sees the formations in it in like manner. Likewise he attains the base consisting \marginnote{\textcolor{teal}{\footnotesize\{797|739\}}}{}of nothingness. On emerging from that he does the fourfold preparatory task, that is to say, about (a) non-damage to others’ property, (b) the Community’s waiting, (c) the Master’s summons, and (d) the limit of the duration. \textcolor{brown}{\textit{[706]}}

                    \vismParagraph{XXIII.35}{35}{}
                    (a) Herein, \emph{non-damage to others’ property} refers to what the bhikkhu has about him that is not his personal property: a robe and bowl, or a bed and chair, or a living room, or any other kind of requisite kept by him but the property of various others. It should be resolved\footnote{\vismAssertFootnoteCounter{14}\vismHypertarget{XXIII.n14}{}“‘\emph{It should be resolved}’: the thought should be aroused. For here the resolve consists in arousing the thought. In the non-arising of consciousness-originated materiality, etc., and in the absence of support by a postnascence condition, etc., the physical body continues the same only for seven days; after that it suffers wastage. So he limits the duration to seven days when he attains cessation, they say” (\textbf{\cite{Vism-mhṭ}903}).} that such property will not be damaged, will not be destroyed by fire, water, wind, thieves, rats, and so on. Here is the form of the resolve: “During these seven days let this and this not be burnt by fire; let it not be swept off by water; let it not be spoilt by wind; let it not be stolen by thieves; let it not be devoured by rats, and so on.” When he has resolved in this way, they are not in danger during the seven days.

                    \vismParagraph{XXIII.36}{36}{}
                    If he does not resolve in this way, they may be destroyed by fire, etc., as in the case of the Elder Mahā Nāga. The elder, it seems, went for alms into the village where his mother, a lay follower, lived. She gave him rice gruel and seated him in the sitting hall. The elder sat down and attained cessation. While he was sitting there the hall caught fire. The other bhikkhus each picked up their seats and fled. The villagers gathered together, and seeing the elder, they said, “What a lazy monk! What a lazy monk!” The fire burned the grass thatch, the bamboos, and timbers, and it encircled the elder. People brought water and put it out. They removed the ashes, did repairs,\footnote{\vismAssertFootnoteCounter{15}\vismHypertarget{XXIII.n15}{}\emph{Paribhaṇḍa—}“repair work”: this meaning is not given in PED; cf. \textbf{\cite{M-a}IV 157} (patching of old robes), and \textbf{\cite{M-a}I 291}.} scattered flowers, and then stood respectfully waiting. The elder emerged at the time he had determined. Seeing them, he said, “I am discovered!,” and he rose up into the air and went to Piyaṅgu Island. This is “non-damage to others’ property.”

                    \vismParagraph{XXIII.37}{37}{}
                    There is no special resolving to be done for what is his own personal property such as the inner and outer robes or the seat he is sitting on. He protects all that by means of the attainment itself, like those of the venerable Sañjīva. And this is said: “There was success by intervention of concentration in the venerable Sañjīva. There was success by intervention of concentration in the venerable Sāriputta” (\textbf{\cite{Paṭis}I 212}—see \hyperlink{XII.30}{XII.30}{}).

                    \vismParagraph{XXIII.38}{38}{}
                    (b) \emph{The Community’s waiting} is the Community’s expecting. The meaning is: till this bhikkhu comes there is no carrying out of acts of the Community. And here it is not the actual Community’s waiting that is the preparatory task, but the adverting to the waiting. So it should be adverted to in this way: “While I am sitting for seven days in the attainment of cessation, if the Community wants to enact a resolution, etc., I shall emerge before any bhikkhu comes to summon me.” \textcolor{brown}{\textit{[707]}} One who attains it after doing this emerges at exactly that time.

                    \vismParagraph{XXIII.39}{39}{}
                    \marginnote{\textcolor{teal}{\footnotesize\{798|740\}}}{}But if he does not do so, then perhaps the Community assembles, and not seeing him, it is asked, “Where is the bhikkhu so and so?” They reply, “He has attained cessation.” The Community dispatches a bhikkhu, telling him, “Go and summon him in the name of the Community.” Then as soon as the bhikkhu stands within his hearing and merely says, “The Community is waiting for you, friend,” he emerges. Such is the importance of the Community’s order. So he should attain in such-wise that, by adverting to it beforehand, he emerges by himself.

                    \vismParagraph{XXIII.40}{40}{}
                    (c) \emph{The Master’s summons}: here too it is the adverting to the Master’s summons that is the preparatory task. So that also should be adverted to in this way: “While I am sitting for seven days in the attainment of cessation, if the Master, after examining a case, makes known a course of training, or teaches the Dhamma, the origin of which discourse is some need that has arisen,\footnote{\vismAssertFootnoteCounter{16}\vismHypertarget{XXIII.n16}{}The word \emph{atthuppatti} (“the origin being a need arisen”) is a technical commentarial term. “There are four kinds of origins (\emph{uppatti}) or setting forth of suttas (\emph{sutta-nikkhepa}): on account of the speaker’s own inclination (\emph{attajjhāsaya}), on account of another’s inclination (\emph{parajjhāsaya}), as the result of a question asked (\emph{pucchāvasika}), and on account of a need arisen (\emph{atthuppattika})’ (\textbf{\cite{M-a}I 15}, see also Ch. \hyperlink{III.88}{III.88}{}).} I shall emerge before anyone comes to summon me.” For when he has seated himself after doing so, he emerges at exactly that time.

                    \vismParagraph{XXIII.41}{41}{}
                    But if he does not do so, when the Community assembles, the Master, not seeing him, asks, “Where is the bhikkhu so and so?” They reply, “He has attained cessation.” Then he dispatches a bhikkhu, telling him, “Go and summon him in my name.” As soon as the bhikkhu stands within his hearing and merely says, “The Master calls the venerable one,” he emerges. Such is the importance of the Master’s summons. So he should attain in such wise that, by adverting to it beforehand, he emerges himself.

                    \vismParagraph{XXIII.42}{42}{}
                    (d) \emph{The limit of duration} is the limit of life’s duration. For this bhikkhu should be very careful to determine what the limit of his life’s duration is. He should attain only after adverting in this way: “Will my own vital formations go on occurring for seven days or will they not?” For if he attains it without adverting when the vital formations are due to cease within seven days, then since the attainment of cessation cannot ward off his death because there is no dying during cessation,\footnote{\vismAssertFootnoteCounter{17}\vismHypertarget{XXIII.n17}{}“‘\emph{Vital formations}’ are the same as 1ife span; though some say that they are the life span, heat and consciousness. These are the object only of his normal consciousness. There is no death during cessation because dying takes place by means of the final life-continuum [consciousness]. He should attain only after adverting thus, ‘Let sudden death not occur.’ For in the case of sudden death he would not be able to declare final knowledge, advise the bhikkhus, and testify to the Dispensation’s power. And there would be no reaching the highest path in the case of a non-returner” (\textbf{\cite{Vism-mhṭ}904}).} he consequently emerges from the attainment meanwhile. So he should attain only after adverting to that. For it is said that while it may be permissible to omit adverting to others, this must be adverted to.

                    \vismParagraph{XXIII.43}{43}{}
                    Now, when he has thus attained the base consisting of nothingness and emerged and done this preparatory task, he then attains the base consisting of \marginnote{\textcolor{teal}{\footnotesize\{799|741\}}}{}neither perception nor non-perception. Then after one or two turns of consciousness have passed, he becomes without consciousness, he achieves cessation. But why do consciousnesses not go on occurring in him after the two consciousnesses? Because the effort is directed to cessation. For this bhikkhu’s mounting through the eight attainments, coupling together the states of serenity and insight, \textcolor{brown}{\textit{[708]}} is directed to successive cessation, not to attaining the base consisting of neither perception nor non-perception. So it is because the effort is directed to cessation that no more than the two consciousnesses occur.

                    \vismParagraph{XXIII.44}{44}{}
                    But if a bhikkhu emerges from the base consisting of nothingness without having done this preparatory task and then attains the base consisting of neither perception nor non-perception, he is unable then to become without consciousness: he returns to the base consisting of nothingness and settles down there.

                    \vismParagraph{XXIII.45}{45}{}
                    And here the simile of the man and the road not previously travelled may be told. A man who had not previously travelled a certain road came to a ravine cut by water, or after crossing a deep morass he came to a rock heated by a fierce sun. Then without arranging his inner and outer garments, he descended into the ravine but came up again for fear of wetting his belongings and remained on the bank, or he walked up on to the rock but on burning his feet he returned to the near side and waited there.

                    \vismParagraph{XXIII.46}{46}{}
                    Herein, just as the man, as soon as he had descended into the ravine, or walked up on to the hot rock, turned back and remained on the near side because he had not seen to the arrangement of his inner and outer garments, so too as soon as the meditator has attained the base consisting of neither perception nor non-perception, he turns back and remains in the base consisting of nothingness because the preparatory task has not been done.

                    \vismParagraph{XXIII.47}{47}{}
                    Just as when a man who has travelled that road before comes to that place, he puts his inner garment on securely, and taking the other in his hand, crosses over the ravine, or so acts as to tread only lightly on the hot rock and accordingly gets to the other side, so too, when the bhikkhu does the preparatory task and then attains the base consisting of neither perception nor non-perception, then he achieves cessation, which is the other side, by becoming without consciousness.

                    \vismParagraph{XXIII.48}{48}{}
                    (vii) \emph{How is it made to last}? It lasts as long as the time predetermined for its duration, unless interrupted meanwhile by the exhaustion of the life span, by the waiting of the Community, or by the Master’s summons.

                    \vismParagraph{XXIII.49}{49}{}
                    (viii) \emph{How does the emergence from it come about}? The emergence comes about in two ways thus: by means of the fruition of non-return in the case of the non-returner, or by means of the fruition of Arahantship in the case of the Arahant.

                    \vismParagraph{XXIII.50}{50}{}
                    (ix) \emph{Towards what does the mind of one who has emerged tend}? It tends towards Nibbāna. For this is said: “When a bhikkhu has emerged from the attainment of the cessation of perception and feeling, friend Visākha, his consciousness inclines to seclusion, leans to seclusion, tends to seclusion” (\textbf{\cite{M}I 302}). \textcolor{brown}{\textit{[709]}}

                    \vismParagraph{XXIII.51}{51}{}
                    \marginnote{\textcolor{teal}{\footnotesize\{800|742\}}}{}(x) \emph{What is the difference between one who has attained and one who is dead}? This is also given in a sutta, according as it is said: “When a bhikkhu is dead, friend, has completed his term, his bodily formations have ceased and are quite still, his verbal formations have ceased and are quite still, his mental formations have ceased and are quite still, his life is exhausted, his heat has subsided, and his faculties are broken up. When a bhikkhu has entered upon the cessation of perception and feeling, his bodily formations have ceased and are quite still, his verbal formations have ceased and are quite still, his mental formations have ceased and are quite still, his life is unexhausted, his heat has not subsided, his faculties are quite whole” (\textbf{\cite{M}I 296}).

                    \vismParagraph{XXIII.52}{52}{}
                    (xi) \emph{As to the question is the attainment of cessation formed or unformed, etc}.? It is not classifiable as formed or unformed, mundane or supramundane. Why? Because it has no individual essence. But since it comes to be attained by one who attains it, it is therefore permissible to say that it is produced, not unproduced.\footnote{\vismAssertFootnoteCounter{18}\vismHypertarget{XXIII.n18}{}The subtleties of the word \emph{nipphanna} are best cleared up by quoting a paragraph from the \emph{Sammohavinodanī} (\textbf{\cite{Vibh-a}29}): “The five aggregates are positively-produced (\emph{parinipphanna}) always, not un-positively-produced (\emph{aparinipphanna}); they are always formed, not unformed. Besides, they are produced (\emph{nipphanna}) as well. For among the dhammas that are individual essences (\emph{sabhāva-dhamma}) it is only Nibbāna that is un-positively-produced and un-produced (\emph{anipphanna}).” The \emph{Mūla Ṭīkā} comments on this: “What is the difference between the positively-produced and the produced? A dhamma that is an individual essence with a beginning and an end in time, produced by conditions, and marked by the three characteristics, is \emph{positively produced}. But besides this, what is \emph{produced} [but not positively produced] is a dhamma with no individual essence (\emph{asabhāva-dhamma}) when it is produced by the taking of a name or by attaining [the attainment of cessation]” (\textbf{\cite{Vibh-a}23}). Cf. also \hyperlink{XIV.72}{XIV.72}{} and 77.}
                    \begin{verse}
                        This too is an attainment which\\{}
                        A Noble One may cultivate;\\{}
                        The peace it gives is reckoned as\\{}
                        Nibbāna here and now.
                    \end{verse}

                    \begin{verse}
                        A wise man by developing\\{}
                        The noble understanding can\\{}
                        With it himself endow;\\{}
                        So this ability is called\\{}
                        A boon of understanding, too,\\{}
                        The noble paths allow.
                    \end{verse}

                \subsection[\vismAlignedParas{§53–60}D. Worthiness to Receive Gifts]{D. Worthiness to Receive Gifts}

                    \vismParagraph{XXIII.53}{53}{}
                    And not only the ability to attain the attainment of cessation but also achievement of worthiness to receive gifts should be understood as a benefit of this supramundane development of understanding.

                    \vismParagraph{XXIII.54}{54}{}
                    For, generally speaking, it is because understanding has been developed in these four ways that a person who has developed it, is fit for the gifts of the \marginnote{\textcolor{teal}{\footnotesize\{801|743\}}}{}world with its deities, fit for its hospitality, fit for its offerings, and fit for its reverential salutation, and an incomparable field of merit for the world.

                    \vismParagraph{XXIII.55}{55}{}
                    But in particular, firstly, one who arrives at development of under-standing of the first path with sluggish insight and limp faculties is called, “one who will be reborn seven times at most”; he traverses the round of rebirths seven times in the happy destinies. One who arrives with medium insight and medium faculties is called, “one who goes from noble family to noble family”; with two or three rebirths in noble families he makes an end of suffering. One who arrives with keen insight and keen faculties is called, “one who germinates only once”; with one rebirth in the human world he makes an end of suffering (see \textbf{\cite{A}I 133}).

                    By developing understanding of the second path, he is called a once-returner.

                    He returns once to this world and makes an end of suffering. \textcolor{brown}{\textit{[710]}}

                    \vismParagraph{XXIII.56}{56}{}
                    By developing understanding of the third path he is called a non-returner. According to the difference in his faculties he completes his course in one of five ways after he has left this world: he becomes “one who attains Nibbāna early in his next existence” or “one who attains Nibbāna more than half way through his next existence” or “one who attains Nibbāna without prompting” or “one who attains Nibbāna with prompting” or “one who is going upstream bound for the Highest Gods” (see \textbf{\cite{D}III 237}).

                    \vismParagraph{XXIII.57}{57}{}
                    Herein, \emph{one who attains Nibbāna early in his next existence} attains Nibbāna after reappearing anywhere in the Pure Abodes, without reaching the middle of his life span there. \emph{One who attains Nibbāna more than half way through his next existence} attains Nibbāna after the middle of his life span there. One who attains Nibbāna without prompting generates the highest path without prompting, with little effort. \emph{One who attains Nibbāna with prompting }generates the highest path with prompting, with effort. \emph{One who is going upstream bound for the Highest Gods} passes on upwards from wherever he is reborn [in the Pure Abodes] to the Highest Gods’ becoming and attains Nibbāna there.

                    \vismParagraph{XXIII.58}{58}{}
                    By developing understanding of the fourth path one becomes “liberated by faith,” another “liberated by understanding,” another “both-ways liberated,” another “one with the triple clear vision,” another “one with the six kinds of direct-knowledge,” another “one of the great ones whose cankers are destroyed who has reached the categories of discrimination.” It was about one who has developed the fourth path that it was said: “But it is at the moment of the path that he is said to be disentangling that tangle: at the moment of fruition he has disentangled the tangle and is worthy of the highest offerings in the world with its deities” (\hyperlink{I.7}{I.7}{}).

                    \vismParagraph{XXIII.59}{59}{}
                    
                    \begin{verse}
                        The noble understanding, when\\{}
                        Developed, will these blessings win;\\{}
                        Accordingly discerning men\\{}
                        Rejoice exceedingly therein.
                    \end{verse}


                    \vismParagraph{XXIII.60}{60}{}
                    And at this point the development of understanding with its benefits, which is shown in the \emph{Path of Purification} with its headings of virtue, concentration, and understanding, in the stanza,
                    \begin{verse}
                        \marginnote{\textcolor{teal}{\footnotesize\{802|744\}}}{}“When a wise man, established well in virtue,\\{}
                        Develops consciousness and understanding,\\{}
                        Then as a bhikkhu ardent and sagacious,\\{}
                        He succeeds in disentangling this tangle” (\hyperlink{I.1}{I.1}{}),\\{}
                        has been fully illustrated.
                    \end{verse}


                    The twenty-third chapter called “The Description of the Benefits of Understanding” in the \emph{Path of Purification }composed for the purpose of gladdening good people.
            \section[\vismAlignedParas{§60}Conclusion]{Conclusion}

                After we quoted this stanza,
                \begin{verse}
                    “When a wise man, established well in virtue,\\{}
                    Develops consciousness and understanding,\\{}
                    Then as a bhikkhu ardent and sagacious\\{}
                    He succeeds in disentangling this tangle” (\hyperlink{I.1}{I.1}{}),
                \end{verse}


                we then said:
                \begin{verse}
                    “My task is now to set out the true sense,\\{}
                    Divided into virtue and the rest,\\{}
                    Of this same verse composed by the Great Sage.\\{}
                    There are here in the Victor’s Dispensation\\{}
                    And who although desiring purity\\{}
                    Have no right knowledge of the sure straight way—\\{}
                    Comprising virtue and the other two,\\{}
                    Right hard to find, that leads to purity—\\{}
                    Who, though they strive, here gain no purity.\\{}
                    To them I shall expound the comforting \emph{Path}\\{}
                    \emph{Of Purification}, pure in expositions\\{}
                    Relying on the teaching of the dwellers\\{}
                    In the Great Monastery; let all those\\{}
                    Good men who do desire purity\\{}
                    Listen intently to my exposition.” (\hyperlink{I.4}{I.4}{})
                \end{verse}


                Now, at this point that has all been expounded. And herein:
                \begin{verse}
                    Now, that the exposition as set forth\\{}
                    Is almost free from errors and from flaws\\{}
                    After collating all the expositions\\{}
                    Of all these meanings classed as virtue and so on\\{}
                    Stated in the commentarial system\\{}
                    Of the five Nikāyas—for this reason\\{}
                    Let meditators pure in understanding\\{}
                    Desiring purification duly show\\{}
                    Reverence for this \emph{Path of Purification}.
                \end{verse}


                \textbf{[TODO: * * *]}
                \begin{verse}
                    \marginnote{\textcolor{teal}{\footnotesize\{804|746\}}}{}What store of merit has been gained by me\\{}
                    Desiring establishment in this Good Dhamma\\{}
                    In doing this, accepting the suggestion\\{}
                    Of the venerable Saṅghapāla,\\{}
                    One born into the line of famous elders\\{}
                    Dwelling within the Great Monastery,\\{}
                     A true Vibhajjavādin, who is wise,\\{}
                    And lives in pure simplicity, devoted\\{}
                    To discipline’s observance, and to practice,\\{}
                    Whose mind the virtuous qualities of patience,\\{}
                    Mildness, loving kindness, and so on, grace—\\{}
                    By the power of that store of merit\\{}
                    May every being prosper happily.\\{}
                    And now just as the \emph{Path of Purification},\\{}
                    With eight and fifty recitation sections\\{}
                    In the text, has herewith been completed\\{}
                    Without impediment, so may all those\\{}
                    Who in the world depend on what is good\\{}
                    Glad-hearted soon succeed without delay.
                \end{verse}

            \section[\vismAlignedParas{§60}Postscript]{Postscript}

                This \emph{Path of Purification} was made by the elder who is adorned with supreme and pure faith, wisdom and energy, in whom are gathered a concourse of upright, gentle, etc., qualities due to the practice of virtue, who is capable of delving into and fathoming the views of his own and others’ creeds, who is possessed of keenness of understanding, who is strong in unerring knowledge of the Master’s Dispensation as divided into three Piṭakas with their commentaries, a great expounder, gifted with sweet and noble speech that springs from the ease born of perfection of the vocal instrument, a speaker of what is appropriately said, a superlative speaker, a great poet, an ornament in the lineage of the elders who dwell in the Great Monastery, and who are shining lights in the lineage of elders with unblemished enlightenment in the superhuman states that are embellished with the special qualities of the six kinds of direct-knowledge and the categories of discrimination, who has abundant purified wit, who bears the name Buddhaghosa conferred by the venerable ones, and who should be called “of Moraṇḍaceṭaka.”
                \begin{verse}
                    May it continue here to show\\{}
                    The way to purity of virtue, etc.,\\{}
                    For clansmen seeking out the means\\{}
                    To ferry them across the worlds\\{}
                    For just as long as in this world\\{}
                    Shall last that name “Enlightened One,”\\{}
                    By which, thus purified in mind,\\{}
                    Is known the Greatest Sage, World Chief.
                \end{verse}


                \marginnote{\textcolor{teal}{\footnotesize\{805|747\}}}{}[\emph{The following verses are only in Sinhalese texts}:]
                \begin{verse}
                    By the performance of such merit\\{}
                    As has been gained by me through this\\{}
                     And any other still in hand\\{}
                    So may I in my next becoming\\{}
                    Behold the joys of Tāvatiṃsā,\\{}
                    Glad in the qualities of virtue\\{}
                    And unattached to sense desires.\\{}
                    By having reached the first fruition,\\{}
                    And having in my last life seen\\{}
                    Metteyya, Lord of Sages, Highest\\{}
                    Of persons in the World, and\\{}
                    Helper Delighting in all beings’ welfare,\\{}
                    And heard that Holy One proclaim\\{}
                    The Teaching of the Noble Dhamma,\\{}
                    May I grace the Victor’s Dispensation\\{}
                    By realizing its highest fruit.
                \end{verse}


                [The following verses are only in the Burmese texts:]
                \begin{verse}
                    The exposition of the \emph{Path of Purification}\\{}
                    Has thus been made for gladdening good people;\\{}
                    But this, by reckoning the Pali text,\\{}
                    Has eight and fifty recitation sections.
                \end{verse}


                \textsc{End}
\backmatter
\chapter[Index]{Index\* {\large of Subjects \& Proper Names}}All references are to chapter and paragraph numbers.\begin{multicols}{2}
\parskip=.2\baselineskip\RaggedRight\begin{vismHanging}
\par\textbf{Abandoning} (\emph{pahāna}) \hyperlink{I.12}{I.12}{}, \hyperlink{I.140}{140f.}{}; \hyperlink{XX.89}{XX.89}{}; \hyperlink{XXII.34}{XXII.34}{}, \hyperlink{XXII.47}{47f.}{}, \hyperlink{XXII.78}{78f.}{}, \hyperlink{XXII.92}{92}{}, \hyperlink{XXII.108}{108}{}, \hyperlink{XXII.113}{113}{}; \hyperlink{XXIII.4}{XXIII.4}{}
\par\textbf{Ābhassara} (Streaming-radiance) Deities \hyperlink{XIII.41}{XIII.41}{}
\par\textbf{Abhaya Thera, Tipiṭaka Cūḷa} \hyperlink{II.35}{II.35}{}; \hyperlink{III.53}{III.53}{}; \hyperlink{XII.89}{XII.89}{}, \hyperlink{XII.101}{101}{}; \hyperlink{XIII.n38}{XIII.n.38}{}
\par\textbf{Abhaya Thera, Dīghabhāṇaka} \hyperlink{I.99}{I.99}{}; \hyperlink{VIII.142}{VIII.142}{}
\par\textbf{Abhaya Thera, Pīṭha} \hyperlink{II.74}{II.74}{}
\par\textbf{Abhayagiri} (Monastery) \hyperlink{I.n18}{I.n.18}{}; \hyperlink{XII.n25}{XII.n.25}{}; \hyperlink{XIV.n31}{XIV.n.31}{}; \hyperlink{XXIII.n5}{XXIII.n.5}{}
\par\textbf{Abhidhamma} \hyperlink{XII.72}{XII.72}{}; \hyperlink{XIII.n20}{XIII.n.20}{}; \hyperlink{XIV.24}{XIV.24}{}, \hyperlink{XIV.58}{58}{}, \hyperlink{XIV.185}{185}{}; \hyperlink{XXI.72}{XXI.72}{}, \hyperlink{XXI.126}{126}{}
\par\textbf{abiding} (\emph{vihāra}) \hyperlink{VII.63}{VII.63}{}, see divine a.
\par\textbf{abode} (\emph{senāsana}), see resting-place; a. of beings (\emph{sattāvāsā}) \hyperlink{VII.38}{VII.38}{}; \hyperlink{XIII.69}{XIII.69}{}; \hyperlink{XVII.148}{XVII.148}{}; \hyperlink{XXI.35}{XXI.35}{}
\par\textbf{absence} (\emph{abhāva}) \hyperlink{VIII.147}{VIII.147}{}; \hyperlink{IX.123}{IX.123}{}; \hyperlink{XVI.68}{XVI.68}{}
\par\textbf{absorption} (\emph{appanā}) \hyperlink{III.5}{III.5}{}, \hyperlink{III.6}{6}{}, \hyperlink{III.106}{106}{}; \hyperlink{IV.33}{IV.33}{}, \hyperlink{IV.72}{72}{}, \hyperlink{IV.74}{74f.}{}, \hyperlink{IV.102}{102}{}; \hyperlink{XIII.5}{XIII.5}{}; \hyperlink{XVI.77}{XVI.77}{}; \hyperlink{XXIII.6}{XXIII.6}{}
\par\textbf{abstaining} (\emph{viramana}) \hyperlink{XXII.39}{XXII.39}{}
\par\textbf{abstention} (\emph{veramaṇī}) \hyperlink{I.28}{I.28}{}, \hyperlink{I.140}{140}{}; \hyperlink{XVII.40}{XVII.40}{}
\par\textbf{abstinence} (\emph{virati}) \hyperlink{I.18}{I.18}{}, \hyperlink{I.28}{28}{}, \hyperlink{I.84}{84}{}; \hyperlink{XIV.133}{XIV.133}{}, \hyperlink{XIV.155}{155f.}{}, \hyperlink{XIV.181}{181}{}, \hyperlink{XIV.184}{184}{}; \hyperlink{XVI.26}{XVI.26}{}, \hyperlink{XVI.78}{78f.}{}
\par\textbf{abuse} (\emph{akkosa}) \hyperlink{I.n24}{I.n.24}{}
\par\textbf{access} (\emph{upacāra}) \hyperlink{III.5}{III.5}{}, \hyperlink{III.6}{6}{}, \hyperlink{III.8}{8f.}{}, \hyperlink{III.15}{15}{}, \hyperlink{III.106}{106}{}; \hyperlink{IV.32}{IV.32f.}{}, \hyperlink{IV.74}{74}{}, \hyperlink{IV.185}{185f.}{}; \hyperlink{X.9}{X.9}{}; \hyperlink{XVIII.1}{XVIII.1}{}; \hyperlink{XXI.129}{XXI.129}{}
\par\textbf{accumulation} (\emph{āyūhana}) \hyperlink{I.140}{I.140}{}; \hyperlink{XIV.132}{XIV.132}{}, \hyperlink{XIV.135}{135}{}; \hyperlink{XVII.61}{XVII.61}{}, \hyperlink{XVII.292}{292f.}{}; \hyperlink{XIX.13}{XIX.13}{}; \hyperlink{XX.90}{XX.90}{}; \hyperlink{XXI.37}{XXI.37}{}, \hyperlink{XXI.38}{38}{}, \hyperlink{XXI.80}{80}{}; \hyperlink{XXII.5}{XXII.5}{}, \hyperlink{XXII.79}{79}{}, \hyperlink{XXII.97}{97}{}, \hyperlink{XXII.113}{113}{}, \hyperlink{XXII.115}{115}{}
\par\textbf{action} (\emph{kammanta}) \hyperlink{XXII.42}{XXII.42}{}, \hyperlink{XXII.45}{45}{}, \hyperlink{XXII.66}{66}{}
\par\textbf{adherence} (\emph{parāmāsa}) \hyperlink{I.35}{I.35}{}; \hyperlink{VII.104}{VII.104}{}; \hyperlink{XVII.293}{XVII.293}{}, \hyperlink{XVII.308}{308}{}, \hyperlink{XXII.48}{XXII.48}{}, \hyperlink{XXII.58}{58}{}, \hyperlink{XXII.71}{71}{}. See also misapprehension 
\par\textbf{adverting} (\emph{āvajjana}) \hyperlink{I.57}{I.57}{}; \hyperlink{IV.74}{IV.74}{}, \hyperlink{IV.78}{78}{}, \hyperlink{IV.132}{132}{}, \hyperlink{IV.138}{138}{}, \hyperlink{IV.n13}{n.13}{}; \hyperlink{XIII.5}{XIII.5}{}, \hyperlink{XIII.27}{27}{}, \hyperlink{XIII.101}{101}{}, \hyperlink{XIII.118}{118}{}; \hyperlink{XIV.107}{XIV.107}{}, \hyperlink{XIV.115}{115f.}{}, \hyperlink{XIV.121}{121f.}{}, \hyperlink{XIV.152}{152}{}; \hyperlink{XV.34}{XV.34f.}{}, \hyperlink{XV.n5}{n.5}{}; \hyperlink{XVII.137}{XVII.137}{}, \hyperlink{XVII.232}{232}{}; \hyperlink{XX.44}{XX.44}{}, \hyperlink{XX.121}{121}{}; \hyperlink{XXI.129}{XXI.129}{}; \hyperlink{XXII.1}{XXII.1f.}{}, \hyperlink{XXII.10}{10}{}, \hyperlink{XXII.19}{19}{}, \hyperlink{XXII.26}{26}{}; \hyperlink{XXIII.27}{XXIII.27}{}, \hyperlink{XXIII.38}{38}{}
\par\textbf{affliction} (\emph{ābādha}) \hyperlink{I.92}{I.92}{}, \hyperlink{I.97}{97}{}; \hyperlink{III.50}{III.50}{}
\par\textbf{aging} (\emph{jarā}) \hyperlink{IV.63}{IV.63}{}; \hyperlink{VII.16}{VII.16}{}; \hyperlink{XI.36}{XI.36}{}; \hyperlink{XIV.68}{XIV.68}{}; \hyperlink{XVI.31}{XVI.31}{}, \hyperlink{XVI.44}{44f.}{}, \hyperlink{XVI.59}{59}{}; \hyperlink{XVII.48}{XVII.48}{}; \hyperlink{XIX.11}{XIX.11}{}
\par\textbf{aging-and-death} (\emph{jarā-maraṇa}) \hyperlink{VII.7}{VII.7f.}{}; \hyperlink{VIII.9}{VIII.9}{}; \hyperlink{XVI.71}{XVI.71}{}; \hyperlink{XVII.2}{XVII.2}{}, \hyperlink{XVII.272}{272f.}{}, \hyperlink{XVII.287}{287}{}; \hyperlink{XIX.11}{XIX.11}{}; \hyperlink{XX.6}{XX.6f.}{}
\par\textbf{aggregate} (\emph{khandha}) \hyperlink{VII.16}{VII.16}{}, \hyperlink{VII.28}{28}{}, \hyperlink{VII.38}{38}{}; \hyperlink{VIII.39}{VIII.39}{}, \hyperlink{VIII.234}{234}{}, \hyperlink{VIII.236}{236}{}; \hyperlink{X.41}{X.41}{}, \hyperlink{X.53}{53}{}; \hyperlink{XII.52}{XII.52}{}; \hyperlink{XIII.13}{XIII.13}{}, \hyperlink{XIII.17}{17f.}{}, \hyperlink{XIII.120}{120}{}; \hyperlink{XIV}{XIV passim}{}, \hyperlink{XIV.19}{19}{}, \hyperlink{XIV.33}{33}{}, \hyperlink{XIV.213}{213f.}{}, \hyperlink{XIV.216}{216f.}{}; \hyperlink{XV.21}{XV.21}{}; \hyperlink{XVI.33}{XVI.33}{}, \hyperlink{XVI.44}{44}{}, \hyperlink{XVI.57}{57}{}, \hyperlink{XVI.68}{68f.}{}, \hyperlink{XVI.73}{73}{}; \hyperlink{XVII.77}{XVII.77}{}, \hyperlink{XVII.113}{113}{}, \hyperlink{XVII.159}{159}{}, \hyperlink{XVII.258}{258}{}, \hyperlink{XVII.263}{263}{}, \hyperlink{XVII.n4}{n.4}{}; \hyperlink{XVIII.13}{XVIII.13f.}{}; \hyperlink{XX.9}{XX.9}{}, \hyperlink{XX.20}{20}{}, \hyperlink{XX.28}{28}{}, \hyperlink{XX.97}{97f.}{}; \hyperlink{XXI.6}{XXI.6}{}, \hyperlink{XXI.18}{18}{}, \hyperlink{XXI.35}{35}{}, \hyperlink{XXI.87}{87}{}, \hyperlink{XXI.111}{111}{}, \hyperlink{XXI.n14}{n.14}{}; \hyperlink{XXII.45}{XXII.45}{}, \hyperlink{XXII.48}{48}{}; \hyperlink{XXIII.4}{XXIII.4}{}, \hyperlink{XXIII.n18}{n.18}{}; a. as object of clinging (\emph{upādāna-kkhandha}) \hyperlink{VII.38}{VII.38}{}; \hyperlink{XIV.214}{XIV.214}{}; \hyperlink{XVI.31}{XVI.31}{}, \hyperlink{XVI.57}{57f.}{}, \hyperlink{XVI.92}{92}{}
\par\textbf{agitation} (\emph{uddhacca}) \hyperlink{I.140}{I.140}{}; \hyperlink{III.95}{III.95}{}; \hyperlink{IV.47}{IV.47}{}, \hyperlink{IV.72}{72}{}; \hyperlink{VIII.74}{VIII.74}{}; \hyperlink{XII.17}{XII.17}{}; \hyperlink{XIV.93}{XIV.93}{}, \hyperlink{XIV.159}{159}{}, \hyperlink{XIV.165}{165}{}, \hyperlink{XIV.170}{170}{}, \hyperlink{XIV.176}{176}{}, \hyperlink{XIV.178}{178}{}; \hyperlink{XVII.61}{XVII.61}{}; \hyperlink{XX.106}{XX.106}{}; \hyperlink{XXII.28}{XXII.28}{}, \hyperlink{XXII.45}{45}{}, \hyperlink{XXII.48}{48}{}, \hyperlink{XXII.49}{49}{}; a.-and-worry (\emph{uddhacca-kukkucca}) \hyperlink{IV.86}{IV.86}{}, \hyperlink{IV.104}{104}{}
\par\textbf{air} (\emph{vāta}) \hyperlink{VIII.182}{VIII.182}{}; \hyperlink{XIII.42}{XIII.42}{}; a. (\emph{vāyo}) \hyperlink{XIII.30}{XIII.30}{}, \hyperlink{XIII.59}{59}{}; \hyperlink{XIV.35}{XIV.35f.}{}; \hyperlink{XV.39}{XV.39}{}; a. element (\emph{vāyo-dhātu}) \hyperlink{XI.28}{XI.28f.}{}, \hyperlink{XI.37}{37}{}, \hyperlink{XI.41}{41}{}, \hyperlink{XI.87}{87}{}; \hyperlink{XIV.35}{XIV.35}{}, \hyperlink{XIV.61}{61}{}, \hyperlink{XIV.n27}{n.27}{}, \hyperlink{XIV.n32}{n.32}{}; \hyperlink{XV.30}{XV.30}{}; a. kasiṇa (\emph{vāyo-kasiṇa}) \hyperlink{III.105}{III.105}{}; \hyperlink{XII.132}{XII.132}{}; \hyperlink{XIII.35}{XIII.35}{}
\par\textbf{Akaniṭṭha} (Highest) Gods \hyperlink{XII.78}{XII.78}{}; \hyperlink{XIV.193}{XIV.193}{}; \hyperlink{XXIII.56}{XXIII.56f.}{}; akaniṭṭhagāmin \hyperlink{XXIII.56}{XXIII.56}{}
\par\textbf{Ākāsacetiya} \hyperlink{IV.96}{IV.96}{}
\par\textbf{Ālāra Kālāma} \hyperlink{X.19}{X.19}{}
\par\textbf{all} (\emph{sabba}) \hyperlink{XXII.106}{XXII.106}{}, \hyperlink{XXII.126}{126}{}
\par\textbf{alms food} (\emph{piṇḍapāta}) \hyperlink{I.68}{I.68}{}, \hyperlink{I.89}{89}{}; \hyperlink{II.5}{II.5}{}; a.-eater (\emph{piṇḍapātika}) \hyperlink{II.2}{II.2f.}{}, \hyperlink{II.27}{27}{}
\par\textbf{aloofness} (\emph{atammayatā}) \hyperlink{XXI.135}{XXI.135}{}
\par\textbf{alterability, alteration} (\emph{vikāra}) \hyperlink{XIII.112}{XIII.112}{}; \hyperlink{XIV.65}{XIV.65}{}, \hyperlink{XIV.77}{77}{}; \hyperlink{XVII.14}{XVII.14}{}
\par\textbf{Ānanda Thera} \hyperlink{I.103}{I.103}{}; \hyperlink{XIV.27}{XIV.27}{}
\par\textbf{Anāthapiṇḍika} \hyperlink{XII.74}{XII.74}{}, \hyperlink{XII.106}{106}{}
\par\textbf{Anāthapiṇḍika, Cūḷa} \hyperlink{XII.74}{XII.74}{}
\par\textbf{Andhaka} \hyperlink{XXIII.n1}{XXIII.n.1}{}
\par\textbf{anger} (\emph{kodha}) \hyperlink{I.151}{I.151}{}; \hyperlink{III.95}{III.95}{}; \hyperlink{VII.59}{VII.59}{}, \hyperlink{VII.103}{103}{}, \hyperlink{VII.n25}{n.25}{}; \hyperlink{IX.15}{IX.15}{}
\par\textbf{Aṅgulimāla Thera} \hyperlink{XII.125}{XII.125}{}
\par\textbf{animal generation} (\emph{tiracchānayoni}) \hyperlink{XIII.93}{XIII.93}{}; \hyperlink{XIV.207}{XIV.207}{}; \hyperlink{XVII.154}{XVII.154}{}
\par\textbf{annihilation view} (\emph{uccheda-diṭṭhi}) \hyperlink{XIII.74}{XIII.74}{}; \hyperlink{XVI.85}{XVI.85}{}; \hyperlink{XVII.10}{XVII.10}{}, \hyperlink{XVII.23}{23}{}, \hyperlink{XVII.235}{235f.}{}, \hyperlink{XVII.286}{286}{}, \hyperlink{XVII.310}{310f.}{}; \hyperlink{XX.102}{XX.102}{}; \hyperlink{XXII.112}{XXII.112}{}
\par\textbf{annoyance} (\emph{āghāta}) \hyperlink{IV.87}{IV.87}{}; \hyperlink{IX.21}{IX.21}{}, \hyperlink{IX.93}{93}{}; \hyperlink{XIV.143}{XIV.143}{}, \hyperlink{XIV.171}{171}{}
\par\textbf{Anojā-devī} \hyperlink{XII.82}{XII.82}{}
\par\textbf{Anotatta, Lake} \hyperlink{XII.73}{XII.73}{}; \hyperlink{XIII.38}{XIII.38}{}
\par\textbf{antarāparinibbāyin} \hyperlink{XXIII.56}{XXIII.56}{}
\par\textbf{Anula Thera, Mahā} \hyperlink{XII.128}{XII.128}{}
\par\textbf{Anurādhapura} \hyperlink{I.55}{I.55}{}; \hyperlink{II.48}{II.48}{}; \hyperlink{III.31}{III.31}{}
\par\textbf{Anuruddha Thera} \hyperlink{II.18}{II.18}{}; \hyperlink{XII.74}{XII.74}{}
\par\textbf{any-bed-user} (\emph{yathāsanthatika}) \hyperlink{II.2}{II.2f.}{}, \hyperlink{II.69}{69}{}
\par\textbf{Aparagoyāna} \hyperlink{VII.436}{VII.436}{}
\par\textbf{aperture} (\emph{vivara}) \hyperlink{XIV.42}{XIV.42}{}, \hyperlink{XIV.63}{63}{}; \hyperlink{XV.39}{XV.39}{}
\par\textbf{apparitionally-born} (\emph{opapātika}) \hyperlink{XVII.154}{XVII.154}{}, \hyperlink{XVII.191}{191}{}, \hyperlink{XVII.286}{286}{}; \hyperlink{XX.26}{XX.26}{}; \hyperlink{XXII.27}{XXII.27}{}
\par\textbf{appearance as terror} (\emph{bhayatupaṭṭhāna}) \hyperlink{XXI.29}{XXI.29f.}{}, \hyperlink{XXI.99}{99}{}, \hyperlink{XXI.131}{131}{}; \hyperlink{XXII.120}{XXII.120}{}
\par\textbf{appellation} (\emph{paññatti}), see concept
\par\textbf{applied thought} (\emph{vitakka}) \hyperlink{I.140}{I.140}{}; \hyperlink{III.5}{III.5}{}, \hyperlink{III.11}{11}{}, \hyperlink{III.21}{21}{}, \hyperlink{III.25}{25f.}{}, \hyperlink{III.122}{122}{}; \hyperlink{IV.74}{IV.74}{}, \hyperlink{IV.86}{86}{}, \hyperlink{IV.88}{88f.}{}, \hyperlink{IV.132}{132}{}; \hyperlink{VI.86}{VI.86}{}; \hyperlink{VII.28}{VII.28}{}, \hyperlink{VII.59}{59}{}, \hyperlink{VII.n25}{n.25}{}; \hyperlink{VIII.233}{VIII.233}{}, \hyperlink{VIII.238}{238}{}; \hyperlink{IX.112}{IX.112f.}{}; \hyperlink{XIV.86}{XIV.86}{}, \hyperlink{XIV.133}{133}{}, \hyperlink{XIV.136}{136}{}, \hyperlink{XIV.157}{157f.}{}, \hyperlink{XIV.170}{170}{}, \hyperlink{XIV.176}{176}{}, \hyperlink{XIV.180}{180}{}; \hyperlink{XVI.86}{XVI.86}{}, \hyperlink{XVI.99}{99}{}; \hyperlink{XVII.160}{XVII.160}{}; \hyperlink{XVIII.3}{XVIII.3}{}; \hyperlink{XX.9}{XX.9}{}; \hyperlink{XXIII.24}{XXIII.24}{}, \hyperlink{XXIII.26}{26}{}
\par\textbf{approval} (\emph{anunaya}) \hyperlink{VI.67}{VI.67}{}; \hyperlink{IX.88}{IX.88}{}, \hyperlink{IX.96}{96}{}; \hyperlink{XXII.51}{XXII.51}{}
\par\textbf{arahant} (\emph{arahant}) \hyperlink{I.139}{I.139}{}; \hyperlink{VII.4}{VII.4f.}{}; \hyperlink{XIII.110}{XIII.110}{}; \hyperlink{XIV.108}{XIV.108f.}{}, \hyperlink{XIV.206}{206}{}; \hyperlink{XXII.45}{XXII.45}{}; \hyperlink{XXIII.7}{XXIII.7}{}, \hyperlink{XXIII.11}{11}{}, \hyperlink{XXIII.14}{14}{}, \hyperlink{XXIII.18}{18}{}, \hyperlink{XXIII.58}{58}{}
\par\textbf{arahantship} (\emph{arahatta}) \hyperlink{I.14}{I.14}{}, \hyperlink{I.37}{37}{}, \hyperlink{I.140}{140}{}; \hyperlink{VIII.224}{VIII.224}{}, \hyperlink{VIII.243}{243}{}; \hyperlink{IX.118}{IX.118}{}; \hyperlink{XIV.124}{XIV.124}{}; \hyperlink{XVI.69}{XVI.69}{}; \hyperlink{XVII.245}{XVII.245}{}; \hyperlink{XXII.1}{XXII.1f.}{}; \hyperlink{XXIII.25}{XXIII.25}{}, \hyperlink{XXIII.n1}{n.1}{}
\par\textbf{arisen} (\emph{uppanna}) \hyperlink{XX.47}{XX.47}{}; \hyperlink{XXII.81}{XXII.81f.}{}
\par\textbf{arising} (\emph{uppāda}) \hyperlink{I.140}{I.140}{}; \hyperlink{IV.n33}{IV.n.33}{}; \hyperlink{VIII.242}{VIII.242}{}, \hyperlink{VIII.n54}{n.54}{}; \hyperlink{XIII.111}{XIII.111}{}; \hyperlink{XIV.80}{XIV.80}{}, \hyperlink{XIV.190}{190}{}; \hyperlink{XX.22}{XX.22}{}, \hyperlink{XX.26}{26}{}; \hyperlink{XXI.10}{XXI.10}{}, \hyperlink{XXI.27}{27}{}, \hyperlink{XXI.37}{37}{}, \hyperlink{XXI.n6}{n.6}{}; \hyperlink{XXII.5}{XXII.5}{}, \hyperlink{XXII.44}{44}{}, \hyperlink{XXII.79}{79}{}; XXI-\hyperlink{II.7}{II.7}{}
\par\textbf{ascetic practice} (\emph{dhutaṅga}) \hyperlink{I.112}{I.112}{}; \hyperlink{II}{II passim}{}
\par\textbf{Asoka} \hyperlink{III.111}{III.111}{}; \hyperlink{VII.23}{VII.23}{}; \hyperlink{VIII.14}{VIII.14}{}
\par\textbf{Assagutta Thera} \hyperlink{III.63}{III.63}{}; \hyperlink{XIII.107}{XIII.107}{}
\par\textbf{Assakaṇṇapabbata} \hyperlink{VII.42}{VII.42}{}
\par\textbf{assembly} (\emph{parisā}) \hyperlink{IV.n28}{IV.n.28}{}
\par\textbf{asura (demon)} \hyperlink{VII.43}{VII.43f.}{}, \hyperlink{VII.n15}{n.15}{}; \hyperlink{XII.137}{XII.137}{}; \hyperlink{XIII.93}{XIII.93}{}
\par\textbf{atom} (\emph{aṇu}) \hyperlink{XI.n31}{XI.n.31}{}; \hyperlink{XVI.72}{XVI.72}{}, \hyperlink{XVI.91}{91}{}; \hyperlink{XVII.117}{XVII.117}{}
\par\textbf{attachment} (\emph{nikanti}) \hyperlink{X.6}{X.6}{}; \hyperlink{XI.3}{XI.3}{}; \hyperlink{XVII.292}{XVII.292}{}; \hyperlink{XIX.13}{XIX.13}{}; \hyperlink{XX.82}{XX.82}{}, \hyperlink{XX.122}{122}{}; \hyperlink{XXI.28}{XXI.28}{}
\par\textbf{attained-to-vision} (\emph{diṭṭhippatta}) \hyperlink{XXI.74}{XXI.74}{}, \hyperlink{XXI.89}{89}{}
\par\textbf{attainment} (\emph{samāpatti}) \hyperlink{XII.2}{XII.2}{}; \hyperlink{XIV.188}{XIV.188}{}, \hyperlink{XIV.197}{197}{}, \hyperlink{XIV.201}{201}{}; \hyperlink{XVII.264}{XVII.264}{}; \hyperlink{XVIII.1}{XVIII.1}{}; \hyperlink{XX.9}{XX.9}{}; \hyperlink{XXII.46}{XXII.46}{}; \hyperlink{XXIII.18}{XXIII.18}{}, \hyperlink{XXIII.20}{20}{}
\par\textbf{attention} (\emph{manasikāra}) \hyperlink{III.22}{III.22}{}, \hyperlink{III.26}{26}{}; \hyperlink{IV.52}{IV.52}{}, \hyperlink{IV.59}{59}{}; \hyperlink{VII.59}{VII.59}{}; \hyperlink{VIII.4}{VIII.4f.}{}, \hyperlink{VIII.48}{48}{}, \hyperlink{VIII.61}{61f.}{}; \hyperlink{XIV.133}{XIV.133}{}, \hyperlink{XIV.152}{152}{}, \hyperlink{XIV.159}{159}{}, \hyperlink{XIV.163}{163}{}, \hyperlink{XIV.170}{170}{}, \hyperlink{XIV.176}{176}{}, \hyperlink{XIV.178}{178f.}{}; \hyperlink{XV.39}{XV.39}{}; \hyperlink{XVIII.8}{XVIII.8}{}; \hyperlink{XIX.8}{XIX.8}{}; \hyperlink{XX.44}{XX.44}{}; \hyperlink{XXIII.12}{XXIII.12}{}. See also bringing-to-mind; a. directed to elements (\emph{dhātu-manasikāra}) \hyperlink{VIII.43}{VIII.43}{}; \hyperlink{XI.27}{XI.27}{}; a. (\emph{avadhāna}) \hyperlink{I.32}{I.32}{}
\par\textbf{avarice} (\emph{macchariya, macchera}) \hyperlink{I.151}{I.151}{}; \hyperlink{III.95}{III.95}{}; \hyperlink{VII.59}{VII.59}{}, \hyperlink{VII.107}{107f.}{}; \hyperlink{XIV.170}{XIV.170}{}, \hyperlink{XIV.173}{173}{}; \hyperlink{XXII.52}{XXII.52}{}, \hyperlink{XXII.67}{67}{}
\par\textbf{aversion} (\emph{arati}) \hyperlink{I.140}{I.140}{}; \hyperlink{IX.95}{IX.95}{}, \hyperlink{IX.100}{100}{}; \hyperlink{XXI.28}{XXI.28}{}. See also boredom
\par\textbf{Avīci} \hyperlink{VII.44}{VII.44}{}; \hyperlink{XII.71}{XII.71}{}, \hyperlink{XII.78}{78}{}, \hyperlink{XII.80}{80}{}; \hyperlink{XIII.93}{XIII.93}{}; \hyperlink{XV.27}{XV.27}{}
\par\textbf{bad way} (\emph{agati}) \hyperlink{VII.59}{VII.59}{}; \hyperlink{XXII.55}{XXII.55}{}, \hyperlink{XXII.69}{69}{}
\par\textbf{Bahula Thera} \hyperlink{III.84}{III.84}{}
\par\textbf{Bakkula Thera} \hyperlink{II.82}{II.82}{}; \hyperlink{XII.26}{XII.26}{}
\par\textbf{Baladeva} \hyperlink{VIII.19}{VIII.19}{}
\par\textbf{Bandhumant} \hyperlink{XIII.123}{XIII.123}{}
\par\textbf{Bandhumatī} \hyperlink{XIII.123}{XIII.123}{}
\par\textbf{bare-insight worker} (\emph{sukkha-vipassaka}) \hyperlink{XXI.112}{XXI.112}{}; \hyperlink{XXIII.18}{XXIII.18}{}
\par\textbf{base} (\emph{āyatana}) \hyperlink{I.2}{I.2}{}; \hyperlink{VII.12}{VII.12}{}, \hyperlink{VII.28}{28}{}, \hyperlink{VII.38}{38}{}; \hyperlink{X.24}{X.24}{}; \hyperlink{XV.1}{XV.1f.}{}, \hyperlink{XV.4}{4}{}; \hyperlink{XVII.1}{XVII.1}{}, \hyperlink{XVII.48}{48}{}, \hyperlink{XVII.51}{51}{}, \hyperlink{XVII.56}{56}{}, \hyperlink{XVII.204}{204f.}{}, \hyperlink{XVII.294}{294}{}; \hyperlink{XVIII.12}{XVIII.12}{}; \hyperlink{XIX.13}{XIX.13}{}; \hyperlink{XX.9}{XX.9}{}; \hyperlink{XXI.35}{XXI.35}{}
\par\textbf{base consisting of boundless consciousness} (\emph{viññāṇañcāyatana}) \hyperlink{I.140}{I.140}{}; \hyperlink{III.105}{III.105f.}{}; \hyperlink{IX.119}{IX.119}{}, \hyperlink{IX.122}{122}{}; \hyperlink{X.25}{X.25f.}{}; \hyperlink{XIV.87}{XIV.87}{}, \hyperlink{XIV.206}{206}{}; \hyperlink{XV.25}{XV.25f.}{}, \hyperlink{XVII.125}{XVII.125}{}, \hyperlink{XVII.135}{135}{}; \hyperlink{XXIII.21}{XXIII.21}{}, \hyperlink{XXIII.26}{26}{}, \hyperlink{XXIII.34}{34}{}; b.c.o. boundless space (\emph{ākāsānañcāyatana}) \hyperlink{I.140}{I.140}{}; \hyperlink{III.105}{III.105f.}{}; \hyperlink{VII.19}{VII.19}{}; \hyperlink{IX.119}{IX.119}{}, \hyperlink{IX.121}{121}{}; \hyperlink{X.1}{X.1f.}{}; \hyperlink{XIV.87}{XIV.87}{}, \hyperlink{XIV.206}{206}{}; \hyperlink{XV.25}{XV.25}{}, \hyperlink{XV.26}{26}{}; \hyperlink{XVII.125}{XVII.125}{}, \hyperlink{XVII.184}{184}{}; \hyperlink{XXIII.21}{XXIII.21}{}, \hyperlink{XXIII.26}{26}{}; b.c.o. neither perception nor non-perception (\emph{nevasaññānāsaññāyatana}) \hyperlink{I.140}{I.140}{}; \hyperlink{III.105}{III.105f.}{}; \hyperlink{IV.78}{IV.78}{}; \hyperlink{IX.104}{IX.104}{}; \hyperlink{XIV.87}{XIV.87}{}, \hyperlink{XIV.206}{206}{}; \hyperlink{XV.25}{XV.25}{}, \hyperlink{XV.26}{26}{}; \hyperlink{XVII.75}{XVII.75}{}, \hyperlink{XVII.125}{125}{}, \hyperlink{XVII.135}{135}{}; \hyperlink{XXIII.14}{XXIII.14}{}, \hyperlink{XXIII.21}{21}{}, \hyperlink{XXIII.26}{26}{}, \hyperlink{XXIII.28}{28}{}, \hyperlink{XXIII.31}{31}{}, \hyperlink{XXIII.43}{43}{}; b.c.o. nothingness (\emph{ākiñcaññāyatana}) \hyperlink{I.140}{I.140}{}; \hyperlink{III.105}{III.105f.}{}; \hyperlink{IX.119}{IX.119}{}, \hyperlink{IX.123}{123}{}; \hyperlink{XIV.87}{XIV.87}{}, \hyperlink{XIV.206}{206}{}; \hyperlink{XV.25}{XV.25f.}{}; \hyperlink{XVII.125}{XVII.125}{}; \hyperlink{XXIII.21}{XXIII.21}{}, \hyperlink{XXIII.26}{26}{}, \hyperlink{XXIII.34}{34}{}, \hyperlink{XXIII.43}{43}{}
\par\textbf{base of mastery} (\emph{abhibhāyatana}) \hyperlink{V.28}{V.28}{}, \hyperlink{V.32}{32}{}; \hyperlink{VIII.n37}{VIII.n.37}{}; \hyperlink{X.n3}{X.n.3}{}
\par\textbf{basic} (\emph{padhāna}) \hyperlink{XVII.107}{XVII.107}{}, \hyperlink{XVII.281}{281}{}
\par\textbf{Basic principle} (\emph{padhāna}) \hyperlink{XVI.85}{XVI.85}{}, \hyperlink{XVI.n23}{n.23}{}
\par\textbf{basis, physical} (\emph{vatthu}) \hyperlink{I.53}{I.53}{}; \hyperlink{X.16}{X.16}{}; \hyperlink{XI.107}{XI.107}{}; \hyperlink{XIV.47}{XIV.47}{}, \hyperlink{XIV.52}{52}{}, \hyperlink{XIV.78}{78}{}; \hyperlink{XVII.51}{XVII.51}{}, \hyperlink{XVII.127}{127f.}{}, \hyperlink{XVII.151}{151}{}, \hyperlink{XVII.189}{189}{}, \hyperlink{XVII.193}{193}{}, \hyperlink{XVII.204}{204}{}; \hyperlink{XVIII.7}{XVIII.7}{}; \hyperlink{XX.25}{XX.25}{}, \hyperlink{XX.31}{31}{}; \hyperlink{XXI.83}{XXI.83}{}; \hyperlink{XXII.29}{XXII.29}{}. See also heart-b.
\par\textbf{basis for success} , see road to power
\par\textbf{beauty, beautiful} (\emph{subha}) \hyperlink{I.n14}{I.n.14}{}; \hyperlink{V.32}{V.32}{}; \hyperlink{IX.120}{IX.120f.}{}; \hyperlink{XIV.226}{XIV.226}{}; \hyperlink{XV.26}{XV.26}{}, \hyperlink{XV.40}{40}{}; \hyperlink{XVI.16}{XVI.16}{}, \hyperlink{XVI.85}{85}{}, \hyperlink{XVI.90}{90}{}; \hyperlink{XVII.283}{XVII.283}{}; \hyperlink{XXI.56}{XXI.56}{}. See also perception of b.; b. element (\emph{subha-dhātu}) \hyperlink{XV.25}{XV.25f.}{}
\par\textbf{becoming} (\emph{bhava}) \hyperlink{I.32}{I.32}{}; \hyperlink{IV.78}{IV.78}{}; \hyperlink{VII.15}{VII.15}{}, \hyperlink{VII.28}{28}{}, \hyperlink{VII.n6}{n.6}{}; \hyperlink{VIII.241}{VIII.241}{}; \hyperlink{XI.2}{XI.2}{}; \hyperlink{XIII.28}{XIII.28}{}, \hyperlink{XIII.69}{69}{}, \hyperlink{XIII.114}{114}{}; \hyperlink{XIV.112}{XIV.112}{}, \hyperlink{XIV.124}{124}{}, \hyperlink{XIV.227}{227}{}; \hyperlink{XVI.34}{XVI.34}{}, \hyperlink{XVI.92}{92}{}; \hyperlink{XVII.2}{XVII.2}{}, \hyperlink{XVII.37}{37}{}, \hyperlink{XVII.40}{40}{}, \hyperlink{XVII.48}{48}{}, \hyperlink{XVII.51}{51}{}, \hyperlink{XVII.126}{126}{}, \hyperlink{XVII.162}{162}{}, \hyperlink{XVII.176}{176}{}, \hyperlink{XVII.235}{235}{}, \hyperlink{XVII.250}{250f.}{}, \hyperlink{XVII.273}{273f.}{}, \hyperlink{XVII.292}{292}{}, \hyperlink{XVII.n20}{n.20}{}; \hyperlink{XVIII.30}{XVIII.30}{}; \hyperlink{XIX.13}{XIX.13}{}; \hyperlink{XX.9}{XX.9}{}; \hyperlink{XXI.34}{XXI.34f.}{}; \hyperlink{XXII.56}{XXII.56}{}; \hyperlink{XXIII.29}{XXIII.29}{}
\par\textbf{beginning} (\emph{ādi}) \hyperlink{VIII.n54}{VIII.n.54}{}; \hyperlink{XVII.36}{XVII.36}{}, \hyperlink{XVII.280}{280f.}{}
\par\textbf{being} (\emph{bhava}) \hyperlink{XVII.n48}{XVII.n.48}{}, see becoming
\par\textbf{being, a living} (\emph{satta}) \hyperlink{III.113}{III.113}{}; \hyperlink{IV.62}{IV.62}{}; \hyperlink{VII.37}{VII.37f.}{}, \hyperlink{VII.n1}{n.1}{}, \hyperlink{VII.n13}{n.13}{}; \hyperlink{VIII.10}{VIII.10}{}, \hyperlink{VIII.39}{39}{}, \hyperlink{VIII.140}{140}{}; \hyperlink{IX.53}{IX.53}{}, \hyperlink{IX.102}{102}{}, \hyperlink{IX.120}{120}{}; \hyperlink{XI.30}{XI.30}{}, \hyperlink{XI.117}{117}{}; \hyperlink{XIII.44}{XIII.44f.}{}, \hyperlink{XIII.74}{74f.}{}, \hyperlink{XIII.n17}{n.17}{}; \hyperlink{XIV.111}{XIV.111}{}; \hyperlink{XVI.54}{XVI.54}{}; \hyperlink{XVII.113}{XVII.113f.}{}, \hyperlink{XVII.162}{162}{}; \hyperlink{XVIII.24}{XVIII.24}{}, \hyperlink{XVIII.28}{28}{}; \hyperlink{XIX.19}{XIX.19}{}; \hyperlink{XX.82}{XX.82}{}; \hyperlink{XXI.58}{XXI.58}{}; \hyperlink{XXII.48}{XXII.48}{}
\par\textbf{Bhaddiya Thera} \hyperlink{XII.110}{XII.110}{}
\par\textbf{bhikkhu} (\emph{bhikkhu}) \hyperlink{I.7}{I.7}{}, \hyperlink{I.40}{40}{}, \hyperlink{I.43}{43}{}; \hyperlink{II.90}{II.90f.}{}; \hyperlink{II.n12}{II.n.12}{}
\par\textbf{bhikkhunī} (\emph{bhikkhunī}) \hyperlink{I.40}{I.40}{}; \hyperlink{II.90}{II.90f.}{}
\par\textbf{Bhīmasena} \hyperlink{VIII.19}{VIII.19}{}
\par\textbf{Bhūtapāla Thera} \hyperlink{XII.26}{XII.26}{}, \hyperlink{XII.29}{29}{}
\par\textbf{bile} (\emph{pitta}) \hyperlink{VIII.127}{VIII.127}{}; \hyperlink{XI.17}{XI.17}{}; \hyperlink{XIII.2}{XIII.2}{}, \hyperlink{XIII.73}{73}{}
\par\textbf{Bimbisāra} \hyperlink{VII.23}{VII.23}{}
\par\textbf{birth} (\emph{jāti}) \hyperlink{IV.63}{IV.63}{}; \hyperlink{VII.16}{VII.16}{}; \hyperlink{VIII.10}{VIII.10}{}; \hyperlink{XIII.28}{XIII.28}{}; \hyperlink{XVI.32}{XVI.32f.}{}, \hyperlink{XVI.58}{58}{}; \hyperlink{XVII.2}{XVII.2}{}, \hyperlink{XVII.49}{49}{}, \hyperlink{XVII.51}{51}{}, \hyperlink{XVII.63}{63}{}, \hyperlink{XVII.270}{270f.}{}
\par\textbf{Blessed One} (\emph{bhagavant}) \hyperlink{IV.132}{IV.132}{}; \hyperlink{VII.55}{VII.55f.}{}; \hyperlink{XII.71}{XII.71f.}{}; \hyperlink{XX.n20}{XX.n.20}{}
\par\textbf{bliss} (\emph{sukha}) \hyperlink{I.32}{I.32}{}; \hyperlink{III.5}{III.5}{}, \hyperlink{III.9}{9}{}, \hyperlink{III.12}{12}{}, \hyperlink{III.21}{21}{}, \hyperlink{III.n6}{n.6}{}; \hyperlink{IV.74}{IV.74}{}, \hyperlink{IV.86}{86}{}, \hyperlink{IV.99}{99}{}, \hyperlink{IV.182}{182}{}; \hyperlink{VIII.230}{VIII.230}{}; \hyperlink{XII.131}{XII.131}{}; \hyperlink{XIV.86}{XIV.86}{}, \hyperlink{XIV.139}{139}{}; \hyperlink{XX.117}{XX.117}{}; \hyperlink{XXI.37}{XXI.37}{}; \hyperlink{XXIII.8}{XXIII.8}{}, \hyperlink{XXIII.26}{26}{}. See also perception of b., and pleasure 
\par\textbf{bloated} (\emph{uddhumātaka}) \hyperlink{III.105}{III.105}{}; \hyperlink{VI.1}{VI.1}{}; \hyperlink{VII.28}{VII.28}{}; \hyperlink{VIII.43}{VIII.43}{}
\par\textbf{blood} (\emph{lohita}) \hyperlink{VIII.111}{VIII.111f.}{}, \hyperlink{VIII.130}{130}{}; \hyperlink{XI.17}{XI.17}{}, \hyperlink{XI.72}{72}{}; \hyperlink{XIII.9}{XIII.9}{}, \hyperlink{XIII.n5}{n.5}{}; \hyperlink{XIV.60}{XIV.60}{}; (\emph{ruhira}) \hyperlink{XIII.2}{XIII.2}{}, \hyperlink{XIII.73}{73}{}
\par\textbf{Bodhisatta} (\emph{bodhisatta}) \hyperlink{I.41}{I.41}{}; \hyperlink{III.128}{III.128}{}; \hyperlink{XIII.54}{XIII.54}{}
\par\textbf{bodily formation} (\emph{kāya-saṅkhāra}) \hyperlink{VIII.175}{VIII.175f.}{}, \hyperlink{XVII.61}{XVII.61}{}; \hyperlink{XXIII.24}{XXIII.24}{}, \hyperlink{XXIII.51}{51}{}; b. intimation (\emph{kāyaviññatti}) \hyperlink{XIV.61}{XIV.61}{}; b. misconduct (\emph{k.-duccarita}) \hyperlink{XIV.155}{XIV.155}{}, \hyperlink{XIV.160}{160}{}; b.-pain faculty (\emph{dukkhindriya}) \hyperlink{XVI.10}{XVI.10}{}; b.-pleasure faculty (\emph{sukhindriya}) \hyperlink{XVI.10}{XVI.10}{}; b.-volition (\emph{k.-sañcetanā }\hyperlink{XVII.61}{XVII.61}{}
\par\textbf{body} (\emph{kāya}) \hyperlink{III.105}{III.105}{}; \hyperlink{VII.1}{VII.1}{}; \hyperlink{VIII.42}{VIII.42}{}; \hyperlink{XI.28}{XI.28}{}, \hyperlink{XI.92}{92}{}; \hyperlink{XII.130}{XII.130}{}, \hyperlink{XII.133}{133}{}; \hyperlink{XIV.41}{XIV.41}{}, \hyperlink{XIV.46}{46}{}, \hyperlink{XIV.52}{52}{}, \hyperlink{XIV.58}{58}{}, \hyperlink{XIV.117}{117}{}, \hyperlink{XIV.128}{128}{}, \hyperlink{XIV.144}{144}{}; \hyperlink{XV.3}{XV.3}{}; \hyperlink{XVI.10}{XVI.10}{}; \hyperlink{XVII.61}{XVII.61}{}; \hyperlink{XVIII.n9}{XVIII.n.9}{}; \hyperlink{XIX.4}{XIX.4}{}; \hyperlink{XXII.34}{XXII.34}{}, \hyperlink{XXII.54}{54}{}; \hyperlink{XXIII.n14}{XXIII.n.14}{}. See also contemplation of the b., \& mindful. occupied with the b.; b. base (\emph{kāyāyatana}) \hyperlink{XV.3}{XV.3f.}{}; b. consciousness (\emph{kāya-viññāṇa}) \hyperlink{XIV.96}{XIV.96}{}, \hyperlink{XIV.117}{117}{}, \hyperlink{XIV.179}{179}{}, \hyperlink{XIV.220}{220}{}; b.-c. element (\emph{kāya-viññāṇa-dhātu}) \hyperlink{XV.17}{XV.17f.}{}; b. decad (\emph{kāya-dasaka}) \hyperlink{XVII.151}{XVII.151}{}, \hyperlink{XVII.156}{156}{}, \hyperlink{XVII.189}{189}{}; b. door (\emph{kāya-dvāra}) \hyperlink{XVII.61}{XVII.61}{}; b. element (\emph{kāya-dhātu}) \hyperlink{XV.17}{XV.17f.}{}; b. faculty (\emph{kāyindriya}) \hyperlink{XIV.128}{XIV.128}{}; \hyperlink{XVI.1}{XVI.1}{}; b. witness (\emph{kāya-sakkhi}) \hyperlink{XXI.74}{XXI.74}{}, \hyperlink{XXI.77}{77}{}
\par\textbf{bond} (\emph{yoga}) \hyperlink{IV.87}{IV.87}{}; \hyperlink{VII.59}{VII.59}{}; \hyperlink{XIV.202}{XIV.202}{}, \hyperlink{XIV.226}{226f.}{}; \hyperlink{XXII.56}{XXII.56}{}, \hyperlink{XXII.70}{70}{}
\par\textbf{bondage} (\emph{saṃyoga}) \hyperlink{I.140}{I.140}{}; \hyperlink{XIV.162}{XIV.162}{}; \hyperlink{XX.90}{XX.90}{}; \hyperlink{XXII.97}{XXII.97}{}, \hyperlink{XXII.113}{113}{}, \hyperlink{XXII.121}{121}{}
\par\textbf{book} (\emph{gantha}) \hyperlink{III.51}{III.51}{}, \hyperlink{III.126}{126}{}
\par\textbf{boredom} (\emph{arati}), see aversion boundary (\emph{sīmā}) \hyperlink{III.n16}{III.n.16}{}; \hyperlink{IV.127}{IV.127}{}
\par\textbf{bowl-food eater} (\emph{patta-piṇḍika}) \hyperlink{II.2}{II.2}{}, \hyperlink{II.39}{39}{}
\par\textbf{Brahmā} \hyperlink{IX.106}{IX.106}{}; \hyperlink{XII.79}{XII.79}{}, \hyperlink{XII.137}{137}{}; \hyperlink{XVII.156}{XVII.156}{}, \hyperlink{XVII.282}{282}{}; \hyperlink{XVIII.24}{XVIII.24}{}
\par\textbf{Brahmā Sahampati} \hyperlink{VII.23}{VII.23}{}
\par\textbf{Brahmā’s Retinue} (\emph{brahmakāyika}) Deities \hyperlink{XVII.190}{XVII.190}{}
\par\textbf{Brahmavatī} \hyperlink{XIII.127}{XIII.127}{}
\par\textbf{Brahmā-world} (\emph{brahmaloka}) \hyperlink{III.118}{III.118}{}; \hyperlink{VII.18}{VII.18}{}; \hyperlink{XI.123}{XI.123}{}; \hyperlink{XII.71}{XII.71}{}, \hyperlink{XII.80}{80}{}, \hyperlink{XII.136}{136f.}{}; \hyperlink{XIII.7}{XIII.7}{}, \hyperlink{XIII.32}{32f.}{}; \hyperlink{XVII.134}{XVII.134}{}, \hyperlink{XVII.180}{180}{}, \hyperlink{XVII.264}{264}{}; \hyperlink{XX.108}{XX.108}{}
\par\textbf{Brahman} (\emph{brāhmaṇa}) \hyperlink{I.93}{I.93}{}
\par\textbf{brain} (\emph{matthaluṅga}) \hyperlink{VI.25}{VI.25}{}; \hyperlink{VIII.44}{VIII.44}{}, \hyperlink{VIII.126}{126}{}, \hyperlink{VIII.136}{136}{}; \hyperlink{XI.34}{XI.34}{}, \hyperlink{XI.68}{68}{}
\par\textbf{Brazen Palace} (\emph{lohapāsāda}) \hyperlink{III.55}{III.55}{}
\par\textbf{breath} (\emph{assāsa-passāsa}) \hyperlink{VIII.27}{VIII.27}{}, \hyperlink{VIII.164}{164}{}, \hyperlink{VIII.209}{209}{}; \hyperlink{XI.94}{XI.94}{}; \hyperlink{XVIII.6}{XVIII.6}{}, \hyperlink{XVIII.20}{20}{}
\par\textbf{breathing} (\emph{ānāpāna}) \hyperlink{III.105}{III.105}{}; \hyperlink{VI.65}{VI.65}{}; \hyperlink{VII.1}{VII.1}{}; \hyperlink{VIII.43}{VIII.43}{}, \hyperlink{VIII.145}{145f.}{}
\par\textbf{breathing thing} (\emph{pāṇa}) \hyperlink{I.140}{I.140}{}; \hyperlink{IX.54}{IX.54}{}. See also living thing 
\par\textbf{bright principle} (\emph{satta}) \hyperlink{IX.53}{IX.53}{}
\par\textbf{bringing-to-mind} (\emph{manasikāra}), see attention 
\par\textbf{Buddha} (\emph{Buddha}), see Enlightened One
\par\textbf{Buddharakkhita Thera} \hyperlink{IV.135}{IV.135}{}; \hyperlink{XII.9}{XII.9}{}
\par\textbf{Campā} \hyperlink{VII.51}{VII.51}{}
\par\textbf{Candapadumasiri} \hyperlink{XII.42}{XII.42}{}
\par\textbf{canker} (\emph{āsava}) \hyperlink{I.32}{I.32}{}, \hyperlink{I.127}{127}{}, \hyperlink{I.131}{131}{}; \hyperlink{IV.87}{IV.87}{}; \hyperlink{VII.7}{VII.7}{}, \hyperlink{VII.59}{59}{}; \hyperlink{XIV.8}{XIV.8}{}, \hyperlink{XIV.10}{10}{}, \hyperlink{XIV.72}{72}{}, \hyperlink{XIV.202}{202}{}, \hyperlink{XIV.214}{214}{}, \hyperlink{XIV.226}{226f.}{}; \hyperlink{XVI.104}{XVI.104}{}; \hyperlink{XVII.36}{XVII.36}{}, \hyperlink{XVII.53}{53}{}, \hyperlink{XVII.275}{275}{}; \hyperlink{XXII.30}{XXII.30}{}, \hyperlink{XXII.56}{56}{}, \hyperlink{XXII.70}{70}{}; \hyperlink{XXIII.18}{XXIII.18}{}
\par\textbf{Cāṇura} \hyperlink{VIII.19}{VIII.19}{}
\par\textbf{Catumahārājā} (Four Divine Kings) \hyperlink{VII.42}{VII.42}{}; \hyperlink{XIII.41}{XIII.41}{}
\par\textbf{cause} (\emph{hetu}) \hyperlink{XIV.22}{XIV.22f.}{}, \hyperlink{XIV.191}{191}{}, \hyperlink{XIV.n74}{n.74}{}; \hyperlink{XV.24}{XV.24}{}; \hyperlink{XVI.28}{XVI.28}{}, \hyperlink{XVI.85}{85}{}, \hyperlink{XVI.91}{91}{}, \hyperlink{XVI.104}{104}{}; \hyperlink{XVII.14}{XVII.14}{}, \hyperlink{XVII.67}{67}{}, \hyperlink{XVII.105}{105}{}, \hyperlink{XVII.286}{286f.}{}, \hyperlink{XVII.291}{291}{}, \hyperlink{XVII.310}{310}{}; \hyperlink{XIX.2}{XIX.2}{}; \hyperlink{XX.102}{XX.102}{}; \hyperlink{XXII.97}{XXII.97}{}. See also root-c. 
\par\textbf{causeless} (\emph{ahetuka}) \hyperlink{XIX.3}{XIX.3}{}. See also root-c.
\par\textbf{cessation} (\emph{nirodha}) \hyperlink{I.140}{I.140}{}; \hyperlink{IV.78}{IV.78}{}, \hyperlink{IV.186}{186}{}; \hyperlink{VII.27}{VII.27}{}; \hyperlink{XVI.15}{XVI.15}{}, \hyperlink{XVI.18}{18}{}, \hyperlink{XVI.23}{23f.}{}, \hyperlink{XVI.62}{62f.}{}, \hyperlink{XVI.94}{94}{}; \hyperlink{XVII.62}{XVII.62}{}; \hyperlink{XX.7}{XX.7}{}, \hyperlink{XX.100}{100}{}; \hyperlink{XXI.10}{XXI.10}{}, \hyperlink{XXI.77}{77}{}; \hyperlink{XXII.5}{XXII.5}{}, \hyperlink{XXII.46}{46}{}, \hyperlink{XXII.92}{92}{}; \hyperlink{XXIII.6}{XXIII.6}{}, \hyperlink{XXIII.10}{10}{}, \hyperlink{XXIII.28}{28}{}; c. attainment (\emph{nirodha-samāpatti}) \hyperlink{III.120}{III.120}{}; \hyperlink{IX.104}{IX.104}{}; \hyperlink{XI.124}{XI.124}{}; \hyperlink{XII.32}{XII.32}{}; \hyperlink{XVII.47}{XVII.47}{}; \hyperlink{XXIII.14}{XXIII.14}{}, \hyperlink{XXIII.17}{17f.}{} See also contemplation of c. 
\par\textbf{Cetiyagiri} \hyperlink{IV.10}{IV.10}{}
\par\textbf{Cetiyapabbata} \hyperlink{I.55}{I.55}{}; \hyperlink{II.13}{II.13}{}; \hyperlink{VI.88}{VI.88}{}
\par\textbf{Chaddanta, Lake} \hyperlink{XIII.38}{XIII.38}{}; \hyperlink{XXI.43}{XXI.43}{}
\par\textbf{Chaddanta Nāgarājā} \hyperlink{XXI.43}{XXI.43}{}
\par\textbf{change} (\emph{aññathatta}) \hyperlink{VIII.234}{VIII.234}{}; (\emph{pariṇāma, vipariṇāma}) \hyperlink{I.140}{I.140}{}; \hyperlink{XVII.63}{XVII.63}{}; \hyperlink{XX.97}{XX.97}{}; \hyperlink{XXI.116}{XXI.116}{}. See also contemplation of c.
\par\textbf{change-of-lineage} (\emph{gotrabhū}) \hyperlink{IV.74}{IV.74}{}, \hyperlink{IV.n18}{n.18}{}; \hyperlink{XIII.5}{XIII.5}{}; \hyperlink{XIV.28}{XIV.28}{}, \hyperlink{XIV.121}{121}{}; \hyperlink{XVII.81}{XVII.81}{}; \hyperlink{XXI.101}{XXI.101}{}, \hyperlink{XXI.126}{126}{}, \hyperlink{XXI.134}{134}{}; \hyperlink{XXII.5}{XXII.5f.}{}, \hyperlink{XXII.44}{44}{}, \hyperlink{XXII.56}{56}{}; \hyperlink{XXIII.7}{XXIII.7}{}, \hyperlink{XXIII.14}{14}{}
\par\textbf{characteristic} (\emph{lakkhaṇa}) \hyperlink{I.20}{I.20}{}; \hyperlink{VIII.180}{VIII.180}{}, \hyperlink{VIII.n62}{n.62}{}; \hyperlink{XIV.3}{XIV.3}{}, \hyperlink{XIV.8}{8}{}, \hyperlink{XIV.77}{77}{}; \hyperlink{XVII.51}{XVII.51}{}; \hyperlink{XVIII.19}{XVIII.19}{}; \hyperlink{XX.3}{XX.3f.}{}, \hyperlink{XX.45}{45f.}{}; \hyperlink{XXI.2}{XXI.2f.}{}, \hyperlink{XXI.52}{52}{}, \hyperlink{XXI.n4}{n.4}{}; \hyperlink{XXII.99}{XXII.99f.}{}, \hyperlink{XXII.n11}{n.11}{}
\par\textbf{charnel-ground contemplation} (\emph{sīvathikā}) \hyperlink{VIII.43}{VIII.43}{}; \hyperlink{XXIII.20}{XXIII.20}{}; c.-g. dweller (\emph{sosānika}) \hyperlink{II.2}{II.2}{}, \hyperlink{II.64}{64}{}
\par\textbf{child in the womb} (\emph{gabbhaseyyaka}) \hyperlink{XVII.286}{XVII.286}{}; \hyperlink{XX.22}{XX.22}{}
\par\textbf{Cīragumba} \hyperlink{I.122}{I.122}{}, \hyperlink{I.133}{133}{}
\par\textbf{Citrapāṭaliya Tree} \hyperlink{VII.43}{VII.43}{}
\par\textbf{Citta, householder} \hyperlink{XIV.27}{XIV.27}{}
\par\textbf{Citta, Peak} \hyperlink{XXI.43}{XXI.43}{}
\par\textbf{Cittagutta Thera} \hyperlink{I.104}{I.104f.}{}, \hyperlink{V.5}{V.5}{}, \hyperlink{V.15}{15}{}
\par\textbf{Cittalapabbata} \hyperlink{IV.10}{IV.10}{}, \hyperlink{IV.36}{36}{}; \hyperlink{V.15}{V.15}{}; \hyperlink{VIII.243}{VIII.243}{}; \hyperlink{IX.39}{IX.39}{}, \hyperlink{IX.68}{68}{}; \hyperlink{XX.109}{XX.109f.}{}
\par\textbf{clansman} (\emph{kulaputta}) \hyperlink{I.18}{I.18}{}; \hyperlink{XX.105}{XX.105}{}
\par\textbf{cleansing} (\emph{vodāna}) \hyperlink{I.16}{I.16}{}, \hyperlink{I.143}{143}{}; \hyperlink{III.26}{III.26}{}; \hyperlink{IV.117}{IV.117}{}; \hyperlink{XVII.80}{XVII.80}{}, \hyperlink{XVII.n15}{n.15}{}; \hyperlink{XXI.135}{XXI.135}{}; \hyperlink{XXII.45}{XXII.45}{}, \hyperlink{XXII.n7}{n.7}{}
\par\textbf{clear-vision} (\emph{vijjā}) \hyperlink{I.11}{I.11}{}, \hyperlink{I.n6}{n.6}{}; \hyperlink{VII.30}{VII.30}{}; \hyperlink{VII.239}{VII.239}{}
\par\textbf{clinging} (\emph{upādāna}) \hyperlink{I.32}{I.32}{}; \hyperlink{IV.87}{IV.87}{}; \hyperlink{VII.15}{VII.15f.}{}, \hyperlink{VII.59}{59}{}, \hyperlink{VII.n4}{n.4}{}; \hyperlink{XIV.202}{XIV.202}{}, \hyperlink{XIV.214}{214f.}{}, \hyperlink{XIV.225}{225f.}{}; \hyperlink{XVII.2}{XVII.2}{}, \hyperlink{XVII.48}{48}{}, \hyperlink{XVII.51}{51}{}, \hyperlink{XVII.239}{239f.}{}, \hyperlink{XVII.292}{292}{}; \hyperlink{XIX.4}{XIX.4}{}, \hyperlink{XIX.13}{13}{}; \hyperlink{XXII.59}{XXII.59}{}, \hyperlink{XXII.72}{72}{}
\par\textbf{clung-to} (\emph{upādiṇṇa, upādiṇṇaka}) \hyperlink{VIII.130}{VIII.130}{}; \hyperlink{XI.31}{XI.31f.}{}, \hyperlink{XI.97}{97}{}, \hyperlink{XI.102}{102}{}; \hyperlink{XII.104}{XII.104f.}{}, \hyperlink{XII.n20}{n.20}{}; \hyperlink{XIII.91}{XIII.91}{}; \hyperlink{XIV.52}{XIV.52}{}, \hyperlink{XIV.62}{62}{}, \hyperlink{XIV.72}{72f.}{}, \hyperlink{XIV.204}{204}{}, \hyperlink{XIV.n23}{n.23}{}; \hyperlink{XVII.255}{XVII.255}{}; \hyperlink{XX.40}{XX.40}{}. See also kammically-acquired
\par\textbf{code} (\emph{mātikā}) \hyperlink{I.27}{I.27}{}, \hyperlink{I.n11}{n.11}{}; \hyperlink{III.31}{III.31}{}, \hyperlink{III.n9}{n.9}{}; \hyperlink{IV.19}{IV.19}{}; \hyperlink{IX.67}{IX.67}{}
\par\textbf{cognitive series} (\emph{citta-vīthi}) \hyperlink{I.57}{I.57}{}, \hyperlink{I.n16}{n.16}{}; \hyperlink{IV.n13}{IV.n.13}{}; \hyperlink{XIII.117}{XIII.117}{}; \hyperlink{XIV.103}{XIV.103}{}, \hyperlink{XIV.152}{152}{}, \hyperlink{XIV.188}{188}{}, \hyperlink{XIV.197}{197}{}, \hyperlink{XIV.n47}{n.47}{}; \hyperlink{XV.10}{XV.10}{}; \hyperlink{XVII.136}{XVII.136f.}{}; \hyperlink{XIX.14}{XIX.14}{}; \hyperlink{XXII.6}{XXII.6}{}, \hyperlink{XXII.16}{16}{}; \hyperlink{XXIII.3}{XXIII.3}{}, \hyperlink{XXIII.14}{14}{}
\par\textbf{coincidence} (\emph{saṅgati}) \hyperlink{XVII.308}{XVII.308}{}
\par\textbf{colour} (\emph{vaṇṇa}) \hyperlink{IV.29}{IV.29}{}; \hyperlink{XI.88}{XI.88}{}; \hyperlink{XIV.47}{XIV.47}{}; \hyperlink{XVII.156}{XVII.156}{}; \hyperlink{XVIII.5}{XVIII.5f.}{}
\par\textbf{common to} (\emph{sādhāraṇa}) \hyperlink{XVII.107}{XVII.107}{}; \hyperlink{XIX.8}{XIX.8}{}
\par\textbf{community} (\emph{saṅgha}) \hyperlink{I.46}{I.46}{}; \hyperlink{III.57}{III.57}{}; \hyperlink{VII.1}{VII.1}{}, \hyperlink{VII.89}{89f.}{}, \hyperlink{XXIII.38}{XXIII.38}{}. See also Order 
\par\textbf{compact} (\emph{ghana}) \hyperlink{I.140}{I.140}{}; \hyperlink{XI.30}{XI.30}{}; \hyperlink{XIV.213}{XIV.213}{}; \hyperlink{XX.90}{XX.90}{}; \hyperlink{XXI.4}{XXI.4}{}, \hyperlink{XXI.50}{50}{}, \hyperlink{XXI.122}{122}{}, \hyperlink{XXI.n3}{n.3}{}; \hyperlink{XXII.114}{XXII.114}{}
\par\textbf{compassion} (\emph{karuṇā}) \hyperlink{II.105}{II.105f.}{}; \hyperlink{VII.18}{VII.18}{}, \hyperlink{VII.32}{32}{}, \hyperlink{VII.n7}{n.7}{}, \hyperlink{VII.n9}{n.9}{}; \hyperlink{IX.77}{IX.77f.}{}, \hyperlink{IX.92}{92}{}, \hyperlink{IX.94}{94}{}, \hyperlink{IX.99}{99}{}, \hyperlink{IX.119}{119}{}, \hyperlink{IX.121}{121}{}; \hyperlink{XIV.133}{XIV.133}{}, \hyperlink{XIV.154}{154}{}, \hyperlink{XIV.157}{157}{}, \hyperlink{XIV.181}{181}{}, \hyperlink{XIV.n67}{n.67}{}
\par\textbf{comprehending, comprehension} (\emph{sammasana}) \hyperlink{I.133}{I.133}{}, \hyperlink{I.n36}{n.36}{}; \hyperlink{VIII.224}{VIII.224}{}; \hyperlink{X.38}{X.38}{}, \hyperlink{X.53}{53}{}; \hyperlink{XVII.102}{XVII.102}{}; \hyperlink{XVIII.15}{XVIII.15}{}, \hyperlink{XVIII.n5}{n.5}{}; \hyperlink{XX.2}{XX.2f.}{}, \hyperlink{XX.75}{75f.}{}, \hyperlink{XX.93}{93}{}; \hyperlink{XXI.85}{XXI.85f.}{}; \hyperlink{XXII.107}{XXII.107}{}, \hyperlink{XXII.112}{112}{}
\par\textbf{conceit} (\emph{mada}), see vanity conceit (\emph{māna}), see pride conceit “I am” (\emph{asmi-māna}) \hyperlink{III.122}{III.122}{}, \hyperlink{III.n18}{n.18}{}; \hyperlink{VIII.n71}{VIII.n.71}{}
\par\textbf{concentration} (\emph{samādhi}) \hyperlink{I.7}{I.7}{}, \hyperlink{I.32}{32}{}, \hyperlink{I.n3}{n.3}{}, \hyperlink{I.n4}{n.4}{}; \hyperlink{III.2}{III.2}{}, \hyperlink{III.4}{4}{}, \hyperlink{III.21}{21}{}, \hyperlink{III.56}{56}{}, \hyperlink{III.n3}{n.3}{}; \hyperlink{IV.30}{IV.30}{}, \hyperlink{IV.45}{45f.}{}, \hyperlink{IV.51}{51}{}, \hyperlink{IV.86}{86}{}, \hyperlink{IV.99}{99}{}, \hyperlink{IV.117}{117}{}; \hyperlink{VII.n1}{VII.n.1}{}; \hyperlink{VIII.74}{VIII.74}{}, \hyperlink{VIII.231}{231f.}{}; \hyperlink{IX.104}{IX.104}{}, \hyperlink{IX.112}{112f.}{}, \hyperlink{IX.n17}{n.17}{}; \hyperlink{XI.44}{XI.44}{}, \hyperlink{XI.118}{118f.}{}; \hyperlink{XII.17}{XII.17}{}, \hyperlink{XII.30}{30}{}, \hyperlink{XII.50}{50f.}{}; \hyperlink{XIII.n1}{XIII.n.1}{}, \hyperlink{XIII.n3}{n.3}{}; \hyperlink{XIV.86}{XIV.86}{}, \hyperlink{XIV.133}{133}{}, \hyperlink{XIV.139}{139}{}, \hyperlink{XIV.159}{159}{}, \hyperlink{XIV.170}{170}{}, \hyperlink{XIV.177}{177f.}{}, \hyperlink{XIV.183}{183}{}; \hyperlink{XVI.1}{XVI.1}{}, \hyperlink{XVI.86}{86}{}, \hyperlink{XVI.95}{95}{}; \hyperlink{XVII.314}{XVII.314}{}; \hyperlink{XVIII.1}{XVIII.1}{}; \hyperlink{XXII.75}{XXII.75}{}, \hyperlink{XXII.89}{89}{}; \hyperlink{XXII.42}{XXII.42}{}, \hyperlink{XXII.45}{45}{}, \hyperlink{XXII.66}{66}{}, \hyperlink{XXII.128}{128}{}; \hyperlink{XXIII.7}{XXIII.7}{}, \hyperlink{XXIII.26}{26}{}, \hyperlink{XXIII.37}{37}{}
\par\textbf{concept} (\emph{paññatti}) \hyperlink{II.n18}{II.n.18}{}; \hyperlink{IV.29}{IV.29}{}; \hyperlink{VIII.39}{VIII.39}{}, \hyperlink{VIII.66}{66}{}, \hyperlink{VIII.n11}{n.11}{}; \hyperlink{IX.54}{IX.54}{}, \hyperlink{IX.102}{102}{}, \hyperlink{IX.n6}{n.6}{}; \hyperlink{XI.n17}{XI.n.17}{}, \hyperlink{XI.n18}{n.18}{}, \hyperlink{XI.n30}{n.30}{}; \hyperlink{XXI.n4}{XXI.n.4}{}
\par\textbf{concern} (\emph{ābhoga}) \hyperlink{IV.180}{IV.180}{}; \hyperlink{VIII.178}{VIII.178}{}; \hyperlink{IX.123}{IX.123}{}, \hyperlink{IX.n21}{n.21}{}; \hyperlink{X.45}{X.45}{}; \hyperlink{XI.48}{XI.48f.}{}. See also unconcern 
\par\textbf{concomitant} , see consciousness-c.
\par\textbf{concrete matter} (\emph{rūpa-rūpa}) \hyperlink{XIV.77}{XIV.77}{}; \hyperlink{XVII.189}{XVII.189}{}, \hyperlink{XVII.191}{191}{}; \hyperlink{XVIII.13}{XVIII.13}{}
\par\textbf{concurrence} (\emph{sannipāta}) \hyperlink{XIV.134}{XIV.134}{}; \hyperlink{XVII.308}{XVII.308}{}
\par\textbf{condition} (\emph{paccaya}) \hyperlink{VIII.180}{VIII.180}{}, \hyperlink{VIII.n54}{n.54}{}; \hyperlink{IX.109}{IX.109}{}, \hyperlink{IX.111}{111f.}{}; \hyperlink{XIV.23}{XIV.23}{}, \hyperlink{XIV.73}{73}{}, \hyperlink{XIV.122}{122}{}, \hyperlink{XIV.191}{191}{}, \hyperlink{XIV.n74}{n.74}{}; \hyperlink{XV.14}{XV.14}{}, \hyperlink{XV.32}{32}{}, \hyperlink{XV.35}{35f.}{}; \hyperlink{XVII.2}{XVII.2}{}, \hyperlink{XVII.66}{66f.}{}, \hyperlink{XVII.n2}{n.2}{}; \hyperlink{XIX}{XIX passim}{}, \hyperlink{XIX.2}{2}{}, \hyperlink{XIX.13}{13}{}; \hyperlink{XX.27}{XX.27f.}{}, \hyperlink{XX.97}{97}{}; \hyperlink{XXII.5}{XXII.5}{}
\par\textbf{conditionality, specific} (\emph{idappaccayatā}) \hyperlink{XVII.5}{XVII.5f.}{}, \hyperlink{XVII.58}{58}{}
\par\textbf{conformity} (\emph{anuloma}) \hyperlink{IV.74}{IV.74}{}, \hyperlink{IV.n13}{n.13}{}; \hyperlink{XIII.5}{XIII.5}{}; \hyperlink{XIV.28}{XIV.28}{}; \hyperlink{XX.18}{XX.18}{}; \hyperlink{XXI.1}{XXI.1}{}, \hyperlink{XXI.128}{128f.}{}; \hyperlink{XXII.6}{XXII.6}{}, \hyperlink{XXII.16}{16}{}, \hyperlink{XXII.23}{23}{}, \hyperlink{XXII.121}{121}{}, \hyperlink{XXII.n7}{n.7}{}; \hyperlink{XXIII.14}{XXIII.14}{}
\par\textbf{confusion} (\emph{sammoha}) \hyperlink{I.140}{I.140}{}; \hyperlink{XVIII.25}{XVIII.25}{}; \hyperlink{XX.90}{XX.90}{}; \hyperlink{XXII.113}{XXII.113}{}, \hyperlink{XXII.119}{119}{}
\par\textbf{conscience} (\emph{hiri}) \hyperlink{I.22}{I.22}{}, \hyperlink{I.48}{48}{}, \hyperlink{I.88}{88}{}; \hyperlink{VII.n8}{VII.n.8}{}; \hyperlink{XIV.133}{XIV.133}{}, \hyperlink{XIV.142}{142}{}, \hyperlink{XIV.155}{155}{}
\par\textbf{consciencelessness} (\emph{ahiri}) \hyperlink{VII.59}{VII.59}{}; \hyperlink{XIV.159}{XIV.159}{}, \hyperlink{XIV.170}{170}{}, \hyperlink{XIV.176}{176}{}; \hyperlink{XXII.49}{XXII.49}{}
\par\textbf{consciousness, (manner of) consciousness} (\emph{citta}) \hyperlink{I.7}{I.7}{}, \hyperlink{I.33}{33}{}; \hyperlink{II.12}{II.12}{}, \hyperlink{II.78}{78}{}; \hyperlink{III.24}{III.24}{}; \hyperlink{IV.115}{IV.115f.}{}, \hyperlink{IV.n13}{n.13}{}; \hyperlink{VIII.39}{VIII.39}{}, \hyperlink{VIII.145}{145}{}, \hyperlink{VIII.173}{173}{}, \hyperlink{VIII.231}{231}{}, \hyperlink{VIII.241}{241}{}; \hyperlink{X.10}{X.10}{}; \hyperlink{XI.94}{XI.94}{}, \hyperlink{XI.111}{111}{}; \hyperlink{XII.12}{XII.12}{}, \hyperlink{XII.50}{50}{}, \hyperlink{XII.130}{130}{}, \hyperlink{XII.133}{133}{}; \hyperlink{XIII.5}{XIII.5f.}{}, \hyperlink{XIII.118}{118}{}; \hyperlink{XIV.12}{XIV.12}{}, \hyperlink{XIV.47}{47}{}, \hyperlink{XIV.61}{61f.}{}, \hyperlink{XIV.75}{75}{}, \hyperlink{XIV.82}{82}{}, \hyperlink{XIV.116}{116}{}; \hyperlink{XV.4}{XV.4}{}, \hyperlink{XV.26}{26}{}; \hyperlink{XVII.72}{XVII.72}{}, \hyperlink{XVII.193}{193}{}; \hyperlink{XVIII.5}{XVIII.5}{}; \hyperlink{XIX.9}{XIX.9}{}; \hyperlink{XX.26}{XX.26}{}, \hyperlink{XX.30}{30f.}{}, \hyperlink{XX.71}{71}{}; \hyperlink{XXI.11}{XXI.11}{}, \hyperlink{XXI.129}{129}{}; \hyperlink{XXII.15}{XXII.15}{}, \hyperlink{XXII.42}{42}{}
\par\textbf{consciousness} (\emph{viññāṇa}) \hyperlink{I.53}{I.53}{}, \hyperlink{I.57}{57}{}, \hyperlink{I.n14}{n.14}{}; \hyperlink{V.n5}{V.n.5}{}; \hyperlink{VII.10}{VII.10}{}, \hyperlink{VII.28}{28}{}, \hyperlink{VII.n13}{n.13}{}; \hyperlink{IX.122}{IX.122}{}; \hyperlink{X.25}{X.25f.}{}, \hyperlink{X.50}{50}{}; \hyperlink{XI.1}{XI.1f.}{}, \hyperlink{XI.107}{107}{}; \hyperlink{XII.n21}{XII.n.21}{}; \hyperlink{XIV.3}{XIV.3}{}, \hyperlink{XIV.6}{6}{}, \hyperlink{XIV.81}{81f.}{}, \hyperlink{XIV.129}{129}{}, \hyperlink{XIV.214}{214}{}; \hyperlink{XV.9}{XV.9f.}{}, \hyperlink{XV.27}{27}{}; \hyperlink{XVII.2}{XVII.2}{}, \hyperlink{XVII.48}{48}{}, \hyperlink{XVII.51}{51}{}, \hyperlink{XVII.54}{54}{}, \hyperlink{XVII.120}{120f.}{}, \hyperlink{XVII.294}{294}{}; \hyperlink{XVIII.8}{XVIII.8}{}, \hyperlink{XVIII.11}{11}{}, \hyperlink{XVIII.13}{13}{}, \hyperlink{XVIII.21}{21}{}; \hyperlink{XIX.13}{XIX.13}{}, \hyperlink{XIX.23}{23}{}; \hyperlink{XX.6}{XX.6}{}, \hyperlink{XX.9}{9}{}, \hyperlink{XX.31}{31}{}, \hyperlink{XX.43}{43}{}, \hyperlink{XX.78}{78}{}, \hyperlink{XX.94}{94}{}; \hyperlink{XXI.11}{XXI.11}{}, \hyperlink{XXI.56}{56}{}; \hyperlink{XXII.36}{XXII.36}{}, \hyperlink{XXII.42}{42}{}, \hyperlink{XXII.53}{53}{}, \hyperlink{XXII.126}{126}{}; \hyperlink{XXIII.13}{XXIII.13}{}, \hyperlink{XXIII.18}{18}{}, \hyperlink{XXIII.22}{22}{}, \hyperlink{XXIII.30}{30}{}
\par\textbf{consciousness-concomitant} (\emph{cetasika}) \hyperlink{I.17}{I.17}{}, \hyperlink{I.n7}{n.7}{}; \hyperlink{II.12}{II.12}{}; \hyperlink{III.3}{III.3}{}, \hyperlink{III.n2}{n.2}{}; \hyperlink{IV.n13}{IV.n.13}{}; \hyperlink{X.22}{X.22}{}, \hyperlink{X.48}{48}{}; \hyperlink{XIV.6}{XIV.6}{}; \hyperlink{XVII.72}{XVII.72}{}; \hyperlink{XVIII.8}{XVIII.8}{}; \hyperlink{XXIII.18}{XXIII.18}{}
\par\textbf{consciousness-originated} (\emph{cittasamuṭṭhāna}) \hyperlink{VIII.n54}{VIII.n.54}{}; \hyperlink{XIV.61}{XIV.61f.}{}; \hyperlink{XIX.9}{XIX.9}{}; \hyperlink{XX.29}{XX.29}{}
\par\textbf{constituent of becoming} (\emph{vokāra}) \hyperlink{VII.n6}{VII.n.6}{}; \hyperlink{XVII.254}{XVII.254}{}
\par\textbf{contact} (\emph{phassa}) \hyperlink{IV.140}{IV.140}{}; \hyperlink{VII.13}{VII.13}{}, \hyperlink{VII.28}{28}{}; \hyperlink{X.50}{X.50}{}; \hyperlink{XI.1}{XI.1f.}{}, \hyperlink{XI.n2}{n.2}{}; \hyperlink{XIV.6}{XIV.6}{}, \hyperlink{XIV.133}{133f.}{}, \hyperlink{XIV.159}{159}{}, \hyperlink{XIV.170}{170}{}, \hyperlink{XIV.176}{176}{}, \hyperlink{XIV.179}{179}{}, \hyperlink{XIV.227}{227}{}, \hyperlink{XIV.n61}{n.61}{}, \hyperlink{XIV.n81}{n.81}{}; \hyperlink{XVII.2}{XVII.2}{}, \hyperlink{XVII.48}{48}{}, \hyperlink{XVII.51}{51}{}, \hyperlink{XVII.56}{56}{}, \hyperlink{XVII.220}{220f.}{}, \hyperlink{XVII.294}{294}{}; \hyperlink{XVIII.8}{XVIII.8}{}, \hyperlink{XVIII.11}{11}{}, \hyperlink{XVIII.18}{18f.}{}; \hyperlink{XIX.13}{XIX.13}{}; \hyperlink{XX.9}{XX.9}{}, \hyperlink{XX.77}{77}{}, \hyperlink{XX.97}{97}{}; c. pentad (\emph{phassa-pañcamaka}) \hyperlink{XX.77}{XX.77}{}
\par\textbf{contemplation} (\emph{anupassanā}) \hyperlink{VIII.234}{VIII.234}{}, \hyperlink{VIII.236}{236}{}; the seven c. (\emph{sattānupassanā}) \hyperlink{XX.4}{XX.4}{}; \hyperlink{XXI.15}{XXI.15f.}{}, \hyperlink{XXI.43}{43}{}; \hyperlink{XXII.114}{XXII.114}{}; c. of body (\emph{kāyānup.}) \hyperlink{VIII.168}{VIII.168}{}; \hyperlink{IX.113}{IX.113}{}; c. of cessation (\emph{nirodhānup.}) \hyperlink{I.140}{I.140}{}; \hyperlink{VIII.233}{VIII.233}{}; \hyperlink{XX.90}{XX.90}{}; \hyperlink{XXII.113}{XXII.113}{}; \hyperlink{XXIII.22}{XXIII.22f.}{}; c. of change (\emph{vipariṇāmānup}.) \hyperlink{I.140}{I.140}{}; \hyperlink{XX.90}{XX.90}{}; \hyperlink{XXII.113}{XXII.113}{}, \hyperlink{XXII.116}{116}{}; c. of danger (\emph{ādināvanup}.) \hyperlink{I.140}{I.140}{}; \hyperlink{VIII.43}{VIII.43}{}; \hyperlink{XX.90}{XX.90}{}; \hyperlink{XXI.35}{XXI.35f.}{}; \hyperlink{XXII.113}{XXII.113}{}, \hyperlink{XXII.120}{120}{}; c. of the desireless (\emph{appaṇihitānup}.) \hyperlink{I.140}{I.140}{}; \hyperlink{XX.90}{XX.90}{}; \hyperlink{XXII.113}{XXII.113}{}, \hyperlink{XXII.117}{117}{}; c. of destruction (\emph{khayānup}.) \hyperlink{I.140}{I.140}{}; \hyperlink{XX.90}{XX.90}{}; \hyperlink{XXII.113}{XXII.113}{}, \hyperlink{XXII.114}{114}{}; c. of dispassion (\emph{nibbidānup.}) \hyperlink{I.140}{I.140}{}; \hyperlink{VIII.233}{VIII.233}{}; \hyperlink{XX.90}{XX.90}{}; \hyperlink{XXI.43}{XXI.43f.}{}; \hyperlink{XXII.113}{XXII.113}{}; \hyperlink{XXIII.22}{XXIII.22f.}{}; c. of dissolution (\emph{bhaṅgānup.}) \hyperlink{VIII.224}{VIII.224}{}; \hyperlink{XX.4}{XX.4}{}; \hyperlink{XXI.10}{XXI.10f.}{}, \hyperlink{XXI.25}{25}{}, \hyperlink{XXI.131}{131}{}; \hyperlink{XXII.108}{XXII.108}{}, \hyperlink{XXII.114}{114}{}; c. of fading away (\emph{virāgānup}.) \hyperlink{I.140}{I.140}{}; \hyperlink{VIII.233}{VIII.233}{}; \hyperlink{XX.90}{XX.90}{}; \hyperlink{XXII.113}{XXII.113}{}; \hyperlink{XXIII.22}{XXIII.22f.}{}; c. of fall (\emph{vayānup.}) \hyperlink{I.140}{I.140}{}; \hyperlink{XX.90}{XX.90}{}; \hyperlink{XXII.113}{XXII.113}{}, \hyperlink{XXII.115}{115}{}; c. of impermanence (\emph{aniccānup}.) \hyperlink{I.140}{I.140}{}; \hyperlink{VIII.233}{VIII.233}{}; \hyperlink{XX.4}{XX.4}{}, \hyperlink{XX.20}{20}{}, \hyperlink{XX.90}{90}{}; \hyperlink{XXI.26}{XXI.26}{}, \hyperlink{XXI.122}{122}{}; \hyperlink{XXII.46}{XXII.46}{}, \hyperlink{XXII.108}{108}{}, \hyperlink{XXII.113}{113}{}; \hyperlink{XXIII.22}{XXIII.22f.}{} c. of notself (\emph{anattānup}.) \hyperlink{I.140}{I.140}{}; \hyperlink{VIII.233}{VIII.233}{}; \hyperlink{XX.4}{XX.4}{}, \hyperlink{XX.20}{20}{}, \hyperlink{XX.90}{90}{}; \hyperlink{XXI.122}{XXI.122}{}; \hyperlink{XXII.113}{XXII.113}{}; \hyperlink{XXIII.22}{XXIII.22f.}{}; c. of pain (\emph{dukkhānup}.) \hyperlink{I.140}{I.140}{}; \hyperlink{VIII.233}{VIII.233}{}; \hyperlink{XX.4}{XX.4}{}, \hyperlink{XX.20}{20}{}, \hyperlink{XX.90}{90}{}; \hyperlink{XXI.122}{XXI.122}{}; \hyperlink{XXII.113}{XXII.113}{}; \hyperlink{XXIII.22}{XXIII.22f.}{}; c. of reflection (\emph{paṭisaṅkhānup.}) \hyperlink{I.140}{I.140}{}; \hyperlink{XX.90}{XX.90}{}; \hyperlink{XXI.47}{XXI.47}{}, \hyperlink{XXI.62}{62}{}, \hyperlink{XXI.82}{82}{}; \hyperlink{XXII.113}{XXII.113}{}, \hyperlink{XXII.120}{120}{}; c. of relinquishment (\emph{paṭinissaggānup.}) \hyperlink{VIII.233}{VIII.233}{}, \hyperlink{VIII.236}{236}{}; \hyperlink{XX.90}{XX.90}{}; \hyperlink{XXII.113}{XXII.113}{}; \hyperlink{XXIII.22}{XXIII.22f.}{}; c. of rise and fall (\emph{udayabbayānup.}) \hyperlink{VIII.224}{VIII.224}{}; \hyperlink{XX.4}{XX.4}{}, \hyperlink{XX.93}{93f.}{}; \hyperlink{XXI.1}{XXI.1}{}; c. of the signless (\emph{animittānup.}) \hyperlink{I.140}{I.140}{}; \hyperlink{XX.90}{XX.90}{}; \hyperlink{XXII.113}{XXII.113}{}, \hyperlink{XXII.117}{117}{}; c. of turning away (\emph{vivaṭṭānup.}) \hyperlink{I.140}{I.140}{}; \hyperlink{XX.90}{XX.90}{}; \hyperlink{XXII.113}{XXII.113}{}, \hyperlink{XXII.121}{121}{}; \hyperlink{XXIII.25}{XXIII.25}{}; c. of voidness (\emph{suññatānupassanā}) \hyperlink{I.140}{I.140}{}; \hyperlink{XX.90}{XX.90}{}; \hyperlink{XXII.113}{XXII.113}{}, \hyperlink{XXII.117}{117}{}
\par\textbf{contentment} (\emph{santuṭṭhitā}) \hyperlink{I.151}{I.151}{}; \hyperlink{II.1}{II.1}{}, \hyperlink{II.83}{83}{}; \hyperlink{XVI.86}{XVI.86}{}
\par\textbf{contiguous objective field} (\emph{sampattavisaya}) \hyperlink{XIV.46}{XIV.46}{}, \hyperlink{XIV.76}{76}{}; \hyperlink{XVII.56}{XVII.56}{}
\par\textbf{continuity} (\emph{santāna}) \hyperlink{VIII.n11}{VIII.n.11}{}; \hyperlink{XIII.13}{XIII.13}{}, \hyperlink{XIII.111}{111f.}{}; \hyperlink{XIV.114}{XIV.114f.}{}, \hyperlink{XIV.123}{123}{}; (\emph{santati}) \hyperlink{XI.112}{XI.112}{}, \hyperlink{XI.n21}{n.21}{}; \hyperlink{XIII.111}{XIII.111}{}, \hyperlink{XIII.113}{113}{}; \hyperlink{XIV.66}{XIV.66}{}, \hyperlink{XIV.114}{114}{}, \hyperlink{XIV.124}{124}{}, \hyperlink{XIV.186}{186}{}, \hyperlink{XIV.188}{188f.}{}, \hyperlink{XIV.197}{197}{}; \hyperlink{XVII.74}{XVII.74}{}, \hyperlink{XVII.165}{165}{}, \hyperlink{XVII.170}{170}{}, \hyperlink{XVII.189}{189}{}, \hyperlink{XVII.204}{204}{}, \hyperlink{XVII.223}{223}{}, \hyperlink{XVII.271}{271}{}, \hyperlink{XVII.310}{310}{}; \hyperlink{XIX.23}{XIX.23}{}; \hyperlink{XX.22}{XX.22}{}, \hyperlink{XX.26}{26}{}, \hyperlink{XX.102}{102}{}, \hyperlink{XX.n23}{n.23}{}; \hyperlink{XXI.3}{XXI.3}{}, \hyperlink{XXI.n3}{n.3}{}, \hyperlink{XXI.n41}{n.41}{}; \hyperlink{XXII.89}{XXII.89}{}, \hyperlink{XXII.128}{128}{}
\par\textbf{contraction} (\emph{saṃvaṭṭa}) \hyperlink{XIII.28}{XIII.28f.}{}
\par\textbf{conventional, convention} (\emph{sammuti}) \hyperlink{VII.n7}{VII.n.7}{}, \hyperlink{VII.n19}{n.19}{}; \hyperlink{VIII.1}{VIII.1}{}; \hyperlink{XVI.n18}{XVI.n.18}{}; \hyperlink{XVII.171}{XVII.171}{}
\par\textbf{conveying} (\emph{abhinīhāra}) \hyperlink{XI.93}{XI.93}{}, \hyperlink{XI.117}{117}{}; \hyperlink{XIII.16}{XIII.16}{}, \hyperlink{XIII.95}{95}{}
\par\textbf{coolness} (\emph{sītibhāva}) \hyperlink{VIII.77}{VIII.77}{}
\par\textbf{co-presence} (\emph{saṇṭhāna}) \hyperlink{XVII.76}{XVII.76}{}; (\emph{sahaṭṭhāna}) \hyperlink{XIII.116}{XIII.116}{}
\par\textbf{cords of sense desire, the five} (\emph{pañcakāmaguṇa}) \hyperlink{I.155}{I.155f.}{}; \hyperlink{IV.87}{IV.87}{}, \hyperlink{IV.n24}{n.24}{}. See also sense desire 
\par\textbf{core} (\emph{sāra}) \hyperlink{I.140}{I.140}{}; \hyperlink{XIV.91}{XIV.91}{}; \hyperlink{XX.16}{XX.16}{}, \hyperlink{XX.90}{90}{}; \hyperlink{XXI.56}{XXI.56}{}, \hyperlink{XXI.59}{59}{}; \hyperlink{XXII.118}{XXII.118}{}
\par\textbf{correct knowledge} (\emph{yathābhūtañāṇa}) \hyperlink{XIX.25}{XIX.25}{}
\par\textbf{correct knowledge and vision} (\emph{yathābhūtañāṇadassana}) \hyperlink{I.32}{I.32}{}, \hyperlink{I.140}{140}{}; \hyperlink{XX.90}{XX.90}{}; \hyperlink{XXII.113}{XXII.113}{}, \hyperlink{XXII.119}{119}{}
\par\textbf{coupling, coupled} (\emph{yuganaddha}) \hyperlink{IV.117}{IV.117}{}; \hyperlink{XXII.46}{XXII.46}{}; \hyperlink{XXIII.43}{XXIII.43}{}
\par\textbf{course of action} (\emph{kamma-patha}) \hyperlink{I.17}{I.17}{}; \hyperlink{VII.59}{VII.59}{}, \hyperlink{VII.n25}{n.25}{}; \hyperlink{XXII.63}{XXII.63}{}, \hyperlink{XXII.75}{75}{}
\par\textbf{course of an existence} (\emph{pavatta, pavatti}) \hyperlink{XVI.23}{XVI.23}{}, \hyperlink{XVI.42}{42}{}; \hyperlink{XVII.89}{XVII.89}{}, \hyperlink{XVII.126}{126f.}{}, \hyperlink{XVII.193}{193}{}; \hyperlink{XIX.16}{XIX.16}{}; \hyperlink{XX.44}{XX.44}{}. See also occurrence 
\par\textbf{covetousness} (\emph{abhijjhā}) \hyperlink{I.42}{I.42}{}, \hyperlink{I.140}{140}{}; \hyperlink{XII.19}{XII.19}{}; \hyperlink{XVII.251}{XVII.251}{}
\par\textbf{craving} (\emph{taṇhā}) \hyperlink{I.2}{I.2}{}, \hyperlink{I.13}{13}{}; \hyperlink{III.17}{III.17}{}, \hyperlink{III.78}{78}{}; \hyperlink{IV.87}{IV.87}{}; \hyperlink{VII.7}{VII.7f.}{}, \hyperlink{VII.15}{15}{}, \hyperlink{VII.27}{27f.}{}, \hyperlink{VII.59}{59}{}, \hyperlink{VII.n25}{n.25}{}; \hyperlink{VIII.247}{VIII.247}{}; \hyperlink{XI.26}{XI.26}{}; \hyperlink{XIV.162}{XIV.162}{}, \hyperlink{XIV.n14}{n.14}{}; \hyperlink{XVI.23}{XVI.23}{}, \hyperlink{XVI.28}{28}{}, \hyperlink{XVI.31}{31}{}, \hyperlink{XVI.61}{61}{}, \hyperlink{XVI.86}{86}{}, \hyperlink{XVI.93}{93}{}; \hyperlink{XVII.37}{XVII.37}{}, \hyperlink{XVII.48}{48}{}, \hyperlink{XVII.51}{51}{}, \hyperlink{XVII.163}{163}{}, \hyperlink{XVII.233}{233f.}{}, \hyperlink{XVII.286}{286}{}, \hyperlink{XVII.292}{292}{}; \hyperlink{XIX.4}{XIX.4}{}, \hyperlink{XIX.13}{13}{}; \hyperlink{XX.9}{XX.9}{}, \hyperlink{XX.82}{82}{}, \hyperlink{XX.97}{97}{}, \hyperlink{XX.125}{125}{}; \hyperlink{XXI.19}{XXI.19}{}
\par\textbf{cruelty} (\emph{vihiṃsā}) \hyperlink{IX.94}{IX.94}{}, \hyperlink{IX.99}{99}{}; \hyperlink{XV.28}{XV.28}{}
\par\textbf{Cūḷa-Abhaya} , etc., see under individual names, Abhaya, etc. 
\par\textbf{Cūḷa-Nāga-Lena} (Cave) \hyperlink{IV.36}{IV.36}{}
\par\textbf{curiosity} (\emph{īhaka}) \hyperlink{XVIII.31}{XVIII.31}{}
\par\textbf{cutting off} (\emph{samuccheda}) \hyperlink{I.12}{I.12}{}; \hyperlink{XXII.122}{XXII.122}{}
\par\textbf{Dakkhiṇagiri} \hyperlink{IV.10}{IV.10}{}
\par\textbf{Dānava} (Demon) \hyperlink{XI.97}{XI.97}{}
\par\textbf{danger} (\emph{ādīnava}), see contemplation of d.
\par\textbf{Datta Thera, Mahā} \hyperlink{XX.110}{XX.110}{}; \hyperlink{XXI.n38}{XXI.n.38}{}
\par\textbf{Datta-Abhaya Thera} \hyperlink{III.84}{III.84}{}
\par\textbf{death} (\emph{cuti}) \hyperlink{IV.n13}{IV.n.13}{}; \hyperlink{VIII.241}{VIII.241}{}; \hyperlink{XIII.14}{XIII.14}{}, \hyperlink{XIII.17}{17f.}{}, \hyperlink{XIII.24}{24}{}, \hyperlink{XIII.76}{76}{}; \hyperlink{XIV.98}{XIV.98}{}, \hyperlink{XIV.110}{110}{}, \hyperlink{XIV.123}{123}{}; \hyperlink{XVII.113}{XVII.113f.}{}, \hyperlink{XVII.131}{131f.}{}, \hyperlink{XVII.135}{135f.}{}, \hyperlink{XVII.164}{164}{}, \hyperlink{XVII.232}{232}{}, \hyperlink{XVII.n43}{n.43}{}, \hyperlink{XVII.n45}{n.45}{}; \hyperlink{XIX.24}{XIX.24}{}; \hyperlink{XX.31}{XX.31}{}, \hyperlink{XX.43}{43}{}, \hyperlink{XX.47}{47}{}; (\emph{maraṇa}) \hyperlink{III.105}{III.105}{}; \hyperlink{IV.63}{IV.63}{}; \hyperlink{VII.1}{VII.1}{}, \hyperlink{VII.16}{16}{}, \hyperlink{VII.59}{59}{}; \hyperlink{VIII.1}{VIII.1f.}{}; \hyperlink{IX.7}{IX.7}{}, \hyperlink{IX.75}{75}{}; \hyperlink{XIII.91}{XIII.91}{}; \hyperlink{XIV.111}{XIV.111}{}; \hyperlink{XVI.31}{XVI.31}{}, \hyperlink{XVI.46}{46f.}{}, \hyperlink{XVI.59}{59}{}; \hyperlink{XVII.48}{XVII.48}{}, \hyperlink{XVII.163}{163}{}, \hyperlink{XVII.278}{278}{}; \hyperlink{XVIII.30}{XVIII.30}{}; \hyperlink{XIX.15}{XIX.15}{}; \hyperlink{XX.25}{XX.25}{}; \hyperlink{XXI.24}{XXI.24}{}, \hyperlink{XXI.34}{34}{}; \hyperlink{XXII.116}{XXII.116}{}, \hyperlink{XXII.118}{118}{}; \hyperlink{XXIII.42}{XXIII.42}{}, \hyperlink{XXIII.n17}{n.17}{}
\par\textbf{deathless} (\emph{amata}) \hyperlink{XV.42}{XV.42}{}; \hyperlink{XVI.10}{XVI.10}{}, \hyperlink{XVI.15}{15}{}, \hyperlink{XVI.90}{90}{}; \hyperlink{XXI.103}{XXI.103}{}; \hyperlink{XXII.20}{XXII.20}{}
\par\textbf{decad} (\emph{dasaka}) \hyperlink{XVII.149}{XVII.149}{}, \hyperlink{XVII.151}{151f.}{}, \hyperlink{XVII.156}{156}{}, \hyperlink{XVII.189}{189f.}{}, \hyperlink{XVII.n26}{n.26}{}; \hyperlink{XVIII.5}{XVIII.5f.}{}; \hyperlink{XX.22}{XX.22}{}, \hyperlink{XX.28}{28}{}, \hyperlink{XX.70}{70}{}
\par\textbf{decade} (\emph{dasaka}) \hyperlink{XX.50}{XX.50f.}{}
\par\textbf{deceit} (\emph{māyā}) \hyperlink{I.151}{I.151}{}; \hyperlink{III.95}{III.95}{}; \hyperlink{VII.59}{VII.59}{}
\par\textbf{dedicated} (\emph{niyyātita}) \hyperlink{III.124}{III.124}{}
\par\textbf{deeds} , see kamma
\par\textbf{defilement} (\emph{kilesa, saṅkilesa}) \hyperlink{I.13}{I.13}{}, \hyperlink{I.54}{54}{}, \hyperlink{I.140}{140}{}; \hyperlink{III.18}{III.18}{}, \hyperlink{III.26}{26}{}; \hyperlink{IV.31}{IV.31}{}, \hyperlink{IV.84}{84f.}{}; \hyperlink{VII.59}{VII.59}{}, \hyperlink{VII.n25}{n.25}{}; \hyperlink{VIII.236}{VIII.236}{}; \hyperlink{XII.17}{XII.17}{}; \hyperlink{XIV.145}{XIV.145}{}, \hyperlink{XIV.199}{199}{}; \hyperlink{XVI.68}{XVI.68}{}; \hyperlink{XVII.136}{XVII.136}{}, \hyperlink{XVII.140}{140}{}, \hyperlink{XVII.244}{244}{}, \hyperlink{XVII.281}{281}{}; \hyperlink{XX.110}{XX.110}{}; \hyperlink{XXI.18}{XXI.18}{}, \hyperlink{XXI.41}{41}{}, \hyperlink{XXI.105}{105}{}, \hyperlink{XXI.117}{117}{}; \hyperlink{XXII.7}{XXII.7}{}, \hyperlink{XXII.19}{19f.}{}, \hyperlink{XXII.45}{45}{}, \hyperlink{XXII.49}{49}{}, \hyperlink{XXII.65}{65}{}; \hyperlink{XXIII.2}{XXIII.2}{}
\par\textbf{defining} (\emph{vavatthāna}) \hyperlink{XI.27}{XI.27}{}; \hyperlink{XIV.11}{XIV.11}{}; \hyperlink{XVIII.37}{XVIII.37}{}; \hyperlink{XX.130}{XX.130}{}; d. of states (\emph{dhamma-vavatthāna}) \hyperlink{I.140}{I.140}{}; d. of the four elements (\emph{catudhātu-vavatthāna}) \hyperlink{III.6}{III.6}{}, \hyperlink{III.105}{105}{}; \hyperlink{XI.27}{XI.27f.}{}
\par\textbf{deity} (\emph{deva, devatā) }\hyperlink{I.150}{I.150}{}; \hyperlink{III.58}{III.58}{}, \hyperlink{III.118}{118}{}; \hyperlink{VII.1}{VII.1}{}, \hyperlink{VII.59}{59}{}, \hyperlink{VII.115}{115f.}{}, \hyperlink{VII.n14}{n.14}{}; \hyperlink{IX.69}{IX.69}{}; \hyperlink{X.24}{X.24}{}; \hyperlink{XI.97}{XI.97}{}; \hyperlink{XII.72}{XII.72}{}; \hyperlink{XIV.111}{XIV.111}{}, \hyperlink{XIV.193}{193}{}; \hyperlink{XVII.134}{XVII.134}{}, \hyperlink{XVII.154}{154}{}, \hyperlink{XVII.278}{278}{}, \hyperlink{XVII.n43}{n.43}{}; \hyperlink{XX.15}{XX.15}{}
\par\textbf{delight} (\emph{nandi, abhinandana}) \hyperlink{I.140}{I.140}{}; \hyperlink{VII.59}{VII.59}{}, \hyperlink{VII.n25}{n.25}{}; \hyperlink{VIII.233}{VIII.233}{}; \hyperlink{XVI.61}{XVI.61}{}, \hyperlink{XVI.93}{93}{}; \hyperlink{XVII.30}{XVII.30}{}; \hyperlink{XX.90}{XX.90}{}; \hyperlink{XXI.11}{XXI.11}{}; \hyperlink{XXII.113}{XXII.113}{}; \hyperlink{XXIII.23}{XXIII.23}{}; (\emph{rati}) \hyperlink{XXI.28}{XXI.28}{}
\par\textbf{delimitation of formations} (\emph{saṅkhārapariccheda}) \hyperlink{XVIII.37}{XVIII.37}{}; \hyperlink{XX.4}{XX.4}{}; \hyperlink{XXII.112}{XXII.112}{}
\par\textbf{delimiting-matter} (\emph{pariccheda-rūpa}) \hyperlink{XIV.77}{XIV.77}{}
\par\textbf{deliverance} (\emph{vimutti}) \hyperlink{I.32}{I.32}{}; \hyperlink{IV.117}{IV.117}{}; \hyperlink{VIII.239}{VIII.239}{}; \hyperlink{XIII.12}{XIII.12}{}; \hyperlink{XXII.66}{XXII.66}{}, see mind-d. 
\par\textbf{delusion} (\emph{moha}) \hyperlink{I.90}{I.90}{}, \hyperlink{I.n14}{n.14}{}; \hyperlink{II.84}{II.84}{}, \hyperlink{II.86}{86}{}; \hyperlink{III.74}{III.74f.}{}, \hyperlink{III.95}{95}{}, \hyperlink{III.128}{128}{}; \hyperlink{IV.87}{IV.87}{}; \hyperlink{VII.59}{VII.59}{}; \hyperlink{XII.63}{XII.63}{}; \hyperlink{XIII.64}{XIII.64}{}, \hyperlink{XIII.77}{77}{}; \hyperlink{XIV.93}{XIV.93}{}, \hyperlink{XIV.159}{159}{}, \hyperlink{XIV.161}{161}{}, \hyperlink{XIV.170}{170}{}, \hyperlink{XIV.176}{176}{}; \hyperlink{XVI.69}{XVI.69}{}; \hyperlink{XVII.52}{XVII.52}{}, \hyperlink{XVII.292}{292}{}; \hyperlink{XXII.11}{XXII.11}{}, \hyperlink{XXII.49}{49}{}, \hyperlink{XXII.61}{61}{}
\par\textbf{dependent origination} (\emph{paṭicca-samuppāda}) \hyperlink{VII.9}{VII.9f.}{}, \hyperlink{VII.22}{22}{}, \hyperlink{VII.28}{28}{}; \hyperlink{XVII}{XVII passim}{}; \hyperlink{XVIII.n8}{XVIII.n.8}{}; \hyperlink{XIX.11}{XIX.11f.}{}; \hyperlink{XX.6}{XX.6}{}, \hyperlink{XX.9}{9}{}, \hyperlink{XX.43}{43}{}, \hyperlink{XX.98}{98}{}, \hyperlink{XX.101}{101}{}
\par\textbf{dependently-originated} (\emph{paṭiccasamuppanna}) \hyperlink{XVII.3}{XVII.3f.}{}; \hyperlink{XX.101}{XX.101}{}
\par\textbf{deportment} (\emph{iriyāpatha}) \hyperlink{I.61}{I.61}{}, \hyperlink{I.70}{70}{}. See also posture 
\par\textbf{derived materiality, derivative m.} (\emph{upādārūpa}) \hyperlink{VIII.180}{VIII.180}{}; \hyperlink{XIV.36}{XIV.36f.}{}; \hyperlink{XVII.77}{XVII.77}{}; \hyperlink{XVIII.4}{XVIII.4}{}, \hyperlink{XVIII.14}{14}{}
\par\textbf{Descent of the Gods} (\emph{devorohaṇa}) \hyperlink{XIII.72}{XIII.72f.}{}
\par\textbf{desirable} (\emph{iṭṭha}) \hyperlink{XVII.127}{XVII.127}{}, \hyperlink{XVII.178}{178}{}
\par\textbf{desire} (\emph{paṇidhi}) \hyperlink{I.140}{I.140}{}; \hyperlink{XX.90}{XX.90}{}; \hyperlink{XXI.73}{XXI.73}{}, \hyperlink{XXI.122}{122}{}; \hyperlink{XXII.113}{XXII.113}{}, \hyperlink{XXII.117}{117}{}
\par\textbf{desire, sensual} see sense d.
\par\textbf{desire for deliverance} (\emph{muccitukamyatā}) \hyperlink{XXI.45}{XXI.45}{}, \hyperlink{XXI.79}{79}{}, \hyperlink{XXI.99}{99}{}, \hyperlink{XXI.131}{131f.}{}
\par\textbf{desireless} (\emph{appaṇihita}) \hyperlink{I.140}{I.140}{}. See also contemplation of the d.; d. element (\emph{appaṇihita-dhātu}) \hyperlink{XXI.67}{XXI.67}{}; d. liberation (\emph{appaṇihita-vimokkha}) \hyperlink{XXI.70}{XXI.70f.}{}
\par\textbf{despair} (\emph{upāyāsa}) \hyperlink{XVI.31}{XVI.31}{}, \hyperlink{XVI.52}{52f.}{}; \hyperlink{XVII.2}{XVII.2}{}, \hyperlink{XVII.48}{48}{}
\par\textbf{destiny, destination} (\emph{gati}) \hyperlink{VIII.34}{VIII.34}{}; \hyperlink{XIII.69}{XIII.69}{}, \hyperlink{XIII.92}{92f.}{}; \hyperlink{XIV.111}{XIV.111}{}, \hyperlink{XIV.113}{113}{}; \hyperlink{XVII.135}{XVII.135}{}, \hyperlink{XVII.148}{148}{}, \hyperlink{XVII.160}{160}{}
\par\textbf{destruction} (\emph{khaya}) \hyperlink{I.140}{I.140}{}; \hyperlink{VIII.227}{VIII.227}{}, \hyperlink{VIII.231}{231}{}, \hyperlink{VIII.233}{233}{}; \hyperlink{X.38}{X.38}{}; \hyperlink{XI.104}{XI.104}{}; \hyperlink{XIV.69}{XIV.69}{}; \hyperlink{XVI.69}{XVI.69}{}; \hyperlink{XVII.102}{XVII.102}{}; \hyperlink{XX.14}{XX.14}{}, \hyperlink{XX.21}{21}{}, \hyperlink{XX.40}{40}{}; \hyperlink{XXI.10}{XXI.10}{}, \hyperlink{XXI.24}{24}{}, \hyperlink{XXI.69}{69}{}; \hyperlink{XXII.122}{XXII.122}{}. See also contemplation of d.
\par\textbf{determining} (\emph{voṭṭhapana}) \hyperlink{I.57}{I.57}{}; \hyperlink{IV.n13}{IV.n.13}{}; \hyperlink{XIV.108}{XIV.108}{}, \hyperlink{XIV.120}{120}{}; \hyperlink{XV.36}{XV.36}{}; \hyperlink{XVII.138}{XVII.138}{}; \hyperlink{XX.44}{XX.44}{}
\par\textbf{Devadatta} \hyperlink{XII.138}{XII.138}{}
\par\textbf{Devaputtaraṭṭha} \hyperlink{VIII.243}{VIII.243}{}
\par\textbf{Deva Thera, Mahā} \hyperlink{VIII.49}{VIII.49}{}
\par\textbf{development, developing} (\emph{bhāvanā}) \hyperlink{I.140}{I.140}{}; \hyperlink{III.27}{III.27f.}{}; \hyperlink{XIV.13}{XIV.13}{}, \hyperlink{XIV.206}{206}{}; \hyperlink{XVI.95}{XVI.95}{}, \hyperlink{XVI.102}{102}{}; \hyperlink{XVII.60}{XVII.60}{}, \hyperlink{XVII.76}{76}{}; \hyperlink{XXII.92}{XXII.92}{}, \hyperlink{XXII.124}{124}{}, \hyperlink{XXII.128}{128}{}
\par\textbf{dhamma} (\emph{dhamma}) \hyperlink{I.34}{I.34}{}; \hyperlink{VII.1}{VII.1}{}, \hyperlink{VII.60}{60}{}, \hyperlink{VII.68}{68f.}{}, \hyperlink{VII.n1}{n.1}{}, \hyperlink{VII.n4}{n.4}{}, \hyperlink{VII.n7}{n.7}{}; \hyperlink{VIII.245}{VIII.245}{}, \hyperlink{VIII.n54}{n.54}{}, \hyperlink{VIII.n65}{n.65}{}, \hyperlink{VIII.n68}{n.68}{}, \hyperlink{VIII.n70}{n.70}{}; \hyperlink{X.n7}{X.n.7}{}; \hyperlink{XI.104}{XI.104}{}; \hyperlink{XII.n21}{XII.n.21}{}; \hyperlink{XIV.23}{XIV.23}{}, \hyperlink{XIV.n27}{n.27}{}, \hyperlink{XIV.n78}{n.78}{}; \hyperlink{XVI.n25}{XVI.n.25}{}; \hyperlink{XXII.79}{XXII.79}{}; \hyperlink{XXIII.n18}{XXIII.n.18}{}. See also law \& state; dh. body (\emph{dhamma-kāya}) \hyperlink{VII.60}{VII.60}{}; \hyperlink{VIII.23}{VIII.23}{}; dh. devotee (\emph{dhammānusārin}) \hyperlink{XXI.74}{XXI.74}{}, \hyperlink{XXI.89}{89}{}
\par\textbf{Dhammadinna Thera} \hyperlink{XII.80}{XII.80}{}; \hyperlink{XX.111}{XX.111}{}
\par\textbf{Dhammaguttā} \hyperlink{XII.39}{XII.39}{}
\par\textbf{Dhammarakkhita Thera, Mahā} \hyperlink{III.53}{III.53}{}
\par\textbf{Dhammāsoka} \hyperlink{III.111}{III.111}{}
\par\textbf{Dhammika} \hyperlink{XIV.27}{XIV.27}{}
\par\textbf{Dhanañjaya} \hyperlink{XII.42}{XII.42}{}
\par\textbf{Dīpaṅkara} \hyperlink{VII.34}{VII.34}{}
\par\textbf{direct-knowledge} (\emph{abhiññā}) \hyperlink{I.11}{I.11}{}, \hyperlink{I.140}{140}{}, \hyperlink{I.n6}{n.6}{}; \hyperlink{III.5}{III.5}{}, \hyperlink{III.14}{14f.}{}, \hyperlink{III.120}{120}{}; \hyperlink{IV.75}{IV.75}{}, \hyperlink{IV.78}{78}{}; \hyperlink{VII.30}{VII.30}{}; \hyperlink{XI.122}{XI.122}{}; \hyperlink{XII}{XII passim}{}; \hyperlink{XIII}{XIII passim}{}; \hyperlink{XVII.61}{XVII.61}{}, \hyperlink{XVII.102}{102}{}; \hyperlink{XX.3}{XX.3}{}, \hyperlink{XX.31}{31}{}; \hyperlink{XXII.106}{XXII.106}{}; \hyperlink{XXIII.58}{XXIII.58}{}
\par\textbf{directing on to} (\emph{abhiniropana}) \hyperlink{IV.90}{IV.90}{}; \hyperlink{XVI.77}{XVI.77}{}, \hyperlink{XVI.100}{100}{}
\par\textbf{discerning} (\emph{pariggaha}) \hyperlink{IV.114}{IV.114}{}; \hyperlink{VIII.180}{VIII.180}{}; \hyperlink{XVIII.3}{XVIII.3f.}{}; \hyperlink{XIX.4}{XIX.4f.}{}; \hyperlink{XXII.39}{XXII.39}{}; d. of conditions (\emph{paccayapariggaha}) \hyperlink{XIX}{XIX passim}{}; \hyperlink{XX.4}{XX.4}{}, \hyperlink{XX.130}{130}{}; \hyperlink{XXII.112}{XXII.112}{}
\par\textbf{disciple} (\emph{sāvaka}) \hyperlink{I.98}{I.98}{}, \hyperlink{I.131}{131}{}; \hyperlink{IV.55}{IV.55}{}; \hyperlink{XIV.31}{XIV.31}{}
\par\textbf{discipline} (\emph{vinaya}) \hyperlink{I.32}{I.32}{}
\par\textbf{discord} (\emph{vivāda}) \hyperlink{VII.59}{VII.59}{}
\par\textbf{discrimination} (\emph{paṭisambhidā}) \hyperlink{I.11}{I.11}{}; \hyperlink{XIV.21}{XIV.21f.}{}; \hyperlink{XVII.33}{XVII.33}{}, \hyperlink{XVII.305}{305}{}; \hyperlink{XX.111}{XX.111}{}; \hyperlink{XXIII.58}{XXIII.58}{}
\par\textbf{disease} (\emph{roga}) \hyperlink{XI.21}{XI.21}{}; \hyperlink{XX.18}{XX.18}{}; \hyperlink{XXI.35}{XXI.35}{}, \hyperlink{XXI.48}{48}{}, \hyperlink{XXI.59}{59}{}; \hyperlink{XXII.98}{XXII.98}{}
\par\textbf{dispassion} (\emph{nibbidā}) \hyperlink{I.32}{I.32}{}, \hyperlink{I.140}{140}{}; \hyperlink{III.22}{III.22}{}; \hyperlink{VIII.224}{VIII.224}{}; \hyperlink{X.52}{X.52}{}; \hyperlink{XXI.26}{XXI.26}{}, \hyperlink{XXI.135}{135}{}. See also contemplation of d. 
\par\textbf{dispensation} (\emph{sāsana}) \hyperlink{I.10}{I.10}{}; \hyperlink{VIII.152}{VIII.152}{}
\par\textbf{dissolution} (\emph{bhaṅga}) \hyperlink{VIII.234}{VIII.234}{}, \hyperlink{VIII.242}{242}{}; \hyperlink{XIII.111}{XIII.111}{}; \hyperlink{XIV.59}{XIV.59}{}, \hyperlink{XIX.11}{XIX.11}{}; \hyperlink{XX.22}{XX.22}{}, \hyperlink{XX.26}{26}{}; \hyperlink{XXI.11}{XXI.11}{}; \hyperlink{XXII.115}{XXII.115}{}, \hyperlink{XXII.118}{118}{}. See also contemplation of d. 
\par\textbf{distension} (\emph{vitthambhana}) \hyperlink{XI.37}{XI.37}{}, \hyperlink{XI.84}{84f.}{}, \hyperlink{XI.89}{89f.}{}, \hyperlink{XI.93}{93}{}, \hyperlink{XI.n23}{n.23}{}
\par\textbf{distinction} (\emph{visesa}) \hyperlink{III.22}{III.22}{}, \hyperlink{III.26}{26}{}, \hyperlink{III.128}{128}{}
\par\textbf{distraction} (\emph{vikkhepa}) \hyperlink{III.4}{III.4}{}, \hyperlink{III.n3}{n.3}{}
\par\textbf{diversification} (\emph{papañca}) \hyperlink{VII.59}{VII.59}{}; \hyperlink{XVI.n17}{XVI.n.17}{}
\par\textbf{diversity} (\emph{nānatta}) \hyperlink{XVII.309}{XVII.309}{}, \hyperlink{XVII.311}{311}{}; \hyperlink{XX.102}{XX.102}{}
\par\textbf{divine abiding} (\emph{brahma-vihāra}) \hyperlink{III.105}{III.105f.}{}; \hyperlink{VII.63}{VII.63}{}; \hyperlink{IX}{IX passim}{}
\par\textbf{divine ear element} (\emph{dibba-sotadhātu}) \hyperlink{III.109}{III.109}{}; \hyperlink{XII.2}{XII.2}{}, \hyperlink{XII.136}{136}{}; \hyperlink{XIII.1}{XIII.1f.}{}, \hyperlink{XIII.109}{109}{}
\par\textbf{divine eye} (\emph{dibba-cakkhu}) \hyperlink{III.109}{III.109}{}; \hyperlink{V.30}{V.30}{}, \hyperlink{V.35}{35f.}{}; \hyperlink{XII.100}{XII.100}{}, \hyperlink{XII.129}{129}{}, \hyperlink{XII.136}{136}{}; \hyperlink{XIII.9}{XIII.9}{}, \hyperlink{XIII.72}{72f.}{}, \hyperlink{XIII.80}{80}{}, \hyperlink{XIII.124}{124}{}; \hyperlink{XX.120}{XX.120}{}
\par\textbf{divine world} (\emph{deva-loka}) \hyperlink{XVII.141}{XVII.141}{}
\par\textbf{doer} (\emph{kāraka}) \hyperlink{XVI.90}{XVI.90}{}; \hyperlink{XVII.273}{XVII.273}{}, \hyperlink{XVII.282}{282}{}, \hyperlink{XVII.302}{302}{}; \hyperlink{XIX.20}{XIX.20}{}; \hyperlink{XX.16}{XX.16}{}
\par\textbf{dog-practice} (\emph{kukkura-kiriyā}) \hyperlink{XVII.246}{XVII.246}{}
\par\textbf{domain} (\emph{gocara}) \hyperlink{X.20}{X.20}{}; \hyperlink{XIV.26}{XIV.26}{}; \hyperlink{XV.11}{XV.11}{}; \hyperlink{XXII.46}{XXII.46}{}
\par\textbf{door} (\emph{dvāra}) \hyperlink{I.53}{I.53}{}, \hyperlink{I.n16}{n.16}{}; \hyperlink{IV.n13}{IV.n.13}{}; \hyperlink{X.17}{X.17}{}; \hyperlink{X.22}{X.22}{}; \hyperlink{XIV.47}{XIV.47}{}, \hyperlink{XIV.78}{78}{}, \hyperlink{XIV.98}{98}{}, \hyperlink{XIV.108}{108}{}, \hyperlink{XIV.115}{115f.}{}, \hyperlink{XIV.121}{121}{}, \hyperlink{XIV.141}{141}{}, \hyperlink{XIV.152}{152}{}; \hyperlink{XV.4}{XV.4}{}, \hyperlink{XV.36}{36f.}{}; \hyperlink{XVII.51}{XVII.51}{}, \hyperlink{XVII.61}{61}{}, \hyperlink{XVII.127}{127}{}, \hyperlink{XVII.136}{136f.}{}, \hyperlink{XVII.228}{228}{}; \hyperlink{XIX.8}{XIX.8}{}; \hyperlink{XX.9}{XX.9}{}, \hyperlink{XX.70}{70}{}. See also mind-d., etc. 
\par\textbf{dominance} (\emph{adhipateyya}) \hyperlink{XXII.102}{XXII.102}{}
\par\textbf{doubt} (\emph{kaṅkhā}) \hyperlink{VIII.224}{VIII.224}{}; \hyperlink{XIX}{XIX passim}{}; \hyperlink{XXII.112}{XXII.112}{}
\par\textbf{dream} (\emph{supina}) \hyperlink{IX.62}{IX.62}{}; \hyperlink{XIV.114}{XIV.114}{}, \hyperlink{XIV.n45}{n.45}{}
\par\textbf{duty} (\emph{vatta}) \hyperlink{I.17}{I.17}{}; \hyperlink{III.66}{III.66f.}{}; \hyperlink{IV.3}{IV.3}{}
\par\textbf{dyad} (\emph{duka}) \hyperlink{XIII.n20}{XIII.n.20}{}
\par\textbf{ear} (\emph{sota}) \hyperlink{XIV.36}{XIV.36}{}, \hyperlink{XIV.38}{38}{}, \hyperlink{XIV.49}{49}{}, \hyperlink{XIV.117}{117}{}; \hyperlink{XV.3}{XV.3}{}; \hyperlink{XX.70}{XX.70}{}; e. base (\emph{sotāyatana}) \hyperlink{XV.3}{XV.3f.}{}; e. consciousness (\emph{sotaviññāṇa}) \hyperlink{XIV.96}{XIV.96f.}{}, \hyperlink{XIV.101}{101}{}, \hyperlink{XIV.117}{117}{}, \hyperlink{XIV.179}{179}{}; e.-c. element (\emph{sota-viññāṇa-dhātu}) \hyperlink{XV.17}{XV.17f.}{}; e. element (\emph{sota-dhātu}) \hyperlink{XIII.2}{XIII.2}{}; \hyperlink{XV.17}{XV.17f.}{}; e. faculty (\emph{sotindriya}) \hyperlink{XVI.1}{XVI.1}{}
\par\textbf{earth} (\emph{pathavī}) \hyperlink{VII.41}{VII.41}{}; \hyperlink{XI.28}{XI.28f.}{}, \hyperlink{XI.33}{33}{}, \hyperlink{XI.41}{41}{}, \hyperlink{XI.87}{87}{}; \hyperlink{XIII.43}{XIII.43}{}, \hyperlink{XIII.99}{99}{}; \hyperlink{XIV.35}{XIV.35}{}, \hyperlink{XIV.62}{62}{}; \hyperlink{XV.30}{XV.30}{}, \hyperlink{XV.34}{34}{}, \hyperlink{XV.39}{39}{}; \hyperlink{XVIII.19}{XVIII.19}{}; e. element (\emph{pathavī-dhātu}), see earth; e. kasiṇa (\emph{pathavī-kasiṇa}) \hyperlink{III.105}{III.105f.}{}, \hyperlink{III.131}{131}{}; \hyperlink{IV}{IV passim}{}; \hyperlink{XII.95}{XII.95}{}, \hyperlink{XII.132}{132}{}, \hyperlink{XII.138}{138}{}
\par\textbf{effacement} (\emph{sallekha}) \hyperlink{I.116}{I.116}{}, \hyperlink{I.151}{151}{}; \hyperlink{II.84}{II.84}{}, \hyperlink{II.86}{86}{}
\par\textbf{effort} (\emph{vāyāma}) \hyperlink{XXII.39}{XXII.39}{}, \hyperlink{XXII.66}{66}{}; (\emph{padhāna}), see endeavour 
\par\textbf{eightfold path} (\emph{aṭṭhaṅgika-magga}) \hyperlink{XVI.75}{XVI.75f.}{}, \hyperlink{XVI.95}{95}{}; \hyperlink{XXII.33}{XXII.33}{}
\par\textbf{ekabījin} \hyperlink{XXIII.55}{XXIII.55}{}
\par\textbf{element} (\emph{dhātu}) \hyperlink{I.86}{I.86}{}, \hyperlink{I.97}{97}{}; \hyperlink{III.80}{III.80}{}; \hyperlink{VII.28}{VII.28}{}, \hyperlink{VII.38}{38}{}, \hyperlink{VII.n1}{n.1}{}; \hyperlink{VIII.43}{VIII.43}{}, \hyperlink{VIII.60}{60}{}, \hyperlink{VIII.159}{159}{}, \hyperlink{VIII.180}{180}{}, \hyperlink{VIII.n43}{n.43}{}, \hyperlink{VIII.n68}{n.68}{}; \hyperlink{IX.38}{IX.38}{}; \hyperlink{XI.27}{XI.27f.}{}, \hyperlink{XI.87}{87f.}{}, \hyperlink{XI.93}{93}{}, \hyperlink{XI.104}{104}{}, \hyperlink{XI.n19}{n.19}{}; \hyperlink{XII.37}{XII.37}{}; \hyperlink{XIV.47}{XIV.47}{}, \hyperlink{XIV.65}{65}{}; \hyperlink{XV.17}{XV.17f.}{}, \hyperlink{XV.n11}{n.11}{}; \hyperlink{XVI.59}{XVI.59}{}; \hyperlink{XVII.156}{XVII.156}{}; \hyperlink{XVIII.5}{XVIII.5}{}, \hyperlink{XVIII.9}{9}{}, \hyperlink{XVIII.19}{19}{}; \hyperlink{XX.9}{XX.9}{}, \hyperlink{XX.64}{64}{}
\par\textbf{elephant} (\emph{hatthin}) \hyperlink{XXI.n15}{XXI.n.15}{}
\par\textbf{embryo} (\emph{kalala}) \hyperlink{VIII.30}{VIII.30}{}; \hyperlink{XVII.117}{XVII.117}{}, \hyperlink{XVII.152}{152}{}
\par\textbf{emergence} (\emph{vuṭṭhāna}) \hyperlink{IV.131}{IV.131}{}; \hyperlink{XVI.23}{XVI.23}{}; \hyperlink{XX.44}{XX.44f.}{}; \hyperlink{XXI.83}{XXI.83f.}{}; \hyperlink{XXIII.13}{XXIII.13}{}, \hyperlink{XXIII.49}{49}{}
\par\textbf{endeavour} (\emph{padhāna}) \hyperlink{IV.55}{IV.55}{}; \hyperlink{XII.50}{XII.50f.}{}; \hyperlink{XXII.33}{XXII.33f.}{}, \hyperlink{XXII.39}{39}{}, \hyperlink{XXII.42}{42}{}
\par\textbf{energy} (\emph{viriya}) \hyperlink{I.18}{I.18}{}, \hyperlink{I.26}{26}{}, \hyperlink{I.33}{33}{}, \hyperlink{I.111}{111}{}; \hyperlink{II.88}{II.88}{}; \hyperlink{III.24}{III.24}{}; \hyperlink{IV.45}{IV.45f.}{}, \hyperlink{IV.51}{51}{}, \hyperlink{IV.72}{72}{}, \hyperlink{IV.113}{113}{}; \hyperlink{VII.7}{VII.7}{}, \hyperlink{VII.n8}{n.8}{}; \hyperlink{IX.124}{IX.124}{}; \hyperlink{XII.12}{XII.12}{}, \hyperlink{XII.17}{17}{}, \hyperlink{XII.50}{50}{}; \hyperlink{XIV.133}{XIV.133}{}, \hyperlink{XIV.137}{137}{}, \hyperlink{XIV.159}{159}{}, \hyperlink{XIV.170}{170}{}, \hyperlink{XIV.176}{176}{}, \hyperlink{XIV.183}{183}{}; \hyperlink{XVI.1}{XVI.1}{}, \hyperlink{XVI.86}{86}{}; \hyperlink{XVII.72}{XVII.72}{}; \hyperlink{XX.119}{XX.119}{}; \hyperlink{XXII.35}{XXII.35f.}{}
\par\textbf{engagement} (\emph{abhinipāta}) \hyperlink{XVIII.19}{XVIII.19}{}
\par\textbf{Enlightened One} (\emph{buddha}) \hyperlink{III.128}{III.128}{}; \hyperlink{IV.55}{IV.55}{}; \hyperlink{VII.1}{VII.1f.}{}, \hyperlink{VII.26}{26}{}, \hyperlink{VII.52}{52}{}; \hyperlink{VIII.23}{VIII.23}{}, \hyperlink{VIII.155}{155}{}; \hyperlink{IX.124}{IX.124}{}; \hyperlink{XI.17}{XI.17}{}; \hyperlink{XII.1}{XII.1}{}; \hyperlink{XIII.16}{XIII.16}{}, \hyperlink{XIII.18}{18}{}, \hyperlink{XIII.31}{31}{}; \hyperlink{XIV.31}{XIV.31}{}; \hyperlink{XVI.20}{XVI.20}{}; \hyperlink{XXI.118}{XXI.118}{}; \hyperlink{XXIII.11}{XXIII.11}{}
\par\textbf{enlightenment} (\emph{bodhi, sambodhi}) \hyperlink{I.140}{I.140}{}
\par\textbf{enlightenment factor} (\emph{bojjhaṅga, sambojjhaṅga}) \hyperlink{IV.51}{IV.51f.}{}; \hyperlink{VIII.141}{VIII.141}{}, \hyperlink{VIII.239}{239}{}; \hyperlink{XVI.86}{XVI.86}{}; \hyperlink{XX.21}{XX.21}{}; \hyperlink{XXI.111}{XXI.111}{}; \hyperlink{XXII.22}{XXII.22}{}, \hyperlink{XXII.33}{33}{}, \hyperlink{XXII.38}{38f.}{}, \hyperlink{XXII.42}{42}{}
\par\textbf{ennead} (\emph{navaka}) \hyperlink{XVII.194}{XVII.194}{}. See also life-e. and sound-e.
\par\textbf{entering into} (\emph{pakkhandana}) \hyperlink{XIV.140}{XIV.140}{}; \hyperlink{XXI.18}{XXI.18}{}; \hyperlink{XXII.n2}{XXII.n.2}{}
\par\textbf{envy} (\emph{issā}) \hyperlink{III.95}{III.95}{}; \hyperlink{VII.59}{VII.59}{}; \hyperlink{XIV.170}{XIV.170}{}, \hyperlink{XIV.172}{172}{}
\par\textbf{equanimity} (\emph{upekkhā}) \hyperlink{I.n14}{I.n.14}{}; \hyperlink{III.5}{III.5}{}, \hyperlink{III.9}{9}{}, \hyperlink{III.12}{12}{}, \hyperlink{III.21}{21}{}, \hyperlink{III.105}{105f.}{}, \hyperlink{III.n6}{n.6}{}; \hyperlink{IV.51}{IV.51}{}, \hyperlink{IV.112}{112f.}{}, \hyperlink{IV.156}{156f.}{}, \hyperlink{IV.182}{182}{}, \hyperlink{IV.193}{193}{}; \hyperlink{VII.18}{VII.18}{}; \hyperlink{VIII.74}{VIII.74}{}; \hyperlink{IX.88}{IX.88f.}{}, \hyperlink{IX.92}{92}{}, \hyperlink{IX.96}{96}{}, \hyperlink{IX.101}{101}{}, \hyperlink{IX.119}{119}{}, \hyperlink{IX.123}{123}{}; \hyperlink{XII.37}{XII.37}{}, \hyperlink{XII.131}{131}{}; \hyperlink{XIII.9}{XIII.9}{}; \hyperlink{XIV.12}{XIV.12}{}, \hyperlink{XIV.83}{83}{}, \hyperlink{XIV.85}{85f.}{}, \hyperlink{XIV.119}{119}{}, \hyperlink{XIV.126}{126f.}{}, \hyperlink{XIV.154}{154}{}; \hyperlink{XV.28}{XV.28}{}; \hyperlink{XVI.86}{XVI.86}{}; \hyperlink{XVII.160}{XVII.160}{}; \hyperlink{XX.44}{XX.44}{}, \hyperlink{XX.121}{121}{}; \hyperlink{XXI.63}{XXI.63}{}, \hyperlink{XXI.83}{83}{}, \hyperlink{XXI.114}{114}{}; \hyperlink{XXII.42}{XXII.42}{}; e. about formations (\emph{saṅkhārupekkhā}) \hyperlink{XXI.61}{XXI.61f.}{}, \hyperlink{XXI.79}{79}{}, \hyperlink{XXI.131}{131}{}, \hyperlink{XXI.135}{135}{}; \hyperlink{XXII.23}{XXII.23}{}, \hyperlink{XXII.26}{26}{}, \hyperlink{XXII.29}{29}{}, \hyperlink{XXII.121}{121}{}; e. faculty (\emph{upekkhindriya}) \hyperlink{XIII.9}{XIII.9}{}; \hyperlink{XVI.1}{XVI.1}{}
\par\textbf{eon} (\emph{kappa}) \hyperlink{XI.102}{XI.102}{}; \hyperlink{XIII.16}{XIII.16}{}; \hyperlink{XI.205}{XI.205}{}; \hyperlink{XX.73}{XX.73}{}
\par\textbf{equipoise} (\emph{tādibhāva}) \hyperlink{I.10}{I.10}{}; \hyperlink{VII.71}{VII.71}{}
\par\textbf{escape} (\emph{nissaraṇa}) \hyperlink{I.32}{I.32}{}; \hyperlink{IV.82}{IV.82}{}; \hyperlink{VII.29}{VII.29}{}; \hyperlink{XVI.15}{XVI.15}{}
\par\textbf{essence} (\emph{bhāva}) \hyperlink{I.32}{I.32}{}; \hyperlink{VIII.234}{VIII.234}{}, \hyperlink{VIII.n68}{n.68}{}; \hyperlink{XVII.14}{XVII.14}{}; e., individual (\emph{sabhāva}), see  individual e. 
\par\textbf{eternity view, eternalism} (\emph{sassatadiṭṭhi}) \hyperlink{XVI.85}{XVI.85}{}; \hyperlink{XVII.22}{XVII.22}{}, \hyperlink{XVII.235}{235}{}, \hyperlink{XVII.286}{286}{}, \hyperlink{XVII.310}{310}{}; \hyperlink{XVII.29}{XVII.29}{}; \hyperlink{XX.102}{XX.102}{}; \hyperlink{XXI.28}{XXI.28}{}; \hyperlink{XXII.112}{XXII.112}{}
\par\textbf{eunuch} (\emph{paṇḍaka}) \hyperlink{V.41}{V.41}{}; \hyperlink{XIV.111}{XIV.111}{}
\par\textbf{evilness of wishes} (\emph{pāpicchatā}) \hyperlink{III.95}{III.95}{}
\par\textbf{exalted} (\emph{mahaggata}) \hyperlink{II.5}{II.5}{}, \hyperlink{II.13}{13}{}; \hyperlink{IV.74}{IV.74}{}; \hyperlink{X.34}{X.34}{}; \hyperlink{XIII.12}{XIII.12}{}, \hyperlink{XIII.106}{106}{}, \hyperlink{XIII.120}{120}{}; \hyperlink{XIV.15}{XIV.15}{}; \hyperlink{XVII.53}{XVII.53}{}, \hyperlink{XVII.140}{140}{}; \hyperlink{XIX.15}{XIX.15}{}; \hyperlink{XX.44}{XX.44}{}; \hyperlink{XXIII.4}{XXIII.4}{}
\par\textbf{event} (\emph{samaya}) \hyperlink{VII.n1}{VII.n.1}{}
\par\textbf{exclusive, absolute} (\emph{advaya}) \hyperlink{V.38}{V.38}{}
\par\textbf{exertion} (\emph{paggaha}) \hyperlink{VIII.74}{VIII.74}{}; \hyperlink{XX.119}{XX.119}{}; \hyperlink{XXII.45}{XXII.45}{}
\par\textbf{existence} (\emph{atthitā}) \hyperlink{XVII.n1}{XVII.n.1}{}; (\emph{atthibhāva}) \hyperlink{XVII.n16}{XVII.n.16}{}; (\emph{bhava}) \hyperlink{VII.n6}{VII.n.6}{}; \hyperlink{IX.97}{IX.97}{}. See also becoming 
\par\textbf{existing} (\emph{vijjamāna}) \hyperlink{IX.123}{IX.123}{}; (\emph{vattamāna}) \hyperlink{XVI.68}{XVI.68}{}
\par\textbf{expansion} (\emph{vivaṭṭa}) \hyperlink{XIII.28}{XIII.28f.}{}; \hyperlink{XX.73}{XX.73}{}
\par\textbf{experiencer} (\emph{upabhuñjaka}) \hyperlink{XVII.171}{XVII.171}{}; (\emph{vedaka}) \hyperlink{XVI.90}{XVI.90}{}; \hyperlink{XVII.273}{XVII.273}{}, \hyperlink{XVII.282}{282}{}; \hyperlink{XX.16}{XX.16}{}
\par\textbf{extension} (\emph{vaḍḍhana}) \hyperlink{III.109}{III.109}{}; \hyperlink{IV.126}{IV.126f.}{}; \hyperlink{V.26}{V.26}{}; \hyperlink{XIII.6}{XIII.6}{}
\par\textbf{extent} (\emph{addhā, addhāna}) \hyperlink{IV.78}{IV.78}{}; \hyperlink{VIII.165}{VIII.165}{}; \hyperlink{XIII.111}{XIII.111f.}{}; \hyperlink{XIV.186}{XIV.186f.}{}
\par\textbf{external} (\emph{bahiddhā, bāhira}) \hyperlink{XI.107}{XI.107}{}; \hyperlink{XIII.106}{XIII.106f.}{}; \hyperlink{XIV.19}{XIV.19}{}, \hyperlink{XIV.73}{73}{}, \hyperlink{XIV.192}{192}{}, \hyperlink{XIV.198}{198}{}; \hyperlink{XX.73}{XX.73}{}; \hyperlink{XXI.83}{XXI.83f.}{}; \hyperlink{XXII.5}{XXII.5}{}, \hyperlink{XXII.45}{45}{}; \hyperlink{XXIII.4}{XXIII.4}{}
\par\textbf{extinction} (of craving, etc.) (\emph{nibbāna, parinibbāna}), see Nibbāna 
\par\textbf{eye} (\emph{cakkhu}) \hyperlink{I.42}{I.42}{}, \hyperlink{I.53}{53}{}, \hyperlink{I.n14}{n.14}{}; \hyperlink{VII.27}{VII.27}{}; \hyperlink{X.16}{X.16}{}; \hyperlink{XIV.36}{XIV.36f.}{}, \hyperlink{XIV.47}{47}{}, \hyperlink{XIV.73}{73}{}, \hyperlink{XIV.115}{115}{}, \hyperlink{XIV.117}{117}{}, \hyperlink{XIV.213}{213}{}; \hyperlink{XV.3}{XV.3}{}; \hyperlink{XVI.6}{XVI.6}{}; \hyperlink{XVII.127}{XVII.127}{}; \hyperlink{XX.6}{XX.6f.}{}, \hyperlink{XX.44}{44}{}, \hyperlink{XX.94}{94}{}, \hyperlink{XXI.11}{XXI.11}{}, \hyperlink{XXI.55}{55f.}{}, \hyperlink{XXIII.22}{XXIII.22}{}; e. base (\emph{cakkhāyatana}) \hyperlink{XV.3}{XV.3f.}{}; \hyperlink{XVI.10}{XVI.10}{}; e.-consciousness (\emph{cakkhuviññāṇa}) \hyperlink{I.57}{I.57}{}; \hyperlink{XIV.47}{XIV.47}{}, \hyperlink{XIV.54}{54}{}, \hyperlink{XIV.95}{95f.}{}, \hyperlink{XIV.101}{101}{}, \hyperlink{XIV.107}{107}{}, \hyperlink{XIV.117}{117}{}, \hyperlink{XIV.179}{179}{}; \hyperlink{XVI.10}{XVI.10}{}; \hyperlink{XVII.73}{XVII.73}{}, \hyperlink{XVII.n20}{n.20}{}; \hyperlink{XX.44}{XX.44}{}; e.-c. element (\emph{cakkhuviññāṇa-dhātu}) \hyperlink{XV.17}{XV.17}{}; \hyperlink{XVI.10}{XVI.10}{}; e. decad (\emph{cakkhu-dasaka}) \hyperlink{XVII.156}{XVII.156}{}, \hyperlink{XVII.190}{190}{}; e-door (\emph{cakkhudvāra}) \hyperlink{XIII.118}{XIII.118}{}; \hyperlink{XIV.117}{XIV.117}{}; e. element (\emph{cakkhudhātu}) \hyperlink{XV.17}{XV.17}{}; e. faculty (\emph{cakkhundriya}) \hyperlink{XVI.1}{XVI.1}{}
\par\textbf{faculty} (\emph{indriya}) \hyperlink{I.42}{I.42}{}, \hyperlink{I.56}{56f.}{}, \hyperlink{I.100}{100}{}; \hyperlink{III.18}{III.18}{}; \hyperlink{IV.45}{IV.45}{}, \hyperlink{IV.61}{61}{}, \hyperlink{IV.117}{117f.}{}, \hyperlink{IV.186}{186}{}; \hyperlink{XI.107}{XI.107}{}; \hyperlink{XIV.58}{XIV.58f.}{}, \hyperlink{XIV.73}{73}{}, \hyperlink{XIV.79}{79}{}, \hyperlink{XIV.115}{115}{}, \hyperlink{XIV.134}{134}{}; \hyperlink{XVI.1}{XVI.1f.}{}, \hyperlink{XVI.10}{10}{}, \hyperlink{XVI.86}{86}{}; \hyperlink{XVII.91}{XVII.91}{}, \hyperlink{XVII.127}{127}{}, \hyperlink{XVII.150}{150}{}, \hyperlink{XVII.163}{163}{}; \hyperlink{XVIII.n8}{XVIII.n.8}{}; \hyperlink{XX.9}{XX.9}{}, \hyperlink{XX.21}{21}{}; \hyperlink{XXI.75}{XXI.75}{}, \hyperlink{XXI.89}{89}{}, \hyperlink{XXI.n31}{n.31}{}; \hyperlink{XXII.22}{XXII.22}{}, \hyperlink{XXII.33}{33}{}, \hyperlink{XXII.37}{37}{}, \hyperlink{XXII.42}{42}{}; \hyperlink{XXIII.51}{XXIII.51}{}, \hyperlink{XXIII.55}{55}{}, \hyperlink{XXIII.56}{56}{}
\par\textbf{fading away} (\emph{virāga}) \hyperlink{I.32}{I.32}{}, \hyperlink{I.140}{140}{}; \hyperlink{VIII.235}{VIII.235}{}, \hyperlink{VIII.245}{245}{}; \hyperlink{XX.7}{XX.7}{}; \hyperlink{XXI.11}{XXI.11}{}. See also contemplation of f.a. 
\par\textbf{faith} (\emph{saddhā}) \hyperlink{I.26}{I.26}{}, \hyperlink{I.68}{68}{}, \hyperlink{I.98}{98}{}; \hyperlink{III.74}{III.74f.}{}, \hyperlink{III.122}{122}{}; \hyperlink{IV.45}{IV.45}{}, \hyperlink{IV.142}{142}{}; \hyperlink{VII.7}{VII.7}{}, \hyperlink{VII.115}{115f.}{}, \hyperlink{VII.n4}{n.4}{}, \hyperlink{VII.n8}{n.8}{}; \hyperlink{XII.17}{XII.17}{}; \hyperlink{XIV.133}{XIV.133}{}, \hyperlink{XIV.140}{140}{}, \hyperlink{XIV.148}{148}{}, \hyperlink{XIV.155}{155}{}; \hyperlink{XVI.1}{XVI.1}{}, \hyperlink{XVI.10}{10}{}, \hyperlink{XVI.86}{86}{}; \hyperlink{XVII.84}{XVII.84}{}; \hyperlink{XX.118}{XX.118}{}; \hyperlink{XXI.74}{XXI.74}{}, \hyperlink{XXI.89}{89}{}, \hyperlink{XXI.128}{128}{}, \hyperlink{XXI.n34}{n.34}{}; \hyperlink{XXII.37}{XXII.37}{}, \hyperlink{XXII.42}{42}{}
\par\textbf{faith devotee} (\emph{saddhānusārin}) \hyperlink{XXI.74}{XXI.74}{}, \hyperlink{XXI.89}{89}{}
\par\textbf{fall} (\emph{vāya}) \hyperlink{I.140}{I.140}{}; \hyperlink{VIII.227}{VIII.227}{}, \hyperlink{VIII.231}{231}{}, \hyperlink{VIII.233}{233}{}; \hyperlink{XIV.69}{XIV.69}{}; \hyperlink{XV.15}{XV.15}{}; \hyperlink{XVI.35}{XVI.35}{}; \hyperlink{XVII.102}{XVII.102}{}; \hyperlink{XX.7}{XX.7}{}, \hyperlink{XX.94}{94}{}; \hyperlink{XXI.6}{XXI.6}{}, \hyperlink{XXI.22}{22}{}. See also rise and f., contemplation of f., and contemplation of rise and f. 
\par\textbf{false speech} (\emph{musāvāda}) \hyperlink{I.140}{I.140}{}; \hyperlink{XXII.66}{XXII.66}{}
\par\textbf{far} (\emph{dūra}) \hyperlink{XIV.73}{XIV.73}{}, \hyperlink{XIV.194}{194}{}, \hyperlink{XIV.209}{209}{}
\par\textbf{fatalism} (\emph{niyata-vāda}) \hyperlink{XVII.313}{XVII.313}{}
\par\textbf{fate} (\emph{niyati}) \hyperlink{XVII.117}{XVII.117}{}
\par\textbf{fear} (\emph{bhaya}) \hyperlink{III.124}{III.124}{}. See also terror, and appearance as t. 
\par\textbf{feeling} (\emph{vedanā}) \hyperlink{IV.182}{IV.182}{}, \hyperlink{IV.193}{193}{}; \hyperlink{VII.14}{VII.14}{}, \hyperlink{VII.28}{28}{}, \hyperlink{VII.38}{38}{}, \hyperlink{VII.n13}{n.13}{}; \hyperlink{VIII.230}{VIII.230}{}; \hyperlink{IX.111}{IX.111}{}; \hyperlink{X.50}{X.50}{}; \hyperlink{XIV.125}{XIV.125f.}{}, \hyperlink{XIV.134}{134}{}, \hyperlink{XIV.144}{144}{}, \hyperlink{XIV.197}{197f.}{}; \hyperlink{XV.14}{XV.14}{}; \hyperlink{XVI.35}{XVI.35}{}; \hyperlink{XVII.2}{XVII.2}{}, \hyperlink{XVII.9}{9}{}, \hyperlink{XVII.32}{32}{}, \hyperlink{XVII.51}{51}{}, \hyperlink{XVII.163}{163}{}, \hyperlink{XVII.228}{228f.}{}, \hyperlink{XVII.294}{294}{}; \hyperlink{XVIII.8}{XVIII.8}{}, \hyperlink{XVIII.13}{13}{}, \hyperlink{XVIII.91}{91f.}{}; \hyperlink{XIX.13}{XIX.13}{}; \hyperlink{XX.7}{XX.7}{}, \hyperlink{XX.9}{9}{}, \hyperlink{XX.94}{94}{}; \hyperlink{XXI.11}{XXI.11}{}, \hyperlink{XXI.56}{56}{}; \hyperlink{XXII.22}{XXII.22}{}, \hyperlink{XXII.34}{34}{}, \hyperlink{XXII.126}{126}{}; \hyperlink{XXIII.13}{XXIII.13}{}, \hyperlink{XXIII.22}{22}{}, \hyperlink{XXIII.24}{24}{}
\par\textbf{femininity faculty} (\emph{itthindriya}) \hyperlink{XIV.58}{XIV.58}{}; \hyperlink{XV.1}{XV.1}{}
\par\textbf{foetus} (\emph{gabbha}) \hyperlink{IV.34}{IV.34}{}. See also embryo
\par\textbf{fetter} (\emph{saṃyojana}) \hyperlink{IV.87}{IV.87}{}; \hyperlink{XIV.172}{XIV.172}{}; \hyperlink{XXII.45}{XXII.45}{}, \hyperlink{XXII.47}{47}{}, \hyperlink{XXII.48}{48}{}, \hyperlink{XXII.122}{122}{}; \hyperlink{XXIII.2}{XXIII.2}{}
\par\textbf{fewness of wishes} (\emph{appicchatā}) \hyperlink{I.151}{I.151}{}; \hyperlink{II.12}{II.12f.}{}, \hyperlink{II.83}{83f.}{}; \hyperlink{XIV.155}{XIV.155}{}; \hyperlink{XVI.86}{XVI.86}{}
\par\textbf{fictitious-cause view} (\emph{visamahetu-diṭṭhi}) \hyperlink{XVII.22}{XVII.22}{}; \hyperlink{XXII.112}{XXII.112}{}
\par\textbf{field} (\emph{khetta}) \hyperlink{XIII.31}{XIII.31}{}
\par\textbf{final knowledge} (\emph{aññā}) \hyperlink{XVI.3}{XVI.3}{}; \hyperlink{XXIII.n17}{XXIII.n.17}{}
\par\textbf{fine-material becoming} (\emph{rūpa-bhava}) \hyperlink{IV.78}{IV.78}{}; \hyperlink{VII.9}{VII.9f.}{}; \hyperlink{VIII.241}{VIII.241}{}; f.-m. sphere (\emph{rūpāvacara}) \hyperlink{III.13}{III.13}{}, \hyperlink{III.23}{23}{}; \hyperlink{IV.74}{IV.74}{}, \hyperlink{IV.138}{138}{}; \hyperlink{X.2}{X.2f.}{}, \hyperlink{X.18}{18}{}; \hyperlink{XIII.5}{XIII.5}{}; \hyperlink{XIV.15}{XIV.15}{}, \hyperlink{XIV.86}{86}{}, \hyperlink{XIV.104}{104}{}, \hyperlink{XIV.112}{112}{}, \hyperlink{XIV.127}{127}{}, \hyperlink{XIV.157}{157}{}, \hyperlink{XIV.182}{182}{}, \hyperlink{XIV.n36}{n.36}{}; \hyperlink{XVI.92}{XVI.92}{}; \hyperlink{XX.31}{XX.31}{}
\par\textbf{fire} (\emph{tejo}) \hyperlink{XIII.32}{XIII.32f.}{}; \hyperlink{XIV.35}{XIV.35}{}; \hyperlink{XV.34}{XV.34}{}; \hyperlink{XVIII.6}{XVIII.6}{}; f. element (\emph{tejo-dhātu}) \hyperlink{V.30}{V.30}{}; \hyperlink{XI.28}{XI.28f.}{}, \hyperlink{XI.36}{36}{}, \hyperlink{XI.41}{41}{}, \hyperlink{XI.87}{87}{}; \hyperlink{XIV.35}{XIV.35}{}; \hyperlink{XV.30}{XV.30}{}; \hyperlink{XX.40}{XX.40}{}; f. \emph{kasiṇa }(\emph{tejo-kasiṇa}) \hyperlink{III.105}{III.105f.}{}; \hyperlink{V.5}{V.5}{}; \hyperlink{XIII.95}{XIII.95}{}
\par\textbf{Five-crest} (\emph{pañcasikha}) \hyperlink{XII.79}{XII.79}{}
\par\textbf{flavour} (\emph{rasa}) \hyperlink{XIV.47}{XIV.47}{}, \hyperlink{XIV.57}{57}{}; \hyperlink{XVII.156}{XVII.156}{}; \hyperlink{XVII.2}{XVII.2}{}, \hyperlink{XVII.11}{11}{}; f. base (\emph{rasāyatana}) \hyperlink{XV.3}{XV.3f.}{}; f. element (\emph{rasa-dhātu}) \hyperlink{XV.17}{XV.17}{}
\par\textbf{flood} (\emph{ogha}) \hyperlink{IV.87}{IV.87}{}; \hyperlink{VII.59}{VII.59}{}; \hyperlink{XIV.202}{XIV.202}{}, \hyperlink{XIV.226}{226f.}{}; \hyperlink{XXII.56}{XXII.56}{}, \hyperlink{XXII.70}{70}{}
\par\textbf{focus} (\emph{āpātha}) \hyperlink{I.57}{I.57}{}; \hyperlink{XIII.99}{XIII.99}{}; \hyperlink{XIV.116}{XIV.116}{}, \hyperlink{XIV.134}{134}{}; \hyperlink{XVII.127}{XVII.127}{}, \hyperlink{XVII.136}{136}{}, \hyperlink{XVII.235}{235}{}; \hyperlink{XXII.89}{XXII.89}{}
\par\textbf{food} (\emph{āhāra}), see nutriment; (\emph{bhojana}) \hyperlink{I.93}{I.93}{}; \hyperlink{IV.40}{IV.40}{}
\par\textbf{forest} (\emph{arañña}) \hyperlink{II.9}{II.9}{}, \hyperlink{II.48}{48}{}; \hyperlink{VIII.158}{VIII.158}{}; f. dweller (\emph{āraññika}) \hyperlink{II.2}{II.2}{}, \hyperlink{II.47}{47}{}
\par\textbf{formation} (\emph{saṅkhāra}) \hyperlink{IV.62}{IV.62}{}; \hyperlink{VII.7}{VII.7f.}{}, \hyperlink{VII.37}{37}{}, \hyperlink{VII.59}{59}{}; \hyperlink{VIII.175}{VIII.175}{}, \hyperlink{VIII.180}{180}{}, \hyperlink{VIII.236}{236}{}, \hyperlink{VIII.243}{243}{}; \hyperlink{X.44}{X.44f.}{}; \hyperlink{XIII.41}{XIII.41}{}; \hyperlink{XIV.131}{XIV.131f.}{}, \hyperlink{XIV.214}{214}{}, \hyperlink{XIV.n81}{n.81}{}; \hyperlink{XV.14}{XV.14}{}; \hyperlink{XVI.35}{XVI.35}{}, \hyperlink{XVI.52}{52}{}, \hyperlink{XVI.89}{89}{}; \hyperlink{XVII.2}{XVII.2}{}, \hyperlink{XVII.44}{44}{}, \hyperlink{XVII.51}{51}{}, \hyperlink{XVII.53}{53}{}, \hyperlink{XVII.60}{60f.}{}, \hyperlink{XVII.163}{163}{}, \hyperlink{XVII.173}{173}{}, \hyperlink{XVII.199}{199}{}, \hyperlink{XVII.251}{251}{}, \hyperlink{XVII.292}{292}{}; \hyperlink{XVIII.13}{XVIII.13}{}, \hyperlink{XVIII.19}{19}{}; \hyperlink{XIX.11}{XIX.11}{}, \hyperlink{XIX.13}{13}{}; \hyperlink{XX.6}{XX.6f.}{}, \hyperlink{XX.21}{21}{}, \hyperlink{XX.83}{83}{}, \hyperlink{XX.94}{94}{}; \hyperlink{XXI.21}{XXI.21}{}, \hyperlink{XXI.34}{34}{}, \hyperlink{XXI.57}{57}{}, \hyperlink{XXI.61}{61f.}{}, \hyperlink{XXI.129}{129}{}; \hyperlink{XXII.22}{XXII.22}{}; \hyperlink{XXIII.10}{XXIII.10}{}, \hyperlink{XXIII.13}{13}{}, \hyperlink{XXIII.22}{22}{}, \hyperlink{XXIII.30}{30}{}
\par\textbf{formed} (\emph{saṅkhata}) \hyperlink{VII.83}{VII.83}{}; \hyperlink{VIII.245}{VIII.245}{}; \hyperlink{XIV.216}{XIV.216}{}, \hyperlink{XIV.223}{223}{}; \hyperlink{XV.15}{XV.15}{}, \hyperlink{XV.25}{25}{}, \hyperlink{XV.40}{40}{}; \hyperlink{XVI.23}{XVI.23}{}, \hyperlink{XVI.102}{102}{}; \hyperlink{XVII.45}{XVII.45}{}; \hyperlink{XX.17}{XX.17}{}, \hyperlink{XX.101}{101}{}; \hyperlink{XXI.18}{XXI.18}{}
\par\textbf{foul, foulness} (\emph{asubha}) \hyperlink{I.103}{I.103}{}, \hyperlink{I.122}{122}{}; \hyperlink{III.57}{III.57f.}{}, \hyperlink{III.122}{122}{}, \hyperlink{III.n27}{n.27}{}; \hyperlink{VI}{VI passim}{}; \hyperlink{VIII.148}{VIII.148}{}; \hyperlink{XIV.224}{XIV.224}{}; \hyperlink{XXII.34}{XXII.34}{}, \hyperlink{XXII.53}{53}{}
\par\textbf{foundation of mindfulness} (\emph{satipaṭṭhāna}) \hyperlink{I.6}{I.6}{}, \hyperlink{I.51}{51}{}; \hyperlink{VIII.239}{VIII.239}{}, \hyperlink{VIII.n47}{n.47}{}; \hyperlink{XIV.141}{XIV.141}{}; \hyperlink{XVI.86}{XVI.86}{}; \hyperlink{XXII.33}{XXII.33}{}, \hyperlink{XXII.39}{39}{}, \hyperlink{XXII.42}{42}{}
\par\textbf{Four Divine Kings} , see Catumahārājā
\par\textbf{fraud} (\emph{sāṭheyya}) \hyperlink{I.151}{I.151}{}; \hyperlink{III.95}{III.95}{}; \hyperlink{VII.59}{VII.59}{}
\par\textbf{friend} (\emph{mitta}) \hyperlink{I.49}{I.49}{}; \hyperlink{III.61}{III.61f.}{}, \hyperlink{III.95}{95}{}
\par\textbf{fruit of cause} (\emph{phala}) \hyperlink{XIV.22}{XIV.22}{}; \hyperlink{XV.24}{XV.24}{}; \hyperlink{XVI.63}{XVI.63}{}, \hyperlink{XVI.85}{85}{}; \hyperlink{XVII.105}{XVII.105}{}, \hyperlink{XVII.168}{168}{}, \hyperlink{XVII.174}{174}{}, \hyperlink{XVII.250}{250}{}, \hyperlink{XVII.288}{288f.}{}, \hyperlink{XVII.291}{291}{}, \hyperlink{XVII.310}{310}{}; \hyperlink{XX.102}{XX.102}{}; \hyperlink{XXII.48}{XXII.48}{}
\par\textbf{fruit of asceticism} (\emph{sāmañña-phala}) \hyperlink{XVI.89}{XVI.89}{}; \hyperlink{XXIII.3}{XXIII.3}{}
\par\textbf{fruition} (\emph{phala}) \hyperlink{I.35}{I.35}{}, \hyperlink{I.37}{37}{}; \hyperlink{IV.78}{IV.78}{}; \hyperlink{VII.91}{VII.91}{}; \hyperlink{XIII.120}{XIII.120}{}; \hyperlink{XIV.105}{XIV.105}{}, \hyperlink{XIV.121}{121}{}; \hyperlink{XXI.125}{XXI.125f.}{}; \hyperlink{XXII.15}{XXII.15f.}{}; \hyperlink{XXIII.3}{XXIII.3}{}, \hyperlink{XXIII.28}{28}{}, \hyperlink{XXIII.49}{49}{}; f. attainment (\emph{phala-samāpatti}) \hyperlink{IV.78}{IV.78}{}; \hyperlink{XXIII.3}{XXIII.3}{}, \hyperlink{XXIII.5}{5}{}
\par\textbf{full awareness} (\emph{sampajañña}) \hyperlink{I.48}{I.48}{}; \hyperlink{III.95}{III.95}{}; \hyperlink{IV.172}{IV.172}{}; \hyperlink{VIII.43}{VIII.43}{}
\par\textbf{full understanding} (\emph{pariññā) }\hyperlink{XI.26}{XI.26}{}; \hyperlink{XX.2}{XX.2f.}{}, \hyperlink{XX.89}{89}{}; \hyperlink{XXII.92}{XXII.92}{}, \hyperlink{XXII.105}{105f.}{}, \hyperlink{XXII.128}{128}{}
\par\textbf{function} (\emph{kicca, rasa}) \hyperlink{I.21}{I.21}{}; \hyperlink{XXII.92}{XXII.92}{}
\par\textbf{functional} (\emph{kriyā}) \hyperlink{I.57}{I.57}{}; \hyperlink{X.14}{X.14}{}; \hyperlink{XIV.106}{XIV.106f.}{}, \hyperlink{XIV.127}{127}{}, \hyperlink{XIV.183}{183f.}{}; \hyperlink{XIX.8}{XIX.8}{}; \hyperlink{XX.31}{XX.31}{}, \hyperlink{XX.44}{44}{}
\par\textbf{gain} (\emph{lābha}) \hyperlink{I.31}{I.31}{}, \hyperlink{I.42}{42}{}, \hyperlink{I.61}{61}{}, \hyperlink{I.65}{65}{}
\par\textbf{Gangā} (Ganges in India, Mahaveli in Sri Lanka) \hyperlink{III.37}{III.37}{}, \hyperlink{III.53}{53}{}; \hyperlink{IV.24}{IV.24}{}; \hyperlink{XII.128}{XII.128}{}; \hyperlink{XIII.n19}{XIII.n.19}{}
\par\textbf{Garuḷa} (demon) \hyperlink{VII.43}{VII.43}{}; \hyperlink{XII.n19}{XII.n.19}{}
\par\textbf{General of the Dhamma} (\emph{dhamma-senāpati}). See also Sāriputta Thera 
\par\textbf{generation} (\emph{yoni}) \hyperlink{XIII.69}{XIII.69}{}; \hyperlink{XVIII.148}{XVIII.148}{}
\par\textbf{generosity} (\emph{cāga}) \hyperlink{VII.107}{VII.107f.}{}
\par\textbf{Ghosita} \hyperlink{XII.40}{XII.40f.}{}
\par\textbf{ghost} (\emph{peta}) \hyperlink{XIII.93}{XIII.93}{}; \hyperlink{XVII.154}{XVII.154}{}, \hyperlink{XVII.178}{178}{}
\par\textbf{gift, giving} (\emph{dāna}) \hyperlink{VII.94}{VII.94}{}, \hyperlink{VII.107}{107f.}{}; \hyperlink{IX.39}{IX.39}{}, \hyperlink{IX.124}{124}{}; \hyperlink{XIV.84}{XIV.84f.}{}, \hyperlink{XIV.206}{206}{}; \hyperlink{XVII.60}{XVII.60}{}, \hyperlink{XVII.81}{81}{}, \hyperlink{XVII.293}{293}{}
\par\textbf{Girikaṇḍaka-vihāra} \hyperlink{IV.96}{IV.96}{}
\par\textbf{giving up} (\emph{pariccāga}) \hyperlink{VIII.236}{VIII.236}{}; \hyperlink{XXI.18}{XXI.18}{}
\par\textbf{gladdening} (\emph{pāmujja}) \hyperlink{I.32}{I.32}{}, \hyperlink{I.140}{140}{}
\par\textbf{gladness} (\emph{muditā}) \hyperlink{III.105}{III.105f.}{}; \hyperlink{VII.18}{VII.18}{}; \hyperlink{IX.84}{IX.84f.}{}, \hyperlink{IX.95}{95}{}, \hyperlink{IX.100}{100}{}, \hyperlink{IX.119}{119}{}, \hyperlink{IX.122}{122}{}; \hyperlink{XIV.133}{XIV.133}{}, \hyperlink{XIV.154}{154f.}{}, \hyperlink{XIV.181}{181}{}
\par\textbf{Godatta Thera, Abhidhammika} \hyperlink{IV.76}{IV.76}{}
\par\textbf{goer} (\emph{gamaka}) \hyperlink{XVI.90}{XVI.90}{}
\par\textbf{going and coming back with the meditation subject} (\emph{gatapaccāgata}) \hyperlink{XIV.28}{XIV.28}{}
\par\textbf{gossip} (\emph{samphappalāpa}) \hyperlink{I.140}{I.140}{}; \hyperlink{XXII.66}{XXII.66}{}
\par\textbf{graspable entity} (\emph{viggaha}) \hyperlink{XV.n1}{XV.n.1}{}; \hyperlink{XVI.n6}{XVI.n.6}{}; \hyperlink{XX.n28}{XX.n.28}{}
\par\textbf{grasping} (\emph{ādāna}) \hyperlink{I.140}{I.140}{}; \hyperlink{VIII.233}{VIII.233}{}; \hyperlink{XVII.308}{XVII.308}{}; \hyperlink{XX.90}{XX.90}{}; \hyperlink{XXI.11}{XXI.11}{}; \hyperlink{XXII.113}{XXII.113}{}, \hyperlink{XXII.118}{118}{}; \hyperlink{XXIII.23}{XXIII.23}{}
\par\textbf{Great Monastery} (\emph{mahāvihāra}) \hyperlink{I.4}{I.4}{}; \hyperlink{III.53}{III.53}{}; \hyperlink{IX.65}{IX.65}{}
\par\textbf{great primary} (\emph{mahā-bhūta}), see primary Great Shrine (\emph{mahācetiya}) \hyperlink{IV.95}{IV.95}{}
\par\textbf{greed} (\emph{lobha}) \hyperlink{II.84}{II.84}{}; \hyperlink{III.95}{III.95}{}, \hyperlink{III.128}{128}{}; \hyperlink{IV.87}{IV.87}{}; \hyperlink{VII.59}{VII.59}{}; \hyperlink{XIII.64}{XIII.64}{}, \hyperlink{XIII.77}{77}{}; \hyperlink{XIV.90}{XIV.90}{}, \hyperlink{XIV.159}{159}{}, \hyperlink{XIV.161}{161}{}, \hyperlink{XIV.205}{205}{}; \hyperlink{XVII.129}{XVII.129}{}; \hyperlink{XXII.11}{XXII.11}{}, \hyperlink{XXII.49}{49}{}, \hyperlink{XXII.61}{61}{}; (\emph{rāga}) \hyperlink{I.90}{I.90}{}, \hyperlink{I.137}{137}{}, \hyperlink{I.140}{140}{}; \hyperlink{II.86}{II.86}{}; \hyperlink{III.74}{III.74f.}{}, \hyperlink{III.122}{122}{}; \hyperlink{IV.85}{IV.85}{}, \hyperlink{IV.192}{192}{}, \hyperlink{IV.n24}{n.24}{}; \hyperlink{VII.76}{VII.76}{}; \hyperlink{VIII.233}{VIII.233}{}, \hyperlink{VIII.247}{247}{}; \hyperlink{IX.97}{IX.97f.}{}, \hyperlink{IX.101}{101}{}; \hyperlink{XII.63}{XII.63}{}; \hyperlink{XVI.61}{XVI.61}{}, \hyperlink{XVI.69}{69}{}; \hyperlink{XVII.103}{XVII.103}{}, \hyperlink{XVII.138}{138}{}, \hyperlink{XVII.235}{235}{}; \hyperlink{XX.90}{XX.90}{}; \hyperlink{XXI.11}{XXI.11}{}, \hyperlink{XXI.123}{123}{}; \hyperlink{XXII.28}{XXII.28}{}, \hyperlink{XXII.48}{48}{}, \hyperlink{XXII.60}{60}{}, \hyperlink{XXII.113}{113}{}; \hyperlink{XXIII.23}{XXIII.23}{}
\par\textbf{grief} (\emph{domanassa}) \hyperlink{I.42}{I.42}{}; \hyperlink{IV.184}{IV.184}{}, \hyperlink{IV.n48}{n.48}{}; \hyperlink{IX.99}{IX.99}{}, \hyperlink{IX.111}{111}{}; \hyperlink{XIII.9}{XIII.9}{}; \hyperlink{XIV.90}{XIV.90}{}, \hyperlink{XIV.127}{127f.}{}; \hyperlink{XV.28}{XV.28}{}; \hyperlink{XVI.31}{XVI.31}{}, \hyperlink{XVI.50}{50f.}{}; \hyperlink{XVII.48}{XVII.48}{}; \hyperlink{XX.71}{XX.71}{}; g. element (\emph{domanassadhātu}) \hyperlink{XV.28}{XV.28}{}; g. faculty (\emph{domanassa-indriya}) \hyperlink{XIII.9}{XIII.9}{}; \hyperlink{XVI.1}{XVI.1}{}
\par\textbf{gross} (\emph{oḷārika}) \hyperlink{VIII.176}{VIII.176}{}; \hyperlink{XIV.72}{XIV.72}{}, \hyperlink{XIV.192}{192}{}, \hyperlink{XIV.198}{198}{}, \hyperlink{XIV.203}{203}{}
\par\textbf{group} (\emph{kalāpa}) \hyperlink{VIII.224}{VIII.224}{}, \hyperlink{VIII.n45}{n.45}{}; \hyperlink{XI.88}{XI.88}{}; \hyperlink{XIV.n27}{XIV.n.27}{}, \hyperlink{XIV.n32}{n.32}{}; \hyperlink{XVII.156}{XVII.156}{}; \hyperlink{XVII.n5}{XVII.n.5}{}; \hyperlink{XIX}{XIX passim}{}; \hyperlink{XX.76}{XX.76f.}{}
\par\textbf{growth} (\emph{upacaya}) \hyperlink{XIV.66}{XIV.66}{}; \hyperlink{XVII.74}{XVII.74}{}
\par\textbf{Gutta Thera, Mahā Rohaṇa} \hyperlink{IV.135}{IV.135}{}; \hyperlink{XII.9}{XII.9}{}
\par\textbf{habit} (\emph{sīla}) \hyperlink{I.38}{I.38}{}
\par\textbf{Haṃsapātana, Lake} \hyperlink{XIII.38}{XIII.38}{}
\par\textbf{“hand-grasping question”} (\emph{hatthagahaṇa-pañhā}) \hyperlink{VIII.142}{VIII.142}{}
\par\textbf{Haṅkana} \hyperlink{XX.110}{XX.110}{}
\par\textbf{happiness} (\emph{pīti}) \hyperlink{I.32}{I.32}{}, \hyperlink{I.140}{140}{}; \hyperlink{III.5}{III.5}{}, \hyperlink{III.8}{8}{}, \hyperlink{III.12}{12}{}, \hyperlink{III.21}{21}{}, \hyperlink{III.n6}{n.6}{}; \hyperlink{IV.51}{IV.51}{}, \hyperlink{IV.74}{74}{}, \hyperlink{IV.86}{86}{}, \hyperlink{IV.94}{94}{}, \hyperlink{IV.182}{182}{}; \hyperlink{VIII.226}{VIII.226}{}, \hyperlink{VIII.230}{230}{}; \hyperlink{IX.112}{IX.112}{}; \hyperlink{XIV.86}{XIV.86}{}, \hyperlink{XIV.128}{128}{}, \hyperlink{XIV.133}{133}{}, \hyperlink{XIV.136}{136}{}, \hyperlink{XIV.156}{156}{}, \hyperlink{XIV.159}{159}{}, \hyperlink{XIV.169}{169}{}, \hyperlink{XIV.180}{180}{}; \hyperlink{XVI.86}{XVI.86}{}; \hyperlink{XVII.160}{XVII.160}{}; \hyperlink{XX.115}{XX.115}{}; \hyperlink{XXI.19}{XXI.19}{}; \hyperlink{XXII.42}{XXII.42}{}; \hyperlink{XXIII.26}{XXIII.26}{}
\par\textbf{happy destiny} (\emph{sugati}) \hyperlink{XVII.135}{XVII.135f.}{}, \hyperlink{XVII.160}{160}{}
\par\textbf{harsh speech} (\emph{pharusa-vācā}) \hyperlink{I.140}{I.140}{}; \hyperlink{XXII.66}{XXII.66}{}
\par\textbf{hate} (\emph{dosa}) \hyperlink{I.90}{I.90}{}; \hyperlink{II.86}{II.86}{}; \hyperlink{III.74}{III.74f.}{}, \hyperlink{III.95}{95}{}, \hyperlink{III.128}{128}{}; \hyperlink{IV.87}{IV.87}{}, \hyperlink{IV.192}{192}{}; \hyperlink{VII.59}{VII.59}{}; \hyperlink{IX.2}{IX.2}{}, \hyperlink{IX.15}{15}{}; \hyperlink{XII.63}{XII.63}{}; \hyperlink{XIII.64}{XIII.64}{}, \hyperlink{XIII.77}{77}{}; \hyperlink{XIV.89}{XIV.89}{}, \hyperlink{XIV.92}{92}{}, \hyperlink{XIV.170}{170f.}{}, \hyperlink{XIV.205}{205}{}; \hyperlink{XVI.69}{XVI.69}{}; \hyperlink{XXII.11}{XXII.11}{}, \hyperlink{XXII.49}{49}{}, \hyperlink{XXII.61}{61}{}
\par\textbf{Hatthikucchipabbhāra} \hyperlink{III.102}{III.102}{}; \hyperlink{IV.10}{IV.10}{}
\par\textbf{head hair} (\emph{kesa}) \hyperlink{VII.28}{VII.28}{}; \hyperlink{VIII.83}{VIII.83}{}; \hyperlink{XI.48}{XI.48}{}
\par\textbf{hearing} (\emph{savana}) \hyperlink{XIV.117}{XIV.117}{}; \hyperlink{XVII.127}{XVII.127}{}
\par\textbf{heart} (\emph{hadaya}) \hyperlink{VIII.111}{VIII.111}{}; \hyperlink{XI.59}{XI.59}{}; \hyperlink{XIII.9}{XIII.9}{}; \hyperlink{XIV.78}{XIV.78}{}; \hyperlink{XV.3}{XV.3}{}; \hyperlink{XVIII.4}{XVIII.4}{}; h. basis (\emph{hadayavatthu}) \hyperlink{XIII.99}{XIII.99}{}, \hyperlink{XIII.n5}{n.5}{}; \hyperlink{XIV.60}{XIV.60}{}, \hyperlink{XIV.78}{78}{}, \hyperlink{XIV.97}{97}{}, \hyperlink{XIV.108}{108}{}, \hyperlink{XIV.128}{128}{}, \hyperlink{XIV.n13}{n.13}{}, \hyperlink{XIV.n26}{n.26}{}, \hyperlink{XIV.n47}{n.47}{}; \hyperlink{XVII.128}{XVII.128}{}, \hyperlink{XVII.163}{163}{}, \hyperlink{XVII.209}{209}{}, \hyperlink{XVII.n36}{n.36}{}; \hyperlink{XVIII.7}{XVIII.7}{}; \hyperlink{XX.70}{XX.70}{}; \hyperlink{XXIII.n12}{XXIII.n.12}{}
\par\textbf{heaven} (\emph{sagga}) \hyperlink{VII.17}{VII.17}{}, \hyperlink{VII.n14}{n.14}{}; \hyperlink{XIII.83}{XIII.83}{}, \hyperlink{XIII.94}{94}{}
\par\textbf{heavenly abiding} (\emph{dibba-vihāra}) \hyperlink{VII.63}{VII.63}{}, \hyperlink{VII.n28}{n.28}{}
\par\textbf{hell} \emph{ }(\emph{niraya}) \hyperlink{I.156}{I.156}{}; \hyperlink{VII.n14}{VII.n.14}{}; \hyperlink{IX.20}{IX.20}{}, \hyperlink{IX.n3}{n.3}{}; \hyperlink{XIII.33}{XIII.33}{}, \hyperlink{XIII.79}{79}{}, \hyperlink{XIII.92}{92}{}; \hyperlink{XIV.193}{XIV.193}{}, \hyperlink{XIV.207}{207}{}; \hyperlink{XVII.137}{XVII.137}{}, \hyperlink{XVII.153}{153}{}, \hyperlink{XVII.178}{178}{}
\par\textbf{higher consciousness} (\emph{adhicitta}) \hyperlink{I.10}{I.10}{}; \hyperlink{VIII.73}{VIII.73f.}{}, \hyperlink{VIII.173}{173}{}
\par\textbf{higher than human state} (\emph{uttari manussadhamma}) \hyperlink{I.69}{I.69}{}
\par\textbf{higher understanding} (\emph{adhipaññā}) \hyperlink{I.10}{I.10}{}; \hyperlink{VIII.173}{VIII.173}{}
\par\textbf{higher virtue} (\emph{adhisīla}) \hyperlink{I.10}{I.10}{}; \hyperlink{VIII.173}{VIII.173}{}
\par\textbf{Himalaya} (\emph{himavant}) \hyperlink{VII.42}{VII.42}{}, \hyperlink{VII.n16}{n.16}{}; \hyperlink{XIII.38}{XIII.38}{}, \hyperlink{XIII.48}{48}{}; \hyperlink{XXI.43}{XXI.43}{}
\par\textbf{hindrance} (\emph{nivaraṇa}) \hyperlink{I.140}{I.140}{}, \hyperlink{III.15}{III.15}{}; \hyperlink{IV.31}{IV.31f.}{}, \hyperlink{IV.86}{86}{}, \hyperlink{IV.104}{104}{}; \hyperlink{VI.67}{VI.67}{}; \hyperlink{VII.59}{VII.59}{}; \hyperlink{VIII.233}{VIII.233}{}; \hyperlink{IX.102}{IX.102}{}; \hyperlink{XIV.202}{XIV.202}{}; \hyperlink{XXIII.57}{XXIII.57}{}, \hyperlink{XXIII.71}{71}{}
\par\textbf{house-to-house seeker} (\emph{sapadāna-cārika}) \hyperlink{II.2}{II.2}{}, \hyperlink{II.31}{31}{}
\par\textbf{human} (\emph{manussa}) \hyperlink{XIV.111}{XIV.111}{}; \hyperlink{XVII.154}{XVII.154}{}
\par\textbf{humour} (\emph{dhātu}) \hyperlink{VIII.159}{VIII.159}{}; (\emph{dosa}) \hyperlink{III.80}{III.80}{}
\par\textbf{hypocrisy} (\emph{vimhāpana}) \hyperlink{I.67}{I.67}{}
\par\textbf{idea} (\emph{anubodha}) \hyperlink{XVI.84}{XVI.84}{}
\par\textbf{identity} (\emph{ekatta}) \hyperlink{XVII.167}{XVII.167}{}, \hyperlink{XVII.309}{309}{}; \hyperlink{XX.102}{XX.102}{}; \hyperlink{XXI.52}{XXI.52}{}
\par\textbf{idleness} (\emph{kosajja}) \hyperlink{IV.47}{IV.47}{}, \hyperlink{IV.72}{72}{}; \hyperlink{VIII.74}{VIII.74}{}; \hyperlink{XII.17}{XII.17}{}
\par\textbf{ignorance} (\emph{avijjā}) \hyperlink{I.140}{I.140}{}; \hyperlink{III.17}{III.17}{}; \hyperlink{IV.87}{IV.87}{}; \hyperlink{VII.7}{VII.7f.}{}, \hyperlink{VII.59}{59}{}; \hyperlink{XII.17}{XII.17}{}; \hyperlink{XIV.229}{XIV.229}{}; \hyperlink{XV.28}{XV.28}{}; \hyperlink{XVII.2}{XVII.2}{}, \hyperlink{XVII.36}{36}{}, \hyperlink{XVII.43}{43}{}, \hyperlink{XVII.48}{48}{}, \hyperlink{XVII.51}{51f.}{}, \hyperlink{XVII.58}{58f.}{}, \hyperlink{XVII.103}{103}{}, \hyperlink{XVII.163}{163}{}, \hyperlink{XVII.274}{274}{}, \hyperlink{XVII.292}{292}{}; \hyperlink{XX.8}{XX.8}{}, \hyperlink{XX.97}{97}{}; \hyperlink{XXII.28}{XXII.28}{}, \hyperlink{XXII.45}{45}{}, \hyperlink{XXII.48}{48}{}, \hyperlink{XXII.56}{56}{}, \hyperlink{XXII.60}{60}{}
\par\textbf{illumination} (\emph{obhāsa}) \hyperlink{XII.17}{XII.17}{}; \hyperlink{XX.107}{XX.107}{}
\par\textbf{ill-will} (\emph{vyāpāda}) \hyperlink{I.140}{I.140}{}; \hyperlink{III.122}{III.122}{}; \hyperlink{IV.86}{IV.86}{}, \hyperlink{IV.104}{104}{}; \hyperlink{IX.93}{IX.93}{}, \hyperlink{IX.98}{98}{}; \hyperlink{XIV.227}{XIV.227}{}; \hyperlink{XV.28}{XV.28}{}; \hyperlink{XVI.10}{XVI.10}{}; \hyperlink{XXII.25}{XXII.25}{}, \hyperlink{XXII.54}{54}{}, \hyperlink{XXII.62}{62}{}
\par\textbf{Illīsa} \hyperlink{XII.127}{XII.127}{}
\par\textbf{immaterial} (\emph{arūpa}) \hyperlink{VIII.180}{VIII.180}{}; \hyperlink{XIV.n36}{XIV.n.36}{}; XVI\hyperlink{II.8}{II.8}{}, \hyperlink{II.15}{15}{}; \hyperlink{XX.43}{XX.43}{}; \hyperlink{XXI.86}{XXI.86}{}; i. becoming (\emph{arūpabhava}) \hyperlink{III.118}{III.118}{}; \hyperlink{IV.78}{IV.78}{}; \hyperlink{VII.9}{VII.9f.}{}; \hyperlink{XVII.150}{XVII.150}{}; \hyperlink{XXI.34}{XXI.34}{}; i. septad (\emph{arūpa-sattaka}) \hyperlink{XX.76}{XX.76}{}, \hyperlink{XX.126}{126}{}; i. sphere (\emph{arūpāvacara}) \hyperlink{III.13}{III.13}{}, \hyperlink{III.23}{23}{}; \hyperlink{X.10}{X.10}{}; \hyperlink{XIV.15}{XIV.15}{}, \hyperlink{XIV.87}{87}{}, \hyperlink{XIV.104}{104}{}, \hyperlink{XIV.109}{109}{}, \hyperlink{XIV.182}{182}{}, \hyperlink{XIV.206}{206}{}; \hyperlink{XVI.92}{XVI.92}{}; \hyperlink{XX.31}{XX.31}{}; i. state (\emph{āruppa}) \hyperlink{III.105}{III.105}{}; \hyperlink{V.n5}{V.n.5}{}; \hyperlink{VII.28}{VII.28}{}; \hyperlink{XI.104}{XI.104}{}; \hyperlink{X}{X passim}{}; \hyperlink{XIV.87}{XIV.87}{}; i. world (\emph{arūpa-loka}) \hyperlink{XVI.85}{XVI.85}{}; \hyperlink{XVII.134}{XVII.134}{}; \hyperlink{XXIII.n12}{XXIII.n.12}{}
\par\textbf{impact} (\emph{abhighāta}) \hyperlink{XIV.37}{XIV.37}{}
\par\textbf{impediment} (\emph{palibodha}) \hyperlink{III.29}{III.29f.}{}; \hyperlink{IV.20}{IV.20}{}; \hyperlink{XVI.23}{XVI.23}{}; \hyperlink{XXII.97}{XXII.97}{}
\par\textbf{imperfection} (\emph{upakkilesa}) \hyperlink{VIII.224}{VIII.224}{}; \hyperlink{XIII.2}{XIII.2}{}; \hyperlink{XX.105}{XX.105f.}{}; \hyperlink{XXI.1}{XXI.1}{}
\par\textbf{impermanent, impermanence} (\emph{anicca}) \hyperlink{I.140}{I.140}{}; \hyperlink{VIII.234}{VIII.234}{}, \hyperlink{VIII.n64}{n.64}{}; \hyperlink{XI.104}{XI.104}{}, \hyperlink{XI.n18}{n.18}{}; \hyperlink{XII.26}{XII.26}{}; \hyperlink{XIV.3}{XIV.3}{}, \hyperlink{XIV.69}{69}{}, \hyperlink{XIV.224}{224}{}, \hyperlink{XIV.229}{229}{}; \hyperlink{XVI.89}{XVI.89}{}, \hyperlink{XVI.99}{99}{}; \hyperlink{XIX.26}{XIX.26}{}; \hyperlink{XX}{XX passim}{}, \hyperlink{XX.47}{47f.}{}, \hyperlink{XX.85}{85}{}, \hyperlink{XX.103}{103}{}, \hyperlink{XX.126}{126}{}; \hyperlink{XXI.3}{XXI.3f.}{}, \hyperlink{XXI.10}{10f.}{}, \hyperlink{XXI.33}{33}{}, \hyperlink{XXI.48}{48}{}, \hyperlink{XXI.51}{51}{}, \hyperlink{XXI.59}{59}{}, \hyperlink{XXI.68}{68}{}, \hyperlink{XXI.88}{88}{}, \hyperlink{XXI.129}{129}{}, \hyperlink{XXI.n3}{n.3}{}; \hyperlink{XXII.22}{XXII.22}{}, \hyperlink{XXII.53}{53}{}; \hyperlink{XXIII.32}{XXIII.32}{}
\par\textbf{imperturbable} (\emph{āneñja}) \hyperlink{X.19}{X.19}{}; \hyperlink{XII.16}{XII.16}{}, \hyperlink{XII.n1}{n.1}{}; \hyperlink{XVII.44}{XVII.44}{}, \hyperlink{XVII.60}{60}{}, \hyperlink{XVII.252}{252}{}
\par\textbf{impinging} (\emph{ghaṭṭana, saṅghaṭṭana}) \hyperlink{I.53}{I.53}{}; \hyperlink{XIV.73}{XIV.73}{}, \hyperlink{XIV.115}{115}{}, \hyperlink{XIV.134}{134}{}; \hyperlink{XVII.308}{XVII.308}{}
\par\textbf{impulsion} (\emph{javana}) \hyperlink{I.57}{I.57}{}, \hyperlink{I.n16}{n.16}{}; \hyperlink{IV.33}{IV.33}{}, \hyperlink{IV.74}{74f.}{}, \hyperlink{IV.132}{132}{}, \hyperlink{IV.138}{138}{}, \hyperlink{IV.n13}{n.13}{}, \hyperlink{IV.n21}{n.21}{}, \hyperlink{IV.n22}{n.22}{}; \hyperlink{X.10}{X.10}{}; \hyperlink{XIII.5}{XIII.5}{}, \hyperlink{XIII.117}{117}{}; \hyperlink{XIV.98}{XIV.98}{}, \hyperlink{XIV.121}{121}{}, \hyperlink{XIV.152}{152}{}, \hyperlink{XIV.188}{188}{}, \hyperlink{XIV.197}{197}{}, \hyperlink{XIV.n27}{n.27}{}; \hyperlink{XV.36}{XV.36f.}{}; \hyperlink{XVII.87}{XVII.87}{}, \hyperlink{XVII.103}{103}{}, \hyperlink{XVII.129}{129}{}, \hyperlink{XVII.136}{136f.}{}, \hyperlink{XVII.293}{293}{}; \hyperlink{XIX.14}{XIX.14}{}; \hyperlink{XX.44}{XX.44}{}; \hyperlink{XXI.129}{XXI.129}{}; \hyperlink{XXII.16}{XXII.16}{}, \hyperlink{XXII.19}{19}{}
\par\textbf{incalculable} (\emph{asaṅkheyya}) \hyperlink{XIII.16}{XIII.16}{}, \hyperlink{XIII.29}{29}{}, \hyperlink{XIII.55}{55}{}
\par\textbf{indeterminate} (\emph{avyākata}) \hyperlink{II.78}{II.78}{}; \hyperlink{XIV.94}{XIV.94}{}, \hyperlink{XIV.126}{126}{}, \hyperlink{XIV.129}{129}{}, \hyperlink{XIV.132}{132}{}, \hyperlink{XIV.179}{179f.}{}, \hyperlink{XIV.198}{198}{}, \hyperlink{XIV.209}{209}{}; \hyperlink{XVII.51}{XVII.51}{}
\par\textbf{individual essence} (\emph{sabhāva}) \hyperlink{I.n14}{I.n.14}{}; \hyperlink{III.115}{III.115f.}{}; \hyperlink{IV.45}{IV.45}{}, \hyperlink{IV.53}{53}{}; \hyperlink{V.n5}{V.n.5}{}; \hyperlink{VI.35}{VI.35}{}, \hyperlink{VI.84}{84}{}; \hyperlink{VII.n1}{VII.n.1}{}; \hyperlink{VIII.40}{VIII.40}{}, \hyperlink{VIII.246}{246}{}, \hyperlink{VIII.n12}{n.12}{}, \hyperlink{VIII.n16}{n.16}{}, \hyperlink{VIII.n68}{n.68}{}, \hyperlink{VIII.n70}{n.70}{}; \hyperlink{IX.123}{IX.123}{}; \hyperlink{X.20}{X.20}{}, \hyperlink{X.n7}{n.7}{}; \hyperlink{XI.25}{XI.25}{}, \hyperlink{XI.27}{27}{}, \hyperlink{XI.42}{42}{}, \hyperlink{XI.n18}{n.18}{}; \hyperlink{XIV.7}{XIV.7f.}{}, \hyperlink{XIV.68}{68}{}, \hyperlink{XIV.73}{73}{}, \hyperlink{XIV.82}{82}{}, \hyperlink{XIV.91}{91}{}, \hyperlink{XIV.126}{126f.}{}, \hyperlink{XIV.129}{129}{}, \hyperlink{XIV.143}{143}{}, \hyperlink{XIV.163}{163}{}, \hyperlink{XIV.198}{198}{}, \hyperlink{XIV.200}{200}{}, \hyperlink{XIV.203}{203f.}{}, \hyperlink{XIV.n4}{n.4}{}; \hyperlink{XV.14}{XV.14}{}, \hyperlink{XV.15}{15}{}, \hyperlink{XV.20}{20f.}{}, \hyperlink{XV.26}{26}{}, \hyperlink{XV.32}{32}{}; \hyperlink{XVI.35}{XVI.35}{}, \hyperlink{XVI.85}{85}{}, \hyperlink{XVI.n23}{n.23}{}; \hyperlink{XVII.68}{XVII.68}{}, \hyperlink{XVII.110}{110}{}, \hyperlink{XVII.312}{312}{}, \hyperlink{XVII.n6}{n.6}{}; \hyperlink{XX.103}{XX.103}{}; \hyperlink{XXI.n4}{XXI.n.4}{}, \hyperlink{XXI.n33}{n.33}{}; \hyperlink{XXII.58}{XXII.58}{}, \hyperlink{XXII.92}{92}{}, \hyperlink{XXII.100}{100}{}; \hyperlink{XXIII.52}{XXIII.52}{}, \hyperlink{XXIII.n18}{n.18}{}
\par\textbf{individuality} (\emph{sakkāya}) \hyperlink{IV.155}{IV.155}{}, \hyperlink{IV.192}{192}{}; \hyperlink{XVI.85}{XVI.85}{}; \hyperlink{XVII.243}{XVII.243}{}; \hyperlink{XXII.48}{XXII.48}{}, \hyperlink{XXII.112}{112}{}; \hyperlink{XXIII.2}{XXIII.2}{}
\par\textbf{Indra} \hyperlink{XII.137}{XII.137}{}; \hyperlink{XX.114}{XX.114}{}, \hyperlink{XX.121}{121}{}
\par\textbf{inductive insight} (\emph{naya-vipassanā)}\hyperlink{XX.2}{XX.2}{}, \hyperlink{XX.21}{21}{}
\par\textbf{inference} (\emph{anumāna}) \hyperlink{XII.n7}{XII.n.7}{}; \hyperlink{XVII.202}{XVII.202}{}; \hyperlink{XX.n13}{XX.n.13}{}
\par\textbf{inferential knowledge} (\emph{anvaya-ñāṇa}) \hyperlink{XXI.17}{XXI.17}{}; \hyperlink{XXII.115}{XXII.115}{}
\par\textbf{inferior} (\emph{hīna}) \hyperlink{I.33}{I.33}{}; \hyperlink{XIV.193}{XIV.193}{}, \hyperlink{XIV.208}{208}{}; \hyperlink{XV.29}{XV.29}{}; \hyperlink{XVII.53}{XVII.53}{}, \hyperlink{XVII.138}{138}{}
\par\textbf{inferiority} (\emph{hīnatā}) \hyperlink{XVII.271}{XVII.271}{}
\par\textbf{inherence} (\emph{samavāya}) \hyperlink{XVI.91}{XVI.91}{}
\par\textbf{inherent tendency} (\emph{anusaya}) \hyperlink{I.13}{I.13}{}; \hyperlink{VII.59}{VII.59}{}; \hyperlink{IX.88}{IX.88}{}; \hyperlink{XVI.64}{XVI.64}{}, \hyperlink{XVI.n18}{n.18}{}; \hyperlink{XVII.238}{XVII.238}{}; \hyperlink{XX.n33}{XX.n.33}{}; \hyperlink{XXII.45}{XXII.45}{}, \hyperlink{XXII.60}{60}{}, \hyperlink{XXII.73}{73}{}, \hyperlink{XXII.83}{83f.}{}
\par\textbf{initiative, element of} (\emph{ārambha-dhātu}) \hyperlink{IV.52}{IV.52}{}, \hyperlink{IV.124}{124}{}; \hyperlink{XV.25}{XV.25}{}, \hyperlink{XV.28}{28}{}
\par\textbf{inquiry} (\emph{vīmaṃsā}) \hyperlink{I.33}{I.33}{}; \hyperlink{II.24}{II.24}{}; \hyperlink{XII.12}{XII.12}{}, \hyperlink{XII.50}{50}{}; \hyperlink{XVI.86}{XVI.86}{}; \hyperlink{XVII.72}{XVII.72}{}; \hyperlink{XXII.36}{XXII.36}{}
\par\textbf{insight} (\emph{vipassanā}) \hyperlink{I.6}{I.6}{}, \hyperlink{I.7}{7}{}, \hyperlink{I.n3}{n.3}{}; \hyperlink{III.56}{III.56}{}, \hyperlink{III.n7}{n.7}{}; \hyperlink{VII.30}{VII.30}{}; \hyperlink{VIII.43}{VIII.43}{}, \hyperlink{VIII.180}{180}{}, \hyperlink{VIII.222}{222}{}, \hyperlink{VIII.233}{233}{}, \hyperlink{VIII.236}{236f.}{}; \hyperlink{IX.97}{IX.97}{}, \hyperlink{IX.104}{104}{}; \hyperlink{X.52}{X.52}{}; \hyperlink{XI.121}{XI.121}{}; \hyperlink{XVIII.5}{XVIII.5}{}, \hyperlink{XVIII.8}{8}{}, \hyperlink{XVIII.n2}{n.2}{}; \hyperlink{XX.81}{XX.81}{}, \hyperlink{XX.83}{83}{}, \hyperlink{XX.91}{91}{}, \hyperlink{XX.105}{105f.}{}, \hyperlink{XX.n33}{n.33}{}; \hyperlink{XXI.1}{XXI.1f.}{}, \hyperlink{XXI.73}{73}{}, \hyperlink{XXI.130}{130}{}; \hyperlink{XXII.1}{XXII.1}{}, \hyperlink{XXII.25}{25}{}, \hyperlink{XXII.46}{46}{}, \hyperlink{XXII.89}{89}{}, \hyperlink{XXII.113}{113}{}, \hyperlink{XXII.118}{118}{}; \hyperlink{XXIII.7}{XXIII.7}{}, \hyperlink{XXIII.20}{20}{}, \hyperlink{XXIII.31}{31}{}, \hyperlink{XXIII.33}{33}{}; i., eighteen principal (\emph{aṭṭhārasa mahā-vipassanā}) \hyperlink{I.n38}{I.n.38}{}; \hyperlink{XX.89}{XX.89f.}{}; \hyperlink{XXII.113}{XXII.113}{}; i. into states that is the higher understanding (\emph{adhipaññā-vipassanā}) \hyperlink{I.140}{I.140}{}; \hyperlink{XX.90}{XX.90}{}; \hyperlink{XXI.11}{XXI.11}{}, \hyperlink{XXI.25}{25}{}; \hyperlink{XXII.113}{XXII.113}{}, \hyperlink{XXII.118}{118}{}; i. knowledge (\emph{vipassanāñāṇa}) \hyperlink{VII.30}{VII.30}{}; \hyperlink{VIII.43}{VIII.43}{}, \hyperlink{VIII.60}{60}{}; \hyperlink{XXI.11}{XXI.11}{}; i. leading to emergence (\emph{vuṭṭhānagāmini-vipassanā}) \hyperlink{XXI.83}{XXI.83f.}{}
\par\textbf{insistence} (\emph{abhinivesa}) \hyperlink{I.140}{I.140}{}. See also interpreting 
\par\textbf{intention} (\emph{adhippāya}) \hyperlink{XIV.61}{XIV.61}{}, \hyperlink{XIV.n27}{n.27}{}
\par\textbf{interestedness} (\emph{vyāpāra}) \hyperlink{XVIII.31}{XVIII.31}{}. See also uninterest 
\par\textbf{internal} (\emph{ajjhatta}) \hyperlink{IV.141}{IV.141}{}; \hyperlink{XI.32}{XI.32f.}{}; \hyperlink{XIII.105}{XIII.105f.}{}; \hyperlink{XIV.10}{XIV.10}{}, \hyperlink{XIV.19}{19}{}, \hyperlink{XIV.73}{73}{}, \hyperlink{XIV.192}{192}{}, \hyperlink{XIV.198}{198}{}, \hyperlink{XIV.224}{224}{}, \hyperlink{XIV.n75}{n.75}{}; \hyperlink{XXI.83}{XXI.83f.}{}
\par\textbf{interpreting} (\emph{abhinivesa}) \hyperlink{I.140}{I.140}{}, \hyperlink{I.n14}{n.14}{}; \hyperlink{XIII.10}{XIII.10}{}; \hyperlink{XIV.8}{XIV.8}{}, \hyperlink{XIV.19}{19}{}, \hyperlink{XIV.130}{130}{}, \hyperlink{XIV.164}{164}{}, \hyperlink{XIV.228}{228}{}, \hyperlink{XIV.n6}{n.6}{}; \hyperlink{XVII.244}{XVII.244}{}; \hyperlink{XX.90}{XX.90}{}; \hyperlink{XXI.73}{XXI.73}{}, \hyperlink{XXI.83}{83f.}{}; \hyperlink{XXII.54}{XXII.54}{}, \hyperlink{XXII.113}{113}{}, \hyperlink{XXII.118}{118}{}, \hyperlink{XXII.120}{120}{}. See also misinterpreting \& insistence 
\par\textbf{intervention} (\emph{vipphāra}) \hyperlink{IV.89}{IV.89}{}; \hyperlink{XII.27}{XII.27}{}; \hyperlink{XIV.132}{XIV.132}{}, \hyperlink{XIV.n58}{n.58}{}
\par\textbf{intimation} (\emph{viññatti}) \hyperlink{I.n16}{I.n.16}{}; \hyperlink{XI.107}{XI.107}{}; \hyperlink{XIV.61}{XIV.61f.}{}, \hyperlink{XIV.79}{79}{}, \hyperlink{XIV.n27}{n.27}{}, \hyperlink{XIV.n33}{n.33}{}; \hyperlink{XVII.61}{XVII.61}{}; \hyperlink{XX.31}{XX.31}{}
\par\textbf{investigation} (\emph{santīraṇa}) \hyperlink{I.57}{I.57}{}; \hyperlink{XIV.97}{XIV.97f.}{}, \hyperlink{XIV.119}{119}{}; \hyperlink{XV.36}{XV.36}{}; \hyperlink{XVII.129}{XVII.129}{}; \hyperlink{XX.44}{XX.44}{}
\par\textbf{investigation-of-states} (\emph{dhammavicaya}) \hyperlink{IV.51}{IV.51}{}; \hyperlink{XVI.86}{XVI.86}{}
\par\textbf{Īsadharapabbata} \hyperlink{VII.42}{VII.42}{}
\par\textbf{I-shall-come-to-know-the-unknown faculty} (\emph{anaññātaññassāmītindriya}) \hyperlink{XVI.1}{XVI.1f.}{}
\par\textbf{Jambudīpa (India)} \hyperlink{V.2}{V.2}{}; \hyperlink{VII.23}{VII.23}{}, \hyperlink{VII.42}{42}{}, \hyperlink{VII.44}{44}{}; \hyperlink{XII.75}{XII.75}{}; \hyperlink{XX.n1}{XX.n.1}{}
\par\textbf{Jambu} (Rose-apple) River \hyperlink{I.n37}{I.n.37}{}
\par\textbf{Jambu Tree} \hyperlink{VII.42}{VII.42}{}
\par\textbf{Jaṭilaka} \hyperlink{VIII.18}{VIII.18}{}; \hyperlink{XII.40}{XII.40}{}
\par\textbf{jhāna} (\emph{jhāna}) \hyperlink{I.6}{I.6}{}, \hyperlink{I.140}{140}{}; \hyperlink{III.5}{III.5}{}, \hyperlink{III.8}{8}{}, \hyperlink{III.11}{11}{}, \hyperlink{III.21}{21}{}, \hyperlink{III.107}{107}{}; \hyperlink{IV.78}{IV.78}{}, \hyperlink{IV.79}{79f.}{}; \hyperlink{VIII.141}{VIII.141f.}{}, \hyperlink{VIII.179}{179}{}, \hyperlink{VIII.227}{227}{}, \hyperlink{VIII.233}{233}{}, \hyperlink{VIII.241}{241}{}; \hyperlink{IX.90}{IX.90}{}; \hyperlink{X}{X passim}{}; \hyperlink{XII.2}{XII.2f.}{}, \hyperlink{XII.130}{130}{}; \hyperlink{XIII.5}{XIII.5f.}{}, \hyperlink{XIII.35}{35}{}; \hyperlink{XIV.12}{XIV.12}{}, \hyperlink{XIV.86}{86f.}{}, \hyperlink{XIV.127}{127}{}, \hyperlink{XIV.158}{158}{}, \hyperlink{XIV.206}{206}{}; \hyperlink{XVII.66}{XVII.66}{}, \hyperlink{XVII.92}{92}{}, \hyperlink{XVII.143}{143}{}; \hyperlink{XVIII.3}{XVIII.3}{}; \hyperlink{XX.9}{XX.9}{}; \hyperlink{XXI.75}{XXI.75}{}, \hyperlink{XXI.111}{111}{}, \hyperlink{XXI.116}{116}{}; \hyperlink{XXIII.11}{XXIII.11}{}, \hyperlink{XXIII.21}{21}{}, \hyperlink{XXIII.26}{26}{}, \hyperlink{XXIII.29}{29}{}; j. factor (\emph{jhānaṅga}) \hyperlink{III.5}{III.5}{}, \hyperlink{III.21}{21}{}, \hyperlink{III.25}{25}{}; \hyperlink{IV.32}{IV.32}{}, \hyperlink{IV.86}{86}{}, \hyperlink{IV.132}{132}{}; \hyperlink{VI.67}{VI.67}{}; \hyperlink{VII.28}{VII.28}{}; \hyperlink{XII.7}{XII.7}{}; \hyperlink{XIV.86}{XIV.86}{}; \hyperlink{XVIII.3}{XVIII.3}{}; \hyperlink{XXI.113}{XXI.113f.}{}
\par\textbf{Jīvaka} \hyperlink{XII.61}{XII.61}{}
\par\textbf{Jotika} \hyperlink{VIII.18}{VIII.18}{}; \hyperlink{XII.40}{XII.40}{}
\par\textbf{joy} (\emph{somanassa}) \hyperlink{I.140}{I.140}{}; \hyperlink{IV.184}{IV.184}{}; \hyperlink{IX.88}{IX.88}{}, \hyperlink{IX.100}{100}{}, \hyperlink{IX.111}{111}{}; \hyperlink{XIII.9}{XIII.9}{}; \hyperlink{XIV.12}{XIV.12}{}, \hyperlink{XIV.83}{83}{}, \hyperlink{XIV.98}{98f.}{}, \hyperlink{XIV.109}{109}{}, \hyperlink{XIV.119}{119}{}, \hyperlink{XIV.126}{126f.}{}, \hyperlink{XIV.180}{180}{}; \hyperlink{XV.28}{XV.28}{}; \hyperlink{XX.31}{XX.31}{}, \hyperlink{XX.71}{71}{}; \hyperlink{XXI.114}{XXI.114}{}; j. \emph{faculty }(\emph{somanassindriya}) \hyperlink{XIII.9}{XIII.9}{}; \hyperlink{XVI.1}{XVI.1}{}
\par\textbf{Kaccāna Thera, Mahā} \hyperlink{XXII.83}{XXII.83}{}
\par\textbf{Kadamba Tree} \hyperlink{VII.43}{VII.43}{}
\par\textbf{Kākavaḷiya} \hyperlink{XII.127}{XII.127}{}
\par\textbf{Kāḷadīghavāpi} \hyperlink{VI.77}{VI.77}{}
\par\textbf{Kalyāṇagāma} \hyperlink{XXII.89}{XXII.89}{}
\par\textbf{Kambojā} \hyperlink{X.28}{X.28}{}
\par\textbf{kamma} (\emph{kamma}) \hyperlink{I.6}{I.6}{}, \hyperlink{I.155}{155}{}, \hyperlink{I.n9}{n.9}{}; \hyperlink{III.83}{III.83f.}{}; \hyperlink{V.40}{V.40f.}{}, \hyperlink{V.n6}{n.6}{}; \hyperlink{VII.16}{VII.16f.}{}; \hyperlink{VIII.3}{VIII.3}{}; \hyperlink{IX.23}{IX.23}{}, \hyperlink{IX.96}{96}{}; \hyperlink{XI.111}{XI.111}{}; \hyperlink{XII.n12}{XII.n.12}{}; \hyperlink{XIII.2}{XIII.2}{}, \hyperlink{XIII.35}{35}{}, \hyperlink{XIII.73}{73}{}, \hyperlink{XIII.78}{78}{}; \hyperlink{XIV.37}{XIV.37f.}{}, \hyperlink{XIV.45}{45}{}, \hyperlink{XIV.74}{74f.}{}, \hyperlink{XIV.111}{111f.}{}, \hyperlink{XIV.122}{122}{}, \hyperlink{XIV.188}{188}{}, \hyperlink{XIV.220}{220}{}, \hyperlink{XIV.n14}{n.14}{}, \hyperlink{XIV.n21}{n.21}{}, \hyperlink{XIV.n40}{n.40}{}, \hyperlink{XIV.n74}{n.74}{}; \hyperlink{XVI.5}{XVI.5}{}; \hyperlink{XVII.38}{XVII.38f.}{}, \hyperlink{XVII.45}{45}{}, \hyperlink{XVII.51}{51}{}, \hyperlink{XVII.66}{66}{}, \hyperlink{XVII.88}{88}{}, \hyperlink{XVII.136}{136}{}, \hyperlink{XVII.139}{139}{}, \hyperlink{XVII.174}{174}{}, \hyperlink{XVII.250}{250}{}; \hyperlink{XIX.4}{XIX.4}{}, \hyperlink{XIX.8}{8f.}{}, \hyperlink{XIX.13}{13f.}{}; \hyperlink{XX.22}{XX.22}{}, \hyperlink{XX.27}{27f.}{}, \hyperlink{XX.43}{43}{}, \hyperlink{XX.97}{97}{}; \hyperlink{XXI.38}{XXI.38}{}; \hyperlink{XXII.48}{XXII.48}{}, \hyperlink{XXII.81}{81}{}, \hyperlink{XXII.85}{85}{}, \hyperlink{XXII.88}{88}{}; k.-born (\emph{kamma-ja}) \hyperlink{X.2}{X.2}{}; \hyperlink{XIV.74}{XIV.74}{}; \hyperlink{XVII.196}{XVII.196}{}; \hyperlink{XX.27}{XX.27f.}{}, \hyperlink{XX.70}{70}{}; k.-originated (\emph{kammasamuṭṭhāna}) \hyperlink{XI.35}{XI.35}{}, \hyperlink{XI.88}{88}{}, \hyperlink{XI.94}{94}{}, \hyperlink{XI.111}{111}{}; \hyperlink{XIV.188}{XIV.188}{}; \hyperlink{XVII.194}{XVII.194}{}, \hyperlink{XVII.199}{199}{}, \hyperlink{XVII.202}{202}{}; \hyperlink{XVIII.5}{XVIII.5}{}, \hyperlink{XVIII.22}{22}{}; \hyperlink{XIX.9}{XIX.9}{}, \hyperlink{XIX.27}{27f.}{}; k. performed (\emph{kaṭatta}) \hyperlink{XVII.89}{XVII.89}{}, \hyperlink{XVII.122}{122}{}, \hyperlink{XVII.174}{174}{}; k.-process becoming (\emph{kamma-bhava}) \hyperlink{VII.16}{VII.16f.}{}; \hyperlink{XVII.250}{XVII.250f.}{}, \hyperlink{XVII.292}{292}{}; \hyperlink{XIX.15}{XIX.15}{}; k.-re-suit (\emph{vipāka}) \hyperlink{V.40}{V.40f.}{}; \hyperlink{XV.34}{XV.34}{}; \hyperlink{XVII.51}{XVII.51}{}, \hyperlink{XVII.66}{66}{}, \hyperlink{XVII.89}{89}{}, \hyperlink{XVII.120}{120}{}, \hyperlink{XVII.134}{134}{}; \hyperlink{XIX.8}{XIX.8}{}; \hyperlink{XXII.81}{XXII.81}{}. See also result 
\par\textbf{kammically acquired} (\emph{upādiṇṇa}), see clung-to 
\par\textbf{Kaṇāda} \hyperlink{XIV.n19}{XIV.n.19}{}
\par\textbf{Kaṇṇamuṇḍaka, Lake} \hyperlink{XIII.38}{XIII.38}{}
\par\textbf{Kappa Tree} \hyperlink{VII.43}{VII.43}{}
\par\textbf{Kappina Thera, Mahā} \hyperlink{XII.82}{XII.82}{}
\par\textbf{Karañjiya-vihāra, Mahā} \hyperlink{VIII.243}{VIII.243}{}
\par\textbf{Karavīka bird} \hyperlink{III.111}{III.111}{}
\par\textbf{Karavīka-pabbata} \hyperlink{VII.42}{VII.42}{}
\par\textbf{Karuliyagiri} (Karaliya-, Karuḷiya-) \hyperlink{III.52}{III.52}{}
\par\textbf{Kasiṇa} (\emph{kasiṇa}) \hyperlink{III.97}{III.97}{}, \hyperlink{III.105}{105f.}{}, \hyperlink{III.119}{119}{}; IV passim; \hyperlink{V}{V passim}{}, \hyperlink{V.n5}{n.5}{}; \hyperlink{VII.28}{VII.28}{}; \hyperlink{IX.104}{IX.104}{}, \hyperlink{IX.121}{121}{}; \hyperlink{X.1}{X.1f.}{}; \hyperlink{XI.n18}{XI.n.18}{}; \hyperlink{XII.2}{XII.2f.}{}, \hyperlink{XII.88}{88f.}{}; \hyperlink{XIII.95}{XIII.95}{}; \hyperlink{XVII.143}{XVII.143}{}; \hyperlink{XX.9}{XX.9}{}; \hyperlink{XXIII.20}{XXIII.20}{}
\par\textbf{Kassapa Thera} \hyperlink{I.41}{I.41}{}; \hyperlink{II.32}{II.32}{}; \hyperlink{XII.126}{XII.126}{}; \hyperlink{XIII.107}{XIII.107}{}
\par\textbf{Kaṭakandhakāra} \hyperlink{VII.127}{VII.127}{}
\par\textbf{Khāṇu-Kondañña Thera} \hyperlink{XII.30}{XII.30}{}, \hyperlink{XII.33}{33}{}
\par\textbf{Khattiya} (Warrior Noble) \hyperlink{XIII.54}{XIII.54}{}
\par\textbf{Khujjuttarā upāsikā} \hyperlink{XIV.27}{XIV.27}{}
\par\textbf{killing living things} (\emph{pāṇātipāta}) \hyperlink{I.17}{I.17}{}, \hyperlink{I.140}{140}{}; \hyperlink{XVII.39}{XVII.39}{}, \hyperlink{XVII.60}{60}{}; \hyperlink{XXII.62}{XXII.62}{}
\par\textbf{knowledge} (\emph{ñāṇa}) \hyperlink{I.18}{I.18}{}, \hyperlink{I.140}{140}{}; \hyperlink{II.84}{II.84}{}; \hyperlink{IV.118}{IV.118}{}; \hyperlink{VII.7}{VII.7}{}, \hyperlink{VII.n7}{n.7}{}; \hyperlink{VIII.174}{VIII.174}{}; \hyperlink{IX.124}{IX.124}{}; \hyperlink{XII.26}{XII.26}{}; \hyperlink{XIII}{XIII passim}{}, \hyperlink{XIII.n6}{n.6}{}; \hyperlink{XIV.2}{XIV.2}{}, \hyperlink{XIV.20}{20f.}{}, \hyperlink{XIV.83}{83}{}, \hyperlink{XIV.126}{126}{}; \hyperlink{XV.21}{XV.21}{}; \hyperlink{XX.94}{XX.94}{}, \hyperlink{XX.114}{114}{}, \hyperlink{XX.129}{129f.}{}; \hyperlink{XXI.12}{XXI.12}{}, \hyperlink{XXI.52}{52}{}; \hyperlink{XXII.25}{XXII.25}{}, \hyperlink{XXII.46}{46}{}, \hyperlink{XXII.66}{66}{}; \hyperlink{XXIII.20}{XXIII.20}{}; in conformity with truth (\emph{saccānulomika-ñāṇa}) \hyperlink{XXI.1}{XXI.1}{}; k. of dispassion (\emph{nibbidā-ñāṇa}) \hyperlink{XXI.81}{XXI.81}{}, \hyperlink{XXI.131}{131}{}; k. of faring according to deeds (\emph{yathā-kammūpagañāṇa}) \hyperlink{XIII.78}{XIII.78f.}{}, \hyperlink{XIII.103}{103}{}, \hyperlink{XIII.122}{122}{}, \hyperlink{XIII.128}{128}{}; k. of the future (\emph{anāgataṃsa-ñāṇa}) \hyperlink{XIII.80}{XIII.80}{}, \hyperlink{XIII.103}{103}{}, \hyperlink{XIII.122}{122}{}, \hyperlink{XIII.125}{125}{}; k. of passing away and re-appearance (\emph{cutūpapāta-ñāṇa}) \hyperlink{XII.2}{XII.2}{}; \hyperlink{XIII.72}{XIII.72}{}; k. of penetration of minds (\emph{cetopariya-ñāṇa}) \hyperlink{III.96}{III.96}{}; \hyperlink{XII.2}{XII.2}{}, \hyperlink{XII.136}{136}{}; \hyperlink{XIII.8}{XIII.8}{}, \hyperlink{XIII.110}{110}{}, \hyperlink{XIII.120}{120}{}; k. of relations of states (\emph{dhammaṭṭhiti-ñāṇa}) \hyperlink{VII.20}{VII.20}{}; \hyperlink{XIX.25}{XIX.25}{}; \hyperlink{XXI.135}{XXI.135}{}; k. of rise and fall (\emph{udayabbaya-ñāṇa}) \hyperlink{XXI.131}{XXI.131}{}; k. of the path (\emph{magga-ñāṇa}) \hyperlink{XXII.3}{XXII.3f.}{}, \hyperlink{XXII.22}{22f.}{}, \hyperlink{XXII.25}{25f}{}, \hyperlink{XXII.28}{28}{}; k. of reviewing (\emph{paccavekkhaṇa-ñāṇa}) \hyperlink{I.32}{I.32}{}; k. and vision of deliverance (\emph{vimutti-ñāṇadassana}) \hyperlink{I.32}{I.32}{}
\par\textbf{kolaṅkola} \hyperlink{XXIII.55}{XXIII.55}{}
\par\textbf{Koraṇḍaka-vihāra} \hyperlink{III.36}{III.36}{}
\par\textbf{Kosala} \hyperlink{VII.23}{VII.23}{}
\par\textbf{Koṭapabbata} \hyperlink{VIII.243}{VIII.243}{}
\par\textbf{Kumbhakāragāma} \hyperlink{III.33}{III.33}{}
\par\textbf{Kumbhaṇḍa} (demon) \hyperlink{XII.n19}{XII.n.19}{}
\par\textbf{Kuṇāla, Lake} \hyperlink{XIII.38}{XIII.38}{}
\par\textbf{Kuraṇḍaka-mahā-leṇa} \hyperlink{I.104}{I.104}{}
\par\textbf{lakes, 7 great} (\emph{satta mahāsarā}) \hyperlink{XIII.38}{XIII.38}{}; \hyperlink{XXI.43}{XXI.43}{}
\par\textbf{lamentation} (\emph{parideva}) \hyperlink{XVI.31}{XVI.31}{}, \hyperlink{XVI.49}{49}{}; \hyperlink{XVII.2}{XVII.2}{}, \hyperlink{XVII.48}{48}{}
\par\textbf{language} (\emph{nirutti}) \hyperlink{VII.58}{VII.58}{}; \hyperlink{XIV.21}{XIV.21f.}{}
\par\textbf{lapsed kamma} (\emph{ahosi-kamma}) \hyperlink{XIX.14}{XIX.14}{}
\par\textbf{lastingness} (\emph{dhuva-bhāva}) \hyperlink{XVI.16}{XVI.16}{}, \hyperlink{XVI.85}{85}{}, \hyperlink{XVI.90}{90}{}; \hyperlink{XVII.283}{XVII.283}{}
\par\textbf{later-food refuser} (\emph{khalupacchābhattika}) \hyperlink{II.2}{II.2}{}, \hyperlink{II.43}{43}{}
\par\textbf{law} (\emph{dhamma}) \hyperlink{VII.68}{VII.68}{}, \hyperlink{VII.n1}{n.1}{}; \hyperlink{XIV.21}{XIV.21f.}{}; \hyperlink{XVII.25}{XVII.25}{}. See also dhamma 
\par\textbf{lay follower} (\emph{upāsaka}) \hyperlink{I.40}{I.40}{}; \hyperlink{II.92}{II.92}{}
\par\textbf{lesser stream-enterer} (\emph{cūḷa-sotāpanna}) \hyperlink{XIX.27}{XIX.27}{}
\par\textbf{liberated in both ways} (\emph{ubhatobhāga-vimutta}) \hyperlink{XXI.74}{XXI.74}{}, \hyperlink{XXI.89}{89}{}; \hyperlink{XXIII.58}{XXIII.58}{}; I. by faith (\emph{saddhā-vimutta}) \hyperlink{XXI.74}{XXI.74}{}, \hyperlink{XXI.89}{89}{}; \hyperlink{XXIII.38}{XXIII.38}{}; I. by understanding (\emph{paññā-vimutta}) \hyperlink{XXI.74}{XXI.74}{}, \hyperlink{XXI.89}{89}{}; \hyperlink{XXIII.58}{XXIII.58}{}
\par\textbf{liberation} (\emph{vimokkha}) \hyperlink{V.32}{V.32}{}; \hyperlink{VII.48}{VII.48}{}, \hyperlink{VII.63}{63}{}; \hyperlink{IX.120}{IX.120}{}; \hyperlink{X.n3}{X.n.3}{}; \hyperlink{XIV.31}{XIV.31}{}; \hyperlink{XVII.281}{XVII.281}{}; \hyperlink{XXI.66}{XXI.66f.}{}, \hyperlink{XXI.119}{119}{}
\par\textbf{life} (\emph{jīvita}) \hyperlink{VII.108}{VII.108}{}; \hyperlink{VIII.27}{VIII.27f.}{}, \hyperlink{VIII.35}{35}{}; \hyperlink{XIV.47}{XIV.47}{}, \hyperlink{XIV.59}{59}{}, \hyperlink{XIV.133}{133}{}, \hyperlink{XIV.138}{138}{}, \hyperlink{XIV.159}{159}{}, \hyperlink{XIV.170}{170}{}, \hyperlink{XIV.176}{176}{}, \hyperlink{XIV.179}{179}{}; \hyperlink{XVII.156}{XVII.156}{}, \hyperlink{XVII.190}{190}{}, \hyperlink{XVII.192}{192}{}, \hyperlink{XVII.217}{217}{}; \hyperlink{XVIII.5}{XVIII.5f.}{}; \hyperlink{XXIII.42}{XXIII.42}{}; l.-continuum (\emph{bhavaṅga}) \hyperlink{I.57}{I.57}{}; \hyperlink{IV.33}{IV.33}{}, \hyperlink{IV.74}{74f.}{}, \hyperlink{IV.78}{78}{}, \hyperlink{IV.132}{132}{}, \hyperlink{IV.138}{138}{}, \hyperlink{IV.n13}{n.13}{}; \hyperlink{XIV.98}{XIV.98f.}{}, \hyperlink{XIV.107}{107}{}, \hyperlink{XIV.114}{114}{}, \hyperlink{XIV.115}{115f.}{}, \hyperlink{XIV.n45}{n.45}{}; \hyperlink{XV.10}{XV.10}{}, \hyperlink{XV.37}{37}{}, \hyperlink{XV.n5}{n.5}{}; \hyperlink{XVII.129}{XVII.129f.}{}, \hyperlink{XVII.136}{136f.}{}, \hyperlink{XVII.193}{193}{}, \hyperlink{XVII.201}{201}{}, \hyperlink{XVII.232}{232}{}; \hyperlink{XIX.8}{XIX.8}{}; \hyperlink{XX.24}{XX.24}{}, \hyperlink{XX.31}{31}{}, \hyperlink{XX.43}{43}{}; \hyperlink{XXI.129}{XXI.129}{}, \hyperlink{XXI.n41}{n.41}{}; \hyperlink{XXII.19}{XXII.19}{}; \hyperlink{XXII.14}{XXII.14}{}; l. ennead (\emph{jīvitadasaka}) \hyperlink{XVII.156}{XVII.156}{}, \hyperlink{XVII.190}{190}{}, \hyperlink{XVII.192}{192}{}; l. faculty (\emph{jīvitindriya}) \hyperlink{I.91}{I.91}{}; \hyperlink{XI.88}{XI.88}{}; \hyperlink{XIII.91}{XIII.91}{}; \hyperlink{XIV.59}{XIV.59}{}; \hyperlink{XVI.1}{XVI.1}{}, \hyperlink{XVI.8}{8}{}, \hyperlink{XVI.10}{10}{}, \hyperlink{XVI.46}{46}{}; \hyperlink{XVII.190}{XVII.190}{}, \hyperlink{XVII.192}{192}{}; l. of purity (\emph{brahmacariya}) \hyperlink{I.92}{I.92}{}, \hyperlink{I.144}{144}{}; \hyperlink{VII.69}{VII.69}{}, \hyperlink{VII.72}{72}{}; l. span (\emph{āyu}) \hyperlink{VIII.3}{VIII.3}{}, \hyperlink{VIII.243}{243}{}; \hyperlink{XIII.44}{XIII.44}{}; \hyperlink{XXIII.42}{XXIII.42f.}{}, \hyperlink{XXIII.48}{48}{}
\par\textbf{light} (\emph{āloka}) \hyperlink{I.n14}{I.n.14}{}; \hyperlink{V.n5}{V.n.5}{}; \hyperlink{XIII.9}{XIII.9}{}, \hyperlink{XIII.79}{79}{}; \hyperlink{XV.39}{XV.39}{}; \hyperlink{XX.108}{XX.108f.}{}; l. kasiṇa (\emph{āloka-kasiṇa}) \hyperlink{III.105}{III.105}{}; \hyperlink{V.21}{V.21}{}; \hyperlink{XIII.95}{XIII.95}{}
\par\textbf{lightness} (\emph{lahutā}) \hyperlink{XIV.64}{XIV.64}{}, \hyperlink{XIV.76}{76}{}, \hyperlink{XIV.79}{79}{}, \hyperlink{XIV.133}{133}{}, \hyperlink{XIV.145}{145}{}; \hyperlink{XVIII.13}{XVIII.13}{}; \hyperlink{XX.23}{XX.23}{}, \hyperlink{XX.32}{32}{}, \hyperlink{XX.36}{36}{}
\par\textbf{limited} (\emph{paritta}) \hyperlink{III.5}{III.5}{}, \hyperlink{III.13}{13}{}, \hyperlink{III.20}{20}{}, \hyperlink{III.112}{112}{}; \hyperlink{IV.74}{IV.74}{}; \hyperlink{XIII.105}{XIII.105f.}{}; \hyperlink{XIV.15}{XIV.15}{}; \hyperlink{XVII.53}{XVII.53}{}
\par\textbf{lineage} (\emph{gotta}) \hyperlink{IV.74}{IV.74}{}; \hyperlink{XIII.123}{XIII.123}{}
\par\textbf{livelihood} (\emph{ājīva}) \hyperlink{I.18}{I.18}{}, \hyperlink{I.42}{42}{}, \hyperlink{I.44}{44}{}, \hyperlink{I.60}{60}{}, \hyperlink{I.84}{84}{}, \hyperlink{I.111}{111}{}, \hyperlink{I.123}{123}{}; \hyperlink{XXII.42}{XXII.42}{}, \hyperlink{XXII.45}{45}{}, \hyperlink{XXII.66}{66}{}
\par\textbf{living being} (\emph{satta}), see being
\par\textbf{logical relation, double \& quadruple} (\emph{dvi-, catu-koṭika}) \hyperlink{XXI.53}{XXI.53}{}
\par\textbf{lordship} (\emph{issariya}) \hyperlink{VII.61}{VII.61}{}
\par\textbf{loving-kindness} (\emph{mettā}) \hyperlink{III.57}{III.57f.}{}, \hyperlink{III.105}{105f.}{}, \hyperlink{III.122}{122}{}; \hyperlink{VII.18}{VII.18}{}, \hyperlink{VII.28}{28}{}; \hyperlink{IX.1}{IX.1f.}{}, \hyperlink{IX.92}{92f.}{}, \hyperlink{IX.98}{98}{}, \hyperlink{IX.119}{119f.}{}; \hyperlink{XII.34}{XII.34}{}, \hyperlink{XII.37}{37}{}; \hyperlink{XIII.34}{XIII.34}{}; \hyperlink{XIV.154}{XIV.154}{}
\par\textbf{lust} (\emph{kāmacchanda}) \hyperlink{I.140}{I.140}{}; \hyperlink{IV.85}{IV.85}{}, \hyperlink{IV.104}{104}{}, \hyperlink{IV.n24}{n.24}{}; \hyperlink{XVI.10}{XVI.10}{}; \hyperlink{XXII.57}{XXII.57}{}\emph{; (}\emph{rāga}) \hyperlink{IX.6}{IX.6}{}. See also greed 
\par\textbf{Magadha} \hyperlink{XIV.25}{XIV.25}{}, \hyperlink{XIV.30}{30}{}; \hyperlink{XVIII.25}{XVIII.25}{}
\par\textbf{Māgandiya} \hyperlink{XII.35}{XII.35}{}
\par\textbf{magnanimous ordinary man} (\emph{kalyāṇaputhujjana}) \hyperlink{I.35}{I.35}{}, \hyperlink{I.131}{131}{}
\par\textbf{Mahā-Anula Thera} , etc., see under individual names Anula, etc. 
\par\textbf{Mahā-Brahmā} \hyperlink{XII.79}{XII.79}{}
\par\textbf{Mahā-cetiya} , see Great Shrine
\par\textbf{Mahāgāma} \hyperlink{I.106}{I.106}{}
\par\textbf{Mahaka} \hyperlink{XII.84}{XII.84}{}
\par\textbf{Mahānāma} \hyperlink{VII.111}{VII.111}{}; \hyperlink{XXII.21}{XXII.21}{}
\par\textbf{Mahāsaṅghika} \hyperlink{XIV.n16}{XIV.n.16}{}
\par\textbf{Mahāsammata} \hyperlink{VIII.17}{VIII.17}{}; \hyperlink{XIII.54}{XIII.54}{}
\par\textbf{Mahātittha} \hyperlink{V.2}{V.2}{}
\par\textbf{Mahāvaṭṭani Forest} \hyperlink{I.99}{I.99}{}
\par\textbf{Mahāvihāra} , see Great Monastery
\par\textbf{Mahinda Thera} \hyperlink{XII.83}{XII.83}{}
\par\textbf{Mahinda-guhā} (M.’s Cave) \hyperlink{III.102}{III.102}{}
\par\textbf{Malaya} (Hill Country, Sri Lanka) \hyperlink{III.51}{III.51}{}; \hyperlink{VIII.49}{VIII.49}{}
\par\textbf{malicious speech} (\emph{pisuṇā-vācā}) \hyperlink{I.140}{I.140}{}; \hyperlink{XXII.66}{XXII.66}{}
\par\textbf{Mallaka Thera} \hyperlink{IV.23}{IV.23}{}; \hyperlink{VIII.142}{VIII.142}{}
\par\textbf{malleability} (\emph{mudutā}) \hyperlink{XIV.64}{XIV.64}{}, \hyperlink{XIV.133}{133}{}, \hyperlink{XIV.146}{146}{}; \hyperlink{XVIII.13}{XVIII.13}{}
\par\textbf{man} (\emph{purisa}) \hyperlink{I.n14}{I.n.14}{}; \hyperlink{XI.30}{XI.30}{}; \hyperlink{XVII.n4}{XVII.n.4}{}
\par\textbf{Maṇiliyā} \hyperlink{IX.69}{IX.69}{}
\par\textbf{Maṇḍuka Devaputta} \hyperlink{VII.51}{VII.51}{}
\par\textbf{Māra} \hyperlink{VII.59}{VII.59}{}, \hyperlink{VII.128}{128}{}, \hyperlink{VII.n14}{n.14}{}; \hyperlink{XII.10}{XII.10}{}; \hyperlink{XX.19}{XX.19}{}
\par\textbf{masculinity faculty} (\emph{purisindriya}) \hyperlink{XIV.58}{XIV.58}{}; \hyperlink{XVI.1}{XVI.1}{}
\par\textbf{mastery} (\emph{vasi}) \hyperlink{IV.131}{IV.131}{}; \hyperlink{XX.102}{XX.102}{}
\par\textbf{material becoming} (\emph{rūpa-bhava}) \hyperlink{XXI.34}{XXI.34}{}; m. body (\emph{rūpa-kāya}) \hyperlink{XVIII.36}{XVIII.36}{}; m. septad (\emph{rūpa-sattaka}) \hyperlink{XX.45}{XX.45f.}{}
\par\textbf{materiality, matter} (\emph{rūpa}) \hyperlink{I.140}{I.140}{}, \hyperlink{I.n14}{n.14}{}; \hyperlink{VII.28}{VII.28}{}, \hyperlink{VII.38}{38}{}; \hyperlink{VIII.180}{VIII.180}{}, \hyperlink{VIII.233}{233}{}; \hyperlink{IX.121}{IX.121}{}; \hyperlink{X.1}{X.1f.}{}; \hyperlink{XI.2}{XI.2}{}, \hyperlink{XI.26}{26}{}, \hyperlink{XI.96}{96}{}; \hyperlink{XII.n20}{XII.n.20}{}, \hyperlink{XII.n21}{n.21}{}; \hyperlink{XIII.9}{XIII.9}{}, \hyperlink{XIII.113}{113}{}, \hyperlink{XIII.124}{124}{}, \hyperlink{XIII.n17}{n.17}{}; \hyperlink{XIV.8}{XIV.8}{}, \hyperlink{XIV.11}{11}{}, \hyperlink{XIV.33}{33f.}{}, \hyperlink{XIV.195}{195}{}, \hyperlink{XIV.214}{214}{}, \hyperlink{XIV.244}{244}{}; \hyperlink{XV.13}{XV.13f.}{}; \hyperlink{XVI.93}{XVI.93}{}; \hyperlink{XVII.48}{XVII.48}{}, \hyperlink{XVII.51}{51}{}, \hyperlink{XVII.69}{69}{}, \hyperlink{XVII.72}{72}{}, \hyperlink{XVII.148}{148f.}{}, \hyperlink{XVII.187}{187}{}, \hyperlink{XVII.193}{193}{}, \hyperlink{XVII.197}{197}{}; \hyperlink{XVIII}{XVIII passim}{}, \hyperlink{XVIII.8}{8}{}; \hyperlink{XX.7}{XX.7}{}, \hyperlink{XX.9}{9}{}, \hyperlink{XX.22}{22f.}{}, \hyperlink{XX.68}{68}{}, \hyperlink{XX.73}{73}{}; \hyperlink{XXI.10}{XXI.10}{}, \hyperlink{XXI.56}{56}{}, \hyperlink{XXI.86}{86}{}; \hyperlink{XXII.22}{XXII.22}{}, \hyperlink{XXII.126}{126}{}; \hyperlink{XXIII.13}{XXIII.13}{}, \hyperlink{XXIII.22}{22}{}
\par\textbf{mātikā} , see schedule \& code
\par\textbf{meaning} (\emph{attha}) \hyperlink{VII.72}{VII.72}{}; \hyperlink{XIV.21}{XIV.21f.}{}; \hyperlink{XVII.25}{XVII.25}{}
\par\textbf{means} (\emph{upāya}) \hyperlink{I.85}{I.85}{}; \hyperlink{XVI.28}{XVI.28}{}
\par\textbf{measureless} (\emph{appamāṇa}) \hyperlink{III.5}{III.5}{}, \hyperlink{III.13}{13}{}, \hyperlink{III.20}{20}{}, \hyperlink{III.112}{112}{}; \hyperlink{XIII.120}{XIII.120}{}; \hyperlink{XIV.15}{XIV.15}{}; m. state (\emph{appamaññā}) \hyperlink{VII.28}{VII.28}{}; \hyperlink{IX}{IX passim}{}, \hyperlink{IX.105}{105}{}, \hyperlink{IX.110}{110}{}; \hyperlink{XX.9}{XX.9}{}; XXI\hyperlink{II.4}{II.4}{}
\par\textbf{medicine} (\emph{bhesajja}) \hyperlink{I.96}{I.96}{}, \hyperlink{I.115}{115}{}
\par\textbf{meditation subject} (\emph{kammaṭṭhāna}) \hyperlink{III.57}{III.57}{}, \hyperlink{III.103}{III.103f.}{}, \hyperlink{XI.119}{XI.119}{}; \hyperlink{XIV.28}{XIV.28}{}
\par\textbf{Meṇḍaka} \hyperlink{VIII.18}{VIII.18}{}; \hyperlink{XII.40}{XII.40f.}{}
\par\textbf{mental body} (\emph{nāma-kāya}) \hyperlink{XIV.133}{XIV.133f.}{}; XVI\hyperlink{II.36}{II.36}{}; \hyperlink{XIX.5}{XIX.5}{}; m. datum, m. object (\emph{dhamma}) \hyperlink{I.n1}{I.n.1}{}; \hyperlink{XXII.34}{XXII.34}{}; m.-data base (\emph{dhammāyatana}) \hyperlink{X.49}{X.49}{}; \hyperlink{XV.3}{XV.3}{}; \hyperlink{XVIII.14}{XVIII.14}{}; m.-data element (\emph{dhamma-dhātu}) \hyperlink{XV.17}{XV.17}{}; m. šformation (\emph{citta-saṅkhāra}) \hyperlink{VIII.229}{VIII.229}{}; \hyperlink{XVII.61}{XVII.61}{}; \hyperlink{XXIII.24}{XXIII.24}{}, \hyperlink{XXIII.51}{51}{}; m. volition (\emph{mano-sañcetanā}) \hyperlink{XIV.228}{XIV.228}{}; \hyperlink{XVII.61}{XVII.61}{}
\par\textbf{mentality} (\emph{nāma}) \hyperlink{VII.38}{VII.38}{}; \hyperlink{XIV.8}{XIV.8}{}, \hyperlink{XIV.11}{11}{}, \hyperlink{XIV.n35}{n.35}{}; \hyperlink{XV.13}{XV.13}{}; \hyperlink{XVII.48}{XVII.48}{}, \hyperlink{XVII.51}{51}{}, \hyperlink{XVII.187}{187}{}, \hyperlink{XVII.206}{206f.}{}; \hyperlink{XVIII}{XVIII passim}{}, \hyperlink{XVIII.8}{8}{}
\par\textbf{mentality-materiality} (\emph{nāma-rūpa}) \hyperlink{VII.11}{VII.11}{}, \hyperlink{VII.38}{38}{}; \hyperlink{VIII.180}{VIII.180}{}, \hyperlink{VIII.222}{222f.}{}; \hyperlink{XI.2}{XI.2}{}; \hyperlink{XII.24}{XII.24f.}{}; \hyperlink{XV.13}{XV.13}{}; \hyperlink{XVI.92}{XVI.92}{}; \hyperlink{XVII.2}{XVII.2}{}, \hyperlink{XVII.55}{55}{}, \hyperlink{XVII.186}{186f.}{}, \hyperlink{XVII.218}{218f.}{}, \hyperlink{XVII.294}{294}{}; \hyperlink{XVII}{XVII passim}{}; \hyperlink{XIX.1}{XIX.1f.}{}; \hyperlink{XX.2}{XX.2f.}{}
\par\textbf{merit} (\emph{puñña}) \hyperlink{I.68}{I.68}{}; \hyperlink{VI.22}{VI.22}{}; \hyperlink{VII.n1}{VII.n.1}{}; \hyperlink{XII.40}{XII.40}{}; \hyperlink{XVII.60}{XVII.60f.}{}, \hyperlink{XVII.102}{102}{}, \hyperlink{XVII.119}{119}{}, \hyperlink{XVII.177}{177}{}, \hyperlink{XVII.251}{251}{}
\par\textbf{merriment} (\emph{pahāsa}) \hyperlink{IX.95}{IX.95}{}
\par\textbf{method} (\emph{naya}) \hyperlink{XVII.11}{XVII.11}{}, \hyperlink{XVII.33}{33}{}, \hyperlink{XVII.309}{309f.}{}; \hyperlink{XX.102}{XX.102}{}; \hyperlink{XXI.52}{XXI.52}{}
\par\textbf{Metteyya Bhagavant} \hyperlink{I.135}{I.135}{}; \hyperlink{XIII.127}{XIII.127}{}
\par\textbf{Mīḷhābhaya Therā} , see Abhaya Thera, Pīṭha 
\par\textbf{mind} (\emph{citta}) \hyperlink{I.103}{I.103}{}; \hyperlink{XIV.n35}{XIV.n.35}{}. See also consciousness; (\emph{mano}) \hyperlink{XIV.82}{XIV.82}{}; \hyperlink{XV.3}{XV.3}{}, \hyperlink{XV.12}{12}{}; \hyperlink{XVI.10}{XVI.10}{}; \hyperlink{XX.70}{XX.70}{}; m. base (\emph{manāyatana}) \hyperlink{X.49}{X.49}{}; \hyperlink{XV.3}{XV.3f.}{}; \hyperlink{XVIII.12}{XVIII.12}{}; m. consciousness (\emph{manoviññāṇa}) \hyperlink{XVII.120}{XVII.120}{}; m.-c. element (\emph{manoviññāṇa-dhātu}) \hyperlink{I.57}{I.57}{}; \hyperlink{IV.n13}{IV.n.13}{}; \hyperlink{VIII.111}{VIII.111}{}; \hyperlink{X.20}{X.20}{}; \hyperlink{XIV.60}{XIV.60}{}, \hyperlink{XIV.95}{95}{}, \hyperlink{XIV.97}{97}{}, \hyperlink{XIV.99}{99}{}, \hyperlink{XIV.108}{108}{}, \hyperlink{XIV.116}{116}{}, \hyperlink{XIV.120}{120f.}{}, \hyperlink{XIV.180}{180}{}, \hyperlink{XIV.n26}{n.26}{}; \hyperlink{XV.17}{XV.17}{}; \hyperlink{XVII.73}{XVII.73}{}, \hyperlink{XVII.120}{120f.}{}; \hyperlink{XVIII.8}{XVIII.8}{}, \hyperlink{XVIII.11}{11}{}; \hyperlink{XX.31}{XX.31}{}, \hyperlink{XX.44}{44}{}; m. deliverance (\emph{ceto-vimutti}) \hyperlink{IV.191}{IV.191}{}; \hyperlink{IX.50}{IX.50}{}, \hyperlink{IX.115}{115f.}{}; m. door (\emph{mano-dvāra}) \hyperlink{XIV.116}{XIV.116}{}; \hyperlink{XVII.61}{XVII.61}{}; \hyperlink{XX.44}{XX.44}{}, \hyperlink{XX.70}{70}{}, \hyperlink{XX.121}{121}{}; \hyperlink{XXI.129}{XXI.129}{}; \hyperlink{XXII.19}{XXII.19}{}; m. element (\emph{mano-dhātu}) \hyperlink{I.57}{I.57}{}; \hyperlink{IV.n13}{IV.n.13}{}; \hyperlink{VIII.111}{VIII.111}{}; \hyperlink{X.20}{X.20}{}; \hyperlink{XIV.60}{XIV.60}{}, \hyperlink{XIV.95}{95f.}{}, \hyperlink{XIV.107}{107}{}, \hyperlink{XIV.115}{115}{}, \hyperlink{XIV.118}{118f.}{}, \hyperlink{XIV.180}{180}{}, \hyperlink{XIV.n26}{n.26}{}; \hyperlink{XV.7}{XV.7}{}; \hyperlink{XVII.73}{XVII.73}{}, \hyperlink{XVII.120}{120f.}{}; \hyperlink{XVIII.8}{XVIII.8}{}, \hyperlink{XVIII.11}{11}{}; \hyperlink{XIX.23}{XIX.23}{}, \hyperlink{XIX.31}{31}{}; m. faculty (\emph{manindriya}) \hyperlink{XVI.1}{XVI.1}{}, \hyperlink{XVI.10}{10}{}
\par\textbf{mindfulness} (\emph{sati}) \hyperlink{I.18}{I.18}{}, \hyperlink{I.26}{26}{}, \hyperlink{I.51}{51}{}, \hyperlink{I.56}{56}{}, \hyperlink{I.100}{100}{}; \hyperlink{III.95}{III.95}{}; \hyperlink{IV.45}{IV.45}{}, \hyperlink{IV.49}{49}{}, \hyperlink{IV.172}{172}{}, \hyperlink{IV.194}{194}{}; \hyperlink{VII.n8}{VII.n.8}{}; \hyperlink{XII.17}{XII.17}{}; \hyperlink{XIII.13}{XIII.13}{}; \hyperlink{XIV.133}{XIV.133}{}, \hyperlink{XIV.141}{141}{}; \hyperlink{XVI.1}{XVI.1}{}, \hyperlink{XVI.86}{86}{}; \hyperlink{XX.120}{XX.120}{}; \hyperlink{XXI.10}{XXI.10}{}; \hyperlink{XXII.34}{XXII.34}{}, \hyperlink{XXII.38}{38f.}{}, \hyperlink{XXII.42}{42}{}, \hyperlink{XXII.45}{45}{}, \hyperlink{XXII.66}{66}{}; m. occupied with the body (\emph{kāyagatā sati}) \hyperlink{III.105}{III.105f.}{}; \hyperlink{VIII.42}{VIII.42f.}{}; \hyperlink{XI.26}{XI.26}{}; m. of breathing (\emph{ānāpāna-sati}) \hyperlink{III.105}{III.105f.}{}, \hyperlink{III.122}{122}{}; \hyperlink{VIII.43}{VIII.43}{}, \hyperlink{VIII.145}{145f.}{}; \hyperlink{XXIII.20}{XXIII.20}{}; m. of death (\emph{maraṇa-sati}) \hyperlink{III.6}{III.6}{}, \hyperlink{III.57}{57f.}{}, \hyperlink{III.105}{105}{}; \hyperlink{VIII.1}{VIII.1f.}{}
\par\textbf{mind-made} (\emph{mano-maya}) \hyperlink{VII.30}{VII.30}{}; \hyperlink{XII.135}{XII.135}{}, \hyperlink{XII.139}{139}{}
\par\textbf{miracle, marvel} (\emph{pāṭihāriya}) \hyperlink{XII.71}{XII.71}{}, \hyperlink{XII.74}{74}{}
\par\textbf{misapprehension} , see adherence
\par\textbf{misconduct} (\emph{duccarita}) \hyperlink{I.13}{I.13}{}; \hyperlink{VII.59}{VII.59}{}, \hyperlink{VII.n25}{n.25}{}; \hyperlink{XIV.155}{XIV.155}{}, \hyperlink{XIV.160}{160}{}
\par\textbf{misinterpretation} (\emph{abhinivesa}), see interpreting 
\par\textbf{Missaka Grove} \hyperlink{XIII.79}{XIII.79}{}
\par\textbf{Mitta Thera, Mahā} \hyperlink{I.104}{I.104}{}, \hyperlink{I.109}{109}{}
\par\textbf{mode} (\emph{ākāra}) \hyperlink{I.n14}{I.n.14}{}; \hyperlink{XIV.61}{XIV.61f.}{}, \hyperlink{XIV.66}{66}{}; \hyperlink{XVII.14}{XVII.14}{}; \hyperlink{XVIII.13}{XVIII.13}{}; \hyperlink{XXI.6}{XXI.6f.}{}, See also aspect 
\par\textbf{Moggallāna Thera, Mahā} \hyperlink{I.117}{I.117}{}; \hyperlink{IV.133}{IV.133}{}; \hyperlink{VIII.20}{VIII.20}{}; \hyperlink{XII.76}{XII.76}{}, \hyperlink{XII.105}{105f.}{}, \hyperlink{XII.111}{111f.}{}, \hyperlink{XII.122}{122}{}, \hyperlink{XII.127}{127}{}; \hyperlink{XXI.118}{XXI.118}{}
\par\textbf{moment} (\emph{khaṇa}) \hyperlink{IV.78}{IV.78}{}, \hyperlink{IV.99}{99}{}, \hyperlink{IV.n22}{n.22}{}, \hyperlink{IV.n33}{n.33}{}; \hyperlink{XIV.190}{XIV.190}{}, \hyperlink{XIV.197}{197}{}; \hyperlink{XVI.75}{XVI.75}{}; \hyperlink{XVII.193}{XVII.193}{}; \hyperlink{XIX.9}{XIX.9}{}; \hyperlink{XX.22}{XX.22}{}, \hyperlink{XX.97}{97}{}, \hyperlink{XX.100}{100f.}{}; \hyperlink{XXII.92}{XXII.92f.}{} See also instant
\par\textbf{momentary concentration} (\emph{khaṇika-samādhi}) \hyperlink{I.n3}{I.n.3}{}, \hyperlink{I.n4}{n.4}{}; \hyperlink{IV.99}{IV.99}{}; \hyperlink{VIII.n63}{VIII.n.63}{}; \hyperlink{IX.n17}{IX.n.17}{}; \hyperlink{XIII.n1}{XIII.n.1}{}, \hyperlink{XIII.n3}{n.3}{};
\par\textbf{momentary unification} (\emph{khaṇika-ekaggatā}) \hyperlink{VIII.232}{VIII.232}{}, \hyperlink{VIII.n63}{n.63}{}
\par\textbf{monastery} (\emph{Vihāra}) \hyperlink{I.69}{I.69}{}; \hyperlink{IV.2}{IV.2f.}{}
\par\textbf{moon} (\emph{canda}) \hyperlink{I.n10}{I.n.10}{}; \hyperlink{VII.44}{VII.44}{}; \hyperlink{XII.102}{XII.102}{}; \hyperlink{XIII.46}{XIII.46}{}
\par\textbf{moral-inefficacy-of-action view} (\emph{akiriyadiṭṭhi}) \hyperlink{XVI.85}{XVI.85}{}; \hyperlink{XVII.23}{XVII.23}{}, \hyperlink{XVII.313}{313}{}; \hyperlink{XX.102}{XX.102}{}
\par\textbf{mortification of self} (\emph{atta-kilamatha}) \hyperlink{I.93}{I.93}{}; \hyperlink{II.84}{II.84}{}
\par\textbf{motion} (\emph{calana}) \hyperlink{XIV.n7}{XIV.n27}{}, \hyperlink{XIV.n29}{n.29}{}
\par\textbf{movement} (\emph{gati}) \hyperlink{VIII.n54}{VIII.n.54}{}
\par\textbf{mundane} (\emph{lokiya}) \hyperlink{I.29}{I.29}{}, \hyperlink{I.32}{32}{}, \hyperlink{I.n4}{n.4}{}; \hyperlink{III.5}{III.5}{}, \hyperlink{III.7}{7}{}, \hyperlink{III.n5}{n.5}{}; \hyperlink{XIV.9}{XIV.9}{}, \hyperlink{XIV.202}{202}{}; \hyperlink{XVI.102}{XVI.102}{}; \hyperlink{XVII.120}{XVII.120}{}; \hyperlink{XVIII.8}{XVIII.8f}{}; \hyperlink{XX.43}{XX.43}{}, \hyperlink{XX.130}{130}{}; \hyperlink{XXI.16}{XXI.16}{}; \hyperlink{XXII.39}{XXII.39}{}, \hyperlink{XXII.46}{46}{}, \hyperlink{XXII.124}{124}{}, \hyperlink{XXII.128}{128}{}; \hyperlink{XXIII.2}{XXIII.2}{}, \hyperlink{XXIII.52}{52}{}
\par\textbf{Nāga} \hyperlink{XII.100}{XII.100}{}, \hyperlink{XII.106}{106f.}{}, \hyperlink{XII.137}{137}{}, \hyperlink{XII.n19}{n.19}{}; \hyperlink{XIII.93}{XIII.93}{}; \hyperlink{XXI.43}{XXI.43}{}, \hyperlink{XXI.46}{46}{}
\par\textbf{Nāgapabbata} \hyperlink{IV.36}{IV.36}{}
\par\textbf{Nāga Thera, Mahā} \hyperlink{XXIII.36}{XXIII.36}{}
\par\textbf{Nāga Thera, Karuliyagiri-vāsin} \hyperlink{III.52}{III.52}{}
\par\textbf{Nāga Thera, Mahā, Uccāvālika-vāsin} \hyperlink{XX.110}{XX.110f.}{}
\par\textbf{Nāga Thera, Tipiṭaka Cūḷa} \hyperlink{XII.105}{XII.105}{}; \hyperlink{XXI.n38}{XXI.n.38}{}
\par\textbf{name} (\emph{nāma}) \hyperlink{II.n18}{II.n.18}{}; \hyperlink{VII.54}{VII.54}{}; \hyperlink{VIII.n11}{VIII.n.11}{}; \hyperlink{XIII.123}{XIII.123}{}; \hyperlink{XVIII.n4}{XVIII.n.4}{}; \hyperlink{XXIII.n18}{XXIII.n.18}{}
\par\textbf{naming} (\emph{abhidhāna}) \hyperlink{IV.n18}{IV.n.18}{}; \hyperlink{IX.n6}{IX.n.6}{}
\par\textbf{Nanda} , see Nandopananda
\par\textbf{Nanda the brahman student} (\emph{Nandamāṇava}) \hyperlink{XXII.83}{XXII.83}{}
\par\textbf{Nanda Thera} \hyperlink{XXII.99}{XXII.99}{}
\par\textbf{Nandana Grove} \hyperlink{XIII.79}{XIII.79}{}
\par\textbf{Nandopananda} \hyperlink{IV.133}{IV.133}{}; \hyperlink{XII.106}{XII.106f.}{}
\par\textbf{nature} (\emph{pakati}) \hyperlink{I.38}{I.38}{}; as Universal N., see Primordial Essence; (\emph{rasa}) \hyperlink{I.21}{I.21}{}, see function; (\emph{sabhāva}) \hyperlink{XVI.85}{XVI.85}{}; \hyperlink{XVII.n3}{XVII.n.3}{}
\par\textbf{natural materiality} (\emph{dhammatā-rūpa}) \hyperlink{XX.73}{XX.73}{}
\par\textbf{negligence} (\emph{pamāda}) \hyperlink{I.140}{I.140}{}; \hyperlink{VII.59}{VII.59}{}; \hyperlink{XII.17}{XII.17}{}
\par\textbf{neither-painful-nor-pleasant} (\emph{adukkhamasukha}) \hyperlink{III.n6}{III.n.6}{}; \hyperlink{IV.193}{IV.193}{}; \hyperlink{XIV.200}{XIV.200}{}
\par\textbf{Nemindharapabbata} \hyperlink{VII.42}{VII.42}{}
\par\textbf{Netti} \hyperlink{XXII.n11}{XXII.n.11}{}
\par\textbf{neutral} (\emph{majjhatta}) \hyperlink{IV.163}{IV.163}{}; \hyperlink{IX.88}{IX.88f.}{}, \hyperlink{IX.92}{92}{}, \hyperlink{IX.96}{96}{}; \hyperlink{XIV.200}{XIV.200}{}; \hyperlink{XVI.10}{XVI.10}{}; \hyperlink{XVII.127}{XVII.127}{}
\par\textbf{neutrality, specific} (\emph{tatramajjhattatā}) \hyperlink{IV.116}{IV.116}{}, \hyperlink{IV.156}{156}{}, \hyperlink{IV.164}{164}{}; \hyperlink{XIV.133}{XIV.133}{}, \hyperlink{XIV.153}{153f.}{}
\par\textbf{Nibbāna} (\emph{nibbāna}) \hyperlink{I.5}{I.5}{}, \hyperlink{I.32}{32}{}, \hyperlink{I.140}{140}{}; \hyperlink{III.129}{III.129}{}, \hyperlink{III.n6}{n.6}{}; \hyperlink{VII.33}{VII.33}{}, \hyperlink{VII.74}{74f.}{}; \hyperlink{VIII.235}{VIII.235f.}{}, \hyperlink{VIII.245}{245}{}, \hyperlink{VIII.n65}{n.65}{}, \hyperlink{VIII.n68}{n.68}{}, \hyperlink{VIII.n72}{n.72}{}; \hyperlink{XI.124}{XI.124}{}; \hyperlink{XIV.15}{XIV.15}{}, \hyperlink{XIV.67}{67}{}; \hyperlink{XV.14}{XV.14}{}; \hyperlink{XVI.26}{XVI.26}{}, \hyperlink{XVI.31}{31}{}, \hyperlink{XVI.64}{64f.}{}, \hyperlink{XVI.n6}{n.6}{}, \hyperlink{XVI.n18}{n.18}{}, \hyperlink{XVI.n25}{n.25}{}; \hyperlink{XVII.n16}{XVII.n.16}{}; \hyperlink{XXI.18}{XXI.18}{}, \hyperlink{XXI.37}{37}{}, \hyperlink{XXI.64}{64}{}, \hyperlink{XXI.71}{71}{}, \hyperlink{XXI.106}{106}{}, \hyperlink{XXI.124}{124}{}, \hyperlink{XXI.n33}{n.33}{}; \hyperlink{XXII.5}{XXII.5f.}{}, \hyperlink{XXII.20}{20}{}, \hyperlink{XXII.40}{40}{}, \hyperlink{XXII.44}{44}{}, \hyperlink{XXII.88}{88}{}, \hyperlink{XXII.127}{127}{}, \hyperlink{XXII.n1}{n.1}{}; \hyperlink{XXIII.9}{XXIII.9}{}, \hyperlink{XXIII.30}{30}{}, \hyperlink{XXIII.50}{50}{}, \hyperlink{XXIII.n4}{n.4}{}
\par\textbf{nihilism} (\emph{natthi-vāda}) \hyperlink{XVII.23}{XVII.23}{}
\par\textbf{Nikapenna} \hyperlink{XX.110}{XX.110}{}
\par\textbf{noble} (\emph{ariya}) \hyperlink{VII.n3}{VII.n.3}{}; n. disciple (\emph{ariyasāvaka}) \hyperlink{VII.121}{VII.121}{}; \hyperlink{XX.105}{XX.105}{}; \hyperlink{XXIII.10}{XXIII.10}{}; n. one (\emph{ariya}) \hyperlink{XI.124}{XI.124}{}; \hyperlink{XIII.82}{XIII.82}{}; \hyperlink{XIV.164}{XIV.164}{}; \hyperlink{XVI.20}{XVI.20f.}{}, \hyperlink{XVI.86}{86}{}; \hyperlink{XXIII.6}{XXIII.6}{}, \hyperlink{XXIII.8}{8}{}; n. ones’ heritages (\emph{ariyavaṃsā}) \hyperlink{I.112}{I.112}{}; \hyperlink{II.1}{II.1}{}, \hyperlink{II.28}{28f.}{}; \hyperlink{III.n15}{III.n.15}{}; \hyperlink{XX.78}{XX.78}{}, \hyperlink{XX.83}{83}{}; n. path (\emph{ariya-magga}) \hyperlink{XXI.71}{XXI.71}{}; n. person (\emph{ariya-puggala}) \hyperlink{XXI.74}{XXI.74f.}{}; \hyperlink{XXII.85}{XXII.85}{}
\par\textbf{no-cause view} (\emph{ahetuka-diṭṭhi}) \hyperlink{XVII.22}{XVII.22}{}, \hyperlink{XVII.313}{313}{}; \hyperlink{XXII.112}{XXII.112}{}
\par\textbf{non-becoming} (\emph{vibhava}) \hyperlink{XVI.93}{XVI.93}{}; \hyperlink{XVII.135}{XVII.135}{}
\par\textbf{non-confusion} (\emph{asammoha}) \hyperlink{VIII.226}{VIII.226}{}
\par\textbf{non-covetousness} (\emph{anabhijjhā}) \hyperlink{I.17}{I.17}{}
\par\textbf{non-cruelty} (\emph{avihiṃsā}) \hyperlink{XV.28}{XV.28}{}
\par\textbf{non-delusion} (\emph{amoha}) \hyperlink{II.84}{II.84}{}; \hyperlink{III.128}{III.128}{}; \hyperlink{XIII.77}{XIII.77}{}; \hyperlink{XIV.7}{XIV.7}{}, \hyperlink{XIV.133}{133}{}, \hyperlink{XIV.143}{143}{}, \hyperlink{XIV.156}{156}{}
\par\textbf{non-distraction} (\emph{avikkhepa}) \hyperlink{I.140}{I.140}{}; \hyperlink{III.5}{III.5}{}; \hyperlink{XXII.45}{XXII.45}{}; \hyperlink{XXIII.20}{XXIII.20}{}
\par\textbf{non-existence} (\emph{abhāva}) \hyperlink{III.115}{III.115}{}; \hyperlink{X.45}{X.45}{}. See also absence 
\par\textbf{non-greed} (\emph{alobha}) \hyperlink{II.84}{II.84}{}; \hyperlink{III.128}{III.128}{}; \hyperlink{XIII.77}{XIII.77}{}; \hyperlink{XIV.100}{XIV.100}{}, \hyperlink{XIV.106}{106}{}, \hyperlink{XIV.133}{133}{}, \hyperlink{XIV.143}{143}{}
\par\textbf{non-hate} (\emph{adosa}) \hyperlink{III.128}{III.128}{}; \hyperlink{XIII.77}{XIII.77}{}; \hyperlink{XIV.133}{XIV.133}{}, \hyperlink{XIV.143}{143}{}, \hyperlink{XIV.154}{154}{}
\par\textbf{non-human} (\emph{amanussa}) \hyperlink{II.65}{II.65}{}, \hyperlink{II.70}{70}{}; \hyperlink{VI.26}{VI.26}{}
\par\textbf{non-ill-will} (\emph{avyāpāda}) \hyperlink{I.17}{I.17}{}, \hyperlink{I.140}{140}{}; \hyperlink{XXIII.20}{XXIII.20}{}
\par\textbf{non-owning} (\emph{akiñcana}) \hyperlink{X.39}{X.39}{}
\par\textbf{non-percipient} (\emph{asañña, asaññin}) \hyperlink{XVII.134}{XVII.134}{}, \hyperlink{XVII.192}{192}{}, \hyperlink{XVII.201}{201}{}
\par\textbf{non-reflection} (\emph{appaṭisaṅkhā}) \hyperlink{I.140}{I.140}{}
\par\textbf{non-remorse} (\emph{avippaṭisāra}) \hyperlink{I.23}{I.23}{}, \hyperlink{I.32}{32}{}, \hyperlink{I.140}{140}{}
\par\textbf{non-returner} (\emph{anāgāmin}) \hyperlink{I.14}{I.14}{}, \hyperlink{I.140}{140}{}; \hyperlink{III.128}{III.128}{}; \hyperlink{XIV.206}{XIV.206}{}; \hyperlink{XXII.2}{XXII.2f.}{}, \hyperlink{XXII.21}{21}{}, \hyperlink{XXII.28}{28f.}{}, \hyperlink{XXII.45}{45}{}; \hyperlink{XXIII.7}{XXIII.7}{}, \hyperlink{XXIII.18}{18}{}, \hyperlink{XXIII.25}{25}{}, \hyperlink{XXIII.28}{28}{}, \hyperlink{XXIII.56}{56f.}{}
\par\textbf{non-trainer} (\emph{asekha}) \hyperlink{I.35}{I.35}{}, \hyperlink{I.37}{37}{}; \hyperlink{XIV.27}{XIV.27}{}; \hyperlink{XVI.104}{XVI.104}{}
\par\textbf{non-transgression} (\emph{avītikkama}) \hyperlink{I.17}{I.17}{}, \hyperlink{I.41}{41}{}, \hyperlink{I.140}{140}{}
\par\textbf{non-wavering} (\emph{avikampana}) \hyperlink{III.4}{III.4}{}; \hyperlink{XXIII.20}{XXIII.20}{}
\par\textbf{nose} (\emph{ghāna}) \hyperlink{XIV.36}{XIV.36}{}, \hyperlink{XIV.39}{39}{}, \hyperlink{XIV.46}{46}{}, \hyperlink{XIV.50}{50}{}, \hyperlink{XIV.117}{117}{}; \hyperlink{XVI.10}{XVI.10}{}; \hyperlink{XX.70}{XX.70}{}; n. base (\emph{ghānāyatana}) \hyperlink{XV.3}{XV.3f.}{}; n. consciousness (\emph{ghānaviññāṇa}) \hyperlink{XIV.95}{XIV.95f.}{}, \hyperlink{XIV.117}{117}{}, \hyperlink{XIV.179}{179}{}; n.-c. element (\emph{ghāṇaviññāṇadhātu}) \hyperlink{XV.17}{XV.17f.}{}; n. faculty (\emph{ghānindriya}) \hyperlink{XVI.1}{XVI.1}{}
\par\textbf{noseless} (\emph{aghānaka}) \hyperlink{XVII.157}{XVII.157}{}
\par\textbf{not-self} (\emph{anattā}) \hyperlink{I.140}{I.140}{}; \hyperlink{VII.n7}{VII.n.7}{}; \hyperlink{XI.104}{XI.104}{}; \hyperlink{XIV.3}{XIV.3}{}, \hyperlink{XIV.224}{224}{}; \hyperlink{XVI.99}{XVI.99}{}, \hyperlink{XVI.n25}{n.25}{}; \hyperlink{XIX.26}{XIX.26}{}; \hyperlink{XX}{XX passim}{}; \hyperlink{XXI.3}{XXI.3f.}{}, \hyperlink{XXI.48}{48}{}, \hyperlink{XXI.51}{51}{}, \hyperlink{XXI.59}{59}{}, \hyperlink{XXI.70}{70}{}, \hyperlink{XXI.88}{88}{}, \hyperlink{XXI.129}{129}{}, \hyperlink{XXI.n3}{n.3}{}; \hyperlink{XXII.22}{XXII.22}{}, \hyperlink{XXII.53}{53}{}; \hyperlink{XXIII.32}{XXIII.32}{}. See also contemplation of n.-s. 
\par\textbf{nothingness} (\emph{ākiñcañña}) \hyperlink{X.32}{X.32}{}
\par\textbf{not-so-classifiable} (\emph{na-vattabba}) \hyperlink{III.n32}{III.n.32}{}; \hyperlink{XVII.134}{XVII.134}{}
\par\textbf{novice} (\emph{sāmaṇera}) \hyperlink{I.40}{I.40}{}; \hyperlink{II.92}{II.92}{}
\par\textbf{nutriment} (\emph{āhāra}) \hyperlink{I.89}{I.89}{}; \hyperlink{IV.52}{IV.52}{}, \hyperlink{IV.63}{63}{}; \hyperlink{VII.37}{VII.37f.}{}, \hyperlink{VII.n13}{n.13}{}; \hyperlink{VIII.27}{VIII.27f.}{}; \hyperlink{XI.1}{XI.1f.}{}, \hyperlink{XI.111}{111}{}; \hyperlink{XIV.47}{XIV.47}{}, \hyperlink{XIV.75}{75}{}, \hyperlink{XIV.79}{79}{}, \hyperlink{XIV.188}{188}{}, \hyperlink{XIV.226}{226f.}{}; \hyperlink{XVI.92}{XVI.92}{}; \hyperlink{XVII.66}{XVII.66}{}, \hyperlink{XVII.90}{90}{}, \hyperlink{XVII.194}{194}{}; \hyperlink{XVIII.5}{XVIII.5}{}; \hyperlink{XIX.9}{XIX.9}{}; \hyperlink{XX.27}{XX.27}{}, \hyperlink{XX.68}{68}{}, \hyperlink{XX.97}{97}{}; n. originated (\emph{āhārasamuṭṭhāna}) \hyperlink{XI.111}{XI.111}{}; \hyperlink{XVII.194}{XVII.194}{}; \hyperlink{XIX.9}{XIX.9}{}; \hyperlink{XX.29}{XX.29}{}, \hyperlink{XX.35}{35f.}{}
\par\textbf{nutritive essence} (\emph{ojā}) \hyperlink{XI.2}{XI.2}{}, \hyperlink{XI.88}{88}{}; \hyperlink{XIV.70}{XIV.70}{}; \hyperlink{XVII.256}{XVII.256}{}; \hyperlink{XVIII.5}{XVIII.5f.}{}; \hyperlink{XX.29}{XX.29}{}
\par\textbf{object} (\emph{ārammaṇa}) \hyperlink{I.2}{I.2}{}, \hyperlink{I.53}{53}{}, \hyperlink{I.57}{57}{}; \hyperlink{III.3}{III.3}{}, \hyperlink{III.5}{5}{}, \hyperlink{III.20}{20}{}, \hyperlink{III.108}{108}{}, \hyperlink{III.112}{112}{}; \hyperlink{IV.74}{IV.74}{}; \hyperlink{VIII.40}{VIII.40}{}, \hyperlink{VIII.226}{226f.}{}, \hyperlink{VIII.236}{236}{}; \hyperlink{IX.102}{IX.102}{}; \hyperlink{X.15}{X.15}{}, \hyperlink{X.28}{28}{}; \hyperlink{XIII.2}{XIII.2}{}, \hyperlink{XIII.73}{73}{}, \hyperlink{XIII.122}{122}{}; \hyperlink{XIV.15}{XIV.15}{}, \hyperlink{XIV.96}{96}{}, \hyperlink{XIV.111}{111f.}{}, \hyperlink{XIV.128}{128}{}, \hyperlink{XIV.139}{139}{}, \hyperlink{XIV.147}{147}{}, \hyperlink{XIV.150}{150}{}, \hyperlink{XIV.163}{163}{}, \hyperlink{XIV.201}{201}{}; \hyperlink{XV.4}{XV.4}{}; \hyperlink{XVI.104}{XVI.104}{}; \hyperlink{XVII.52}{XVII.52}{}, \hyperlink{XVII.66}{66}{}, \hyperlink{XVII.71}{71}{}, \hyperlink{XVII.127}{127}{}, \hyperlink{XVII.134}{134f.}{}; \hyperlink{XVIII.17}{XVIII.17}{}, \hyperlink{XVIII.21}{21}{}, \hyperlink{XVIII.n4}{n.4}{}; \hyperlink{XIX.8}{XIX.8}{}; \hyperlink{XX.9}{XX.9}{}, \hyperlink{XX.43}{43}{}; \hyperlink{XXI.n3}{XXI.n.3}{}; \hyperlink{XXII.4}{XXII.4}{}, \hyperlink{XXII.20}{20}{}, \hyperlink{XXII.44}{44}{}, \hyperlink{XXII.82}{82}{}, \hyperlink{XXII.89}{89}{}, \hyperlink{XXII.118}{118}{}; \hyperlink{XXIII.10}{XXIII.10}{}; o. triad (\emph{ārammaṇa-ttika}) \hyperlink{XIII.104}{XIII.104f.}{}
\par\textbf{objective basis} (\emph{vatthu}) \hyperlink{XXI.83}{XXI.83}{}; o. field (\emph{visaya}) \hyperlink{VII.n7}{VII.n.7}{}; \hyperlink{XIV.46}{XIV.46}{}, \hyperlink{XIV.54}{54}{}, \hyperlink{XIV.76}{76}{}, \hyperlink{XIV.130}{130}{}, \hyperlink{XIV.134}{134}{}, \hyperlink{XIV.197}{197}{}, \hyperlink{XIV.213}{213}{}; \hyperlink{XV.11}{XV.11}{}; \hyperlink{XVII.51}{XVII.51}{}, \hyperlink{XVII.163}{163}{}; \hyperlink{XVII.11}{XVII.11}{}
\par\textbf{obsession} (\emph{pariyuṭṭhāna}) \hyperlink{I.13}{I.13}{}; \hyperlink{VII.65}{VII.65}{}; \hyperlink{XVI.85}{XVI.85}{}
\par\textbf{occurrence} (\emph{pavatta, pavatti}) \hyperlink{XVI.23}{XVI.23}{}, \hyperlink{XVI.28}{28}{}, \hyperlink{XVI.92}{92}{}; \hyperlink{XIX.26}{XIX.26}{}; \hyperlink{XX.40}{XX.40}{}; \hyperlink{XXI.27}{XXI.27}{}, \hyperlink{XXI.33}{33f.}{}, \hyperlink{XXI.37}{37}{}, \hyperlink{XXI.51}{51f.}{}, \hyperlink{XXI.80}{80}{}, \hyperlink{XXI.83}{83}{}; \hyperlink{XXII.4}{XXII.4f.}{}, \hyperlink{XXII.44}{44}{}; \hyperlink{XXIII.7}{XXIII.7}{}
\par\textbf{octad} (\emph{aṭṭhaka, aṭṭhamaka}) \hyperlink{XI.2}{XI.2}{}, \hyperlink{XI.88}{88}{}; \hyperlink{XVII.193}{XVII.193}{}; \hyperlink{XVIII.5}{XVIII.5f.}{}
\par\textbf{odour} (\emph{gandha}) \hyperlink{XI.86}{XI.86}{}; \hyperlink{XIV.56}{XIV.56}{}; \hyperlink{XVII.156}{XVII.156}{}; \hyperlink{XVIII.5}{XVIII.5}{}, \hyperlink{XVIII.11}{11}{}; o. base (\emph{gandhāyatana}) \hyperlink{XV.2}{XV.2}{}; o. element (\emph{gandha-dhātu}) \hyperlink{XV.17}{XV.17}{}
\par\textbf{offence} (\emph{āpatti}) \hyperlink{I.60}{I.60}{}, \hyperlink{I.125}{125}{}; \hyperlink{IV.3}{IV.3}{}
\par\textbf{ogre} (\emph{rakkhasa}) \hyperlink{XIII.100}{XIII.100}{}
\par\textbf{omnipotent being} (\emph{vasavatti}), see powerwielder 
\par\textbf{omniscience} (\emph{sabbaññutā}) \hyperlink{VI.32}{VI.32}{}; \hyperlink{VII.n7}{VII.n.7}{}; \hyperlink{XII.121}{XII.121}{}
\par\textbf{once-returner} (\emph{sakadāgāmin}) \hyperlink{I.14}{I.14}{}, \hyperlink{I.140}{140}{}; \hyperlink{III.128}{III.128}{}; \hyperlink{XIV.206}{XIV.206}{}; \hyperlink{XXII.2}{XXII.2}{}, \hyperlink{XXII.23}{23}{}, \hyperlink{XXII.45}{45}{}; \hyperlink{XXIII.7}{XXIII.7}{}, \hyperlink{XXIII.18}{18}{}, \hyperlink{XXIII.25}{25}{}, \hyperlink{XXIII.55}{55}{}
\par\textbf{one defining, the} (\emph{eka-vavatthāna}) \hyperlink{II.105}{II.105}{}
\par\textbf{one-pointedness} (\emph{ekaggatā}), see unification
\par\textbf{one-sessioner} (\emph{ekāsanika}) \hyperlink{II.2}{II.2}{}, \hyperlink{II.35}{35}{}
\par\textbf{open-air dweller} (\emph{abbhokāsika}) \hyperlink{I.68}{I.68}{}; \hyperlink{II.2}{II.2}{}, \hyperlink{II.60}{60}{}
\par\textbf{opposites, substitution of} (\emph{tadaṅga}), see substitution of o. 
\par\textbf{order} (\emph{kama}) \hyperlink{XIV.211}{XIV.211}{}; o. of bhikkhus (\emph{bhikkhu-saṅgha}), see Community; o. of beings (\emph{satta-nikāya}) \hyperlink{XIII.69}{XIII.69}{}
\par\textbf{ordinary man} (\emph{puthujjana}) \hyperlink{I.35}{I.35}{}, \hyperlink{I.135}{135}{}, \hyperlink{I.137}{137}{}; \hyperlink{II.78}{II.78}{}; \hyperlink{III.56}{III.56}{}; \hyperlink{XI.121}{XI.121}{}; \hyperlink{XIII.110}{XIII.110}{}; \hyperlink{XIV.109}{XIV.109}{}, \hyperlink{XIV.202}{202}{}; \hyperlink{XV.42}{XV.42}{}; \hyperlink{XVI.67}{XVI.67}{}; \hyperlink{XVII.39}{XVII.39}{}, \hyperlink{XVII.261}{261}{}; \hyperlink{XXII.5}{XXII.5}{}, \hyperlink{XXII.85}{85}{}; \hyperlink{XXIII.6}{XXIII.6}{}, \hyperlink{XXIII.18}{18}{}
\par\textbf{organic continuity} (\emph{santati-sīsa}) \hyperlink{XVII.189}{XVII.189f.}{}
\par\textbf{origin, origination} (\emph{samudaya}) \hyperlink{VII.27}{VII.27}{}; \hyperlink{XVI.13}{XVI.13}{}, \hyperlink{XVI.61}{61}{}; \hyperlink{XX.90}{XX.90}{}, \hyperlink{XX.100}{100}{}, \hyperlink{XX.130}{130}{}; \hyperlink{XXII.44}{XXII.44}{}, \hyperlink{XXII.92}{92}{}, \hyperlink{XXII.113}{113}{}; o. of a sutta (\emph{uppatti}) \hyperlink{III.88}{III.88}{}; \hyperlink{VII.69}{VII.69}{}; \hyperlink{XXIII.n166}{XXIII.n.166}{}
\par\textbf{origination} (\emph{samuṭṭhāna}) \hyperlink{XI.94}{XI.94}{}; \hyperlink{XVII.196}{XVII.196}{}; \hyperlink{XX.30}{XX.30f.}{}
\par\textbf{or-whatever-state} (\emph{ye-vā-panaka-dhamma}) \hyperlink{XIV.133}{XIV.133f.}{}
\par\textbf{outlet} (\emph{niyyāna}) \hyperlink{XVI.15}{XVI.15}{}, \hyperlink{XVI.25}{25}{}; \hyperlink{XXI.67}{XXI.67}{}; \hyperlink{XXII.38}{XXII.38}{}, \hyperlink{XXII.97}{97}{}
\par\textbf{over-generalization} (\emph{atippasaṅga}) \hyperlink{XIV.186}{XIV.186}{}
\par\textbf{Overlord} (\emph{issara}) \hyperlink{XVI.30}{XVI.30}{}, \hyperlink{XVI.85}{85}{}, \hyperlink{XVI.n23}{n.23}{}; \hyperlink{XVII.50}{XVII.50}{}, \hyperlink{XVII.117}{117}{}; \hyperlink{XIX.3}{XIX.3}{}; \hyperlink{XXI.48}{XXI.48}{}, \hyperlink{XXI.57}{57}{}, \hyperlink{XXI.n3}{n.3}{}; \hyperlink{XXII.119}{XXII.119}{}
\par\textbf{owning} (\emph{kiñcana}) \hyperlink{X.39}{X.39}{}, \hyperlink{X.n9}{n.9}{}; \hyperlink{XXI.54}{XXI.54}{}, \hyperlink{XXI.n19}{n.19}{}
\par\textbf{ox-asceticism} (\emph{go-sīla}) \hyperlink{XVII.241}{XVII.241}{}; o. practice (\emph{go-kiriyā}) \hyperlink{XVII.246}{XVII.246}{}
\par\textbf{Paccekabuddha} (Undeclared Enlightened One) \hyperlink{I.131}{I.131}{}; \hyperlink{III.128}{III.128}{}; \hyperlink{IV.55}{IV.55}{}; \hyperlink{VIII.22}{VIII.22}{}, \hyperlink{VIII.155}{155}{}, \hyperlink{VIII.211}{211}{}; \hyperlink{XI.17}{XI.17}{}; \hyperlink{XII.11}{XII.11}{}; \hyperlink{XIII.16}{XIII.16}{}; \hyperlink{XIV.31}{XIV.31}{}; \hyperlink{XXIII.11}{XXIII.11}{}
\par\textbf{Pācīnakhaṇḍārājī} \hyperlink{III.31}{III.31}{}
\par\textbf{pain, painful} (\emph{dukkha}) \hyperlink{I.140}{I.140}{}; \hyperlink{IV.184}{IV.184}{}; \hyperlink{IX.123}{IX.123}{}; \hyperlink{XI.104}{XI.104}{}; \hyperlink{XIV.3}{XIV.3}{}, \hyperlink{XIV.102}{102}{}, \hyperlink{XIV.127}{127f.}{}, \hyperlink{XIV.200}{200}{}, \hyperlink{XIV.220}{220}{}; \hyperlink{XV.28}{XV.28}{}; \hyperlink{XVI.1}{XVI.1}{}, \hyperlink{XVI.10}{10}{}, \hyperlink{XVI.31}{31}{}, \hyperlink{XVI.50}{50}{}, \hyperlink{XVI.99}{99}{}; \hyperlink{XVII.2}{XVII.2}{}, \hyperlink{XVII.48}{48}{}; \hyperlink{XIX.26}{XIX.26}{}; \hyperlink{XX}{XX passim}{}; \hyperlink{XXI.3}{XXI.3f.}{}, \hyperlink{XXI.34}{34}{}, \hyperlink{XXI.48}{48}{}, \hyperlink{XXI.51}{51}{}, \hyperlink{XXI.59}{59}{}, \hyperlink{XXI.69}{69}{}, \hyperlink{XXI.88}{88}{}, \hyperlink{XXI.129}{129}{}, \hyperlink{XXI.n3}{n.3}{}; \hyperlink{XXII.22}{XXII.22}{}, \hyperlink{XXII.53}{53}{}; \hyperlink{XXIII.32}{XXIII.32}{}. See also contemplation of p., \& suffering 
\par\textbf{Pañcasikhā} (Five-crest) \hyperlink{XII.79}{XII.79}{}
\par\textbf{Paṇḍukambala-silā} \hyperlink{XII.72}{XII.72}{}
\par\textbf{Panthaka Thera, Cūḷa} \hyperlink{XII.59}{XII.59f.}{}
\par\textbf{Panthaka Thera, Mahā} \hyperlink{XII.60}{XII.60}{}
\par\textbf{Paranimmitavasavatti Deities} (Who Wield Power Over Others’ Creations) \hyperlink{VII.n14}{VII.n.14}{}; \hyperlink{XIV.207}{XIV.207}{}; \hyperlink{XV.27}{XV.27}{}
\par\textbf{Pāricchattaka Tree} \hyperlink{VII.43}{VII.43}{}
\par\textbf{Pāṭaliputta} (Patna) \hyperlink{IX.64}{IX.64}{}; \hyperlink{XII.123}{XII.123}{}
\par\textbf{past} (\emph{atīta}) \hyperlink{XIV.185}{XIV.185f.}{}
\par\textbf{path} (\emph{magga}) \hyperlink{I.137}{I.137}{}; \hyperlink{III.7}{III.7}{}, \hyperlink{III.13}{13}{}; \hyperlink{IV.78}{IV.78}{}; \hyperlink{VII.27}{VII.27}{}, \hyperlink{VII.33}{33}{}, \hyperlink{VII.76}{76}{}, \hyperlink{VII.91}{91}{}; \hyperlink{VIII.224}{VIII.224}{}; \hyperlink{XIII.83}{XIII.83}{}, \hyperlink{XIII.120}{120}{}; \hyperlink{XIV.3}{XIV.3}{}, \hyperlink{XIV.9}{9}{}, \hyperlink{XIV.23}{23}{}, \hyperlink{XIV.105}{105}{}, \hyperlink{XIV.121}{121}{}, \hyperlink{XIV.158}{158}{}, \hyperlink{XIV.206}{206}{}, \hyperlink{XIV.n67}{n.67}{}; \hyperlink{XVI.26}{XVI.26}{}, \hyperlink{XVI.68}{68}{}; \hyperlink{XVII.62}{XVII.62}{}, \hyperlink{XVII.66}{66}{}, \hyperlink{XVII.81}{81}{}, \hyperlink{XVII.93}{93}{}; \hyperlink{XX.100}{XX.100}{}, \hyperlink{XX.107}{107}{}, \hyperlink{XX.130}{130}{}; \hyperlink{XXI.71}{XXI.71}{}, \hyperlink{XXI.83}{83}{}, \hyperlink{XXI.85}{85}{}, \hyperlink{XXI.111}{111}{}, \hyperlink{XXI.116}{116}{}, \hyperlink{XXI.120}{120}{}, \hyperlink{XXI.129}{129}{}, \hyperlink{XXI.n34}{n.34}{}; \hyperlink{XXII.2}{XXII.2f.}{}, \hyperlink{XXII.33}{33}{}, \hyperlink{XXII.42}{42}{}, \hyperlink{XXII.78}{78}{}; \hyperlink{XXIII.3}{XXIII.3}{}, \hyperlink{XXIII.14}{14}{}, \hyperlink{XXIII.33}{33}{}
\par\textbf{patience} (\emph{khanti}) \hyperlink{I.18}{I.18}{}; \hyperlink{IX.2}{IX.2}{}, \hyperlink{IX.124}{124}{}; \hyperlink{XXI.28}{XXI.28}{}
\par\textbf{Pātimokkha} (\emph{pātimokkha}) \hyperlink{I.18}{I.18}{}, \hyperlink{I.42}{42f.}{}, \hyperlink{I.98}{98}{}, \hyperlink{I.126}{126}{}, \hyperlink{I.n10}{n.10}{}, \hyperlink{I.n11}{n.11}{}
\par\textbf{Pavāraṇā} (\emph{pavāraṇā}) \hyperlink{III.n10}{III.n.10}{}
\par\textbf{peace} (\emph{upasama}) \hyperlink{I.140}{I.140}{}; \hyperlink{III.105}{III.105}{}; \hyperlink{VII.1}{VII.1}{}; \hyperlink{VIII.245}{VIII.245f.}{}
\par\textbf{penetration} (\emph{nibbedha}) \hyperlink{I.39}{I.39}{}; \hyperlink{III.32}{III.32}{}
\par\textbf{penetration to} (\emph{abhisamaya}) \hyperlink{XVI.15}{XVI.15}{}; \hyperlink{XXII.79}{XXII.79}{}, \hyperlink{XXII.92}{92}{}. See also convergence 
\par\textbf{penetration of minds} (\emph{cetopariya}), see knowledge of p. 
\par\textbf{perception} (\emph{saññā}) \hyperlink{I.55}{I.55}{}, \hyperlink{I.140}{140}{}; \hyperlink{III.22}{III.22}{}, \hyperlink{III.26}{26}{}; \hyperlink{VII.28}{VII.28}{}, \hyperlink{VII.59}{59}{}; \hyperlink{VIII.216}{VIII.216}{}, \hyperlink{VIII.230}{230}{}, \hyperlink{VIII.233}{233}{}; \hyperlink{X.12}{X.12}{}, \hyperlink{X.50}{50f.}{}; \hyperlink{XI.n1}{XI.n.1}{}; \hyperlink{XII.49}{XII.49}{}; \hyperlink{XIV.3}{XIV.3}{}, \hyperlink{XIV.129}{129f.}{}, \hyperlink{XIV.141}{141}{}, \hyperlink{XIV.213}{213f.}{}, \hyperlink{XIV.218}{218}{}, \hyperlink{XIV.224}{224}{}; \hyperlink{XV.14}{XV.14}{}, \hyperlink{XV.42}{42}{}; \hyperlink{XVIII.8}{XVIII.8}{}, \hyperlink{XVIII.13}{13}{}, \hyperlink{XVIII.20}{20}{}; \hyperlink{XX.6}{XX.6}{}, \hyperlink{XX.9}{9}{}, \hyperlink{XX.94}{94}{}; \hyperlink{XXI.11}{XXI.11}{}, \hyperlink{XXI.56}{56}{}; \hyperlink{XXII.53}{XXII.53}{}, \hyperlink{XXII.126}{126}{}; \hyperlink{XXIII.13}{XXIII.13}{}; p. of beauty (\emph{subha-saññā}) \hyperlink{XXII.34}{XXII.34}{}; p. of bliss (\emph{sukha-s.}) \hyperlink{XII.119}{XII.119}{}, p. of compactness (\emph{ghana-s.}) \hyperlink{I.140}{I.140}{}; \hyperlink{XX.90}{XX.90}{}; \hyperlink{XXII.113}{XXII.113}{}, \hyperlink{XXII.114}{114}{}; p. of foulness (\emph{asubha-s.}) \hyperlink{III.57}{III.57}{}; \hyperlink{VI}{VI passim}{}; \hyperlink{XI.26}{XI.26}{}; p. of impermanence (\emph{aniccā.}) \hyperlink{III.122}{III.122}{}; p. of lastingness (\emph{dhuva-s.}) \hyperlink{I.140}{I.140}{}; \hyperlink{XX.90}{XX.90}{}; \hyperlink{XXII.113}{XXII.113}{}, \hyperlink{XXII.116}{116}{}; p. of light (\emph{āloka-s.}) \hyperlink{I.140}{I.140}{}; \hyperlink{XXIII.20}{XXIII.20}{}; p. of lightness (\emph{lahu-s.}) \hyperlink{XII.119}{XII.119}{}; p. of a living being (\emph{satta-s.}) \hyperlink{XX.82}{XX.82}{}; \hyperlink{XXI.122}{XXI.122}{}; p. of permanence (\emph{nicca-s.}) \hyperlink{I.140}{I.140}{}; \hyperlink{VIII.233}{VIII.233}{}; \hyperlink{XX.90}{XX.90}{}; \hyperlink{XXI.11}{XXI.11}{}; \hyperlink{XXII.34}{XXII.34}{}, \hyperlink{XXII.113}{113}{}; \hyperlink{XXIII.23}{XXIII.23}{}; p. of pleasure (\emph{sukha-s.}) \hyperlink{I.140}{I.140}{}; \hyperlink{VIII.233}{VIII.233}{}; \hyperlink{XX.90}{XX.90}{}; \hyperlink{XXI.11}{XXI.11}{}, \hyperlink{XXI.122}{122}{}; \hyperlink{XXII.34}{XXII.34}{}, \hyperlink{XXII.113}{113}{}; \hyperlink{XXIII.23}{XXIII.23}{}; p. of repulsiveness in nutriment (\emph{āhāre paṭikkūla-s.}) \hyperlink{III.6}{III.6}{}; \hyperlink{XI.4}{XI.4f.}{}; p. of self (\emph{atta-s}.) \hyperlink{I.140}{I.140}{}; \hyperlink{VIII.233}{VIII.233}{}; \hyperlink{XX.90}{XX.90}{}; \hyperlink{XXI.11}{XXI.11}{}, \hyperlink{XXI.122}{122}{}; \hyperlink{XXII.34}{XXII.34}{}, \hyperlink{XXII.113}{113}{}; \hyperlink{XXIII.23}{XXIII.23}{}; p., fictitious (\emph{visama-s.}) \hyperlink{VII.59}{VII.59}{}
\par\textbf{perdition} (\emph{vinipāta}) \hyperlink{XIII.92}{XIII.92}{}
\par\textbf{Perfect One} (\emph{tathāgata}) \hyperlink{VII.n10}{VII.n.10}{}
\par\textbf{perfection} (\emph{pāramī, pāramitā}) \hyperlink{I.33}{I.33}{}; \hyperlink{VII.34}{VII.34}{}; \hyperlink{IX.124}{IX.124}{}
\par\textbf{performedness of kamma} (\emph{kaṭatta}) \hyperlink{XVII.89}{XVII.89}{}, \hyperlink{XVII.122}{122}{}, \hyperlink{XVII.174}{174}{}
\par\textbf{period} (\emph{samaya}) \hyperlink{XIV.186}{XIV.186}{}, \hyperlink{XIV.189}{189}{}
\par\textbf{permanent} (\emph{nicca}) \hyperlink{I.140}{I.140}{}; \hyperlink{XXI.56}{XXI.56}{}. See also perception of p. 
\par\textbf{person} (\emph{puggala}) \hyperlink{I.52}{I.52}{}; \hyperlink{IX.93}{IX.93}{}; \hyperlink{XI.30}{XI.30}{}; \hyperlink{XIV.201}{XIV.201}{}. See also noble p.; in terms of p. (\emph{puggalādhiṭṭhāna}) \hyperlink{IV.92}{IV.92}{}
\par\textbf{personality} (\emph{sakkāya}), see individuality
\par\textbf{perverseness} (\emph{vipariyesa}) \hyperlink{VII.59}{VII.59}{}
\par\textbf{perversion} (\emph{vipallāsa}) \hyperlink{XIV.226}{XIV.226f.}{}; \hyperlink{XVII.63}{XVII.63}{}; \hyperlink{XXII.53}{XXII.53}{}, \hyperlink{XXII.68}{68}{}
\par\textbf{Phārusaka Grove} \hyperlink{XII.79}{XII.79}{}
\par\textbf{phenomenon} (\emph{dhamma}) \hyperlink{VII.n1}{VII.n.1}{}; \hyperlink{XIX.20}{XIX.20}{}
\par\textbf{phlegm} (\emph{semha}) \hyperlink{VIII.128}{VIII.128}{}; \hyperlink{IX.17}{IX.17}{}, \hyperlink{IX.70}{70}{}; \hyperlink{XIII.2}{XIII.2}{}, \hyperlink{XIII.73}{73}{}; \hyperlink{XVII.16}{XVII.16}{}
\par\textbf{Phussa-Deva Thera} \hyperlink{VII.128}{VII.128}{}
\par\textbf{Phussa Mittā} \hyperlink{XII.39}{XII.39}{}
\par\textbf{physical} (\emph{karaja}) \hyperlink{X.2}{X.2}{}; \hyperlink{XII.131}{XII.131}{}
\par\textbf{physical basis} (\emph{vatthu}), see basis physical nutriment (\emph{kabaliṅkārāhāra}) \hyperlink{XI.1}{XI.1}{}; \hyperlink{XIV.70}{XIV.70}{}, \hyperlink{XIV.226}{226}{}
\par\textbf{pisāca} (\emph{goblin}) \hyperlink{X.4}{X.4}{}
\par\textbf{Piyaṅkara-mātar} \hyperlink{XII.39}{XII.39}{}
\par\textbf{plane} (\emph{bhūmi}) \hyperlink{XIV.83}{XIV.83}{}, \hyperlink{XIV.206}{206}{}
\par\textbf{pleasure, pleasant} (\emph{sukha}) \hyperlink{I.140}{I.140}{}; \hyperlink{III.n6}{III.n.6}{}; \hyperlink{IV.184}{IV.184}{}; \hyperlink{VIII.233}{VIII.233}{}; \hyperlink{IX.123}{IX.123}{}; \hyperlink{XIV.99}{XIV.99}{}, \hyperlink{XIV.102}{102}{}, \hyperlink{XIV.127}{127}{}, \hyperlink{XIV.200}{200}{}, \hyperlink{XIV.227}{227}{}; \hyperlink{XV.28}{XV.28}{}, \hyperlink{XV.40}{40}{}; \hyperlink{XVI.1}{XVI.1}{}, \hyperlink{XVI.16}{16}{}, \hyperlink{XVI.85}{85}{}, \hyperlink{XVI.90}{90}{}; \hyperlink{XVII.282}{XVII.282}{}; \hyperlink{XX.86}{XX.86}{}; \hyperlink{XXI.56}{XXI.56}{}; \hyperlink{XXII.53}{XXII.53}{}, \hyperlink{XXII.117}{117}{}. See also bliss 
\par\textbf{posture} (\emph{iriyāpatha}) \hyperlink{I.94}{I.94}{}; \hyperlink{III.88}{III.88f.}{}, \hyperlink{III.97}{97f.}{}; \hyperlink{IV.41}{IV.41}{}; \hyperlink{VIII.27}{VIII.27f.}{}, \hyperlink{VIII.43}{43}{}, \hyperlink{VIII.159}{159}{}; \hyperlink{XI.92}{XI.92}{}, \hyperlink{XI.107}{107}{}; \hyperlink{XX.31}{XX.31}{}; \hyperlink{XXI.4}{XXI.4}{}
\par\textbf{power} (\emph{bala}) \hyperlink{IX.124}{IX.124}{}; \hyperlink{XVI.86}{XVI.86}{}; \hyperlink{XXII.25}{XXII.25}{}, \hyperlink{XXII.42}{42}{}; \hyperlink{XXIII.20}{XXIII.20}{}; (\emph{iddhi}) \hyperlink{XII.20}{XII.20f.}{}; \hyperlink{XXII.36}{XXII.36}{}. See also supernormal p.; p. of the Perfect One (\emph{tathāgatabala}) \hyperlink{XIV.31}{XIV.31}{}
\par\textbf{precept} (\emph{sīla}) \hyperlink{I.n10}{I.n.10}{}. See also training p.
\par\textbf{preceptor} (\emph{upajjhāya}) \hyperlink{III.48}{III.48}{}
\par\textbf{predominance} (\emph{adhipati}) \hyperlink{III.5}{III.5}{}, \hyperlink{III.24}{24}{}; \hyperlink{XXI.119}{XXI.119}{}, \hyperlink{XXI.n39}{n.39}{}; \hyperlink{XXII.37}{XXII.37}{}. See also dominance 
\par\textbf{preparatory task} (\emph{pubba-kicca}) \hyperlink{III.16}{III.16}{}; \hyperlink{XXIII.31}{XXIII.31f.}{}
\par\textbf{preliminary work} (\emph{parikamma}) \hyperlink{III.6}{III.6}{}; \hyperlink{IV.25}{IV.25}{}, \hyperlink{IV.74}{74}{}; \hyperlink{XII.46}{XII.46}{}, \hyperlink{XII.59}{59}{}; \hyperlink{XIII.4}{XIII.4f.}{}, \hyperlink{XIII.9}{9}{}, \hyperlink{XIII.23}{23}{}, \hyperlink{XIII.98}{98}{}; \hyperlink{XXI.129}{XXI.129}{}
\par\textbf{presence} (\emph{ṭhāna}) \hyperlink{XX.31}{XX.31}{}; (\emph{ṭhiti}) \hyperlink{IV.n33}{IV.n.33}{}; \hyperlink{VIII.242}{VIII.242}{}; \hyperlink{XIII.111}{XIII.111f.}{}; \hyperlink{XVII.68}{XVII.68}{}, \hyperlink{XVII.193}{193}{}; \hyperlink{XIX.9}{XIX.9}{}; \hyperlink{XX.25}{XX.25f.}{}, \hyperlink{XX.47}{47}{}; \hyperlink{XXI.10}{XXI.10}{}, \hyperlink{XXI.27}{27}{}, \hyperlink{XXI.n6}{n.6}{}; p., moment of (\emph{atthi-kkhaṇa}) \hyperlink{XIV.59}{XIV.59}{}
\par\textbf{present} (\emph{paccuppanna}) \hyperlink{XIII.111}{XIII.111f.}{}; \hyperlink{XIV.186}{XIV.186f.}{}
\par\textbf{pride} (\emph{māna}) \hyperlink{I.151}{I.151}{}; \hyperlink{III.78}{III.78}{}, \hyperlink{III.95}{95}{}, \hyperlink{III.n18}{n.18}{}; \hyperlink{VII.59}{VII.59}{}; \hyperlink{XIV.146}{XIV.146}{}, \hyperlink{XIV.168}{168}{}; \hyperlink{XX.82}{XX.82}{}, \hyperlink{XX.125}{125}{}; \hyperlink{XXII.28}{XXII.28}{}, \hyperlink{XXII.48}{48f.}{}, \hyperlink{XXII.60}{60}{}. See also conceit 
\par\textbf{primary element} (\emph{bhūta}), great primary (\emph{mahā-bhūta}) \hyperlink{VIII.27}{VIII.27f.}{}, \hyperlink{VIII.45}{45}{}, \hyperlink{VIII.180}{180}{}; \hyperlink{XI.96}{XI.96}{}; \hyperlink{XIV.34}{XIV.34}{}, \hyperlink{XIV.n17}{n.17}{}, \hyperlink{XIV.n18}{n.18}{}, \hyperlink{XIV.n27}{n.27}{}, \hyperlink{XIV.n32}{n.32}{}; \hyperlink{XVII.156}{XVII.156}{}; \hyperlink{XVIII.4}{XVIII.4}{}, \hyperlink{XVIII.14}{14}{}, \hyperlink{XVIII.24}{24}{}; \hyperlink{XXI.35}{XXI.35}{}, \hyperlink{XXI.86}{86}{}
\par\textbf{Primordial Essence} (\emph{pakati}) \hyperlink{XI.n29}{XI.n.29}{}; \hyperlink{XVI.91}{XVI.91}{}; \hyperlink{XVII.8}{XVII.8}{}, \hyperlink{XVII.36}{36}{}; \hyperlink{XVIII.n9}{XVIII.n.9}{}
\par\textbf{produced} (\emph{nipphanna}) \hyperlink{XIV.72}{XIV.72f.}{}, \hyperlink{XIV.77}{77}{}; \hyperlink{XVIII.13}{XVIII.13}{}; \hyperlink{XXIII.52}{XXIII.52}{}, \hyperlink{XXIII.n18}{n.18}{}
\par\textbf{proficiency} (\emph{pāguññatā}) \hyperlink{XIV.133}{XIV.133}{}, \hyperlink{XIV.148}{148}{}
\par\textbf{profitable} (\emph{kusala}) \hyperlink{I.n9}{I.n.9}{}, \hyperlink{I.n16}{n.16}{}; \hyperlink{II.78}{II.78}{}, \hyperlink{II.n18}{n.18}{}; \hyperlink{III.13}{III.13}{}, \hyperlink{III.23}{23}{}, \hyperlink{III.75}{75}{}; \hyperlink{X.14}{X.14}{}; \hyperlink{XIV.23}{XIV.23}{}, \hyperlink{XIV.82}{82f.}{}, \hyperlink{XIV.94}{94}{}, \hyperlink{XIV.109}{109}{}, \hyperlink{XIV.117}{117}{}, \hyperlink{XIV.126}{126}{}, \hyperlink{XIV.129}{129}{}, \hyperlink{XIV.132}{132}{}, \hyperlink{XIV.133}{133f.}{}, \hyperlink{XIV.179}{179}{}, \hyperlink{XIV.193}{193}{}, \hyperlink{XIV.204}{204}{}, \hyperlink{XIV.209}{209}{}, \hyperlink{XIV.n35}{n.35}{}, \hyperlink{XIV.n76}{n.76}{}; \hyperlink{XV.26}{XV.26}{}, \hyperlink{XV.34}{34}{}; \hyperlink{XVI.104}{XVI.104}{}; \hyperlink{XVII.53}{XVII.53}{}, \hyperlink{XVII.93}{93}{}, \hyperlink{XVII.120}{120}{}; \hyperlink{XIX.8}{XIX.8}{}; \hyperlink{XX.28}{XX.28}{}, \hyperlink{XX.31}{31}{}, \hyperlink{XX.44}{44}{}; \hyperlink{XXII.35}{XXII.35}{}, \hyperlink{XXII.85}{85}{}
\par\textbf{profundity} (\emph{gambhīratta}) \hyperlink{VII.71}{VII.71}{}; \hyperlink{XVII.11}{XVII.11}{}, \hyperlink{XVII.33}{33}{}, \hyperlink{XVII.304}{304}{}
\par\textbf{progress} (\emph{paṭipadā}) \hyperlink{II.86}{II.86}{}; \hyperlink{III.5}{III.5}{}, \hyperlink{III.14}{14f.}{}; \hyperlink{XXI.117}{XXI.117}{}
\par\textbf{prominence} (\emph{ussada}) \hyperlink{III.81}{III.81}{}, \hyperlink{III.83}{83}{}; \hyperlink{XI.88}{XI.88}{}
\par\textbf{prompted, prompting} (\emph{sasaṅkhāra}) \hyperlink{XIV.83}{XIV.83}{}, \hyperlink{XIV.90}{90}{}, \hyperlink{XIV.100}{100}{}, \hyperlink{XIV.126}{126}{}, \hyperlink{XIV.156}{156}{}; \hyperlink{XVII.52}{XVII.52}{}, \hyperlink{XVII.122}{122}{}
\par\textbf{proper way} (\emph{sāmīci}) \hyperlink{VII.90}{VII.90}{}; \hyperlink{XVII.24}{XVII.24}{}, \hyperlink{XVII.33}{33}{}
\par\textbf{Pubbavideha} \hyperlink{VII.43}{VII.43f.}{}
\par\textbf{Puṇṇa} \hyperlink{XII.42}{XII.42}{}
\par\textbf{Puṇṇaka} \hyperlink{VIII.18}{VIII.18}{}; \hyperlink{XII.34}{XII.34}{}
\par\textbf{Puṇṇavallika} \hyperlink{IV.95}{IV.95}{}
\par\textbf{Pure Abodes} (\emph{suddhāvāsa}) \hyperlink{XII.79}{XII.79}{}; \hyperlink{XXIII.57}{XXIII.57}{}
\par\textbf{purity} (\emph{soceyya}) \hyperlink{I.22}{I.22}{}
\par\textbf{purification} (\emph{suddhi, visuddhi}) \hyperlink{I.29}{I.29}{}, \hyperlink{I.126}{126}{}; \hyperlink{XVI.85}{XVI.85}{}
\par\textbf{purpose} (\emph{attha}) \hyperlink{XIV.22}{XIV.22}{}
\par\textbf{pus} \emph{ }(\emph{pubba}) \hyperlink{VI.3}{VI.3}{}; \hyperlink{VIII.129}{VIII.129}{}; \hyperlink{XI.17}{XI.17}{}, \hyperlink{XI.71}{71}{}
\par\textbf{Rāhu} \hyperlink{XXI.46}{XXI.46}{}
\par\textbf{Rāhula Thera} \hyperlink{XII.110}{XII.110}{}
\par\textbf{Rājagaha} \hyperlink{XII.126}{XII.126}{}
\par\textbf{Rakkhita} , see Buddharakkhita Thera
\par\textbf{Rathakāra, Lake} \hyperlink{XIII.38}{XIII.38}{}
\par\textbf{Raṭṭhapāla Thera} \hyperlink{XII.110}{XII.110}{}
\par\textbf{razor-wheel} (\emph{khura-cakka}) \hyperlink{XV.42}{XV.42}{}
\par\textbf{real, reality} (\emph{tatha, tathatā}) \hyperlink{XVI.24}{XVI.24f.}{}; \hyperlink{XVII.5}{XVII.5}{}; \hyperlink{XXII.97}{XXII.97}{}
\par\textbf{realization} (\emph{sacchikiriyā}) \hyperlink{XXII.92}{XXII.92}{}, \hyperlink{XXII.124}{124f.}{}
\par\textbf{reappearance} (\emph{upapāta}) \hyperlink{XIII.72}{XIII.72}{}; \hyperlink{XVII.114}{XVII.114}{}
\par\textbf{reasoning} (\emph{cintā}) \hyperlink{XIV.14}{XIV.14}{}
\par\textbf{rebirth-linking} (\emph{paṭisandhi}) \hyperlink{I.7}{I.7}{}; \hyperlink{III.86}{III.86}{}; \hyperlink{IV.n13}{IV.n.13}{}; \hyperlink{VII.10}{VII.10}{}; \hyperlink{VIII.10}{VIII.10}{}; \hyperlink{XI.2}{XI.2}{}; \hyperlink{XIII.14}{XIII.14}{}, \hyperlink{XIII.17}{17f.}{}, \hyperlink{XIII.24}{24}{}, \hyperlink{XIII.76}{76}{}; \hyperlink{XIV.98}{XIV.98}{}, \hyperlink{XIV.111}{111f.}{}, \hyperlink{XIV.124}{124}{}, \hyperlink{XIV.187}{187}{}; \hyperlink{XVI.32}{XVI.32}{}, \hyperlink{XVI.n8}{n.8}{}; \hyperlink{XVII.51}{XVII.51}{}, \hyperlink{XVII.89}{89}{}, \hyperlink{XVII.126}{126}{}, \hyperlink{XVII.133}{133f.}{}, \hyperlink{XVII.164}{164}{}, \hyperlink{XVII.189}{189f.}{}, \hyperlink{XVII.232}{232}{}, \hyperlink{XVII.292}{292}{}; \hyperlink{XIX.13}{XIX.13}{}, \hyperlink{XIX.15}{15}{}, \hyperlink{XIX.23}{23f.}{}; \hyperlink{XX.22}{XX.22}{}, \hyperlink{XX.31}{31}{}, \hyperlink{XX.43}{43}{}, \hyperlink{XX.47}{47}{}; \hyperlink{XXI.37}{XXI.37f.}{}, \hyperlink{XXI.80}{80}{}
\par\textbf{rebirth-process becoming} (\emph{uppatti-bhava}) \hyperlink{VII.16}{VII.16f.}{}; \hyperlink{XVI.92}{XVI.92}{}; \hyperlink{XVII.250}{XVII.250}{}, \hyperlink{XVII.294}{294f.}{}; \hyperlink{XIX.13}{XIX.13}{}
\par\textbf{receiving} (\emph{sampaṭicchana}) \hyperlink{I.57}{I.57}{}; \hyperlink{IV.n13}{IV.n.13}{}; \hyperlink{XI.93}{XI.93}{}; \hyperlink{XIV.95}{XIV.95}{}, \hyperlink{XIV.101}{101}{}, \hyperlink{XIV.118}{118}{}, \hyperlink{XIV.n47}{n.47}{}; \hyperlink{XV.34}{XV.34}{}, \hyperlink{XV.36}{36}{}; \hyperlink{XVII.128}{XVII.128}{}, \hyperlink{XVII.138}{138}{}, \hyperlink{XVII.231}{231}{}; \hyperlink{XX.44}{XX.44}{}
\par\textbf{recollection} (\emph{anussati}) \hyperlink{III.6}{III.6}{}, \hyperlink{III.105}{105f.}{}; \hyperlink{VII}{VII passim}{}, \hyperlink{VII.28}{28}{}; \hyperlink{VIII}{VIII passim}{}; \hyperlink{XIII.13}{XIII.13f.}{}; \hyperlink{XXIII.20}{XXIII.20}{}; r. of the Community (\emph{saṅghānussati}) \hyperlink{III.105}{III.105f.}{}; \hyperlink{VII.89}{VII.89}{}; r. of death (\emph{maraṇānussati}), see mindfulness of d.; r. of deities (\emph{devatānussati}) \hyperlink{III.105}{III.105}{}; \hyperlink{VII.115}{VII.115f.}{}; r. of the Enlightened One (\emph{buddhānussati}) \hyperlink{III.6}{III.6}{}, \hyperlink{III.105}{105f.}{}; \hyperlink{VII.2}{VII.2f.}{}; r. of generosity (\emph{cāgānussati}) \hyperlink{III.105}{III.105f.}{}; \hyperlink{VII.107}{VII.107f.}{}, r. of the Law (\emph{dhammānussati}) \hyperlink{III.105}{III.105f.}{}; \hyperlink{VII.68}{VII.68f.}{}; r. of past life (\emph{pubbenivāsānussati}) \hyperlink{XII.2}{XII.2}{}; \hyperlink{XIII.13}{XIII.13f.}{}, \hyperlink{XIII.120}{120}{}; r. of peace (\emph{upasamānussati}) \hyperlink{III.6}{III.6}{}, \hyperlink{III.105}{105f.}{}; \hyperlink{VIII.245}{VIII.245f.}{}; r. of virtue (\emph{sīlānussati}) \hyperlink{III.105}{III.105f.}{}; \hyperlink{VII.101}{VII.101f.}{}
\par\textbf{rectitude} (\emph{ujukatā}) \hyperlink{XIV.133}{XIV.133}{}, \hyperlink{XIV.149}{149}{}
\par\textbf{reflection} (\emph{paṭisaṅkhā}) \hyperlink{I.85}{I.85}{}, \hyperlink{I.140}{140}{}; \hyperlink{XXI.11}{XXI.11}{}, \hyperlink{XXI.23}{23}{}. See also contemplation of r. 
\par\textbf{refuse-rag wearer} (\emph{paṃsukūlika}) \hyperlink{II.2}{II.2}{}, \hyperlink{II.14}{14}{}
\par\textbf{registration} (\emph{tadārammaṇa}) \hyperlink{IV.n13}{IV.n.13}{}; \hyperlink{XIV.98}{XIV.98}{}, \hyperlink{XIV.100}{100}{}, \hyperlink{XIV.122}{122}{}, \hyperlink{XIV.n39}{n.39}{}, \hyperlink{XIV.n52}{n.52}{}; \hyperlink{XVII.129}{XVII.129f.}{}, \hyperlink{XVII.137}{137f.}{}, \hyperlink{XVII.231}{231}{}; \hyperlink{XX.43}{XX.43}{}
\par\textbf{reliance} (\emph{ālaya}) \hyperlink{I.140}{I.140}{}; \hyperlink{VIII.245}{VIII.245}{}; \hyperlink{XX.90}{XX.90}{}; \hyperlink{XXII.113}{XXII.113}{}, \hyperlink{XXII.120}{120}{}
\par\textbf{relic} (\emph{dhātu}) \hyperlink{XIII.107}{XIII.107}{}
\par\textbf{relinquishment} (\emph{paṭinissagga}) \hyperlink{I.140}{I.140}{}; \hyperlink{III.128}{III.128}{}; \hyperlink{VIII.236}{VIII.236}{}; \hyperlink{XXI.18}{XXI.18}{}. See also contemplation of r. 
\par\textbf{renunciation} (\emph{nekkhamma}) \hyperlink{I.140}{I.140}{}; \hyperlink{III.128}{III.128}{}; \hyperlink{IX.124}{IX.124}{}; \hyperlink{XV.28}{XV.28}{}; \hyperlink{XVI.86}{XVI.86}{}; \hyperlink{XXIII.20}{XXIII.20}{}
\par\textbf{repetition} (\emph{āsevanā}) \hyperlink{I.140}{I.140}{}; \hyperlink{IV.113}{IV.113}{}, \hyperlink{IV.n33}{n.33}{}; \hyperlink{VIII.40}{VIII.40}{}; \hyperlink{XVII.66}{XVII.66}{}, \hyperlink{XVII.87}{87}{}; \hyperlink{XXI.130}{XXI.130}{}; \hyperlink{XXII.6}{XXII.6}{}, \hyperlink{XXII.16}{16}{}
\par\textbf{repulsive} (\emph{paṭikkūla}) \hyperlink{I.n33}{I.n.33}{}; \hyperlink{III.6}{III.6}{}; \hyperlink{VI.1}{VI.1f.}{}; \hyperlink{VIII.43}{VIII.43f.}{}, \hyperlink{VIII.69}{69}{}, \hyperlink{VIII.84}{84}{}, \hyperlink{VIII.n16}{n.16}{}; \hyperlink{XI.4}{XI.4f.}{}; \hyperlink{XII.36}{XII.36f.}{}; \hyperlink{XXI.63}{XXI.63}{}
\par\textbf{requisite} (\emph{parikkhāra}) \hyperlink{I.2}{I.2}{}, \hyperlink{I.68}{68}{}, \hyperlink{I.96}{96}{}; (\emph{paccaya}) \hyperlink{I.18}{I.18}{}, \hyperlink{I.42}{42}{}, \hyperlink{I.85}{85}{}, \hyperlink{I.96}{96}{}, \hyperlink{I.112}{112f.}{}
\par\textbf{resentment} (\emph{paṭigha}) \hyperlink{I.n14}{I.n.14}{}; \hyperlink{IX.14}{IX.14}{}, \hyperlink{IX.88}{88}{}, \hyperlink{IX.96}{96}{}, \hyperlink{IX.101}{101}{}; \hyperlink{XIV.92}{XIV.92}{}; \hyperlink{XXII.45}{XXII.45}{}, \hyperlink{XXII.48}{48}{}, \hyperlink{XXII.51}{51}{}, \hyperlink{XXII.60}{60}{}
\par\textbf{resistance} (\emph{paṭigha}) \hyperlink{I.140}{I.140}{}, \hyperlink{I.n14}{n.14}{}; \hyperlink{X.12}{X.12}{}, \hyperlink{X.16}{16}{}; \hyperlink{XIV.74}{XIV.74}{}; \hyperlink{XV.11}{XV.11}{}
\par\textbf{resolve, resolving} (\emph{adhiṭṭhāna}) \hyperlink{XII.23}{XII.23}{}, \hyperlink{XII.57}{57}{}; \hyperlink{XXIII.27}{XXIII.27}{}, \hyperlink{XXIII.35}{35f.}{}
\par\textbf{resolution} (\emph{adhimutti, adhimokkha}) \hyperlink{III.128}{III.128}{}; \hyperlink{XIV.133}{XIV.133}{}, \hyperlink{XIV.151}{151}{}, \hyperlink{XIV.159}{159}{}, \hyperlink{XIV.170}{170}{}, \hyperlink{XIV.178}{178f.}{}; \hyperlink{XX.118}{XX.118}{}; \hyperlink{XXI.70}{XXI.70}{}, \hyperlink{XXI.75}{75}{}, \hyperlink{XXI.89}{89}{}
\par\textbf{resolution into elements} (\emph{vinibbhoga}), resolved into \emph{e. }(\emph{vinibbhutta}) \hyperlink{IX.38}{IX.38}{}; \hyperlink{XI.30}{XI.30}{}, \hyperlink{XI.105}{105}{}; \hyperlink{XVII.308}{XVII.308}{}; \hyperlink{XXI.4}{XXI.4}{}, \hyperlink{XXI.122}{122}{}; \hyperlink{XXII.114}{XXII.114}{}
\par\textbf{resort} (\emph{gocara}) \hyperlink{I.45}{I.45}{}, \hyperlink{I.49}{49}{}; \hyperlink{XVI.1}{XVI.1}{}
\par\textbf{resting-place} (\emph{senāsana}) \hyperlink{I.68}{I.68}{}, \hyperlink{I.95}{95}{}; \hyperlink{III.97}{III.97}{}; \hyperlink{IV.19}{IV.19}{}; \hyperlink{VIII.158}{VIII.158}{}, \hyperlink{VIII.n42}{n.42}{}
\par\textbf{restraint} (\emph{saṃvara}) \hyperlink{I.17}{I.17}{}, \hyperlink{I.32}{32}{}, \hyperlink{I.42}{42f.}{}, \hyperlink{I.53}{53f.}{}, \hyperlink{I.126}{126}{}, \hyperlink{I.140}{140}{}
\par\textbf{result, resultant} (\emph{vipāka}) \hyperlink{I.57}{I.57}{}; \hyperlink{X.14}{X.14}{}; \hyperlink{XIV.22}{XIV.22}{}, \hyperlink{XIV.94}{94}{}, \hyperlink{XI.1}{XI.1f.}{}, \hyperlink{XI.127}{127}{}, \hyperlink{XI.179}{179}{}, \hyperlink{XI.199}{199}{}, \hyperlink{XI.205}{205}{}; \hyperlink{XV.34}{XV.34}{}; \hyperlink{XVII.109}{XVII.109}{}, \hyperlink{XVII.120}{120f.}{}, \hyperlink{XVII.252}{252}{}; \hyperlink{XIX.8}{XIX.8}{}; \hyperlink{XX.28}{XX.28}{}, \hyperlink{XX.44}{44}{}; \hyperlink{XXI.38}{XXI.38}{}. See also kamma-r. 
\par\textbf{Revata Thera, Majjhimabhāṇaka} \hyperlink{III.51}{III.51}{}
\par\textbf{Revata Thera, Malayavāsin} \hyperlink{III.51}{III.51}{}
\par\textbf{reviewing} (\emph{paccavekkhaṇa}) \hyperlink{I.32}{I.32}{}, \hyperlink{I.85}{85}{}, \hyperlink{I.124}{124}{}; \hyperlink{IV.78}{IV.78}{}, \hyperlink{IV.129}{129}{}; \hyperlink{VII.77}{VII.77}{}; \hyperlink{VIII.224}{VIII.224}{}; \hyperlink{XI.48}{XI.48}{}; \hyperlink{XXII.19}{XXII.19}{}. See also knowledge of r. 
\par\textbf{right action} (\emph{sammā-kammanta}) \hyperlink{XVI.79}{XVI.79}{}, \hyperlink{XVI.86}{86}{}, \hyperlink{XVI.95}{95}{}; \hyperlink{XXII.45}{XXII.45}{}; r. concentration (\emph{s.-samādhi}) \hyperlink{XVI.83}{XVI.83}{}, \hyperlink{XVI.86}{86}{}, \hyperlink{XVI.95}{95}{}; \hyperlink{XXII.45}{XXII.45}{}; r. endeavour (\emph{s.ppadhāna}) \hyperlink{I.6}{I.6}{}; \hyperlink{XII.51}{XII.51}{}; \hyperlink{XXII.33}{XXII.33}{}, \hyperlink{XXII.39}{39}{}, \hyperlink{XXII.42}{42}{}; r. effort (\emph{s.-vāyāma}) \hyperlink{XVI.81}{XVI.81}{}, \hyperlink{XVI.86}{86}{}, \hyperlink{XVI.95}{95f.}{}; \hyperlink{XXII.45}{XXII.45}{}; r. livelihood (\emph{s.-ājīva}) \hyperlink{XVI.80}{XVI.80}{}, \hyperlink{XVI.86}{86}{}, \hyperlink{XVI.95}{95}{}; \hyperlink{XXII.45}{XXII.45}{}; r. mindfulness (\emph{s.-sati}) \hyperlink{XVI.82}{XVI.82}{}, \hyperlink{XVI.86}{86}{}, \hyperlink{XVI.95}{95f.}{}; \hyperlink{XXII.45}{XXII.45}{}; r. speech (\emph{s.vācā}) \hyperlink{XVI.78}{XVI.78}{}, \hyperlink{XVI.86}{86}{}, \hyperlink{XVI.95}{95f.}{}; \hyperlink{XXII.45}{XXII.45}{}; r. thinking (\emph{s.-saṅkappa}) \hyperlink{XVI.77}{XVI.77}{}, \hyperlink{XVI.86}{86}{}, \hyperlink{XVI.95}{95f.}{}; \hyperlink{XXII.45}{XXII.45}{}; r. view (\emph{s.-diṭṭhi}) \hyperlink{I.17}{I.17}{}; \hyperlink{XIV.84}{XIV.84}{}; \hyperlink{XVI.76}{XVI.76}{}, \hyperlink{XVI.86}{86}{}, \hyperlink{XVI.95}{95}{}; \hyperlink{XVII.9}{XVII.9}{}; \hyperlink{XXII.38}{XXII.38}{}, \hyperlink{XXII.45}{45}{}; \hyperlink{XXIII.4}{XXIII.4}{}; r. vision (\emph{s.-dassana}) \hyperlink{XIX.25}{XIX.25}{}
\par\textbf{rightness} (\emph{sammatta}) \hyperlink{V.40}{V.40}{}; \hyperlink{XX.18}{XX.18}{}
\par\textbf{rise} (\emph{udaya}) \hyperlink{VIII.234}{VIII.234}{}; \hyperlink{XV.15}{XV.15}{}; \hyperlink{XVI.35}{XVI.35}{}; \hyperlink{XX.94}{XX.94}{}, \hyperlink{XX.100}{100f.}{}; \hyperlink{XXI.68}{XXI.68}{}
\par\textbf{rise and fall} (\emph{udayabbaya}) \hyperlink{XIV.224}{XIV.224}{}; \hyperlink{XVI.35}{XVI.35}{}; \hyperlink{XVII.283}{XVII.283}{}; \hyperlink{XX.84}{XX.84}{}, \hyperlink{XX.93}{93f.}{}; \hyperlink{XXI.2}{XXI.2f.}{}; \hyperlink{XXIII.10}{XXIII.10}{}. See also contemplation of r. \& f. 
\par\textbf{rules and vows} (\emph{sīlabbata}) \hyperlink{XIV.229}{XIV.229}{}; \hyperlink{XVII.240}{XVII.240f.}{}; \hyperlink{XXII.48}{XXII.48}{}, \hyperlink{XXII.54}{54}{}
\par\textbf{rivers, five great} (\emph{mahā-nadī}) \hyperlink{XIII.36}{XIII.36f.}{}
\par\textbf{road to power} (\emph{iddhi-pāda}) \hyperlink{XII.50}{XII.50}{}; \hyperlink{XVI.86}{XVI.86}{}; \hyperlink{XXII.33}{XXII.33}{}, \hyperlink{XXII.36}{36}{}, \hyperlink{XXII.39}{39}{}, \hyperlink{XXII.42}{42}{}
\par\textbf{robe} (\emph{cīvara}) \hyperlink{I.68}{I.68}{}, \hyperlink{I.86}{86}{}
\par\textbf{Rohaṇa} \hyperlink{III.36}{III.36}{}, \hyperlink{III.53}{53}{}
\par\textbf{Rohaṇa-Gutta Thera, Mahā} \hyperlink{IV.135}{IV.135}{}; \hyperlink{XII.9}{XII.9}{}
\par\textbf{root} (\emph{mūla}) \hyperlink{IV.87}{IV.87}{}; \hyperlink{VII.27}{VII.27}{}, \hyperlink{VII.59}{59}{}
\par\textbf{root-cause} (\emph{hetu}) \hyperlink{III.83}{III.83}{}; \hyperlink{XIV.72}{XIV.72}{}, \hyperlink{XIV.94}{94f.}{}, \hyperlink{XIV.111}{111}{}, \hyperlink{XIV.116}{116}{}, \hyperlink{XIV.127}{127}{}, \hyperlink{XIV.179}{179}{}, \hyperlink{XIV.206}{206}{}, \hyperlink{XIV.n2}{n.2}{}; \hyperlink{XVII.54}{XVII.54}{}, \hyperlink{XVII.66}{66f.}{}, \hyperlink{XVII.160}{160}{}
\par\textbf{root-causeless} (\emph{ahetuka}) \hyperlink{I.57}{I.57}{}; \hyperlink{XIV.95}{XIV.95}{}, \hyperlink{XIV.113}{113f.}{}
\par\textbf{round of defilement} (\emph{kilesa-vaṭṭa}) \hyperlink{XVII.298}{XVII.298}{}; r. of kamma (\emph{kamma-vaṭṭa}) \hyperlink{XVII.298}{XVII.298}{}; \hyperlink{XIX.17}{XIX.17}{}; r. of kamma-result (\emph{vipāka-vaṭṭa}) \hyperlink{XVII.298}{XVII.298}{}; \hyperlink{XIX.17}{XIX.17}{}; r. of rebirths (\emph{saṃsāra}) \hyperlink{XV.4}{XV.4}{}, \hyperlink{XV.20}{20}{}; \hyperlink{XVII.62}{XVII.62}{}, \hyperlink{XVII.115}{115}{}, \hyperlink{XVII.244}{244}{}, \hyperlink{XVII.286}{286}{}, \hyperlink{XVII.314}{314}{}; \hyperlink{XIX.14}{XIX.14}{}; \hyperlink{XXII.14}{XXII.14}{}, \hyperlink{XXII.18}{18}{}
\par\textbf{roundabout talk} (\emph{parikathā}) \hyperlink{I.63}{I.63}{}, \hyperlink{I.79}{79}{}, \hyperlink{I.113}{113f.}{}
\par\textbf{Sabbatthivādin} , see Sarvāstivādin
\par\textbf{sacrifice} (\emph{yañña}) \hyperlink{XVII.62}{XVII.62}{}
\par\textbf{Sāketa} \hyperlink{XII.71}{XII.71}{}
\par\textbf{Sakka Ruler of Gods} (\emph{Sakka-devinda}) \hyperlink{XII.77}{XII.77}{}, \hyperlink{XII.79}{79}{}
\par\textbf{Sāmāvatī upāsikā} \hyperlink{XII.30}{XII.30}{}, \hyperlink{XII.35}{35}{}
\par\textbf{Samudda Thera, Cūḷa} \hyperlink{XII.123}{XII.123}{}
\par\textbf{Saṅgharakkhita Thera} \hyperlink{VI.88}{VI.88}{}
\par\textbf{Saṅgharakkhita Thera, Mahā} \hyperlink{I.135}{I.135}{}; \hyperlink{III.85}{III.85}{} Saṅgharakkhita Thera, (\& sāmaṇera),
\par\textbf{Bhāgineyya} \hyperlink{I.130}{I.130}{}, \hyperlink{I.135}{135}{}
\par\textbf{Saṅkassa-nagara} \hyperlink{XII.75}{XII.75}{}, \hyperlink{XII.122}{122}{}
\par\textbf{Saṅkicca Sāmaṇera} \hyperlink{IX.71}{IX.71}{}
\par\textbf{Saṅkicca Thera} \hyperlink{VII.26}{VII.26}{}, \hyperlink{VII.28}{28}{}
\par\textbf{Sañjīva Thera} \hyperlink{XII.30}{XII.30}{}, \hyperlink{XII.32}{32}{}; \hyperlink{XIII.37}{XIII.37}{}
\par\textbf{Sāriputta Thera} \hyperlink{I.117}{I.117}{}; \hyperlink{II.82}{II.82}{}; \hyperlink{VII.n23}{VII.n.23}{}; \hyperlink{VII.213}{VII.213}{}; \hyperlink{X.53}{X.53}{}; \hyperlink{XII.30}{XII.30f.}{}; \hyperlink{XXI.118}{XXI.118}{}; \hyperlink{XXIII.37}{XXIII.37}{}
\par\textbf{Sarvāstivādin} (\emph{Sabbatthivādin}) \hyperlink{VII.n36}{VII.n.36}{}
\par\textbf{Sasaṅkhāra-parinibbāyin} \hyperlink{XXIII.56}{XXIII.56}{}
\par\textbf{sattakkhattuparama} \hyperlink{XXIII.55}{XXIII.55}{}
\par\textbf{Sāvatthī} \hyperlink{XII.71}{XII.71}{}, \hyperlink{XII.122}{122}{}
\par\textbf{Schedule of Abhidhamma} (\emph{mātikā}) \hyperlink{XIII.n20}{XIII.n.20}{}
\par\textbf{science} (\emph{vijjā}) \hyperlink{XIV.14}{XIV.14}{}
\par\textbf{scripture} (\emph{pariyatti}) \hyperlink{VII.n1}{VII.n.1}{}
\par\textbf{search} (\emph{esanā}) \hyperlink{III.58}{III.58}{}, \hyperlink{III.124}{124}{}
\par\textbf{season} (\emph{utu}) \hyperlink{VII.159}{VII.159}{}; \hyperlink{XX.55}{XX.55}{}
\par\textbf{seclusion} (\emph{viveka, paviveka}) \hyperlink{II.83}{II.83}{}; \hyperlink{III.128}{III.128}{}; \hyperlink{IV.82}{IV.82}{}, \hyperlink{IV.n23}{n.23}{}; \hyperlink{VII.63}{VII.63}{}; \hyperlink{XXIII.50}{XXIII.50}{}
\par\textbf{sectarian} (\emph{titthiya}) \hyperlink{I.45}{I.45}{}; \hyperlink{XV.21}{XV.21}{}; \hyperlink{XVI.63}{XVI.63}{}
\par\textbf{seeing} (\emph{dassana}) \hyperlink{I.5}{I.5}{}; \hyperlink{IV.45}{IV.45}{}; \hyperlink{XIV.13}{XIV.13}{}, \hyperlink{XIV.117}{117}{}, \hyperlink{XIV.123}{123}{}; \hyperlink{XVI.95}{XVI.95}{}; \hyperlink{XVII.127}{XVII.127}{}; \hyperlink{XXII.45}{XXII.45}{}, \hyperlink{XXII.127}{127}{}
\par\textbf{seen} (\emph{diṭṭha}) \hyperlink{XVII.202}{XVII.202}{}; \hyperlink{XXI.17}{XXI.17}{}, \hyperlink{XXI.21}{21}{}
\par\textbf{seen, heard, sensed, cognized} (\emph{diṭṭha-sutamuta viññāta}) \hyperlink{XIV.76}{XIV.76}{}
\par\textbf{self} (\emph{attā}) \hyperlink{I.34}{I.34}{}, \hyperlink{I.93}{93}{}, \hyperlink{I.140}{140}{}; \hyperlink{VIII.233}{VIII.233}{}, \hyperlink{VIII.n65}{n.65}{}; \hyperlink{IX.10}{IX.10}{}, \hyperlink{IX.47}{47}{}, \hyperlink{IX.54}{54}{}, \hyperlink{IX.n6}{n.6}{}; \hyperlink{XI.32}{XI.32}{}, \hyperlink{XI.n21}{n.21}{}, \hyperlink{XI.n29}{n.29}{}; \hyperlink{XIV.213}{XIV.213}{}, \hyperlink{XIV.216}{216}{}, \hyperlink{XIV.228}{228}{}; \hyperlink{XV.21}{XV.21}{}, \hyperlink{XV.40}{40}{}; \hyperlink{XVI.16}{XVI.16}{}, \hyperlink{XVI.24}{24}{}, \hyperlink{XVI.85}{85}{}, \hyperlink{XVI.90}{90}{}, \hyperlink{XVI.n25}{n.25}{}; \hyperlink{XVII.116}{XVII.116}{}, \hyperlink{XVII.282}{282}{}, \hyperlink{XVII.303}{303}{}, \hyperlink{XVII.312}{312}{}; \hyperlink{XVIII.n9}{XVIII.n.9}{}; \hyperlink{XX.16}{XX.16}{}, \hyperlink{XX.84}{84}{}, \hyperlink{XX.90}{90}{}, \hyperlink{XX.126}{126}{}; \hyperlink{XXI.53}{XXI.53}{}, \hyperlink{XXI.56}{56}{}, \hyperlink{XXI.n10}{n.10}{}; \hyperlink{XXII.117}{XXII.117f.}{}
\par\textbf{self-doctrine} (\emph{atta-vāda}) \hyperlink{XVII.240}{XVII.240f.}{} sense-base, see base
\par\textbf{sense becoming, sense-desire b.} (\emph{kāmabhava}) \hyperlink{VII.9}{VII.9f.}{}; \hyperlink{XVII.150}{XVII.150}{}, \hyperlink{XVII.180}{180}{}
\par\textbf{sense desire, sensual desire} (\emph{kāma}) \hyperlink{III.26}{III.26}{}; \hyperlink{IV.82}{IV.82f.}{}, \hyperlink{IV.n24}{n.24}{}; \hyperlink{VII.n4}{VII.n.4}{}; \hyperlink{XIV.91}{XIV.91}{}, \hyperlink{XIV.226}{226}{}, \hyperlink{XIV.n36}{n.36}{}; \hyperlink{XV.27}{XV.27}{}; \hyperlink{XVI.93}{XVI.93}{}; \hyperlink{XVII.63}{XVII.63}{}, \hyperlink{XVII.240}{240f.}{}, \hyperlink{XVII.262}{262}{}; \hyperlink{XXI.n13}{XXI.n.13}{}; \hyperlink{XXII.48}{XXII.48}{}
\par\textbf{sense of urgency} (\emph{saṃvega}) \hyperlink{III.95}{III.95}{}; \hyperlink{IV.63}{IV.63}{}; \hyperlink{XIII.35}{XIII.35}{}
\par\textbf{sense sphere, sensual sphere, sense-desire sphere} (\emph{kāmāvacara}) \hyperlink{III.5}{III.5}{}, \hyperlink{III.23}{23}{}; \hyperlink{IV.74}{IV.74}{}, \hyperlink{IV.138}{138}{}; \hyperlink{X.10}{X.10}{}; \hyperlink{XIII.5}{XIII.5}{}; \hyperlink{XIV.15}{XIV.15}{}, \hyperlink{XIV.83}{83}{}, \hyperlink{XIV.95}{95}{}, \hyperlink{XIV.106}{106}{}, \hyperlink{XIV.111}{111}{}, \hyperlink{XIV.122}{122}{}, \hyperlink{XIV.127}{127}{}, \hyperlink{XIV.133}{133}{}, \hyperlink{XIV.181}{181}{}, \hyperlink{XIV.206}{206}{}, \hyperlink{XIV.n36}{n.36}{}; \hyperlink{XVI.92}{XVI.92}{}; \hyperlink{XVII.129}{XVII.129}{}, \hyperlink{XVII.136}{136}{}, \hyperlink{XVII.180}{180}{}, \hyperlink{XVII.262}{262}{}; \hyperlink{XX.31}{XX.31}{}, \hyperlink{XX.43}{43}{}
\par\textbf{sensed} (\emph{muta}) \hyperlink{XIV.76}{XIV.76}{}
\par\textbf{sensitivity, sensitive} (\emph{pasāda}) \hyperlink{I.53}{I.53}{}, \hyperlink{I.57}{57}{}; \hyperlink{XIII.2}{XIII.2}{}; \hyperlink{XIV.37}{XIV.37f.}{}, \hyperlink{XIV.72}{72}{}, \hyperlink{XIV.78}{78}{}, \hyperlink{XIV.115}{115}{}; \hyperlink{XV.34}{XV.34}{}; \hyperlink{XVII.294}{XVII.294}{}; \hyperlink{XVIII.5}{XVIII.5}{}, \hyperlink{XVIII.9}{9}{}; \hyperlink{XIX.13}{XIX.13}{}; \hyperlink{XX.14}{XX.14}{}
\par\textbf{sequence of meaning} (\emph{anusandhi}) \hyperlink{VII.69}{VII.69}{}, \hyperlink{VII.n31}{n.31}{}
\par\textbf{serenity} (\emph{samatha}) \hyperlink{I.8}{I.8}{}; \hyperlink{III.17}{III.17}{}, \hyperlink{III.n7}{n.7}{}; \hyperlink{IV.64}{IV.64}{}, \hyperlink{IV.111}{111}{}; \hyperlink{VIII.60}{VIII.60}{}, \hyperlink{VIII.179}{179}{}, \hyperlink{VIII.237}{237}{}; \hyperlink{IX.104}{IX.104}{}; \hyperlink{XVIII.3}{XVIII.3}{}, \hyperlink{XVIII.5}{5}{}, \hyperlink{XVIII.8}{8}{}, \hyperlink{XVIII.n2}{n.2}{}; \hyperlink{XX.110}{XX.110}{}, \hyperlink{XX.n33}{n.33}{}; \hyperlink{XXII.46}{XXII.46}{}, \hyperlink{XXII.89}{89}{}; \hyperlink{XXIII.20}{XXIII.20f.}{}, \hyperlink{XXIII.43}{43}{}
\par\textbf{setting up} (\emph{ācaya}) \hyperlink{XIV.66}{XIV.66}{}
\par\textbf{sex} (\emph{bhāva}) \hyperlink{XI.88}{XI.88}{}; \hyperlink{XIV.n74}{XIV.n.74}{}; \hyperlink{XVII.150}{XVII.150}{}, \hyperlink{XVII.189}{189}{}; (\emph{liṅga}) \hyperlink{XI.89}{XI.89}{}
\par\textbf{sexual misconduct} (\emph{kāmesu micchācāra}) \hyperlink{I.140}{I.140}{}; \hyperlink{XXII.62}{XXII.62}{}
\par\textbf{shackle, mental} (\emph{cetaso vinibandha}) \hyperlink{VII.59}{VII.59}{}, \hyperlink{VII.n25}{n.25}{}
\par\textbf{shame} (\emph{ottappa}) \hyperlink{I.22}{I.22}{}, \hyperlink{I.48}{48}{}; \hyperlink{VII.n8}{VII.n.8}{}; \hyperlink{XIV.133}{XIV.133}{}, \hyperlink{XIV.142}{142}{}, \hyperlink{XIV.155}{155}{}
\par\textbf{shamelessness} (\emph{anottappa}) \hyperlink{VII.59}{VII.59}{}; \hyperlink{XIV.159}{XIV.159f.}{}, \hyperlink{XIV.170}{170}{}, \hyperlink{XIV.176}{176}{}; \hyperlink{XXII.49}{XXII.49}{}
\par\textbf{shape} (\emph{saṇṭhāna}) \hyperlink{I.n14}{I.n.14}{}; \hyperlink{VIII.82}{VIII.82}{}; \hyperlink{XIV.n32}{XIV.n.32}{}
\par\textbf{sickness} (\emph{byādhi}) \hyperlink{IV.63}{IV.63}{}
\par\textbf{sign} (\emph{nimitta}) \hyperlink{I.42}{I.42}{}, \hyperlink{I.54}{54}{}, \hyperlink{I.63}{63}{}, \hyperlink{I.77}{77}{}, \hyperlink{I.100}{100}{}, \hyperlink{I.140}{140}{}, \hyperlink{I.n14}{n.14}{}; \hyperlink{III.113}{III.113f.}{}, \hyperlink{III.132}{132}{}, \hyperlink{III.n31}{n.31}{}; \hyperlink{IV.22}{IV.22f.}{}, \hyperlink{IV.30}{30}{}, \hyperlink{IV.72}{72}{}, \hyperlink{IV.74}{74}{}, \hyperlink{IV.111}{111}{}, \hyperlink{IV.126}{126f.}{}; \hyperlink{VI.66}{VI.66}{}; \hyperlink{VII.51}{VII.51}{}, \hyperlink{VII.107}{107}{}; \hyperlink{VIII.74}{VIII.74}{}, \hyperlink{VIII.141}{141}{}, \hyperlink{VIII.204}{204}{}, \hyperlink{VIII.206}{206}{}, \hyperlink{VIII.214}{214}{}; \hyperlink{IX.43}{IX.43}{}; \hyperlink{X.9}{X.9}{}; \hyperlink{XI.25}{XI.25}{}; \hyperlink{XIV.111}{XIV.111f.}{}, \hyperlink{XIV.130}{130}{}; \hyperlink{XV.26}{XV.26}{}; \hyperlink{XVII.136}{XVII.136f.}{}, \hyperlink{XVII.278}{278}{}; \hyperlink{XIX.26}{XIX.26}{}; \hyperlink{XX.21}{XX.21}{}, \hyperlink{XX.90}{90}{}; \hyperlink{XXI.10}{XXI.10}{}, \hyperlink{XXI.27}{27}{}, \hyperlink{XXI.33}{33}{}, \hyperlink{XXI.37}{37f.}{}, \hyperlink{XXI.51}{51f.}{}, \hyperlink{XXI.73}{73}{}, \hyperlink{XXI.83}{83}{}; \hyperlink{XXII.4}{XXII.4}{}, \hyperlink{XXII.11}{11}{}, \hyperlink{XXII.44}{44f.}{}, \hyperlink{XXII.79}{79}{}, \hyperlink{XXII.113}{113}{}, \hyperlink{XXII.117}{117}{}; \hyperlink{XXIII.4}{XXIII.4}{}, \hyperlink{XXIII.7}{7}{}, \hyperlink{XXIII.12}{12}{}
\par\textbf{signless} (\emph{animitta}) \hyperlink{I.140}{I.140}{}; \hyperlink{VIII.29}{VIII.29}{}; \hyperlink{XVI.23}{XVI.23}{}. See also contemplation of the s.; s. element (\emph{animitta-dhātu}) \hyperlink{XXI.67}{XXI.67}{}; \hyperlink{XXIII.9}{XXIII.9}{}; s. liberation (\emph{animitta-vimokkha}) \hyperlink{XXI.70}{XXI.70}{}, \hyperlink{XXI.89}{89}{}, \hyperlink{XXI.121}{121}{}; s. mind-deliverance (\emph{animitta-cetovimutti}) \hyperlink{XXIII.12}{XXIII.12}{}
\par\textbf{Sīhapapāta, Lake} \hyperlink{XIII.38}{XIII.38}{}
\par\textbf{Simbali Tree} \hyperlink{VII.43}{VII.43}{}
\par\textbf{Sineru, Mount} \hyperlink{VII.23}{VII.23}{}, \hyperlink{VII.42}{42}{}, \hyperlink{VII.n14}{n.14}{}, \hyperlink{VII.n15}{n.15}{}; \hyperlink{XII.72}{XII.72}{}, \hyperlink{XII.78}{78}{}, \hyperlink{XII.109}{109f.}{}, \hyperlink{XII.121}{121}{}, \hyperlink{XII.n19}{n.19}{}; \hyperlink{XIII.34}{XIII.34}{}, \hyperlink{XIII.41}{41}{}, \hyperlink{XIII.48}{48}{}
\par\textbf{Sirimā} \hyperlink{XII.34}{XII.34}{}
\par\textbf{Sirīsa Tree} \hyperlink{VII.43}{VII.43}{}
\par\textbf{sitter} (\emph{nesajjika}) \hyperlink{II.2}{II.2}{}, \hyperlink{II.73}{73}{}
\par\textbf{Siva Thera, Cūḷa} \hyperlink{V.2}{V.2}{}
\par\textbf{Siva Thera, Cūḷa Saṃyuttabhāṇaka} \hyperlink{IX.71}{IX.71}{}
\par\textbf{sixfold base} (\emph{saḷāyatana}), see base
\par\textbf{sleep} (\emph{niddā}) \hyperlink{XIV.114}{XIV.114}{}, \hyperlink{XIV.167}{167}{}, \hyperlink{XIV.n68}{n.68}{}
\par\textbf{solid food} (\emph{kabaliṅkārāhāra}), see physical nutriment 
\par\textbf{solidity} (\emph{paṭhavi}), see earth
\par\textbf{Sopāka} \hyperlink{III.110}{III.110}{}; \hyperlink{XI.n1}{XI.n.1}{}
\par\textbf{Soreyya} \hyperlink{XXII.83}{XXII.83}{}
\par\textbf{sorrow} (\emph{soka}) \hyperlink{IX.94}{IX.94}{}; \hyperlink{XVI.31}{XVI.31}{}, \hyperlink{XVI.48}{48}{}; \hyperlink{XVII.2}{XVII.2}{}, \hyperlink{XVII.48}{48}{}, \hyperlink{XVII.57}{57}{}, \hyperlink{XVII.272}{272}{}
\par\textbf{soul} (\emph{jīva}) \hyperlink{IV.143}{IV.143}{}; \hyperlink{VII.n1}{VII.n.1}{}; \hyperlink{XV.22}{XV.22}{}, \hyperlink{XV.32}{32}{}; \hyperlink{XVII.n9}{XVII.n.9}{}
\par\textbf{soulless} (\emph{nijjīva}) \hyperlink{VII.n1}{VII.n.1}{}; \hyperlink{XI.41}{XI.41}{}; \hyperlink{XV.22}{XV.22}{}; \hyperlink{XVII.31}{XVII.31}{}, \hyperlink{XVII.162}{162}{}, \hyperlink{XVII.308}{308}{}
\par\textbf{sound} (\emph{sadda}) \hyperlink{I.59}{I.59}{}; \hyperlink{XIII.3}{XIII.3f.}{}, \hyperlink{XIII.109}{109}{}, \hyperlink{XIII.112}{112}{}; \hyperlink{XIV.55}{XIV.55}{}, \hyperlink{XIV.96}{96}{}, \hyperlink{XIV.134}{134}{}, \hyperlink{XIV.n22}{n.22}{}, \hyperlink{XIV.n27}{n.27}{}, \hyperlink{XIV.n33}{n.33}{}; \hyperlink{XV.3}{XV.3}{}; \hyperlink{XVII.193}{XVII.193}{}; \hyperlink{XVIII.6}{XVIII.6}{}, \hyperlink{XVIII.10}{10}{}; s. base (\emph{saddāyatana}) \hyperlink{XIV.76}{XIV.76}{}, \hyperlink{XIV.79}{79}{}; \hyperlink{XV.3}{XV.3}{}; s. element (\emph{sadda-dhātu}) \hyperlink{XV.17}{XV.17}{}; s. ennead (\emph{saddanavaka}) \hyperlink{XVII.193}{XVII.193}{}; \hyperlink{XX.40}{XX.40}{}
\par\textbf{space} (\emph{ākāsa}) \hyperlink{V.28}{V.28}{}, \hyperlink{V.n5}{n.5}{}; \hyperlink{IX.122}{IX.122f.}{}; \hyperlink{X.1}{X.1f.}{}; \hyperlink{XIII.41}{XIII.41}{}; s. element (\emph{ākāsa-dhātu}) \hyperlink{XV.25}{XV.25}{}; \hyperlink{XVII.13}{XVII.13}{}; \hyperlink{XX.32}{XX.32}{}; s. kasiṇa (\emph{ākāsa-kasiṇa}) \hyperlink{III.105}{III.105f.}{}; \hyperlink{V.29}{V.29}{}, \hyperlink{V.37}{37}{}; \hyperlink{XII.88}{XII.88}{}
\par\textbf{special quality} (\emph{guṇa}) \hyperlink{VII.1}{VII.1}{}, \hyperlink{VII.66}{66}{}; \hyperlink{VIII.245}{VIII.245}{}
\par\textbf{speech} (\emph{vācā}) \hyperlink{XXII.42}{XXII.42}{}. See also right s.; s.door (\emph{vacīdvāra}) \hyperlink{XVII.61}{XVII.61}{}; s. utterance (\emph{vacībheda}) \hyperlink{XIV.62}{XIV.62}{}
\par\textbf{spirit, sprite} (\emph{yakkha}) \hyperlink{VII.42}{VII.42}{}; \hyperlink{XI.98}{XI.98f.}{}; \hyperlink{XII.31}{XII.31}{}, \hyperlink{XII.n19}{n.19}{}; \hyperlink{XIII.100}{XIII.100}{}
\par\textbf{stage of life} (\emph{vaya}) \hyperlink{XX.48}{XX.48}{}
\par\textbf{stain} (\emph{mala}) \hyperlink{VII.59}{VII.59}{}; \hyperlink{XXII.61}{XXII.61}{}, \hyperlink{XXII.74}{74}{}
\par\textbf{state} (\emph{dhamma}) \hyperlink{VII.n1}{VII.n.1}{}, \hyperlink{VII.n4}{n.4}{}; \hyperlink{XI.104}{XI.104}{}; \hyperlink{XIV.7}{XIV.7}{}; \hyperlink{XVII.n1}{XVII.n.1}{}; \hyperlink{XIX.26}{XIX.26}{}; \hyperlink{XX.8}{XX.8}{}; \hyperlink{XXII.20}{XXII.20}{}, \hyperlink{XXII.47}{47}{}. See also dhamma 
\par\textbf{state partaking of enlightenment} (\emph{bodhipakkhiya-dhamma}) \hyperlink{XXI.130}{XXI.130}{}; \hyperlink{XXII.33}{XXII.33}{}
\par\textbf{state of loss} (\emph{apāya}) \hyperlink{IV.63}{IV.63}{}; \hyperlink{VII.16}{VII.16}{}; \hyperlink{XIII.92}{XIII.92}{}; \hyperlink{XIV.113}{XIV.113}{}; \hyperlink{XVII.262}{XVII.262}{}; \hyperlink{XXII.14}{XXII.14}{}
\par\textbf{state of peace} (\emph{santi-pada}) \hyperlink{XXI.37}{XXI.37}{}
\par\textbf{station of consciousness} (\emph{viññāṇaṭṭhiti}) \hyperlink{VII.38}{VII.38}{}, \hyperlink{VII.n13}{n.13}{}; \hyperlink{XIII.69}{XIII.69}{}; \hyperlink{XVII.148}{XVII.148}{}; \hyperlink{XXI.35}{XXI.35}{}
\par\textbf{stationariness} (\emph{ṭhiti}) \hyperlink{I.39}{I.39}{}; \hyperlink{III.22}{III.22}{}
\par\textbf{steadiness of consciousness} (\emph{cittaṭṭhiti}) \hyperlink{IV.145}{IV.145}{}; \hyperlink{XIV.139}{XIV.139}{}, \hyperlink{XIV.176}{176f.}{}, \hyperlink{XIV.179}{179}{}; \hyperlink{XVIII.8}{XVIII.8}{}
\par\textbf{stealing} (\emph{adinnādāna}), see taking what is not given 
\par\textbf{stiffness} (\emph{thīna}) \hyperlink{III.95}{III.95}{}; \hyperlink{XIV.167}{XIV.167}{}; \hyperlink{XXII.49}{XXII.49}{}
\par\textbf{stiffness and torpor} (\emph{thīna-middha}) \hyperlink{I.140}{I.140}{}; \hyperlink{IV.86}{IV.86}{}, \hyperlink{IV.104}{104}{}; \hyperlink{V.35}{V.35f.}{}; \hyperlink{XIV.166}{XIV.166}{}, \hyperlink{XIV.175}{175}{}
\par\textbf{stream-enterer} (\emph{sotāpanna}) \hyperlink{III.128}{III.128}{}; \hyperlink{XIII.110}{XIII.110}{}; \hyperlink{XIX.27}{XIX.27}{}; \hyperlink{XXII.18}{XXII.18}{}; \hyperlink{XXIII.7}{XXIII.7}{}, \hyperlink{XXIII.18}{18}{}, \hyperlink{XXIII.55}{55}{}
\par\textbf{stream-entry} (\emph{sotāpatti}) \hyperlink{I.14}{I.14}{}, \hyperlink{I.140}{140}{}; \hyperlink{XIV.206}{XIV.206}{}, \hyperlink{XIV.n63}{n.63}{}; \hyperlink{XVII.245}{XVII.245}{}; \hyperlink{XXI.75}{XXI.75}{}; \hyperlink{XXII.2}{XXII.2}{}, \hyperlink{XXII.14}{14}{}, \hyperlink{XXII.45}{45}{}; \hyperlink{XXIII.4}{XXIII.4}{}, \hyperlink{XXIII.7}{7}{}, \hyperlink{XXIII.25}{25}{}
\par\textbf{structure of conditions} (\emph{paccayākāra}) \hyperlink{XVII.9}{XVII.9}{}
\par\textbf{Subhaddā, Cūḷa} \hyperlink{XII.71}{XII.71}{}
\par\textbf{Subhakiṇha} (Refulgent Glory) Deities \hyperlink{XIII.57}{XIII.57}{}
\par\textbf{Subrahmā} \hyperlink{XIII.127}{XIII.127}{}
\par\textbf{substance} (\emph{drabya}) \hyperlink{XVIII.n8}{XVIII.n.8}{}
\par\textbf{substitution of opposites} (\emph{tad-aṅga}) \hyperlink{I.12}{I.12}{}; \hyperlink{VIII.236}{VIII.236}{}; \hyperlink{XXI.18}{XXI.18}{}; \hyperlink{XXII.110}{XXII.110}{}
\par\textbf{subtle} (\emph{sukhuma}) \hyperlink{VIII.176}{VIII.176}{}; \hyperlink{XIV.73}{XIV.73}{}; \hyperlink{XVI.34}{XVI.34}{}
\par\textbf{success} (\emph{iddhi}), see power, supernormal power, road to power 
\par\textbf{successive arising in adjacent locations} (\emph{desantaruppatti}) \hyperlink{VIII.n45}{VIII.n.45}{}, \hyperlink{VIII.n54}{n.54}{}; \hyperlink{XI.n37}{XI.n.37}{}; \hyperlink{XII.n21}{XII.n.21}{}; \hyperlink{XIV.n27}{XIV.n.27}{}, \hyperlink{XIV.n29}{n.29}{}
\par\textbf{Sudassa} (Fair to See) Deities \hyperlink{XIV.193}{XIV.193}{}
\par\textbf{Sudassanapabbata} \hyperlink{VII.42}{VII.42}{}
\par\textbf{Sudassin} (Fair-seeing) Deities \hyperlink{XIV.193}{XIV.193}{}
\par\textbf{Suddhāvāsa} , see Pure Abodes
\par\textbf{suffering} (\emph{dukkha}) \hyperlink{IV.63}{IV.63}{}; \hyperlink{VII.27}{VII.27}{}; \hyperlink{IX.94}{IX.94}{}; \hyperlink{XVI.13}{XVI.13}{}, \hyperlink{XVI.16}{16}{}, \hyperlink{XVI.32}{32f.}{}; \hyperlink{XVII.2}{XVII.2}{}, \hyperlink{XVII.62}{62}{}; \hyperlink{XX.47}{XX.47}{}, \hyperlink{XX.100}{100}{}, \hyperlink{XX.130}{130}{}; \hyperlink{XXI.37}{XXI.37}{}, \hyperlink{XXI.41}{41}{}; \hyperlink{XXII.14}{XXII.14}{}, \hyperlink{XXII.48}{48}{}, \hyperlink{XXII.93}{93}{}. See also pain, contemplation of pain 
\par\textbf{suitable} (\emph{sappāya}) \hyperlink{III.16}{III.16}{}, \hyperlink{III.97}{97f.}{}, \hyperlink{III.121}{121}{}; \hyperlink{IV.35}{IV.35}{}
\par\textbf{Sumana-devi} \hyperlink{XII.42}{XII.42}{}
\par\textbf{Sumana Thera, Cūḷa} \hyperlink{XX.110}{XX.110}{}
\par\textbf{sun} (\emph{suriya}) \hyperlink{VII.44}{VII.44}{}; \hyperlink{XII.102}{XII.102}{}; \hyperlink{XIII.36}{XIII.36}{}, \hyperlink{XIII.45}{45}{}
\par\textbf{Sundarī} \hyperlink{XXII.99}{XXII.99}{}
\par\textbf{Supaṇṇa (demon)} \hyperlink{IV.135}{IV.135}{}; \hyperlink{VII.n17}{VII.n.17}{}; \hyperlink{XII.100}{XII.100}{}, \hyperlink{XII.115}{115}{}, \hyperlink{XII.137}{137}{}, \hyperlink{XII.n19}{n.19}{}; \hyperlink{XXI.46}{XXI.46}{}
\par\textbf{superior} (\emph{paṇīta}) \hyperlink{I.33}{I.33}{}
\par\textbf{supernormal power} (\emph{iddhi}) \hyperlink{III.56}{III.56}{}; \hyperlink{VII.30}{VII.30}{}; \hyperlink{XII}{XII passim}{}, \hyperlink{XII.20}{20f.}{}; \hyperlink{XIII.106}{XIII.106}{}, \hyperlink{XIII.122}{122}{}
\par\textbf{support} (\emph{nissaya}) \hyperlink{XIV.46}{XIV.46}{}, \hyperlink{XIV.60}{60}{}; \hyperlink{XVII.66}{XVII.66}{}, \hyperlink{XVII.79}{79}{}
\par\textbf{suppression} (\emph{vikkhambhana}) \hyperlink{I.12}{I.12}{}; \hyperlink{IV.31}{IV.31}{}, \hyperlink{IV.87}{87}{}; \hyperlink{VI.67}{VI.67}{}
\par\textbf{supramundane} (\emph{lokuttara}) \hyperlink{I.29}{I.29}{}, \hyperlink{I.32}{32}{}, \hyperlink{I.135}{135}{}, \hyperlink{I.n4}{n.4}{}; \hyperlink{III.5}{III.5}{}, \hyperlink{III.7}{7}{}, \hyperlink{III.n5}{n.5}{}; \hyperlink{VIII.40}{VIII.40}{}; \hyperlink{XIV.8}{XIV.8f.}{}, \hyperlink{XIV.88}{88}{}, \hyperlink{XIV.105}{105}{}, \hyperlink{XIV.127}{127}{}, \hyperlink{XIV.158}{158}{}, \hyperlink{XIV.182}{182}{}, \hyperlink{XIV.202}{202}{}, \hyperlink{XIV.n36}{n.36}{}; \hyperlink{XVI.102}{XVI.102}{}; \hyperlink{XVII.120}{XVII.120}{}; \hyperlink{XVIII.8}{XVIII.8}{}; \hyperlink{XX.12}{XX.12}{}, \hyperlink{XX.31}{31}{}; \hyperlink{XXII.36}{XXII.36}{}, \hyperlink{XXII.122}{122}{}, \hyperlink{XXII.124}{124}{}, \hyperlink{XXII.128}{128}{}; \hyperlink{XXIII.2}{XXIII.2}{}, \hyperlink{XXIII.52}{52}{}; s. states, the nine \hyperlink{VII.68}{VII.68}{}, \hyperlink{VII.74}{74f.}{}
\par\textbf{sustained thought} (\emph{vicāra}) \hyperlink{I.140}{I.140}{}; \hyperlink{III.5}{III.5}{}, \hyperlink{III.21}{21}{}; \hyperlink{IV.74}{IV.74}{}, \hyperlink{IV.86}{86}{}, \hyperlink{IV.88}{88f.}{}, \hyperlink{IV.132}{132}{}; \hyperlink{VII.28}{VII.28}{}; \hyperlink{IX.112}{IX.112f.}{}; \hyperlink{XIV.86}{XIV.86}{}, \hyperlink{XIV.133}{133}{}, \hyperlink{XIV.136}{136}{}, \hyperlink{XIV.157}{157}{}, \hyperlink{XIV.170}{170}{}, \hyperlink{XIV.176}{176}{}, \hyperlink{XIV.180}{180}{}; \hyperlink{XVI.86}{XVI.86}{}; \hyperlink{XVII.160}{XVII.160}{}; \hyperlink{XX.9}{XX.9}{}; \hyperlink{XXIII.24}{XXIII.24}{}, \hyperlink{XXIII.26}{26}{}
\par\textbf{Suyāma} \hyperlink{XII.79}{XII.79}{}
\par\textbf{syllogism, member of} (\emph{vacanāvayava}) \hyperlink{XVII.67}{XVII.67}{}
\par\textbf{taking what is not given} (\emph{adinnādāna}) \hyperlink{I.140}{I.140}{}; \hyperlink{XXII.62}{XXII.62}{}
\par\textbf{Taḷaṅgara} \hyperlink{XII.80}{XII.80}{}; \hyperlink{XX.111}{XX.111}{}
\par\textbf{Tālaveli-magga} \hyperlink{II.16}{II.16}{}
\par\textbf{talk, the ten instances of} (\emph{dasa-kathāvatthu}) \hyperlink{I.49}{I.49}{}; \hyperlink{IV.38}{IV.38}{}; t., thirty-two kinds of aimless (\emph{tiracchāna-kathā}) \hyperlink{IV.n15}{IV.n.15}{}
\par\textbf{talking} (\emph{lapanā}) \hyperlink{I.42}{I.42}{}, \hyperlink{I.62}{62}{}, \hyperlink{I.72}{72}{}
\par\textbf{Tambapaṇṇi-dīpa} (Sri Lanka) \hyperlink{I.99}{I.99}{}; \hyperlink{IV.36}{IV.36}{}; \hyperlink{IX.64}{IX.64}{}; \hyperlink{XII.80}{XII.80}{}, \hyperlink{XII.83}{83}{}, \hyperlink{XII.123}{123}{}; \hyperlink{XX.n1}{XX.n.1}{}
\par\textbf{tangible datum} (\emph{phoṭṭhabba}) \hyperlink{XIV.128}{XIV.128}{}, \hyperlink{XIV.n32}{n.32}{}; \hyperlink{XVIII.11}{XVIII.11}{}; t.-d. base (\emph{phoṭṭhabbāyatana}) \hyperlink{XV.3}{XV.3}{}; t.-d. element (\emph{phoṭṭhabba-dhātu}) \hyperlink{XV.17}{XV.17}{}, \hyperlink{XV.30}{30}{}
\par\textbf{Tāvatiṃsā Thirty-three} (Gods’) Heaven \hyperlink{VII.43}{VII.43f.}{}; \hyperlink{XII.72}{XII.72}{}, \hyperlink{XII.108}{108}{}; \hyperlink{XIII.41}{XIII.41}{}
\par\textbf{teacher} (\emph{ācariya}) \hyperlink{III.47}{III.47}{}, \hyperlink{III.126}{126}{}
\par\textbf{teaching} (\emph{desanā}) \hyperlink{I.126}{I.126}{}
\par\textbf{temperament, behaviour} (\emph{carita, cariya}) \hyperlink{II.86}{II.86}{}; \hyperlink{III.74}{III.74f.}{}, \hyperlink{III.95}{95}{}, \hyperlink{III.121}{121}{}; \hyperlink{VI.83}{VI.83}{}, \hyperlink{VI.85}{85}{}; \hyperlink{VIII.112}{VIII.112}{}, \hyperlink{VIII.159}{159}{}; \hyperlink{XVII.286}{XVII.286}{}
\par\textbf{temperature} (\emph{utu}) \hyperlink{I.86}{I.86}{}; \hyperlink{XIV.47}{XIV.47}{}, \hyperlink{XIV.188}{188}{}; \hyperlink{XVII.106}{XVII.106}{}, \hyperlink{XVII.193}{193}{}; \hyperlink{XVIII.5}{XVIII.5}{}; \hyperlink{XIX.9}{XIX.9}{}; t.-originated (\emph{utu-samuṭṭhāna}) \hyperlink{XIV.61}{XIV.61}{}, \hyperlink{XIV.75}{75}{}; \hyperlink{XVII.193}{XVII.193}{}; \hyperlink{XIX.9}{XIX.9}{}; \hyperlink{XX.27}{XX.27}{}, \hyperlink{XX.39}{39f.}{}
\par\textbf{terms of, in} (\emph{adhiṭṭhāna}) \hyperlink{I.52}{I.52}{}; \hyperlink{IV.n27}{IV.n.27}{}
\par\textbf{terror} (\emph{bhaya}) \hyperlink{XX.15}{XX.15}{}; \hyperlink{XXI.26}{XXI.26}{}, \hyperlink{XXI.29}{29f.}{}, \hyperlink{XXI.61}{61}{}, \hyperlink{XXI.69}{69}{} see also appearance as t., and fear
\par\textbf{theory} (\emph{paṭipatti}) \hyperlink{XIV.163}{XIV.163}{}, \hyperlink{XIV.177}{177}{}; \hyperlink{XVI.85}{XVI.85}{}; \hyperlink{XVII.52}{XVII.52}{}, \hyperlink{XVII.303}{303}{}
\par\textbf{Therambatthala} \hyperlink{IV.135}{IV.135}{}; \hyperlink{XII.9}{XII.9}{}
\par\textbf{thing} (\emph{dhamma}) \hyperlink{VII.n1}{VII.n.1}{}. See also dhamma
\par\textbf{thinking} (\emph{saṅkappa}) \hyperlink{XXII.42}{XXII.42}{}, \hyperlink{XXII.66}{66}{}. See also right t. 
\par\textbf{Thirty-three Gods} , see Tāvatiṃsa
\par\textbf{thirty-two aspects of the body} (\emph{dvattiṃsākāra}) \hyperlink{III.1}{III.1}{}, \hyperlink{III.105}{105}{}; \hyperlink{VII.28}{VII.28}{}; \hyperlink{VIII.44}{VIII.44f.}{}; \hyperlink{XVIII.5}{XVIII.5}{}; \hyperlink{XX.9}{XX.9}{}
\par\textbf{thought, thought-arising} (\emph{cittuppāda}) \hyperlink{IV.87}{IV.87}{}; \hyperlink{XXII.63}{XXII.63}{}, \hyperlink{XXII.76}{76}{}
\par\textbf{Thūpārāma} \hyperlink{III.31}{III.31}{}
\par\textbf{tie} (\emph{gantha}) \hyperlink{IV.87}{IV.87}{}; \hyperlink{VII.59}{VII.59}{}; \hyperlink{XIV.202}{XIV.202}{}, \hyperlink{XIV.226}{226}{}; \hyperlink{XXII.54}{XXII.54}{}, \hyperlink{XXII.69}{69}{}
\par\textbf{time} (\emph{kā1a}) \hyperlink{VII.n7}{VII.n.7}{}; \hyperlink{VIII.32}{VIII.32}{}; \hyperlink{XIV.n71}{XIV.n.71}{}; \hyperlink{XVI.85}{XVI.85}{}; \hyperlink{XVII.75}{XVII.75}{}, \hyperlink{XVII.n3}{n.3}{}
\par\textbf{Tissa-macca-mātar} \hyperlink{II.16}{II.16}{}
\par\textbf{Tissamahāvihāra} \hyperlink{XII.80}{XII.80}{}
\par\textbf{Tissa Thera, Ciragumbavasik-ambakhādakamahā} \hyperlink{I.122}{I.122}{}, \hyperlink{I.133}{133}{}
\par\textbf{Tissa Thera, Cūḷa-piṇḍapātika} \hyperlink{III.127}{III.127}{}; \hyperlink{VI.77}{VI.77}{}
\par\textbf{Tissa Thera, Koṭapabbatavāsin} \hyperlink{VIII.243}{VIII.243}{}
\par\textbf{Tissa Thera, Kuṭumbiyaputta} \hyperlink{I.137}{I.137}{}
\par\textbf{Tissa Thera, Mahā} \hyperlink{I.55}{I.55}{}; \hyperlink{IV.95}{IV.95}{}; \hyperlink{VI.81}{VI.81}{}, \hyperlink{VI.88}{88}{}; \hyperlink{XII.89}{XII.89}{}
\par\textbf{Tissa Thera, Mahā, Mahā-Karañjiya-vihāra-vāsin} \hyperlink{VIII.243}{VIII.243}{}
\par\textbf{Tissa Thera, Padhāniya, Nāgapabbatavāsin} \hyperlink{IV.36}{IV.36}{}
\par\textbf{Tissa Thera, Piṇḍapātika, Devaputtaraṭṭhavāsin} \hyperlink{VIII.243}{VIII.243}{}
\par\textbf{Tissadatta Thera} \hyperlink{XII.124}{XII.124}{}
\par\textbf{tongue} (\emph{jivhā}) \hyperlink{XIV.40}{XIV.40}{}, \hyperlink{XIV.46}{46}{}, \hyperlink{XIV.51}{51}{}, \hyperlink{XIV.117}{117}{}; \hyperlink{XVI.10}{XVI.10}{}; \hyperlink{XVII.156}{XVII.156}{}; \hyperlink{XX.70}{XX.70}{}; t. base (\emph{jivhāyatana}) \hyperlink{XV.3}{XV.3}{}; t. consciousness (\emph{jivhā-viññāṇa}) \hyperlink{XIV.96}{XIV.96}{}, \hyperlink{XIV.117}{117}{}, \hyperlink{XIV.179}{179}{}; t.-c. element (\emph{jivhāviññāṇa-dhātu}) \hyperlink{XV.17}{XV.17}{}; t. element (\emph{jivhā-dhātu}) \hyperlink{XV.17}{XV.17}{}; t. faculty (\emph{jivhindriya}) \hyperlink{XVI.1}{XVI.1}{}
\par\textbf{torpor} (\emph{middha}) \hyperlink{III.95}{III.95}{}; \hyperlink{XIV.71}{XIV.71}{}
\par\textbf{trainer} (\emph{sekha}) \hyperlink{I.35}{I.35}{}, \hyperlink{I.37}{37}{}, \hyperlink{I.127}{127}{}, \hyperlink{I.131}{131}{}, \hyperlink{I.137}{137}{}; \hyperlink{II.78}{II.78}{}; \hyperlink{XI.121}{XI.121}{}; \hyperlink{XIV.27}{XIV.27}{}, \hyperlink{XIV.109}{109}{}; \hyperlink{XVI.104}{XVI.104}{}; \hyperlink{XVII.81}{XVII.81}{}; \hyperlink{XXII.21}{XXII.21}{}; \hyperlink{XXIII.10}{XXIII.10}{}
\par\textbf{training} (\emph{sikkhā}) \hyperlink{I.10}{I.10}{}; t. precept (\emph{sikkhā-pada}) \hyperlink{I.40}{I.40}{}, \hyperlink{I.52}{52}{}, \hyperlink{I.98}{98}{}, \hyperlink{I.131}{131}{}; \hyperlink{XIV.8}{XIV.8}{}; training rule, minor (\emph{sekhiya-dhamma}) \hyperlink{I.52}{I.52}{}; \hyperlink{II.29}{II.29}{}
\par\textbf{tranquillity} (\emph{passaddhi}) \hyperlink{I.32}{I.32}{}, \hyperlink{I.140}{140}{}; \hyperlink{IV.51}{IV.51}{}, \hyperlink{IV.99}{99}{}; \hyperlink{XIV.128}{XIV.128}{}, \hyperlink{XIV.133}{133}{}, \hyperlink{XIV.144}{144}{}; \hyperlink{XVI.86}{XVI.86}{}; \hyperlink{XX.116}{XX.116}{}; \hyperlink{XXI.75}{XXI.75}{}, \hyperlink{XXI.89}{89}{}; \hyperlink{XXII.42}{XXII.42}{}
\par\textbf{transformation} (\emph{vikubbana}) \hyperlink{XII.2}{XII.2}{}, \hyperlink{XII.137}{137}{}. See also versatility 
\par\textbf{transgression} (\emph{vītikkama}) \hyperlink{I.13}{I.13}{}, \hyperlink{I.44}{44}{}
\par\textbf{transmigration} (\emph{saṅkanti, saṅkamana}) \hyperlink{XVII.113}{XVII.113}{}, \hyperlink{XVII.162}{162}{}, \hyperlink{XVII.302}{302}{}
\par\textbf{treasures, the seven} (\emph{satta dhanāni}) \hyperlink{XXII.14}{XXII.14}{}
\par\textbf{tree-root dweller} (\emph{rukkhamūlika}) \hyperlink{I.68}{I.68}{}; \hyperlink{II.2}{II.2f.}{}, \hyperlink{II.56}{56}{}
\par\textbf{triad} (\emph{tika}) \hyperlink{II.n18}{II.n.18}{}; \hyperlink{XIII.104}{XIII.104}{}, \hyperlink{XIII.n20}{n.20}{}
\par\textbf{triple continuity} (\emph{ti-santati}) \hyperlink{XI.112}{XI.112}{}; \hyperlink{XX.22}{XX.22}{}
\par\textbf{triple origination} (\emph{ti-samuṭṭhāna}) \hyperlink{XVII.196}{XVII.196}{}
\par\textbf{triple-robe wearer} (\emph{ti-cīvarika}) \hyperlink{II.2}{II.2}{}, \hyperlink{II.23}{23f.}{}
\par\textbf{truth} (\emph{sacca}) \hyperlink{VII.27}{VII.27}{}, \hyperlink{VII.62}{62}{}, \hyperlink{VII.n1}{n.1}{}; \hyperlink{XIV.218}{XIV.218}{}; \hyperlink{XVI.3}{XVI.3}{}, \hyperlink{XVI.13}{13f.}{}; \hyperlink{XVII.59}{XVII.59}{}, \hyperlink{XVII.300}{300}{}; \hyperlink{XVIII.n8}{XVIII.n.8}{}; \hyperlink{XX.98}{XX.98}{}, \hyperlink{XX.100}{100}{}, \hyperlink{XX.130}{130}{}; \hyperlink{XXI.1}{XXI.1}{}, \hyperlink{XXI.130}{130}{}; \hyperlink{XXII.7}{XXII.7}{}, \hyperlink{XXII.92}{92}{}
\par\textbf{Tulādhārapabbata-vihāra} \hyperlink{III.53}{III.53}{}
\par\textbf{turning away} (\emph{vivaṭṭa}) \hyperlink{I.140}{I.140}{}; \hyperlink{VI.43}{VI.43}{}. See also contemplation of t. a. 
\par\textbf{twin marvel} (\emph{yamaka-pāṭihāriya}) \hyperlink{IV.132}{IV.132}{}; \hyperlink{VII.n7}{VII.n.7}{}; \hyperlink{XII.72}{XII.72}{}, \hyperlink{XII.84}{84}{}
\par\textbf{Uccāvālika} \hyperlink{XX.110}{XX.110f.}{}
\par\textbf{Uddhaṃsota} \hyperlink{XXIII.56}{XXIII.56}{}
\par\textbf{Udena-rājā} \hyperlink{XII.35}{XII.35}{}
\par\textbf{Ugga} \hyperlink{VIII.18}{VIII.18}{}
\par\textbf{ultimate sense} (\emph{paramattha}) \hyperlink{I.n14}{I.n.14}{}; \hyperlink{II.n18}{II.n.18}{}; \hyperlink{VIII.39}{VIII.39}{}; \hyperlink{XVI.n18}{XVI.n.18}{}; \hyperlink{XX.72}{XX.72}{}, \hyperlink{XX.n20}{n.20}{}
\par\textbf{uncertainty} (\emph{vicikicchā}) \hyperlink{I.140}{I.140}{}; \hyperlink{III.95}{III.95}{}; \hyperlink{IV.86}{IV.86}{}, \hyperlink{IV.104}{104}{}; \hyperlink{XIV.93}{XIV.93}{}, \hyperlink{XIV.176}{176}{}, \hyperlink{XIV.177}{177f.}{}; \hyperlink{XIX.6}{XIX.6}{}, \hyperlink{XIX.10}{10}{}; \hyperlink{XXII.48}{XXII.48}{}, \hyperlink{XXII.60}{60}{}
\par\textbf{unconcern} (\emph{anābhoga}) \hyperlink{IV.171}{IV.171}{}; \hyperlink{IX.108}{IX.108}{}
\par\textbf{unconscious beings} (\emph{asañña-satta}) \hyperlink{VIII.n57}{VIII.n.57}{}; \hyperlink{XVII.134}{XVII.134}{}
\par\textbf{understanding} (\emph{paññā}) \hyperlink{I.7}{I.7}{}, \hyperlink{I.123}{123}{}; \hyperlink{III.15}{III.15}{}; \hyperlink{IV.45}{IV.45}{}, \hyperlink{IV.117}{117f.}{}; \hyperlink{V.41}{V.41}{}; \hyperlink{VII.n8}{VII.n.8}{}, \hyperlink{VII.n9}{n.9}{}; \hyperlink{VIII.111}{VIII.111}{}, \hyperlink{VIII.173}{173}{}; \hyperlink{IX.124}{IX.124}{}; \hyperlink{XII.17}{XII.17}{}; \hyperlink{XIV.2}{XIV.2f.}{}; \hyperlink{XVI.1}{XVI.1}{}, \hyperlink{XVI.86}{86}{}, \hyperlink{XVI.99}{99}{}; \hyperlink{XX.3}{XX.3}{}, \hyperlink{XX.7}{7}{}; \hyperlink{XXI.12}{XXI.12}{}, \hyperlink{XXI.37}{37}{}, \hyperlink{XXI.74}{74f.}{}, \hyperlink{XXI.89}{89}{}; \hyperlink{XXII.42}{XXII.42}{}, \hyperlink{XXII.45}{45}{}, \hyperlink{XXII.98}{98}{}; \hyperlink{XIII.2}{XIII.2}{}
\par\textbf{unformed} (\emph{asaṅkhata}) \hyperlink{VII.85}{VII.85}{}; \hyperlink{VIII.245}{VIII.245}{}; \hyperlink{XV.25}{XV.25}{}; \hyperlink{XVI.23}{XVI.23}{}, \hyperlink{XVI.102}{102}{}; u. element (\emph{asaṅkhata-dhātu}) \hyperlink{XV.34}{XV.34}{}, \hyperlink{XV.42}{42}{}; \hyperlink{XVI.94}{XVI.94}{}; \hyperlink{XXII.n1}{XXII.n.1}{}
\par\textbf{unhappy destiny} (\emph{duggati}) \hyperlink{XVII.135}{XVII.135f.}{}, \hyperlink{XVII.160}{160}{}
\par\textbf{unification} (\emph{ekaggatā}) \hyperlink{III.2}{III.2}{}, \hyperlink{III.n2}{n.2}{}; \hyperlink{IV.74}{IV.74}{}; \hyperlink{VIII.232}{VIII.232}{}; \hyperlink{XI.119}{XI.119}{}; \hyperlink{XXI.114}{XXI.114}{}; \hyperlink{XXII.36}{XXII.36}{}, \hyperlink{XXII.46}{46}{}; \hyperlink{XXIII.20}{XXIII.20}{}, \hyperlink{XXIII.26}{26}{}
\par\textbf{unincluded} (\emph{apariyāpanna}) \hyperlink{III.23}{III.23}{}
\par\textbf{uninterest} (\emph{avyāpāra}) \hyperlink{XVII.312}{XVII.312}{}, \hyperlink{XVII.n14}{n.14}{}; \hyperlink{XX.102}{XX.102}{}
\par\textbf{unity} (\emph{ekatta}) \hyperlink{IV.112}{IV.112f.}{}, \hyperlink{IV.n31}{n.31}{}; \hyperlink{XI.95}{XI.95}{}; \hyperlink{XXII.97}{XXII.97}{}
\par\textbf{universal} , see kasiṇa
\par\textbf{unknowing} (\emph{aññāṇa}) \hyperlink{I.57}{I.57}{}; \hyperlink{IX.96}{IX.96}{}, \hyperlink{IX.110}{110}{}
\par\textbf{unprofitable} (\emph{akusala}) \hyperlink{I.42}{I.42}{}, \hyperlink{I.52}{52}{}; \hyperlink{IV.85}{IV.85}{}; \hyperlink{VII.59}{VII.59}{}, \hyperlink{VII.n25}{n.25}{}; \hyperlink{X.16}{X.16}{}; \hyperlink{XIII.64}{XIII.64}{}; \hyperlink{XIV.23}{XIV.23}{}, \hyperlink{XIV.89}{89f.}{}, \hyperlink{XIV.101}{101f.}{}, \hyperlink{XIV.113}{113}{}, \hyperlink{XIV.117}{117}{}, \hyperlink{XIV.126}{126f.}{}, \hyperlink{XIV.129}{129}{}, \hyperlink{XIV.132}{132}{}, \hyperlink{XIV.159}{159f.}{}, \hyperlink{XIV.179}{179}{}, \hyperlink{XIV.193}{193}{}, \hyperlink{XIV.199}{199}{}, \hyperlink{XIV.205}{205}{}, \hyperlink{XIV.209}{209}{}; \hyperlink{XVI.104}{XVI.104}{}; \hyperlink{XVII.120}{XVII.120}{}; \hyperlink{XIX.8}{XIX.8}{}; \hyperlink{XX.28}{XX.28}{}, \hyperlink{XX.31}{31}{}, \hyperlink{XX.44}{44}{}, \hyperlink{XX.124}{124}{}; \hyperlink{XXII.35}{XXII.35}{}, \hyperlink{XXII.62}{62}{}, \hyperlink{XXII.75}{75}{}, \hyperlink{XXII.85}{85}{}
\par\textbf{unseen} (\emph{adiṭṭha}) \hyperlink{XVII.202}{XVII.202}{}; \hyperlink{XXI.17}{XXI.17}{}, \hyperlink{XXI.21}{21}{}
\par\textbf{unworldly} (\emph{nirāmisa}) \hyperlink{XXI.37}{XXI.37}{}
\par\textbf{upahaccaparinibbāyin} \hyperlink{XXIII.56}{XXIII.56}{}
\par\textbf{Upananda Thera} \hyperlink{II.82}{II.82}{}
\par\textbf{Upatissa Thera} \hyperlink{III.n19}{III.n.19}{}
\par\textbf{uposatha} (\emph{uposatha}–observance day) \hyperlink{I.40}{I.40}{}, \hyperlink{I.n10}{n.10}{}; \hyperlink{II.60}{II.60}{}; \hyperlink{VII.125}{VII.125}{}; \hyperlink{XVII.81}{XVII.81}{}; \hyperlink{XXI.n15}{XXI.n.15}{}
\par\textbf{Uppalavaṇṇā Theri} \hyperlink{XXII.83}{XXII.83}{}
\par\textbf{urgency} (\emph{saṃvega}), see sense of u. 
\par\textbf{use} (\emph{paribhoga}) \hyperlink{I.124}{I.124f.}{}
\par\textbf{Uttarakuru} \hyperlink{I.41}{I.41}{}; \hyperlink{VII.43}{VII.43f.}{}; \hyperlink{XII.73}{XII.73}{}
\par\textbf{Uttara-mātar} \hyperlink{XII.39}{XII.39}{}
\par\textbf{Vakkali Thera} \hyperlink{IV.45}{IV.45}{}
\par\textbf{Vaṅgīsa Thera} \hyperlink{I.103}{I.103}{}
\par\textbf{vanity} (\emph{mada}) \hyperlink{II.67}{II.67}{}; \hyperlink{VII.59}{VII.59}{}; \hyperlink{VIII.247}{VIII.247}{}, \hyperlink{VIII.n71}{n.71}{}. See also intoxication, conceit 
\par\textbf{vanity, personal} (\emph{cāpalya}) \hyperlink{III.95}{III.95}{}, \hyperlink{III.n22}{n.22}{}
\par\textbf{variety} (\emph{nānatta}) \hyperlink{I.140}{I.140}{}; \hyperlink{X.12}{X.12}{}, \hyperlink{X.20}{20}{}; \hyperlink{XI.95}{XI.95f.}{}
\par\textbf{Vattaniya-senāsana} \hyperlink{XIII.107}{XIII.107}{}
\par\textbf{Vasudhamma} \hyperlink{XIV.n16}{XIV.n.16}{}
\par\textbf{Vasudeva} \hyperlink{VIII.19}{VIII.19}{}; \hyperlink{XV.5}{XV.5}{}
\par\textbf{Vattakālakagāma} \hyperlink{IV.96}{IV.96}{}
\par\textbf{Veda} (\emph{veda}) \hyperlink{XII.44}{XII.44}{}
\par\textbf{Vehapphala (Great Fruit) Deities} \hyperlink{XIII.62}{XIII.62}{}
\par\textbf{Vejayanta Palace} \hyperlink{VIII.20}{VIII.20}{}; \hyperlink{XII.110}{XII.110}{}
\par\textbf{verbal formation} (\emph{vacī-saṅkhāra}) \hyperlink{XVII.61}{XVII.61}{}; \hyperlink{XXIII.24}{XXIII.24}{}, \hyperlink{XXIII.51}{51}{}; v. intimation (\emph{vacī-viññatti}) \hyperlink{XIV.61}{XIV.61}{}; v. misconduct (\emph{vacī-duccarita}) \hyperlink{XIV.133}{XIV.133}{}; v. volition (\emph{vacī-sañcetanā}) \hyperlink{XVII.61}{XVII.61}{}
\par\textbf{versatility} (\emph{vikubbana}) \hyperlink{IX.44}{IX.44}{}. See also transformation 
\par\textbf{vehicle} (\emph{yāna}) \hyperlink{XVIII.3}{XVIII.3}{}, \hyperlink{XVIII.5}{5}{}
\par\textbf{Vibhajjavādin} \hyperlink{XVII.25}{XVII.25}{}
\par\textbf{view} (\emph{diṭṭhi}) \hyperlink{I.13}{I.13}{}, \hyperlink{I.137}{137}{}, \hyperlink{I.140}{140}{}; \hyperlink{III.78}{III.78}{}; \hyperlink{VII.59}{VII.59}{}, \hyperlink{VII.n25}{n.25}{}; \hyperlink{XIII.74}{XIII.74}{}; \hyperlink{XIV.90}{XIV.90}{}, \hyperlink{XIV.146}{146}{}, \hyperlink{XIV.205}{205}{}, \hyperlink{XIV.218}{218}{}, \hyperlink{XIV.229}{229}{}; \hyperlink{XVI.93}{XVI.93}{}; \hyperlink{XVII.240}{XVII.240f.}{}, \hyperlink{XVII.265}{265}{}, \hyperlink{XVII.286}{286}{}, \hyperlink{XVII.310}{310f.}{}; \hyperlink{XIX.24}{XIX.24}{}; \hyperlink{XX.82}{XX.82f.}{}, \hyperlink{XX.125}{125}{}; \hyperlink{XXI.26}{XXI.26}{}, \hyperlink{XXI.42}{42}{}, \hyperlink{XXI.92}{92}{}; \hyperlink{XXII.48}{XXII.48}{}, \hyperlink{XXII.60}{60}{}. See also wrong v. 
\par\textbf{village} (\emph{gāma}) \hyperlink{II.48}{II.48}{}; \hyperlink{VIII.158}{VIII.158}{}
\par\textbf{Vimuttimagga} \hyperlink{III.n19}{III.n.19}{}
\par\textbf{Vinatakapabbata} \hyperlink{VII.42}{VII.42}{}
\par\textbf{Vipassin Bhagavant} \hyperlink{XIII.123}{XIII.123}{}
\par\textbf{virtue} (\emph{sīla}) \hyperlink{I}{I passim}{}, \hyperlink{I.19}{19f.}{}; \hyperlink{II}{II passim}{}; \hyperlink{VII.7}{VII.7}{}, \hyperlink{VII.101}{101f.}{}; \hyperlink{VIII.173}{VIII.173}{}; \hyperlink{IX.124}{IX.124}{}; \hyperlink{XIV.206}{XIV.206}{}, \hyperlink{XIV.219}{219}{}; \hyperlink{XVI.86}{XVI.86}{}; \hyperlink{XVII.60}{XVII.60}{}, \hyperlink{XVII.81}{81}{}; \hyperlink{XVIII.1}{XVIII.1}{}; \hyperlink{XXII.128}{XXII.128}{}
\par\textbf{Visākha Thera} \hyperlink{IX.64}{IX.64f.}{}
\par\textbf{visible} (\emph{sanidassana}) \hyperlink{XIV.74}{XIV.74f.}{} visible datum, visible object (\emph{rūpa}) \hyperlink{I.20}{I.20}{}, \hyperlink{I.53}{53}{}, \hyperlink{I.57}{57}{}, \hyperlink{I.n14}{n.14}{}; \hyperlink{III.109}{III.109}{}; \hyperlink{VII.28}{VII.28}{}; \hyperlink{X.16}{X.16}{}; \hyperlink{XIII.101}{XIII.101}{}; \hyperlink{XIV.54}{XIV.54}{}, \hyperlink{XIV.74}{74}{}, \hyperlink{XIV.96}{96}{}, \hyperlink{XIV.99}{99}{}, \hyperlink{XIV.107}{107}{}, \hyperlink{XIV.115}{115}{}, \hyperlink{XIV.134}{134}{}; \hyperlink{XVII.127}{XVII.127}{}, \hyperlink{XVII.180}{180}{}; \hyperlink{XVIII.11}{XVIII.11}{}; \hyperlink{XX.44}{XX.44}{}; v.-d. base (\emph{rūpāyatana}) \hyperlink{XIV.76}{XIV.76}{}; \hyperlink{XV.3}{XV.3}{}; v.-d. element (\emph{rūpa-dhātu}) \hyperlink{XV.17}{XV.17}{}
\par\textbf{Vissakamma} \hyperlink{XII.71}{XII.71}{}, \hyperlink{XII.77}{77}{}, \hyperlink{XII.n14}{n.14}{}
\par\textbf{vital formation} (\emph{āyu-saṅkhāra}) \hyperlink{VIII.244}{VIII.244}{}; \hyperlink{XXIII.42}{XXIII.42}{}
\par\textbf{void} (\emph{suñña}) \hyperlink{X.33}{X.33}{}; \hyperlink{XVI.90}{XVI.90}{}, \hyperlink{XVI.n25}{n.25}{}; \hyperlink{XVII.283}{XVII.283}{}; \hyperlink{XX.47}{XX.47}{}; \hyperlink{XXI.24}{XXI.24}{}, \hyperlink{XXI.34}{34}{}, \hyperlink{XXI.53}{53f.}{}, \hyperlink{XXI.69}{69f.}{}, \hyperlink{XXI.121}{121}{}, \hyperlink{XXI.123}{123f.}{}; v. liberation (\emph{suññata-vimokkha}) \hyperlink{XXI.70}{XXI.70f.}{}, \hyperlink{XXI.89}{89}{}
\par\textbf{voidness} (\emph{suññatā}) \hyperlink{II.40}{II.40}{}; \hyperlink{VII.n1}{VII.n.1}{}; \hyperlink{XI.117}{XI.117}{}, \hyperlink{XI.n20}{n.20}{}; v. element (\emph{suññatā-dhātu}) \hyperlink{XXI.67}{XXI.67}{}
\par\textbf{volition} (\emph{cetanā}) \hyperlink{I.17}{I.17}{}, \hyperlink{I.140}{140}{}; \hyperlink{II.12}{II.12}{}, \hyperlink{II.83}{83}{}, \hyperlink{II.89}{89}{}; \hyperlink{VII.28}{VII.28}{}; \hyperlink{XI.1}{XI.1f.}{}, \hyperlink{XI.n2}{n.2}{}; \hyperlink{XIV.133}{XIV.133}{}, \hyperlink{XIV.135}{135}{}, \hyperlink{XIV.159}{159}{}, \hyperlink{XIV.170}{170}{}, \hyperlink{XIV.176}{176}{}, \hyperlink{XIV.179}{179}{}, \hyperlink{XIV.n81}{n.81}{}; \hyperlink{XVI.10}{XVI.10}{}; \hyperlink{XVII.44}{XVII.44}{}, \hyperlink{XVII.5}{5}{}\hyperlink{I.60}{I.60f.}{}, \hyperlink{I.88}{88}{}, \hyperlink{I.251}{251}{}, \hyperlink{I.292}{292f.}{}; \hyperlink{XVIII.8}{XVIII.8}{}, \hyperlink{XVIII.19}{19}{}; \hyperlink{XX.9}{XX.9}{}, \hyperlink{XX.29}{29}{}; \hyperlink{XXII.66}{XXII.66}{}
\par\textbf{water} (\emph{udaka}) \hyperlink{XIII.43}{XIII.43}{}, \hyperlink{XIII.67}{67}{}; (\emph{āpo}) \hyperlink{XI.35}{XI.35}{}, \hyperlink{XI.41}{41}{}, \hyperlink{XI.87}{87}{}; \hyperlink{XIII.30}{XIII.30}{}, \hyperlink{XIII.56}{56}{}; \hyperlink{XIV.35}{XIV.35}{}, \hyperlink{XIV.73}{73}{}, \hyperlink{XIV.n32}{n.32}{}; w. element (\emph{āpo-dhātu}) \hyperlink{XI.28}{XI.28f.}{}, \hyperlink{XI.87}{87f.}{}; w. kasiṇa, (\emph{āpo-kasiṇa}) \hyperlink{III.105}{III.105}{}; \hyperlink{V.1}{V.1}{}; \hyperlink{XII.92}{XII.92}{}
\par\textbf{way} (\emph{paṭipadā}) \hyperlink{III.42}{III.42}{}, \hyperlink{III.n15}{n.15}{}; \hyperlink{VII.74}{VII.74}{}, \hyperlink{VII.90}{90}{}; \hyperlink{XVI.75}{XVI.75f.}{}; (\emph{patha}) \hyperlink{I.85}{I.85}{}
\par\textbf{wheel of becoming} (\emph{bhava-cakka}) \hyperlink{XVII.273}{XVII.273f.}{}
\par\textbf{wheel of the round of rebirths} (\emph{saṃsāracakka}) \hyperlink{VII.7}{VII.7f.}{}
\par\textbf{wheel-turning monarch} (\emph{cakkavattin}) \hyperlink{VIII.n61}{VIII.n.61}{}; \hyperlink{XII.40}{XII.40}{}
\par\textbf{white kasiṇa} (\emph{odāta-kasiṇa}) \hyperlink{III.105}{III.105f.}{}; \hyperlink{V.16}{V.16}{}; \hyperlink{XIII.95}{XIII.95}{}
\par\textbf{wieldiness} (\emph{kammaññatā}) \hyperlink{XIV.64}{XIV.64}{}, \hyperlink{XIV.133}{133}{}, \hyperlink{XIV.147}{147}{}
\par\textbf{wilderness in the heart} (\emph{ceto-khila}) \hyperlink{VII.59}{VII.59}{}
\par\textbf{wisdom} (\emph{veda}) \hyperlink{XXII.70}{XXII.70}{}, \hyperlink{XXII.75}{75}{}, \hyperlink{XXII.89}{89}{}
\par\textbf{wishes} , see evilness of w. \& fewness of w.
\par\textbf{woman} (\emph{itthi}) \hyperlink{I.n14}{I.n.14}{}; \hyperlink{XVII.n4}{XVII.n.4}{}
\par\textbf{world} (\emph{loka}) \hyperlink{I.34}{I.34}{}; \hyperlink{III.n5}{III.n.5}{}; \hyperlink{VII.36}{VII.36f.}{}; \hyperlink{VIII.39}{VIII.39}{}, \hyperlink{VIII.n11}{n.11}{}; \hyperlink{XIII.94}{XIII.94}{}; \hyperlink{XIV.n36}{XIV.n.36}{}; \hyperlink{XVI.85}{XVI.85}{}, \hyperlink{XVI.n23}{n.23}{}; \hyperlink{XVII.134}{XVII.134}{}; \hyperlink{XX.72}{XX.72}{}; world apex, world shrine (\emph{loka-thūpika}) \hyperlink{XVI.85}{XVI.85}{}; w. element (\emph{loka-dhātu}) \hyperlink{VII.44}{VII.44}{}, \hyperlink{VII.n14}{n.14}{}; \hyperlink{XII.78}{XII.78}{}, \hyperlink{XII.106}{106}{};
\par\textbf{world-marshal deities} (\emph{loka-vyūha-devā}) \hyperlink{XIII.34}{XIII.34}{}; w. inter-space (\emph{lokantara}) \hyperlink{VII.n14}{VII.n.14}{}; \hyperlink{XVI.43}{XVI.43}{}; w. soul, Puruṣa (\emph{purisa}) \hyperlink{XVII.8}{XVII.8}{}, \hyperlink{XVII.n3}{n.3}{}; \hyperlink{XVIII.n9}{XVIII.n.9}{}; w. sphere (\emph{cakka-vāla}) \hyperlink{VII.40}{VII.40}{}, \hyperlink{VII.44}{44}{}, \hyperlink{VII.n14}{n.14}{}; \hyperlink{IX.103}{IX.103}{}; \hyperlink{X.6}{X.6}{}; \hyperlink{XII.72}{XII.72}{}, \hyperlink{XII.78}{78}{}, \hyperlink{XII.88}{88}{}; \hyperlink{XIII.3}{XIII.3}{}, \hyperlink{XIII.31}{31}{}, \hyperlink{XIII.48}{48f.}{}
\par\textbf{worldliness} (\emph{āmisa}) \hyperlink{XXI.n13}{XXI.n.13}{}
\par\textbf{worldly} (\emph{sāmisa}) \hyperlink{XXI.37}{XXI.37}{}, \hyperlink{XXI.41}{41}{}
\par\textbf{worldly state} (\emph{loka-dhamma}) \hyperlink{VII.38}{VII.38}{}; \hyperlink{XXII.51}{XXII.51}{}, \hyperlink{XXII.67}{67}{}
\par\textbf{worm} (\emph{kimi}) \hyperlink{VI.77}{VI.77}{}; \hyperlink{VIII.25}{VIII.25}{}, \hyperlink{VIII.121}{121}{}
\par\textbf{worry} (\emph{kukkucca}) \hyperlink{III.95}{III.95}{}; \hyperlink{XIV.170}{XIV.170}{}, \hyperlink{XIV.174}{174}{}. See also agitation and w.
\par\textbf{wrongdoing} (\emph{dukkata}) \hyperlink{I.60}{I.60}{}
\par\textbf{wrongness} (\emph{micchatta}) \hyperlink{VII.59}{VII.59}{}; \hyperlink{XVII.53}{XVII.53}{}; \hyperlink{XXII.50}{XXII.50}{}, \hyperlink{XXII.66}{66}{}
\par\textbf{wrong path} (\emph{micchā-magga}) \hyperlink{XXII.14}{XXII.14}{}; w. speech (\emph{micchā-vācā}) \hyperlink{XXII.50}{XXII.50}{}; w. view (\emph{micchā-diṭṭhi}) \hyperlink{I.140}{I.140}{}; \hyperlink{V.41}{V.41}{}; \hyperlink{XIV.159}{XIV.159}{}, \hyperlink{XIV.164}{164}{}; \hyperlink{XVII.9}{XVII.9}{}, \hyperlink{XVII.243}{243}{}; \hyperlink{XXII.45}{XXII.45}{}, \hyperlink{XXII.50}{50}{}, \hyperlink{XXII.56}{56}{}, \hyperlink{XXII.58}{58}{}, \hyperlink{XXII.66}{66}{}
\par\textbf{Yama-rājā} (King of the Underworld) \hyperlink{VII.n14}{VII.n.14}{}
\par\textbf{Yasa Thera} \hyperlink{XII.82}{XII.82}{}
\par\textbf{Yuddhiṭṭhila} \hyperlink{VIII.19}{VIII.19}{}
\par\textbf{Yugandharapabbata} \hyperlink{VII.42}{VII.42}{}; \hyperlink{XII.72}{XII.72}{}, \hyperlink{XII.121}{121}{}
\par\textbf{zeal} (\emph{chanda}) \hyperlink{I.33}{I.33}{}; \hyperlink{III.24}{III.24}{}; \hyperlink{IV.85}{IV.85}{}, \hyperlink{IV.n24}{n.24}{}; \hyperlink{IX.102}{IX.102}{}; \hyperlink{XII.12}{XII.12}{}, \hyperlink{XII.50}{50}{}; \hyperlink{XIV.133}{XIV.133}{}, \hyperlink{XIV.150}{150}{}, \hyperlink{XIV.159}{159}{}, \hyperlink{XIV.170}{170}{}; \hyperlink{XVI.86}{XVI.86}{}; \hyperlink{XVII.72}{XVII.72}{}; \hyperlink{XXII.36}{XXII.36}{}, \hyperlink{XXII.39}{39}{}, \hyperlink{XXII.42}{42}{}, \hyperlink{XXII.55}{55}{}
\end{vismHanging}\end{multicols}\chapter[Pali-English Glossary]{Pali-English Glossary\* {\large of Some Subjects and Technical Terms}}This Glossary only includes (a) some epistemological and technical terms, and (b) meanings or words not in the PED, which are marked with an asterisk(*), though such compounds prefixes as e.g. \emph{anukaḍḍhati} = to keep dragging along (\hyperlink{III.68}{III.68}{}) or suffixes as e.g. \emph{vattabbatā} = ability to be called (\hyperlink{IV.148}{IV.148}{}) and such verbal substantives as \emph{udikkhana} from \emph{udikkhati} are not always included.\begin{multicols}{2}
\parskip=.2\baselineskip\RaggedRight\begin{vismHanging}
\par\textbf{akiriya-diṭṭhi} (moral-) inefficacy-of-action view 
\par\textbf{akusala} (1) unskilful, (2) (kammically) un-profitable 
\par\textbf{agati} bad way (the four)
\par\textbf{*agaru} aloe wood (spelled agaḷu in PED); \hyperlink{VIII.47}{VIII.47}{} 
\par\textbf{aṅga} (l) limb, (2) factor (of path, jhāna, etc.), (3) practice, etc. 
\par\textbf{ajjhatta} internally, in oneself
\par\textbf{*ajjhottharamāna} \emph{also }threatening: \hyperlink{VI.56}{VI.56}{}
\par\textbf{*añcita} outstretched: \hyperlink{XX.112}{XX.112}{}
\par\textbf{aññā} final knowledge (in the Arahant)
\par\textbf{*aññāti} to know ( = ājānāti): \hyperlink{VII.22}{VII.22}{} (\textbf{\cite{Paṭis} I 122}) 
\par\textbf{aṭṭhaka, aṭṭhamaka} octad
\par\textbf{aṭṭhaṅgika-magga} eightfold path
\par\textbf{*aṇimā} minuteness: \hyperlink{VII.61}{VII.61}{}
\par\textbf{*atammayatā} aloofness; \hyperlink{XXI.135}{XXI.135}{} (\textbf{\cite{M} III 220}) 
\par\textbf{*atippasaṅga} over-generalization (logic): \hyperlink{XIV.186}{XIV.186}{} 
\par\textbf{*atisāra} flux (of bowels), diarrhoea: \hyperlink{XI.21}{XI.21}{}
\par\textbf{atīta} past
\par\textbf{*attatā} selfness, oneself: \hyperlink{IX.47}{IX.47}{} 
\par\textbf{attabhāva} person, personality, selfhood, re-birth 
\par\textbf{attavāda} self-doctrine
\par\textbf{attā} self
\par\textbf{attānudiṭṭhi} self-view, wrong view as self 
\par\textbf{*attānuvāda} self-reproach: \hyperlink{VII.106}{VII.106}{}
\par\textbf{attha} (1) benefit, result, (2) purpose, aim, goal, (3) meaning 
\par\textbf{adinnādāna} taking what is not given, stealing 
\par\textbf{adukkha-m-asukha} neither-painful-nor-pleasant (feeling) 
\par\textbf{adosa} non-hate
\par\textbf{addhā, addhāna} extent, period
\par\textbf{advaya} exclusive, absolute
\par\textbf{*adha-r-āraṇi} lower fire-stick: \hyperlink{XV.41}{XV.41}{}
\par\textbf{*adhikāra} \emph{also }treatise, heading: \hyperlink{III.133}{III.133}{} (\textbf{\cite{Dhs-a} 58}) 
\par\textbf{*adhikicca} as an integral part of, dependent on 
\par\textbf{adhicitta} higher consciousness (i.e. jhāna) 
\par\textbf{adhiṭṭhāna} (l) steadying, (2) resolve
\par\textbf{*adhiṭṭhāna} \emph{also }(3) in terms of: \hyperlink{IV.92}{IV.92}{}, (4) habitat: \hyperlink{X.24}{X.24}{}; \hyperlink{XIV.134}{XIV.134}{} 
\par\textbf{adhipaññā} higher understanding (i.e. insight) 
\par\textbf{adhipati} predominance
\par\textbf{adhippāya} intention, purport 
\par\textbf{adhimokkha} resolution 
\par\textbf{adhimutti} resolution
\par\textbf{adhisīla} higher virtue (i.e. virtue as basis for jhāna and insight) 
\par\textbf{anaññātaññassāmī-t-indriya} I-shall-come-to-know-the-unknown faculty 
\par\textbf{anattā} not-self 
\par\textbf{anāgata} future
\par\textbf{anāgataṃsa-ñāṇa} knowledge of the future 
\par\textbf{anāgāmin} non-returner (third stage of realization) 
\par\textbf{anicca} impermanent
\par\textbf{animitta} signless 
\par\textbf{*anutthunana} brooding: \hyperlink{XVI.59}{XVI.59}{} 
\par\textbf{anunaya} approval
\par\textbf{anupabandhana} anchoring (of the mind) 
\par\textbf{anupālana} maintenance
\par\textbf{anubodha} idea, ideation
\par\textbf{anubhāva} power, influence
\par\textbf{anuloma} (1) in conformity with, (2) in forward order, or as “arising” (of dependent origination), (3) conformity (stage in development of jhāna or insight) 
\par\textbf{*anuvattāpana} causing occurrence parallel to: \hyperlink{XVI.10}{XVI.10}{} (cf. \textbf{\cite{Dhs} p.5}) 
\par\textbf{anusaya} inherent (underlying) tendency (the 7) 
\par\textbf{anussati} recollection (the 10)
\par\textbf{*aneñja, aneja} unperturbed: \hyperlink{XII.55}{XII.55}{}
\par\textbf{anesanā} improper search
\par\textbf{anottappa} shamelessness
\par\textbf{anvaya-ñāṇa} inferential knowledge
\par\textbf{apariyāpanna} unincluded (of supramundane states) 
\par\textbf{*aparisaṇṭhita} turbulent: \hyperlink{VI.86}{VI.86}{}
\par\textbf{*apavārita} opened up: \hyperlink{VI.4}{VI.4}{}
\par\textbf{apāya} state of loss
\par\textbf{appaṇihita} desireless
\par\textbf{appanā} absorption
\par\textbf{appamaññā} measureless state ( = divine abiding) 
\par\textbf{appamāṇa} measureless
\par\textbf{*appāyati} to satisfy: \hyperlink{XI.87}{XI.87}{}
\par\textbf{appicchatā} fewness of wishes
\par\textbf{*appita} done away with: \hyperlink{IV.146}{IV.146}{} (\textbf{\cite{Vibh} 258}) 
\par\textbf{*abbhaṅga} unguent: \hyperlink{I.86}{I.86}{} 
\par\textbf{*abyābhicārin} without exception (gram. and log.): \hyperlink{XIV.25}{XIV.25}{} 
\par\textbf{*abyosāna} not stopping halfway: \hyperlink{XX.21}{XX.21}{}
\par\textbf{abhāva} absence, non-existence, nonentity 
\par\textbf{*abhāva} without sex: XVII. 150 
\par\textbf{*abhigacchati} to rely on: \hyperlink{VII.60}{VII.60}{} 
\par\textbf{abhighāta} impact
\par\textbf{abhijjhā} covetousness
\par\textbf{abhiññā} direct-knowledge
\par\textbf{abhinandanā} delight, delighting
\par\textbf{abhinipāta} conjunction, engagement
\par\textbf{abhiniropana} directing on to
\par\textbf{*abhinivesa} \emph{also }insistence, interpreting: \hyperlink{I.140}{I.140}{}; \hyperlink{XIV.130}{XIV.130}{}; \hyperlink{XXI.84}{XXI.84f.}{}, etc. 
\par\textbf{*abhinihāra} (1) conveying, (2) guidance: \hyperlink{XI.93}{XI.93}{}, \hyperlink{XI.117}{117}{}; \hyperlink{XIII.16}{XIII.16}{}, \hyperlink{XIII.95}{95}{} (\textbf{\cite{Paṭis} I 17}, 61)
\par\textbf{abhibhāyatana} base of mastery, base for transcending (the sense-desire sphere)
\par\textbf{*abhisaṃharati} to make (a profit): \hyperlink{IX.65}{IX.65}{}
\par\textbf{abhisaṅkhāra} (1) volitional formation, kamma-formation, formation, (2) momentum 
\par\textbf{abhisamaya} penetration to, convergence upon (the 4 Truths) 
\par\textbf{amata} deathless (term for Nibbāna)
\par\textbf{amoha} non-delusion 
\par\textbf{*aya} \emph{also }a reason: \hyperlink{XIII.92}{XIII.92}{}; \hyperlink{XVI.17}{XVI.17}{}
\par\textbf{arati} aversion, boredom
\par\textbf{arahant} arahant (4th and last stage of realization) 
\par\textbf{*ariṭṭhaka} kind of thorny plant: \hyperlink{VIII.83}{VIII.83}{}
\par\textbf{ariya} noble, noble one (i.e. one who has attained a path) 
\par\textbf{*ariyati} to be served (CPD has “to approach”): \hyperlink{XIV.22}{XIV.22}{} 
\par\textbf{arūpa} immaterial
\par\textbf{alobha} non-greed
\par\textbf{*allīna} unsheltered (pp. a+līyati): \hyperlink{XX.19}{XX.19}{}
\par\textbf{*allīyituṃ} to give shelter (not in CPD; inf. ā+līyati; see leṇa in CPD): (allīyitabba) \hyperlink{XXII.120}{XXII.120}{}; (allīyana) \hyperlink{VII.83}{VII.83}{}
\par\textbf{*avakkhaṇḍana} hiatus: \hyperlink{II.6}{II.6}{}
\par\textbf{*avagaha} grasping: \hyperlink{XVI.104}{XVI.104}{}
\par\textbf{*avatthā} occasion, position: \hyperlink{IV.167}{IV.167}{}; \hyperlink{XVII.306}{XVII.306}{}; \hyperlink{XX.19}{XX.19}{} 
\par\textbf{*avadhāna} attention: \hyperlink{I.32}{I.32}{} (\textbf{\cite{Paṭis} I 1}; \textbf{\cite{M} II 175}) 
\par\textbf{*avadhi} limit (= odhi): \hyperlink{I.86}{I.86}{}
\par\textbf{avabodha} awareness, discovery 
\par\textbf{*avarodha} inclusion: \hyperlink{XIV.216}{XIV.216}{}, \hyperlink{XIV.219}{219}{} 
\par\textbf{*avāsa} eviction: \hyperlink{IV.9}{IV.9}{}, \hyperlink{IV.12}{12}{}
\par\textbf{*avi} goat or sheep: \hyperlink{XVII.110}{XVII.110}{}
\par\textbf{avikkhepa} non-distraction
\par\textbf{avijjā} ignorance
\par\textbf{avyākata} (1) (kammically) indeterminate (i.e. neither profitable nor unprofitable), (2) un-answered (by the Buddha) 
\par\textbf{avyāpāda} non-ill-will
\par\textbf{asaṅkhata} unformed
\par\textbf{asaññin} non-percipient
\par\textbf{asammoha} non-confusion, non-delusion 
\par\textbf{asubha} foulness, foul, ugly 
\par\textbf{assāsa-passāsa} in-breath and outbreath 
\par\textbf{asekha} non-trainer (i.e. one who has reached the fruition of arahantship) 
\par\textbf{asmi-māna} the conceit “I am”
\par\textbf{ahiri} consciencelessness
\par\textbf{ahetuka} without root-cause
\par\textbf{ahetuka-diṭṭhi} no-cause view
\par\textbf{ahosi-kamma} lapsed kamma
\par\textbf{ākāra} mode, aspect, structure
\par\textbf{ākāra-rūpa} matter as mode (e.g. “mark of the female”) 
\par\textbf{ākāsa} space
\par\textbf{ākiñcañña} nothingness
\par\textbf{āghāta} annoyance
\par\textbf{ācaya} setting-up (of matter)
\par\textbf{ājīva} livelihood
\par\textbf{ādāna} grasping, taking
\par\textbf{*ādina} wretched: \hyperlink{XX.19}{XX.19}{}
\par\textbf{ādīnava} danger, disability
\par\textbf{*ādhāraṇa} subserving: \hyperlink{XIV.60}{XIV.60}{} (\textbf{\cite{M-a} II 52}) 
\par\textbf{ānantarika} (kamma) with immediate result (on rebirth)
\par\textbf{ānāpāna} breathing
\par\textbf{āneñja} imperturbable, the (term for the 4th jhāna) 
\par\textbf{*āpajjati} \emph{also }to follow logically: \hyperlink{II.79}{II.79}{}; \hyperlink{XVI.68}{XVI.68f.}{} 
\par\textbf{*āpajjana} logical consequence: \hyperlink{I.n19}{I.n.19}{}; \hyperlink{XV.68}{XV.68}{} 
\par\textbf{*āpatti} \emph{also }logical consequent: \hyperlink{XVI.72}{XVI.72}{}; \hyperlink{XIX.3}{XIX.3}{} 
\par\textbf{āpatti} offence
\par\textbf{*āpādana} production: \hyperlink{II.21}{II.21}{}
\par\textbf{*āpo} water 
\par\textbf{ābandhana} cohesion 
\par\textbf{ābhoga} concern
\par\textbf{*āyatati} to actuate \hyperlink{XV.4}{XV.4}{}
\par\textbf{āyatana} base
\par\textbf{*āyatana} actuating: \hyperlink{XV.4}{XV.4}{}
\par\textbf{āyu} life 
\par\textbf{āyu-saṅkhāra} vital formation 
\par\textbf{*āyūhana} \emph{also }accumulation (of kamma)
\par\textbf{ārammaṇa} object (of consciousness or its concomitants), support
\par\textbf{āruppa} immaterial state (the 4)
\par\textbf{āropeti} \emph{also }to attribute to: \hyperlink{XX.47}{XX.47}{}
\par\textbf{ālaya} reliance, thing relied on
\par\textbf{āloka} light 
\par\textbf{āvajjana} adverting (consciousness) 
\par\textbf{*āvatthika} denoting a period: \hyperlink{VII.54}{VII.54}{}
\par\textbf{*āviñjana} picking up (see PED āvijjhati): \hyperlink{XIV.37}{XIV.37}{} 
\par\textbf{*āsana} \emph{also }(flower) altar: \hyperlink{V.15}{V.15}{}
\par\textbf{*āsava} canker (the 4)
\par\textbf{āsevanā} (1) cultivation, (2) repetition
\par\textbf{*āhanana} striking at: \hyperlink{IV.88}{IV.88}{}
\par\textbf{āhāra} nutriment, food
\par\textbf{āhāra-samuṭṭhāna} nutriment-originated (matter) 
\par\textbf{iṭṭha} desirable
\par\textbf{*itarathā} otherwise: \hyperlink{III.53}{III.53}{} (\textbf{\cite{Dhs-a} 44})
\par\textbf{itthindriya} femininity faculty
\par\textbf{idappaccayatā} specific conditionality (term for dependent origination) 
\par\textbf{iddhi} power, success, supernormal power 
\par\textbf{iddhipāda} road to power, basis for success (the 4) 
\par\textbf{indriya} faculty (the 22)
\par\textbf{iriyāpatha} posture, deportment (the 4)
\par\textbf{issara} overlord, Lord Creator
\par\textbf{īhaka} having curiosity, activity
\par\textbf{uggaha} learning
\par\textbf{uggaha} nimitta-learning sign
\par\textbf{*uggaṇhita (ugghaṭita?)} decayed: \hyperlink{VI.42}{VI.42}{}
\par\textbf{*ugghāti} removal: \hyperlink{III.115}{III.115}{}
\par\textbf{*ugghāta} exhilaration: \hyperlink{I.117}{I.117}{}
\par\textbf{uccheda-diṭṭhi} annihilation view
\par\textbf{ujukatā} rectitude
\par\textbf{utu} (1) climate, (2) season, (3) temperature
\par\textbf{utu-samuṭṭhāna} temperature-originated (matter)
\par\textbf{udaya} rise
\par\textbf{udaya-bbaya} rise and fall 
\par\textbf{*udāhariyati} to be uttered: \hyperlink{XV.3}{XV.3}{} 
\par\textbf{uddhacca} agitation
\par\textbf{uddhacca-kukkucca} agitation and worry
\par\textbf{upakkilesa} imperfection
\par\textbf{upacaya} growth (of matter)
\par\textbf{upacāra} (1) approach, neighbourhood, precinct, (2) access (concentration) 
\par\textbf{*upacāra} \emph{also }(3) metaphor. \hyperlink{XVI.70}{XVI.70}{}; \hyperlink{XVII.15}{XVII.15}{}; \hyperlink{XXII.51}{XXII.51}{} 
\par\textbf{*upaṭṭhāna} \emph{also }(1) establishment \hyperlink{VIII.168}{VIII.168}{}: (2) appearance: \hyperlink{XXI.29}{XXI.29}{} 
\par\textbf{*upadhāraṇa} upholding: \hyperlink{I.19}{I.19}{}, \hyperlink{I.141}{141}{}
\par\textbf{*upanaya} inducement, application (log): \hyperlink{VII.83}{VII.83}{} 
\par\textbf{*upanayana} \emph{also }applying (log.), inducing, leading on: \hyperlink{VII.83}{VII.83}{}; \hyperlink{XIV.68}{XIV.68}{}
\par\textbf{upapatti} reappearance, rebirth
\par\textbf{upatthambhana} consolidation, stiffening, supporting 
\par\textbf{upabrūhana} intensification
\par\textbf{*upabrūhayati} to intensify: \hyperlink{VIII.121}{VIII.121}{}
\par\textbf{upabhuñjaka} experiencer, user
\par\textbf{*upasaṭṭhatā} menacedness: \hyperlink{XX.16}{XX.16}{}
\par\textbf{upasama} peace (term for Nibbāna)
\par\textbf{upādāna} clinging
\par\textbf{upādāna-kkhandha} aggregate (as object) of clinging 
\par\textbf{upādā-rūpa} derivative (or secondary) materiality 
\par\textbf{upādiṇṇa, upādiṇṇaka} clung-to, kammically acquired (matter), organic (matter) 
\par\textbf{upāya} means
\par\textbf{upāyāsa} despair
\par\textbf{upekkhā} equanimity, onlooking
\par\textbf{uppatti} arising, rebirth
\par\textbf{*uppatti} \emph{also }origin of a sutta (tech. term): \hyperlink{III.88}{III.88}{}; \hyperlink{VII.69}{VII.69}{} 
\par\textbf{uppatti-bhava} rebirth-process becoming, being as result of action 
\par\textbf{uppanna} arisen
\par\textbf{uppāda} arising
\par\textbf{ussada} prominence
\par\textbf{ussāha} activity
\par\textbf{*ūhana} hitting upon: \hyperlink{IV.88}{IV.88}{}
\par\textbf{ekaggatā} unification (of consciousness)
\par\textbf{ekatta} (1) unity, (2) identity, (3) singleness 
\par\textbf{*eta-parama} that at most: \hyperlink{XIV.216}{XIV.216}{}; \hyperlink{XVI.28}{XVI.28}{} (\textbf{\cite{M} I 339}) 
\par\textbf{evaṃ-dhammatā} ineluctable regularity 
\par\textbf{esanā} search 
\par\textbf{okāsa} (1) location, (2) opportunity
\par\textbf{*okāseti} to scatter on (not as in PED): \hyperlink{XII.85}{XII.85}{} (\textbf{\cite{S} IV 190}) 
\par\textbf{*okkhandhati} to descend into: \hyperlink{XX.120}{XX.120}{}; \hyperlink{XXII.34}{XXII.34}{} 
\par\textbf{*ogaḷati} to run downwards: \hyperlink{VIII.124}{VIII.124}{} 
\par\textbf{ogha} flood (the 4)
\par\textbf{ojaṭṭhamaka} material octad with nutritive essence as eighth 
\par\textbf{ojā} nutritive essence, metabolism
\par\textbf{ottappa} shame
\par\textbf{opapātika} apparitionally reborn
\par\textbf{*obhagga} looped: \hyperlink{VIII.118}{VIII.118}{}; \hyperlink{XI.64}{XI.64}{} 
\par\textbf{*obhañjati (or obhuñjati)} to loop, to coil: \hyperlink{XI.64}{XI.64}{} 
\par\textbf{obhāsa} illumination
\par\textbf{*omatta} subordinate: \hyperlink{XX.64}{XX.64}{}
\par\textbf{*oruhati} to come down: \hyperlink{IV.64}{IV.64}{}
\par\textbf{oḷārika} gross
\par\textbf{*ovaṭṭha} showered down: \hyperlink{XI.72}{XI.72}{}
\par\textbf{kaṅkhā} doubt
\par\textbf{kaṭatta} performedness (of kamma), (kamma) performed 
\par\textbf{*kaṇḍuyati} to be itchy: \hyperlink{VIII.127}{VIII.127}{}
\par\textbf{*kaṇṇika} fungus: \hyperlink{VIII.88}{VIII.88}{}
\par\textbf{*kataka} \emph{also }a kind of seed (used for clearing water) 
\par\textbf{kathā-vatthu} (1) name of Abhidhamma book, (2) instance of talk (the 10) 
\par\textbf{kappa} eon, age
\par\textbf{kabaliṅkārāhāra} physical nutriment
\par\textbf{kampana} wavering, shaking
\par\textbf{kammaññatā} wieldiness
\par\textbf{kamma} (1) kamma, deeds, action, (2) work, (3) (legal) enactment 
\par\textbf{kammaṭṭhāna} meditation subject
\par\textbf{kamma-patha} course of action, of kamma 
\par\textbf{kamma-bhava} kamma-process becoming, being as action 
\par\textbf{kamma-samuṭṭhāna} kamma-originated (matter) 
\par\textbf{kammanta} action, work
\par\textbf{karaja} physical
\par\textbf{karuṇā} compassion
\par\textbf{kalāpa} (1) group, (2) material group (term for material octad, etc.) 
\par\textbf{kalāpa-sammasana} comprehension by groups (does not refer to the material octad, etc.) 
\par\textbf{kalyāṇa-puthujjana} magnanimous ordinary man 
\par\textbf{kasiṇa} kasiṇa, universal (a contemplation device, and concept based thereon) 
\par\textbf{kāma} sense desire, sensual desire 
\par\textbf{kāma-guṇa} cord of sense-desire (the 5), dimension of sensual desire 
\par\textbf{kāma-cchanda} lust, zeal for sense desires 
\par\textbf{kāma-rāga} greed for sense desires
\par\textbf{kāmāvacara} sense-desire sphere, sense sphere 
\par\textbf{kāmesu micchācāra} sexual misconduct 
\par\textbf{kāya} (1) body, group, order, (2) the material body, (3) the mental body (i.e. the 3 nāmakkhandha) 
\par\textbf{kāyasakkhin} body witness
\par\textbf{kāya-saṅkhāra} bodily formation (term for in-breath and out-breath) 
\par\textbf{kāraka} doer
\par\textbf{kāla} time
\par\textbf{kicca} function
\par\textbf{*kiñcana} owning, ownership: \hyperlink{XXI.53}{XXI.53}{}
\par\textbf{*kiṇāti} \emph{also }to combat: \hyperlink{VI.8}{VI.8}{}
\par\textbf{kiriya} (kammically) functional, inoperative 
\par\textbf{kilesa} defilement
\par\textbf{*kukata} villainy: \hyperlink{XIV.174}{XIV.174}{}
\par\textbf{kukkucca} worry
\par\textbf{*kuṇḍika} \emph{also }a four-footed water pot: \hyperlink{V.3}{V.3}{} 
\par\textbf{*kuṇapa} \emph{also }ordure: \hyperlink{VIII.121}{VIII.121}{}; \hyperlink{XI.19}{XI.19}{}, \hyperlink{XI.21}{21}{}
\par\textbf{kusala} (1) skilful, (2) profitable (consciousness), (3) good 
\par\textbf{kuhanā} scheming
\par\textbf{*kūṭa} wild, savage: \hyperlink{VIII.53}{VIII.53}{} (\textbf{\cite{M-a} II 82})
\par\textbf{*kūṭāgāra} \emph{also }(1) catafalque (comy. To \textbf{\cite{A} I 150}), (2) palanquin: \hyperlink{XII.71}{XII.71}{} (\textbf{\cite{M-a} V 90}) 
\par\textbf{*kūpaka-yaṭṭhi} masthead (?), spar (?): \hyperlink{XXI.65}{XXI.65}{} 
\par\textbf{*koṭṭhaṭṭhi} shoulder-blade bone (lit. “flat-bone”; not as in PED): \hyperlink{VIII.101}{VIII.101}{} 
\par\textbf{*koṭṭhalika} flattened: \hyperlink{VII.97}{VII.97}{}
\par\textbf{*kosa} measure of length (about 1 mile): \hyperlink{IV.37}{IV.37}{} 
\par\textbf{khaṇa} moment, instant
\par\textbf{*khaṇati} \emph{also }to consume: \hyperlink{IV.100}{IV.100}{}; \hyperlink{XVII.48}{XVII.48}{} 
\par\textbf{khanti} (1) patience, (2) choice 
\par\textbf{khandha} aggregate
\par\textbf{khaya} destruction, exhaustion
\par\textbf{khara} harsh
\par\textbf{*kharigata} harsh: \hyperlink{XI.31}{XI.31}{} (\textbf{\cite{M} I 185})
\par\textbf{*khinna} exhausted: \hyperlink{IV.100}{IV.100}{}; see khijjana 14, n.2VI. 
\par\textbf{*gaṇḍuppādaka} \emph{also }sort of intestinal worm: \hyperlink{VIII.121}{VIII.121}{} 
\par\textbf{*gata-paccāgata} (1) duty of going to and returning from the alms round with the meditation subject, (2) kind of refuse rag: \hyperlink{II.17}{II.17}{}; \hyperlink{XIV.28}{XIV.28}{} 
\par\textbf{gati} (1) destiny, destination (on rebirth), movement 
\par\textbf{*gadati} to enunciate (see gada in PED) \hyperlink{VII.35}{VII.35}{} 
\par\textbf{gantha} (1) tie (the 4), (2) book
\par\textbf{gandha} odour
\par\textbf{*gandhayati} to be smelled: \hyperlink{XV.3}{XV.3}{}
\par\textbf{guṇa} special quality
\par\textbf{gocara} resort, domain, scope
\par\textbf{gotrabhū} change-of-lineage (consciousness) 
\par\textbf{*gopa} guardian, \hyperlink{IV.190}{IV.190}{}; \hyperlink{VIII.153}{VIII.153}{} (\textbf{\cite{M} II 180}) 
\par\textbf{ghaṭṭana} impinging, knocking together
\par\textbf{ghana} compact
\par\textbf{ghana-vinibbhoga} resolution of the compact (into elements) 
\par\textbf{ghāna} nose
\par\textbf{cakkavāḷa} world-sphere
\par\textbf{*cakkhati} to relish: \hyperlink{XV.3}{XV.3}{}
\par\textbf{cakkhu} eye
\par\textbf{catusamuṭṭhāna} (matter) of fourfold origination (i.e. by consciousness, kamma, temperature and nutriment) 
\par\textbf{cariya, carita} temperament; behaviour, exercise 
\par\textbf{cāga} generosity
\par\textbf{*cāpalya} \emph{also }personal vanity: \hyperlink{III.95}{III.95}{} (this meaning not in CPD, under acāpalya or acapala) 
\par\textbf{*cāraka} prison: \hyperlink{XIV.221}{XIV.221}{}; \hyperlink{XVI.18}{XVI.18}{}
\par\textbf{*cikicchā} wish to think: \hyperlink{XIV.177}{XIV.177}{}
\par\textbf{citta} (manner of) consciousness, consciousness, cognizance, mind 
\par\textbf{citta-ṭṭhiti} steadiness of consciousness
\par\textbf{citta-vīthi} cognitive series (of consciousnesses) 
\par\textbf{citta-saṅkhāra} mental formation (term for perception and feeling) 
\par\textbf{citta-samuṭṭhāna} consciousnessoriginated (matter) 
\par\textbf{cittuppāda} thought, thought-arising
\par\textbf{cintā} reasoning 
\par\textbf{cuti} death 
\par\textbf{cetanā} volition
\par\textbf{cetasika} consciousness concomitant (i.e. feeling, perception and formations) 
\par\textbf{ceto} mind, heart, will
\par\textbf{cetopariya} penetration of minds
\par\textbf{ceto-vimutti} heart-deliverance, mind-d.
\par\textbf{chanda} zeal 
\par\textbf{*jatuka} bat, pipistrelle: \hyperlink{III.97}{III.97}{}; \hyperlink{XI.7}{XI.7}{} 
\par\textbf{*janaka} \emph{also }father: \hyperlink{XVII.271}{XVII.271}{}
\par\textbf{*jara} fever: \hyperlink{XI.36}{XI.36}{} (\textbf{\cite{A} V 100})
\par\textbf{jarā} aging, old age
\par\textbf{jarā-maraṇa} aging-and-death
\par\textbf{javana} (1) speed, (2) impulsion (consciousness) 
\par\textbf{jāti} (1) birth, (2) sort, kind
\par\textbf{jivhā} tongue
\par\textbf{jīva} soul
\par\textbf{jīvita} life
\par\textbf{*juṭṭha} fostered: \hyperlink{XVI.4}{XVI.4}{}
\par\textbf{jhāna} jhāna
\par\textbf{ñāṇa} knowledge (in general)
\par\textbf{ṭhiti} (1) presence, (2) station, (3) relation, (4) steadiness, stability, (5) stationariness, stagnation 
\par\textbf{*tacchati} \emph{also }to pare: \hyperlink{VIII.103}{VIII.103}{} (\textbf{\cite{M} I 31})
\par\textbf{taṇhā} craving
\par\textbf{tatramajjhattatā} specific neutrality
\par\textbf{tathāgata} perfect one
\par\textbf{tadaṅga} substitution of opposites (function of insight) 
\par\textbf{*tadārammaṇa} (1) having that (aforesaid thing) as its object, (2) registration (consciousness): \hyperlink{XIV.98}{XIV.98}{}; \hyperlink{XVII.139}{XVII.139}{} 
\par\textbf{*tanana} range: \hyperlink{XV.4}{XV.4}{}
\par\textbf{*tāvatva} just so much: XV 18
\par\textbf{tādi-bhāva} equipoise
\par\textbf{tiracchāna-yoni} animal generation
\par\textbf{tilakkhaṇa} three characteristics (of impermanence, pain and not-self) 
\par\textbf{*ti-santati-rūpa} materiality of triple continuity (term for the three decads at moment of rebirth-linking): \hyperlink{XI.112}{XI.112}{}; \hyperlink{XX.22}{XX.22}{} 
\par\textbf{*ti-samuṭṭhāna} materiality of triple origination (by kamma, temperature and nutriment only): \hyperlink{XVII.196}{XVII.196}{} 
\par\textbf{tīraṇa} judgement, investigation
\par\textbf{thaddha} stiffened
\par\textbf{thīna-middha} stiffness and torpor 
\par\textbf{*theriya} belonging to the Elders: epil. verse 
\par\textbf{*dakasītalika} edible white water lily: \hyperlink{VIII.119}{VIII.119}{} 
\par\textbf{dasaka} (1) decad (of matter), (2) decade
\par\textbf{dassana} (l) seeing (the eye’s function), (2) vision, (3) term for the first path 
\par\textbf{dāna} gift, giving 
\par\textbf{*dāna} gap: \hyperlink{II.6}{II.6}{}
\par\textbf{diṭṭha} seen
\par\textbf{diṭṭhi} view, (wrong) view, (right) view
\par\textbf{diṭṭhi-pāta} one attained to vision
\par\textbf{dibba-cakkhu} divine eye
\par\textbf{dibba-sotadhātu} divine ear element
\par\textbf{dukkha} pain, painful, bodily pain, suffering 
\par\textbf{dukkha-dukkha} intrinsic suffering
\par\textbf{duggati} unhappy destination (on rebirth) 
\par\textbf{duccarita} misconduct, misbehaviour
\par\textbf{*duṭṭhulla} \emph{also }(1) inertia, (2) irritability: \hyperlink{IV.124}{IV.124}{} (\textbf{\cite{M} III 151}, 159) 
\par\textbf{dūra} far
\par\textbf{desanā} teaching, instruction
\par\textbf{*desantar-uppatti} successive arising in adjacent locations (description of phenomenon of motion); \hyperlink{VII.n45}{VII.n.45}{} 
\par\textbf{domanassa} grief
\par\textbf{dosa} (1) hate, (2) flaw, (3) humour (of the body) 
\par\textbf{*drabya} substance: \hyperlink{XVIII.n8}{XVIII.n.8}{}
\par\textbf{*drava} fluid: \hyperlink{XI.41}{XI.41}{}
\par\textbf{dvattiṃsākāra} the thirty-two aspects (of the body) 
\par\textbf{dvāra} door (i.e. the 6 d. of consciousness by the 6 bases; also the 3 d. of kamma by body, speech and mind)
\par\textbf{dhamma} (1) the Dhamma or Law (as discovered by the Buddha), (2) dhamma, state, thing, phenomenon, (3) mental object, mental datum (12th base) 
\par\textbf{dhamma-ṭṭhiti-ñāṇa} knowledge of relations of states, knowledge of structure of ideas 
\par\textbf{dhammatā-rūpa} natural materiality (i.e trees, stones, etc.) 
\par\textbf{*dhammani} rat-snake: \hyperlink{XI.64}{XI.64}{}
\par\textbf{dhamma-vicaya} investigation of states
\par\textbf{dhammānusārin} dhamma devotee 
\par\textbf{dhātu} (l) element, (2) humour (of the body), (3) relic 
\par\textbf{*dhātu} \emph{also }(metallic) ore: \hyperlink{XI.20}{XI.20}{}; \hyperlink{XV.20}{XV.20}{}
\par\textbf{dhutaṅga} ascetic practice
\par\textbf{*dhura-bhatta} meal given in a principal house (not as in PED): \hyperlink{II.27}{II.27}{} 
\par\textbf{dhuva} everlasting 
\par\textbf{nandi} delight 
\par\textbf{naya} method
\par\textbf{naya-vipassanā} inductive insight
\par\textbf{navaka} ennead
\par\textbf{*nahanā} tying: \hyperlink{I.73}{I.73}{}
\par\textbf{*nāgabalā} kind of plant: \hyperlink{XI.17}{XI.17}{}
\par\textbf{nāna-kkhaṇika} (kamma) acting from a different time 
\par\textbf{nānatta} variety, difference
\par\textbf{nāma} (1) mentality, (2) name
\par\textbf{nāma-kaya} mentality body, mental body (aggregates of feeling, perception and formations) 
\par\textbf{nāma-rūpa} mentality-materiality (term for the five aggregates, or for the four aggregates excluding consciousness)
\par\textbf{nāma-rūpa-pariccheda} definition of mentality-materiality 
\par\textbf{*nāyare} they are known (—ñāyanti): \hyperlink{VIII.29}{VIII.29}{}; cf. \hyperlink{IX.42}{IX.42}{} (nāyati—ñāyati) 
\par\textbf{nikanti} attachment, attraction
\par\textbf{*nigghāta} depression: \hyperlink{XI.117}{XI.117}{}
\par\textbf{nicca} permanent
\par\textbf{nijjīva} soulless
\par\textbf{*nippadesa} comprehensive: \hyperlink{XVI.95}{XVI.95}{}
\par\textbf{*nippharipphanda} inactive: \hyperlink{V.4}{V.4}{}
\par\textbf{*nippiṃsati} to scrape, to grind: \hyperlink{I.81}{I.81}{}
\par\textbf{*nippuñchati} to wipe off: \hyperlink{I.81}{I.81}{}
\par\textbf{*nippesikatā} belittling (not as in PED): \hyperlink{I.64}{I.64}{} (\textbf{\cite{M} III 75}) 
\par\textbf{*nippeseti} to scrape off: \hyperlink{I.81}{I.81}{}
\par\textbf{*nipphanna} produced (term for certain kinds of derived materiality) 
\par\textbf{*nibbacana} verbal derivative (gram): \hyperlink{II.4}{II.4}{}; \hyperlink{XVI.14}{XVI.14}{} 
\par\textbf{nibbatti} generation, production, rebirth 
\par\textbf{Nibbāna} nirvana, extinction (of greed, hate and delusion) 
\par\textbf{*nibbikappa} “without dismay,” without thinking: \hyperlink{II.71}{II.71}{}; \hyperlink{VI.81}{VI.81}{} 
\par\textbf{nibbidā} dispassion, revulsion
\par\textbf{*nibbisa} without poison: \hyperlink{XII.115}{XII.115}{} 
\par\textbf{*nibbedha} penetration 
\par\textbf{nimitta} sign
\par\textbf{niyati-vāda} determinism, fatalism
\par\textbf{niyyāna} outlet (from the round of rebirths; term for the path) 
\par\textbf{nirutti} language
\par\textbf{nirodha} cessation
\par\textbf{nissaya} (1) support, (2) the dependence (given by teacher to pupil) 
\par\textbf{nissatta} not-a-living-being
\par\textbf{nissaraṇa} escape (from defilement by Nibbāna) 
\par\textbf{nīvaraṇa} hindrance (the 5 or the 7)
\par\textbf{*nīharati} \emph{also }to fix: \hyperlink{II.50}{II.50}{}
\par\textbf{nekkhamma} renunciation
\par\textbf{*nemittika} (name) signifying (an acquirement): \hyperlink{VII.55}{VII.55}{} 
\par\textbf{*nemittikatā} hinting (not as in PED): \hyperlink{I.63}{I.63}{} (\textbf{\cite{M} III 75}) 
\par\textbf{*pakaṭṭha} distant: \hyperlink{VII.81}{VII.81}{}
\par\textbf{pakati} (1) nature, natural, normal, (2) Primordial Essence, Prakṛti
\par\textbf{*pakāsa} illumination: \hyperlink{XVII.77}{XVII.77}{}
\par\textbf{pakkhandati} to enter into, to launch out into 
\par\textbf{pāguññatā} proficiency
\par\textbf{paccakkha} personal experience
\par\textbf{paccatta} for oneself
\par\textbf{paccaya} (1) condition (for what is conditionally arisen), (2) requisite (the 4 for the bhikkhu) 
\par\textbf{paccaya-pariggaha} discernment of conditions 
\par\textbf{paccayākāra} structure of conditions (term for dependent origination) 
\par\textbf{paccavekkhaṇa} reviewing
\par\textbf{paccuppanna} present, presently arisen
\par\textbf{paññatti} (1) making-known, announcement, (2) appellation, designation, (3) concept, description 
\par\textbf{paññā} understanding (insight and path)
\par\textbf{paññā-vimutta} one liberated by understanding 
\par\textbf{*paṭatantuka} intestinal worm: \hyperlink{VIII.121}{VIII.121}{} 
\par\textbf{*paṭikkamana} refectory: \hyperlink{II.28}{II.28}{} 
\par\textbf{paṭikkūla} repulsive
\par\textbf{paṭigha} resentment, resistance
\par\textbf{paṭicca} (indecl. ger. of paṭiyeti) having depended, due to, dependent on 
\par\textbf{*paṭicca} (decl. adj.) ought to be arrived at: \hyperlink{XVII.16}{XVII.16}{} 
\par\textbf{paṭicca-samuppanna} conditionally arisen, dependently originated 
\par\textbf{paṭicca-samuppāda} dependent origination 
\par\textbf{*paṭiñña} \emph{also }proposition (log.) \hyperlink{XVII.67}{XVII.67}{} (\textbf{\cite{Kv}}.2) 
\par\textbf{paṭiñña} claim 
\par\textbf{paṭinissagga} relinquishment 
\par\textbf{paṭipatti} way, progress, practice
\par\textbf{*paṭipatti} theory: \hyperlink{XIV.163}{XIV.163}{}, \hyperlink{XIV.177}{177}{}; \hyperlink{XVII.52}{XVII.52}{}, \hyperlink{XVII.303}{303}{} 
\par\textbf{paṭipassaddhi} tranquilization (of defilement by fruition) 
\par\textbf{*paṭipassanā} looking back: \hyperlink{VIII.189}{VIII.189}{}, \hyperlink{VIII.225}{225}{} 
\par\textbf{*paṭipātiyāmana} following successively: \hyperlink{VIII.69}{VIII.69}{} 
\par\textbf{*paṭipadāna} maintaining (on course): \hyperlink{IV.42}{IV.42}{} 
\par\textbf{paṭibhāga-nimitta} counterpart sign
\par\textbf{*paṭiveti} to vanish: \hyperlink{XX.96}{XX.96}{}
\par\textbf{paṭivedha} penetration (of 4 Truths)
\par\textbf{paṭisaṅkhā} reflection
\par\textbf{paṭisandhi} rebirth-linking (consciousness) 
\par\textbf{paṭisambhidā} discrimination (the 4)
\par\textbf{*paṭisiddha} excluded, rejected, refuted (log.): \hyperlink{XVII.150}{XVII.150}{} 
\par\textbf{*paṭihaññati} to resent (as verb for paṭigha): \hyperlink{IX.101}{IX.101}{} (cf. \textbf{\cite{Dhs-a} 72}, \textbf{\cite{Netti} 13})
\par\textbf{*paṭihita (paṇihita?)} drawn on: \hyperlink{VIII.26}{VIII.26}{}
\par\textbf{paṇidhi} desire, aspiration
\par\textbf{paṇīta} superior, sublime
\par\textbf{*patati} to gather, to wander for: \hyperlink{II.5}{II.5}{}
\par\textbf{*patīyamāna} going back to: \hyperlink{XVII.16}{XVII.16}{}
\par\textbf{*patthanīyatā} famousness: \hyperlink{IV.2}{IV.2}{}, \hyperlink{IV.10}{10}{}
\par\textbf{pathavī} earth
\par\textbf{padhāna} (1) endeavour, effort, (2) basic
\par\textbf{*padhāna} Basic Principle, Pradhāna: \hyperlink{XVI.85}{XVI.85}{} 
\par\textbf{papañca} (1) obstacle, (2) diffuseness, (2) diversification (as function of craving, conceit and wrong view; not in this sense in PED) 
\par\textbf{*pabbhāra} \emph{also }overhang of rock: \hyperlink{II.61}{II.61}{}
\par\textbf{*pabhāvanā} production: \hyperlink{VIII.182}{VIII.182}{} (\textbf{\cite{Paṭis} I 184}) 
\par\textbf{*pabhuti} \textbf{[TODO: missing parenthesis in BPS2011?:]}encl.) and so on, etcetera ( = ādi in that sense): \hyperlink{VIII.17}{VIII.17}{}, \hyperlink{VIII.121}{121}{}; \hyperlink{X.51}{X.51}{}, etc. 
\par\textbf{pabheda} class, category 
\par\textbf{*pamukha} veranda, forecourt: \hyperlink{IV.13}{IV.13}{}; \hyperlink{XI.7}{XI.7}{}; \hyperlink{XIII.6}{XIII.6}{} 
\par\textbf{paramattha} highest sense, ultimate sense 
\par\textbf{parāmaṭṭha} misapprehended, adheredto 
\par\textbf{parāmāsa} misapprehension, adherence, pre-assumption 
\par\textbf{parikathā} roundabout talk
\par\textbf{*parikappanatā} conjecturing: \hyperlink{III.77}{III.77}{}
\par\textbf{parikamma} preliminary work
\par\textbf{parikkhāra} (1) equipment, (2) requisite
\par\textbf{pariggaha} (1) inclusion, (2) embracing (as definition of right speech), (3) reinforcement, (4) discerning, etc. 
\par\textbf{pariccāga} giving up
\par\textbf{pariccheda-rūpa} delimiting-materiality (term for space) 
\par\textbf{pariññā} full-understanding (the 3)
\par\textbf{pariṇāma} change
\par\textbf{paritta} (1) small, (2) limited (term for the sense-desire sphere), (3) protection (term for certain discourses recited for that purpose) 
\par\textbf{parideva} lamentation
\par\textbf{*parinijjhāpana} obsessing, burning up: \hyperlink{XVI.48}{XVI.48}{} 
\par\textbf{*parinipphanna} positively produced XXIII. n.18 
\par\textbf{parinibbāna} attainment of Nibbāna
\par\textbf{paripācana} maintaining, maturing, ripening 
\par\textbf{*paripphandana} \emph{also }interference, activity: \hyperlink{IV.89}{IV.89}{}; \hyperlink{XIV.144}{XIV.144}{} 
\par\textbf{*paribhaṇḍa} \emph{also }repair: \hyperlink{XXIII.36}{XXIII.36}{}
\par\textbf{paribhoga} use
\par\textbf{pariyatti} (1) mastery, (2) scripture
\par\textbf{pariyāpanna} included
\par\textbf{pariyāhanana} threshing, striking on: \hyperlink{IV.88}{IV.88}{} 
\par\textbf{pariyuṭṭhāna} obsession
\par\textbf{*pariyonahana} covering, envelope: \hyperlink{VIII.115}{VIII.115}{} 
\par\textbf{*pariyosāna} \emph{also }intensity: \hyperlink{VI.49}{VI.49}{} 
\par\textbf{*pariveṇa} \emph{also} surroundings of a building, surrounding walk: \hyperlink{IV.127}{IV.127}{}; \hyperlink{XI.7}{XI.7}{}; \hyperlink{XIII.6}{XIII.6}{} 
\par\textbf{*parissavati} to run away: \hyperlink{XI.90}{XI.90}{}
\par\textbf{*parihāra-vacana} explanation: \hyperlink{XVII.109}{XVII.109}{} 
\par\textbf{palibodha} impediment 
\par\textbf{pavatta, pavatti} (1) occurrence, (2) course of an existence (between rebirth-linking and death) 
\par\textbf{*pavana} draught: \hyperlink{XI.19}{XI.19}{}; \hyperlink{XVI.37}{XVI.37}{}
\par\textbf{pavicaya} investigation
\par\textbf{paviveka} seclusion
\par\textbf{pasāda} sensitivity (of matter) 
\par\textbf{pasādana} confidence, clarification 
\par\textbf{passaddhi} tranquillity
\par\textbf{pahāna} abandoning
\par\textbf{*paheyya} abandonable: \hyperlink{XVI.93}{XVI.93}{}
\par\textbf{*pāṭibhoga} agent (not as in PED): \hyperlink{XVII.174}{XVII.174}{}; \textbf{\cite{Peṭ} 215} 
\par\textbf{pāṇātipāta} killing living things
\par\textbf{pāduddhāra} footstep, lifting of foot
\par\textbf{*pāpaka} what reaches, causes to reach: \hyperlink{XIV.5}{XIV.5}{}, \hyperlink{XIV.68}{68}{} 
\par\textbf{*pāpana} reaching \hyperlink{XVI.68}{XVI.68}{}
\par\textbf{*pāpana} denigrating: \hyperlink{I.81}{I.81}{}
\par\textbf{pāpicchatā} evilness of wishes
\par\textbf{pāramī, pāramitā} perfection
\par\textbf{*pārāvata} pigeon (pārāpata in PED): \hyperlink{XI.7}{XI.7}{}
\par\textbf{*pāvāra} \emph{also }a cloth, cloth: \hyperlink{VIII.117}{VIII.117}{}
\par\textbf{piṇḍapāta} alms
\par\textbf{*piṇḍika} the calf of the leg: \hyperlink{VIII.97}{VIII.97}{}; \hyperlink{XI.11}{XI.11}{}; cf. piṇḍa at \hyperlink{VIII.126}{VIII.126}{} 
\par\textbf{*pidhānī} lid: \hyperlink{XI.24}{XI.24}{}
\par\textbf{pisuṇa-vācā} malicious speech
\par\textbf{pīṇana} act of refreshing
\par\textbf{pīti} happiness
\par\textbf{puggala} person
\par\textbf{puthujjana} ordinary man (i.e. one who has not reached the path)
\par\textbf{*pupphaka} balloon, swelling: \hyperlink{VIII.117}{VIII.117}{}
\par\textbf{purisa} man, male
\par\textbf{*purisa} World Soul, Puruṣa: \hyperlink{XVII.8}{XVII.8}{}
\par\textbf{peta} ghost
\par\textbf{*pesika} scraper: \hyperlink{I.81}{I.81}{}
\par\textbf{pharaṇa} pervasion, intentness upon
\par\textbf{pharusa-vācā} harsh speech
\par\textbf{phala} (1) fruit of (plants), (2) fruit of cause, (3) fruition (of path; the 4) 
\par\textbf{*phalakasata} target: \hyperlink{XXII.12}{XXII.12}{} 
\par\textbf{phassa} contact
\par\textbf{*phāsu} convenient: \hyperlink{IV.1}{IV.1}{} (\textbf{\cite{D} II 99}; \textbf{\cite{M} I 10})
\par\textbf{phoṭṭhabba} tangible datum, t. object
\par\textbf{bala} power (the 5; the 10 of a Perfect One) 
\par\textbf{bahiddhā, bāhira} external, externally 
\par\textbf{*bālatta} dotage: \hyperlink{XVI.45}{XVI.45}{}
\par\textbf{buddha} enlightened one
\par\textbf{buddhi} (l) enlightenment, (2) intellect, discretion, speculation, (3) sensation 
\par\textbf{*budha} possessed of wit: \hyperlink{IV.66}{IV.66}{}
\par\textbf{bojjhaṅga} enlightenment factor
\par\textbf{bodhi} enlightenment, awakening 
\par\textbf{bodhisatta} Bodhisatta, Being Destined to Enlightenment
\par\textbf{*byatti} see vyatti
\par\textbf{brahmacariya} life of purity, the good life, the life divine 
\par\textbf{brahma-vihāra} divine abiding (the 4)
\par\textbf{brūhana} intensification
\par\textbf{bhagavant} Blessed One
\par\textbf{bhaṅga} dissolution
\par\textbf{*bhattar} employer: \hyperlink{IV.121}{IV.121}{} (cf. \textbf{\cite{M} II 123})
\par\textbf{*bhanti} they shine (3rd p. pl. of bhāti): \hyperlink{VII.36}{VII.36}{} (\textbf{\cite{M} I 328}) 
\par\textbf{bhayat’ upaṭṭhāna} appearance as terror (stage in insight) 
\par\textbf{bhava} becoming, being, existence
\par\textbf{bhavaṅga} life-continuum (consciousness) 
\par\textbf{*bhāti} brother: \hyperlink{XXI.54}{XXI.54}{}
\par\textbf{bhāva} (1) essence, stateness, (2) sex, (3) verbal substantive (gram.) 
\par\textbf{bhāvanā} (1) development (lit. making be’), (2) term for the 3 higher paths
\par\textbf{bhāva-sādhana} formula of establishment by substantive (gram.): \hyperlink{XVII.12}{XVII.12}{} 
\par\textbf{bhikkhu} bhikkhu, Buddhist monk
\par\textbf{bhūta} (1) become, been, (2) creature, (3) primary element (entity) of matter, etc.
\par\textbf{bhūtūpādā-rūpa} matter derived upon the (four) primary elements (the 24 kinds) 
\par\textbf{bhūmi} (l) ground, soil, (2) plane (of existence; the 4) 
\par\textbf{*bheda, vacī- } speech utterance: \hyperlink{XIV.62}{XIV.62}{} (cf. \textbf{\cite{Dhs-a} 90}; \textbf{\cite{M} I 301}) 
\par\textbf{magga} path
\par\textbf{macchariya, macchera} avarice
\par\textbf{majjhatta} neutral, central 
\par\textbf{majjhima} middle, medium
\par\textbf{mada} vanity, intoxication
\par\textbf{manasikāra} attention, bringing to mind
\par\textbf{mano} mind
\par\textbf{maraṇa} death, dying 
\par\textbf{*maru} \emph{also }cliff: \hyperlink{XVII.63}{XVII.63}{}
\par\textbf{mala} stain (the 3)
\par\textbf{mahaggata} exalted (a term for consciousness “exalted” from the “limited” sense-desire sphere to the fine-material or immaterial spheres) 
\par\textbf{*mahacca} great pomp: \hyperlink{X.46}{X.46}{} (\textbf{\cite{D} I 49})
\par\textbf{mahā-bhūta} great primary, great entity (the 4) 
\par\textbf{mahā-vipassanā} principal insight (the 18)
\par\textbf{mātikā} (1) schedule of the Abhidhamma, (2) codes of the Pātimokkha (the 2), (3) schedule, etc. 
\par\textbf{māna} conceit (pride)
\par\textbf{māyā} deceit
\par\textbf{micchā} wrong
\par\textbf{micchatta} wrongness (the 10)
\par\textbf{middha} torpor
\par\textbf{*milāpana} withering, causing to wither: \hyperlink{XIV.128}{XIV.128}{} 
\par\textbf{muccitukamyatā} desire for deliverance
\par\textbf{muta} sensed (i.e. smelled, tasted or touched) 
\par\textbf{muditā} gladness (at others’ success)
\par\textbf{mudutā} malleability
\par\textbf{musā-vāda} false speech, lying
\par\textbf{mūla} root
\par\textbf{mettā} loving-kindness, amity
\par\textbf{*mehana} private parts: \hyperlink{VII.64}{VII.64}{}
\par\textbf{moha} delusion
\par\textbf{yathā-kammūpaga-ñāṇa} knowledge of (beings’) faring according to deeds 
\par\textbf{yathābhūta} correct
\par\textbf{yuganaddha} coupling, yoking (of serenity and insight) 
\par\textbf{ye-vā-panaka} or-whatever (state) (term for certain formations) 
\par\textbf{yoga} bond (the 4)
\par\textbf{yoni} (1) womb, (2) generation, (3) cause, reason 
\par\textbf{yoniso} wise, wisely, with ordered reasoning
\par\textbf{rati} delight
\par\textbf{rasa} (1) taste, flavour, (2) nature as function or achievement, (3) stimulus (for feeling), (4) essential juice, filtrate 
\par\textbf{*rasati} to taste: \hyperlink{XV.3}{XV.3}{}
\par\textbf{*rasāyana} elixir. \hyperlink{XVII.236}{XVII.236}{}
\par\textbf{rāga} greed, lust
\par\textbf{rūpa} (1) materiality (aggregate), fine materiality of fine-material Brahmāworld, matter in general, material form, (2) visible datum, visible object, visible matter, visible form 
\par\textbf{rūpa-kāya} material body
\par\textbf{rūpa-kkhandha} materiality aggregate
\par\textbf{rūpūpādānakkhandha} materiality aggregate (as object) of clinging 
\par\textbf{rūpa-rūpa} concrete materiality (term for certain derived kinds of materiality) 
\par\textbf{*rūpayati} to be made visible: \hyperlink{XV.3}{XV.3}{}
\par\textbf{rūpāvacara} fine-material sphere
\par\textbf{lakkhaṇa} characteristic
\par\textbf{*laghimā} lightness: \hyperlink{VII.61}{VII.61}{}
\par\textbf{lahutā} lightness
\par\textbf{lābha} gain
\par\textbf{loka} world
\par\textbf{lokiya} mundane (i.e. not associated with the path, fruition or Nibbāna) 
\par\textbf{lokiya dhamma} worldly state (the 8)
\par\textbf{lokuttara} supramundane (i.e. the 9 states consisting of the 4 paths, 4 fruitions and Nibbāna, and states associated with them) 
\par\textbf{lobha} greed
\par\textbf{vaggulī} fruit bat, flying fox; \hyperlink{XXI.91}{XXI.91}{}
\par\textbf{*vacanāvayava} member of a syllogism: \hyperlink{XVII.67}{XVII.67}{} 
\par\textbf{*vacī-bheda} speech utterance: \hyperlink{XIV.62}{XIV.62}{}
\par\textbf{vacī-saṅkhāra} verbal formation (i.e. vitakka and vicāra) 
\par\textbf{vaṭṭa} round (of kamma, etc.; term for the dependent origination as arising) 
\par\textbf{vaḍḍhana} extension, increase
\par\textbf{vata} vow, duty, ritual
\par\textbf{vatta} duty
\par\textbf{*vattana} performance of duties: \hyperlink{III.71}{III.71}{} (\textbf{\cite{Vin} I 61}) 
\par\textbf{*vatthika} clothable: \hyperlink{VII.79}{VII.79}{}
\par\textbf{vatthu} (1) basis, physical basis (term for the six internal bases), (2) object, (3) instance, example, (4) story, etc. 
\par\textbf{vaya} (1) fall, (2) stage of life
\par\textbf{vāsaṭṭhāna} defining 
\par\textbf{vasa-vattana} exercise of mastery 
\par\textbf{*vahanika} catamaran float (?): \hyperlink{XVII.196}{XVII.196}{}
\par\textbf{vācā} speech
\par\textbf{vāta} air, wind
\par\textbf{*vāna} fastening: \hyperlink{VIII.247}{VIII.247}{}
\par\textbf{vāyāma} effort 
\par\textbf{vāyo} air 
\par\textbf{*vikappa} alternative: \hyperlink{XI.89}{XI.89}{} (cf. \textbf{\cite{M-a} I 67})
\par\textbf{vikampana} shaking, wavering
\par\textbf{vikāra} alteration
\par\textbf{vikāra-rūpa} materiality as alteration (term for certain of the 24 kinds of derived materiality, i.e. impermanence, etc.) 
\par\textbf{*vikuppati} to be damaged: \hyperlink{XXIII.35}{XXIII.35}{} 
\par\textbf{vikubbana} (1) versatility (in development of divine abidings), (2) transformation (by supernormal power) 
\par\textbf{vikkhambhana} suppression (of defilements by serenity)
\par\textbf{vikkhepa} distraction
\par\textbf{*vikkhepa} \emph{also }spreading out: \hyperlink{IV.89}{IV.89}{}; gesture: \hyperlink{XI.100}{XI.100}{} 
\par\textbf{vicāra} sustained thought
\par\textbf{vicikicchā} uncertainty
\par\textbf{*vijambhati} to stretch, yawn: \hyperlink{IX.61}{IX.61}{}
\par\textbf{vijjamāna} existing, actual
\par\textbf{vijjā} (1) clear-vision (the 3 or the 8), (2) science, knowledge 
\par\textbf{viññatti} intimation
\par\textbf{viññāta} cognized
\par\textbf{viññāṇa} consciousness, cognition
\par\textbf{viññāṇaṭṭhiti} station of consciousness (the 7) 
\par\textbf{vitakka} applied thought
\par\textbf{*vitthambhana} \emph{also }distension: \hyperlink{XI.37}{XI.37}{}, \hyperlink{XI.84}{84}{}
\par\textbf{*vinana} joining together: \hyperlink{VIII.247}{VIII.247}{}
\par\textbf{vinaya} (1) Vinaya Piṭaka or Book of Discipline, (2) discipline, removal, leading away 
\par\textbf{*viniddhunana} shaking off: \hyperlink{XVI.82}{XVI.82}{}
\par\textbf{vinipāta} perdition
\par\textbf{vinibbhoga} resolution (into elements)
\par\textbf{*vippaṭipatti} wrong theory: \hyperlink{XVI.85}{XVI.85}{}
\par\textbf{vipariṇāma} change
\par\textbf{vipariṇāma-dukkha} suffering due to change 
\par\textbf{vipariyesa} perverseness (the 4)
\par\textbf{vipallāsa} perversion (the 3)
\par\textbf{vipassanā} insight (the vision of what is formed as impermanent, painful, notself) 
\par\textbf{vipassanā-yānika} one whose vehicle is insight 
\par\textbf{vipāka} (kamma-) result
\par\textbf{*vipphandana} \emph{also }excitement, wrong excitement: \hyperlink{VI.42}{VI.42}{}; \hyperlink{VIII.190}{VIII.190}{} 
\par\textbf{*vipphāra} \emph{also }intervention: \hyperlink{IV.89}{IV.89}{}; \hyperlink{XII.27}{XII.27}{}; \hyperlink{XIV.132}{XIV.132}{} 
\par\textbf{vibhava} (1) non-being, non-becoming, (2) success 
\par\textbf{vimokkha} liberation (the 3 and the 8)
\par\textbf{vimutti} deliverance
\par\textbf{*viyojeti} to separate: \hyperlink{VIII.95}{VIII.95}{} 
\par\textbf{virati} abstinence (the 3) 
\par\textbf{viramana} abstaining
\par\textbf{*viraha} (subst.) absence: \hyperlink{IV.148}{IV.148}{}
\par\textbf{virāga} fading away (of greed)
\par\textbf{viriya} energy
\par\textbf{vivaṭṭa} (1) cessation of the round (of kamma, etc.), the dependent origination as cessation, (2) turning away, (3) expansion (of world after contraction) 
\par\textbf{viveka} seclusion
\par\textbf{*visaṅkharoti} to analyze: \hyperlink{XX.68}{XX.68}{}
\par\textbf{visama-hetu} fictitious cause
\par\textbf{visaya} (1) abode, (2) objective field (of consciousness) 
\par\textbf{*visavitā} majesty: \hyperlink{XII.49}{XII.49}{} (\textbf{\cite{Paṭis} I 174}; II 205; \textbf{\cite{Dhs-a} 109}) 
\par\textbf{*visahati} to suffer: \hyperlink{II.38}{II.38}{}
\par\textbf{*visādana} dejection: \hyperlink{XVI.59}{XVI.59}{}
\par\textbf{*visesa} distinction
\par\textbf{vihāra} (1) dwelling place, abode, (2) monastery, (3) mode of abiding
\par\textbf{*vihaṭamāna} being carded: \hyperlink{XXI.66}{XXI.66}{}
\par\textbf{vihiṃsā} cruelty
\par\textbf{*vītiharaṇa} \emph{also }shifting sideways: \hyperlink{XI.115}{XI.115}{}
\par\textbf{vīthi} (1) street, (2) cognitive series (of consciousness) 
\par\textbf{vīthi-citta} a consciousness of the cognitive series 
\par\textbf{vīmaṃsā} enquiry
\par\textbf{vuṭṭhāna} emergence
\par\textbf{vuṭṭhānagāmini-vipassanā} insight leading to emergence (of the path) 
\par\textbf{veda} (1) wisdom, (2) joy, inspiration, (3) the Vedas 
\par\textbf{vedanā} feeling (i.e. of pleasure, pain, or neither) 
\par\textbf{vedaka} experiencer, one who feels 
\par\textbf{vedayita} feeling what is felt
\par\textbf{veramaṇi} abstention
\par\textbf{vokāra} constituent
\par\textbf{voṭṭhapana} determining (consciousness) 
\par\textbf{vodāna} cleansing (term for consciousness preceding absorption or path)
\par\textbf{vohāra} conventional usage, common speech 
\par\textbf{*vyatti} particular distinction: \hyperlink{VIII.72}{VIII.72}{} (\textbf{\cite{M-a} I 6}) 
\par\textbf{*vyappita} \emph{also }gone away: \hyperlink{IV.146}{IV.146}{} (\textbf{\cite{Vibh} 258}) 
\par\textbf{vyāpāda} ill will
\par\textbf{*vyāpāra} \emph{also }interest, interestedness: \hyperlink{XVII.309}{XVII.309}{}; \hyperlink{XVIII.31}{XVIII.31}{}
\par\textbf{saṃyoga} bondage
\par\textbf{saṃyojana} fetter (the 10)
\par\textbf{saṃvaṭṭa} contraction (of world)
\par\textbf{*saṃvaṇṇita} \emph{also }in detail: \hyperlink{XIII.14}{XIII.14}{}
\par\textbf{saṃvara} restraint
\par\textbf{saṃvega} sense of urgency
\par\textbf{*saṃvedanika} which feels: \hyperlink{XIV.213}{XIV.213}{}
\par\textbf{saṃsāra} round of rebirths
\par\textbf{sakadāgāmin} once-returner (term for 2nd stage of realization) 
\par\textbf{*sakalika} \emph{also }scale (of fish): \hyperlink{VIII.91}{VIII.91}{}
\par\textbf{sakkarā} sugar (spelled sakkharā in PED) 
\par\textbf{sakkāya} individuality
\par\textbf{sakkāya-diṭṭhi} false view of individuality (the 20 kinds) 
\par\textbf{sagga} heaven
\par\textbf{saṅkanti} transmigration
\par\textbf{saṅkappa} thinking
\par\textbf{saṅkamana} transmigrating
\par\textbf{*saṅkara} confounding, confusing: \hyperlink{XIV.58}{XIV.58}{}; epil. verses (see CPD asaṅkara) 
\par\textbf{saṅkilesa} defilement, corruption
\par\textbf{*saṅku-patha} \emph{also }a path set on piles: \hyperlink{IX.36}{IX.36}{} 
\par\textbf{saṅkhata} formed
\par\textbf{saṅkhāra} formation, formed thing
\par\textbf{saṅkhāra-dukkha} suffering due to formations 
\par\textbf{saṅkhāra-pariccheda} delimiting of formations 
\par\textbf{saṅkhārupekkhā} equanimity about formations 
\par\textbf{saṅgati} coincidence, chance
\par\textbf{*saṅgaha} \emph{also }holding together: \hyperlink{XI.93}{XI.93}{}
\par\textbf{*saṅgahīta} \emph{also }held together:\hyperlink{XI.90}{XI.90}{}
\par\textbf{saṅgha} the Order,the Community
\par\textbf{saṅghaṭṭana} knocking together, impingement 
\par\textbf{sacca} truth
\par\textbf{saccānulomika-ñāṇa} knowledge in conformity with truth 
\par\textbf{*sacchika} based on realization: \hyperlink{VII.55}{VII.55}{} (\textbf{\cite{Paṭis} I 174}) 
\par\textbf{sacchikiriyā} realization
\par\textbf{*sañña} restrained: \hyperlink{I.158}{I.158}{} 
\par\textbf{saññā} (1) perception, (2) sign, signal, label 
\par\textbf{saññāvedayitanirodha} cessation of perception and feeling
\par\textbf{saṇṭhāna} (1) shape
\par\textbf{*saṇṭhāna} \emph{also }(2) settling down, stationariness: \hyperlink{III.22}{III.22}{}; \hyperlink{VIII.69}{VIII.69}{}, (3) co-presence: \hyperlink{XVII.76}{XVII.76}{} 
\par\textbf{sati} mindfulness
\par\textbf{satta} a being, a living being
\par\textbf{*satta} Bright Principle, Sattva: \hyperlink{IX.53}{IX.53}{}
\par\textbf{satta-saññā} (1) perception of a living being, (2) the seven perceptions (first of the 18 principal insights) 
\par\textbf{*sattāvāsa} abode of beings (the 9)
\par\textbf{sadda} (1) sound, (2) word, (3) grammar
\par\textbf{sadda-lakkhaṇa} etymology
\par\textbf{saddhā} faith
\par\textbf{saddhā-vimutta} one liberated by faith
\par\textbf{saddhānusārin} faith devotee
\par\textbf{*saddheyya} inspiring faith: \hyperlink{VII.72}{VII.72}{}
\par\textbf{sa-nidassana} visible
\par\textbf{santati} continuity
\par\textbf{santati-sīsa} organic continuity
\par\textbf{santāna} continuity
\par\textbf{santi-pada} state of peace (term for Nibbāna) 
\par\textbf{santīraṇa} investigation (consciousness)
\par\textbf{*sandhāraṇa} \emph{also }upholding: \hyperlink{XIV.44}{XIV.44}{}
\par\textbf{*sannikkhepana} \emph{also }putting down: \hyperlink{XX.62}{XX.62}{} 
\par\textbf{*sanniṭṭheyya} fit to be convinced about: \hyperlink{XIV.151}{XIV.151}{} 
\par\textbf{sannipāta} concurrence
\par\textbf{*sannirujjhana} \emph{also }fixing down: \hyperlink{IV.91}{IV.91}{}; \hyperlink{XII.51}{XII.51}{}; \hyperlink{XX.62}{XX.62}{} 
\par\textbf{*sannissaya} waiting on, dependence: \hyperlink{XIV.29}{XIV.29}{} 
\par\textbf{*sappati} to be emitted (pass. of sapati, to swear): \hyperlink{XV.3}{XV.3}{} 
\par\textbf{*sabbhāva} (presence): \hyperlink{I.141}{I.141}{}; \hyperlink{II.21}{II.21}{}; \hyperlink{XIV.98}{XIV.98}{}; \hyperlink{XVI.73}{XVI.73}{} 
\par\textbf{sabhāva} individual essence 
\par\textbf{*sabhāva} with sex: \hyperlink{XVII.150}{XVII.150}{}
\par\textbf{*sabhāva} Nature, Svabhāva: \hyperlink{XVI.85}{XVI.85}{}
\par\textbf{samatha} serenity (term for jhāna)
\par\textbf{samatha-yānika} one whose vehicle is serenity 
\par\textbf{samaya} period, event, occasion, etc.
\par\textbf{*samabbhāhata} \emph{also }stretched flat: \hyperlink{IV.129}{IV.129}{} 
\par\textbf{*samabbhāhata} \emph{also }impelled: \hyperlink{XI.92}{XI.92}{} 
\par\textbf{samavāya} inherence
\par\textbf{*samaveta} inherent: \hyperlink{XVI.91}{XVI.91}{}
\par\textbf{samādhi} concentration
\par\textbf{samāpatti} attainment (the 9)
\par\textbf{*samāhata} \emph{also }brought in: \hyperlink{IV.190}{IV.190}{}
\par\textbf{samuccheda} cutting off (of defilements by the path) 
\par\textbf{samuṭṭhāna} origination (4 kinds), moulding 
\par\textbf{*samuṭṭhāpaya} rousable: \hyperlink{IV.51}{IV.51}{}
\par\textbf{samudaya} origin
\par\textbf{samudīraṇa} moving
\par\textbf{sampajañña} full awareness
\par\textbf{sampaṭicchana} receiving (consciousness) 
\par\textbf{*sampaṭipādana} keeping on the track: \hyperlink{VI.59}{VI.59}{} 
\par\textbf{*sampatta-visaya} having a contiguous objective field (i.e. smell, taste and touch) 
\par\textbf{*sampasādayati} to make confident: \hyperlink{IV.142}{IV.142}{} 
\par\textbf{*sampiṇḍana} \emph{also }conjunction (gram.): \hyperlink{IV.154}{IV.154}{} 
\par\textbf{samphappalāpa} gossip, idle chatter
\par\textbf{sambojjhaṅga} enlightenment factor (the 7) 
\par\textbf{*sambhāveti} \emph{also }to judge: \hyperlink{IX.109}{IX.109}{}
\par\textbf{*sambhoga} \emph{also }exploiting: \hyperlink{XIV.128}{XIV.128}{}; \hyperlink{XVII.51}{XVII.51}{} 
\par\textbf{sammatta} rightness (the 10)
\par\textbf{sammappadhāna} right endeavour (the 4)
\par\textbf{sammā} right
\par\textbf{sammā-sambuddha} fully enlightened one 
\par\textbf{sammuti} convention, conventional 
\par\textbf{sammuti-sacca} conventional truth (e.g. kasiṇa concept) 
\par\textbf{*sammussana} forgetting: \hyperlink{XVI.82}{XVI.82}{}
\par\textbf{sammosa} forgetfulness
\par\textbf{sammoha} delusion 
\par\textbf{*sarūpena} \emph{also }in its own form: \hyperlink{XVI.70}{XVI.70}{} 
\par\textbf{salakkhaṇa} specific characteristic (e.g. hardness of earth) 
\par\textbf{sallakkhaṇa} observation
\par\textbf{sallekha} effacement
\par\textbf{saḷāyatana} sixfold base (for contact)
\par\textbf{savana} hearing
\par\textbf{savana} flowing 
\par\textbf{*savana} exudation: \hyperlink{XVII.56}{XVII.56}{} 
\par\textbf{sa-saṅkhāra} prompted
\par\textbf{sa-sambhāra-kathā} “accessory locution” (log.)
\par\textbf{sassata} eternal
\par\textbf{sassata-diṭṭhi} eternity view
\par\textbf{*sahaṭṭhāna} co-presence: \hyperlink{XIII.116}{XIII.116}{}
\par\textbf{sāṭheyya} fraud
\par\textbf{*sādhika} accomplishing: \hyperlink{IV.105}{IV.105}{}
\par\textbf{sādhāraṇa} common to, shared with
\par\textbf{sāmañña-phala} fruit of asceticism
\par\textbf{sāmañña-lakkhaṇa} general characteristic (of what is formed, i.e. the 3 beginning with impermanence) 
\par\textbf{sāra} core
\par\textbf{sāvaka} disciple, hearer
\par\textbf{sāsana} dispensation
\par\textbf{sikkhā} training
\par\textbf{sikkhāpada} training precept
\par\textbf{*siṅga} \emph{also }foppery: \hyperlink{III.95}{III.95}{} (\textbf{\cite{Vibh} 351})
\par\textbf{*siṭṭha} prepared: \hyperlink{XVI.4}{XVI.4}{}
\par\textbf{*sippikā} bag (?): \hyperlink{XI.68}{XI.68}{}
\par\textbf{*silesa} cement: \hyperlink{XI.51}{XI.51}{}
\par\textbf{*sīta (?)} measure of area: \hyperlink{XII.41}{XII.41}{}
\par\textbf{*sīmā} chapter house: \hyperlink{IX.66}{IX.66}{}
\par\textbf{sīla} (1) virtue, (2) habit, (3) rite
\par\textbf{sīlabbata} rules and vows (Ñāṇamoli’s original translation was “rites and rituals,” but was changed in accordance with his later translation of this term. ) 
\par\textbf{*sīlaka} good-tempered: \hyperlink{III.84}{III.84}{}
\par\textbf{*sīlana} composing: \hyperlink{I.19}{I.19}{}
\par\textbf{sukkha-vipassaka} bare(or dry-) insight worker (one who attains the path without previously having attained jhāna) 
\par\textbf{sukha} pleasure, pleasant, bliss, blissful, bodily pleasure
\par\textbf{*sukha} tepid: \hyperlink{X.52}{X.52}{}
\par\textbf{*sukhana} act of pleasing: \hyperlink{IV.100}{IV.100}{}
\par\textbf{sugata} Sublime One (the Buddha)
\par\textbf{suñña, suññata} void 
\par\textbf{suññatā} voidness 
\par\textbf{suta} heard
\par\textbf{*suttaka} intestinal worm: \hyperlink{VIII.121}{VIII.121}{}
\par\textbf{subha} beautiful, beauty
\par\textbf{*surabhi} perfume: \hyperlink{III.100}{III.100}{}; \hyperlink{VI.90}{VI.90}{}; \hyperlink{X.60}{X.60}{}
\par\textbf{*sūcayati} to betray, reveal: \hyperlink{XV.3}{XV.3}{}
\par\textbf{*sūdana} cleansing: \hyperlink{XI.125}{XI.125}{}
\par\textbf{sekha} trainer (term for one possessing one of the four paths or first three fruitions, so with training still to do) 
\par\textbf{soka} sorrow 
\par\textbf{sotāpanna} stream enterer (1st stage of realization) 
\par\textbf{somanassa} joy, mental pleasure
\par\textbf{hadaya} heart
\par\textbf{hadaya-vatthu} heart-basis (physical basis of mind)
\par\textbf{hiri} conscience
\par\textbf{hīna} (1) abandoned, (2) inferior
\par\textbf{hetu} root-cause, cause
\par\textbf{*hetu} \emph{also }middle  term  (in  syllogism; log.): \hyperlink{XVII.67}{XVII.67}{}
\end{vismHanging}\end{multicols}
	\backmatter
		% \chapter{References}
\ifplastex
	\chapter{Bibliography}
\else
	\addcontentsline{toc}{chapter}{Bibliography}
\fi
\begin{thebibliography}{xxxxxxxxxxx}
\bibitem[A]{A} Aṅguttara Nikāya
\bibitem[A-a]{A-a} \emph{Aṅguttara Nikāya Aṭṭhakathā = Manorathapurāṇī}
\bibitem[Cp]{Cp} Cariyāpiṭaka
\bibitem[Cp-a]{Cp-a} \emph{Cariyāpiṭaka Aṭṭhakathā}
\bibitem[Dhp]{Dhp} Dhammapada
\bibitem[Dhp-a]{Dhp-a} \emph{Dhammapada Aṭṭhakathā}
\bibitem[Dhs]{Dhs} Dhammasaṅgaṇī
\bibitem[Dhs-a]{Dhs-a} \emph{Dhammasaṅgaṇi Aṭṭhakathā = Atthasālinī}
\bibitem[Dhs-ṭ]{Dhs-ṭ} \emph{Dhammasaṅgaṇī Ṭīkā = Mūla Ṭīkā II}
\bibitem[Dhātuk]{Dhātuk} Dhātukathā
\bibitem[D]{D} Dīgha Nikāya
\bibitem[D-a]{D-a} \emph{Dīgha Nikāya Aṭṭhakathā = Sumaṅgala-vilāsinī}
\bibitem[It]{It} Itivuttaka
\bibitem[J-a]{J-a} \emph{Jātaka-aṭṭhakathā}
\bibitem[Kv]{Kv} Kathāvatthu
\bibitem[Mhv]{Mhv} \emph{Mahāvaṃsa}
\bibitem[M]{M} Majjhima Nikāya
\bibitem[M-a]{M-a} \emph{Majjhima Nikāya Aṭṭhakathā = Papañca-sūdanī}
\bibitem[Mil]{Mil} \emph{Milindapañhā}
\bibitem[Netti]{Netti} \emph{Nettipakaraṇa}
%\bibitem[Nidd I]{Nidd I} Mahā Niddesa
%\bibitem[Nidd II]{Nidd II} Cūḷa Niddesa (Siamese ed.)
\bibitem[Nidd]{Nidd} I: Mahā Niddesa;\\ II: Cūḷa Niddesa (Siamese ed.)
\bibitem[Nikāya-s]{Nikāya-s} \emph{Nikāyasaṃgrahaya}
\bibitem[Paṭis]{Paṭis} Paṭisambhidāmagga
\bibitem[Paṭis-a]{Paṭis-a} \emph{Paṭisambhidāmagga Aṭṭhakathā = Saddhammappakāsinī} (Sinhalese Hewavitarne ed.).
% Paṭṭh I has numerous continuations: , 34, 44 (in index)
%\bibitem[Paṭṭh I]{Paṭṭh I} Paṭṭhāna, Tika Paṭṭhāna 
%\bibitem[Paṭṭh II]{Paṭṭh II} Paṭṭhāna, Duka Paṭṭhāna (Se and Be.)
\bibitem[Paṭṭh]{Paṭṭh} I: Paṭṭhāna, Tika Paṭṭhāna;\\ II: Paṭṭhāna, Duka Paṭṭhāna (Se and Be.)
\bibitem[Peṭ]{Peṭ} \emph{Peṭakopadesa}
\bibitem[Pv]{Pv} Petavatthu
\bibitem[S]{S} Saṃyutta Nikāya
\bibitem[S-a]{S-a} \emph{Saṃyutta Nikāya Aṭṭhakathā = Sāratthappakāsinī}
\bibitem[Sn]{Sn} Sutta-nipāta
\bibitem[Sn-a]{Sn-a} \emph{Sutta-nipāta Aṭṭhakathā = Paramatthajotikā}
\bibitem[Th]{Th} Thera-gāthā
\bibitem[Ud]{Ud} Udāna
\bibitem[Ud-a]{Ud-a} ???
\bibitem[Vibh]{Vibh} Vibhaṅga
\bibitem[Vibh-a]{Vibh-a} \emph{Vibhaṅga Aṭṭhakathā = Sammohavinodanī}
\bibitem[Vibh-ṭ]{Vibh-ṭ} \emph{Vibhaṅga Ṭīkā = Mūla Ṭīkā II}
\bibitem[Vv]{Vv} Vimānavatthu
%\bibitem[Vin I]{Vin I} Vinaya Piṭaka I: (3)—Mahāvagga
%\bibitem[Vin II]{Vin II} Vinaya Piṭaka II: (4)—Cūḷavagga
%\bibitem[Vin III]{Vin III} Vinaya Piṭaka III: (1)—Suttavibhaṅga 1
%\bibitem[Vin IV]{Vin IV} Vinaya Piṭaka IV: (2)—Suttavibhaṅga 2
%\bibitem[Vin V]{Vin V} Vinaya Piṭaka V: (5)—Parivāra
\bibitem[Vin]{Vin} Vinaya Piṭaka:\\ I: (3)—Mahāvagga \\ II: (4)—Cūḷavagga \\ III: (1)—Suttavibhaṅga 1 \\ IV: (2)—Suttavibhaṅga 2 \\ V: (5)—Parivāra
\bibitem[Vin-a]{Vin-a} ???
\bibitem[Vism]{Vism} \emph{Visuddhimagga} (PTS ed. [= Ee] and Harvard Oriental Series ed. [= Ae])
\bibitem[Vism-mhṭ]{Vism-mhṭ} Paramatthamañjūsā, Visuddhimagga Aṭṭhakathā = Mahā Ṭīkā (Chs. I to XVII Sinhalese Vidyodaya ed.; Chs. XVIII to XXIII Be ed.)
\end{thebibliography}

		\input{vism-index.tex}
		\input{vism-glossary.tex}
\end{document}
