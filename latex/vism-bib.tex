% \chapter{References}
\ifplastex
	\chapter{Bibliography}
\else
	\addcontentsline{toc}{chapter}{Bibliography}
\fi
\begin{thebibliography}{xxxxxxxxxxx}
\bibitem[A]{A} Aṅguttara Nikāya
\bibitem[A-a]{A-a} \emph{Aṅguttara Nikāya Aṭṭhakathā = Manorathapurāṇī}
\bibitem[Cp]{Cp} Cariyāpiṭaka
\bibitem[Cp-a]{Cp-a} \emph{Cariyāpiṭaka Aṭṭhakathā}
\bibitem[Dhp]{Dhp} Dhammapada
\bibitem[Dhp-a]{Dhp-a} \emph{Dhammapada Aṭṭhakathā}
\bibitem[Dhs]{Dhs} Dhammasaṅgaṇī
\bibitem[Dhs-a]{Dhs-a} \emph{Dhammasaṅgaṇi Aṭṭhakathā = Atthasālinī}
\bibitem[Dhs-ṭ]{Dhs-ṭ} \emph{Dhammasaṅgaṇī Ṭīkā = Mūla Ṭīkā II}
\bibitem[Dhātuk]{Dhātuk} Dhātukathā
\bibitem[D]{D} Dīgha Nikāya
\bibitem[D-a]{D-a} \emph{Dīgha Nikāya Aṭṭhakathā = Sumaṅgala-vilāsinī}
\bibitem[Dīp]{Dīp} \emph{Dīpavaṃsa}
\bibitem[It]{It} Itivuttaka
\bibitem[J-a]{J-a} \emph{Jātaka-aṭṭhakathā}
\bibitem[Kv]{Kv} Kathāvatthu
\bibitem[Mhv]{Mhv} \emph{Mahāvaṃsa}
\bibitem[M]{M} Majjhima Nikāya
\bibitem[M-a]{M-a} \emph{Majjhima Nikāya Aṭṭhakathā = Papañca-sūdanī}
\bibitem[Mil]{Mil} \emph{Milindapañhā}
\bibitem[Netti]{Netti} \emph{Nettipakaraṇa}
%\bibitem[Nidd I]{Nidd I} Mahā Niddesa
%\bibitem[Nidd II]{Nidd II} Cūḷa Niddesa (Siamese ed.)
\bibitem[Nidd]{Nidd} I: Mahā Niddesa;\\ II: Cūḷa Niddesa (Siamese ed.)
\bibitem[Nikāya-s]{Nikāya-s} \emph{Nikāyasaṃgrahaya}
\bibitem[Paṭis]{Paṭis} Paṭisambhidāmagga
\bibitem[Paṭis-a]{Paṭis-a} \emph{Paṭisambhidāmagga Aṭṭhakathā = Saddhammappakāsinī} (Sinhalese Hewavitarne ed.).
% Paṭṭh I has numerous continuations: , 34, 44 (in index)
%\bibitem[Paṭṭh I]{Paṭṭh I} Paṭṭhāna, Tika Paṭṭhāna 
%\bibitem[Paṭṭh II]{Paṭṭh II} Paṭṭhāna, Duka Paṭṭhāna (Se and Be.)
\bibitem[Paṭṭh]{Paṭṭh} I: Paṭṭhāna, Tika Paṭṭhāna;\\ II: Paṭṭhāna, Duka Paṭṭhāna (Se and Be.)
\bibitem[Peṭ]{Peṭ} \emph{Peṭakopadesa}
\bibitem[Pv]{Pv} Petavatthu
\bibitem[S]{S} Saṃyutta Nikāya
\bibitem[S-a]{S-a} \emph{Saṃyutta Nikāya Aṭṭhakathā = Sāratthappakāsinī}
\bibitem[Sn]{Sn} Sutta-nipāta
\bibitem[Sn-a]{Sn-a} \emph{Sutta-nipāta Aṭṭhakathā = Paramatthajotikā}
\bibitem[Th]{Th} Thera-gāthā
\bibitem[Ud]{Ud} Udāna
\bibitem[Ud-a]{Ud-a} ???
\bibitem[Vibh]{Vibh} Vibhaṅga
\bibitem[Vibh-a]{Vibh-a} \emph{Vibhaṅga Aṭṭhakathā = Sammohavinodanī}
\bibitem[Vibh-ṭ]{Vibh-ṭ} \emph{Vibhaṅga Ṭīkā = Mūla Ṭīkā II}
\bibitem[Vv]{Vv} Vimānavatthu
%\bibitem[Vin I]{Vin I} Vinaya Piṭaka I: (3)—Mahāvagga
%\bibitem[Vin II]{Vin II} Vinaya Piṭaka II: (4)—Cūḷavagga
%\bibitem[Vin III]{Vin III} Vinaya Piṭaka III: (1)—Suttavibhaṅga 1
%\bibitem[Vin IV]{Vin IV} Vinaya Piṭaka IV: (2)—Suttavibhaṅga 2
%\bibitem[Vin V]{Vin V} Vinaya Piṭaka V: (5)—Parivāra
\bibitem[Vin]{Vin} Vinaya Piṭaka:\\ I: (3)—Mahāvagga \\ II: (4)—Cūḷavagga \\ III: (1)—Suttavibhaṅga 1 \\ IV: (2)—Suttavibhaṅga 2 \\ V: (5)—Parivāra
\bibitem[Vin-a]{Vin-a} ???
\bibitem[Vism]{Vism} \emph{Visuddhimagga} (PTS ed. [= Ee] and Harvard Oriental Series ed. [= Ae])
\bibitem[Vism-mhṭ]{Vism-mhṭ} Paramatthamañjūsā, Visuddhimagga Aṭṭhakathā = Mahā Ṭīkā (Chs. I to XVII Sinhalese Vidyodaya ed.; Chs. XVIII to XXIII Be ed.)
\end{thebibliography}
